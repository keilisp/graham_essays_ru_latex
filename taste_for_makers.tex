\documentclass[ebook,12pt,oneside,openany]{memoir}
\usepackage[utf8x]{inputenc} \usepackage[russian]{babel}
\usepackage[papersize={90mm,120mm}, margin=2mm]{geometry}
\sloppy
\usepackage{url} \title{По следам величайших творцов } \author{Пол
  Грэм} \date{}
\begin{document}
\maketitle

Недавно я говорил со своим другом, преподавателем Массачусетского
Технологического института. Его область сейчас очень популярна и
каждый год на него обрушивается шквал заявок от потенциальных
дипломников. «Многие из них кажутся очень неглупыми.» — сказал он, —
«Но я не могу сказать, обладают ли они каким-либо вкусом». Вкус.
Нечасто теперь можно услышать это слово. И все же, нам не обойтись без
такого понятия — как бы его ни называли. Мой друг имел в виду, что
нужны студенты, являющиеся не только хорошими специалистами, но и
способные использовать свои знания для создания красивых вещей.
Математики называют хорошую работу «красивой» точно так же, как —
сейчас и в прошлом — это делают и делали ученые, инженеры, музыканты,
архитекторы, конструкторы, писатели и художники. Вопрос в том,
используют ли все они одно слово случайно или же в случайности есть
своя закономерность? Если таковая присутствует, то можем ли мы
критерии красоты, определенные в одной области, применить в другой?
Для людей, создающих новое, это не отвлеченные теоретические вопросы.
Если существует такая вещь, как красота, нам нужно уметь распознавать
ее. Хороший вкус нужен для создания хороших вещей. Чтобы не рассуждать
об отвлеченных абстракциях, зададимся практическим вопросом: как
делать хорошие вещи? Стоит упомянуть вкус и многие скажут: «вкус
субъективен». И в этом утверждении они совершенно искренни, людям
действительно так кажется. Когда им что-то нравится, они не имеют ни
малейшего представления, почему. Может быть, это оттого, что объект их
симпатий на самом деле красив. Или же у их матери была такая вещица /
они видели ее на знаменитости в глянцевом журнале / знают о ее немалой
цене. Мысли этих людей — клубок неподотчетных им импульсов.
Родительское воспитание чаще всего не способствует распутыванию этого
клубка. Когда вы поднимаете на смех своего братишку за инопланетные
цветовые гаммы в его книжке раскрасок, то мама, скорее всего, скажет
вам: «на вкус и цвет товарища нет». В этот момент она не пытается
донести до вас какую-либо важную истину об эстетике. Она просто хочет
помирить вас. Как и многие другие половинчатые истины, что говорят
детям взрослые, эта противоречит другим их речам. Внушив нам, что вкус
— лишь вопрос личных предпочтений, они ведут нас в музей и заставляют
всматриваться в вывешенные там картины, поскольку Леонардо — великий
художник. Что же происходит в голове ребенка в эту минуту? Чем
становится для него «великий художник»? После того, как ему годы
повторяли, что каждому мил свой вкус и цвет, он вряд ли откроет для
себя, что великий художник — это тот, чьи работы лучше всех прочих.
Скорее всего, в его Птолемеевой модели Вселенной «великий» будет
означать вещь, полезную самому ребенку (как брокколи), — ведь так
сказано в книге. Сказать, что вкусы субъективны, — неплохой способ
погасить споры о них. Беда в том, что это ложь. Это начинаешь понимать
при создании нового. В какое бы дело ни вкладывались люди, они всегда
хотят делать его лучше. Футболисты хотят выигрывать. Управляющие
директора хотят увеличивать прибыли. Сделать лучше — это и вопрос
гордости, и удовольствие само по себе. Но если работа заключается в
создании нового, и нет такой вещи как красота, то нет возможности
сделать работу лучше. Если вкус — лишь вопрос личных предпочтений, то
каждый уже совершенен: у каждого человека есть своя мера, и они друг
другу подходят. Как и в любой другой работе, продолжая созидать, вы
растете над собой. Ваш вкус меняется. И, как любой другой, кто
совершенствуется в своем деле, вы ощущаете свой рост. Значит. прежние
вкусы не просто иные, они — хуже. Так рассыпается в пух и прах
утверждение, что вкус не может быть ложным. Релятивизм нынче в моде,
что может заставлять не обращать внимания на вкус, даже когда он
улучшается. Но если признаться, по крайней мере самому себе, что есть
хороший дизайн — красота — и плохой дизайн — уродство, то можно начать
детально изучать красоту. Как изменился ваш вкус? Чем были вызваны
ваши ошибки? Что узнали другие о красоте? Начав изучать этот вопрос,
вы будете удивлены тем, насколько схожи могут быть принципы красоты в
совершенно несхожих областях человеческой деятельности. Хороший дизайн
прост. Так говорят все: от математиков до художников. В математике это
означает, что чем короче доказательство, тем оно лучше. Когда речь
заходит об аксиомах, то воистину: меньше — да лучше. Во многом то же
самое верно и в программировании. Для архитекторов и дизайнеров это
значит, что красота должна основываться на нескольких тщательно
выбранных структурных элементах, а не на обилии внешнего орнамента.
(Орнамент плох не сам по себе, а лишь когда он пытается скрыть
безвкусную форму). Аналогично в живописи, натюрморт, составленный из
нескольких точно подобранных объектов с тщательной моделировкой, будет
более интересен, чем, например вычурное переплетение кружев. В
писательском деле этот принцип говорит: скажи все, что имеешь в виду,
и без лишних слов. Кто-то подумает: странно обращать внимание на
простоту. Казалось бы, простота присутствует изначально, а сложность
требует дополнительных усилий. Но, видимо, что-то находит на людей,
когда те пытаются блеснуть творчеством. Начинающие писатели
изъясняются напыщенными фразами, которые никогда не употребляют в
обычной речи. Дизайнеры в стремлении к артистизму прибегают к
выписыванию «красивостей». Художники преображаются в экспрессионистов.
Но это лишь увертки, за витиеватыми словесами и «экспрессивными»
мазками ничего нет — и это страшно. Минимум средств выражения
заставляет вас обращаться к реальной проблеме. Когда вы не можете
изощряться в форме, остается донести суть. Хороший дизайн вне времени.
В математике не устаревает ни одно из доказательств, если только оно
не содержит ошибки. Так что же имел в виду Харди, сказав, что в
вечности нет места для некрасивой математики? Да то же, что и Келли
Джонсон: некрасивое решение не может быть лучшим. Обязательно должно
быть лучшее, и кто-нибудь в конце концов найдет его. Устремление в
будущее — один из способов найти лучший ответ: если вы чувствуете, что
кто-то может вас превзойти, то сделайте это сами. Некоторые из великих
мастеров настолько приблизились к этому идеалу, что оставили очень
мало места для тех, кто шел следом. Каждый гравер после Дюрера обречен
творить в его тени. Устремитесь в будущее — и избегнете веяний
сиюминутной моды. Ведь мода — почти по определению — меняется со
временем, и если создать вещь, что будет выглядеть хорошо и в далеком
будущем, то ее притягательность будет гораздо больше основываться на
благих, вневременных качествах и меньше — на текущей моде. Почти
парадокс: если вы хотите привлечь своим творением будущие поколения,
один из способов — это привлечь поколения ушедшие. Трудно представить,
каким будет будущее, но оно, несомненно, как и прошлое, будет
равнодушно к модам этого дня. Поэтому, сделав вещь, которая пришлась
бы по нраву не только вашим современникам, но и жившим в XVI веке,
есть все шансы полагать, что она придется по нраву и обитателям века
XXVI-го. Хороший дизайн решает верную проблему. На типичной кухонной
плите есть четыре конфорки, расположенных по углам квадрата, и ручка
для регулировки огня на каждой из них. Как расположить эти ручки?
Простейший ответ — поместить их в ряд. Но это ответ на неправильный
вопрос. Ручки делаются для людей, и если разместить их в один ряд, то
несчастная домохозяйка первое время будет постоянно вспоминать, какая
ручка какой конфорке соответствует. Лучше расположить ручки квадратом,
как и конфорки. Немало плохих дизайнерских решений приняты совершенно
осознанно, но с использованием неверных предпосылок. В середине
двадцатого века существовала мода на набор текстов рублеными шрифтами.
Эти шрифты по своему облику ближе к исходным формам букв. Но это не та
цель, которую нужно достигнуть. Что важно, так это читаемость текста,
и она напрямую зависит от того, насколько легко отличить одну букву от
другой. Может быть, это и отдает викторианской эпохой, но в шрифтах
Times Roman строчную g легко отличить от строчной y (в отличие от
шрифтов без засечек). Рассматриваемые проблемы можно улучшать также,
как и решения. В разработке ПО неразрешимая проблема обычно может быть
заменена эквивалентной ей и вполне разрешимой. Физика сделала огромные
успехи с того момента, как ее задачей стало предсказание наблюдаемых
явлений, а не согласование их с Писанием.

Хороший дизайн зачаровывает. В романах Джейн Остин вы почти не найдете
описаний, вместо этого ее стиль настолько хорош, что вы сами
представляете себе всю обстановку. Также и картина, оставляющая место
воображению, захватывает нас гораздо больше, чем та, что точно
фиксирует события. Каждый самостоятельно продолжает историю Моны Лизы.

В архитектуре и дизайне тот же принцип гласит, что здание или объект
должны допускать любое его использование. Хороший дом позволить
организовать жизнь его обитателей так, как те пожелают, и не будет
диктовать единственно возможный уклад, словно они выполняют программу,
написанную архитектором. В программном обеспечении это означает, что
следует дать пользователем несколько базовых элементов, которые те
смогут комбинировать любым образом, по образу конструктора Лего. В
математике доказательство, становящееся основой для множества новых
работ, предпочтительнее того, что не ведет к будущим открытиям. В
научной среде в целом частота цитирования служит грубым индикатором
ценности работы. Хороший дизайн часто слегка забавен. Этот признак
верен не всегда. Но гравюры Дюрера, кресло Womb Эро Сааринена, Пантеон
и оригинальный Порше 911 кажутся мне чем-то забавными.

Кресло Womb Эро Сааринена

“Пилат умывает руки”, Альбрехт Дюрер

Пантеон Теорема Гёделя о неполноте похожа на розыгрыш. Мне кажется,
причина — в связи юмора и силы. Обладать чувством юмора — значит
обладать силой: способность отвечать улыбкой на удары судьбы защищает
от жизненных невзгод, а неспособность воспринимать их с юмором
уменьшает защищенность. Поэтому признак силы — или хотя бы ее право —
не относиться к себе чересчур серьезно. Нередко уверенные в себе люди
создают словно играючи — как Хичкок в своих фильмах или Брегель в
своих картинах, или Шекспир, если уж на то пошло. Возможно, красота и
не обязана быть забавной, но тяжело представить что-либо, не
вызывающее ни малейшей улыбки и в то же время обладающее красотой.
Хороший дизайн — тяжелый труд. Если посмотреть на людей, что создавали
великие творения, то можно выделить общую черту: они работали на
пределе своих сил. Если вы не выкладываетесь полностью, то, вполне
возможно, впустую тратите время. Неразрешимые проблемы ждут своих
титанов. В математике сложные доказательства требуют искусных решений,
они же являются наиболее интересными. Сходная ситуация — в инженерном
деле. Когда вам предстоит подъем на гору, вы выкидываете из рюкзака
все ненужное. Точно так же архитектор, возводящий здание на неудобном
участке или в условиях ограниченного бюджета, будет вынужден найти
элегантное решение. Все прикрасы становятся ненужны, когда речь идет о
решении задачи на пределе возможностей. Но не всякие трудности хороши.
Есть хорошая боль и плохая. Нормальный человек всегда предпочтет ту
боль, что возникает в ногах после долгой пробежки, а не после
вонзившегося в ступню гвоздя. Сложная проблема будет благом для
конструктора, но ненадежный клиент или плохие материалы — вряд ли. В
живописи высочайшее место традиционно отводится изображениям людей.
Это не случайно — но не по одной лишь причине, что портреты посылают в
наш мозг какие-то сигналы, которые не посылают другие картины. Мы все
видели столько людских лиц, что видим малейшую фальшь в портрете. Если
нарисовать дерево и изменить угол наклона ветви на пять градусов, то
никто не заметит. Но если поменять угол наклона глаз на те же пять
градусов, то разницу заметят все. Когда конструкторы Баухаус приняли
девиз Салливана «Форма следует за функцией», они подразумевали: форма
должна следовать функции. И чем больше требований предъявляется к
функциональности, тем более отточенной должна быть форма — ведь не
допускается и возможности ошибки. Дикие животные потому и красивы, что
жизнь не дает им поблажек. Хороший дизайн выглядит легко. Как и у
выдающихся спортсменов, результаты труда великих творцов выглядят
легкодостижимыми. Чаще всего это иллюзия. Легкий стиль, разговорная
манера хорошего произведения получается только с восьмой попытки. В
науке и инженерном деле некоторые из величайших открытий кажутся
настолько элементарными, что мы говорим себе: «Я сам мог бы до этого
додуматься». В этот момент первооткрыватель имеет полное право
спросить: «Ну и почему же не додумался?» Некоторые портреты Леонардо
состоят всего из нескольких линий. Смотря на них, думаешь: стоило бы
провести восемь-десять линий в нужном месте — и красивый портрет был
бы готов. Ну да, так оно и есть, но линии-то нужно провести в том
самом нужном месте. Малейшая ошибка разрушает рисунок. На самом деле,
штриховые рисунки — сложнейшие из всех, поскольку они требуют почти
совершенной техники. Выражаясь математически, это решение в замкнутом
виде; художники уровнем пониже обычно решают ту же задачу
последовательными приближениями. Одна из причин, почему дети бросают
рисовать примерно в десять лет, — это желание начать рисовать как
взрослые. И вот в числе их первых попыток — штриховой рисунок лица. Не
тут-то было! В большинстве областей подобная легкость появляется с
практикой. Вероятно, роль практики — в тренировке вашего подсознания,
чтобы переложить на него задачи, ранее занимавшие сознание. Иногда это
сводится к тренировке тела. Пианист-виртуоз может играть ноты быстрее,
чем его мозг посылает сигналы пальцам. Схожим образом художник после
некоторой тренировки может воспринимать картину визуально и
воспроизводить руками так же автоматически, как кто-то отстукивает
ногой ритм. Когда люди говорят о нахождении в «состоянии потока», я
думаю, они подразумевают взятие спинным мозгом ситуации под свой
контроль. Спинной мозг реже сомневается и освобождает сознательную
мысль для сложных задач. Хороший дизайн содержит симметрию. Я считаю,
что симметрия может быть одним из путей достижения простоты, но она
достаточно важна сама по себе и заслуживает отдельного упоминания.
Симметрия часто встречается в природе, а это хороший знак. Существуют
два вида симметрии: повторение и рекурсия. Рекурсия означает
повторение в подэлементах, например расположение прожилок в листе.
Симметрия непопулярна сейчас в некоторых областях, это обратная
реакция на ее чрезмерное применение в прошлом. Архитекторы стали
намеренно создавать асимметричные постройки в викторианскую эпоху, и к
1920-м годам асимметрия была непременным признаком архитектуры
модерна. Но даже тогда, убирая симметрию относительно главных осей,
они сохраняли сотни симметрий в малых деталях. В литературе симметрия
присутствует на любом уровне, от фраз в предложении до композиции
целого романа. Подобное можно обнаружить в музыке и живописи. Мозаика
(и некоторые произведения Сезанна) обретают дополнительный визуальный
эффект за счет создания единой картины из отдельных частиц.
Композиционная симметрия приводит к появлению наиболее запоминающихся
картин, особенно когда две половины соотносятся друг с другом, как в
Сотворении Адама или Американской готике.

Сотворение Адама

Американская готика В математике и инженерном деле рекурсия приносит
особенно щедрые плоды. Индуктивные доказательства изумительно
компактны. В разработке программного обеспечения задача, которая может
быть решена с помощью рекурсии, практически всегда решается ею лучше
всего. Эйфелева башня выглядит так поразительно отчасти потому, что
это рекурсивное решение — башня на башне. Опасность симметрии, и
повторения в особенности, в том, что она может заменять мысль вообще.
Хороший дизайн подражает природе. Ценен не факт подражания сам по
себе, а то, что у природы было немало времени для поиска решения.
Поэтому использование в вашей работе решений, наработанных природой, —
хороший знак. Копировать не зазорно. Мало кто станет отрицать, что
литература должна быть подобна жизни. Черпание сюжетов из жизни весьма
помогает и в живописи, но суть процесса здесь часто не понимают. Цель
ведь не просто сделать слепок реальной жизни. Смысл рисования с натуры
в том, что оно дает художнику пищу для размышлений. Когда его глаза
смотрят на что-то, то кисть создает более интересные картины.
Подражание природе срабатывает и в конструировании. Длительное время у
лодок имелся хребет и ребра, как в грудной клетке животного. Порой
приходится ждать лучшей технологии: первые конструкторы самолетов
ошибочно замышляли свои машины похожими на птиц, не обладая ни
достаточно легкими материалами и двигателями (двигатель братьев Райт
весил 70 кг при мощности в 12 л.с.), ни системами управления,
позволяющими машинам двигаться как птицы. Но я могу представить
небольшие беспилотные самолеты-разведчики, летающие как птицы, лет
через пятьдесят.

Двигатель братьев Райт Сейчас, обладая достаточной вычислительной
мощью, мы можем моделировать как метод природы, так и ее результаты.
Генетические алгоритмы могут позволить создавать вещи слишком сложные
для обычных методов. Хороший дизайн — редизайн. Редко когда удается
сделать все хорошо с первого раза. Мастера готовы выбросить часть
старого труда. И в планах их есть место новым планам. Чтобы отказаться
от уже сделанного, нужно быть уверенным в своих силах. Нужно уметь
помыслить: я могу сделать лучше. К примеру, когда люди начинают
рисовать, они довольно часто отказываются перерисовывать части,
которые выполнены неправильно. Им кажется, что имеющийся результат —
уже большое везение, и если они станут его переделывать, то непременно
получится хуже. Поэтому они убеждают себя в том, что рисунок не так уж
и плох, более того, возможно, они так и хотели все нарисовать.

Набросок Леонардо Но это скользкий путь; если и следует что-либо
взращивать в себе, так это неудовлетворенность. В рисунках Леонардо
часто присутствует по пять-шесть попыток провести нужную линию.
Характерный облик Порше 911 возник только после переделки неказистого
прототипа.

Porshe 695 — прототип 911 У Райта в ранних чертежах музея Гугенхайма
правая половина была выполнена в виде зиккурата, перевернув ее, он
получил текущий облик музея.

Музей Гугенхайма Ошибки естественны. Учитесь не воспринимать их как
катастрофу, а с легкостью признавать и исправлять. В какой-то мере
Леонардо изобрел эскиз — для большей свободы в экспериментах. Открытое
программное обеспечение содержит меньше ошибок, потому что при его
создании признается возможность ошибок. Хорошим подспорьем является
носитель, который легко допускает исправления. Когда в XV веке темперу
сменила масляная краска, она немало облегчила написание таких сложных
объектов, как фигуры людей, потому что масляную краску, в отличие от
темперы, можно было смешивать и закрашивать. Хороший дизайн допускает
копирование. Отношение к копированию в процессе становления творца
часто совершает полный оборот. Начинающий художник имитирует мастера,
не подозревая об этом; затем он осознанно пытается быть оригинальным;
наконец, он решает, что важнее следовать истине, чем быть
оригинальным. Невольная имитация почти всегда — гарантия плохого
дизайна. Если вы не знаете, откуда появляются ваши задумки, возможно,
вы подражаете подражателю. Рафаэль так повлиял на вкусы середины XIX
века, что практически каждый художник подражал ему, зачастую он
подражал подражателям подражателей Рафаэля и т.д. Последнее беспокоило
прерафаэлитов даже больше, чем подражание самому Рафаэлю. Амбиции не
позволяют удовлетвориться подражанием. Вторая фаза в росте вкуса —
осознанная попытка быть оригинальным. Мне кажется, великие мастера
пошли еще дальше, достигнув самоотвержения. Они стремились к истине, и
если часть ее была уже открыта кем-то другим, не было никакой причины
не воспользоваться этим. Они были достаточно уверенны в себе, чтобы
копировать чужие работы без страха, что их собственный стиль в
результате будет утерян. Хороший дизайн часто бывает странный.
Некоторые из лучших работ обладают непривычным качеством: формула
Эйлера, Охотники на снегу Брегеля, SR-71, Лисп. Они не только красивы,
они странно красивы.

Формула Эйлера

Охотники на снегу

SR-71 Я затрудняюсь объяснить причину этого. Возможно, дело в моей
глупости — консервный ключ, должно быть, собаке покажется чудом.
Вероятно, если бы я был достаточно умен, формула $ e^(i* \pi) = -1 $
казалась бы мне самой естественной вещью в мире. В конце концов, она
истинна. Большинство уже упомянутых качеств могут быть сознательно
достигнуты, но я не думаю, что можно сознательно добиться странности.
Лучшее, что здесь можно сделать, — это не душить странную идею в
зародыше. Эйнштейн не пытался сделать Теорию Относительности такой
странной. Он лишь хотел открыть истину, но открытая истина оказалась
странной. Учащиеся художественной школы, где я когда-то обучался,
больше всего хотели развить свой собственный стиль. Но если вы
стремитесь делать вещи хорошо, вы неизбежно станете делать это
по-своему, точно так же, как неповторима походка каждого человека.
Микеланджело не стремился рисовать как Микеланджело, он лишь хотел
рисовать хорошо и не мог ничего поделать с тем стилем, который у него
выработался. Единственный стиль, который стоит иметь, — это тот, с
которым вы ничего не можете поделать. И это особенно верно в случае
странности: к ней нет царских путей. Северо-западный проход, что
искали маньеристы, романтики и два поколения американских студентов,
судя по всему, не существует. Единственный способ попасть туда —
пройти путем красоты и истины. Хороший дизайн концентрируется во
времени и пространстве. Среди жителей Флоренции XV века мы находим
Брунеллески, Гиберти, Донателло, Мазаччо, Филиппо Липпи, Фра
Анджелико, Веррокьо, Боттичелли, Леонардо и Микеланджело . Милан в то
время был не меньшим городом, но сколько художников он подарил миру в
XV веке? Что-то такое происходило во Флоренции в XV веке. И это нельзя
отнести к наследственности, поскольку это не повторяется сейчас.
Следует признать, что какими бы врожденными способностями ни обладали
Леонардо и Микеланджело, в Милане были точно такие же люди. Так что же
случилось с Леонардо из Милана? Сейчас в США проживает примерно в
тысячу раз больше людей, чем жило тогда во Флоренции. Тысячи Леонардо
и тысячи Микеланджело бродят среди нас. Если бы за все отвечала ДНК,
то в искусстве каждый день происходили бы открытия. Но это не так, и
причина в том, что для появления Леонардо недостаточно его врожденных
способностей. Нужна еще Флоренция 1450-го года. Нет ничего более
мощного, чем сообщество талантливых людей, работающих над сходными
проблемами. Гены в этом случае не играют определяющую роль: даже
обладай вы полным набором генов Леонардо, ничего не вышло, родись вы
не во Флоренции, а в Милане. Сегодня мы меньше ограничены в
передвижениях, но лучшие результаты по-прежнему достигаются в
определенных местах: Баухаус, Проект Манхэттен, Нью-Йоркер, Lockheed’s
“Skunk Works”, Xerox PARC. В любой момент времени существовало
несколько актуальнейших областей и несколько групп, добивавшихся в них
великих результатов. Но практически невозможно достичь больших успехов
ни в одной из этих областей, если вы слишком удалены от таких групп.
Этот закон довольно гибок, но обойти его нельзя. (Может быть, вы
можете, но Леонардо из Милана не смог). Хороший дизайн вызывает
сопротивление. Какую историческую эпоху ни возьми, ее современники
верили в вещи, кажущиеся теперь смешными, и верили настолько сильно,
что вы могли стать изгоем и даже жертвой насилия, высказывая
противоположные суждения. Было бы поразительно, если бы наше время
отличалось в лучшую сторону. И насколько я могу судить, это не так.
Эта проблема затрагивает не только каждую эпоху, но в некоторой
степени и каждую область. Немалая часть искусства Возрождения
считалась в свое время чрезмерно мирским: если верить Васани,
Боттичелли покаялся и бросил занятия живописью, а Фра Бартоломео и
Лоренцо ди Креди сожгли некоторые из своих работ. Теория
относительности Эйнштейна вызвала неприятие многих современных ему
физиков и не получала окончательного признания несколько десятилетий
(во Франции — до 1950 года). Сегодняшняя экспериментальная ошибка —
теория завтрашнего дня. Если вы мечтаете о великих открытиях, то надо
не закрывать глаза на те места, где обыденная мудрость и истина
противоречат друг другу, а пристально изучать их. На практике, я
считаю, легче опознать уродство, чем представить красоту. Большинство
людей, созидавших красивые вещи, делали это путем исправления чего-то,
что они считали уродливым. Великие труды, похоже, зачастую появляются
оттого, что кто-то смотрит и думает: «Я мог бы сделать лучше». Так,
изображения Мадонн, выполненные согласно византийским традициям,
удовлетворявшим всех на протяжении веков, Джотто казались
безжизненными и неестественными. Коперник был настолько недоволен
«заплаткой», устраивавшей его современников, что почувствовал: должно
быть лучшее решение. Одной только нетерпимости к уродству
недостаточно. Вам нужно хорошо знать соответствующую область прежде,
чем вы поймете, что в ней требует вмешательства. Нужно сделать свою
домашнюю работу. Но когда вы станете экспертом в этой области, вы
начнете слышать голоса, шепчущие: «Что за поделка! Должно быть лучшее
решение!» Не глушите в себе эти голоса. Хольте и лелейте их. Ибо
секрет великого труда таков: точнейший вкус и способность его
воплотить. Примечания На самом деле, Салливан сказал «форма всегда
следует функции», но я считаю, неточность, обычно допускаемая в этой
цитате, точнее отражает мысль архитекторов модерна. Stephen G. Brush,
“Why was Relativity Accepted?” Phys. Perspect. 1 (1999) 184–214.


\end{document}
