\documentclass[ebook,12pt,oneside,openany]{memoir}
\usepackage[utf8x]{inputenc} \usepackage[russian]{babel}
\usepackage[papersize={90mm,120mm}, margin=2mm]{geometry}
\sloppy
\usepackage{url} \title{Экономическое неравенство} \author{Пол Грэм}
\date{}
\begin{document}
\maketitle

Начиная с 1970-х уровень экономического неравенства в США сильно
возрос. В частности, богатые становились и становятся еще богаче. Для
некоторых это является знаком раскола общества внутри страны.

Меня интересует данная тематика так как я сам являюсь одним из
создателей экономического неравенства. Я был одним из основателей
компании под названием Y Combinator, которая помогает людям основывать
стартапы. Практически по определению основатели становятся богатыми
людьми, если стартап успешный. И даже если богатство не является
единственной целью основателей стартапов, многие становятся таковыми,
и лишь единицы — нет.

Я стал экспертом в том, как увеличить экономическое неравенство, я
потратил последнее десятилетие, трудясь именно над этим, а не только
помогая 2400 основателям стартапам, которым YC выделил средства. Мы
также писали статьи, которые были призваны воодушевить людей на
повышение уровня экономического неравенства и в которых давались
подробные инструкции о том, как это сделать.

Поэтому, когда я слышу, что люди говорят об экономическом неравенстве
как о крайне негативном явлении, которое должно быть изжито, то я
чувствую себя диким животным, подслушавшим беседу охотников. Но то,
что поражает меня больше всего в таких разговорах — насколько они
сбиты с толку. Кажется, что они и сами не уверены, хотят они убивать
меня или нет.

Самой распространенной ошибкой людей в данном вопросе является
отношение к экономическому неравенству как к единственному феномену.
Самая наивная версия основывается на ошибочном утверждении касательно
распределения богатства: богатые становятся еще богаче, отнимая деньги
у бедных.

Обычно люди с ходу принимают это суждение вместо того, чтобы изучить
вопрос и прийти к логическим умозаключениям. Иногда такие ошибочные
утверждения о распределении богатства начинаются весьма
недвусмысленно:

… часть национального дохода, которую присваивают себе богатые
постоянно растет, из-за чего уменьшается то, что осталось… (см.
примечание [1] в конце статьи

Иногда это более подсознательно. Но подсознательная форма очень широко
распространена. Я думаю, так происходит потому, что мы росли в мире,
где эта ошибочная концепция распределения богатства была вполне верна.
Для детей богатство — это и есть тот пирог, который делится на всех, и
если один получит больший кусок, то это лишь за счет другого. И нужно
сделать сознательное усилие, чтобы напомнить себе о том, что реальный
мир функционирует по-другому.

В реальном мире вы можете создать богатство так же, как и отнять его у
других. Столяр создает богатство. Он делает стул, и вы с радостью
платите ему за его работу. Трейдер в области алгоритмической торговли
не создает богатство — он зарабатывает доллар только тогда, когда
кто-то по другую сторону этот доллар теряет.

Если богатые люди в обществе аккумулируют капитал путем отнятия его у
бедных, то мы сталкиваемся с вырожденным случаем экономического
неравенства, где причина бедности и богатства одна и та же. Но далеко
не все случаи экономического неравенства таковы. Если один столяр
сделал 5 стульев, а второй — ничего, то последний и заработает меньше,
но не потому, что у него кото-то что-то отнял.

Даже люди, которые разбираются в экономике настолько, чтобы понимать
заблуждение о распределении богатства, все равно поддаются ему ввиду
привычки описывать экономическое неравенство как соотношение одного
вида дохода или богатства к другому. Довольно легко перейти от
разговоров о доходе из одной статистики к другой и действительно
поверить в то, что именно это и происходит.

Если исключить вырожденный случай, то экономическое неравенство не
может быть описано соотношениями или кривыми графиков. В общем случае
оно складывается из множества путей как к богатству, так и к бедности.
А это означает, что для понимания экономического неравенства в стране,
вам нужно найти людей, которые разбогатели или стали бедными, и
разобраться, как это произошло (см. примечание [2] в конце статьи)

Если вы хотите понять изменения в экономическом неравенстве, то вам
следует спросить, что бы эти люди сделали, если бы все было наоборот.
Я знаю, что именно так богатые не становятся таковыми просто из-за
какой-то зловещей новой системы перехода к ним богатства, отнимаемого
у других. Когда вы используете этот «условный» метод с основателями
стартапов, вы понимаете, что многие сделали бы в 1960 году, когда
уровень экономического неравенства был ниже — присоединились бы к
крупной компании или получили бы звание профессора. До того, как Марк
Цукерберг основал Facebook, его жизненной амбицией по умолчанию было
работать в Microsoft. Причиной тому, что он и множество других
основателей стартапов богаче, чем они были бы в середине XX века,
является не то, что страна приняла правильный курс во времена
Администрации Рейгана, а развитие технологий, которое упростило
процесс основания новой быстро растущей компании.

Как бы странно это ни звучало, но, судя по всему, традиционные
экономисты не расположены к анализу отдельных людей. Кажется, что для
них существует неписанное правило — все должно начинаться со
статистики. Таким образом, они дают очень точные числа об изменении
уровня богатства и доходах, а затем следуют с наивными предположениями
о первопричинах.

Я как раз знаю первопричины экономического неравенства – как
производитель и его «соучастник». Да, немало людей становятся
богатыми, не создавая материальных ценностей, много и тех, кто
получает богатство, занимаясь бизнесом, в котором деньги переходят от
одних другим. Однако велико число тех, кто становятся богатыми,
создавая богатство.

И эта группа представляет две проблемы для охотников, мечтающих
уничтожить экономическое неравенство. Первая — ускорение изменений для
эффективности производства. Уровень, на котором люди могут создать
богатство, зависит от технологий, которые имеются в их распоряжении, а
они развиваются экспоненциально. Другая проблема с созданием богатства
как первопричины неравенства — это то, что оно может разрастись и
затронуть многих людей.

Я за то, чтобы пресечь окольные пути к достижению богатства. Но это не
уберет экономического неравенства, так как пока доступна возможность
богатеть, создавая богатство, люди будут это делать.

Большинство людей, которые разбогатели, — это честные люди. И какими
бы ни были их недостатки, лень, как правило, не является одним из них.
Предположим, что новая политика не способствует тому, чтобы можно было
сколотить состояние, занимаясь финансами. Кажется ли вам вероятным,
что люди, которые в настоящее время входят в финансы, чтобы нажить
состояние, продолжат заниматься тем же, довольствуясь обычными
зарплатами? Они занимаются финансами не потому, что любят это, а
потому, что хотят разбогатеть. Если единственный путь к богатству
лежит через открытие стартапа, то они станут основывать стартапы. Они
преуспеют и в этом тоже, потому что целеустремленность — это главный
фактор для достижения успеха стартапа (см. примечание [3] в конце
статьи). Возможно, для мира было бы куда лучше, если бы все, кто
занимался бизнесом, в котором капитал переходит от проигравшего к
победителю, перешли бы к созданию богатства. Однако это все равно не
устранит экономическое неравенство — как раз напротив, это будет
способствовать его развитию. В бизнесе, где одни отнимают добычу у
других, по крайней мере, есть верхний предел выигрыша. К тому же
множество новых стартапов создали бы новые технологии, которые
ускорили бы изменения в производительности.

Изменения в производительности — это далеко не единственный источник
экономического неравенства, но это его ядро, которое не удастся
сокрушить, даже если вы уже исключили остальные источники. И если так,
то это ядро расширится, плюс к тому же появится монополия, и каждый,
кто мог бы разбогатеть, создавая богатство, будет получать достаточно,
чтобы это могло удержать его от подобных мыслей.

Экономическое неравенство не может быть изжито, если только вы не
начнете препятствовать тому, что люди становятся богатыми. А
последнего нельзя достигнуть, если не препятствовать основанию ими
стартапов.

Поэтому давайте расставим точки над «i». Окончание экономического
неравенства означало бы и гибель стартапов. Уверены ли вы, охотники,
что хотите застрелить именно этого зверя? Это означало бы, что вы
изжили стартапы в своей собственной стране. Амбициозные люди уже
объехали пол мира, развивая свою карьеру, и стартапы в наше время
могут работать отовсюду. Таким образом, если бы вы лишили людей
возможности разбогатеть, создавая богатство в своей стране, то
амбициозные люди просто были бы вынуждены уехать из нее и сделать это
то где-то в другом месте. Это, конечно, снизило бы коэффициент Джини,
а заодно и послужило бы уроком — надо быть осторожным в своих желаниях
(см. примечание [4] в конце статьи).

Я думаю, что рост экономического неравенства является неизбежным для
стран, которые не выбрали что-то худшее. В середине XX века у нас был
отрезок в 40 лет, который часть людей убедил в обратном. Но, как я уже
объяснял в одном из моих недавних постов The Refragmentation), это
была аномалия — уникальное сочетание обстоятельств, которые сжали
американское общество не только экономически, но и культурно (см.
примечание [5] в конце статьи).

Частично рост экономического неравенства с тех пор был заслугой
отрицательных факторов, но одновременно с этим выросли и возможности
для создания богатства. Практически все стартапы являются продуктами
того периода. И даже не в мире стартапов последнее десятилетие
наблюдаются качественные изменения. Технологии уменьшили затраты при
открытии стартапа настолько, что теперь основатели имеют власть над
инвесторами. Для основателей теперь типично дальнейшее удержание
контроля над компанией. И те, и другие далее способствуют росту
экономического неравенства: во-первых, потому что основатели сохраняют
больше акций, а во-вторых, потому что инвесторы уже поняли, что
основатели лучше них справляются с управлением компании.

Начиная с 1970-х уровень экономического неравенства в США сильно
возрос. В частности, богатые становились и становятся еще богаче. Для
некоторых это является знаком раскола общества внутри страны.

Меня интересует данная тематика так как я сам являюсь одним из
создателей экономического неравенства. Я был одним из основателей
компании под названием Y Combinator, которая помогает людям основывать
стартапы. Практически по определению основатели становятся богатыми
людьми, если стартап успешный. И даже если богатство не является
единственной целью основателей стартапов, многие становятся таковыми,
и лишь единицы — нет.

И если внешние проявления переменчивы, то глубинные предпосылки очень
и очень стары. Увеличение продуктивности, которое мы наблюдаем в
Кремниевой Долине, происходит уже на протяжении тысяч лет. Если вы
посмотрите на историю каменных инструментов, то сделаете вывод, что
технологии уже развивались в мезолите. Эти изменения были слишком
медленными, чтобы можно было за ними проследить в течение одной жизни.
Такова природа левой части экспоненциальной кривой. Но это была все та
же кривая.

Никто не желает создавать общество, которое было бы несовместимо с
этой кривой. Эволюция технологий — это одна из самых мощных сил в
истории.

Луис Брандейс (Louis Brandeis) говорил: «У нас может быть либо
демократия, либо богатство, сосредоточенное в руках небольшого числа
людей, но нам не дано получить и то и другое». Это похоже на правду.
Но если мне пришлось бы выбирать между его утверждениями и
экспоненциальной кривой, которой подчиняется миру уже тысячи лет, то я
делаю ставку на последнюю. Игнорирование любой тенденции, которая
актуальна тысячи лет, в высшей степени небезопасно. Но
экспоненциальный рост требует особого внимания.

И если ускорение изменений в продуктивности всегда способствует росту
экономического неравенства, то неплохой идеей было бы посветить
некоторое время на размышления о будущем. Возможно ли создание
здорового общества с огромными различиями в богатстве? Как бы оно
выглядело?

Подумайте, как ново кажется рассуждать об этом. Общественность до сих
пор была озабочена исключительно необходимостью уменьшения уровня
экономического неравенства. Мы едва ли задумывались над тем, как жить
с этим.

Я надеюсь, что у нас это получится. Брандейс был продуктом
«позолоченного века», и с тех пор многое изменилось. Сейчас куда
сложнее сокрыть правонарушения. И для того чтобы разбогатеть, не нужно
подкупать политиков, как это делали нефтяные или железнодорожные
магнаты (см. примечание [6] в конце статьи). Похоже на то, что
огромная концентрация богатства, которое крутится в Кремниевой Долине,
не разрушает демократию.

В США есть масса негативных факторов, симптомом которых является
экономическое неравенство. Нам нужно бороться с этими факторами, и
тогда, может быть, нам удастся уменьшить и неравенство. Но нельзя
начинать с врачевания симптомов и надеяться исправить первопричины
(см. примечание [7] в конце статьи).

Самая очевидная причина — это бедность. Я уверен, что большинство
людей, которые ратуют за уменьшение уровня экономического неравенства,
в основном хотят помочь бедным, а не навредить богатым (см. примечание
[8] в конце статьи). В действительности большинство просто путают
понятия, и, говоря об уменьшении экономического неравенства, они имеют
в виду уменьшение уровня бедности. Но это как раз та ситуация, в
которой не помешает быть точными, описывая свои желания. Бедность и
экономическое неравенство не идентичны. Когда вам отключают воду за
неуплату, то совершенно не важно, насколько основатель Google Ларри
Пейдж богаче вас. Он мог бы быть лишь в несколько раз богаче вас, но
это не повлияло бы на вашу проблему с отключенной водой.

Тесно связано с бедностью отсутствие социальной мобильности. Я сам был
свидетелем данного явления: вы не обязаны быть богатым или выходцем из
верхнего слоя среднего класса, чтобы создать стартап и разбогатеть, но
мало кто из основателей стартапов изначально живут в глубокой
бедности. Но опять же, проблемой здесь является не просто
экономическое неравенство. Несмотря на то, что Ларри Пейдж рос в
гораздо менее богатом доме, нежели многие основатели стартапов, ему
удалось войти в их ряды. По своей сути экономическое неравенство не
блокирует социальную мобильность, но дает специфическое сочетание
определенных факторов, которые негативно влияют на детей, если они
растут в крайней нищете.

Один из наиболее важных принципов Кремниевой Долины — «вы делаете то,
что вы измеряете». Это значит, что если вы выбрали какой-то
показатель, на котором вы сосредотачиваете свое внимание, то он будет
улучшаться, но вам нужно правильно выбрать показатель, потому что
улучшиться только он, другие же, даже концептуально близкие, могут
этого не сделать. К примеру, если вы директор университета, и вы
решили сосредоточить внимание на проценте выпускников, то процент
выпускников вырастет. Но это совсем не значит, что качество
образования тоже увеличится, — оно может и уменьшиться, если для роста
процента выпускников вы упростите учебную программу.

Экономическое неравенство совсем не является идентичным с различными
проблемами, симптомом которых оно является. Если наши усилия будут
направлены на борьбу с экономическим неравенством, то мы не решим этих
исходных проблем. Поэтому давайте обратим наше внимание на сами
проблемы.

К примеру, давайте бороться с бедностью, и наносить урон богатству в
процессе, если это необходимо. Это скорее всего будет иметь гораздо
большую эффективность, нежели бороться с богатством в надежде победить
бедность (см. примечание [9] в конце статьи). И если люди богатеют,
обманывая потребителей и лоббируя правительство для создания
антиконкурентных правил или налоговых лазеек, то давайте останавливать
их. Не потому, что это приводит к экономическому неравенству, а
потому, что это воровство (см. примечание [10] в конце статьи).

Все, что у вас есть, это статистика. И кажется, что именно она требует
доработок. Но за такими статистическими показателями, как
экономическое неравенство всегда стоят различные факторы — как
хорошие, так и плохие. Некоторые из них являются историческими
трендами, имеющими огромное влияние, другие же — простыми
случайностями. Если мы хотим улучшить мир, который призвана отобразить
статистика, мы должны его понять и сосредоточить свои усилия там, где
они будут наиболее полезны.

Примечания

[1] Стиглиц, Джозеф. «Цена неравенства» (The Price of Inequality).
Norton, 2012, c. 32.

[2] Так как экономическое неравенство является вопросом резко
отклоняющихся значений, и эти значения оказались там, где оказались не
из-за тех причин, о которых обычно не задумываются экономисты (отдавая
предпочтение таким факторам, как заработная плата и
производительность), а скорее, скажем, из-за того, что в войне с
наркотиками некоторые люди оказались не с той стороны баррикад.

[3] Целеустремленность является самым важным фактором, определяющим
успех или поражение, который в стартапах сильно дифференцирован. Но
для создания одного из в высшей степени успешных стартапов
недостаточно одной целеустремленности. Хотя большинство основателей в
начале взволнованы от идеи сколотить состояние, основатели, которыми
движут исключительно корыстные цели, как правило, продают успешные
стартапы, принимая предложения о поглощении, которые получают
большинство успешных молодых компаний. Те учредители, которые
продвигаются на следующую ступень, обычно считают, что у них есть
определенная миссия, и они ее выполняют. Они так же привязаны к своим
компаниям, как писатель или художник — к своей работе. Но на первых
порах очень нелегко определить, кто именно из основателей будет именно
таким. Это не просто отражение их изначального отношения. Основание
компании меняет людей.

[4] После чтения черновика этой статьи Ричард Флорида рассказал мне,
что когда-то ему довелось беседовать с группой европейцев, которые
говорили, что хотели бы, чтобы Европа стала более предпринимательской
и похожей на Кремниевую Долину. Он сказал, что по определению это
приведет к росту неравенства. Европейцы подумали, что он сумасшедший —
они просто не могли этого принять.

[5] Экономическое неравенство уменьшается глобально. Но это происходит
главным образом из-за разрушения клептократий, которые раньше
доминировали во всех бедных странах. Как только произойдет
политическое выравнивание, мы вновь увидим рост экономического
неравенства. Наша страна стала первопроходцем. Ситуация, с которой мы
сталкиваемся в США, рано или поздно затронет и другие страны.

[6] Некоторые люди до сих пор богатеют, подкупая политиков. Я же хочу
акцентировать внимание именно на том, что это больше не является
обязательным условием получения богатства.

[7] Существуют такие проблемы, симптомом которых является
экономическое неравенство, а есть и такие, для которых оно является
причиной. Но для большинства, если не для всех таких проблем
экономическое неравенство не является главной причиной. Как правило,
существует несправедливость, из-за которой экономическое неравенство
принимает другие формы неравенства, и с этой несправедливостью нам и
нужно бороться. К примеру, полиция в США к бедным относится хуже, чем
к богатым людям. Но решение не заключается в том, чтобы сделать людей
богаче. Просто нужно заставить полицию относиться к людям одинаково.
Иначе они и будут продолжать плохо обращаться с теми людьми, которые
так или иначе слабы.

[8] Некоторые читатели непременно скажут, что я невежественен или даже
целенаправленно ввожу людей в заблуждение, сосредотачивая внимание на
более богатой стороне экономического неравенства, и что на самом деле
экономическое неравенство связано именно с бедностью. Но это как раз
то, о чем я и говорю, каким бы запутанными не казались мои
рассуждения. Настоящая проблема — это бедность, а не экономическое
неравенство. И если вы отождествляете их, то вы целитесь не в ту
мишень.

Другие скажут то же самое: что я невежественен или ввожу в заблуждение
читателей, обращая внимание именно на людей, которые разбогатели,
создавая богатство, и что проблема не в стартапах, а в коррупции в
секторе финансов, здравоохранении и т. д. И опять же, это как раз то,
о что я и пытаюсь донести. Проблема не в экономическом неравенстве, а
в этих специфических негативных факторах.

Довольно странно писать статью, в которой ты пытаешься объяснить, что
какое-то явление не является проблемой, но приходится это делать в
надежде развеять заблуждения огромного количества людей.

[9] Особенно учитывая то, что многие причины бедности лишь частично
вызваны людьми, которые пытаются заработать на этом. К примеру, крайне
высокий уровень лишения свободы в США является главной причиной
бедности. Но хотя коммерческие тюремные компании и союзы тюремной
охраны занимаются лоббированием, чтобы замалчивались некоторые аспекты
карательной политики, тем не менее, они не являются их
первоисточником.

[10] Кстати, лазейки в налоговом законодательстве не являются
результатом каких-либо мощных изменений из-за недавнего роста
экономического неравенства. Золотой век экономического неравенства,
который пришелся на середину XX столетия, был также золотым веком
уклонения от налогов. Действительно, это было так широко
распространено и так эффективно, что я скептически настроен по поводу
суждений о том, что экономическое неравенство действительно тогда было
настолько низким, как мы считаем. В те периоды, когда люди пытаются
сокрыть свое богатство от правительства, они склонны скрывать его и от
статистики. Один из признаков потенциальной значительности этой
проблемы — это несоответствие между бюджетными поступлениями в
процентах ВВП, которые были более или менее постоянными со времен II
Мировой Войны и до сегодняшних дней, и налоговыми ставками, которые
значительно выросли.
\end{document}
