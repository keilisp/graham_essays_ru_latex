\documentclass[ebook,12pt,oneside,openany]{memoir}
\usepackage[utf8x]{inputenc} \usepackage[russian]{babel}
\usepackage[papersize={90mm,120mm}, margin=2mm]{geometry}
\sloppy
\usepackage{url} \title{Веб 2.0} \author{Пол Грэм} \date{}
\begin{document}
\maketitle
Означает ли что-нибудь словосочетание «Веб 2.0»? До последнего времени
я считал, что нет, но на самом деле вопрос оказался более сложным.
Изначально оно действительно было бессмысленным, но теперь у него
появилось значение. И всё же правы те, кому не нравится данный термин,
потому что если он означает то, что я думаю, он нам не нужен.

Впервые я услышал фразу «Веб 2.0» в названии соответствующей
конференции проходившей в 2004 году. В то время оно должно было
означать использование всемирной паутины как платформы, под чем я
понимаю веб-приложения. [1]

Поэтому я был удивлён, когда на конференции того лета Тим О'Рейли
возглавил секцию, назначением которой было выяснение значения термина
«Веб 2.0». Разве уже в то время он не означал использование веба в
качестве платформы? А если он ничего не означал, зачем он вообще был
нужен?

Происхождение Тим говорит, что в первый раз фраза «Веб 2.0» прозвучала
во время совместного мозгового штурма О'Reilly и Medialive
International. Что это за компания — Medialive International? По
информации с их сайта — «производители выставок и конференций,
посвящённых технологии». Так что, по-видимому, этому и был посвящён
совместный мозговой штурм. Компания О'Рейли хотела организовать
конференцию о всемирной паутине и они гадали как её назвать.

Я думаю, ни у кого не было намерения предложить новую версию веба. Они
только хотели сказать, что веб снова имеет значение. Это был своего
рода семантический аванс: они знали, что новые возможности на подходе
и номер «2.0» относился к ним, чем бы эти возможности не являлись.

И они оказались правы. Новые возможности были на подходе. Но новый
номер версии вызвал затруднение на первых порах. Во время подготовки
речи для первой конференции кто-то решил, что лучше попытаться
объяснить, к чему относится номер версии «2.0». Что бы это не значило,
«веб как платформа» звучало по крайней мере не слишком ограничивающе.

История о том, что «Веб 2.0» означает веб как платформу не надолго
пережила первую конференцию. Во время второй конференции, термин «Веб
2.0», казалось, означал что-то о демократии. По крайней мере, он
означал это, когда люди писали о нём в Интернете. Сама конференция
была не очень демократичной. Участие в ней стоило \$2800, поэтому
позволить себе присутствовать на ней могли только венчурные
капиталисты и представители больших компаний.

Тем не менее, как это ни странно, в статье Раяна Сингела (Ryan Singel)
в Wired News говорилось о «толпах гиков». Когда мой друг спросил об
этом Райана, для него самого это оказалось новостью. Он сказал, что
сначала написал нечто вроде «толпы венчурных капиталистов и
бизнесменов», затем сократил это до слова «толпы», которое, видимо,
было в свою очередь расширены редакторами до «толпы гиков». В конце
концов, на конференции, посвящённой Веб 2.0, должны быть толпы гиков,
верно?

Оказывается, нет. Их было около семи человек. Даже Тим О'Рейли был в
костюме — зрелище столь необычное, что я не сразу смог его узнать. Я
увидел его мимоходом и сказал одному из участников от О'Рейли: «Этот
парень выглядит точно как Тим».

«О, это Тим. Он купил костюм».

Я догнал его и, конечно, это был Тим. Он рассказал мне, что купил
костюм в Таиланде.

Конференция «Веб 2.0» напомнила мне Интернет-выставки времён
dotcom-пузыря, полные праздношатающихся венчурных капиталистов, ищущих
очередной модный стартап. Также как и прежде, присутствие большого
количества людей с намерением не упустить возможность, создавало
странную атмосферу. Не упустить что? Они не знали. Чего-либо, что
должно было появиться — чем бы не оказался Веб 2.0.

Я бы не назвал это «пузырём 2.0», потому что венчурные капиталисты
снова хотели инвестировать. Интернет является действительно
грандиозным событием. Банкротство было чрезмерной реакцией на бум.
Вполне ожидаемо, что как только начался выход из состояния
банкротства, произошёл значительный рост в данной области, точно так
же как это было в период Великой Депрессии с наиболее успешными
отраслями.

Причина, по которой это не повторный пузырь — то, что времена рынка
первичного размещения акций прошли. Венчурные инвесторы
руководствуются стратегиями выхода. Причиной, по которой они
финансировали все эти смехотворные стартапы в конце девяностых, была
надежда на продажу их доверчивым мелким инвесторам; они надеялись
посмеяться по дороге в банк. В настоящее время этот путь закрыт.
Сейчас стратегией выхода по умолчанию является продажа, а покупатели
менее расположены к бессмысленному швырянию деньгами, чем инвесторы.
Наиболее близкой к оценкам стоимости времён пузыря доткомов явилась
покупка Рупертом Мёрдоком (Rupert Murdoch) сайта Myspace за 580
миллионов долларов. Цена отличается лишь на один порядок или около
того.

1. Ajax Означает ли «Веб 2.0» больше, чем название конференции? Не
хочу соглашаться, но уже начинает означать. Когда сейчас говорят «Веб
2.0», у меня есть некоторое представление о том, что имеется в виду. А
тот факт, что я презираю данную фразу и понимаю её, является
безусловным доказательством того, что она начала что-то означать.

Одним из ингредиентов её значения является, конечно, Ajax, который я
только недавно начал употреблять без кавычек. По существу, «Ajax»
означает «Javascript наконец начал работать». А это, в свою очередь,
означает, что веб-приложения теперь могут стать более похожими на
традиционные.

По мере того, как вы читаете данное эссе, создаётся новое поколение
приложений, использующих преимущества Ajax. Такой волны новых
приложений не было со времени появления микрокомпьютеров. Даже
Microsoft видит это, но для них слишком поздно делать что-либо, кроме
как допускать утечки «внутренних» документов, предназначенные для
создания впечатления, что они лидируют в данной области.

На самом деле, новое поколение программ создаётся слишком быстро,
чтобы Microsoft могла повлиять на это, не говоря уж о создании
собственных разработок в данной области. Единственная их надежда —
скупить наиболее перспективные Ajax-стартапы прежде чем это сделает
Google. Но даже это будет слишком сложно, потому что Google обладает
таким же преимуществом в покупке микростартапов, как и несколько лет
назад в области поиска. В конце-концов, Google Maps, каноническое
Ajax-приложение, было создано приобретённым стартапом.

По иронии судьбы, первоначальное название конференции «Веб 2.0»
оказалось отчасти верным: веб-приложения являются важным компонентом
Веб 2.0. Но я убеждён, что они получили это право случайно. Бум Ajax
произошёл в начале 2005 года, с появлением сайта Google Maps и термина
Ajax.

2. Демократия Вторым важным компонентом Веб 2.0 является демократия.
Мы можем видеть примеры того, что любители могут превосходить
профессионалов, особенно когда у них есть система, направляющая усилия
в нужное русло. Википедия — один из самых известных примеров. Эксперты
характеризируют качество содержимого Википедии как среднее, но они
упускают важный момент: оно достаточно хорошо. И оно бесплатно, а это
означает, что люди могут его читать. В Интернете, статьи, за которые
необходимо платить, всё равно что не существуют. Даже если вы готовы
заплатить за возможность их прочтения, вы всё равно не можете
поставить на них ссылку. Они не могут стать предметом обсуждения.

Другим выигрышным аспектом демократии является решение того, что
является новостями. Я больше не читаю никаких новостных сайтов, кроме
Reddit. [2] Я знаю, что если случилось что-то важное, или кто-либо
написал чрезвычайно интересную статью, информация об этом появится на
Reddit. Зачем просматривать первые полосы какого-либо журнала или
газеты? Reddit как канал RSS для всех сайтов в Интернете, с фильтром
качества. К числу подобных сайтов относится Digg, сайт с техническими
новостями, быстро догоняющий Slashdot по популярности, и del.icio.us —
сеть для хранения и обмена закладками, положивший начало применению
тегов. Главная привлекательная особенность Википедии — то, что её
содержимое достаточно качественно и свободно, а данные сайты
демонстрируют, что распределённое сообщество может выполнять некоторые
задачи намного лучше, чем редакторы выделенные специально для
выполнения данных задач.

Наиболее значительный пример демократии Веб 2.0 — не отбор идей, а их
производство. Некоторое время назад я заметил, что статьи, прочитанные
мной на персональных сайтах, часто настолько же хороши или даже лучше
чем те, которые я читаю в газетах и журналах. И теперь у меня есть
объективное доказательство: наиболее популярные ссылки на Reddit чаще
ведут на персональные сайты, а не на журнальные статьи или газетные
материалы.

Мой опыт написания статей для журналов подсказывает ответ: редакторы.
Они выбирают темы, на которые вы можете писать, и, в общем случае,
могут переписать всё, что вы напишете. В результате происходит
фильрация. Благодаря редактированию статьи улучшаются на 95\%, но
одновременно ухудшаются на 5\%. Из-за этих 5\% получаются «толпы
гиков».

Люди могут публиковать в Интернете всё что хотят. Почти весь
опубликованный таким образом контент не прошёл бы редакторской
проверки в печатных изданиях. Но общее количество писателей очень,
очень велико. Если оно достаточно велико, отсутствие фильтрации
означает, что лучшая статья в Интернете может превзойти лучшую
печатную статью. [3] Теперь, когда в Интернете появились средства
отбора лучших произведений, в целом Интернет начал выигрывать. Отбор
побеждает фильтрацию по той же причине, по которой рыночные экономики
побеждают экономики с централизованным планированием.

В настоящее время изменились даже стартапы. Они относятся к стартапам
времён пузыря как блоггеры к печатным изданиям. Во времена пузыря,
стартап представлял собой компанию, возглавляемую магистром делового
администрирования, пускающую на ветер миллионы долларов венчурных
инвесторов, чтобы «быстрее стать большой» в буквальном значении
данного выражения. Сейчас данное слово означает более молодую, более
технически подготовленную группу людей, решивших создать что-либо
значительное. Со временем, они решат, необходимо ли привлекать
венчурные инвестиции, и если даже решат привлечь их, то сделают это на
своих условиях.

3. Не обращайтесь плохо с пользователями Мне кажется, что все
согласятся с тем, что декмократия и Ajax являются элементами «Веб
2.0». Я также вижу третий элемент: отсутствие плохого обращения с
пользователями. Во времена пузыря, большинство популярных сайтов вели
себя высокомерно по отношению к пользователям. Не всегда это было
очевидно, как, например, необходимость регистрироваться или
раздражающие баннеры. Сам дизайн среднего сайта 90-х был
отвратительным. Большинство наиболее популярных сайтов было
перегружено навязчивым брендингом, замедляющим загрузку и говорящим
пользователю: это наш сайт, а не ваш. (Аналогичное явление существует
в физическом мире в виде наклеек Intel и Microsoft на некоторых
ноутбуках.)

Думаю, суть проблемы была в том, что большинство сайтов считало, что
они отдают что-либо бесплатно, а до недавнего времени, компания
отдающая что-либо бесплатно, могла относиться к этому высокомерно.
Иногда это доходило до экономического садизма: владельцы сайтов
считали, что чем больше страданий они причинят пользователю, тем
больше будет их выигрыш. Выдающимся пережитком данного подхода
является сайт salon.com, на котором можно прочитать начало истории, но
чтобы дочитать до конца необходимо просмотреть целый фильм.

В нашей компании Y Combinator, мы советуем финансируемым нами
стартапам никогда не относиться к пользователям высокомерно. Никогда
не заставляйте пользователей регистрироваться, кроме тех случаев,
когда необходимо хранить какую-либо информацию для них. Если же вы
настаиваете на регистрации пользователей, никогда не заставляйте их
ждать получения ссылки с подтверждением по email. На самом деле, даже
не спрашивайте адрес электронной почты, если он действительно не нужен
по какой-либо причине. Не задавайте ненужные вопросы. Не посылайте им
электронные письма, кроме тех, что были явно запрошены. Не открывайте
ссылки в фреймах или новых окнах. Если у вас есть платная и бесплатные
версии, не делайте бесплатную версию слишком ограниченной. Если вы
задаёте себе вопрос: «Должны ли мы позволить пользователю выполнять
операцию x?», отвечайте «да». Округляйте в сторону великодушия.

В статье Как создать стартап я советовал стартапам не пропускать
никого, подразумевая под этим, что нельзя позволить другой компании
предлагать более простое, дешёвое решение. Другим способом добиться
этого является предоставление пользователям больших возможностей.
Позволяйте пользователям делать то, что им хочется. Если вы этого не
делаете, а конкурент делает, то вы в опасности.

В этом смысле, iTunes является в некотором роде сервисом Web 2.0.
Наконец-то появилась возможность покупать треки по отдельности, а не
целыми альбомами. Звукозаписывающие компании ненавидели данную идею и
долго сопротивлялись её введению. Но это очевидно было тем, в чём
нуждались пользователи, поэтому Apple обскакала данные компании. [4]
Хотя было бы правильнее охарактеризовать iTunes как сервис Web 1.5.
Применительно к музыке, Web 2.0 наверное означал бы бесплатную раздачу
записей без цифрового управления правами (DRM) отдельными группами.

Наилучшее отношение к пользователям — это бесплатная раздача контента,
за который конкуренты берут деньги. В 90-ых годах множество людей
считало, что к настоящему времени у нас будет какая-либо работающая
система микроплатежей. На самом деле, развитие пошло в другом
направление. Наиболее успешные сайты — те, которые нашли новые способы
раздавать контент бесплатно. Сайт Craigslist уничтожил значительную
часть сайтов с тематическими рекламными объявлениями, а OkCupid,
кажется, сделает то же самое с предыдущим поколением сайтов знакомств.

Обслуживание веб-страниц обходится очень, очень дёшево. Даже если если
просмотр страницы приносит менее цента дохода, вы можете получить
прибыль. При этом технологии целенаправленного показа рекламы
продолжают совершенствоваться. Не удивлюсь, если через десять лет eBay
заменит freeBay (или, что более вероятно, gBay), существующий за счёт
показа баннеров.

Хотя это звучит странно, мы советуем стартапам, что они должны
стараться зарабатывать как можно меньше. Если вы можете найти способ
превратить отрасль с миллиардными доходами в отрасль с миллионными
доходами, тем лучше, если все эти миллионы попадут в ваш карман.
Впрочем, на самом деле, удешевление вещей нередко в конечном счёте
приводит к увеличению прибыли, так же как автоматизация производства
часто приводит к увеличению количества рабочих мест.

Основной целью является Microsoft. С каким звуком лопнет этот пузырь,
когда кто-либо проткнёт его предложив бесплатную альтернативу MS
Office, реализованную в виде веб-приложения. [5] Кто это будет?
Google? Кажется, они замешкались. Подозреваю, что игла окажется в
руках пары 20-летних хакеров, слишком наивных, чтобы испугаться этой
идеи. (Насколько сложно реализовать данную затею?)

Общее направление Ajax, демократия и отсутствие пренебрежительного
отношения к пользователям. Что общего в этих вещах? До недавнего
времени я не осознавал, что в них есть что-то общее, и это являлось
одной из причин, по которой мне так не нравился термин «Веб 2.0».
Казалось, что он используется как ярлык для всего нового — что он
ничего не предсказывает.

Но существует общее направление. Веб 2.0 означает использование веба
так как он должена быть использован. «Тенденции», которые мы сейчас
наблюдаем, являются просто первоначальной природой всемирной паутины,
пробивающейся сквозь неправильные модели наложенные на неё во время
пузыря.

Я осознал это, когда читал ещё не опубликованное тогда интервью с Джо
Краусом (Joe Kraus), сооснователем Excite. [6]

Бизнес-модель Excite никогда не была верной. Мы натолкнулись на
классическую проблему того, как при появлении новой среды
распространения информации, она перенимает обычаи, содержимое и
бизнес-модели старой среды, что приводит к провалу. После этого,
придумываются новые модели. Казалось бы, что в годы, последовавшие за
пузырём, происходило немногое. Но при взгляде назад, становится
очевидно, что кое-что всё же происходило: всемирная паутина искала
оптимально сбалансированное положение. Демократия, например, не
является инновацией, в том смысле, что кто-то приложил усилия, чтобы
она появилась. Она является тем, что естественно вытекает из природы
всемирной паутины.

То же самое касается идеи предоставления через сеть приложений,
подобных традиционным. Эта идея почти так же стара, как и сама сеть.
Сначала она была кооптирована Sun и мы получили Java-апплеты. С тех
пор, Java была преобразована в обобщённую замену C++, но в 1996-ом
году Java представляла собой новую модель программного обеспечения.
Вместо традиционнх приложений, вы бы запускали «апплеты» Java,
полученные с сервера.

Но этот план рухнул под собственной тяжестью. Его добила Microsoft, но
он умер бы в любом случае. Он не был воспринят хакерами. Когда вы
видите, что фирмы, занимающиеся связями с общественностью рекламируют
нечто, как следующую платформу для разработчиков, можете быть уверены,
что оно ей не является. Если бы это было не так, данные фирмы не
потребовались бы, чтобы сказать вам об этом, потому что хакеры уже
создавали бы творения на основе этой платформы, как, например,
Busmonster использовал Google Maps в качестве платформы ещё до того,
как Google предполагала подобное использование.

Доказательство, что Ajax является следующей популярной платформой —
это то, что тысячи хакеров спонтанно начали создавать приложения на
его основе.

Существует ещё одна вещь, присущая всем трём компонентам Веб 2.0.
Подсказка. Предположим, вы пришли к инвесторам со следующей идеей
стартапа Веб 2.0:

Сайты, такие как del.icio.us и flickr, позволяют пользователям
назначать содержимому метки с наглядным описанием. Но существует также
огромное количество неявных тегов, которые они игнорируют: текст
веб-ссылок. Более того, эти ссылки представляют собой социальную сеть,
соединяющую индивидуумов и организации, создавшие страницы, и, при
помощи теории графов, мы можем вычислить на основе данной сети,
репутацию каждого члена. Мы планируем перекопать всемирную паутину в
поисках данных неявных тегов и использовать их, совместно с иерархией
репутаций, для улучшения поиска во всемирной сети. Как много времени
потребуется, чтобы понять, что это описание того, чем занимается
Google?

Google был первым, кто начал применять все три компонента Веб 2.0:
основные идеи их бизнеса описываются в терминах Веб 2.0. «Не
обращаться с пользователями плохо» является более слабой версией
слогана Google «Не быть плохими», и, конечно, весть этот Ajax-бум
начался с Google Maps.

Веб 2.0 означает использование веба так как он должен быть
использован, и Google это делает. Это их секрет. Они плывут под
парусами, вместо того, чтобы оставаться на месте, пережидая штиль и
молясь о появлении бизнес-модели, как печатные издания, или изменить
направление подавая в суд на своих клиентов, как Microsoft или
звукозаписывающие компании. [7]

Google не пытается изменить естественный порядок вещей. Они пытаются
догадаться, что будет дальше, чтобы оказаться в этом месте заранее.
Это правильный подход к технологии, и, так как бизнес включает в себя
значительную часть технологии, правильный подход к бизнесу.

Тот факт, что Google является компанией «Веб 2.0» ясно демонстрирует,
что этот термин неправильный, несмотря на явную осмысленность. Он как
слово «аллопатический». Оно означает правильное положение вещей,
поэтому плохо, когда для него придумывается специальное слово.

Примечания [1] С сайта конференции, Июнь 2004: «В то время, как первое
поколение веб-сайтов было тесно привязано к браузерам, второе
поколение распространяет возможности приложений на всемирную сеть и
создаёт возможности для нового поколения служб и бизнес-возможностей».
В каком-то смысле, это, видимо, означает веб-приложения.

[2] Признание: Reddit был профинансирован нашей компанией Y
Combinator. Но, хотя я начал использовать его в знак признательности
нашей команде, скоро я стал заядлым его приверженцем. Здесь же стоит
признаться в том, что я инвестировал в Microsoft, но продал все мои
акции в этом году.

[3] Я не против редактирования. Большая часть моего времени уходит на
редактирование, а не написание статей и у меня есть несколько
придирчивых друзей, которые вычитывают почти всё, что я пишу. Я не
люблю редактирование кем-либо другим после публикации.

[4] Это ещё мягко сказано. Пользователи лазили через окно годами,
прежде чем Apple наконец открыла дверь.

[5] Подсказка: способ создания веб-альтернативы Microsoft Office
заключается не в самостоятельном написании каждого компонента, а в
создании протокола для веб-приложений, позволяющего совместно
использовать виртуальный домашний каталог, распределённый по
нескольким серверам. А может быть и в том, чтобы написать всё
самостоятельно.

[6] Интервью из книги Джессики Ливингстон (Jessica Livingston)
Основатели за работой, которая должна быть опубликована издательством
O'Reilly в 2006-ом году.

[7] Microsoft не пыталась засудить своих клиентов напрямую, но
постаралась сделать всё возможное, чтобы помочь преуспеть в этом SCO.


\end{document}
