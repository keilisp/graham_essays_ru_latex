\documentclass[ebook,12pt,oneside,openany]{memoir}
\usepackage[utf8x]{inputenc} \usepackage[russian]{babel}
\usepackage[papersize={90mm,120mm}, margin=2mm]{geometry}
\sloppy
\usepackage{url} \title{Пугающе амбициозные идеи стартапов}
\author{Пол Грэм} \date{}
\begin{document}
\maketitle

Одна из наиболее удивительных вещей, которые я заметил, работая в Y
Combinator, это какими пугающими бывают самые амбициозные идеи
стартапов. В этой статье я собираюсь продемонстрировать этот феномен
посредством описания некоторых из них. Любая из них может сделать вас
миллиардером. Это может звучать как привлекательная перспектива,
однако когда я описываю эти идеи, вы можете заметить, как съеживаетесь
от них.

Не волнуйтесь, это не признак слабости. Вероятно это признак
вменяемости. Крупнейшие идеи стартапов выглядят пугающе. И не просто
потому, что они потребовали бы много работы. Крупнейшие идеи похоже
угрожают вашей личности: вы удивительны, если у вас достаточно
амбиций, чтобы довести их до конца.

Есть сцена в «Быть Джоном Малковичем», где занудный герой встречает
очень привлекательную, изысканную женщину. Она говорит ему:

Дело вот в чем: Если вы когда-нибудь заполучили меня, вы бы не имели
понятия что со мной делать.

Это то, что говорят эти идеи нам.

Этот феномен является одной из самых важных вещей, которые вы можете
понять о стартапах. [1] Вы ожидали, что идеи большого стартапа будет
привлекательной, но, в действительности, они, как правило, отталкивают
вас. И это имеет кучу последствий. Это означает, эти идеи невидимы для
большинства людей, которые пытаются думать об идеях стартапа, потому
что их подсознание отфильтровывает их. Даже самые амбициозные люди,
вероятно, в лучшем случае подступаются к ним не напрямую.

1. Новый поисковый движок. Лучшие идеи есть только на лицевой стороне
невозможного. Я не знаю, возможна ли именно эта идея, но есть
признаки, что она может быть возможна. Создание нового поисковика
означает конкурировать с Google, и в последнее время я заметил
некоторые трещины в их крепости.

Момент, когда мне стало ясно, что Microsoft потеряла свой путь, был
когда они решили войти в поисковый бизнес. Это было не естественным
шагом для Microsoft. Они делали это потому, что они боялись Google, и
Google был в поисковом бизнесе. Но это означает (а) Google был теперь
поставлен на повестку дня у Microsoft, и (б) повестка дня Microsoft
состояла из не очень хороших вещей.

Microsoft : Google :: Google : Facebook.

Это само по себе не означает, что нет места для новой поисковой
системы, но в последнее время, когда использование поиска Google я
нашел, что ностальгирую по старым дням, когда Google был верен своему
собственному еле различимому Я. Google использовался, чтобы выдать мне
страницу правильных ответов, быстро, без лишнего шума. Сейчас
результаты кажутся вдохновленными принципом сайентолога, где верно то,
что верно для вас. И страницы не имеют чистоты, редкое ощущение,
которое было привычным. Результаты поиска Google обычно выглядели как
вывод Unix утилиты. Сейчас, если я случайно наведу курсор не в то
место, может случиться все что угодно.

Путь к победе здесь это создать поисковик, который используют все
хакеры. Поисковик, чьи пользователи состоят из лучших 10 тысяч хакеров
и никто не был бы в более выгодном положении, несмотря на его малый
размер, как и Google, когда он был таким поисковиком. И впервые за всё
десятилетие идея перейти на него выглядит для меня возможной. Так как
кто-нибудь, способный запустить такую компанию это один из тех 10
тысяч хакеров, путь по крайней мере прост: делать поисковик, который
хотите вы сами. Вы вольны сделать его чрезмерно любительским. Сделать
его действительно хорошим для поиска кода, например. Хотели бы вы,
чтобы поисковые запросы были Тьюринг-полными? Всё, что получаете вы от
этих 10 тысяч пользователей, хорошо в силу самого факта.

Не волнуйтесь, если что-то что вы хотите делать будет вас ограничивать
в долгосрочной перспективе, потому что если вы не получите это
первоначальное ядро пользователей, не будет долгосрочной перспективы.
Если вы можете просто сделать что-то, что вы и ваши друзья искренне
предпочтете Google’у, вы уже на 10\% пути к IPO, как был Facebook
(хотя они, вероятно, не понимали этого), когда они заполучили всех
старшекурсников Гарварда.

2. Замените Email Электронная почта не предназначена для использования
так, как мы сейчас ее используем. Эл. почта это не протокол обмена
сообщениями. Это список задач. Или, скорее, мой почтовый ящик это
список задач, и эл. почта это способ, как все это получить. Но это
катастрофически плохой список задач.

Я открыт для различных типов решения этой проблемы, но я подозреваю,
что настройки почтового ящика не достаточно, и что эл. почта должна
быть заменена новым протоколом. Этот новый протокол должен быть
протоколом списка задач, не протоколом обмена сообщениями, хотя есть
вырожденный случай, когда кто-то хочет от вас, чтобы вы прочитали
следующий текст.

Как протокол списка задач, новый протокол должен давать бОльшие
возможности получателю, чем дает эл. почта. Я хочу, чтобы там было
больше ограничений на то, что кто-то может положить в мой список дел.
И когда кто-то может положить что-то в мой список дел, я хочу, чтобы
они рассказали мне больше о том, чего они хотят от меня. Они хотят,
чтобы я сделал нечто большее, чем простого прочитал текст? Насколько
это важно? (Там, очевидно, должен быть какой-то механизм,
препятствующий тому, чтобы люди говорили, что важно всё.) Когда это
должно быть сделано?

Это одна из тех идей, которые похожи на непреодолимую силу встречи с
недвижимым объектом. С одной стороны, укоренившиеся протоколы
невозможно заменить. С другой – кажется неправдоподобным, что люди
через 100 лет все еще будут жить в том же email аду, что и мы сейчас.
И если эл. почта в конечном счете будет заменена, почему не сейчас?

Если вы сделаете все правильно, вы сможете избежать обычной «проблемы
курицы и яйца» от столкновения с новым протоколом, потому что
несколько самых влиятельных людей в мире будут одними из первых, кто
перейдет на него. Они все тоже зависимы от эл. почты.

Что бы вы не создавали, делайте это быстрым. Gmail стал мучительно
медленным.[2] Если вы делаете что-то не лучше, чем Gmail, но быстрее,
это одно могло бы позволить вам начать перетягивать пользователей с
Gmail.

Gmail медленный, потому что Google не может позволить себе много
тратить на нее. Но люди будут платить за это. Для меня не проблема
платить \$50 в месяц. Учитывая, сколько времени я трачу в эл. почте,
страшно подумать, насколько оправданной может быть оплата. По меньшей
мере \$1000 в месяц. Если я провожу несколько часов в день на чтение и
написание эл. почты, то это был бы дешевый способ сделать мою жизнь
лучше.

3. Замените университеты В последнее время повсюду встречаются люди с
этой идеей, и я, думаю, в этом что-то есть. Я неохотно соглашаюсь, что
с институтом, который просуществовал в течение тысячелетия, покончено
просто из-за некоторых ошибок, которые они сделали в последние
десятилетия, но, конечно, в последние десятилетия университеты США,
кажется, движутся в неверном направлении. Можно было бы сделать
гораздо лучше за гораздо меньшие деньги.

Я не думаю, что университеты исчезнут. Они не будут массово заменены.
Они просто потеряют монополию де-факто на определенные типы обучения,
на которые раньше имели. Будет множество способов изучать различные
вещи, и некоторые могут выглядеть весьма отлично от университетов. Y
Combinator сам, пожалуй, один из них.

Обучение является такой большой проблемой, что изменения, которые
делают люди, будут иметь волну второго эффекта. Например, само
название университета ушло и воспринимается большинством людей(верно
или нет) как подтверждение собственного права заниматься чем-либо.
Если обучение разбить на много маленьких кусочков, дипломирование от
него можно отделить. Там может быть даже необходимы замены в
студенческой социальной жизни(и как ни странно, у Y Combinator есть на
это свои взгляды).

Вы можете заменить вузы тоже, но там вы лицом к лицу столкнетесь с
бюрократическими препятствиями, которые замедлили бы стартап.
Университеты выглядят местом для начала замен.

4. Интернет драма. Голливуд не торопиться охватить Интернет. Это
ошибка, потому что, я думаю, сейчас мы можем назвать победителя в
гонке между механизмами доставки, и это Интернет, а не кабельное.

Главная причина, ужас клиентов кабельного, также известна как
телевизоры. Наша семья не ждет Apple TV. Мы ненавидим наш последний
телевизор так сильно, что пару месяцев назад мы заменили его на iMac,
прикрученный болтами к стене. Немного неудобно управлять им
беспроводной мышью, но в общем это гораздо лучше, чем кошмарный
интерфейс, с которым мы имели дело прежде.

Некоторая часть внимания, которое люди в настоящее время уделяют
просмотру фильмов и ТВ, может быть украдена вещами, казалось бы,
совершенно не связанными, например приложения для социальных сетей.
Больше может быть украдено вещами, которые немного более тесно
связаны, как например игры. Но, вероятно, всегда будет остаточный
спрос на обыкновенную драму, где вы пассивно сидите и наблюдаете за
сюжетом. Так как вы доставите драму через Интернет? Что бы вы ни
делали, это должно быть в большем масштабе, чем клипы на Youtube.
Когда люди садятся смотреть шоу, они хотят знать, что они получат: или
это очередная серия со знакомыми героями, или один большой фильм,
основной посыл которого они уже заранее знают.

Есть два способа доставки и оплаты, которые можно разыграть. Либо
некоторые компании, такие как Netflix или Apple сделают магазин
приложения для развлечения, и вы доберетесь до аудитории через них.
Или будущие магазины приложений будут слишком затратные, или слишком
негибкие технически, и возникнут компании, которые будут, на выбор,
снабжать оплатой и потоковой передачей данных производителей драмы.

5. Следующий Стив Джобс Я разговаривал недавно с тем, кто знал Apple
очень хорошо, и я спросил его, если сейчас люди запускают компанию,
могла бы она продолжать создавать новые вещи так, как Apple под
управлением Стива Джобса. Его ответ был просто «нет». Я уже испугался,
что это мог быть ответ. Я спросил еще, чтобы понять, как он это
определил. Но он вовсе не определил. Нет, больше не будет новых
великолепных вещей по ту сторону любого нынешнего конвейера. Доходы
Apple могут продолжать расти долгое время, но, как показала Microsoft,
доход – запаздывающий индикатор в технологичном бизнесе.

Таким образом, если не Apple собирается делать следующий iPad, то кто?
Никто из существующих игроков. Никто из них не запускался
дальновидными творческими гениями продукта, и опытным путем вы,
кажется, не можете получить их по найму. Эмпирически, способ получить
дальновидного творческого гения в качестве генерального директора –
это найти для него компанию и не увольнять. Так что компанией, которая
создает новую волну оборудования, вероятно, должен быть стартап.

Я понимаю, это звучит нелепо амбициозно для стартапа, чтобы пытаться
стать таким же крупным, как Appleподли. Но не более амбициозно, чем
было для Apple стать таким же крупным, как Apple, и они сделали это.
Кроме того, стартап, занимающийся этой проблемой сейчас имеет
преимущества, которого не было у нного Apple: пример Apple. Стив Джобс
показал нам, что это возможно. Это помогает потенциальным приемникам
непосредственно, как делал Роджер Баннистер , показывая, насколько
лучше можете делать вы, чем делали люди раньше, и косвенно, как делал
Августус, внедряя в разум пользователей идею о том, что один человек
может открыть будущее для них.[3]

Теперь Стив Джобс ушел, мы все чувствуем пустоту. Если новая компания
смело направится в будущее аппаратного обеспечения, пользователи
последовали бы за ней. Генеральный директор этой компании, “следующий
Стив Джобс”, может не достигнуть уровня Стива Джобса. Но он и не
обязан. Он просто должен делать работу лучше, чем Samsung и HP и
Nokia, и это выглядит вполне выполнимо.

6. Вернуть Закон Мура

Последние 10 лет напомнили нам, что Закон Мура говорит на самом деле.
Примерно до 2002 года вы могли спокойно перетолковывать его, как
обещание, что тактовая частота будет удваиваться каждые 18 месяцев. На
самом деле, он говорит, что плотность монтажа схемы будет удваиваться
каждые 18 месяцев. Раньше казалось педантичным указывать на это. Не
более. Intel не может больше давать нам более быстрые процессоры,
только больше процессоров.

Этот Закон Мура не так хорош, как старый. Закон Мура использовался
чтобы показать, что если ваше ПО было медленным, всё, что вы должны
сделать, это подождать, и неумолимый прогресс аппаратного обеспечения
решил бы ваши проблемы. Теперь, если ваше ПО медленное, вы должны
переписывать его, чтобы сделать больше вещей исполняемыми параллельно,
что является большей работой, чем ожидание.

Было бы здорово, если стартап мог вернуть нам что-то вроде старого
Закона Мура, за счет написания ПО, которое могло бы сделать большое
количество процессоров выглядящими для разработчика как один очень
быстрый процессор.Есть несколько способов решить эту проблему. Самый
амбициозный это попытаться сделать это автоматически: написать
компилятор, который будет распараллеливать наш код для нас. Название
для этого компилятора, достаточно умный компилятор, и это синоним
невозможного. Но действительно ли это невозможно? Нет ли конфигурации
бигтов в памяти нынешних компьютеров, которые были бы этим
компилятором? Если вы действительно так думаете, вам следует
попытаться доказать это, потому что результат может быть интересным. И
если это не невозможно, а просто очень трудно, может быть стоит
попробовать написать это. Ожидаемая ценность может быть высокой, даже
если шанс успеха был низким.

Причина, по которой ожидаемая ценность так высока это веб-сервисы.
Если вы можете написать ПО, которое дало бы программистам те же
удобства, как в старые добрые времена, вы могли бы предложить им его в
качестве веб-сервиса. И это могло бы, в свою очередь, означать, что вы
получили практически всех пользователей.

Представьте себе, что был бы другой производитель процессоров, который
мог бы продолжать преобразовывать увеличивающуюся плотность монтажа
схемы в увеличивающуюся тактовую частоту. Они забрали бы большую часть
бизнеса Intel’а. И поскольку веб-сервисы означают, что никто не видит
больше их процессоров, написав достаточно умный компилятор, вы можете
создать ситуацию, неотличимую от той, где вы являетесь таким
производителем, по крайней мере на рынке серверов.

Наименее амбициозный способ подойти к проблеме это начать с другого
конца, и предложить программистам больше распараллеливаемых
Лего-блогов, чтобы делать программы, как Hadoop и MapReduce. В таком
случае, программист по-прежнему делает большую часть работы по
оптимизации.

Тут есть интригующая середина, где вы строите полуавтоматическое
оружие - где в цепочке есть люди. Вы делаете что-то, что выглядит для
пользователя как достаточно умный компилятор, но внутри есть люди,
которые используют высокоразвитые инструменты оптимизации для
нахождения и устранения узких мест в программах пользователей. Эти
люди могут быть вашими сотрудниками, или вы можете создать рынок для
оптимизации.

Рынок оптимизации был бы путем к созданию достаточно умного
компилятора по частям, потому что участники будут незамедлительно
начинать писать ботов. Было бы любопытное положение дел, если вы могли
бы добраться до точки, где всё могли бы сделать боты, потому что тогда
вы сделали достаточно умный компилятор, но ни один человек не будет
иметь его полную копию.

Я понимаю как безумно это звучит. На самом деле, что мне нравится в
этой идее, это что в ней все различные пути неверны. Сама идея
фокусировки на оптимизации идет вразрез с общей тенденцией в
разработке ПО в течение последних десятилетий. Попытка написать
достаточно умный компилятор это, по определению, ошибка. И даже если
бы не была ошибкой, компиляторы это вид ПО, который должен создаваться
проектами с открытым исходным кодом, не компаниями. Плюс, если он
работает, он лишит всех программистов, которые с удовольствием делают
многопоточные приложения, такого большого количества смешной
сложности. Форумный троль, который к этому моменту уже появился внутри
меня, даже не знает с чего начать в наращивании возражений против
этого проекта. Вот что я называю идея для стартапа.

7. Непрерывная диагностика

Но подождите, вот еще кое-что, что может столкнуться с еще большим
сопротивлением: непрерывная, автоматическая медицинская диагностика.

Один из моих трюков для генерации идей стартапов это представить,
какими отсталыми мы будем выглядеть для будущих поколений. И я
абсолютно уверен, что для людей через 50 или 100 лет в будущем, будет
казать варварским, что люди в наше время ждали, пока у них появятся
симптомы, для того, чтобы поставить диагноз, такой как болезнь сердца
или рак.

Например, в 2004 Билл Клинтон обнаружил, что ощущает одышку. Врачи
выявили, что некоторые из его артерий были более чем на 90\%
заблокированы и 3 дня спустя ему поставили четверной шунт. Разумно
предположить, что у Билла Клинтона лучшее доступное медицинское
обслуживание. И все же даже он должен был ждать, пока его артерии
стали более чем на 90\% заблокированы, чтобы узнать, что число было
более 90\%. Конечно, в какой-то момент в будущем, мы будем знать эти
числа так же, как мы сейчас знаем, например, наш вес. Аналогично для
рака. Будет казаться нелепым для будущих поколений, что мы ждали, пока
у пациента появятся физические симптомы, чтобы диагностировать рак.
Рак будет сразу же отображаться на чем-то, вроде экрана радара.

(Конечно, то, что появляется на экране радара, может отличаться от
того, что мы сейчас представляем в качестве рака. Я не удивлюсь, если
в любой момент времени у нас десяток или даже сотни микро раков
существуют зараз, ни один из которых, обычно, не имеет значения.)

Много препятствий на пути к непрерывной диагностике выйдет из того
факта, что она идет против медицинской профессии. То, как медицина
всегда работала – пациент приходит к доктору с проблемой, и доктор
выясняет, в чем дело. Многим докторам не нравится идея медицинского
эквивалента того, что юристы называют «рыбалка», когда вы ищете
проблемы, не зная, что вы ищете. Они называют вещи, которые были
обнаружены таким способом «инциденталомами», и они являются своего
рода помехой.

Например, моя подруга, однажды ее мозг был просканирован в рамках
исследования. Она была в ужасе, когда доктора в процессе исследования
обнаружили то, что оказалось большой опухолью. После дополнительного
тестирования, оно оказалось безвредной кистой. Но это стоило ей
нескольких дней ужаса. Многие врачи беспокоятся, что если вы начнете
сканировать людей без симптомов, вы получите такое в гигантском
масштабе: огромное число ложных срабатываний, которые заставляют
пациентов паниковать и требуют затрат и, возможно, даже опасных
тестов. Но я думаю, это просто искажение из-за существующих
ограничений. Если бы люди сканировались все время и мы стали бы лучше
в решении того, что было реальной проблемой, моя подруга знала бы об
этой кисте всю ее жизнь и знала бы, что это безвредно, как родинка.

Здесь существует возможность для многих стартапов. В дополнение к
техническим трудностям, с которыми сталкиваются все стартапы, и
бюрократических сложностей, с которыми сталкиваются все медицинские
стартапы, они будут идти против тысяч лет медицинских традиций. Но это
произойдет, и это будет великой вещью – настолько великой, что люди в
будущем будут чувствовать жалость к нам, как и мы к поколениям,
которые жили до наркоза и антибиотиков.

Тактика

Позвольте мне в заключение некоторый тактический совет. Если вы хотите
взять на себя проблему настолько большую, как те, о которых я
рассуждал, не приступайте к ней напрямую в лоб. Не говорите, например,
что вы собираетесь заменить электронную почту. Если вы сделаете это,
вы взрастите слишком много ожиданий. Ваши сотрудники и инвесторы будут
постоянно спрашивать «уже готово?» и у вас будет армия ненавистников,
ожидающих увидеть ваш провал. Просто скажите, что вы делаете программу
TODO-лист. Это звучит безобидно. Люди могут заметить, что вы заменили
электронную почту, как свершившийся факт.[4]

На деле, способом делать действительно большие вещи, кажется, начинать
с обманчиво маленьких вещей. Хотите доминировать в микрокомпьютерном
ПО? Начните с написания Основного интерпретатора для машины с
несколькими тысячами пользователей. Хотите сделать универсальный сайт?
Начните с создания сайта для Гарвардских старшекурсников, чтобы
подкрасться от одного к другому.

На деле, не просто ради других людей вы должны начинать с малого. Вам
это нужно для вашей же пользы. Ни Билл Гейтс, ни Цукерберг не знали
сначала, насколько большими их компании собираются стать. Все, что они
знали, что в них что-то есть. Может быть, это плохая идея, иметь
действительно большие амбиции вначале, потому что чем больше ваши
амбиции, тем дольше это займет, и чем дальше ваш проект продвинется в
будущее, тем больше вы будете понимать, что он, скорее всего,
ошибочен.

Я думаю, способ использования этих больших идей – не пытаться
определить момент в будущем и затем спросите себя, как добраться
отсюда туда, как популярный образ мечтателя. Лучше будет, если вы
действуете как Колумб и просто направляетесь в основном на запад. Не
пытайтесь строить будущее как здание, потому что ваш текущий план
почти наверняка ошибочен. Начните с чего-то, что, как вы знаете,
работает, и когда вы расширяетесь, расширяйтесь на запад.

Популярный образ провидца – кто-то с чистым видением будущего, но на
деле может быть лучше, чтобы оно было туманное.

Примечания

[1] Это также одна из самых важный вещей, которую венчурные
капиталисты не понимают в стартапах. Большинство ожидает, что
основатели идут с четким планом будущего, и судят на основе этого.
Мало сознательно понимать, что в крупнейших успехах наименьшая
корреляция между первоначальным планом и тем, чем стартап в конце
концов станет.

[2] Это предложение изначально читается «Gmail мучительно медленный».
Спасибо Полу Бачхайт за поправку.

[3] Роджер Баннистер известен, как первый человек, пробежавший милю
менее чем за 4 минуты. Но его мировой рекорд выдержал только 46 дней.
Когда-то показав, что это можно сделать, многие другие последовали его
примеру. Десять лет спустя Джим Рьян пробежал милю за 3:59 будучи
старшеклассником.

[4] Если вы хотите быть следующей Apple, может быть вы даже не хотите
начинать с потребительской электроники. Может быть сначала вы сделаете
что-то, что используют хакеры. Или вы сделаете что-то популярное, но,
очевидно, незначительное, как наушники или маршрутизатор. Всё, что вам
нужно, это плацдарм.

\end{document}
