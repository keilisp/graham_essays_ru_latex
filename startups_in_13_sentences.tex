\documentclass[ebook,12pt,oneside,openany]{memoir}
\usepackage[utf8x]{inputenc} \usepackage[russian]{babel}
\usepackage[papersize={90mm,120mm}, margin=2mm]{geometry}
\sloppy
\usepackage{url} \title{13 главных принципов в жизни стартапа}
\author{Пол Грэм} \date{}
\begin{document}
\maketitle

Я непременно рассказываю основателям стартапов о принципе, который
узнал от Пола Букхайта: лучше сделать счастливыми немногих людей, чем
недостаточно счастливыми – множество людей. Недавно я говорил
журналисту, что если бы я мог рассказывать предпринимателям о 10
главных принципах, то этот был бы одним из них. Тогда я подумал: а
какие будут остальные девять?

Я составил список, и принципов оказалось 13:

1. Выбирайте подходящих сооснователей.

Сооснователи для стартапа – то же самое, что месторасположение для
недвижимости. В доме можно поменять практически все, за исключением
того участка земли, на котором он стоит. В стартапе легко поменять
основную идею, но поменять сооснователей трудно. И успех стартапа –
практически всегда производная от действий его сооснователей.

2. Запускайтесь быстро.

Дело не в том, что крайне важно вывести продукт на рынок пораньше, а в
том, что реальная работа над продуктом начнется именно тогда, когда он
уже запущен. Запуск дает вам понять, что именно вы должны были
создать. Так что главная ценность запуска – это повод для вовлечения
пользователей.

3. Пусть ваша идея эволюционирует.

Это логичное продолжение быстрого запуска. Запускайте быстро и меняйте
продукт. Большая ошибка – относиться к стартапу так, как будто все
дело в правильном исполнении некой гениальной изначальной идеи. Как и
при написании эссе, большинство идей возникают уже по ходу реализации.

4. Поймите своих пользователей.

Стоимость, создаваемую стартапом, можно представить в виде
прямоугольника, одна сторона которого – число пользователей, а другая
– насколько вы улучшаете их жизнь. Над второй стороной у вас больше
всего власти. И увеличение первой стороны будет зависеть от того,
насколько хорошо вы поработаете над второй. Как и в науке, самое
сложное – не отвечать на вопросы, а задавать их: самое трудное в том,
чтобы увидеть нечто новое, чего не хватает пользователям. Чем лучше вы
их понимаете, тем выше шансы, что у вас это получится. Вот почему так
много успешных стартапов создают то, в чем нуждались их основатели.

5. Лучше завоевать любовь немногих пользователей, чем неоднозначное
отношение многих.

В идеале вас должны полюбить широкие массы пользователей, но вряд ли
можно рассчитывать на это с самого начала. Изначально нужно выбрать
между удовлетворением всех потребностей некоторой части потенциальных
пользователей и удовлетворением части потребностей всех потенциальных
пользователей. Идите по первому пути. Такую модель легче расширять в
плане числа пользователей, чем в плане уровня удовлетворенности. А
самое главное, труднее врать самому себе. Если вы думаете, что на 85\%
подобрались к отличному продукту, то откуда вы знаете, что это именно
85\%, а не 70\%? Или не 10\%? А вот сколько у вас пользователей,
узнать легко.

6. Обеспечьте неожиданно хороший уровень обслуживания пользователей.

Потребители привыкли, что с ними обращаются недостойно. Большинство
компаний, с которыми люди сталкиваются – это квази-монополисты,
которым сходит с рук отвратительный сервис. И ваше собственное
представление о возможном и невозможном подсознательно находится под
влиянием подобного опыта. Попробуйте сделать ваш собственный сервис не
просто достойным, но неожиданно хорошим. Из кожи вон лезьте, чтобы
осчастливить клиентов. Они будут ошеломлены, вот увидите. На ранней
стадии стартапа вполне имеет смысл предоставлять сервис на таком
уровне, который потом невозможно расширить на массу пользователей –
для вас это способ узнать что-то новое о клиентах.

7. У вас получается то, что вы измеряете.

На эту мысль меня натолкнул Джо Краус. Даже когда вы что-то просто
измеряете, в этом измерении заложена возможность улучшить продукт.
Если вы хотите, чтобы у вас выросло число пользователей, то повесьте
на стену большой лист бумаги и каждый день заносите туда число
пользователей. Вы будете в восторге, когда показатель пойдет вверх, и
в расстройстве, когда он пойдет вниз. Довольно скоро вы начнете
замечать, благодаря чему показатель идет вверх, и вы начнете делать
акцент именно на это. Еще одно следствие: осторожнее в выборе того,
что вы измеряете.

8. Тратьте немного.

Для стартапа необычайно важно оставаться дешевым. Большинство
стартапов терпят крах до того, как им удается создать нечто, желаемое
людьми. И самая распространенная форма провала – это когда кончаются
деньги. Так что оставаться дешевым – это почти то же самое, что быстро
переориентироваться. Но и нечто большее. Культура экономии сохраняет в
компаниях молодость, как физические упражнения сохраняют молодость в
людях.

9. Добейтесь минимальной прибыльности.

Минимальная прибыльность в данном случае – это уровень прибыли,
достаточный, чтобы оплачивать жизненно необходимые расходы основателей
стартапа. Это способ увильнуть от многих трудностей инвестиционного
процесса. Как только вы достигаете такой минимальной прибыльности, это
полностью меняет ваши отношения с инвесторами. И это полезно для
поддержания морального духа.

10. Не отвлекайтесь.

Ничто не убивает стартапы так быстро, как склонность отвлекаться. Хуже
всего – та деятельность, что приносит деньги: основная работа,
консультирование, выгодные побочные проекты. Стартап, возможно, имеет
больший долгосрочный потенциал, но вы постоянно отвлекаетесь от работы
над ним, отвечая на звонки людей, что платят вам прямо сейчас. Как ни
парадоксально, сбор средств тоже относится к такой отвлекающей
деятельности, так что постарайтесь его минимизировать.

11. Боритесь с деморализацией.

Хотя непосредственной причиной смерти стартапа, как правило, бывает
конец финансирования, более глубокой причиной обычно оказывается
недостаточная сосредоточенность. Или компанией руководят глупые люди
(этого советами не исправишь), или люди умны, но деморализованы.
Работа в стартапе – это большой моральный груз. Поймите это и
сознательно старайтесь сделать так, чтобы этот груз поменьше на вас
давил. Вы же сгибаете колени, когда поднимаете тяжелый ящик?

12. Не сдавайтесь.

Даже если вы деморализованы, не сдавайтесь. Если вы попросту не
сдаетесь, вы можете зайти удивительно далеко. Так случается далеко не
во всех областях. Многие люди не смогут стать хорошими математиками,
как бы долго они ни упорствовали. Но в стартапах именно так. Одних
усилий бывает достаточно, если вы при этом продолжаете развивать свою
идею.

13. Сделки идут сами собой.

Один из первых полезных навыков, которым нас научил Viaweb – это не
слишком пылать надеждой. У нас намечалось порядка 20 разного рода
сделок. И примерно после первых 10 мы научились относиться к ним как к
фоновым процессам, которые следует игнорировать, пока они не
закончатся. Очень опасно, когда настрой команды зависит от того, как
завершаются сделки – и не только потому, что зачастую они не
завершаются ничем, но потому, что именно такое отношение подрывает
шансы на их успех.

Теперь, когда я выразил все это в 13 предложениях, время понять, какой
бы принцип я выбрал, если бы требовалось выбрать всего один.

Нужно понимать ваших пользователей – вот ключевой. Основная задача
стартапа – создать ценность; вы больше всего влияете на один аспект
этой ценности – насколько вы улучшаете жизнь пользователей; и самое
сложное – это понять, что именно нужно им предложить. Как только вы
это поймете, нужны просто усилия, чтобы это создать, и большинство
нормальных хакеров на это способны.

Понять пользователей – это основа доброй половины принципов в этом
списке. Именно поэтому нужно запускать продукт рано – чтобы понять
пользователей. Развитие идеи – это следствие возникающего понимания.
Понимание пользователей будет подталкивать вас к тому, чтобы создать
продукт, делающий глубоко счастливыми пусть даже небольшое число
людей. А главная причина, почему важно иметь неожиданно высокий
уровень сервиса – это то, что такой сервис помогает понять ваших
пользователей. А такое понимание будет укреплять моральный настрой
команды: даже если все вокруг рушится, достаточно будет десятка
пользователей, которые вас по-настоящему любят, чтобы продолжать
дальше.



\end{document}
