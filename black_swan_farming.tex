\documentclass[ebook,12pt,oneside,openany]{memoir}
\usepackage[utf8x]{inputenc} \usepackage[russian]{babel}
\usepackage[papersize={90mm,120mm}, margin=2mm]{geometry}
\sloppy
\usepackage{url} \title{Как отличать гениальные бизнес-идеи от
  никчемных} \author{Пол Грэм} \date{}
\begin{document}
\maketitle

За годы моей жизни я занимался несколькими довольно разными делами, но
я не знаю столь парадоксального занятия, как инвестирование в
стартапы. Две самых важных вещи, которые нужно понять об
инвестировании в стартапы как бизнесе, это: 1) практически всю прибыль
приносят несколько суперуспешных проектов; 2) самые лучшие идеи
вначале кажутся ужасными.

Первое правило я осознавал рационально, но не ощущал его в полной
мере, пока именно это не произошло с нами самими. Общая стоимость
компаний, в которые мы инвестировали — это плюс-минус 10 миллиардов.
Но три четверти этой суммы приходится всего на две компании — Dropbox
и Airbnb.

В сфере стартапов сверхуспешные игроки оказываются настолько велики,
что это разрушает все наши представления о вероятностях и вариации. Не
знаю, врожденные ли это представления или благоприобретенные, но как
бы то ни было, мы просто не готовы к той вариации, которая возникает
при инвестировании в стартапы — результаты разных компаний могут
отличаться в тысячу раз.

Это приводит к разного рода странным последствиям. К примеру, с чисто
финансовой точки зрения, в каждой серии стартапов нашего фонда Y
Combinator есть (вероятно) в лучшем случае одна компания, которая
существенно повлияет на наши доходы, а остальные представляют собой
лишь издержки ведения бизнеса. Я все еще не в полной мере усвоил этот
факт, отчасти в силу его парадоксальности, а отчасти потому, что мы
занимаемся всем этим не только по денежным мотивам: Y Combinator был
бы ужасно одиноким местом, если бы в каждой «партии» стартапов у нас
была бы только одна компания. И все же это так.

Чтобы добиться успеха в области, которая противоречит всем вашим
интуитивным представлениям, вам нужно научиться отключать их, как
летчик, пролетающий через облака. Вам нужно делать то, что правильно с
рациональной точки зрения, даже если это выглядит неправильным.

Для нас это постоянная битва. Трудно заставить себя идти на достаточно
серьезный риск. Когда говоришь с основателями стартапа и думаешь: «У
них вполне может получиться», трудно не профинансировать эту компанию.
И все же — по крайней мере, с финансовой точки зрения — успех бывает
только один: либо компания войдет в число суперуспешных игроков, либо
нет — и если нет, то не важно, профинансируете вы ее или нет,
поскольку даже в случае ее успеха отдача на инвестиции будет
незначительной. В тот же день собеседований вы можете наткнуться на
19-летних парней, которые даже не решили, чем им по-настоящему
заниматься. Их шансы на успех кажутся маленькими. Но опять-таки, важна
не вероятность их успеха. а вероятность, что они добьются реально
большого успеха. Шансы войти в число суперуспешных игроков
микроскопически малы для любой организации, но у кучки 19-летних эти
шансы могут быть выше, чем у другого стартапа, вложения в который
кажутся вам более безопасными.

Вероятность большого успеха — это не просто фиксированный процент от
вероятности успеха как такового. Если бы было так, то можно было бы
финансировать любую компанию, которая имеет шансы на успех как
таковой, и получать свою долю больших прорывов. Увы, выбирать
победителей труднее. Нужно игнорировать стоящего перед вами слона,
игнорировать вероятность, что эта компания добьется успеха, и вместо
этого сосредоточиться на самостоятельном и практически неощутимом
вопросе о том, сможет ли эта компания добиться реально большого
успеха.

Осложнения

Все осложняет тот факт, что лучшие идеи в мире стартапов сперва
выглядят дурными идеями. Я уже писал об этом: если бы хорошая идея
была очевидно хорошей, то ее уже реализовал бы кто-то еще. Поэтому
самые успешные стартаперы обычно работают над идеями, качество которых
мало кто, кроме них самих, осознает. Это смахивает на описание безумия
— пока вы не дойдете до точки, в которой видны результаты.

Когда Питер Тиль выступал в Y Combinator, он изобразил диаграмму
Венна, блестяще иллюстрирующую эту ситуацию. Он нарисовал два
пересекающихся круга, один помечен «Похоже, это плохая идея», другой —
«Хорошая идея». Их пересечение — это «золотая середина» для стартапов.

Концепция простая, но в виде диаграммы она многое проясняет. Она
напоминает нам, что существует область пересечения — хорошие идеи,
которые кажутся плохими. Она также напоминает нам, что подавляющее
большинство идей, кажущихся плохими, и вправду плохие.

В силу того факта, что хорошие идеи выглядят плохими, еще труднее
выявлять настоящих победителей. Это означает, что вероятность
действительно большого успеха для стартапа — это не просто процент от
вероятности его успеха, и более того, что стартапы с высокой
вероятностью большого успеха будут отличаться непропорционально низкой
вероятностью успеха как такового.

Большие успехи переписывают историю: в ретроспективе кажется
очевидным, что те или иные компании должны были преуспеть. Именно в
силу этого одно из самых ценных моих воспоминаний — то, как убого
звучала для меня идея Facebook, когда я впервые о ней услышал. Сайт,
где студенты будут убивать время? Казалось, это абсолютный пример
плохой идеи — сайт: 1) для нишевого рынка; 2) для безденежной
аудитории; 3) для занятий чем-то совершенно не важным.

Microsoft и Apple когда-то можно было описать в тех же выражениях.

Новые осложнения

Но стойте: все еще хуже. Вам не только предстоит решить эту сложную
проблему, вам еще и предстоит это сделать, не имея перед собой никаких
индикаторов, говорящих, преуспеваете вы или нет. Если вы и выбрали
будущего победителя, вы не будете знать об этом еще два года.

В то же время тот показатель, что вы все-таки можете измерить, опасен
и способен ввести вас в заблуждение. То, что мы можем точно отследить
— это насколько успешно стартапы привлекают деньги после демонстрации
своих проектов (Demo Day — презентации для инвесторов, которые дважды
в год проводят стартапы, выбранные Y Combinator). Но мы знаем, что это
ложный индикатор. Нет никакой корреляции между процентом стартапов,
которые получают финансирование, и финансово значимым показателем:
есть ли в этой группе стартапов будущий сверхуспешный игрок или нет.

Но есть показатель обратного. Это может испугать: успехи в привлечении
средств — это не просто бесполезный индикатор, это однозначно вводящий
в заблуждение индикатор. В нашем бизнесе приходится искать аномалии,
которые выглядят совсем не многообещающе, и масштаб их успехов
означает, что мы можем раскидывать свою сеть очень широко. Отдача от
сверхуспешных стартапов может превысить инвестиции в 10 000 раз. Это
значит, что на каждую такую компанию мы можем выбрать тысячу
стартапов, которые не принесут ничего, и все равно получить
десятикратную отдачу.

И если бы хоть раз случилось так, что 100\% профинансированных нами
стартапов смогли привлечь инвестиции после демонстрации своих
проектов, то это почти наверняка означало бы, что мы отбирали их
слишком консервативно.

И чтобы воздержаться от этого, нужны сознательные усилия. После 15
циклов подготовки стартапов к общению с инвесторами и дальнейшего
наблюдения за их результатами я могу теперь посмотреть на те стартапы,
что мы отбираем, глазами будущих инвесторов. Но это неправильный
взгляд!

Мы можем позволить себе как минимум в десять раз больше риска, чем эти
инвесторы. И поскольку риск обычно пропорционален вознаграждению, то
если вы можете позволить себе больший риск — вам стоит на него пойти.
Что означает десятикратно более высокий риск? Что мы должны быть
готовы финансировать в десять раз больше стартапов, чем более
традиционные инвесторы.

Я не знаю, какая доля наших стартапов «поднимает» больше денег после
демонстрации проектов. Я сознательно избегаю подсчета этого числа,
поскольку если ты что-то измеряешь, ты начинаешь это оптимизировать; а
я знаю, что это не та вещь, которую нужно оптимизировать. Но эта доля
определенно выше 30\% (и заметно). Честно говоря, при мысли о 30\%
успехе в привлечении средств мне становится нехорошо. Demo Day, в
результате которого лишь 30\% стартапов окажутся способны привлекать
средства — это катастрофа. Все скажут, что Y Combinator пошел под
откос. Мы бы сами чувствовали, что идем под откос. И все же все мы
были бы не правы.

К лучшему или нет, подобные рассуждения никогда не выйдут за пределы
мысленного эксперимента. Мы бы не выдержали подобного. Как вам это,
достаточно парадоксально? Я могу изложить то, что — я знаю — является
правильным методом действий, и все же действовать не так. Я могу
выдумать всевозможные правдоподобные обоснования. Наш бренд оказался
бы под угрозой (по крайней мере, среди людей, не умеющих считать),
если бы мы инвестировали в гигантское количество рискованных и быстро
стухающих стартапов. Это может размыть ценность сети наших
«выпускников». И, что самое убедительное, постоянные провалы
деморализовали бы нас самих. Но я знаю, что подлинная причина нашего
столь консервативного подхода — то, что мы еще не прочувствовали факт
тысячекратной вариации в отдаче на наши инвестиции.

Наверное, мы никогда не сможем заставить себя идти на риски,
пропорциональные отдаче в этом бизнесе. Лучшее, на что мы можем
надеяться — это что когда мы начнем проводить очередно собеседование с
группой предпринимателей, и нам придет в голову мысль: «Вроде бы это
достойные ребята, но что подумают инвесторы об их безумной идее?», то
мы сможем ответить себе: «Да кого волнуют эти инвесторы?» Вот что мы
подумали, выбирая Airbnb, и если мы хотим получить новые Airbnb, нам
надо и дальше уметь думать именно так.

\end{document}
