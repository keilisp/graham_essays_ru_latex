\documentclass[ebook,12pt,oneside,openany]{memoir}
\usepackage[utf8x]{inputenc} \usepackage[russian]{babel}
\usepackage[papersize={90mm,120mm}, margin=2mm]{geometry}
\sloppy
\usepackage{url} \title{Анатомия целеустремленности} \author{Пол Грэм}
\date{}
\begin{document}
\maketitle

Как и все инвесторы, мы тратим много времени, пытаясь понять, как
предугадать успешность стартапа. Вероятно, большую часть времени мы
проводим, размышляя над этим, потому что инвестируем в самом начале.
Обычно, прогноз – это то, на что мы можем положиться. Мы поняли, что
наиболее важный фактор успешности – это целеустремленность. Вначале мы
думали, что это, вероятно, интеллект. Каждый хочет верить, что именно
благодаря ему (интеллекту) стартап будет успешным. Как замечательно
выглядит рассказ об успешной компании, когда причиной успеха
упоминаются выдающиеся умственные способности ее основателей.
Наверняка, даже пиарщики и журналисты, распространяющие подобные
истории, сами же в них верят. Ум, несомненно, помогает, но это не
решающий фактор. В мире множество людей, таких же умных, как и Билл
Гейтс, но при этом ничего не достигших.


В большинстве сфер талант переоценен по сравнению с
целеустремленностью – частично потому, что так история выглядит
интересней, частично в угоду лени наблюдателей и частично потому, что
через некоторое время целеустремленность становится похожа на талант.
Мне трудно вспомнить хотя бы одну область, где целеустремленность была
бы переоценена, но соотношение целеустремленности и таланта в
некоторой степени отличаются. Возможно, талант более значим в «чистых»
областях, в том смысле, что в основном решается один тип проблем,
вместо многих различных. Я подозреваю, целеустремленность не позволила
б вам добиться успеха в математике, как она сделала бы, скажем, в
организованной преступности. Я ни в коем случае не считаю, что работа,
зависящая от таланта, всегда достойна большего восхищения. Большинство
согласится с тем, что лучше быть математиком, чем запоминать длинные
строки чисел, даже при том, что последнее больше зависит от природных
способностей. Одна из причин, почему люди верят в успех умных
основателей стартапов, заключается в том, что сейчас умственные
способности более востребованы в технологических стартапах, по
сравнению с другими более ранними проектами. Чтобы достичь успеха на
рынке интернет-технологий, наверняка нужно быть несколько умней, по
сравнению со специалистами в области железных дорог, отелей или газет.
И это уже скорее правило, чем исключение. Но даже в самой
высокотехнологичной отрасли успех всё равно зависит больше от
исполнительности, чем от мозгов. Если целеустремленность настолько
важна, можем ли мы разложить на отдельные составляющие? Какие из них
более важные, а какими можно пренебречь? Может, что-то из них можно
развить? Простейшей формой целеустремленности является тупое
упрямство. Когда вы что-то хотите, вы должны получить это, неважно
что. Очевидно, чрезмерное упрямство – врожденная черта характера,
потому как мы часто можем увидеть семьи, где один ребенок более
упрямый по сравнению с другим. Непреодолимые обстоятельства могут
сломать сильную личность, но вряд ли из слабохарактерного можно
воспитать волевого человека. Однако быть волевым человеком не
достаточно. Вы также обязаны трудиться над собой. Решительного, но не
организованного, нельзя назвать целеустремленным. Целеустремленность
подразумевает дисциплинированность сбалансированную упрямством. Слово
«баланс» здесь важно. Чем более упрямым вы являетесь, тем более
дисциплинированным вы обязаны быть. Чем сильнее ваше желание, тем
меньше кто-либо способен спорить с вами, кроме вас самих. А в мире все
равно найдется кто-нибудь, кто вынужден спорить с вами, потому как у
каждого есть свои мотивы, и если у вас желания сильнее дисциплины, вы
просто остановитесь на каком-то промежуточном этапе в жизни, наихудший
пример из которого – наркотическая зависимость. Мы можем представить
волю и дисциплину в виде двух пальцев, сдавливающих сколькую дынную
семечку. Чем сильнее они сжимаются, тем дальше семечко полетит, но при
этом они оба должны давить равномерно, иначе семя улетит в сторону.
Если это правда, то это имеет интересное следствие, потому что
дисциплинированность может быть развита и, фактически, может довольно
таки сильно изменяться в течение жизни индивидуума. Если
целеустремленность является продуктом желаний и дисциплинированности,
тогда вы можете стать более целеустремленными, будучи более
дисциплинированными. [1] Другое следствие модели дынной семечки
заключается в том, что чем более вы упрямы, тем опаснее
недисциплинированность. Существует множество примеров в подтверждение
этому. У некоторых активных людей, вы заметите нечто похожее на
метания, они разрываются между деланием важных дел или ничего
неделанием. Внешне это немного похоже на раздвоение личности. Модель
семечки дыни неточна, по крайней мере, в одном отношении: она
статична. Фактически опасность недисциплинированности возрастает под
влиянием различных соблазнов. Занимательно, это означает, что
целеустремленность разрушает сама себя. Если вы достаточно
целеустремленны для совершения великих дел, это наверняка увеличит
количество соблазнов вокруг вас. Пока вы будете становиться
дисциплинированными, упрямство возьмет верх и вы недалеко уйдёте.
Поэтому Илий Цезарь считал, худых людей очень опасными. Их не
прельщали малые привилегии власти. Модель семечки дыни предполагает,
что можно быть слишком дисциплинированным. Так ли это? Я считаю, что
есть люди, чье упрямство разбилось из-за чрезмерной
дисциплинированности, и которые достигли бы большего, если бы они не
были бы так строги к себе. Одна из причин, почему иногда молодые
успешны там, где старики терпят неудачу, в том, что они не осознают
настолько они некомпетентны. Это позволяет им быть оптимистами. Когда
они впервые начинают работать над чем-нибудь, они переоценивают свои
достижения. Но это даёт им уверенность продолжать работу, и их
производительность увеличивается. Тогда как кто-нибудь трезвомыслящий
увидел бы их начальную некомпетентность для той работы, и возможно
потерял бы уверенность, необходимую для продолжения работы. Есть
другой главный компонент целеустремленности: амбиции. Если упрямство и
дисциплина – это ваш путь к цели, то амбиции – это то, каким вы его
выбираете. Я не знаю, правильно ли утверждать, что амбиции – это черта
целеустремленности, но они точно не полностью независимы. Это было бы
неправильно, если кто-то будет утверждать, что он был полон решимости,
чтобы сделать что-то тривиально простое. К счастью, амбиции
представляются вполне гибкими; и вы многое можете сделать, чтобы
преумножить их. Большинство людей не подозревают, каково это быть
честолюбивым, особенно когда они ещё юные. Они не знают, что даётся
тяжело, на что они способны. Этот вопрос обостряется при наличии
нескольких однодумцев. Амбициозных людей мало, поэтому если собрать
всех вместе и перемешать, так как бывает в жизни поначалу, то у
каждого амбициозного индивидуума не будет много однодумцев вокруг.
Когда вы берете таких людей и объединяете с другими амбициозными
людьми, они расцветают, как вянущий цветок от полива. Наверняка,
большинство амбициозных людей жаждет поощрения, со стороны таких же
амбициозных однодумцев, независимо от их возраста. Достижения также
увеличивают ваши амбиции. С каждым шагом вы получаете уверенность
достичь большего в следующий раз. Итак, суммируя всё, как же работает
целеустремленность: она состоит из упрямства, сбалансированного
дисциплиной, направляемая амбициями. И к счастью, по крайней мере, два
из этих трех качеств можно развить. Вы можете увеличить желание
чего-либо; вы, определенно, можете научиться само-организовываться; и
практически каждый сдается, когда дело доходит до амбиций. Мне
кажется, я понимаю целеустремленность сейчас немножко лучше. Но только
немного: упрямство, дисциплинированность и амбиции – все эти понятия
почти так же сложны, как и целеустремлённость Заметьте также, что
целеустремленность и талант это ещё не всё. Существует третий фактор в
успехе: насколько сильно вы любите работу. Если вам действительно
нравится работать над чем-то, вас не нужно заставлять, чтобы увлечься;
это вы сделаете в любом случае. Но большинство работ имеют неприятные
аспекты, потому что они заключаются в делании вещей для других людей,
и очень не похоже, что задачи, продиктованные их нуждами, в точности
совпадут с вашими. Конечно, для достижения максимального успеха
следует сфокусироваться более на их нуждах, чем на ваших интересах, и
компенсировать разницу целеустремленностью.

Ссылки

[1] Целеустремленность в моем понимании пропорциональна к
$ wd^m – k|w – d|^n $, где w – это желание и d – дисциплина.

[2] Что означает, один из лучших способов помочь сообществу – это
организовывать мероприятия, события и институты, которые соберут
амбициозных людей вместе. Это как вытягивание защитных стержней из
реактора: энергия, которую они выделяют, поддерживает других людей,
вместо того, чтобы быть поглощенной обычными людьми, которыми они
обычно окружены. Наоборот, это вероятно ошибка делать так, как сделали
некоторые европейские страны и пытаются убедить, что ни один из
университетов не лучше, по сравнению с другим.

[3] Например, упрямство состоит из двух компонентов – настойчивость и
энергия. Первая уступает нетерпеливости. Вторая – изменчивости.
Упертые люди становятся старше или теряют энергетику. Они склоны
становиться просто упорными.

\end{document}
