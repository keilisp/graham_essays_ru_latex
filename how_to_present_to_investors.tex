\documentclass[ebook,12pt,oneside,openany]{memoir}
\usepackage[utf8x]{inputenc} \usepackage[russian]{babel}
\usepackage[papersize={90mm,120mm}, margin=2mm]{geometry}
\sloppy
\usepackage{url} \title{Как провести презентацию для инвесторов}
\author{Пол Грэм} \date{}
\begin{document}
\maketitle

Осталось несколько дней до Дня Ангела, когда профинансированные нами
этим летом стартапы будут представлять себя инвесторам. Y Combinator
финансирует стартапы дважды в год: в январе и в июне. А по истечении
десяти недель мы приглашаем всех наших знакомых инвесторов на
презентацию достигнутых результатов.

Мы называем этот день Днем Ангела, потому что большинство инвесторов
являются независимыми «ангелами». Но на презентации приходит все
больше венчурных инвесторов, а в последнее время и несколько
потенциальных покупателей.

Десять недель - это не так уж много. Типичному стартапу, скорее всего,
нечем похвастать после десяти недель существования, но типичный
стартап терпит крах. Если вы посмотрите на стартапы, создавшие великие
вещи, то обнаружите, что многие из них начали с прототипа, выпущенного
за одну-две недели непрерывной работы. Стартапы служат опровержением
правила «поспешишь - людей насмешишь».

(Избыток денег, по-видимому, для стартапов не менее вреден, чем
избыток времени, поэтому мы также даем им не слишком много денег.)

За неделю до Дня Ангела мы проводим репетицию, которая называется День
Демо. На другие мероприятия Y Combinator мы допускаем гостей, но
только не в этот день. Никому, за исключением других основателей, не
суждено увидеть репетиции.

Презентации на Дне Демо обычно оказываются довольно невзрачными, но
этого и следовало ожидать. Мы стараемся подбирать основателей,
способных создать продукт, а не просто красиво рассуждать о нем.
Некоторые из основателей только что закончили колледж, или даже до сих
пор учатся в нем, и им никогда не доводилось выступать перед группой
незнакомых людей.

Поэтому мы концентрируемся на ключевых вещах. В День Ангела у каждого
стартапа будет только десять минут, и мы убеждаем их создателей
сосредоточиться всего на двух задачах: объяснить (a), чем они
занимаются, и (b) почему это будет востребовано пользователями.

Подобные задачи могут показаться несложными, но на самом деле все не
так просто, ведь у докладчика нет опыта выступлений, и ему приходится
объяснять технические вопросы преимущественно не-технической
аудитории.

Когда стартапы проводят презентации для инвесторов, постоянно
повторяется одна и та же ситуация: люди, не способные что-либо
объяснить, говорят с людьми, неспособными что-либо понять. Практически
каждому успешному стартапу, включая таких звезд, как Google, в
какой-то момент приходилось сталкиваться с непониманием и отказом
инвесторов. Было ли причиной тому неумение основателей объяснять или
тупость инвесторов? В большинстве случаев и то, и другое.

На последнем Дне Демо, мы, четыре основателя Y Combinator, заметили,
что говорим практически то же самое, что и на двух предыдущих
мероприятиях. Поэтому в тот же день за ужином мы собрали воедино все
наши советы, касающиеся выступлений перед инвесторами. Большинство
стартапов сталкиваются с аналогичными проблемами, поэтому мы надеемся,
что эти советы окажутся полезными и широкой аудитории.

1. Объясните, что вы делаете Когда инвестор рассматривает совсем юный
стартап, главное, что его интересует, - создал ли стартап
привлекательный продукт. А прежде чем судить, хорош ли ваш продукт Х,
им необходимо понять, к какой категории он относится. Инвесторы будут
очень раздосадованы, если вместо вразумительного рассказа о том, чем
вы занимаетесь, им придется выслушать что-то вроде вступительного
слова.

Расскажите, что вы делаете, как можно раньше, желательно в первом же
предложении. «Нас зовут Джеф и Боб, и мы создали простую в
использовании веб-ориентированную базу данных. Сейчас мы вам ее
покажем и объясним, почему она будет востребована».

Если вы являетесь мастером публичных выступлений, то вы можете себе
позволить нарушить это правило. В прошлом году один из основателей всю
первую половину своего выступления потратил на захватывающий анализ
недостатков общепринятой десктопной парадигмы. Его выступление прошло
удачно, но если вы, как и большинство хакеров, не являетесь
великолепным оратором, то вам лучше выбрать более надежный путь.

2. Побыстрее переходите к демонстрации Наглядная демонстрация
объясняет, что же вы сделали, намного эффективнее, чем любое словесное
описание. Единственное, что заслуживает упоминания до демонстрации, -
это описание проблемы, которую вы пытаетесь решать, и объяснение ее
важности. Но тратьте на это не более 1/10 отведенного вам времени, а
затем переходите к демонстрации.

В процессе демонстрации не стоит просто перечислять функциональности.
Вместо этого оттолкнитесь от решаемой вами задачи, а затем покажите,
каким образом ваш продукт ее решает. Демонстрируйте функциональности в
последовательности, ведущей к какой-либо цели, а не в том порядке, в
каком им довелось появиться на экране.

Если вы представляете веб-приложение, предполагайте, что сеть
загадочным образом пропадет за тридцать секунд до начала вашего
выступления, и приходите с копией серверного программного обеспечения,
запущенной на вашем ноутбуке.

3. Сжатое описание лучше расплывчатого Одна из причин, по которой
основатели избегают четких определений в описании своих проектов,
состоит в том, что на столь ранней стадии существует множество путей
развития. Самое четкое определение кажется неоправданно сжатым. Так,
например, группа, создавшая удобную веб-ориентированную базу данных,
может избегать этого названия для своего приложения, поскольку оно
может стать гораздо большим. На самом деле оно может стать вообще чем
угодно...

Но, как показывает опыт, по мере того, как вы устремляетесь (в
математическом смысле) к описанию чего-то, что может быть чем угодно,
содержательность вашего описания стремится к нулю. Если вы назовете
свою веб-ориентированную базу данных «системой, позволяющей людям
совместно извлекать пользу из информации», то подобное название войдет
в одно ухо инвестора и выйдет через другое. Они просто пропустят эту
фразу мимо ушей как бессмысленный штамп, и будут ждать, со все
возрастающим нетерпением, что в следующем предложении вы все-таки
объясните, что же вы сделали на самом деле.

Ваша главная задача - не расписать все, во что когда-нибудь может
превратиться ваша система, а просто убедить инвестора в том, что вы
заслуживаете продолжения беседы. Поэтому относитесь к данной задаче
как к алгоритму поиска решения путем последовательных приближений.
Начните с ясного, хотя, может быть, и чересчур узкого определения, а
затем максимально расширяйте его. Принцип тот же, что и в
инкрементальной разработке: начните с простого прототипа, а затем
добавляйте функциональности, причем поддерживая в каждый момент
времени работоспособность кода. В данном случае «работоспособность
кода» означает наличие в голове инвестора внятного определения.

4. Не стоит одновременно говорить и проводить демонстрацию Пусть один
человек говорит, а другой работает за компьютером. Если и первое, и
второе будет делать один человек, то он неизбежно начнет мямлить,
склонившись над экраном монитора, вместо того чтобы четко обращаться к
аудитории.

До тех пор пока вы находитесь перед аудиторией и смотрите на нее,
вежливость (и привычка) заставляет ее уделять вам внимание. Но как
только вы перестаете смотреть на людей, переключив внимание на что-то
в вашем компьютере, они тут же отвлекаются на совершенно посторонние
мысли, а этого не должно происходить до завершения вашей презентации.

5. Не стоит уделять много времени второстепенным вопросам Если у вас
есть всего несколько минут, потратьте их на рассказ о возможностях
вашего продукта и о том, почему это здорово. Второстепенные вопросы,
например, наличие конкурентов и ваши резюме, должны быть представлены
всего одним слайдом, по которому вы вкратце пробежитесь в самом конце.
Если у вас впечатляющие резюме, просто покажите их на экране на 15
секунд и скажите пару слов. Что касается конкурентов, то перечислите
трех основных и произнесите по одному предложению с описанием ваших
преимуществ перед каждым из них. Причем эта часть вашего выступления
должна прозвучать уже после того, как все поймут, что именно вы
создали.

6. Не стоит слишком вдаваться в детали бизнес-модели Вообще-то
говорить о том, как вы собираетесь зарабатывать деньги, - неплохо, но
в основном потому, что это показывает, что вам не безразличен данный
вопрос и вы думали о нем. Не стоит вдаваться в детали вашей
бизнес-модели, потому что (a) это не то, что интересует грамотного
инвестора в короткой презентации, и (b) какая бы бизнес-модель у вас в
этот момент ни была, она, скорее всего, ошибочна.

Венчурный инвестор, выступавший недавно в Y Combinator, рассказывал о
компании, в которую он только что вложил деньги. Он сказал, что их
бизнес-модель была ошибочна и ее пришлось сменить трижды, прежде чем
удалось найти правильное решение. Причем основатели были опытными
ребятами, создававшими стартапы ранее, и которым только что удалось
привлечь несколько миллионов долларов инвестиций от одной из ведущих
венчурных компаний, но даже у них бизнес-модель оказалась неверной. (И
тем не менее этот инвестор вложил в них деньги, поскольку он ожидал,
что на данной стадии она будет ерундой.)

Если вы решаете важную задачу, то лучше всего вы будете выглядеть,
рассказывая именно об этом, а не о бизнес-модели. Бизнес-модель -
всего лишь набор предположений, причем предположений о вещах, в
которых вы, скорее всего, не очень хорошо разбираетесь. Так что не
стоит тратить на ерунду драгоценные минуты, которые можно потратить на
рассказ о ценных и интересных вещах, хорошо вам знакомых: решаемой
вами проблеме и о том, что вы успели сделать к настоящему моменту.

Помимо того, что рассказ о бизнес-модели станет не лучшим
использованием времени. Если она к тому же покажется абсолютно
неверной, это вытеснит из памяти инвесторов факты, которые они должны
были бы запомнить. В итоге они запомнят вас стартапом с совершенно
дурацким планом зарабатывания денег, а не компанией, решающей
определенную важную задачу.

7. Говорите с аудиторией ясно и медленно На Дне Демо легко заметить
разницу между людьми, уже имеющими некоторый опыт свободного плавания
и выступлений на публике, и еще не выступавшими.

Для выступления перед множеством людей необходимы совершенно иной
голос и манера говорить, чем в обычной беседе. Повседневная жизнь не
дает возможности попрактиковаться в этом. Если вы еще не умеете так
говорить, отнеситесь к своему выступлению как к трюку, например,
жонглированию.

Но это ни в коем случае не означает, что вы должны говорить подобно
какому-то диктору. Слушатели отключаются, когда с ними говорят таким
образом. Поэтому вы должны говорить в этой искусственной манере и
одновременно поддерживать видимость естественного разговора. (Подобное
правило относится и к литераторству. Хороший стиль письма - это
сознательное усилие, призванное казаться спонтанным.)

Если вы хотите заранее записать и выучить наизусть всю вашу
презентацию - это нормально. Некоторым группам такой метод уже помог.
Но обязательно запишите несколько фраз, способных произвести
впечатление спонтанной, неформальной речи, и преподнесите их именно
так.

И еще - говорите помедленнее. На Дне Демо один из основателей упомянул
одно актерское правило: если вам кажется, что вы говорите слишком
медленно, это означает, что вы говорите с более или менее правильной
скоростью.

8. Пусть говорит кто-то один Зачастую стартапы стремятся
продемонстрировать, что все основатели - равноправные партнеры.
Инстинктивно они поступают верно; инвесторы не любят
несбалансированные команды. Но пытаться демонстрировать это разбиением
презентации на части - уже чересчур. Это отвлекает. Есть менее
прямолинейные способы показать ваше уважение друг к другу. Например,
когда одна из групп выступала на Дне Демо, основную часть доклада
озвучил более общительный из двух основателей, но зато он описал
своего партнера как лучшего хакера из тех, что ему доводилось
встречать, и не было никаких сомнений в серьезности его слов.

Определите одного, от силы двух лучших ораторов, и пусть говорят в
основном они.

Исключение: если один из основателей является экспертом в какой-то
конкретной технической области, возможно, будет полезным, если он
расскажет о ней в течение минуты. Такое «экспертное суждение» способно
повысить доверие, даже если аудитория и не понимает все детали до
конца. Если бы у Джобса и Возняка было десять минут на представление
Apple II, было бы неплохо отдать девять минут Джобсу и одну минуту
где-то в середине Возняку, для рассказа о какой-нибудь сложной
технической проблеме в дизайне, которую ему удалось решить. (Хотя,
конечно, если речь идет именно об этих двоих, то Джобс говорил бы сам
все десять минут.)

9. Держитесь уверенно Из-за ограниченности времени презентации и
недостатка технических знаний большей части аудитории будет непросто
оценить вашу работу. Наверное, на первых порах самым важным
подтверждением ее ценности должна стать ваша собственная уверенность в
ней. Вам придется продемонстрировать, что вы сами под впечатлением от
сделанного.

Причем именно продемонстрировать, а не сказать. Никогда не говорите
«мы увлечены» или «наши продукты великолепны». Люди это просто
проигнорируют - или, что еще хуже, спишут вас со счетов как демагогов.
Такое сообщение должно подаваться в неявном виде.

Чего вы не должны делать, так это казаться нервными или стесненными.
Если вы действительно создали нечто стоящее, вы оказываете инвесторам
услугу, тем, что рассказываете им об этом. Если вы не верите в то, что
ваш стартап настолько многообещающ, что вы оказываете инвесторам
услугу, давая возможность вложить в него деньги, то зачем вы
вкладываете в стартап свое время? И если это действительно так, то,
пожалуй, вам следует сменить сферу деятельности вашей компании.

10. Не пытайтесь казаться большим, чем на самом деле Не волнуйтесь,
если вашей компании всего несколько месяцев отроду и у нее еще нет
офиса, или если ваши основатели - технари, не имеющие опыта ведения
бизнеса. Google когда-то был таким, и у них все неплохо получилось.
Мудрые инвесторы способны смотреть сквозь такие поверхностные изъяны.
Они ищут не законченные блестящие презентации, а талант, пусть и не
опытный. Все, что вам нужно, - это убедить их, что вы умны и
занимаетесь чем-то стоящим. Если же вы будете чересчур стараться
скрыть свою неопытность, пытаясь выглядеть в корпоративном стиле, или
притворяться, что разбираетесь в незнакомых вам вещах, в итоге ваш
талант может просто остаться незамеченным.

Не бойтесь откровенно признавать свое невежество в вопросах, в которых
еще не разбирались. Но, с другой стороны, не следует целенаправленно
привлекать к этому внимание (например, включая слайд про то, что может
пойти не так), так же как не стоит и притворяться, что вы разбираетесь
в вопросе глубже, чем это есть на самом деле. Если вы хакер и
выступаете перед опытными инвесторами, то, скорее всего, они лучше
распознают ложь, чем вы ее придумываете.

11. Не пишите на слайдах слишком много слов Когда на слайде слишком
много слов, их просто перестают читать. Поэтому взгляните на свои
слайды и задайте для каждого слова вопрос: «Можно ли его вычеркнуть?»
Это же относится и к бессмысленному украшательству. Постарайтесь
включать в каждый слайд не более двадцати слов.

Не читайте текст со слайдов. Слайды должны быть фоном, а вы должны
стоять лицом к аудитории и обращаться к ней, а не смотреть на слайды и
читать их людям, находящимся за вашей спиной.

Сайты со множеством элементов не очень хорошо смотрятся при
демонстрации, особенно в проекции на экран. Как минимум, используйте
такой размер шрифта, при котором весь текст будет легко читаемым. Но в
подобных сайтах все равно нет ничего хорошего, так что, возможно, вам
стоит упростить дизайн.

12. Конкретные цифры - это хорошо Если у вас есть какие-либо данные,
пусть даже предварительные, приведите их. Числа врезаются в память.
Если вы можете утверждать, что средний посетитель просматривает
двенадцать страниц, - отлично.

Причем упоминайте только числа, относящиеся непосредственно к вам, и
не более четырех-пяти. Не следует озвучивать объем рынка, на котором
вы работаете. Кого на самом деле волнует, составляет ли он 500
миллионов или 5 миллиардов долларов в год? Говорить об объеме рынка -
это все равно, если бы начинающий актер поведал своим родителям,
сколько зарабатывает Том Хэнкс. Это все конечно хорошо, но сначала
надо стать Томом Хэнксом. Важнее всего не то, зарабатывает он десять
миллионов в год или сто, а то, как вы сами выйдете на подобный
уровень.

13. Расскажите о пользователях Инвесторы, рассматривающие стартапы на
ранней стадии, больше всего беспокоятся о том, что вы создали нечто
исходя из собственных априорных теорий о том, что востребовано, но на
самом деле это никому не нужно. Поэтому будет неплохо, если вы
расскажете о проблемах конкретных пользователей и о том, как вы их
решили.

По словам Грега Мкаду (Greg Mcadoo), сказал, что одна из вещей,
которую ищет Sequoia Capital - это «опосредованный спрос»: что люди
делают сейчас с помощью неадекватных инструментов, что демонстрирует
их потребность в вашем продукте?

Еще один признак наличия спроса - это готовность потребителей платить
за что-то большие деньги. Нетрудно убедить инвесторов в появлении
спроса на более дешевую альтернативу чего-то востребованного, при
условии сохранения в ней качеств, сделавших продукт популярным.

Самые лучшие рассказы о потребностях пользователей - это рассказы о
ваших собственных потребностях. Все знаменитые стартапы выросли из
потребностей самих основателей: Apple, Microsoft, Yahoo!, Google.
Опытные инвесторы знают об этом, поэтому подобные истории привлекут их
внимание. Почти так же хорош будет рассказ о потребностях лично
знакомых вам людей, например, ваших друзей или братьев и сестер.

14. Пусть им западет в память звучная фраза Профессиональные инвесторы
выслушивают много презентаций. Через некоторое время они просто
перестают их различать. Поэтому ваш самый первый шаг - оказаться в
числе тех, кто им запомнился. А чтобы гарантировать это, необходимо
придумать описывающую вас содержательную фразу, способную отложиться в
памяти.

В Голливуде, такие фразы, по-видимому, выглядят как «X повстречал Y».
В мире стартапов они обычно имеют вид «X для Y» или «X Y». Так, Viaweb
был «Microsoft Word электронной коммерции».

Придумайте подобную фразу и отчетливо (но как бы невзначай)
произнесите ее в своем выступлении, желательно в самом начале.

Кроме того, можно использовать полезное упражнение: сесть и попытаться
придумать звучную фразу, характеризующую ваш стартап. Если у вас это
не получается, возможно, ваши планы недостаточно отчетливы.

\end{document}
