\documentclass[ebook,12pt,oneside,openany]{memoir}
\usepackage[utf8x]{inputenc} \usepackage[russian]{babel}
\usepackage[papersize={90mm,120mm}, margin=2mm]{geometry}
\sloppy
\usepackage{url} \title{Как сделать Питтсбург стартап-хабом}
\author{Пол Грэм} \date{}
\begin{document}
\maketitle

Что необходимо предпринять, чтобы превратить Питтсбург в стартап-хаб,
как Силиконовая Долина? Я чувствую, как мне кажется, Питтсбург очень
хорошо, потому что я вырос здесь, в Монровилле. И я понимаю
Силиконовую Долину очень хорошо, потому что я сейчас живу. Могли бы вы
получить такую ​​стартап-экосистему здесь?

Когда я согласился выступить здесь, не думаю, что я был в состоянии
прочитать очень оптимистичную лекцию. Я думал, что буду говорить о
том, что Питтсбург мог бы сделать, чтобы стать стартап-хабом, но
получилось бы слишком много «бы». Вместо этого я буду говорить о том,
что Питтсбург может сделать.

То, что изменило мое мнение — статья, которую я прочитал во всех
разделах о еде New York Times. Название было «Бум питания,
ориентированного на молодежь Питтсбурга». У большинства это не вызовет
и каплю интереса, не говоря уже о чем-то, связанном со стартапами. Но
меня зацепил этот заголовок. Я не думаю, что смог бы выбрать более
многообещающий заголовок, если бы попробовал. Прочитав статью, я еще
больше впечатлился. Там говорилось, что «люди в возрасте от 25 до 29
лет в настоящее время составляют 7,6 процента от всех жителей, по
сравнению с 7 процентами около десяти лет назад.» Ничего себе, подумал
я, Питтсбург может быть следующим Портлендом. Он может стать клевым
местом для всех людей, которые в свои двадцать хотят найти свое место
в жизни.


Когда я попал сюда пару дней назад, то мог почувствовать разницу. Я
жил здесь с 1968 по 1984 год. Я не понимал, что в то время, в течение
всего этого периода, весь город был в свободном падении. На пике
бегства в пригороды, которое происходило повсеместно, как
сталелитейные, так и атомные предприятия умирают. Ребята, что-то
изменилось! Дело не только в том, что центр кажется намного более
процветающим. Здесь есть энергия, которой не было, когда я был
ребенком.

Когда я был ребенком, это было место, которое молодые люди покидали.
Теперь это место, которое привлекает их.

Что нужно делать со стартапами? Ведь стартапы состоят из людей, а
средний возраст людей в типичном стартапе как раз в промежутке от 25
до 29.

Я видел, как это сильно для города, иметь таких людей. Пять лет назад
они перенесли центр тяжести Силиконовой долины с полуострова в
Сан-Франциско. Google и Facebook находятся на полуострове, но
следующее поколение больших победителей все в Сан-Франциско. Причина,
по которой центр тяжести сместился была война за таланты, особенно за
программистов. Большинство от 25 до 29 лет хотят жить в городе, а не
внизу в скучных пригородах. Так что нравится ли это или нет,
основатели знают, что они должны быть в городе. Я знаю многих
основателей, которые предпочли бы жить именно в долине, но были
вынужденны переехать в Сан-Франциско, иначе бы проиграли в войне за
таланты.

Так что быть магнитом для двадцатилетних людей это очень перспективная
вещь. Трудно представить себе место, которое становится стартап-хабом,
но еще не стало им. Когда я прочитал эту статистику об увеличении
процента людей от 25 до 29 лет, у меня было точно такое же чувство
волнения, которое я получаю, когда вижу как графики стартапа начинают
ползти вверх от оси х.

Национальный процент от 25 до 29 лет составляет 6,8\%. Это означает,
что вы на.0,8\% впереди. Население составляет 306 000, поэтому мы
говорим о профиците около 2500 человек. Это население небольшого
городка, и это только превышение. Таким образом, у вас есть точка
опоры. Теперь вы просто должны расширить его.

И хотя «молодежный бум еды» может показаться легкомысленным, это
далеко не так. Рестораны и кафе являются большой частью
индивидуальности города. Представьте прогулку по улице в Париже. Мимо
чего вы проходите? Маленьких ресторанов и кафе. Представьте, что вы
проезжаете через некий депрессивный пригородный район. Мимо чего вы
проезжаете? Starbucks и McDonalds и Pizza Hut. Как сказала Гертруда
Стайн, ни там, ни сям. Вы можете быть где угодно.

Эти независимые рестораны и кафе не только кормят людей. Они
заставляют быть здесь

Так вот моя конкретная рекомендация для превращения Питтсбурга в
следующую Силиконовую Долину: сделайте все возможное, чтобы
стимулировать этот молодежный бум еды. Что может город сделать?
Изучить людей, создающих эти маленькие рестораны и кафе, как
пользователей, и спросить их, что они хотят. Я догадываюсь, что по
крайней мере, одна вещь, которую они могли бы хотеть, это быстрый
процесс выдачи разрешений. Сан-Франциско оставил вам огромное
количество места, чтобы победить их в этой области.

Я знаю, что рестораны это не основная движущая сила. Основная движущая
сила, как говорит статья Таймс, это дешевое жилье. Это большое
преимущество. Но эта фраза «дешевое жилье» немного вводит в
заблуждение. Есть много мест, которые дешевле. Особенность Питтсбурге
не в том, что дешево, а в том, что это дешевое место, в котором вы на
самом деле хотите жить.

Частично это сами здания. Я давно понял, когда я был бедным двадцати с
чем-то лет, что лучшими предложениями были места, которые когда-то
были богаты, а затем стали бедными. Если место всегда было богато, это
хорошо, но слишком дорого. Если место всегда было бедным, это дешево,
но мрачно. Но если это место когда-то было богато, а затем стало
бедным, вы можете найти дворцы задешево. И это то, что объединяет
людей здесь. Когда Питтсбург был богат, сто лет назад, люди, жившие
здесь построили большие прочные здания. Не всегда в лучшем вкусе, но
определенно крепкие. Так вот еще один совет, чтобы стать
стартап-хабом: не разрушайте здания, привлекающие сюда людей. Когда
города находятся на пути обратно вверх, как Питтсбург теперь,
застройщики стремятся снести старые здания. Не позволяйте этому
случиться. Фокус на сохранение истории. Крупные проекты по развитию
недвижимости это не то, что приведет сюда двадцати с чем-то летних.
Они противоположны новым ресторанам и кафе; они вычитают
индивидуальность из города.

Эмпирические данные свидетельствуют о том, вы не можете быть слишком
строгими в охране исторических памятников. Более прочные города это
те, которые построены лучше, чем кажется.

Но привлекательность Питтсбурга не только в самих зданиях, но и в
кварталах. Как и Сан-Франциско и Нью-Йорк, Питтсбург повезло быть
до-автомобильным городом. Это не слишком распространено. Потому что
эти люди от 25 до 29 лет не любят вождение. Они предпочитают ходить,
или ездить на велосипеде или в общественном транспорте. Если вы были в
Сан-Франциско в последнее время, вы не можете не заметить огромное
количество велосипедистов. И это не просто прихоть, которую приняли
двадцати с чем-то летние. В этом отношении они обнаружили лучший
способ жить. Борода пойдет, но не велосипеды. Города, которые вы
можете обойти без машины как раз лучшая точка. Так что я хотел бы
предложить вам сделать все возможное, чтобы заработать на этом. Как и
в случае сохранения исторического наследия, в этом, кажется,
невозможным зайти слишком далеко.

Почему бы не сделать Питтсбург самым дружелюбным к велосипедам и
пешеходам городом в стране? Смотрите, если вы можете пойти так далеко,
то вы сделаете так, то покажется, что Сан-Франциско движется в
обратном направлении по сравнению с вами. Если вы это сделаете, очень
маловероятно, что вы будете сожалеть об этом. Город будет казаться
раем для молодых людей, которых вы хотите привлечь. Если они будут
вынуждены уйти, чтобы получить работу в другом месте, будет жалко
оставлять позади такие места. И в чем недостатки? Можете ли вы
представить себе заголовок «Город разрушен, став слишком дружелюбным к
велосипедистам?» Этого просто не может быть.

Итак, пусть крутые старые кварталы и немного маленькие крутые
рестораны сделают будущий Портленд. Будет ли этого достаточно? Это
поставит вас в более выгодное положение, чем сам Портленд, потому что
Питтсбург имеет что-то, в чем Портленд испытывает недостаток:
первоклассный исследовательский университет. Университет
Карнеги-Меллона плюс маленькие кафе означают, что вы имеете больше,
чем хипстеры, пьющие латте. Это означает, что у вас есть хипстеры,
пьющие латте во время разговора о распределенных системах. Теперь вы
становитесь очень близки к Сан-Франциско.

На самом деле вы лучше, чем Сан-Франциско, с одной стороны, потому что
Университет Карнеги-Меллона (УКМ) это центр города, а Стэнфорд и
Беркли находятся в пригородах.

Что УКМ может сделать, чтобы помочь Питтсбургу стать стартап-хабом?
Быть еще более лучшим исследовательским университетом. УКМ является
одним из лучших университетов в мире, но представьте себе, что было
бы, как если бы он был самым лучшим, и все это знали. Есть много
амбициозных людей, которые должны поехать в самое лучшее место, где бы
оно ни было — даже если бы оно находилось в Сибири. Если бы УКМ был
таким, они бы все пришли сюда. Дети в Казахстане мечтали бы в один
прекрасный день жить в Питтсбурге.

Быть своего рода магнитом для талантов является наиболее важным
вкладом университетов, который они могут внести в превращение их
города в стартап-хаб. На самом деле это практически единственный
вклад, который они могут сделать.

Но подождите, не должны ли университеты запустить программы с такими
словами в названии, как «инновация» и «предпринимательство»? Нет, не
должны. Такие вещи почти всегда оказываются разочарованиями. Они
преследуют неправильные цели. Способ получить инновации это не
стремиться к инновациям, но стремиться к чему-то более конкретному,
типа лучших батарей или лучшей 3D печати. И способ узнать о
предпринимательской деятельности — делать то, что вы не можете сделать
в школе.

Я знаю, что услышать, что самое лучшее, что университет может сделать,
чтобы поощрить стартапы, это быть великим университетом, может
разочаровать некоторых администраторов. Это как говорить людям,
которые хотят похудеть, что способ сделать это — меньше есть.

Но если вы хотите знать, куда пришли стартапы, посмотрите на
эмпирические данные. Посмотрите на истории самых успешных стартапов, и
вы обнаружите, что они растут органически из нескольких учредителей,
которые стороят что-то, что начинается как интересный побочный проект.
Университеты замечательно объединяют основателей, но помимо этого
самое лучшее, что они могут сделать, это уйти с дороги. Например, не
утверждая свое право собственности на «интеллектуальную
собственность», которую разрабатывают студенты и преподаватели, а
также имея либеральные правила о продленном доступе и отгулах.

На самом деле, одним из наиболее эффективных вещей, которые
университет может сделать чтобы поощрить стартапы, это разработать
форму ухода с дороги, изобретенную Гарвардом. Гарвард имел обыкновение
проводить экзамены за осенний семестр после Рождества. В начале января
у них было что-то под названием «Период Чтения», когда вы должны были
готовиться к экзаменам. И Microsoft и Facebook имеют нечто общее, что
немногие люди понимают: они оба запустились во время Периода Чтения.
Это идеальная ситуация для получения такого вида побочных проектов,
которые превращаются в стартапы. Все студенты находятся на территории
кампуса, но они не должны делать ничего, потому что они должны
готовиться к экзаменам.

Гарвард возможно закрыл это окно, потому что несколько лет назад они
перенесли экзамены на период перед Рождеством и сократили период
чтения с 11 дней до 7. Но если университет действительно хотел помочь
своим студентам начать стартапы, эмпирические данные, взвешенные по
рыночной капитализации, предлагают самое лучшее, что они могут сделать
— буквально ничего.

Культура Питтсбурга является еще одним из его сильных сторон. Похоже,
что город должен быть очень социально либеральным, чтобы быть
стартап-хабом, и довольно понятно почему. Город должен терпеть
странности, чтобы быть домом для стартапов, потому что стартапы такие
странные. И вы не можете разрешить только те формы странности, которые
превратятся в крупные стартапы, потому что они все перемешались. Вы
должны терпеть все странности.

Это сразу исключает большие куски в США. Я настроен оптимистично, это
не исключает Питтсбург. Одна из вещей, которые я помню из взросления
здесь, хотя я не понимал в то время, что не было ничего необычного в
этом, как хорошо люди ладили. Я до сих пор не знаю, почему. Может
быть, одна из причин в том, что каждый чувствовал себя иммигрантом.
Когда я был ребенком в Монровилле, люди не называли себя американцами.
Они называли себя итальянцами или сербами или украинцами. Только
представьте, что было здесь сто лет назад, когда люди стекались из
двадцати разных стран. Толерантность был единственным вариантом.

Что я помню о культуре Питтсбурга, так это то, что она была и
толерантная и прагматичная. Вот как я бы описал и культуру Силиконовой
долины тоже. И это не случайно, потому что Питтсбург был Силиконовой
Долиной своего времени. Это был город, где люди строили новые вещи. И
в то время как вещи, которые люди строят, изменились, дух который вам
нужен что бы сделать такую работу, является тем же самым.

Так что, хотя приток хипстеров, жадно пьющих латте, может быть
раздражающим в каком-то смысле, я отошел бы в сторону, чтобы поощрить
их. И в более общем плане терпеть странность, даже до такой степени,
как это делают сумасшедшие калифорнийцы. Для Питтсбурга это
консервативный выбор: это возвращение к корням города.

К сожалению, я приберег самую жесткую часть напоследок. Существует еще
одна вещь, которая нужна, чтобы сделать стартап-хаб, но Питтсбург
этого не имеет: инвесторы. Силиконовая долина имеет большое сообщество
инвесторов, потому что было 50 лет, чтобы его вырастить. Нью-Йорк
имеет большое сообщество инвесторов, потому что он полон людей,
которые любят много денег и которые быстро замечают новые способы,
чтобы получить их. Но Питтсбург имеет ни чего из этого. А дешевое
жилье, который привлекает других людей сюда, не оказывает никакого
влияния на инвесторов.

Если сообщество инвесторов здесь растет, это будет происходить так же,
как это было в Силиконовой долине: медленно и органично. Так что я бы
не стал ставить на наличие большого сообщества инвесторов в
краткосрочной перспективе. Но, к счастью, есть три тенденции, которые
делают это менее необходимым, чем это было раньше. Одной из них
является то, что стартапы запускаются все более дешево, так что вам
просто не нужно столько денег извне, как вы привыкли. Во-вторых,
благодаря таким вещам, как Kickstarter, стартап может получить доход
быстрее. Вы можете разместить что-то на Kickstarter из любой точки
мира. В-третьих, такие программы, как Y Combinator. Стартап из любой
точки мира может поехать в YC на 3 месяца, взять средства, а затем
вернуться домой, если хочет.

Мой совет сделать Питтсбург отличным местом для стартапов, и
постепенно все больше из них будут задерживаться. Некоторые из них
будут иметь успех; некоторые из их основателей станут инвесторами; и
еще больше стартапов будет задерживаться.

Это не быстрый путь — стать стартап-хабом. Но это по крайней мере
путь, которым идут некоторые другие города. И в то же время это не
так, если вы должны принести болезненные жертвы. Подумайте о том, что
я предложил вам сделать. Поощрять местные рестораны, сохранять старые
здания, воспользоваться преимуществом концентрации, сделать УКМ
лучшим, поощрять терпимость. Это те вещи, которые делают Питтсбург
хорошим для жизни сейчас. Все, что я хочу сказать, это что вы должны
сделать еще больше из этого.

И это обнадеживающая мысль. Если путь превращения Питтсбурга в
стартап-хаб в том, чтобы стать даже больше самого себя, то у него есть
хорошие шансы на успех. На самом деле, вероятно, имеются наилучшие
шансы по сравнению с любым городом его размера. Это потребует
некоторых усилий и много времени, но если любой город может сделать
это, то и Питтсбург может.

\end{document}
