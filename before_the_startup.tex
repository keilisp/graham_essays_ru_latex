\documentclass[ebook,12pt,oneside,openany]{memoir}
\usepackage[utf8x]{inputenc} \usepackage[russian]{babel}
\usepackage[papersize={90mm,120mm}, margin=2mm]{geometry}
\sloppy
\usepackage{url} \title{Перед стартапом} \author{Пол Грэм} \date{}
\begin{document}
\maketitle

Одно из преимуществ детей в семье – это постоянная постановка вопроса
«что бы я сказал собственным детям?» самому себе в тот момент, когда
тебя просят дать совет. Мои дети пока маленькие, но я легко могу
представить, что я бы сказал им о стартапах, будь они уже в колледже.
Именно поэтому я собираюсь сказать это вам. \newline

\textbf{Стартапы контр-интуитивны.} Я сам не уверен, почему. Возможно, потому
что знание о них еще не въелось в нашу культуру глубоко. Вне
зависимости от причин, начинать свое собственное дело — задача, в
которой далеко не всегда возможно доверять собственным инстинктам. \newline

В этом смысле процесс похож на катание на лыжах: когда вы первый раз
встаёте на них, едете и хотите затормозить перед спуском, инстинкты
говорят вам: «наклонись назад». Но если вы сделаете это, то вылетите с
трассы. Поэтому часть процесса обучения катания на лыжах включает в
себя подавление данного импульса. С течением времени у вас появляются
новые привычки, но первые попытки требуют серьезного сосредоточения
сознания. То есть существует некоторый список вещей, которые вы должны
запомнить перед тем, как спускаться с трассы. \newline

Стартапы так же противоестественны поначалу, как и катание на лыжах. И
то, что я делаю сейчас – это пишу список вещей, которые вы никак не
можете «не выучить» перед тем, как начинать. \newline

\subsection{Контр-интуитивность}

Первый факт я уже упомянул – стартапы настолько странны, в общем, что
если вы будете доверять собственным инстинктам, то наделаете массу
ошибок. Если вы не знаете ничего, кроме этого, лучше сделать паузу
перед тем, как начинать. \newline

Когда я вплотную занимался Y Combinator я часто шутил, что наша
основная функция – говорить основателям то, что они проигнорировали бы
в любом другом случае. Это действительно так. Набор за набором,
партнеры YC предупреждали основателей об ошибках, которые они сделают,
основатели игнорировали эти слова, а после, через год, приходили со
словами: «Лучше бы мы прислушались». \newline

Почему основатели игнорируют слова партнеров? Это происходит со всеми
контр-интуитивными идеями — они не соответствуют тому, что говорит вам
интуиция. Они кажутся неправильными. Поэтому, конечно, первый импульс
– просто не принимать эти слова в расчет. На самом деле, мое шутливое
объяснение нашей собственной деятельности ни в коем случае не является
проклятием Y Combinator – это причина его существования. Если бы
инстинкты основателей не подводили их и всегда давали правильные
ответы, мы бы никогда не понадобились им. Другие люди нужны только для
того, чтобы давать советы, которые удивляют вас. Это одна из причин,
по которым существует огромное количество тренеров по езде на лыжам и
почти ни одного — по бегу по земле. \newline

Но, как ни странно, вы можете доверять своим инстинктам, касающихся
других людей. И одна из типичных ошибок многих основателей заключается
в том, что они недостаточно сильны в этом: они вовлекаются в общие
процессы с людьми, которые кажутся впечатляющими, но интуитивно
отталкивающими. Позже, когда все идет наперекосяк, основатели
оправдываются: «Я знаю, что с ним было что-то не так, но я
проигнорировал это по той причине, что он впечатлил меня». \newline

Если вы раздумываете о том, чтобы вступить с кем-то в долгосрочные
отношения, в качестве со-основателя, наёмного работника, инвестора или
покупателя, и у вас есть сомнения на этот счет – доверяйте им. Если
кто-то вам кажется скользким типом, уродом, кидалой – не игнорируйте
это ощущение. \newline

Это один из случаев, когда потворствовать собственным слабостям не
зазорно. Работайте только с теми людьми, с которыми у вас не возникает
проблем и с которыми вы давно уверены в отсутствии коммуникационных, и
других, сложностей. \newline

\subsection{Экспертиза}

Второй контр-интуитивный момент заключается в том, что совершенно
неважно, что и в каком объёме вы знаете о стартапах. Путь к успеху с
собственным инновационным проектом заключается не в том, чтобы быть в
этом экспертом, он заключается в том, чтобы на экспертном уровне
решать проблему пользователей. Марк Цукерберг преуспел не потому, что
он был всезнайкой. Он преуспел даже несмотря на то, что у него вообще
не было никакого опыта, но он очень хорошо понимал своих будущих
пользователей. \newline

Если вы не знаете ничего о, к примеру, ангельских инвестициях в первом
раунде – не нужно чувствовать себя несчастным. Это тот тип знания,
который вы сможете приобрести тогда, когда это будет необходимо и
забыть после того, как нужда прошла и вы все сделали. \newline

На самом же деле, я думаю, что это не просто ненужно – знать всё о
«механике» работы стартапов, это может быть даже опасно. Если бы я
встретил студента, который знает всё о конвертируемых облигациях,
соглашениях с сотрудниками и, не дай бог, акциях класса FF, я не
подумаю: «это тот, кто серьезно опережает своих сверстников». Это
сорвёт все сигнализации в моей внутренней системе. Потому, что то, что
характеризует молодых основателей – это необходимость пройти через все
события основания собственной компании, вместе с ошибками. Сначала
кто-то реализует идею, кажущуюся неплохой, снимает офис с крутым видом
и нанимает несколько человек. Со стороны, всем кажется, что это именно
то, что делают стартапы. На самом деле, внутри, следующим шагом после
офиса и найма людей будет простой факт осознания того, насколько они
вляпались. Потому что в процессе имитации стартап-деятельности во всех
внешних проявлениях часто забывается одна вещь, которая является
ключевой – делать что-то, что хотят использовать люди. \newline

\subsection{Игра}

Мы так часто наблюдали, как это происходит, что даже придумали имя
данному феномену: игровой дом. С течением времени я понял, почему это
вообще происходит. Причина по которой молодые основатели приходят к
выводу о необходимости основания стартапа заключается в том, что они
были натренированы и научены идти по этому пути с самого детства.
Эдакая внеклассная активность. Даже в колледжах большая часть работ
так же искусственна, как и бег хомяка в колесе. \newline

Я не набрасываюсь на образовательную систему потому что она такая. В
вашей работе всегда будет присутствовать определенный процент
фейковости, тем более, когда вас учат чему-то. Если же попытаться
замерять результат такой работы, то вы обнаружите большое количество
артефактов, связанных с подобной псевдо-деятельностью. \newline

Признаюсь – я и сам делал так в колледже. Я быстро обнаружил, что на
многих занятиях существует 20 или 30 идей, которые формировали
выпускные экзамены по предмету. После этого я сообразил, что лучшим
способом готовиться к таким экзаменам является не зубрежка материала,
разбираемого на занятиях, а отработка именно конкретных вопросов,
формирующих итоговый тест. Когда я приходил на экзамен, основным моим
чувством было любопытство – какой из моих же вопросов достанется мне в
форме билета. Это сильно напоминало игру. \newline

Неудивительно, что после того как мы тренируемся вести себя подобным
образом, играя в подобные игры, все основатели стартапов пытаются
делать тоже самое и с ними – понять и усвоить «трюки», позволяющие
играть в эту «игру» без потерь. Так как привлечение средств стало
мерилом успеха стартапов (хотя это глупая ошибка), все хотят знать
трюки для убеждения инвесторов. В Y Combinator мы говорим, что лучшим
способом убедить инвесторов является осознание факта, что у вас все
получается, а это значит – что вы быстро растете. Это можно сказать им
простыми словами. После этого все хотят узнать трюки, позволяющие
вырасти быстро. И мы снова говорим, что лучший способ сделать это –
просто сделать что-то нужное людям. \newline

Огромное количество диалогов партнеров YC с молодыми основателями
начинается с вопроса: «Как нам…» и ответом на него: «Просто…» \newline

Почему же основатели постоянно так всё усложняют? Причина, как я
понял, заключается в постоянном поиске «трюка». \newline

Поэтому это третий контр-интуитивный факт, который необходимо
запомнить о стартапах: здесь не работает игровая система, основанная
на трюках и хаках. Это бы могло сработать, если вы пойдете на работу в
крупную компанию-корпорацию. В зависимости от того, насколько
«сломаны» в ней процессы, вы можете преуспеть просто «присосавшись» к
нужным людям, или имитируя деятельность, и так далее. Но это не
работает со стартапами. Нет босса, которого можно обмануть – только
пользователи, каждый из которых переживает в первую очередь о том,
выполняет ли ваш продукт его задачу и решает ли проблему. Стартапы
также обезличены, как и физика. Вы просто должны сделать что-то, что
необходимо пользователю, и в зависимости от того, насколько хорошо вы
это сделали, вы и будете процветать. \newline

Опасная вещь заключается в том, что имитация в некоторой степени может
сработать на инвесторов. Если у вас очень хорошо получается говорить
то, что люди хотят услышать и производить впечатление профессионала,
вы можете обдурить инвесторов на, как минимум, один, а возможно и
больше, раундов инвестиций. Но это не в ваших интересах, так как
компания будет обречена, причем, ультимативно. Единственное, что вы
сделаете – это потратите своё время и чужие деньги на очень быстром
даунхилле. \newline

Поэтому перестаньте искать трюки. В стартапах они могут иногда
встречаться, как и в любой другой отрасли или виде деятельности, но
это не поможет вам решить проблему пользователя. Основатель, который
не знает ничего о привлечении средств, но имеет несколько
пользователей, которые любят его продукт, с гораздо большей
эффективностью найдет те самые средства, по-сравнению с человеком,
который знает об этом все, но не имеет пользователей и их любви. И,
что даже более важно, тот у кого есть пользователи преуспеет после
того, как получит необходимые средства. \newline

И несмотря на то, что в некотором смысле это «плохие новости» –
отсутствие возможности «обыграть» систему успешных стартапов, я
безумно рад тому, что игровая система в них не работает. Я рад, что
все еще есть такие части света и люди, которые могут выиграть делая
хорошую работу. Только представьте, как депрессивен бы был этот мир в
ситуации, если бы все было как в школе или корпорациях, где вам бы
приходилось тратить огромное количество времени, занимаясь откровенной
хренью или проигрывать людям, которые не знают ничего другого. Я даже
не представляю, кем и где бы я был сегодня, если во времена колледжа я
бы осознал это с такой же силой, что в мире есть места, где игровые
системы значат меньше других, или не значат вообще ничего. Сегодня я
убежден в их существовании и предлагаю это знание и вам, особенно,
когда вы думаете о будущем. Как вы можете быть успешным в любой
работе, и чего вы хотите достичь, выполняя её? \newline

\subsection{Всепоглощающий}

Всё сказанное подводит нас к четвертой контр-интуитивной точке –
стартапы всепоглощающи. Если вы начинаете свой, он займёт настолько
высокое место в вашей жизни, что изначально вы даже не можете себе это
представить. А если вы добьётесь успеха со своим проектом, он войдёт в
вашу жизнь на очень долгое время: многие годы, минимально, скорее
десятилетие, а возможно – и весь остаток вашей жизни. Поэтому вам
придется заплатить определенную цену перспективы: её наличия, или
отсутствия. \newline

Может показаться, что жизнь Ларри Пейджа достойна зависти, но в ней
есть и такие аспекты, которым никто бы не позавидовал. По-большому
счёту, в 25 лет он начал бежать так быстро, как только мог, и с тех
пор у него не было возможности остановиться на передышку. Каждый день
в империи Google происходят проблемы, справиться с которыми может
только CEO, и ему приходится решать их. Если он уедет в отпуск,
предположим, на неделю, возникнет целый стек проблем, требующих
срочного решения. И ему придётся справляться с ним не имея возможности
пожаловаться кому-то, отчасти потому что он «папа», который не имеет
возможности показать слабину или страх, а отчасти потому, что всем
плевать на возможные проблемы и сложности миллиардеров. У этого
явления есть странный сайд-эффект в виде того, что вся сложность
построения и жизни успешного стартапа, и его основателя, скрыта от
посторонних глаз. Никто детально не знает, «как это сделано», кроме
тех, кто это делал. \newline

Y Combinator к текущему моменту профинансировал несколько компаний,
которые могут быть названы «успешными», и в каждом случае основатели
говорят одно и то же: «Легче не становится». И не станет. Природа
проблем меняется с течением времени и объёмом задач: вы будете думать
о задержках в постройке здания для нового офиса в Лондоне вместо того,
чтобы думать о сломавшемся кондиционере жарким летом. Общий объём
переживаний никогда не снизится, будет хорошо, если он не увеличится. \newline

И здесь снова можно сравнить основание стартапа с осознанным решением
завести детей – это как нажать кнопку, которая изменит вашу жизнь до
неузнаваемости. И изначально вы не знаете, как конкретно это будет
происходить. И хотя иметь детей -– это замечательно, вы можете сделать
много всего «перед» этим, для того, чтобы помочь будущему себе. И
многое из этого проще сделать до того, как дети у вас появятся, чем
после. Многое из этого сделает вас лучшим родителем. И так как вы
можете подумать, остановится, перед тем как «нажать кнопку», связанную
с детьми и подготовиться, большинство людей в богатых странах и
городах так и поступает. \newline

Но когда дело доходит до стартапов, множество людей считает, что чем
ты моложе – тем больше у тебя шансов на успех, а начинать нужно в
колледже или университете. Вы с ума сошли? А о чём думает
преподавательский состав? Они озабочены тем, чтобы у студентов были
контрацептивы для того, чтобы не допустить стыда в кампусах, и при
этом они организовывают предпринимательские программы и
стартап-инкубаторы направо и налево. \newline

Но, по-правде, у университетов связаны руки. Огромное количество
поступающих студентов заинтересованы в стартапах, и от университетов,
де-факто, ожидается подготовка студентов к этому – карьере. Поэтому те
студенты, которые хотят основать свои стартапы надеются, что учебное
заведение сможет это сделать. И вне зависимости от того, может ли в
действительности университет справится с этой задачей, он вынужден
говорить, что может, иначе студенты пойдут в другое заведение –
которое ясно заявляет, что может. \newline

Могут ли на самом деле университеты научить студентов тому, как делать
стартапы? И да, и нет. Они могут обучить учащихся знанию о стартапах,
но, как я объяснял ранее, это не то, что вам нужно знать. Что вам
нужно знать – это потребности пользователей, но вам никак не узнать
этого до тех пор, пока вы не организуете свою компанию или проект.
Поэтому основание и управление стартапом, начинающим бизнесом – такая
забавная штука, которую можно узнать только делая это. А в колледже
или университете это сделать почти невозможно, по причине которую я
тоже упоминал – это займёт всю вашу жизнь. Вы не можете основать
компанию будучи студентом, потому что, если вы сделаете это, вы
перестанете быть студентом. Номинально, вы можете числиться таковым,
но фактически это сразу же изменится. \newline

Учитывая это противоречие, каким из этих двух путей вам стоит идти?
Быть студентом и не начинать стартап, или начать делать стартап и
перестать быть студентом? На это я могу ответить. Не основывайте
стартап в колледже или университете. «Как сделать свой бизнес и
продукт успешным» – это лишь маленькая часть большего вопроса, который
звучит как: «Как мне жить лучшей жизнью?». И хотя основание стартапа и
вправду может быть путём к лучшей жизни для многих амбициозных людей,
возраст «около 20» не лучший период для этого. Ведь основание
собственного стартапа – это поиск «в глубину», а в 20 с чем-то вы
должны искать в первую очередь «в ширину». \newline

В 20 лет можно делать вещи, которые вы не можете делать до и не
сможете делать после, такие как погружение в спонтанные проекты по
любой прихоти или путешествовать без ощущения «границ». Для людей
неамбициозных такие вещи обречены «опасением стать не успешным», но
для амбициозных это единственный способ получить ценнейший опыт. Если
в 20 вы сделаете стартап, который станет успешным, вы уже никогда не
сможете это сделать. \newline

Марк Цукерберг никогда не сможет бродить по городам какой-то страны в
поисках впечатлений. Да, он может делать то, что недоступно многим –
например, выкупить чартер, который увезёт его куда угодно. Но «успех»
забрал огромное количество разнообразия из его жизни, а Facebook
задаёт тон его существованию точно так же, как и он сам задал тон
развития Facebook. И хотя это очень круто – ощущать себя неотъемлемой
частью проекта, который вы считаете делом собственной жизни, в
разнообразии тоже есть свои плюсы, особенно на заре жизни. Помимо
всего прочего именно разнообразие привносит в вашу жизнь понимание
того, чем конкретно вы бы хотели заниматься оставшееся время, над чем
работать по жизни. \newline

Здесь нет никакого «размена» или «бартера». Вы ничем не жертвуете
основывая стартап в 20, но, с наибольшей вероятностью, вы добьётесь
успеха в своём деле, если немного подождёте. В достаточно
маловероятном случае развития событий один из ваших сайд-проектов
может взлететь, как это было в случае с Facebook, и вам придётся
решать – отдавать этому проекту всё своё время или нет, и в такой
ситуации, действительно, обоснованно заниматься этим. Но в обычной
ситуации основателям приходится очень сильно напрягаться и
фантастически много работать, для того, чтобы добиться этого, и глупо
было бы тратить самые яркие годы своей жизни на это. \newline

\subsection{Пытайтесь}

Стоит ли вообще начинать заниматься этим, в любом возрасте? Сейчас я
понял, что из сложившегося описания получается, что основание стартапа
– очень сложно и почти неподъемное занятие. Если этого ещё не поняли
вы, я повторю: основание стартапа это тяжелая работа. Что если слишком
тяжелая? Как определить, справитесь ли вы? \newline

Ответ на этот вопрос — пятая контр-интуитивная точка: невозможно
сказать или угадать. Ваша жизнь к текущему моменту уже должна была
намекнуть вам на весь массив информации связанный с тем, что и как
произошло бы, реши вы стать математиком. Или профессиональным
футболистом. Но, только в случае если у вас была крайне странная и
нетипичная жизнь, вы никогда не узнаете как это – быть основателем
успешного стартапа. Если вы начнете – это изменит всё, и всего вас.
Поэтому то, что нужно оценивать, это то, куда и в кого вы бы смогли
вырасти, и кто вообще способен на это. \newline

Последние 9 лет моей работой было предсказание того, сможет ли человек
которого я вижу перед собой стать основателем успешного бизнеса.
Всегда было легко говорить людям, насколько они умны и большинство из
тех, кто читает этот текст, наверняка не раз слышали это в свой адрес.
Почти невозможно было, есть и будет сказать, насколько человек станет
амбициозен и требователен к самому себе. Наверное, найдётся немного
людей с таким же опытом, как у меня, и я говорю вам: не угадаешь. Для
себя я уже давно научился держать ум открытым относительно того, какой
проект из каждого набора окажется настоящей звездой. \newline

Иногда основателям кажется, что «они знают». Многие приходят в Y
Combinator с впечатлением, что им удастся «обыграть» нашу систему как
тест, с которыми они уже справлялись в подобном (искусственном) ключе
несколько раз в прошлом. Другие приходят, искренне удивляясь, как их
умудрились выбрать и пытаясь скрыть ошибки, которые могут произвести
плохое впечатление. Но между этими двумя, диаметрально
противоположными, изначальными настроями и конечным результатом на
практике не возникает почти никакой корреляции. \newline

Я читал, что в мире военных это тоже имеет место быть: невозможно
сказать, кто станет лучшим офицером – солдат тихоня или бравый парень.
Это происходит по той же причине – испытания, отделяющие их от
офицерства и высокого чина, несравнимы с тем, с чем сталкивался каждый
из них в прошлом. \newline

Если вы совершенно напуганы основанием собственной компании –
вероятно, вам и не стоит этого делать. Но если вы слегка не уверены,
единственный способ проверить – это попробовать. Но не забывайте о
времени и возрасте. \newline

\subsection{Идеи}

Итак, если вы хотите однажды сделать стартап, что тогда вы должны
делать сейчас? Изначально вам необходимо только две вещи: идея и
команда. И ваш modus operandi должен заключаться в правильной
временной синхронизации этих двух частей. Что подводит нас к шестой, и
последней, контр-интуитивной точке: путь к получению хороших идей для
стартапов заключается в том, чтобы не думать об идеях для стартапов. \newline

Я написал целое эссе на эту тему, поэтому я не буду повторяться
полностью и приведу лишь краткую версию изложенного. Факт заключается
в том, что пытаясь сознательно думать о стартап-идеях, вы можете
придумать что-то неплохо выглядящее на первый взгляд, но плохое. Что
произойдёт дальше? Вы потратите уйму времени, прежде чем поймете, что
идея всё-таки плоха. \newline

Правильный путь «осознания» хороших стартап-идей заключается в том,
чтобы сделать шаг назад. Вместо того, чтобы делать сознательное усилие
для генерации стартап-идеи, сделайте усилие и сконцентрируйтесь на
том, что не требует сознания для того, чтобы стать стартапом.
Большинство сегодняшних супер-проектов изначально начинались, как,
именно, несознательно выбранные поля действий, которые на старте даже
не были похожи на проекты. \newline

Это не просто «возможно» – так начали свой бизнес Apple, Yahoo, Google
и Facebook. Ни одна из этих компаний даже не пыталась быть компанией –
это были любительские проекты. Лучшим стартапам, фактически,
приходится начинать как любительским проектам, потому что лучшие идеи
всегда кажутся аутсайдерами в сознании любого нормального человека и
не воспринимаются как идея для основания компании. \newline

Как, в таком случае, вам настроить собственное сознание на такой тип
работы, чтобы стартап-идеи формировались неосознанно? 1. Вам нужно
знать многое о вещах, которые имеют значение и 2. работать над
проблемами, которые вас интересуют с 3. теми людьми, которые вам
нравятся и которых вы уважаете. Не случайно третий факт как раз и
формирует успешные команды со-основателей работающих над одной идеей. \newline

В первый раз, когда я написал этот параграф, вместо «знать многое о
вещах, которые имеют значение» я написал «стать специалистом в
какой-нибудь технологии». Но данное описание, пусть и правильно,
слишком заужено. Что было особенного в Брайане Чески и Джо Геббиа, так
это то, что они вообще не владели технологией. Они были хороши в
дизайне, и, что даже более важно, они были хороши в организации групп
и осуществлении проектов. Поэтому вам не нужно уметь работать над или
с технологией, как таковой, до тех пор пока вы заняты задачами,
которые требуют с вас гибкости ума и действий. \newline

Что это за задачи? На этот вопрос действительно сложно ответить в
общем тоне. История полна примеров молодых энтузиастов, работавших над
идеями, о которых никто кроме них даже не догадывался или не считал
критичными, одновременно в разных частях света. Как узнать работаете
ли вы над чем-то стоящим? \newline

Я знаю, что я знаю. Реальные проблемы интересны, и я поощряю себя тем,
что работаю над интересной вещью, даже если до неё нет дела никому
другому (особенно, если до неё никому нет дела). Мне сложно работать
со скучными задачами, даже если они кажутся важными. \newline

Моя жизнь полна примеров того, как я работал над чем-то просто из
своего личного и субъективного интереса, а после оказывалось, что моё
решение понравилось кому-то ещё, иногда – многим людям. Y Combinator
сам по себе был чем-то, что казалось интересным. Я называю это
«внутренним компасом», который помогает мне ориентироваться в
пространстве. Но я не знаю, что находится внутри других людей и как
они себе это представляют. Возможно, если я чуть дольше подумаю об
этом, я смогу придумать какой-то алгоритм распознавания
истинно-интересных проблем, но к настоящему моменту единственное, что
я могу предложить, звучит следующим образом. Если у вас есть цель и
желание разбираться с интересными задачами, поощрять этот интерес –
лучший способ подготовить себя к основанию стартапа. Пожалуй, к жизни
тоже. \newline

И хотя я не могу в общих чертах описать то, что является интересной
проблемой, я могу описать большое количество под-проблем. Если вы
воспринимаете технологию как нечто, растущее как фрактал, каждая
движущаяся точка на краю представляет собой интересную задачу. Поэтому
один из гарантированных способов развернуть свой разум к получению
идей для стартапов это заставить себя, как говорит об этом Пол
Буххайт, «жить в будущем». Когда вы достигнете этой точки, идеи,
которые другим людям кажутся неочевидным, будут казаться вам
банальными. Вы можете даже увидеть в одной из них идею для стартапа,
или не увидеть, но вы будете осознавать что вы думаете о чём-то, что
существует или должно существовать. \newline

К примеру, в 90-х годах в Гарварде один из общих знакомых моих
однокурсников Роберта и Тревога написал свой собственный программный
VoIP-программный комплекс. Он не хотел создавать стартап, он никогда
не пытался превратить это в стартап. Он просто хотел общаться со своей
подругой из Тайваня без фантастических счетов за телефон, а так как он
был экспертом в области сетей, ему казалось очевидным, что голос можно
превратить в пакеты и отправить по Интернету. Он не сделал в программе
ничего кроме того, что бы позволяло ему общаться со своей девушкой, и
именно так рождаются лучшие стартапы. \newline

Поэтому, как ни странно, оптимальное дело, которым можно заниматься в
колледже, если вы хотите когда-то основать собственную IT-компанию,
это не думать о том, как ее основать. Это классическая формулировка.
Если вы хотите заняться чем-то своим после колледжа, то в колледже вы
должны быть заняты всепоглощающими делами, неважно в чем. И учиться
выполнять их как можно более эффективно. И если в вас есть
интеллектуальная любознательность, вы создадите стартап занимаясь
чем-то интересным, или полезным, самому себе. \newline

Та часть предпринимательства, которая действительно играет значение,
это экспертиза в определенной области: навыки + опыт. Единственный
путь стать Ларри Пейджем заключается в том, чтобы быть экспертом в
поиске. А для того, чтобы стать экспертом в поиске, всё, что вам нужно
делать – это самореализовываться и идти «у себя на поводу», в хорошем
смысле. Быть любознательным, истинно, а не «по причине». \newline

Стартапы делаются не из корысти, а из-за скрытности в собственной
любознательности. Когда вы хотите узнать что-то от «начала и до
конца», концом будет, как раз, успешный стартап. В лучшем случае. \newline

Поэтому вот мой последний и всеобъемлющий совет тем, кто хочет в один
день стать основателем успешной IT-компании: \textbf{просто учитесь.}

\end{document}
