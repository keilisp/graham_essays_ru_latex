\documentclass[ebook,12pt,oneside,openany]{memoir}
\usepackage[utf8x]{inputenc} \usepackage[russian]{babel}
\usepackage[papersize={90mm,120mm}, margin=2mm]{geometry}
\sloppy
\usepackage{url} \title{Шесть принципов при создании новых вещей}
\author{Пол Грэм} \date{}
\begin{document}
\maketitle

Бурная реакция на релиз Arc[1] имела неожиданные последствия: я
осознал что у меня есть своя философия при создании вещей. Основная
претензия наиболее внятных критиков заключалась в том, что Arc
выглядит недоделанным. Пара тысяч строк макросов — это все, что я
сделал за несколько лет работы? Почему я не занимался прикладными
задачами?


Пока я раздумывал над этими замечаниями, меня внезапно осенило
насколько знакомы были такие отзывы. Люди отзывались точно так же про
Viaweb[2], Y Combinator и про большинство моих статей.

Когда мы запустили Viaweb, он вызвал только усмешки у венчурных
капиталистов и других «экспертов» в е-коммерции. Нас было всего двое и
мы работали в обычной квартире, в 1995 это не выглядело так прикольно
как сейчас. Проект, который мы создали, по их мнению, даже не считался
программным обеспечением. Они понимали программное обеспечение как
большую, монолитную программу под Windows, а Viaweb — первое
веб-приложение которое они когда-либо видели — выглядел как обычный
веб-сайт. Их отношение к нам стало пренебрежительным, когда они
выяснили, что Viaweb не занимается обработкой транзакций кредитных
карт (мы и на самом деле этого не делали в первый год существования
проекта). Обработка транзакций кредитных карт для них было
краеугольным камнем е-коммерции. Обработка тразакций кредитных карт —
это звучало серьезно и внушительно.

Однако, как ни странно, Viaweb победил всех конкурентов.

Когда мы запустили Y Combinator, реакция была почти идентичной. Проект
казался несерьезным. Тогда венчурное вложение означало миллионы
долларов вложенных в малое количество стартапов, основанных людьми,
которые доказали свою состоятельность после месяцев серьезных
бизнес-переговоров; на условиях, описанных в документе толщиной в
тридцать сантиметров. Y Combinator казался незначительным. Конечно,
пока рано говорить повторит ли Y Combinator успех Viaweb, но судя по
количеству попыток клонировать нашу идею, многие считают что мы
нащупали кое-что интересное.

У меня нет другого показателя успешности моих статей кроме как
количества просмотров, однако реакция на них изменилась по сравнению с
той когда я только начинал. Тогда реакция троллей на Slashdot.org,
если перевести на человеческий язык, была примерно следующей: "Кто
этот парень такой чтобы затрагивать эту тему? Такая корткая статья
написанная таким обычным языком не может сказать ничего полезного в
такой-то области, в то время как люди с ученой степенью пишут
толстенные книги на эту тему." Теперь уже новое поколение троллей
обитает на н овых сайтах, они уже не спрашивают "Кто этот парень?.."

Люди говорят про Arc то же, что они говорили на первых порах про
Viaweb, Y Combinator и про мои статьи. Отчего такое сходство? Как я
понял, ответ заключается в том, что у меня один и тот же "почерк" во
всех четырех проектах.

Суть моего метода: я люблю находить (а) простые решения (б) к никем не
замеченным проблемам (в) которые действительно нужно решать (г)
предоставляя решения в максимально свободном стиле (д) начиная с сырой
версии 1 (е) и постоянно совершенствуя ее.

Когда я впервые сформулировал эти принципы, получился практический
рецепт для генерации негативной реакции. Несмотря на то что простые
решения всегда лучше, они не впечатляют так, как сложные. Люди по
определению считают несущественными те проблемы которых они не
замечают. Предоставление решения в простой форме подразумевает что
человеку придется как следует поломать голову чтобы понять его, а не
просто испытать трепет перед ошеломительным видом сложного решения.
Начинать что-то с сырой версии 1 подразумевает что все изначальные
усилия будут незначительными и неполными.

Я, конечно, замечал что люди могут не ухватить новую идею с первого
раза. Я думал что причиной этому в большинстве случаев является
недалекий ум, но теперь я понимаю что не все так просто. Так же как
контр-инвестор[3], каждый, кто последует упомянутым принципам, обречен
делать то, что обычному человеку видится неправильным.

Это и есть самая суть, так же как и стратегия контр-инвестора. Такая
стратегия дает все те преимущества, которые другие люди упускают,
пытаясь выглядеть внушительно. Если вы работаете над незамеченными
проблемами, то вероятность обнаружения каких-то новых вещей намного
выше, так как конкурентов здесь мало. Если вы предоставляете решения в
свободном стиле, то вы (а) не тратите усилий на то чтобы они выглядели
внушительно, (б) избегаете опасности обмануть себя и ваших
пользователей. Преимущество метода "начать с сырой версии 1 и потом
совершенствовать ее" состоит в том, что решения по улучшению от версии
к версии будут приходить извне, причем совершенно неожиданно.

В случае с Viaweb, простым решением было держать программное
обеспечение запущенным на сервере. Незамеченной проблемой было
автоматическая генерация веб-сайтов: в 1995 все онлайн-магазины были
оформлены вручную, дизайнерами, но мы понимали что такой подход не
поддастся масштабированию. Самой значимой частью проекта оказалось
вовсе не обработка транзакций кредитных карт, а генерация графического
дизайна. Механизмом предоставления решений в свободной форме был я,
который приходил на встречу с потенциальными клиентами в джинсах и
футболке. И сырая версия 1, насколько я помню, была размером менее 10
000 строк кода, в этом же время мы запустили проект.

Применение такой стратегии выходит за рамки программирования,
стартапов и сочинения статей. Возможно, эту технику можно применять
для любой творческой деятельности. И совершенно точно то, что ее можно
применять в живописи: как раз этим занимались Сезан и Клее.

В Y Combinator мы делаем ставку на эту технику, в том смысле что мы
поощряем наших стартапщиков работать в таком стиле. Под вашим носом
всегда есть несколько новых идей, поэтому всегда высматривайте простые
вещи, те, которые никем не замечены — позже они всем будут казаться
очевидными. Особое внимание обратите на те проблемы, которые
затерялись из-за всеобщего заблуждения, либо из-за попыток делать то,
что только кажется впечатляющим. Всегда ищите корень проблемы перед
тем как взяться за решение. И не старайтесь выглядеть коорпоративно,
главное — это продукт — то, что выигрывает в перспективе. Кроме того,
запускайте проект как можно скорее, таким путем пользователи дадут вам
понять, что надо было создавать на самом деле.

Классический пример такого подхода — Reddit[4]. Когда Reddit был
запущен, казалось что этот проект не имеет смысла. Ненагруженное
графикой минималистичное оформление, которое даже не тянуло на дизайн.
Но Reddit решал насущную проблему: показать что нового. В результате
проект стал массовым. Теперь, когда Reddit уже существует в умах
людей, кажется что эта идея была очевидной. Люди видят Reddit и думают
что его основателям просто повезло, но, как и другие подобные вещи,
это только кажется везением. Reddit'ы настолько упорно гребли против
течения, что поток развернулся и теперь сам несет их по воде.

Поэтому, когда вы видите что-то вроде Reddit и думаете: "Я бы хотел
придумать что-нибудь подобное" — помните: такие идеи вокруг вас. Но вы
их игнорируете просто потому, что они кажутся неправильными.

Примечания

[1]Новый диалект Лиспа, создатель — Пол Грэм: arclanguage.org

[2]Веб-приложение, с помощью которого можно создать свой интернет
магазин за несколько минут в окне браузера. Написан на Лиспе. Позже
был продан компании Yahoo, теперь находится здесь.

[3]Инвестор, который вкладывает деньги вопреки всеобщим тенденциям на
рынке. В оригинале - Contrarian investment.

[4]Новостной сайт; рейтинг той или иной новости определяют сами
пользователи, путем голосования.

\end{document}
