\documentclass[ebook,12pt,oneside,openany]{memoir}
\usepackage[utf8x]{inputenc} \usepackage[russian]{babel}
\usepackage[papersize={90mm,120mm}, margin=2mm]{geometry}
\sloppy
\usepackage{url} \title{Тест на зависимость} \author{Пол Грэм} \date{}
\begin{document}
\maketitle

Я обнаружил удобный тест для выяснения того, от чего вы зависимы.
Представьте, что вы собираетесь провести выходные у друзей на
маленьком острове у побережья штата Мэн. На острове нет магазинов, и
вы не сможете его покинуть, пока вы там. Кроме того, вы никогда не
были в этом доме, так что вы не можете предположить, что там есть
нечто большее чем все то, что есть в любом доме.

Что помимо одежды и туалетных принадлежностей вы собираетесь
упаковать? Эти вещи это то, от чего вы зависимы. Например, если вы
застанете себя за тем, что вы берете с собой бутылку водки (на всякий
случай), то возможно вам стоит остановиться и хорошенько над этим
подумать.

Для меня список состоит из четырех вещей: книг, берушей, блокнота и
ручки.

Есть и другие вещи, которые я мог бы взять с собой, если так подумать,
например, музыку или чай, но я могу прожить без них. Я не настолько
пристрастился к кофеину, что не рискнул бы провести выходные в доме, в
котором нет чая.

Тишина это другое дело. Я понимаю, что тот факт, что я беру беруши в
поездку на остров у побережья штата Мэн, кажется немного странным.
Если где и должно быть тихо, то там. Но что если человек в соседней
комнате храпит? Что, если там есть ребенок, играющий в баскетбол?
(Бух, бух, бух… бух.) Зачем рисковать? Беруши маленькие.

Иногда я могу думать даже при шуме, если я уже попал в нужное русло с
каким-либо проектом, то я могу работать и в шумных местах. Я могу
редактировать эссе или отладочный код в аэропорту. Но аэропорты не так
уж плохи: по большей части там белый шум. Я не могу работать, когда за
стеной раздается звук комедии или в машине на улице играет музыка.

Конечно, есть еще и другой вид мышления, когда вы начинаете что-то
новое, что требует полной тишины. Вы никогда не знаете, когда этот
момент наступит. Поэтому лучше носить с собой затычки.

Блокнот и ручка это то, что мне всегда нужно иметь при себе в связи с
моей профессией. Хотя в них тоже есть что-то подобное наркотику, в том
смысле, что их главная задача заставить меня почувствовать себя лучше.
Я вряд ли когда-нибудь вернусь и прочту, то что я пишу в блокнотах.
Просто если я не записываю разные вещи я слишком беспокоюсь о том,
чтобы запомнить одну идею и не даю развиваться другой. Ручка и бумага
это фитиль для идей.

Лучшие блокноты, которые я нашел, были сделаны фирмой Miquelrius. Я
пользуюсь их невероятно маленьким размером 2.5 x 4 дюйма (5 х 10 см).
Секрет письма на таких узких страницах заключается в том, чтобы
разбивать слова только тогда, когда у вас больше нет места подобно
тому как это делается в латинском начертании. Я использую самые
дешевые пластиковые шариковые ручки Bic, отчасти потому что их клейкие
чернила не просачиваются сквозь страницы, и отчасти потому что я не
боюсь их потерять.

Блокнот я начал носить с собой около трех лет назад. До этого я
использовал все клочки бумаги, которые я только мог найти. Но проблема
с клочками бумаги в том, что они не упорядочены. В блокноте ты можешь
понять что значит твоя писанина, пролистав его и посмотрев на другие
страницы. В эпоху, когда я пользовался лишь кусочками бумаги, я
постоянно впоследствии находил записи о том, что я вероятно должен был
запомнить, если бы я только мог выяснить что.

Касательно книг я знаю, что в доме вероятно будет что почитать. В
обычную поездку я беру с собой четыре книги и читаю только одну из
них, потому что в пути я нахожу новые книги для чтения. Я беру с собой
книги фактически для подстраховки.

Я понимаю, что подобная зависимость от книг это не очень хорошо — они
мне нужны для отвлечения. Книги, которые я беру с собой в поездки,
обычно довольно целомудренные, из ряда тех книг, которые могут
считаться обязательными к прочтению в колледже. Но я знаю, что мои
мотивы не целомудренны. Я беру с собой книги, потому что если мир
вокруг станет скучным мне нужно проскользнуть в другой мир, который
подвергся перегонке каким-либо писателем. Это все равно что есть
варенье, когда ты знаешь, что должен есть фрукты.

Есть место, где я могу обойтись без книг. Я как-то гулял в отвесных
горах однажды и решил, что я предпочту просто подумать, если мне вдруг
станет скучно, нежели решу тащить с собой хотя бы один лишний грамм.
Было не так уж и плохо. Я выяснил, что могу развлечь себя
самостоятельно придумывая идеи, а не читая идеи других людей. Если вы
перестанете есть варенье, фрукты начнут казаться лучше на вкус.

Поэтому быть может я попытаюсь не брать книги в следующую поездку.
Нужно вырвать затычки из моих холодных, мертвых ушей так или иначе.


\end{document}
