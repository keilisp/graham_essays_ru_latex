\documentclass[ebook,12pt,oneside,openany]{memoir}
\usepackage[utf8x]{inputenc} \usepackage[russian]{babel}
\usepackage[papersize={90mm,120mm}, margin=2mm]{geometry}
\sloppy
\usepackage{url} \title{Зачем переезжать стартапу} \author{Пол Грэм}
\date{}
\begin{document}
\maketitle

После моего последнего выступления, один из организаторов поднялся на
сцену, чтобы возразить мне. Раньше такого не происходило. После первых
нескольких предложений стало понятно, что он был не согласен с тем,
что, как я сказал, стартапам лучше переезжать в Кремниевую Долину.

Конференция проходила в Лондоне, и большинство людей в аудитории,
похоже, были из Великобритании. Поэтому предложение стартапам
переезжать в Кремниевую Долину выглядело очень по-американски:
напыщенный американец говорит всем, что если они хотят нормально вести
дела, им всем нужно переезжать в Америку.

На самом деле, я гораздо меньше американец, чем кажется. Я не говорил
об этом, но по рождению я британец. И так же, как только евреям
позволено шутить на еврейские темы, так же и я не думаю, что должен
быть дипломатичным перед британской аудиторией.

Идея, что стартапы должны переезжать в Кремниевую Долину, даже не
про-американская. [1] То же самое я говорю и стартапам, находящимися в
США. Y Combinator меняет побережья каждые 6 месяцев. Каждый второй
цикл финансирования происходит в Бостоне. И хотя Бостон является
вторым по размеру центром стартапов в США (и в мире), во время этих
циклов мы советуем стартапам переезжать в Кремниевую Долину. И если
этот совет применим даже для Бостона, он тем более применим для любого
другого города.

Не страна имеет значение, а город.

И я думаю, что могу доказать, что я прав. Вы можете легко опровергнуть
аргументы против, доведя их до абсурда. Мало кто будет заявлять, что
совершенно не имеет значение, где находится стартап, что стартап,
находящийся в маленьком сельскохозяйственном городке, ничего не
выиграет от переезда в технологический центр. Большинство людей
понимают, насколько полезным может быть нахождение в месте, где
существует инфраструктура для стартапов, накопленная база знаний, как
выполнять работу, и другие люди, занимающиеся похожими разработками.
Получается, что любой довод в пользу того, что не нужно переезжать из
Лондона в Кремниевую Долину, можно так же применять к утверждению, что
стартапам не нужно переезжать из маленьких городков в Лондон.

Разница между городами заключается в их уровне. И, поскольку все
согласны, что стартапам лучше в Кремниевой Долине, чем в Бостоне,
можно сказать, что стартапам лучше в Кремниевой Долине, чем где бы то
ни было ещё.

Я понимаю, что со стороны может выглядеть, что я преследую свои
интересы, предлагая это, потому что стартапы, которые переместятся в
США, могут попасть в зону интереса Y Combinator. Но американские
стартапы, которые мы финансировали, могут подтвердить, что то же самое
я предлагал и им.

Разумеется, я не заявляю, что любой стартап должен переехать в
Кремнивую Долину, чтобы достичь успеха. Просто, при прочих равных, чем
более место приспособлено для работы стартапа, тем больше стартап там
преуспеет. Но другие аспекты могут перевесить плюсы переезда. Я не
говорю, что основатели стартапа должны срываться с семьями с родного
места и ехать куда-то через полмира - это может быть слишком сильным
отвлекающим фактором.

Ещё одной причиной остаться на месте могут быть трудности иммиграции.
Решение проблем, связанных с иммиграцией, похоже на поиск инвесторов:
этому приходится уделять всё своё время. Стартап не может позволить
себе этого. Один канадский стартап, который мы спонсировали, потратил
около 6 месяцев, работая над переездом в США. В конечном счете они
отказались от этой затеи, потому что не могли позволить себе тратить
так много времени не на разработку своего ПО.

(Если другая страна захочет создать конкурента Кремниевой долине,
лучшее, что они могут сделать, это сделать специальную визу для
основателей стартапов. Иммиграционная политика США - одна из самых
больших слабостей Кремниевой долины.)

Если ваш стартап относится к определенной индустрии, для вас может
быть лучше находиться в одном из ее центров. Для стартапа,
относящегося к развлечениям, лучшим местоположением может быть
Нью-Йорк или Лос-Анджелес.

И наконец, когда хороший инвестор хочет спонсировать вас без переезда,
вы конечно должны оставаться на месте. Найти инвестора трудно. Вообще,
вам не следует отвергать конкретные субсидии на переезд.[2]

В действительности, качество инвесторов может быть главным
преимуществом центров стартапов. Инвесторы Кремниевой Долины гораздо
более агрессивны, чем бостонские. Неоднократно я видел как западные
инвесторы выхватывали стартапы прямо из под носа бостонских
инвесторов, хотя те и начали заниматься ими первыми, но действовали
слишком медленно. В этом году на конференции Boston Demo Day я сказал
слушателям, что если они нашли стартап, который им нравится, они
должны немедленно сделать им предложение. И в течение первого же
месяца после конференции это случилось опять: агрессивный инвестор
западного берега, который познакомился с основателем одного стартапа,
который мы также финансировали, всего за неделю до того, обошёл
бостонского инвестора, который был знаком с этим стартапом уже
несколько лет. К тому времени, как бостонский инвестор понял, что
происходит, сделка уже была совершена.

Бостонские инвесторы не отрицают, что они более консервативны.
Некоторым из них хотелось бы думать, что это качество их
благоразумного характера янки. Но принцип бритвы Оккама подсказывает,
что правда не такая лестная. Бостонские инвесторы более консервативны,
чем инвесторы Кремниевой Долины по той же причине, почему чикагские
инвесторы более консервативны, чем бостонские. Они плохо понимают
стартапы.

Инвесторы западного побережья более напористы не потому, что они
безбашенные ковбои или потому, что хорошая погода настраивает их на
оптимистичный лад. Они более напористы потому, что они знают, что они
делают. Они - горнолыжники, спускающиеся по чёрным трассам.
Напористость - это сущность инвестиционного финансирования. Вы
получаете большую прибыль не потому, что вы пытаетесь избежать
рискованных вложений, а потому, что вы вкладываетесь в многообещающие
проекты. И подобные проекты поначалу часто выглядят довольно
рискованными.

Вот, например, Facebook. Facebook появился в Бостоне. Сначала с ними
начали взаимодействовать бостонские инвесторы. Но они сказали "нет",
поэтому Facebook переехал в Кремниевую Долину и получил деньги там.
Инвестор, который им отказал, сейчас говорит, что "возможно это было и
ошибкой".

Жизнь показывает, что агрессивное инвестирование оказывается в
выигрыше. Может быть агрессивные методы западных инвесторов и могут
обернуться против них самих, но у них уже было довольно времени
сделать это. Кремниевая Долина опережает Бостон с 1970-х годов. Если
инвесторов западного побережья и подстерегала опасность неудачи, она
бы случилась тогда, когда лопнул Пузырь. Но, с тех пор западный берег
ещё больше преуспевает.

Инвесторы западного побережья достаточно уверены в своих оценках,
чтобы действовать быстро; инвесторы восточного побережья уверены не
настолько, но если кто-нибудь думает, что они действуют осторожно в
силу своего благоразумия, он должен увидеть взбешённого инвестора
восточного побережья, когда западный берег уводит у него из под носа
сделку.

Вдобавок к инфраструктуре, центры стартапов также являются рынками. И
рынки обычно централизованы. Даже в наши дни, когда трейдеры могут
быть где угодно, обычно они группируются в нескольких городах. Сложно
сказать, что за аура, способствующая заключению сделок, присутствует
при личных контактах, но что бы это ни было, с помощью технологии
воспроизвести её пока невозможно.

Прогуливаясь в нужное время мимо здания на 165 University Avenue, в
котором выросли такие компании, как Logitech, Google, PayPal, вы
можете одновременно услышать пять разных человек, обсуждающих сделки
по телефону. Честно говоря, это одна из причин, почему Y Combinator
половину времени проводит в Бостоне - нелегко выдерживать такое
окружение целый год. Но, хотя иногда нахождение среди людей, думающих
лишь об одном, может раздражать, это правильное место для вас, если
то, о чём они думают - именно то, что вы пытаетесь сделать.

Недавно я беседовал с одним человеком, который работает над поисковой
системой в Google. Он знает множество людей из Yahoo, поэтому может
сравнивать обе компании. Я спросил его, почему Google ищет лучше. На
что он ответил, что Google не делает ничего особенного, просто они
намного лучше понимают особенности поиска.

И именно по этой же причине стартапы процветают в таких центрах, как
Кремниевая Долина. Стартапы - это довольно специализированный бизнес,
как, например, огранка алмазов. И в центрах стартапов в этом бизнесе
разбираются.

Примечания

[1] Государственническая идея представляет собой обратное: стартапы
должны оставаться в таком-то городе потому, что он находится в
такой-то стране. Если же вы придерживаетесь точки зрения "единого
мира", решение о переезде из Лондона в Кремниевую Долину ничем не
отличается от решения о переезде из Чикаго в Кремниевую Долину.

[2] Вы можете игнорировать инвестора, который только выглядит так, как
будто собирается финансировать в вас. Туманные разговоры о возможности
финансирования в будущем это просто способ сказать "Нет".

\end{document}
