\documentclass[ebook,12pt,oneside,openany]{memoir}
\usepackage[utf8x]{inputenc} \usepackage[russian]{babel}
\usepackage[papersize={90mm,120mm}, margin=2mm]{geometry}
\sloppy
\usepackage{url} \title{Убеждать XOR описывать} \author{Пол Грэм}
\date{}
\begin{document}
\maketitle

При встрече с малознакомыми людьми обычно стараются выглядеть очень
дружелюбными. Люди улыбаются и говорят: «Рад встрече!» — независимо от
того, правда рады или нет. В этом нет ничего нечестного. Все знают,
что эту маленькую социальную ложь не следует понимать буквально, так
же как «Вы не можете передать соль?» — только грамматически вопрос.

Я совершенно искренне улыбаюсь и говорю: «Рад встрече,» — встречая
новых людей. Но есть обычай льстить в письменной речи, который не так
уж безобиден. Причина лести на письме заключается в том, что
большинство эссе пишутся, чтобы убедить читателя. И любой политик вам
скажет, что для убеждения слушателей недостаточно предоставить голые
факты, надо ещё подсластить пилюлю.

К примеру, говоря о сокращении государственного финансирования
программы, политик не скажет просто: «Программа закрыта». Это
прозвучит грубо, обидно. Напротив, большую часть своей речи он
посвятит благородным усилиям работавших над программой людей.
Опасность этого обычая в том, что мы начинаем так думать. «Рад
встрече» это всего лишь вступление к разговору. Но сладкая пилюля,
добавленная политиками, растворяется в речи. От социальной лжи, мы
переходим к обману.


Вот пример из эссе, которое я написал о трудовых союзах.

Те, кто думает, что рабочее движение было создано героическими
усилиями организаторов, сталкиваются с вопросом: почему трудовые союзы
играют такую малозначительную роль сегодня? Лучшее, что они могут
сделать, отвечая на этот вопрос, описать людей ушедшей эпохи. Наши
пращуры были героями. Рабочие начала XX века обладали духовным
мужеством, которое утрачено сегодня.


А вот тот же отрывок, переписанный чтобы польстить им вместо того,
чтобы задевать.

На заре возникновения лидеры трудовых союзов предпринимали героические
усилия, чтобы улучшить условия труда рабочих. Но, хотя трудовые союзы
не имеют такой роли сегодня, это не потому, что их лидеры менее
мужественны. Сегодня работодатель не вышел бы сухим из воды, если бы
нанял головорезов, чтобы избить профсоюзных лидеров; но если бы он
всё-таки сделал это, то рабочие лидеры приняли бы вызов достойно.
Поэтому я думаю, что уменьшение влияния профсоюзов не является
следствием обмельчания их руководства. Конечно, лидеры ранней эпохи
профсоюзов были герои, но не следует думать, что раз влияние
профсоюзов уменьшилось, то виноваты их лидеры. Причина должна быть
внешней. [1]


Здесь написано то же самое. Успех профсоюзов на ранних этапах
определялся не личностными качествами лидеров, но внешними факторами;
в противном случае, современные профсоюзные руководители хуже. Второй
вариант выглядит больше как защита современных профсоюзных лидеров,
чем как преуменьшение заслуг старых. Это делает вторую заметку более
убедительной для сторонников профсоюзов, потому что соответствует их
взглядам.

Я верю во всё, что написал во втором варианте. Ранние профсоюзные
лидеры предпринимали героические усилия. И современные, вероятно,
сделали бы то же самое, если бы это потребовалось. Я сомневаюсь в том,
что было некое «героическое поколение». [2]

Если я верю во всё, написанное во втором варианте, почему я не
остановился на нем? Для чего задевать людей без необходимости? Потому
что я скорее обижу людей, чем стану потворствовать им. Если вы пишете
на спорные темы, вам приходится встать на какую-либо точку зрения.
Степень мужества лидеров прошлого и настоящего второстепенна; главное,
что они такие же. Но, если вы хотите понравиться людям, которые
ошибаются, вы не можете просто сказать им правду. Вам всегда следует
подложить подушку безопасности, чтобы уберечь людские заблуждения от
ударов реальности.

Так поступают большинство авторов. Они пишут так, чтобы убедить, если
только не по привычке или из вежливости. Но я не стремлюсь убеждать; я
пишу, чтобы понять. Я обращаюсь к абстрактному, совершенно
беспристрастному читателю.

Ныне обыкновенно ставится задача убедить реального читателя, такого,
кто не слишком пристрастен. Фактически хуже, чем пристрастен; с тех
пор, как читатели обращаются к эссе, которые пытаются угодить кому-то,
эссе, которые неприятны одной стороне спора, рассматриваются как
попытка угодить другой стороне. Для многих читателей, положительно
относящихся к профсоюзам, первый отрывок звучит как радиопередача
реакционного радио, чей хозяин старается расшевелить своих
последователей. Но ведь на самом деле это не так. То, что противоречит
чьим-либо убеждениям, трудно отделить от нападок на них; и хотя
результаты выглядят похоже, причины, породившие их различны.

Неужели это так плохо, добавить несколько слов, чтобы люди
почувствовали себя лучше? Может быть нет. Может быть я чересчур
краток. Я пишу код, так же я пишу эссе, строчка за строчкой, постоянно
проверяя, что можно выкинуть без ущерба. Но у меня есть разумные
основания для этого. Нельзя понять идею, пока не выразишь её в
нескольких словах. [3]

Опасность второго отрывка не только в том, что он длиннее. А в том,
что начинаешь врать себе. Правда и ложь начинают смешиваться из-за
того, что вы добавили немного читательских заблуждений.

Я думаю, что задача эссе — доискаться истины, сделать открытие. По
крайней мере, это моя задача. Но открытие означает то, что расходится
с общепринятым представлением. Значит, писать, чтобы убедить и
описывать как есть — диаметрально противоположные задачи. Чем больше
ваши выводы несогласны с убеждениями читателей, тем больше усилий вам
придётся приложить, чтобы расширить продажи своих статей. Когда растут
продажи, растёт и это бремя, пока наконец вы не достигнете точки, где
100\% вашей энергии будет уходить на его преодоление и вы не сможете
расти дальше.

Трудно преодолеть чьи-то заблуждения и не задуматься, как донести свои
идеи другим людям. Меня беспокоит, что если я стану убеждать, то я
бессознательно стану избегать идей, которые по-моему трудно продать.
Когда я замечаю что-то удивительное, неожиданное, поначалу это очень
тревожит. Это не более, чем чувство неудобства. Я не хочу задумываться
об этом.

Примечания

[1] Когда я писал эти строки, у меня было странное чувство, будто я
снова в высшей школе. Чтобы получить хорошую оценку, вы должны
написать ханжеское дерьмо, которое от вас ожидают, но это должно
выглядеть, как ваше убеждение. Решалось это определённым приёмом. Было
привычно отвратительно соскользнуть опять в эту неприятную ситуацию.

[2] Упражнение для читателя: перефразируйте эту идею, чтобы доставить
удовольствие тем читателям, которых оскорбляет первый вариант.

[3] Давайте подумаем об этом. Есть один способ, которым я потворствую
читателям, потому что это не меняет количество слов: я пишу от другого
лица. Это лестное отличие выглядит так естественно для обыкновенных
читателей, что они вероятно даже не замечают, когда я переключаюсь на
полуслове, хотя вы бы заметили это если бы это было сделано как здесь.


\end{document}
