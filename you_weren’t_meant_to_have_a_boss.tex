\documentclass[ebook,12pt,oneside,openany]{memoir}
\usepackage[utf8x]{inputenc} \usepackage[russian]{babel}
\usepackage[papersize={90mm,120mm}, margin=2mm]{geometry}
\sloppy
\usepackage{url} \title{Вы не родились подчинённым} \author{Пол Грэм}
\date{}
\begin{document}
\maketitle

Прогресс разделяет понятия "обычного" и "естественного". Зачастую то,
что для нас обычно - неестестенно, а естественное - необычно.
Например, современная еда и отсутствие физических упражнений
неестественны для нашего организма. Возможно, нечто подобное
происходит и с работой. "Обычная" работа может быть так же
противоестественна и вредна для нас в интеллектуальном смысле, как
мука и сахар - в физиологическом.

Я начал подозревать это после нескольких лет работы с основателями
стартапов. На данный момент я успел поработать более, чем с двумястами
из них и заметил определенное отличие между программистами,
работающими над собственными стартапами, и теми, кто работает на
большие компании. Я бы не сказал, что основатели стартапов обязательно
более счастливы; работа над стартапом может быть большим стрессом.
Наверное, лучше будет сказать, что они счастливее так же, как тело
счастливее во время длинного забега, чем во время поедания пончиков на
диване.

Хотя по статистике основатели стартапов и являются отклонением от
нормы, работают они в более естественной для людей форме.

В прошлом году я был в Африке и видел много животных в естественной
среде обитания, которых до этого видел лишь в зоопарках. Было сразу
заметно, как сильно они отличались. Особенно львы. Львы на воле
кажутся в десять раз более живыми. Как будто это разные животные.
Подозреваю, что работать на себя настолько же приятнее для людей, как
для такого свободолюбивого хищника, как лев, жить на воле. Жизнь в
зоопарке легче, но это не та жизнь, для которой они были созданы.

Деревья

Что же такого неестественного в работе на большую компанию? Корень
проблемы в том, что люди не предназначены для работы в таких больших
группах.

Еще одна вещь, которую замечаешь, наблюдая за животными на воле, это
то, что каждый вид процветает, живя в группах определенного размера.
Стадо антилоп может состоять из сотни взрослых особей; бабуинов -
возможно из двадцати; львов - редко из десяти. Люди, видимо, тоже
созданы для работы в группах, и то, что я читал об
охотниках-собирателях, совпадает с исследованиями работы организаций и
моим личным опытом. Из всего этого можно сделать грубое предположение
об идеальном размере группы: восемь человек работают эффективно;
двадцатью уже сложно управлять, а группа из пятидесяти уже совсем
неповоротлива. [1]

Каким бы ни был верхний предел, мы точно не предназначены для работы в
группах по несколько сотен человек. И всё-таки, по причинам, больше
связанным с технологией, чем с природой человека, огромное число людей
работают на компании с сотнями или тысячами сотрудников.

Компании понимают, что большие группы не будут работать, поэтому они
делят их на подразделения, достаточно маленькие, чтобы работать
вместе. Но, чтобы координировать эти подразделения, необходимо кое-что
еще - боссы.

Эти небольшие группы всегда организованы в древовидную структуру. Босс
- это точка, в которой ваша группа крепится к дереву. Но когда вы
используете этот прием для разделения больших групп на меньшие,
происходит кое-что странное, о чем никогда не говорят открыто. В
группе на уровень выше ваш босс представляет всю свою группу. Группа
из десяти менеджеров - это не просто группа из десяти человек,
работающих вместе как обычно. На самом деле это группа групп. Для
группы из десяти менеджеров это означает работать вместе, как будто
они просто группа из десяти человек, а группы, работающие на этих
менеджеров, должны работать, как будто они - один человек. В таком
случае работники и менеджеры будут разделять свободу только одного
человека.

На практике группа людей никогда не может работать, как один человек.
Но в больших организациях, разделенных на группы, требуется именно
это. Каждая группа делает все, чтобы работать как небольшая группа
индивидуумов, для работы в которой и предназначены люди. В этом
заключается смысл создания таких групп. И когда вы продвигаете этот
принцип, каждый человек в итоге получает свободу действия, обратно
пропорциональную размеру всего дерева.

Любой, кто работал на большую компанию, почувствовал это. Можно
ощутить разницу между работой в компании с сотней сотрудников и в
компании с десятью тысячями, даже если ваша группа состоит всего из
десяти человек.

Кукурузный сироп

Группа из десяти человек в большой организации это что-то вроде
ненастоящего племени. Число людей, с которыми вы взаимодействуете в
принципе нормально. Но чего-то не хватает, а именно, личной
инициативы. У племен охотников-собирателей гораздо больше свободы. Их
лидеры обладают несколько большей властью, чем остальные члены
племени, но лидеры распоряжаются не так, как боссы.

В этом нет вины вашего босса. На самом деле проблема заключается в
том, что для группы выше в иерархии ваша группа - это одна виртуальная
личность. А ваш босс - это всего лишь способ сообщить вам это
ограничение.

Получается, что работать в группе из десяти человек в большой
организации одновременно правильно и неправильно. При первом
приближении кажется, что это группа, для работы в которой вы
предназначены, но чего-то очень важного не хватает. Работа в большой
компании похожа на кукурузный сироп с большим содержанием фруктозы: в
нем есть некоторые качества, которые должны вам нравиться, но ему
катастрофически не хватает остальных.

Действительно, еда - отличная метафора, для того чтобы объяснить, что
же не так с обычными видами работы.

К примеру, работать в большой компания - это само собой разумеющееся,
по крайней мере для программистов. И что же в этом плохого? С помощью
аналогии с пищей, мы сможем легко ответить на этот вопрос. В какой бы
точке Америки вы не находились - практически вся еда вокруг вас
окажется вредной. Ведь, человек изначально не приспособлен для того,
чтобы питаться мукой, сахаром, кукурузным сиропом и растительным
маслом. Однако, если вы посмотрите на товары обычного
продовольственного магазина, возможно вы обнаружите, что эти четыре
ингредиента составляют большую часть калорий. То есть, "обычная" еда
очень вредна для вас. Единственными людьми, которые питаются тем, чем
в принципе людям было предначертано питаться, являются чудаки-хакеры,
разгуливающие в сланцах и шортах по коридорам Калифорнийского
Университета в Беркли.

Если "обычная" еда для нас так плоха, тогда почему она так популярна?
Этому есть две причины. Первая - эта еда имеет мгновенную
притягательность для нас (например, из-за яркой упаковки). Вы можете
почувствовать тяжесть в животе через час после того, как съели пиццу,
но откусить первый кусок по-настоящему здорово. Вторая — это экономия,
возникающая при её производстве из-за масштабируемости. Производить
гамбургеры и пиццу можно в больших объёмах, а свежие овощи — нет. Это
означает: (а) гамбургеры могут быть крайне дешевы, и (б) стоит их
продавать.

Если человека поставить перед выбором между двумя вещами: одна из них
разрекламированная, известная широкому кругу людей и в тоже время
дешевая, а другая наоборот, никому не известная и дорогая, как вы
думаете что он выберет?

Точно также дела обстоят и с работой. Обычно выпускники MIT
(Массачусетский технологический институт) стремятся устроиться на
работу в Google или Microsoft. Ведь, устроившись работать в столь
престижные компании, можно обрести стабильность и высокую заработную
плату. Такую работу можно сравнить с пиццей. "Побочный эффект" также,
как с пиццей, будет виден, но, как и в случае с пиццей, не сразу.

В то же время, как основатели, так и первые наёмные программисты
стартапа, схожи с чудаками в сланцах из Беркли: являясь всего лишь
крошечной долей населения, они являются единственными, кто живёт так,
как людям было предназначено жить. В искусственном мире только
маргиналы живут естественно.

Программисты

Жесткие рамки, накладываемые большой компанией на программиста,
оказывают на него негативное влияние. Ведь смысл программирования, как
такового, это создание чего-то нового. Сотрудники отдела сбыта, как
правило, день ото дня выполняют одну и ту же рутинную работу. Людям,
работающим в технической поддержке, чаще всего приходится отвечать на
одни и те же вопросы. Но если вы написали какой-либо код, вам ведь не
требуется писать его снова. То есть, программист всегда создает что-то
новое. Работая в большой компании, где степень свободы каждого
сотрудника обратно пропорциональна размеру компании, вы неизбежно
будете сталкиваться с сопротивлением при попытке создать что-то новое.

Похоже, это является неизбежным следствием большого размера компании.
Даже самые интеллектуальные компании подвержены этому. Недавно я
общался с одним основателем, который рассматривал вариант запуска
стартапа уже в колледже, но вместо этого он решил поработать в
компании Google, потому что посчитал, что он сможет многому научиться
там. В итоге, его ожидания оправдались только частично. Программисты
приобретают знания вследствие реальной работы, воплощения в жизнь
реальных идей. А у него такой возможности — действительно реально
работать и воплощать свои идеи — не было. Иногда, это происходило
потому что такой возможности не давала сама компания, но чаще — потому
что этого не позволял код. Из-за необходимости разбираться в
написанном ранее другими людьми коде, распределения задач по различным
отделам и, соответственно, наличия узких рамок для каждого из них, а
также других ограничений, присущих большим компаниям, он смог
попробовать лишь мизерную долю тех вещей, которые хотел бы
попробовать. По его словам, он получил гораздо больше знаний при
работе с собственным стартапом, нежели работая в большой компании,
несмотря на то что в своём стартапе, кроме программирования, на его
плечи легли и другие заботы. Ведь в своем стартапе, программируя, он
мог делать всё что хочет и как хочет.

{{Препятствие, возникающее при движении вниз по течению, такое же
    препятствие и при движении вверх.}} Если вам не разрешают
реализовывать новые идеи, вы просто перестаёте их генерировать. И
наоборот: когда вы можете делать всё, что захотите, у вас возникает
ещё больше идей и планов. Работая на себя, вы развиваете ваш мозг,
также как система выпуска с низким разрешением в двигателе делает
двигатель более мощным.

Безусловно, работа на себя совсем не обязательно должна подразумевать
создание собственного стартапа. Однако, делая выбор между типичной
работой в большой компании и работой над собственным стартапом,
программисту следует учитывать, что при работе над последним он
получит гораздо больше знаний.

Количество свободы можно расчитать по размеру компании, в которой вы
работаете. Максимальное количество свободы у вас будет в том случае,
если вы сами откроете свою компанию. Если вы устроитесь в компанию,
где работают примерно 10 человек, уровень вашей свободы будет
практически таким же, как и у её основателя. Даже в компании, где
трудятся 100 человек вы, будете себя чувствовать заметно свободнее чем
в компании, штат которой состоит примерно из 1000 человек.

Сам по себе факт работы в небольшой компании, конечно, не гарантирует
свободу. Древовидная структура больших организаций устанавливает
верхний предел свободы, но не нижний предел. Глава небольшой компании
тоже может захотеть быть тираном. Однако, если в случае с небольшой
компаний - это лишь один из вариантов, в случае большой компании
других вариантов просто нет.

Выводы

Выводы из всего этого могут сделать как организации, так и обычные
люди. Как бы компания не старалась сохранить свою первоначальную
гибкость, по мере своего роста она в любом случае будет становиться
все более и более медлительной, громоздкой и неповоротливой. Это
обусловлено наличием древовидной структуры, присущей большим
компаниям.

Единственный известный мне способ избежать этого - вовсе отказаться от
древовидной структуры в больших компаниях, да и вообще от каких то ни
было структур. Нужно дать свободу всем отделам и позволить им
совместно работать, словно различным элементам рыночной экономики,
которые работают совместно, но в тоже время являются независимыми друг
от друга.

Эти наблюдения могут оказаться весьма ценными. Я думаю некоторые
компании уже осознали это. Однако, лично я, не знаю ни одной такой
технологической компании.

Существует один способ, позволяющий сохранить гибкость - нужно
оставаться "маленькими". Если я не ошибаюсь в своих суждениях, в
разные периоды развития компании следует оставаться максимально
небольшой. Компании нужно искать лучших, наиболее профессиональных
сотрудников. Ведь, обычные, посредственные сотрудники выполняют меньше
работы, делают вашу компанию большей, вынуждая вас нанимать всё больше
и больше таких же посредственных работников.

Обычные люди могут для себя сделать примерно тот же вывод: чем меньше,
тем лучше. Вам всегда будет не нравиться работать на большие компании,
и чем больше компания, тем больше вам это будет не нравится.

Несколько лет назад в своем эссе я советовал выпускникам поработать
пару лет на чужую компанию, прежде чем начинать свою. Сейчас я бы
посоветовал иначе. Работайте на чужую компанию, если хотите, но только
на небольшую. И если хотите создать свой стартап, дерзайте.

Причиной того, что я не советовал выпускникам колледжей сразу же
создавать стартапы, было опасение, что у большинства ничего не выйдет.
И так бы и случилось. Но амбициозным программистам лучше заниматься
своим делом и терпеть неудачу чем работать на большую компанию. Так
они точно большему научатся. Возможно, и доход у них будет лучше.
Многие в свои двадцать с небольшим залезают в долги, потому что их
расходы растут гораздо быстрее зарплаты, которая казалась такой
большой по окончанию школы. По крайней мере, если вы начнете стартап и
потерпите неудачу, ваш капитал будет нулевым, а не отрицательным. [3]

На сегодняшний день мы финансировали огромное количество разных
основателей стартапов, и собрали достаточно информации, чтобы выделить
типовые варианты развития событий. Из этого можно сделать вывод, что
нет никакой выгоды в работе на большую компанию. Люди, поработавшие
несколько лет в компании, выглядят перспективнее выпускников
колледжей, но только потому что они старше.

Люди, приходящие к нам из больших компаний, выглядят несколько
консервативными. Сложно сказать, сделали ли их такими большие
компании, или это врожденный консерватизм заставил их работать в таких
компаниях. Но, определенно, бОльшая часть этого консерватизма
приобретенная. Я знаю это, потому что видел как он улетучивался.

Я наблюдал эту картину столько раз, что теперь могу уверенно заявлять
о том, что работа на себя или небольшую компанию является наиболее
естественной для программистов. Основатели стартапов, приходящие в Y
Combinator, часто напоминают подавленных беженцев. Но, спустя три
месяца это проходит: они становятся более уверенными, словно подросли
на несколько сантиметров[4]. Как бы это странно не звучало, они
выглядят одновременно и более счастливыми и более встревоженными. А
это выглядит как раз так, как я бы описал поведение львов в дикой
природе.

Наблюдая за тем, как сотрудники компаний трансформируются в
основателей стартапов, становится очевидным, что разница между ними
вызвана в основном окружающей средой, и в частности, что эта среда в
больших компаниях токсична для программистов. Поработав несколько
недель на себя, они словно оживают, потому что наконец они начали
работать так, как люди и должны работать.

Примечания

[1] Когда я говорю, что люди родились или были созданы, чтобы жить
определённым образом, я подразумеваю жизнь в ходе эволюции.

[2] Страдают не только листья. Ограничение распространяется как вверх,
так и вниз. Поэтому менеджеры тоже оказываются скованными; вместо
того, чтобы просто выполнить работу, им приходится действовать через
подчиненных.

[3] Не финансируйте свой стартап с кредитных карт. Финансирование
стартапа займом - обычно глупый шаг, а займом по кредитной карте -
самый глупый. Задолженность по кредитной карте - плохая идея, точка.
Это ловушка, созданная коварными компаниями для безрассудных и глупых.

[4] Основатели стартапов, которых мы финансировали, раньше были моложе
(изначально мы агитировали студентов последних курсов), и первые
несколько раз, когда я это замечал, я интересовался действительно ли
они становились физически выше.

\end{document}
