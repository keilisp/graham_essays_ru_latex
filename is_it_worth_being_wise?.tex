\documentclass[ebook,12pt,oneside,openany]{memoir}
\usepackage[utf8x]{inputenc} \usepackage[russian]{babel}
\usepackage[papersize={90mm,120mm}, margin=2mm]{geometry}
\sloppy
\usepackage{url} \title{Стоит ли быть мудрым?} \author{Пол Грэм}
\date{}
\begin{document}
\maketitle

Несколько дней назад я наконец понял кое-что, что меня интересовало
уже 25 лет: взаимосвязь между мудростью и интеллектом. Каждый может
заметить, что это не одно и то же. Есть много людей, которые умны, но
не очень мудры. Но интеллект и мудрость как-то связаны. Как?

Что такое мудрость? Я бы сказал, что она в том, чтобы знать, что
делать в разных ситуациях. Я не хочу вдаваться в истинные истоки этого
слова, а просто выяснить, как мы применяем его. Мудрый человек — это
тот, кто в большинстве случаев знает что делать.

Но разве быть умным не означает то же самое знание, что делать в
разных ситуациях? Например, знать, что надо сделать, когда учителя
общеобразовательной школы говорят тебе сложить все числа от 1 до 100?
[1]

Некоторые говорят, что мудрость и интеллект применяются к разным видам
проблем — мудрость к человеческим проблемам, а интеллект к
абстрактным. Но это не так. Иногда мудрость никак не относится к
людям, например: строитель, который знает, что некоторые конструкции
более надежны, чем другие. А также нет сомнений, что умные люди могут
найти искусные решения для человеческих проблем, так же как для
абстрактных [2].

Ещё одно известное объяснение — мудрость приходит с опытом, а
интеллект врожден от природы. Но люди не становятся мудрыми
пропорционально полученному опыту. Иногда мудрость проявляется с
опытом, а иногда может быть от природы, например, врожденные рефлексы,
неприятие некоторых вещей.

Ни одно из общепринятых объяснений разницы между мудростью и
интеллектом не выдерживает критики. Так в чем же эта разница? Если мы
посмотрим, как люди применяют слова «мудрый» и «умный», — то поймем
что это словно разные точки зрения на одно представление.

Кривая

Как «умный», так и «мудрый» - это способы сказать, что кто-то знает
что делать. Отличие в том, что у «мудрого» средний показатель во всех
ситуациях высок, а для «умного» он очень эффектен для некоторых
отдельных ситуаций. Так, если вы нарисуете график, где ось x
представляет ситуации, а ось y представляет выход из них, линия
мудрого человека будет в среднем высоко, но у умного человека будет
достигать высших точек.

Это различие подобно тому, когда о таланте судят по лучшим
показателям, а о характере по худшим. Отличие лишь в том, что мы судим
об интеллекте по лучшим, а о мудрости по средним показателям. Вот как
они связаны: это два разных проявлений ума, у которых одна и та же
кривая может достигать высот.

Так, мудрый человек знает что делать в большинстве ситуаций, тогда как
умный знает что делать там, где только немногие знают. Нам нужно
добавить ещё один критерий: мы должны игнорировать те случаи, когда
кто-то знает что делать потому, что у него есть об этом информация
[3]. Но в остальном, я не думаю, что мы можем более точно что-то
определить, не начав вводить себя в заблуждение.

Но нам это и не нужно. Простое объяснение, какое и есть на самом деле,
подсказывается, или, по крайней мере, соответствует стандартным
версиям — как про интеллект, так и про мудрость. Человеческие проблемы
чаще всего однообразны, так что умение их решать — это ключ в
достижении высоких показателей в подобной оценке. И естественно, что
высокий средний результат в основном зависит от опыта, но яркие
результаты могут быть достигнуты только редкими людьми с природными
способностями. Почти любой может стать хорошим пловцом, но чтобы стать
Олимпийским чемпионом, нужно иметь подходящее тело.

Это объяснение также говорит о том, почему мудрость — такое неуловимое
понятие: нет такой вещи как мудрость. «Мудрый» обозначает кое-что —
то, что человек в среднем умеет делать правильный выбор. Но то, что мы
называем некоторые предполагаемые достоинства и особенности
«мудростью», — ещё не значит, что она существует. В народе «мудрость»
означает все что угодно: её связывают с целым набором разных качеств,
как самодисциплина, опыт, сочувствие [4].

Подобным образом, хотя «умный» что-то значит, мы столкнемся с
проблемой, если будем настойчиво искать что-то одно, что называется
«интеллект». И каковы бы ни были его составляющие, они вовсе не
врожденные. Мы используем слово «интеллект» как показатель
способности: умный человек может понять те вещи, которые смогут
немногие другие. Вполне возможно, что есть некоторая склонность с
рождения к интеллекту (и мудрости тоже), но эта предрасположенность —
еще не сам интеллект.

Одна из причин, почему мы склонны считать интеллект врожденным, в том,
что люди, пытающиеся измерить его, сосредотачиваются на тех его
сторонах, которые можно измерить. С врожденными качествами намного
легче работать, чем с теми которые вытекают из опыта, и от этого могут
отличаться по мере их приобретения. Проблема происходит тогда, когда
мы подставляем слово «интеллект» в то что они измеряют. Если они
измеряют что-то врожденное, то это не может быть интеллект. Трехлетние
дети не умны. Когда мы описываем одного из них как умного, то мы
говорим «умнее, чем другие трехлетние дети».

Разделение

Возможно, это и формальность — выяснить то, что склонность к
интеллекту — это ещё не сам интеллект. Но это важная формальность,
потому что это напоминает, что мы можем стать умнее, так же как мы
можем стать мудрее.

Тревожит то, что мы должны выбирать между этими двумя.

Если мудрость и интеллект — это среднее или пики одной и той же
кривой, то они сводятся в одну точку, если количество точек на линии
уменьшается. Если есть только одна точка, то они идентичны: среднее и
максимум одинаковы. Но если увеличивается количество точек, мудрость и
интеллект отличаются. И, вся история показывает то, что количество
этих точек увеличивалось: наши способности проверены в самых разных
ситуациях.

Во времена Конфуция и Сократа люди относились к мудрости, обучению и
интеллекту как более тесно связанным, чем считаем мы. Отличать
«мудрого» от «умного» — это современный обычай [5]. А причина, по
которой мы так делаем, в том, что они отклонились от стандарта. Когда
знание становится более специализированным, — появляется больше точек
на кривой, и различие между пиками и средним значением становится
более заметным. Как на цифровом изображении с большим количеством
точек.

Как следствие — некоторые старые средства могут выйти из употребления.
Как самое меньшее, мы можем присмотреться, чтобы узнать, — были ли эти
средства для достижения мудрости или для интеллекта. Но настоящая
большая перемена, в результате которой разошлись интеллект и мудрость
— это то, что у нас появилась возможность решать, чего мы больше
хотим. У нас может и не быть возможности улучшить оба одновременно.

И, похоже, что общество выбрало интеллект. Мы более не восхищаемся
глубокомыслием и мудрецами, как это делали люди две тысячи лет назад.
Теперь мы восхищаемся гениальностью. Из-за различия, от которого мы
исходим, появляются страшные ошибки: ты можешь быть умным без большой
мудрости, и можешь быть мудрым не будучи слишком умным. Это звучит не
очень хорошо. Это дало нам Джеймса Бонда, который знает, что делать во
многих ситуациях, но должен полагаться на Кью там, где нужна
математика.

Интеллект и мудрость не являются взаимоисключающими. На самом деле
высокое среднее значение может помочь создать ещё более высокие пики.
Но есть некоторые причины верить, что на каком-то этапе вы должны
между ними выбирать. Одна из них — это очень умные люди, которые так
часто не мудры, что в современной культуре это воспринимается больше
как правило, чем исключение. Возможно, рассеянный профессор и мудр
по-своему, или мудрее чем кажется, но он мудр не так, как этого хотели
от людей Сократ и Конфуций [6].

Новое

Для Сократа и Конфуция мудрость, добродетель и счастье были непременно
связаны. Мудрым человеком считался тот, кто знал правильный выбор и
всегда делал его. Чтобы выбор был правильным, он должен быть
правильным и нравственно. Так, он всегда оставался счастлив, зная, что
сделал лучшее что мог. Я не могу припомнить ни одного древнего
философа, кто бы не согласился с этим.

«Благородный муж счастлив, низкий человек грустен», - Конфуций [7].

Как раз несколько лет назад я читал интервью с математиком, который
говорил, что большинство времени он проводит с недовольством, замечая,
что он недостаточно продвинулся [8]. Китайское и греческое слово,
которое мы переводим как «счастье» не обозначает в точности, что мы
делаем в это время. Но есть достаточно оснований полагать, что эта
запись противоречива.

Математик становится маленьким человеком оттого, что испытывает
недовольство? Нет. Просто он делает работу, которая не была так обычна
в дни Конфуция.

Похоже, что человеческие знания растут как фрактал. Раз за разом,
что-то, что могло представляться неинтересной областью, — даже
ошибочный эксперимент, — при ближайшем рассмотрении, оказывается,
содержит в себе так же много, как и все знания полученные до него.
Некоторые части фрактальных сгустков, которые разорвало с древних
времен, содержат открытия и знания о новых вещах. Например, математика
использовалась как нечто вспомогательное для людей. Но теперь это
карьера тысяч. А в работе, которая затрагивает создание чего-то
нового, некоторые старые правила не работают.

Недавно я проводил время, давая людям советы, и я заметил, что старый
способ все ещё работает: постарайся лучше всего понять ситуацию, дай
лучший совет, основываясь на своем опыте, и потом не беспокойся об
этом, зная, что сделал лучшее, что мог. Но у меня нет такого рода
ясности, когда я пишу свои очерки. Тогда я забеспокоился. Что, если у
меня закончились идеи? И, когда пишу, я тоже чувствую недовольство,
думаю что не сделал достаточно.

Советовать людям и писать — это в корне различные дела. Когда люди
приходят к тебе с проблемой и тебе нужно найти правильное решение, ты
(обычно) не изобретаешь ничего нового. Просто взвешиваешь альтернативы
и пытаешься судить, какой выбор самый благоразумный. Но благоразумие
не может мне сказать, какое предложение писать следующим. Область
поиска слишком велика.

Кто-то, как судья или военный, может делать свою работу, полагаясь на
долг, но долг и обязанности — это не способ создать что-то. Создатели
полагаются на что-то менее точное: вдохновение. Как и многие другие,
кто ведет такой образ жизни, — они склонны к беспокойству, а не
удовлетворению. В этом отношении они больше похожи на маленького
человека дней Конфуция, который постоянно собирает урожай, чтобы не
остаться голодным. Только вместо милосердия погоды они на милосердии
своего воображения.

Ограничения

Для меня было облегчением простое понимание того, что быть недовольным
— это нормально. Идея о том, что успешный человек должен быть
счастливым существует уже тысячи лет. Если я в чем-то преуспел, почему
у меня нет уверенности, которая должна быть у победителя? Но сейчас я
думаю, что это то же самое, как если бы бегун спросил: «Если я такой
хороший атлет, почему я все ещё чувствую усталость?» Хорошие бегуны
тоже чувствуют усталость, но чувствуют её от больших нагрузок.

Люди, чья работа изобретать или познавать вещи, в том же положении,
что и бегун. Для них нет способа сделать лучшее, что они могут, потому
что нет пределов того, что они могут сделать. Чтобы примерно себя
оценить, вы можете сделать сравнение с другими людьми. Но чем лучше вы
делаете что-то, тем менее объективно сравнение. Что будет проверкой
хорошей работы для того, кто итак лучший в своей области? Бегуны, по
крайней мере, могут сравнивать себя с другими, которые делают то же
самое. Если ты получаешь золотую медаль Олимпийских игр, ты можешь по
праву довольствоваться, даже если считаешь, что мог бежать немного
быстрее. Но что делать писателю-романтисту?

Если ты выполняешь дело, где задачи тебе представлены и у тебя есть
несколько альтернатив, в твоем выступлении есть высшая планка —
выбирать лучшее каждый раз. В древних обществах почти любая работа
была такого рода. Крестьянин должен был решать, надо ли заштопать
одежду, а король — надо ли захватить соседей, но ни тот, ни другой не
должны были изобретать что-то. В принципе они могли: король мог
изобрести огнестрельное оружие, а потом напасть на соседей. Но на
практике новшества были так редки, что они ожидались не чаще, чем
ожидается, что вратарь начнет коллекционировать голы [9]. Похоже, что
на практике для любой ситуации существовало правильное решение, и если
ты его выбрал, то выполнил свою задачу в совершенстве. Так же как
вратарь, который не дал другой команде получить очки, по общему
мнению, считается отыгравшим в совершенстве.

Похоже, что в этом мире мудрость первостепенна [10]. Даже сейчас
большинство людей делает работу, где задачи поставлены перед ними, и
им надо выбрать лучшую альтернативу. Но когда знание становится более
адаптированным, тогда появляется все больше задач, где людям надо
создавать что-то самим, и в которых благодаря этому их возможности
исполнения не имеют границ. Интеллект становится все более важным
родственником мудрости, потому что более нет места для пиков.

Средства

Ещё один признак, по которому мы должны выбирать между интеллектом и
мудростью в том, каковы средства для их достижения. Мудрость большей
частью приходит от зарождения качеств в детстве, а интеллект большей
частью от их развития.

Средства достижения мудрости, особенно старые, в том, что нужно
получить выправленную личность. Чтобы достичь мудрости, человек должен
вынести весь мусор из головы, оставляя только важные вещи. Такой
эффект имеет самоконтроль и опыт: чтобы исключить беспорядочные
наклонности, которые идут от вашей собственной природы или от
обстоятельств вашего воспитания, соответственно. Это не вся мудрость,
но это её большая часть. Большая доля того, что находится в голове
старца-мудреца, находится также и в голове двадцатилетнего человека.
Разница в том, что в голове двадцатилетнего это все перемешано с
множеством случайного мусора.

Путь к интеллекту, судя по всему, лежит через работу над трудными
задачами. Вы разрабатываете интеллект, как мышцы, — через упражнения.
Но здесь не может быть слишком много принуждения. Никакая дисциплина
не может заменить настоящей любознательности. Так, развитие интеллекта
— это что-то вроде определения некоторых наклонностей человека —
некоторых склонностей в интересах к чему-то — и обучение этому. Вместо
уничтожения ваших особенностей чтобы сделать из себя нейтральный сосуд
для истины, вы выбираете одну из них и пытаетесь прорастить.

Мудрые все похожи в своей мудрости на других, но очень умные люди
имеют тенденцию быть умными по-разному.

Большинство наших педагогических обычаев направлены на мудрость. И
возможно, школы плохо работают потому, что они пытаются прорастить
интеллект, используя средства подходящие для мудрости. Большинство
средств достижения мудрости включают в себя подчинение. По меньшей
мере, вы должны делать то, что говорит учитель. Более экстремальные
пути стремятся сломать вашу индивидуальность, как это делает военная
подготовка. Но это не ведет к интеллекту. Мудрость происходит от
скромности, но для развития интеллекта может быть полезно иметь
завышенную оценку своих способностей, потому что это мотивирует тебя
продолжать свое дело. В идеале до тех пор, пока не станет очевидно
насколько ты ошибался на самом деле.

(Причина, по которой трудно выучить новые навыки в поздней жизни не
только в том, что мозг менее податлив. Ещё одно, и, вероятно, даже
худшее, препятствие в завышенных стандартах.)

Я понимаю, что вышел здесь на опасную тропу. Я не предлагаю сделать
первой целью образования повышение самооценки студентов. Это просто
выведение лени. И во всяком случае, это не обманет их, по крайней мере
умных. Ещё в молодости они могут понять, что в соревновании выигрывают
не все.

Учитель должен ходить по узкой тропе: хочется поощрять детей, когда
они сами справляются с задачами, но нельзя и одобрить все их
результаты. Нужно быть хорошей аудиторией: уметь ценить, но не
впечатляться слишком просто. И это не так легко. У вас должно быть
достаточно хорошее понимание возможностей детей разного возраста,
чтобы знать, чему удивляться.

Это противоположно традиционным средствам образования. Обычно студенты
являются аудиторией, а не учитель. Задача студентов — не изобретать, а
впитать некоторый заранее данный материал. (В некоторых колледжах
зачитывается заученный материал для разных групп, что является
старомодным обучением). Проблема этих старых традиций в том, что они
пропитаны средствами достижения мудрости.

Различие

Я специально дал эссе такое провокационное название. Конечно же,
мудрым быть стоит. Но я думаю, что важно понимать отношением между
интеллектом и мудростью, и особенно тем, что их отдаляет друг от
друга. Таким путем мы сможем избежать применения правил и норм в
развитии интеллекта, которые на самом деле предназначены для развития
мудрости. Эти два понимания о том, «что надо делать» более различны,
чем многие люди считают. Путь к мудрости лежит через дисциплину, а
путь к интеллекту — через тщательно-отобранное развитие своих навыков.
Мудрость универсальна, а интеллект проявляет себя по-разному. Тогда
как мудрость способствует спокойствию, интеллект почти всегда ведет к
неудовлетворению.

Это очень важно запомнить. Мой друг физик недавно сказал мне, что
половина его отдела сидит на антидепрессантах. Возможно, если мы
поймем, что некоторая доля расстройства неизбежна в отдельных видах
работы, то мы сможем смягчить его эффекты. Возможно, мы сможем собрать
расстройство и подержать какое-то время в стороне, вместо того чтобы
разрешать ему проявляться в каждодневным унынии, которое накапливается
в опасную массу. По крайней мере, мы можем не испытывать недовольства
от того что испытываем недовольство.

Если вы чувствуете себя обессиленным, это не обязательно значит, что с
вами что-то не так. Возможно, вы просто слишком быстро бежите.

Примечания

[1] Говорят, что эту задачу дали Гауссу, когда ему было 10 лет. Вместо
того, чтобы усердно складывать числа, как другие студенты, он заметил
что они составляют 50 пар с суммой 101 (100 + 1, 99 + 2 и т. д.). Так,
он мог просто умножить 101 на 50 и получить ответ — 5050.

[2] Есть вариант, что интеллект — это возможность решать задачи, а
мудрость в суждении о том, как использовать эти решения. Хотя это и
важное отношение между мудростью и интеллектом, это не главное
различие между ними. Мудрость тоже хороша при решении задач, а
интеллект может помочь решить, что делать с этими решениями.

[3] В оценке интеллекта и мудрость мы должны избавиться от фактора
знаний. Человек, который знает комбинацию сейфа, преуспеет в открытии
лучше, чем тот, кто не знает, но никто не будет рассматривать это как
оценку мудрости или знаний.

Но знание совмещается с мудростью так же, как и с интеллектом. Знание
человеческой натуры — несомненно, часть мудрости. Так где же нужно
рисовать линию?

Может быть решение в том, чтобы сбросить знания, которые каким-то
образом имеют решающую ценность для ситуации? Например, понимание
французского языка поможет вам во многих ситуациях, но его ценность
теряется, если никто из других вовлеченных не знает его. Тогда как
значение понимания тщеславия не может сразу обесцениться.

Знания, полезность которых резко теряется — это такие знания, у
которых мало общего с другими. Это такие условности, как языки и
комбинации сейфа, и разные «случайные» факты, например, дни рождения
кинозвезд или различия моделей Studebaker'а 1956 и 1957 года.

[4] Люди, которые ищут нечто отдельное, что называется «мудростью»
были одурачены значением слов. Мудрость — это просто знание что
делать, и есть сотни разных качеств, которые помогают это сделать.
Некоторые, как самоотверженность, бескорыстие могут придти от
размышлений в пустой комнате, а другие, как знание человеческой
натуры, могут придти от посещения пьяных вечеринок.

Возможно, осознание этого поможет развеять облако такой загадочности,
окружающее мудрость в глазах многих людей. Загадка происходит в
основном от поиска того, чего не существует. Такое разнообразие учений
о том, как достигнуть мудрости, заключается в том, что они были
сосредоточены на разных её составляющих.

Когда я использую слово «мудрость» в эссе, я имею в виду не более чем
набор каких-либо качеств, которые помогают людям сделать правильный
выбор в большом разнообразии ситуаций.

[5] Даже в английском языке. Наше слово «интеллект» необычайно
недавнее. Более ранние, как «понимание», имели более широкое значение.

[6] Конечно, есть некоторые сомнения по поводу того, насколько близко
эти записи, приписанные Конфуцию и Сократу, отражают их подлинные
мнения. Я использую эти имена как примеры, также как говорят про
Гомера, — чтобы обозначить предположительных людей, которые говорили
вещи, которые им приписаны.

[7] Analects VII:36, Fung trans.

[8] Это мог быть Эндрю Вайлс (Andrew Wiles), но я не уверен. Если
кто-то помнит такое интервью, я был бы рад узнать об этом у вас.

[9] Конфуций гордо заявлял, что он никогда не изобретал ничего нового
— что он просто утвердился за счет древних традиций. [Analects VII:1]
Теперь для нас сложно оценить, как должна быть сложна задача в
обществе без письма по сохранению полученных знаний. Судя по всему,
даже во времена Конфуция это был первый долг образования.

[10] Склонность древней философии к мудрости может быть преувеличена,
так как в обеих странах, Греции и Китае, многие из первых философов
(включая Конфуция и Платона) видели себя как учителей для официальных
лиц, и из-за этого несоразмерно много думали о подобных вещах.
Некоторые люди, которые изобретали вещи, например, рассказчики, могли
иметь свои резко отклоняющиеся взгляды, которые были проигнорированы.

\end{document}
