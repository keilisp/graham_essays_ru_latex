\documentclass[ebook,12pt,oneside,openany]{memoir}
\usepackage[utf8x]{inputenc} \usepackage[russian]{babel}
\usepackage[papersize={90mm,120mm}, margin=2mm]{geometry}
\sloppy
\usepackage{url} \title{Инвестор как стадное животное} \author{Пол
  Грэм} \date{}
\begin{document}
\maketitle

Основным фактором, определяющем мнение большинства инвесторов о вас,
является мнение других инвесторов. Что, конечно же, способствует
экспоненциальному росту. Как только в вас захочет вложиться один
инвестор, за ним тут же решится еще один, потом другие, и так далее.

Неопытные стартаперы иногда ошибочно полагают, что сущность
привлечения инвестиций заключена как раз в этом эффекте. Услышав
где-то про ажиотаж вокруг инвестирования в успешные стартапы, они
заключают, что способность вызвать ажиотаж — признак успешности
стартапа. На самом деле, эти явления коррелируют не так уж сильно.
Многие «ажиотажные» стартапы в конечном итоге сгорают (причем в
некоторых случаях — отчасти как раз из-за ажиотажа), а многие очень
успешные стартапы пользовались у инвесторов первых раундов весьма
умеренной популярностью.

Таким образом, эта статья не учит создавать ажиотаж, а лишь раскрывает
механизмы его возникновения. В процессе привлечения инвестиций эти
механизмы так или иначе срабатывают, что иногда приводит к неожиданным
последствиям. Понимая их суть, вы сможете хотя бы избежать сюрпризов.

Первая причина роста благосклонности инвесторов лишь после привлечения
других инвесторов, заключается в том, что вы просто стали
привлекательнее для инвестирования. Успешное привлечение средств
снижает риск неудачи. Мало того, оно оправдывает увеличение вашей
оценочной стоимости для последующих инвесторов — хотя сами они этому
вовсе не рады. Тот, кто вложился, когда у вас еще не было средств, нес
большие риски и потому имеет право на большую прибыль. Кроме того,
получив деньги, компания в буквальном смысле становится дороже. Когда
привлечен первый миллион, компания становится как минимум на миллион
долларов дороже — ведь это та же самая компания, только с миллионом на
банковском счете. [1]

Тем не менее, будьте осторожны: более поздние инвесторы ненавидят,
когда оценка компании растет, и потому не желают прислушиваться даже к
этим очевиднейшим доводам. Увеличивайте оценку только для тех
инвесторов, с которыми готовы распрощаться, поскольку кто-то из них
ответит на такое предложение гневным отказом. [2]

Вторая причина, по которой инвесторы любят стартапы, уже нашедшие хоть
какое-то финансирование, состоит в том, что основатели в этом случае
обретают уверенность в себе — а мнение инвесторов о вашей компании
опирается на их мнение о вас лично. Основатели часто удивляются: стоит
им привлечь хоть какие-то средства, инвесторы тут же узнают об этом
словно по волшебству. Хотя для получения такой информации и правда
есть немало путей, главный индикатор — это как раз сами стартаперы.
Инвесторы часто ничего не понимают в технологиях, зато они хорошо
разбираются в людях. Когда привлечение средств идет успешно, инвесторы
сразу чувствуют это по вашей возросшей уверенности. (Это тот случай,
когда характерная для обычного стартапера неспособность скрывать свои
эмоции идет ему на пользу.)

И все же — если начистоту — главная причина, по которой инвесторы
добреют, как только привлечены первые средства, заключается в том, что
они плохо разбираются в стартапах. Даже лучшим из инвесторов оценка
стартапов дается нелегко, а уж посредственные с тем же успехом могли
бы просто бросать монетку. Когда посредственный инвестор видит, что в
вас готово вложиться множество людей, он думает, что на то есть
причина. Так возникает явление, которое в Долине называют hot deal —
когда интерес со стороны инвесторов сильнее, чем вы в состоянии
переварить.

Мнение лучших инвесторов не столь сильно зависит от остальных. Таким
людям незачем равняться на чужие мнения, ведь их суждения ценны сами
по себе. Однако на практике на них оказывает косвенное воздействие тот
факт, что заинтересованность других инвесторов устанавливает некий
дедлайн. Это — четвёртая причина того, что наличие одного предложения
приводит к появлению других. Существенно продвинувшись в заключении
сделки с одним фондом, можно подтолкнуть к принятию решения другие
фонды (часто даже вполне неплохие), ведь иначе они упустят сделку.

Впрочем, если вы не гениальный переговорщик (если не уверены на сто
процентов, значит не гениальный), будьте очень осторожны в попытках
приукрасить ситуацию, чтобы подтолкнуть хорошего инвестора к принятию
решения. Стартаперы постоянно пробуют такой трюк, и инвесторы очень
тонко это чувствуют. Даже чересчур тонко. Однако говоря правду, вы
ничем не рискуете. Если вы далеко продвинулись в переговорах с
инвестором Б, но предпочли бы взять деньги у инвестора А, вы можете
рассказать А о происходящем. Тут нет никакой манипуляции. Вы ведь и
правда в трудном положении, поскольку предпочли бы взять деньги у А,
но не рискнете отклонить предложение Б, пока А не принял решения.

При этом не стоит сообщать А, кто такой Б. Представители венчурных
фондов иногда спрашивают, с кем из других фондов вы ведете переговоры,
но отвечать им ни в коем случае не стоит. Ангелу иногда можно
рассказать о других ангелах, поскольку они чаще сотрудничают друг с
другом. Если же об этом спрашивают в венчурном фонде, просто укажите
своему собеседнику, что он сам хотел бы сохранить ваши переговоры в
тайне от других — вот и вы чувствуете себя обязанным поступать так в
отношении всех фондов, с которыми имеете дело. Если собеседник
настаивает, скажите, что вы все же предпочли бы перестраховаться,
поскольку не слишком опытны в привлечении инвестиций (так вы ничем не
рискуете). [3]

Лишь малой доле стартапов доведется испытать ажиотажный интерес,
большинству же (по крайней мере, поначалу) предстоит столкнуться с
обратной стороной того же явления — когда стадо дружно пасется где-то
в отдалении. Инвесторы очень подвержены влиянию других инвесторов, так
что на старте вам в любом случае придется нелегко. Первая сделка
всегда дается очень сложно, но не опускайте руки, ведь большая часть
трудностей обусловлена именно этим внешним фактором. Договориться со
вторым инвестором будет легче.

Примечания

[1] Бухгалтер сказал бы, что компания, которая привлекла миллион
долларов, не стала богаче, если она получила их в виде конвертируемых
долговых обязательств. На практике же, деньги, привлеченные в виде
конвертируемых долговых обязательств, мало чем отличаются от денег
привлеченных в виде вклада в уставной капитал.

[2] К удивлению многих стартаперов, инвесторы бывают весьма
эмоциональны. Вернее, они любят возмущаться (мне кажется, возмущение —
вообще основная эмоция инвесторов), причем заходят в этой любви так
далеко, что иногда забывают о собственных интересах. Я знаю одного
инвестора, который вложился в стартап по оценке в 15 миллионов
долларов. До того он имел возможность инвестировать исходя из оценки в
5 миллионов, но отказался, поскольку его товарищ, вложивший средства
чуть раньше, ухитрился сделать это по оценке в 3 миллиона.

[3] Если инвестор настойчиво добивается информации о переговорах с
другими, то подумайте, нужен ли вам такой инвестор?

Я хотел бы поблагодарить Пола Бакхайта, Джессику Ливингстон, Джеффа
Ролстона и Гарри Тана за чтение черновиков этой статьи.

\end{document}
