\documentclass[ebook,12pt,oneside,openany]{memoir}
\usepackage[utf8x]{inputenc} \usepackage[russian]{babel}
\usepackage[papersize={90mm,120mm}, margin=2mm]{geometry}
\sloppy
\usepackage{url} \title{Барахло} \author{Пол Грэм} \date{}
\begin{document}
\maketitle

У меня слишком много вещей. Как, впрочем, и у большинства обитателей
США. Такое чувство, что чем у человека меньше денег, тем больше у него
вещей. Сложно представить себе человека настолько бедного, чтобы он не
мог позволить себе владеть несколькими подержанными автомобилями.

Но так было далеко не всегда. Когда-то давно, в прошлом, вещи были
ценными и редкими. И мы до сих пор можем найти подтверждение этому
факту, если только немного осмотримся по сторонам. Например, в моём
доме, в Кембридже, который был построен в 1876 году, в спальнях нет
кладовок. В те времена имущество людей с лёгкостью умещалось в ящиках
комода. Да что там девятнадцатый век — ещё каких-то двадцать лет назад
вещей было гораздо меньше. Когда я смотрю на фотографии 1970-х годов,
я каждый раз удивляюсь — какими пустыми выглядели дома. В детстве у
меня было огромное стадо игрушечных машинок, но оно не идёт ни в какое
сравнение с невероятным количеством игрушек моего племянника. Все мои
игрушки вместе взятые занимали примерно треть поверхности кровати. В
комнате моего племянника кровать — это единственное свободное место.

Вещи стали гораздо дешевле, однако наше отношение к вещам осталось
прежним. Мы слишком высоко ценим разное барахло.

Это было большой проблемой для меня, когда я испытывал недостаток
денег. Я ощущал себя бедным, а вещи казались мне ценными, так что я
почти инстинктивно собирал их из разных мест. Какие-то вещи оставляли
мне друзья во время переездов. Кое-что я находил сам, гуляя по улицам
в ночь вывоза мусора (остерегайтесь всего, что Вы сами называете
«очень хорошей вещью»). Что-то доставалось мне по смешной цене во
время гаражных распродаж. Стоило мне зазеваться и хоп — у меня
заводилось новое барахлишко.

Как я сейчас понимаю, эти бесплатные или почти бесплатные вещи вовсе
не были выгодными покупками. Так как их ценность была ещё меньше той
мизерной цены, которую я за них платил. Собираемое мной барахло, по
большей части, не стоило ровным счётом ничего, так как оно было мне не
нужно.

Тогда я не понимал, что выгода от покупки — это вовсе не разница между
розничной ценой в магазине и ценой, за которую мне удалось приобрести
вещь. Выгода от покупки — это то, что я могу получить от вещи. Вещи
ведь крайне сложно превратить обратно в деньги. Ну а если Вы не можете
продать своё «ценное» имущество, какая разница, сколько оно стоило в
магазине? Единственный способ извлечь из такого имущества прибыль —
это использовать его. Если же Вы не собираетесь начать использовать
свежекупленное имущество сразу, то, скорее всего, не соберётесь уже
никогда.

Компании, которые продают вещи, тратят огромные деньги, чтобы убедить
нас в том, что вещи по-прежнему ценны. Но куда более верным будет
считать, что вещи не стоят ровным счётом ничего.

На самом деле, даже меньше, чем просто ничего. Так как стоит Вам
накопить определённое количество вещей, как они перехватывают у Вас
инициативу и начинают владеть Вами. Я знаю одну семейную пару, которая
не могла переехать в город, куда они хотели переехать. У них не
хватало денег, чтобы купить дом, в котором могли бы жить все их вещи.
Понимаете — они покупали дом не для себя. Они покупали дом для своих
вещей.

И если Вы не являетесь крайне педантичным, аккуратным и организованным
человеком, то полный вещей дом может быть довольно гнетущим.
Беспорядок в комнатах высасывает жизненные силы. Очевидно, чем больше
в комнате вещей, тем меньше в ней остаётся места для людей. Но место —
это только часть проблема. Очень важно также и то, что люди постоянно
осматриваются, чтобы построить в своей голове модель окружающего мира.
И чем больше вокруг них вещей, тем больше сил уходит на построение
этой модели. Неубранная комната буквально выматывает.

(Это, кстати, объясняет тот факт, что дети гораздо спокойнее относятся
к беспорядку, чем взрослые. Дети не видят столько деталей. Они строят
куда более грубую модель своего окружения, и у них уходит на это
меньше энергии).

Впервые я осознал бесполезность вещей, когда я в течение года жил в
Италии. Я взял с собой в эту поездку только один большой рюкзак, а всё
остальное барахло оставил в США, на чердаке у хозяйки дома. И знаете
что? Единственным предметами, по которым я скучал, были несколько моих
любимых книг. В конце года я даже не сумел вспомнить, что же ещё я
оставил на чердаке.

Но когда я вернулся, я не смог выкинуть ни единой коробки. Выкинуть
почти новый дисковый телефон? Нет. Когда-нибудь он может мне
пригодиться.

Особенно больно сейчас вспоминать даже не то, что я собирал всё это
бесполезное барахло. Особенно больно вспоминать, как я часто тратил
весьма нужные мне на тот момент деньги на вещи, которые были мне
совершенно не нужны.

Почему я так поступал? Потому что люди, которые продают вещи, являются
настоящими мастерами своего дела. Двадцатипятилетний молодой человек
не может состязаться на равных с компаниями, которые потратили годы,
исследуя способы заставить Вас потратить деньги. Эти компании сделали
покупку вещей настолько приятной, что шоппинг даже стал видом отдыха.

Как защититься от этих людей? Это непросто. Я считаю себя скептиком,
однако их трюки отлично срабатывали на мне, даже когда мне было далеко
за тридцать. Но есть один способ, который может сработать. Спросите
себя перед покупкой чего бы то ни было — «сумеет ли эта вещь
значительно улучшить мою жизнь?».

Одна моя подруга вылечила себя от привычки покупать кучу одежды
следующим образом. Каждый раз перед покупкой она задавала себе вопрос:
«А собираюсь ли я носить это постоянно?». Если она не могла убедить
саму себя, что эта вещь войдёт в число нескольких любимых, она
отказывалась от покупки. Я думаю, это может сработать для любой
покупки. Просто спросите себя, прежде чем достать деньги из кошелька —
«Собираюсь ли я использовать эту вещь каждый день? Или это просто
славная и красивая вещь? Или, возможно, это самая обыкновенная вещь?»

Самые худшие вещи, кстати, это вещи, которые Вы не используете, так
как они слишком хороши. Ничего не порабощает тебя так сильно, как
хрупкие вещи. Например, «изящный фарфор», которым многие владеют, но
не пьют из него чай из боязни разбить.

Ещё один способ сопротивляться покупкам — это считать общую стоимость
владения барахлом. Ведь цена покупки — это всего лишь начало. Далее Вы
вынуждены думать о купленной вещи годами. Возможно, до конца Вашей
жизни. Каждая вещь, которой Вы владеете, забирает у Вас энергию.
Некоторые вещи дают больше чем забирают. И это — единственная
категория вещей, которыми стоит обладать.

Сейчас я перестал копить вещи. Кроме книг. Книги — это совсем другое
дело. Книги больше похожи на жидкость, чем на отдельные объекты. Нет
ничего особо неудобного в обладании несколькими тысячами книг. Тогда
как если бы у Вас было несколько тысяч обычных вещей, Вы были бы
местной знаменитостью. Но, за исключением книг, я теперь активно
избегаю вещей. Если я хочу потратить деньги на развлечения, я выберу
услуги, а не вещи.

Я не заявляю, что, подобно монахам, достиг просветления и освободился
от груза этого бренного мира. Я говорю про более приземлённые материи.
Времена изменились, и я это теперь осознал. Раньше вещи представляли
собой ценность, а теперь — нет.

В середине двадцатого века, в развитых странах то же самое произошло с
едой. После того, как еда стала дешёвой (или мы стали богатыми), стало
куда как опаснее переедать, чем недоедать. Ну а теперь эту невидимую
границу перешли и вещи. Для большинства людей, богатых или бедных,
вещи стали бременем.

Впрочем, у всего этого есть и хорошая сторона. Если Вы внезапно
обнаруживаете, что Вас всю жизнь тянул к земле лишний груз — у Вас
появляется возможность избавиться от него.Только представьте себе, что
Вы годами носили двухкилограммовые гири на ногах и вдруг сумели их
снять.

\end{document}
