\documentclass[ebook,12pt,oneside,openany]{memoir}
\usepackage[utf8x]{inputenc} \usepackage[russian]{babel}
\usepackage[papersize={90mm,120mm}, margin=2mm]{geometry}
\sloppy
\usepackage{url} \title{Пол Грэм задумался о смерти и обновил свой
  TODO-лист} \author{Пол Грэм} \date{}
\begin{document}
\maketitle

Медсестра Бронни Вэир, оказывающая моральную поддержку душевнобольным,
составила список самых больших сожалений перед смертью. Этот список
выглядит очень правдоподобно. И я совершал (и до сих пор совершаю)
минимум 4 из этих 5 ошибок.

Главная мысль — не будьте винтиком в системе. Из этих пяти сожалений и
состоит образ пост-индустриального человека, который вгоняет себя в
рамки своих реалий и покорно вертится в них, пока не остановится.

Настораживает то, что все эти ошибки, которые приводят к сожалению,
это «ошибки упущения». Вы забиваете на свои мечты, игнорируете свою
семью, подавляете свои чувства, пренебрегаете своими друзьями и
забываете быть счастливыми. «Ошибки упущения» = самый опасный тип
ошибок, потому что вы совершаете их по умолчанию.

Мне бы хотелось избегать таких ошибок. Но как? В идеале вы должны
изменить свою жизнь так, чтобы она имела другие «установки по
умолчанию». Но это может быть невозможно сделать полностью. Пока вы
совершаете эти ошибки, вы, вероятно, должны напоминать себе, что их не
стоит делать. Я решил инвертировать этот список сожалений в список
пяти команд:

не забивайте на свои мечты;

не работайте слишком много;

говорите всегда то, что думаете;

поощряйте дружбу;

будьте счастливы.

А потом поместил их в начало списка своих дел.

\end{document}
