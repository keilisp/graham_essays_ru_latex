\documentclass[ebook,12pt,oneside,openany]{memoir}
\usepackage[utf8x]{inputenc} \usepackage[russian]{babel}
\usepackage[papersize={90mm,120mm}, margin=2mm]{geometry}
\sloppy
\usepackage{url} \title{Ложь, которую мы говорим детям} \author{Пол
  Грэм} \date{}
\begin{document}
\maketitle

Взрослые постоянно лгут детям. Я не говорю, что мы должны это
немедленно прекратить, но мне кажется, нам стоит по крайней мере
задуматься о том, какую неправду мы говорим и почему.

Это может оказаться полезным и для нас самих. Всех обманывали в
детстве, и кое-что из той лжи до сих пор влияет на нас. Так что
изучив, как взрослые обманывают детей, мы сможем избавиться от лжи,
которую говорили нам.

Я использую понятие «ложь» в очень широком смысле: не только явная
ложь, но и более тонкие способы введения детей в заблуждение. Хотя
слово «ложь» имеет негативный оттенок, я и в мыслях не держу
предлагать вам никогда не лгать — просто мы должны быть внимательны,
когда это делаем. [1]

Одна из самых поразительных вещей в том, как мы обманываем детей, —
это широта заговора. Все взрослые знают, о чем лжет детям культура:
это вопросы, на которые вы отвечаете «Спроси у своих родителей». Если
ребенок спросил, кто выиграл первенство по бейсболу в США в 1982 году,
или какой атомный вес у углерода, вы можете просто ответить ему. Но
если ребенок спрашивает вас «Бог существует?» или «Кто такая
проститутка?» — вы, наверное, ответите «Спроси у своих родителей».

Поскольку мы все действуем в соответствии с этой негласной
договоренностью, дети видят мало изъянов в картине мира, которую им
показывают. Наибольшие разногласия возникают между родителями и
школой, но даже они невелики. Школы очень осторожно высказываются в
отношении сомнительных тем, и если они противоречат тому, во что
должны верить дети по мнению родителей, родители либо давят на школу,
чтобы та замолчала, либо переводят детей в другую школу.

Заговора придерживаются так тщательно, что большинство детей
обнаруживают его, только если замечают внутренние противоречия в том,
что им говорят. Это может их травмировать так же сильно, как если бы
они проснулись на операционном столе во время операции. Вот что
случилось с Эйнштейном:

Читая научно-популярные книги, я вскоре пришел к выводу, что
большинство историй из Библии — неправда. Потрясение было так велико,
что результатом стало фанатичное свободомыслие вкупе с впечатлением,
что молодежь намеренно вводят в заблуждение очевидными неправдами. [2]

Я помню это чувство. К пятнадцати я был убежден, что мир испорчен от
начала и до конца. Вот почему кинофильмы вроде Матрицы производят
такой резонанс. Каждый ребенок растет в поддельном мире. В каком-то
смысле было бы проще, если бы за этим действительно стояла банда
злобных машин, и можно было избавиться от навязанных фантомов, просто
приняв таблетку.

Защита

Если вы спросите взрослых, зачем они врут детям, наиболее частым
ответом будет «чтобы защитить их». И детям действительно требуется
защита. Например, новорожденному ребенку нужно окружение, весьма
отличающееся от улиц большого города.

Это так очевидно, что кажется неправильным называть это ложью. Ведь
безусловно не плохо обманывать ребенка, создавая у него ощущение, что
мир тихий, теплый и безопасный. Но этот безвредный вид лжи может
испортиться, если оставить его без присмотра.

Представьте, что вы пытаетесь держать кого-либо в настолько же
защищенной среде, как новорожденного, до тех пор, пока ему не
исполнится 18. Такой грубый обман об окружающем мире покажется скорее
не защитой, а жестокостью. Конечно, это крайний случай; когда родители
делают что-либо подобное, это попадает в выпуски новостей. Но такая же
проблема в меньшем масштабе видна в чувстве тревоги подростков из
американского пригорода.

Основное назначение пригородов состоит в том, чтобы обеспечить
безопасную среду для роста детей. И для десятилетних это здорово. Мне
нравилось жить в предместье, когда мне было 10. Тогда я не замечал,
насколько стерильным оно было. Для меня весь мир состоял из домов моих
друзей, к которым я ездил на велосипеде, и леса, в котором мы играли.
На мерной шкале я находился где-то посередине между колыбелью и земным
шаром. Улицы предместий были как раз моего размера. Но когда я
повзрослел, пригород стал казаться удушливо ненастоящим.

Жизнь может быть хорошей в 10 и в 20, но в 15 она часто
разочаровывает. Это слишком большой вопрос, чтобы раскрывать его здесь
полностью, но одна из причин, почему у детей в 15 лет неуютная жизнь,
определенно в том, что они заперты в мире для десятилетних.

От чего именно родители надеются защитить своих детей, воспитывая их в
пригороде? Знакомая, которая переехала из Манхаттена, просто сказала,
что ее трехлетняя дочь "видела слишком многое". Первое, что приходит в
голову по этому поводу: пьяницы и наркоманы, нищета, безумства города,
отвратительные больницы, различные сексуальные извращения, а также
жестокость и злоба.

Я думаю, злоба беспокоила бы меня больше всего, если бы мне было 3
года. Мне было 29, когда я переехал в Нью-Йорк и я был удивлен даже
тогда. Я бы не хотел, будучи трехлетним, видеть те пререкания, которые
я видел. Это было бы слишком страшно. Множество вещей, которые
взрослые скрывают от маленьких детей, они скрывают, потому что эти
вещи были бы страшными, а не потому, что они хотят скрыть их
существование. То что они при этом вводят детей в заблуждение - это
лишь побочное следствие таких действий.

Это кажется одним из наиболее оправданных типов лжи взрослых детям. Но
потому, что эта ложь не явная мы не очень-то их учитываем. Родители
знают, что они скрывали факты о сексе, и многие в какой-то момент
усаживали детей и рассказывали им больше. Но мало кто рассказывает
своим детям о различии между реальным миром и коконом, в котором они
выросли. Соедините это с доверием, которое родители пытаются внушить
своим детям, и каждый год вы будете получать новый выводок 18-летних,
которые думают, что знают как надо жить.

Разве они не всегда думали, что знают, как надо жить? На самом деле,
это нечто новое. Они думают так всего лишь последние сто лет. В
доиндустриальные времена подростки были младшими членами мира взрослых
и сравнительно хорошо осознавали свои недостатки. Они могли видеть,
что они не так сильны и умелы, как сельский кузнец. В прошлом люди
лгали детям о некоторых вещах больше, чем мы, но непрямая ложь в
искуственной, защищенной среде - это недавнее изобретение. Как и
множество новых изобретений, богатые получили его первыми. Дети
королей и великих магнатов были первыми взращенными без контакта с
миром. Предместья означают, что теперь половина населения может жить в
этом отношении как короли.

Секс (и наркотики)

У меня было бы много разных поводов для беспокойства насчет того, как
растить детей-подростков в Нью-Йорке. Я бы меньше беспокоился о том,
что они видят и больше о том, что они делают. Я ходил в колледж со
множеством детей, которые выросли в Манхеттене, и как правило, они
казались довольно пресыщенными. Казалось, что они потеряли невинность
в среднем в 14 лет и попробовали больше наркотиков, чем я знал их
названий.

Причины, по которым родители не хотят, чтобы их дети-подростки
занимались сексом, сложны. Есть некоторые очевидные опасности:
беременность и заболевания, передающиеся половым путем. Но это далеко
не все причины, по которым родители не хотят, чтобы у их детей был
секс. Средние родители 14-летней девушки возненавидели бы мысль о том,
что она занимается сексом, даже если риск забеременеть или заразиться
был бы нулевой.

Дети, вероятно, догадываются, что им рассказывают не всё. В конце
концов, беременность и заболевания, передающиеся половым путем -- это
такая же проблема для взрослых и тем не менее, они занимаются сексом.

Что на самом деле беспокоит родителей в том, что их дети занимаются
сексом? Их неприязнь этой идеи настолько внутренняя, что она выглядит
как врожденная. Но если она врожденная, она должна была быть
универсальной, а в то же время существует множество обществ, в которых
родители не против того, чтобы их дети тинейджеры занимались сексом --
там, где 14-летним нормально становиться матерями. Так в чем же дело?
По всей видимости, существует всеобщее табу на секс с детьми, не
достигшими половой зрелости, и можно легко представить себе
эволюционные причины для этого. И я думаю, в этом-то и состоит
основная причина, по которой родители в индустриальных обществах не
любят подростков, занимающихся сексом. Они все еще считают их детьми,
даже если те биологически уже ими не являются, так что табу против
детского секса все еще в силе.

Одну вещь, которую взрослые скрывают о сексе, они также скрывают о
наркотиках: от них можно получить огромное удовольствие. Это то, что
делает секс и наркотики такими опасными. Сильное желание их может
затмить рассудительность, что особо пугает, если это никудышная
рассудительность подростка.

Здесь желания родителей противоречат друг другу. В старину общества
говорили детям, что их рассудительность плоха, но современные родители
хотят, чтобы их дети были уверены в себе. Это может быть лучшей
схемой, чем старый подход, ставящий детей на место, но у него есть
побочный эффект в том, что после того, как мы неявно лгали своим детям
о том, как хороша их рассудительность, мы затем должны лгать им обо
всех вещах, из-за которых дети могут попасть в беду, поверив нам.

Если бы родители говорили своим детям правду о сексе и наркотиках, это
могло бы звучать так: причина, по которой ты должен избегать этих
вещей в том, что у тебя мало мудрости. Люди с вдвое большим чем у тебя
опытом всё еще обжигаются на этих вещах. Но это может быть один из тех
случаев, когда правда не будет убедительной, потому что один из
признаков плохого уровня рассудительности -- вера в то, что у тебя
хорошая рассудительность. Если у тебя не хватает сил, чтобы поднять
что-то тяжелое, это сразу видно; но если ты принимаешь решения
импульсивно, здесь ты более уверен в себе.

Невинность

Другая причина нежелания родителей того, чтобы их дети занимались
сексом, заключается в том, что они хотят сохранить их невинность. У
взрослых есть определенная модель того, как детям допускается себя
вести, и она отличается от той, которую они ожидают от других
взрослых.

Одним из наиболее очевидных отличий являются разрешенные детям слова.
Большинство родителей в разговорах с другими взрослыми используют те
слова, которые они не желают слышать от своих детей. Они пытаются
сокрыть даже сам факт существования этих слов так долго, насколько это
возможно. И это один их тех сговоров, в котором участвуют все без
исключения: все знают, что недопустимо ругаться в присутствии детей.

Из всего того, что родители говорят детям, я никогда не слышал более
отличающихся друг от друга объяснений почему детям нельзя ругаться.
Все родители, которых я знаю, запрещают своим детям ругаться, и не
найдется и двух человек с одинаковым обоснованием этого. Очевидно, что
многие сначала просто не хотят, чтобы дети ругались, оставляя
объяснение причин запрета на потом.

Моя теория о происходящем гласит о том, что функция бранных слов -
помечать говорящего как взрослого. Нет никакой разницы в значениях
слов "дерьмо" и "кака". Так почему же одно из них детям позволено
употреблять, а другое - нет? Единственное объяснение -- по
определению. [3]

Почему, когда дети делают вещи, дозволенные только взрослым, это так
беспокоит взрослых? Образ сквернословящего, циничного десятилетнего
ребенка, слоняющегося вокруг фонарного столба со свисающей с уголка
рта сигаретой, очень смущает. Но почему?

Одной из причин нашего желания сохранить детей невинными является то,
что мы запрограммированы любить определенные виды беспомощности. Я
несколько раз слышал, как матери говорили, что они нарочно
воздерживались от корректировки неправильного произношения у их
маленьких детей, потому что они такие очаровательные. И если подумать,
очаровательность -- это беспомощность. У игрушек и мультипликационных
персонажей, созданных милыми, всегда растерянное выражение лица и
слабые конечности.

Неудивительно, что у нас врожденное желание любить и защищать
беспомощных созданий, учитывая, что человеческие детеныши так
беспомощны на протяжении столь долгого времени. При отсутствии
беспомощности, которая делает детей очаровательными, они были бы очень
докучливыми. Они просто выглядели бы некомпетентными взрослыми. Но
здесь все гораздо сложнее. Причина, по которой наш гипотетический
пресытившийся десятилетний ребенок так меня беспокоит, состоит в том,
что он будет не просто надоедливым, а в том, что он перестанет
взрослеть по-настоящему. Чтобы быть пресытившимся, приходится думать,
что ты уже знаешь устройство мира, а все теории, возникающие в голове
десятилетнего ребенка, наверняка будут очень недалекими.

Неиспорченность также означает восприимчивость. Мы хотим, чтобы дети
были неиспорченными для того, чтобы они смогли продолжать обучаться.
Звучит парадоксально, но некоторые виды знаний являются помехой для
других видов знаний. Если вам предстоит узнать, что мир -- жестокое
место, полное людей, которые хотят использовать других для собственной
выгоды, вам лучше узнать это в последнюю очередь. Иначе вы перестанете
быть заинтересованными в познании еще многого.

Очень умные взрослые зачастую выглядят необычно наивными, и я не
думаю, что это совпадение. Я думаю, они осознанно уклоняются от
познания определенных вещей. Я поступаю точно так же. Раньше я думал,
что хочу знать обо всем. Теперь я знаю, что не хочу.

Смерть

После секса, смерть является темой, по которой взрослые наиболее
заметно врут детям. Секс, я уверен, скрывается из-за глубоких табу. Но
почему мы скрываем от детей смерть? Возможно потому, что она
чрезвычайно шокирует маленьких детей. Они хотят чувствовать себя
защищенными, а смерть -- самая большая опасность.

Одной из наиболее впечатляющих является ложь наших родителей о смерти
нашего первого кота. Через несколько лет, когда мы просим
подробностей, они вынуждены придумывать еще и еще, так что история
становится довольно замысловатой. Кот умер в кабинете ветеринара. От
чего? От наркоза. Зачем кота привезли к ветеринару? Чтобы
кастрировать. А почему такая обыденная операция его убила? Это не было
виной ветеринара; у кота с рождения было слабое сердце; наркоз был
слишком силен для него; а узнать заранее об этом не было возможности.
Только в двадцать с чем-то лет, я наконец-то узнал правду: моя сестра,
будучи около трех лет отроду, случайно наступила на кота и сломала ему
позвоночник.

Они не чувствовали необходимости говорить нам, что кот сейчас счастлив
в кошачьем раю. Мои родители никогда не заявляли, что умершие люди и
животные "попадают в лучшее место" или что мы встретим их снова.
Кажется, это нам не повредило.

Моя бабушка рассказала нам подкорректированную версию смерти моего
дедушки. Она сказала, что однажды они сидели и читали, и когда она
что-то ему сказала, он не ответил. Казалось, что он спал, но когда она
попыталась разбудить его, у нее ничего не вышло. "Он покинул нас". По
ее словам выходило, что сердечный приступ похож на засыпание. Позже я
узнал, что все было не совсем так: он умирал от сердечного приступа
почти целый день.

Помимо такой откровенной лжи наверняка было много уходов от темы,
когда речь заходила о смерти. Я этого, разумеется, не помню, но могу
догадаться, потому что до девятнадцати лет я так и не осознал, что
умру. Как я так долго мог упускать что-то настолько очевидное? Теперь,
когда я видел, как родители меняют тему разговора, я понимаю почему:
вопрос о смерти мягко, но твёрдо обходится стороной.

Особенно в этом вопросе дети частично идут им навстречу. Дети зачастую
хотят, чтобы их обманывали. Они так же хотят верить, что они живут в
комфортном, безопасном мире, как и их родители хотят им это внушить.
[4]

Собственное "Я"

Некоторые родители чувствуют стойкую приверженность к этнической или
религиозной группе и хотят, чтобы их дети разделяли эти чувства. Это,
обычно, требует два вида лжи: первая говорить ребенку, что он или она
-- это икс, а вторая -- какие угодно виды лжи, специфичные для
самоопределения иксов.

Говорить детям, что они относятся к некототорой этнической или
религиозной группе -- это одна из самых прилипчивых вещей, которую вы
можете ему сказать. Почти всё, что вы говорите детям, они смогут
переосмыслить, когда начнут думать самостоятельно. Но если вы скажете
ребенку, что он член определенной группы, в этом практически
невозможно будет разубедиться.

И это несмотря на то, что это одна из самых предумышленных лжей
родителей. Когда родители исповедуют разные религии, они часто
договариваются между собой, что их дети будут "воспитываться как
иксы". И это работает. Дети услужливо растут, рассматривая себя иксом
несмотря на то, что, если бы их родители решили по-другому, они бы
выросли, считая себя игреком.

Одна причина, по которой это так хорошо работает в том, что тут
вовлечен второй тип лжи. Правда -- общая собственность. Вы не можете
выделить свою группу среди других, делая разумные вещи и веря в то,
что является правдой. Если вы хотите отделить себя от других, вы
должны делать произвольные вещи и верить в то, что является ложью. И
после того, как они потратили свои жизни целиком, делая произвольные
вещи и веря в неправду, и выглядя чужими для "посторонних" с этой
позиции, когнитивный диссонанс подталкивающий детей к тому, чтобы
считать себя иксами, должен быть огромным. Если они не икс, почему они
связаны именно с этими верованиями и обычаями? Если они не икс, почему
все остальные называют их так?

Эта форма лжи небесполезна. Её можно использовать в качестве носителя
ценного груза положительных убеждений, и они также также могут стать
частью личности ребенка. Можно сказать ребенку, что, помимо того, что
иксы никогда не носят желтую одежду, верят, что мир был создан
гигантским кроликом, а, перед тем, как есть рыбу, всегда щелкают
пальцами, они еще весьма честны и трудолюбивы. И тогда дети икс
вырастут, чувствуя, что часть их личности - это честность и
трудолюбие.

Возможно, это было одной из причин распространения современных религий
и объясняет, почему их доктрины являются сочетанием полезного и
странного. Их странность -- это то, что притягивает к себе, а их
полезная часть несет в себе ценный груз.

Авторитет

Одна из самых непростительных причин, по которым взрослые лгут детям
-- получение над ними власти. Иногда эта ложь очень дурна, как в том
случае, когда насильник говорит своим жертвам, что им будет хуже, если
они кому-нибудь расскажут о случившемся. Иногда это более невинно; все
зависит от того, насколько плохо взрослые лгут, чтобы получить власть
и для чего они эту власть используют.

Большинство взрослых стараются скрыть свои изъяны от детей. Обычно
причины этого неоднозначны. К примеру, отец, который закрутил
романчик, обычно скрывает это от своих детей. Часть причины в том, что
это обеспокоит их, часть в том, что это обнажит тему секса, и часть
(гораздо бОльшая, чем он сам признает) в том, что он не хочет
опорочить себя в их глазах.

Если вы хотите узнать какую ложь говорят детям, прочтите практически
любую книгу, призванную рассказать им о "деликатных вопросах". [7]
Питер Мейл (Peter Mayle) написал подобную книгу, под названием "Почему
мы разводимся?". Она начинается с трех наиболее важных моментов,
которые нужно помнить при разводе, первая из них заключается в
следующем:

Вы не должны винить кого-нибудь одного из родителей, потому что при
разводе никогда не бывает, что виноват кто-то один. [8]

Так ли это? Когда мужчина уходит к секретарше, всегда ли в этом
отчасти виновата его жена? Но я могу понять, почему Мейл утверждает
это. Может быть, детям более важно уважать своих родителей, чем знать
всю правду о них.

Но из-за того, что взрослые скрывают свои изъяны, и в то же время
требуют от детей высоких стандартов поведения, большинство детей
растет с чувством собственной ущербности. Они зацикливаются на чувстве
ужасной злости на себя за использование бранных слов, в то время как
фактически большинство окружающих их взрослых делают куда более плохие
вещи.

Такое случается и в интелектуальных вопросах, так же, как и в
моральных. Чем более человек уверен в себе, тем более он готов
ответить на какой-либо вопрос "Я не знаю". Менее уверенные в себе люди
думают, что у них должен быть ответ, иначе они будут выглядеть плохо в
глазах других людей. Мои родители достаточно легко признавали, что они
чего-то не знают, но, должно быть, я часто слышал ложь подобного рода
от учителей, потому что я практически никогда не слышал фразу "я не
знаю" от учителя, пока не пошёл в колледж. Я это запомнил, потому что
было весьма удивительно услышать, что кто-то может сказать такое перед
целым классом.

Первую подсказку о том, что учитель не был всеведущим, я получил в
шестом классе после того, как мой отец опровергнул нечто, выученное
мной в школе. Когда я не согласился, сказав, что учитель сообщил
обратное, мой отец ответил, что тот парень ничего не смыслит в том о
чем говорит, он же всего лишь учитель младших классов, в конце концов.

Всего лишь учитель? Фраза казалась даже грамматически неправильной.
Разве учителя не знают все о предмете, который они преподают? А если
нет, то почему они учат нас?

Печальный факт состоит в том, что учителя государственных американских
школ не понимают хорошо в целом то, чему они учат. Есть некоторые
безупречные исключения, но как правило, люди, планирующие пойти в
учителя, находятся на нижнем уровне успеваемости среди учеников
колледжа. Так что тот факт, что в 11 я всё еще думал, что учителя были
безошибочными, показывает какую работу система, должно быть, проделала
над моим мозгом.

Школа

То, чему детей учат в школе - сложная смесь обманов. Наиболее
простительны те, которые сообщаются дабы упростить идеи, чтобы их
стало проще выучить. Проблема в том, что масса пропаганды влезает в
учебный план во имя упрощения.

Учебники обычных школ представляют из себя компромисс между тем, что
различные влиятельные группы хотят сообщить детям. Ложь редко бывает
неприкрытой. Обычно, она состоит или из упущений или из придавания
слишком большого значения некоторым темам в ущерб другим. Взгляд на
историю, который мы получили в младшей школе был аляповатой
идеализацией с хотя бы одним представителем каждой влиятельной группы.

Известными учеными, которых я припоминаю, были Эйнштейн, Мария Кюри и
Джордж Вашингтон Карвер. Эйнштейн был важной шишкой, потому что его
работа привела к созданию атомной бомбы, Мария Кюри работала над
рентгеновскими лучами. Но я был озадачен Карвером. Он, кажется,
занимался чепухой с арахисом.

Сейчас очевидно, что он был в списке, потому что был черным (и кстати,
Мария Кюри была там, потому что она -- женщина), но будучи ребенком, я
был озадачен его присутствием на многие годы. Мне интересно, не было
бы лучше, просто сказать правду, что не существовало известных черных
ученых. Ставя Джорджа Вашингтона Карвера на одну доску с Эйнштейном,
нас сбили с толку не только в отношении науки, но и в отношении
преград, с которыми встречались черные в то время.

Чем менее точной была наука, тем чаще в ней появлялась ложь. Когда
дело доходило до политики и новейшей истории, то, что нам преподавали,
состояло уже практически из сплошной пропаганды. Например, нас учили
относиться к политическим лидерам, как к святым, особенно к недавно
трагически погибшим Кеннеди и Кингу. Позднее я с изумлением узнал, что
они оба были неисправимыми бабниками, а Кеннеди вдобавок еще и сидел
на амфетаминах (к тому времени, когда я узнал, что Кинг был
плагиатором, я уже утратил способность удивляться порокам
знаменитостей).

Я сомневаюсь, что вы сможете преподать детям новейшую историю без лжи,
потому что практически любой, кому есть что сказать об этом, насаждает
некую свою точку зрения. Большая часть новейшей истории состоит из
впаривания своей версии событий. Возможно, будет лучше просто обучить
их метаутверждениям вроде этого.

Возможно, самая большая ложь, которую дети слышат в школе, это то, что
путь к успеху лежит через соблюдение "правил". На самом же деле,
большинство таких правил -- это просто хитрые приемы эффективного
управления большими группами людей.

Мир

Из всех причин, по поторым мы лжем детям, вероятно, самая веская
причина - это та же банальная причина, почему они врут нам.

Часто мы лжем людям не потому, что это входит в нашу сознательную
стратегию, а потому, что боимся их резкой реакции на правду. Детям,
почти по определению, недостаёт самоконтроля. Они бурно реагируют на
многие вещи, и поэтому им часто говорят неправду. [9]

Несколько лет тому назад, в День благодарения, мой приятель оказался в
ситуации, которая идеально иллюстрирует те сложные мотивы, которыми мы
руководствуемся, когда врем детям. Когда на столе появилась жареная
индейка, его на удивление чувствительный пятилетний сын вдруг спросил,
хотела ли индейка своей смерти. Предвидя беду, мой приятель и его жена
быстро сымпровизировали: да, индейка хотела умереть, и более того, она
прожила всю свою жизнь, мечтая стать их ужином в День благодарения. На
этом (уф, слава Богу!) всё и закончилось.

Всякий раз, когда мы лжем детям, чтобы защитить их, мы также делаем
это, чтобы сохранить мир.

Одним из следствий результата такой успокаивающей лжи -- это то, что
мы растем, считая, что ужасные вещи нормальны. Нам сложно
почувствовать побуждение к действию от событий, о которых нас,
буквально, натренировали не волноваться. Когда мне было 10, я видел
документальный фильм о загрязнении среды, что вызвало у меня панику.
Мне казалось, что планета безвозвратно разрушается. После чего я пошел
к маме, чтобы спросить ее, так ли это на самом деле. Я не помню, что
она мне сказала, но это заставило меня чувствовать себя лучше, и я
перестал беспокоиться об этом.

Это, возможно, было лучшим способом общаться с испуганным десятилетним
ребенком. Но мы должны осознавать цену этого. Этот вид лжи является
одной из причин, по которым плохие вещи продолжают существовать: нас
всех натренировали не замечать их.

Детоксикация

Спринтер после старта гонки почти сразу приходит в состояние под
названием "дефицит кислорода". Его тело переключается на резервный
источник энергии, более быстрый, чем обычное аэробное дыхание. Однако
в результате этого процесса в организме накапливаются токсичные
продукты жизнедеятельности, а для их разложения в конце концов
требуется дополнительный кислород, поэтому в конце гонки он вынужден
останавливаться и тяжело дышать какое-то время, чтобы восстановиться.

Мы подходим к своему совершеннолетию с неким "дефицитом правды". Нам
много врали, чтобы провести нас (и наших родителей) через детство.
Какая-то из этой лжи, возможно, была необходима. Какая-то, наверно,
была не нужна. Но у всех нас голова забита ложью в момент, когда мы
становимся взрослыми.

Никогда не бывает такого момента, чтобы взрослые усадили вас и
объяснили всю ложь, которую они говорили. Они забыли большую ее часть.
Так что если вы собираетесь вычистить эту ложь из своей головы, вам
придется сделать это самостоятельно.

Немногие так поступают. Большинство продолжают жить с кусками
оберточной бумаги, прилипшей к мозгам, и даже не подозревают об этом.
Вероятно, вы никогда не сможете полностью нейтрализовать эффект от лжи
сказанной вам в детстве. Однако попробовать стоит. Я обнаружил, что
если мне удавалось развенчать какую-либо сказанную мне ложь, множество
других вещей становились на свои места.

К счастью, становясь взрослыми, вы можете использовать новые ценные
ресурсы, чтобы выявить сказанную вам ложь. Вы стали одним из лжецов.
Вы теперь видите всю подноготную того, как взрослые преподносят мир
следующему поколению детей.

Первый шаг на пути к ясности мышления заключается в осознании того,
насколько сильно вы отличаетесь от нейтрального наблюдателя. По
окончании школы, я полагал, что являюсь полным скептиком. Я понял, что
школа -- это лицемерие. Я думал, что готов поставить под сомнение все,
что знал. Но среди многочисленных вещей, которые я не знал, было то,
сколько на самом деле мусора в моей голове. Неверно полагать, что ваше
сознание - чистый лист бумаги. Вам нужно сознательно стирать его.

Примечания

[1] Одна из причин использования такого простого слова в том, что
ложь, навязываемая нами детям, вероятно, не является такой безобидной,
как мы думаем. Если вы посмотрите на то, что взрослые говорили детям в
прошлом, вас буквально шокирует, насколько сильно они лгали. Подобно
нам с вами, они делали это из лучших побуждений. Поэтому, мы
обманываем себя, если полагаем, что мы насколько возможно открыто
общаемся в детьми. Как мы бываем шокированы, узнавая о лжи,
распространяемой людьми 100 лет назад, равно и люди спустя 100 лет
будут шокированы, узнав об лжи, которую мы говорим сейчас.

Я не могу предсказать что это будет за ложь, но я не хотел бы написать
заметку, которая покажется глупой спустя 100 лет. Поэтому, вместо
использования особых эвфемизмов для обозначения лжи, которую в
соответствии с сегодняшней модой мы считаем безобидной, я просто
назову всякую нашу ложь ложью.

(Я пропустил один тип: ложь, связанная с игрой на детской
доверчивости. Это начиная с розыгрышей, которые на самом деле не
являются ложью, поскольку мы подмигиваем, говоря ее, и заканчивая
устрашающей ложью старших братьев и сестер. Об этом не так много есть
что сказать: я бы не хотел, чтобы первое исчезло, и я не думаю, что
исчезнет второе).

[2] Алиса Калапрайс, "Цитируемый Эйнштейн", Принстон Университи Пресс,
1996

[3] Если вы спросите родителей, почему дети не должны ругаться, то
менее образованные обычно отвечают что-то вроде "это неправильно" (что
сразу же предполагает новый вопрос вроде "а почему неправильно?"). А
более образованные родители приведут подробное логическое обоснование.
В действительности, менее образованные родители куда ближе к правде.

[4] Как заметил один мой знакомый с маленькими детьми, им достаточно
легко считать себя бессмертными, поскольку время течет для них очень
медленно. Для трехлетнего ребенка день ощущается как месяц для
взрослого. Следовательно, 80 лет звучит для них как 2400.

[5] Я понимаю что получу бесконечное горе классифицируя религию как
разновидность лжи. Обычно люди обходят этот вопрос, при этом
подразумевается, что ложь в которую верит существенно большое
количество людей достаточно длительное время имеет иммунитет к обычным
стандартам для правды. Но поскольку я не могу предсказать какую ложь
будущие поколения будут осуждать как непростительную, я не могу
безопасно пропустить какую-либо, в том числе и эту. Да, сейчас
выглядит сомнительно, что религия выйдет из моды в ближайшие 100 лет,
но не более сомнительно чем для людей в 1880 году выглядело бы
предположение, что школьников в 1980 будут учить что мастурбация это
совершенно нормально, и не следует чувствовать себя виноватым по этому
поводу.

[6] К сожалению, "полезная нагрузка" может состоять не только из
хороших обычаев, но и из плохих. Например, есть определенные качества,
которые некоторые не-белые сообщества в США называют "acting white" -
"подражание белым".

На самом деле большую часть из них можно точно так же назвать
"подражание японцам". Нет ничего конкретно белого в этих обычаях. Они
являются общими для всех культур с давними традициями жизни в городах.
Так что, вероятно, нет большого смысла, если какое-то сообщество будет
намеренно делать наоборот и считать это частью своей
самоидентификации.

[7] В данном контексте "деликатные вопросы" (англ. issues) означает
"вещи, о которых мы хотим соврать". Потому мы и придумали для их
обозначения специальное слово.

[8] Петер Майл, "Почему мы разводимся", 1988

[9] Ирония в том, что по этой же причине и дети врут родителям. Если
вы "взрываетесь", когда люди говорят вам неприятные вещи, они не будут
вам их говорить. Подростки не говорят своим родителям что произошло
прошлой ночью, когда они решили остаться в доме друга по той же
причине, что и родители не говорят пятилетнему ребенку правду об
индейке на день благодарения. Потому что они разозлятся, если узнают.

\end{document}
