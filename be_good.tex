\documentclass[ebook,12pt,oneside,openany]{memoir}
\usepackage[utf8x]{inputenc} \usepackage[russian]{babel}
\usepackage[papersize={90mm,120mm}, margin=2mm]{geometry}
\sloppy
\usepackage{url} \title{Будь хорошим} \author{Пол Грэм} \date{}
\begin{document}
\maketitle

Примерно через месяц после запуска Y

Combinator мы пришли к фразе, которая впоследствии стала нашим
девизом: "Делай то, что хотят люди." С тех пор мы многому научились,
но если бы я выбирал сейчас, я бы не изменил своего решения.

Ещё одна вещь, которую мы говорим основателям -- не слишком
беспокойтесь о бизнес-модели, по крайней мере поначалу. Не потому, что
зарабатывать деньги не важно, а потому, что это гораздо легче, чем
создать что-то стоящее.

Пару недель назад я понял, что если совместить эти две идеи, то
выходит кое-что удивительное. Делайте то, чего хотят люди. Не
беспокойтесь особо о зарабатывании денег. Да это же определение
благотворительности.

Когда приходишь к такому неожиданному результату, это может быть как
глупостью, так и новым открытием. Либо бизнес не должен быть похож на
благотворительность, и исходные принципы были ошибкой, либо у нас
новая идея.

Подозреваю, верно второе, ведь как только у меня появилась эта мысль,
целая цепочка других была расставлена по местам.

Примеры

К примеру, Craigslist. Это не благотворительная организация, но
действуют они так, как будто являются ею. И они поразительно успешны.
Когда просматриваешь список наиболее популярных интернет-сайтов, число
работников Craigslist больше похоже на опечатку. Их доходы не так
высоки, как могли бы быть, но большинство стартап-компаний были бы
счастливы поменяться с ними местами.

Капитаны в романах Патрика О'Брайена (Patrick O'Brian) всегда пытались
заходить с наветренной стороны своих противников. Если вы с
наветренной стороны, вы принимаете решение, когда именно атаковать
другой корабль. Craigslist идет по ветру огромных доходов. Они
встретят некоторые трудности, если захотят зарабатывать еще больше, но
это будут не те трудности, которые вы испытаете, пытаясь продвинуть
паршивый продукт безразличным пользователям, потратив в десять раз
больше на продажи, чем на разработку. [1]

Я не говорю, что все стартапы должны стремиться к тому, чтобы
закончить как Craigslist. Он -- результат необычных обстоятельств. Но
он также хорошая модель для ранних стадий.

Google выглядел скорее благотворителем в начале. Они не показывали
рекламу более года. В первый год Google были неотличимы от
некоммерческой организации. Если некоммерческая или государственная
организация начала бы проект по индексации интернета, Google в первый
год был пределом того, что они могли бы сделать.

Возвращаясь назад -- когда я работал над спам-фильтрами, я думал это
будет хорошая идея -- почтовый сервис с веб-интерфейсом с хорошей
фильтрацией спама. Я не думал об этом как о компании. Я просто хотел
оградить людей от потока спама. Но чем больше я думал об этом проекте,
тем больше я убеждался в том, что, вероятно, это должно быть
компанией. Работа этого проекта будет чего-то стоить, а финансирование
посредством грантов и пожертвований будет головной болью.

Это было удивительное открытие. Компании часто утверждают, что они
приносят пользу обществу, но я внезапно понял, что есть такие
общественно полезные проекты, которые могут существовать, только став
коммерческими компаниями.

Я не хотел создавать еще одну компанию, поэтому я этого не сделал. Но
если бы кто-то сделал, то сейчас он наверное был бы довольно богат.
Было окно длительностью около двух лет, когда поток спама быстро рос,
а фильтры на всех крупных почтовых сервисах никуда не годились. Если
бы кто-то запустил новый сервис без спама, люди побежали бы туда
толпой.

Какие из этого можно сделать выводы? Из разных направлений можно
прийти в одну точку. Если вы взглянете на успешные стартапы, вы можете
обнаружить, что они часто вели себя как некоммерческие компании. А
если вы посмотрите на идеи для некоммерческих сервисов, вы поймёте,
что из них часто могли бы получиться хорошие стартапы.

Сила

Насколько велика эта территория? Могут ли все хорошие некоммерческие
организации стать хорошими коммерческими компаниями? Вероятно нет. Что
сделало Google таким ценным, так это то, что у его пользователей есть
деньги. Если вы добились того, что люди с деньгами вас любят, вы
можете надеяться получить некоторую часть их денег. Но можете ли вы
создать успешный стартап, который будет вести себя как некоммерческая
организация с потребителями, у которых нет денег? Можете ли вы,
например, вырастить успешный стартап из лечения немодной, но
смертельно опасной болезни, например малярии?

Я не уверен, но думаю что если вы вложите свои силы в идею, вы будете
удивлены как далеко она пойдет. Например люди пришедшие в Y Combinator
обычно не богаты, и тем не менее мы получаем прибыль помогая им,
потому что с нашей помощью они могут делать деньги. Может, и с
малярией можно так сделать. Возможно, организация которая помогла
избавить от нее страну, сможет получить пользу от последующего роста.

Я не утверждаю что это серьезная идея, потому что я не знаю ничего о
малярии. Но я работаю с идеями достаточно долго, чтобы понять, когда я
натыкаюсь на стоящую.

Один из способов понять как далеко пойдет идея, это спросить себя, в
какой момент вы бы сделали ставку против нее. Мысль о том, чтобы
сделать ставку против доброжелательности, вызывает тревогу, примерно
так же, как если вы заявите, что нечто технически невозможно. Вы
просто напрашиваетесь на то, чтобы оказаться в дураках, потому что в
обоих случаях вы делаете ставку против очень мощных сил. [2]

В начале я думал что этот принцип применяется только к интернет
стартапам. Очевидно что это работает для Google, но что можно сказать
о Microsoft? Microsoft ведь не доброжелателен? Но если посмотреть в
прошлое, то я думаю они были именно такими. По сравнению с IBM они
казались Робином Гудом. Когда IBM представила PC, они предполагали
зарабатывать деньги продажей аппаратных частей по высоким ценам. Но
получив контроль над стандартом на PC, Microsoft открыла рынок для
всех производителей. Цены на аппаратную составляющую быстро упали и
множество людей получили возможность приобрести компьютер, хотя раньше
не могли себе этого позволить. Подобных свершений вы бы могли ожидать
от Google.

Microsoft не столь великодушна сейчас. Сейчас, если подумать о том,
что Microsoft делает с пользователями, все приходящие на ум слова --
непечатные. [3] И тем не менее не похоже, что это окупается. Цена
акций Microsoft остается постоянной уже многие годы. А в те времена,
когда она была Робином Гудом, ее акции росли как сейчас у Google.
Может быть здесь есть взаимосвязь?

Теперь вы видите как все обстоит. Когда ваша компания мала, вы не
можете диктовать свои условия заказчикам, вы должны очаровывать их.
Когда же вы становитель большой компанией, вы можете запугивать их,
если захотите, и вы будете склоняться именно к этому, потому что это
легче, чем удовлетворить их требования. Чтобы вырасти, нужно быть
хорошим, но чтобы оставаться большим, быть хорошим уже не обязательно.

Это сходит вам с рук, пока не изменятся основные условия, после чего
все ваши жертвы сбегают. Так что, возможно, лозунг "Не будь злым"
(Don't be evil) — это самое ценное, что Пол Бухайт (Paul Buchheit)
сделал для Google, потому что это лозунг может оказаться эликсиром
корпоративной молодости. Я думаю, сейчас они считают его большим
неудобством, но подумайте как это будет ценно, если он спасет их от
падения в фатальную лень, которая поразила Microsoft и IBM.

Любопытно то, что этот эликсир свободно доступен любой другой
компании. Кто угодно может принять девиз "Не будь злым". Проблема в
том, что ему потом придется следовать, иначе вас не поймут. Так что я
не думаю, что рекорд-лейблы или табачные компании смогут использовать
эту находку.

Боевой дух

Есть много внешних признаков, что доброжелательность работает. Но как
работает? Одно из преимуществ инвестирования в большое количество
стартапов в том, что ты получаешь большой объем данных о том как они
работают. Из того, что мы видим, стремление быть хорошими помогает
стартапам в трех направлениях: это повышает их мораль, это побуждает
других людей помогать им, и прежде всего, это помогает им быть
решительными.

Мораль чрезвычайно важна для стартапа -- настолько важна, что ее одной
почти всегда достаточно для достижения успеха. Стартапы часто
описываются как русские горки. В одно мгновение ты владеешь миром, в
следующее -- ты обречен. Проблема с чувством обреченности не только в
том, что это делает тебя несчастным, но и в том, что останавливает
твою работу. Так что спуск по русским горкам чаще исполняемое
пророчество, чем подъем. Если чувство того, что ты двигаешься к успеху
дает тебе силы работатать усерднее, что наверняка увеличивает шансы на
конечный успех, то ощущение надвигающегося провала делает невозможным
продолжение работы, что практически гарантирует его.

Это и есть то место, где доброта приходит на помощь. Если ты
чувствуешь, что реально помогаешь людям, ты продолжаешь работать
несмотря на то, что все указывает на то, что стартап проваливается.
Большинство из нас имеют некоторое количество доброты от природы. Сам
факт того, что кто-то нуждается в вас, заставляет вас захотеть помочь
им. Так что если вы начинает такой стартап, куда пользователи
возвращаются каждый день, вы фактически строите гигантский тамагочи.
Вы делаете нечто, о чем вам нужно заботиться.

Blogger -- известный пример стартапа, который прошел через
действительно самую нижнюю точку и выжил. В один момент у них
кончились деньги и все покинули стартап. Эван Вильямс (Evan Williams)
пришел на работу на следующий день, на рабочем месте не было никого
кроме него. Что остановило его от ухода? Частично то, что пользователи
нуждались в нем. Он обслуживал хостинг тысяч пользовательских блогов.
Он не мог позволить сайту умереть.

Есть много преимуществ быстрого запуска, но важнейшим может быть то,
что как только у вас появляются пользователи -- эффект тамагочи
запускается. Как только появляются пользователи, о которых надо
заботиться, вы усиленно стараетесь понять то, что сделает их
счастливыми, и это действительно очень важная информация.

Дополнительная уверенность, которая приходит вместе с попытками помочь
людям, также может помочь вам и с инвесторами. Один из основателей
Chatterous сказал мне недавно, что он и его партнер были уверены, что
их сервис то, в чем мир нуждался, так что они продолжали работать над
ним не смотря на то, что им пришлось вернуться в Канаду и жить у своих
родителей.

Как только они поняли это, они перестали так сильно беспокоиться от
том, что думают о них инвесторы. Они продолжали с ними встречаться, но
они не собирались закрываться, если не получат денег. И знаете что?
Инвесторы стали намного более заинтересованными. Они поняли, что
Chatterous собираются сделать этот стартап, с ними или без них.

Если вы действительно преданы своему делу и расходы на работу вашего
стартапа невелики, ваше предприятие будет очень живучим. Практически
все стартапы, даже самые успешные, в какой-то момент близки к провалу.
Таким образом ощущение того, что вы помогаете людям, делает вас
сильнее, что само по себе с лихвой компенсирует любые потери от того,
что вы не выбрали более эгоистичный проект.

Помощь

Другое преимущество ""быть хорошим"" -- это то, что другие люди хотят
помочь вам. Это также выглядит врожденной чертой в людях.

Один из стартапов, который мы финансировали, Octopart, сейчас
находится в состоянии классического противоборства между добром и
злом. Они предоставляют собой поисковый сайт для промышленных
компонентов. Многим людям нужен такой поиск, и до появления Octopart
хорошего поиска не существовало. Как выяснилось, это было не случайно.

Octopart организовали правильный способ для поиска компонентов.
Пользователям понравилось это, и их количество стало стремительно
расти. И большую часть жизни Octopart, крупнейший дистрибьютор,
Digi-Key, старался заставить их убрать цены с сайта. Octopart
направлял пользователей к ним бесплатно, но все же Digi-Key старался
остановить этот поток. Почему? Потому что основа их бизнес-модели --
переплата, которую делают люди, не владеющие полной информацией по
ценам. Они не хотят, чтобы поиск работал.

Ребята из Octopart лучшие в мире. Они бросили докторскую программу по
физике в Беркли, чтобы сделать это. Они всего лишь хотели решить
проблему, с которой они столкнулись в своем исследовании. Представьте
сколько времени вы поможете сэкономить инженерам во всем мире, если
дать им возможность искать в онлайн. Так что, когда я услышал, что
большая злая компания пытается остановить их, хочет, чтобы их поиск не
работал, мне действительно захотелось помочь им. Это заставило меня
тратить больше времени на Octopart, чем на другие стартапы, которые мы
финансируем. Это только что заставило меня потратить несколько минут
на то, чтобы рассказать вам какие они замечательные. Почему? А потому
что они хорошие ребята, и пытаются сделать что-то хорошее.

Если вы доброжелательны, люди собираются вокруг вас: инвесторы,
клиенты, другие компании и потенциальные работники. В долгосрочной
перспективе наиболее важными могут оказаться потенциальные сотрудники.
Я думаю что теперь все знают, что [ссылка]хорошие специалисты[/ссылка]
намного лучше посредственностей. Если вы сможете привлечь лучших
специалистов, как смогла Google, у вас есть большое преимущество. А
самые лучшие специалисты часто бывают идеалистами. Им не надо искать
работу, поэтому они могут работать там, где хотят. Так что большинство
из них хотят работать над вещами, которые делают мир лучше.

Компас

Но самое большое преимущество ""быть хорошим"" -- оно действует как
компас. Одна из самых сложных особенностей при работе в стартапе --
очень много ситуаций, где надо делать выбор. Вас всего двое или трое,
и есть тысячи вещей, которые можно сделать. Как вы будете выбирать?

Ответ таков -- делайте то, что лучше всего для ваших пользователей. Вы
можете держаться за это как за трос во время урагана, и это спасет
вас, если вас вообще возможно спасти. Следуйте этому принципу и он
проведет вас через все, что необходимо сделать.

Это даже служит ответом на вопросы, выглядящие не относящимися к делу,
например, как убедить инвестора дать вам денег. Если вы хороший
продавец, то вы можете просто попробовать уговорить их. Но более
надежный путь -- убедить их с помощью ваших пользователей: если вы
делаете что-то, что нравится пользователям настолько, что они хотят
рассказать об этом своим друзьям -- вы растете экспоненциально, а это
убедит любого инвестора.

"Быть хорошим" очень полезная стратегия для принятия решения в сложных
ситуациях, потому что она не зависит от предыдущих действий. Это как
говорить правду. Проблема с враньем в том, что нужно помнить все, что
сказал в прошлом, и быть уверенным в том, что не противоречишь себе.
Если же вы сказали правду, вам не нужно ничего помнить, и это
действительно полезное свойство в области, где события происходят
быстро.

К примеру, Y Combinator к настоящему моменту инвестировал в 80
стартапов, 57 из них до сих живы. (Остальные либо закрылись, либо
слились или были поглощены другими компаниями.) Когда вы пытаетесь
консультировать 57 стартапов, получается, что вы должны иметь алгоритм
без состояний. Вы не можете иметь скрытых мотивов, когда у вас есть 57
событий, происходящих одновременно, потому что вы не можете запомнить
их. Так что наше правило состоит всего лишь в том, чтобы делать то,
что лучше всего для основателей стартапа. Не потому, что мы какие-то
особенно хорошие, а потому, что это единственный алгоритм, который
работает в подобном масштабе.

Когда вы пишете о том, что людям нужно быть хорошими, выглядит так,
как будто вы заявляете, что и сами хороший. Я хочу сказать
определенно, что я не достаточно хороший. Когда я был ребенком, я
крепко засел в лагере плохих. В ситуациях, когда взрослые использовали
слово хороший, выглядело так, как будто это синоним слова "кроткий",
так что я вырос недоверчивым к нему.

Вы знаете, что бывают такие люди -- когда их имя всплывает в
разговоре, каждый говорит: "он такой замечательный". Люди никогда не
говорили так обо мне. Лучшее, что говорили про меня -- "у него хорошие
намерения". Я не заявляю, что я хороший. В лучшем случае, я могу
заставить себя быть хорошим.

Я не предлагаю быть хорошим в обычном ханжеском смысле. Я предлагаю
это, потому что оно работает. Это будет работать не только как
заявление о "ценностях," но и как руководство к выбору стратегии, и
даже как спецификация на дизайн приложения. Не просто "не будь злым."
Будь хорошим.

Примечания

[1] 50 лет назад выглядело бы шокирующим, если бы публичная компания
не платила дивиденды. Сейчас многие технические компании их не платят.
Кажется, рынки научились оценивать потенциальные дивиденды. Вполне
может быть, что и это не последний шаг в этой эволюции. Может быть в
конечном счёте рынки будут удовлетворены потенциальными заработками.
(Венчурные капиталисты уже так работают, и по крайней мере некоторые
из них последовательно зарабатывают деньги.)

Я понимаю, что это звучит так же, как все эти разговоры про "новую
экономику" во времена пузыря доткомов. Поверьте мне, тогда я не был
одним из слепо верующих. Но я убежден, что в пузыре были и хорошие
идеи. Например, что нормально сфокусироваться на росте, а не на
прибыли -- но только если это настоящий рост. Пользователей нельзя
покупать, иначе это будет пирамидой. Но компания с быстрым и настоящим
ростом имеет ценность, а любая настоящая ценность будет в какой-то
момент распознана рынком.

[2] Идея основать компанию с добрыми целями сейчас недооценена, потому
что тот тип людей, которые в наше время объявляют это своей целью,
обычно не слишком хорошо работают.

Для трастоманов (trustafarians) начать некий вродебыполезный бизнес --
один из стандартных путей развития. Проблема большинства этих проектов
в том, что они либо имеют сомнительную политическую программу, либо
слабо исполнены. Предки трастоманов стали богатыми не потому, что
сохраняли свою традиционную культуру; возможно, люди из Боливии тоже
не хотят сохранять свою. Кроме того, создание, например, органической
фермы, несмотря на то, что это прямое благо, не помогает людям в том
масштабе, в котором это делает Google.

Из общественно полезных проектов, заявленных как таковые, большинство
ведет себя недостаточно ответственно. Они действуют так, как будто
хороших намерений достаточно, чтобы гарантировать хорошие результаты.

[3] Пользователям настолько не нравится их новая ОС, что они пишут
петиции, чтобы сохранить предыдущую версию. Хотя и предыдущая версия
не представляет из себя ничего особенного. Хакеры из Microsoft
наверняка в глубине души понимают, что если бы компания действительно
заботилась о своих пользователях, она бы просто предложила им
переключиться на OSX.

\end{document}
