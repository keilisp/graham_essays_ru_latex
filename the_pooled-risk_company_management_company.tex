\documentclass[ebook,12pt,oneside,openany]{memoir}
\usepackage[utf8x]{inputenc} \usepackage[russian]{babel}
\usepackage[papersize={90mm,120mm}, margin=2mm]{geometry}
\sloppy
\usepackage{url} \title{Управляющая компания с объединённым страховым
  фондом} \author{Пол Грэм} \date{}
\begin{document}
\maketitle

В этом году в школе стартапов Девид Хайнемайер Ханссон (David
Heinemeier Hansson) выступил с лекцией, в которой высказал мнение о
том, что создатели стартапов должны все делать по-старому. Вместо того
чтобы надеяться разбогатеть за счет создания дорогостоящей компании и
последующей продажи основного капитала при «выходе из дела»,
учредители должны создавать компании, которые приносят деньги, и жить
на доходы.

Звучит хорошо. Давайте подумаем об оптимальном пути реализации этого
плана.

Недостаток того, чтобы жить на доходы вашей компании в том, что вы
должны продолжать управлять ею. И любой, кто управляет собственным
бизнесом, скажет вам, что это требует полного внимания. Вы не можете
начать дело и затем устраниться, когда дело идет хорошо, иначе на
удивление быстро все станет плохо.

Кажется, что основные экономические стимулы учредителей стартапа --
это свобода и гарантированность. Они хотят иметь достаточно денег, (а)
чтобы не беспокоится о том, что окажутся на мели, и (б) они хотят
распоряжаться своим временем так, как им хочется. Управление своим
делом не дает ни того, ни другого. Безусловно, вы лишены свободы:
никто так не нужен, как босс. И, конечно же, нет гарантированности,
потому что как только вы перестанете уделять внимание вашей компании,
ее доходы уплывут, а вместе с ними и ваши собственные доходы.

Для многих было бы самым лучшим нанять управляющего компании, когда
она вырастет до определенных размеров. Предположим, вы смогли найти,
действительно, хорошего управляющего. Тогда вы имеете и свободу, и
гарантированность. Вы сможете уделять внимание вашему бизнесу в той
мере, в какой вам будет угодно, зная, что ваш управляющий будет
следить за тем, чтобы все шло гладко. В этом случае доходы будут
продолжать расти, а вы будете иметь и гарантированность.

Конечно, найдутся учредители, которым эта идея придется не по нраву:
это те, кому так сильно нравится управлять своей компанией, что они не
хотели бы заниматься чем-либо еще. Но, похоже, что таких мало. Залог
успеха в большинстве видов бизнеса – это фанатичная забота о нуждах
потребителей. Каковы шансы, что ваши собственные желания точно
совпадут с требованиями этой мощной внешней силы?

Естественно, управление собственной компанией может быть весьма
интересным. Viaweb была намного интереснее, чем любая другая работа,
какую мне приходилось выполнять прежде. Я на этом заработал намного
больше денег, и соотношение дохода к скуке, сопровождающей то, что я
делал, составляет порядки величины. Но было ли это самой интересной
работой, какую я мог бы себе представить? Нет.

Является ли число учредителей в подобном положении неограниченно
приближающимся к большинству или просто большим, неважно; несомненно
одно: их много. Для них правильным было бы со временем передать
компанию профессиональному управляющему, если они смогут найти
достаточно хорошего кандидата.

Ну, пока все, кажется, складывается хорошо. Но что, если ваш
управляющий попал под автобус? То, что вам, действительно нужно, так
это управляющая компания для управления вашей компанией вместо вас.
Тогда вы больше не зависите от одного человека.

Если вы владеете недвижимостью, предназначенной для сдачи в аренду,
существуют компании, которые вы можете нанять, чтобы управлять ею
вместо вас. Некоторые из них будут делать все: искать арендаторов и
устранять течь в водопроводе. Конечно, управление компаниями намного
сложнее, чем управление недвижимостью, предназначенной для сдачи в
аренду, но давайте предположим, что есть такие управляющие компании,
которые могли бы делать это за вас. Они будут взимать с вас огромные
деньги, так стоит ли шкурка выделки? Лично я пожертвовал бы большой
процент с моих доходов за дополнительное душевное спокойствие.

Я понимаю, все, что я описываю, уже звучит слишком хорошо, чтобы быть
правдой, и все же я могу придумать еще какой-нибудь способ, чтобы
сделать эту идею еще более привлекательной. Если бы существовали
управляющие компании, существовала бы дополнительная услуга, которую
они могли бы предложить своим клиентам: клиенты могут позволить им
застраховать свои доходы посредством объединения страховых фондов. В
конце концов, даже самый профессиональный управляющий не сможет спасти
компанию, как порой случается, в случае полного исчезновения ее рынка,
точно так же, управляющий недвижимостью не сможет спасти вас от того,
что ваше здание может сгореть дотла. Но компания, управляющая
достаточно большим числом компаний, могла бы заявить всем своим
клиентам: мы объединим доходы всех ваших компаний и выплатим вам ваши
пропорциональные доли.

Если бы существовали такие управляющие компании, они предложили бы
максимум свободы и гарантированности. Некто управлял бы вашей
компанией, и даже в случае ее краха вы были бы защищены.

Давайте подумаем, как можно было бы организовать такую управляющую
компанию. Самый простой способ – это создать новый вид акционерного
капитала, который представлял бы собой совокупный общий фонд
управляемых компаний. В этом случае, подписываясь, вы обмениваете
капитал вашей компании на доли из этого общего фонда, которые
пропорциональны оценке стоимости вашей компании, с которой согласны
обе стороны. После чего вы автоматически получаете свою долю доходов
из общего фонда.

Здесь загвоздка в том, что из-за сложности аннулирования подобного
обмена, вы не сможете перейти к другим управляющим компаниям. Однако
есть способ, чтобы исправить данную ситуацию: предположим, что все
управляющие компании собрались вместе и согласились позволять своим
клиентам обменивать доли во всех их общих фондах. Тогда вы, в
сущности, могли бы одновременно выбрать все управляющие компании для
управления вашей вместо вас в любой устраивающей вас пропорции, а
позже менять свои решения так часто, как вам хочется.

Если бы существовали такие управляющие компании с объединенными
страховыми фондами, для многих людей, следующих по пути, который
пропагандируется Девидом, сотрудничество с такой компанией было бы,
как кажется, идеальным планом. Хорошая новость: они действительно
существуют. То, что я сейчас описал, есть поглощение любой открытой
акционерной компанией.

К сожалению, несмотря на то, что открытые акционерные
компании-поглотители идентичны управляющим компаниям по своей
структуре, они не считают себя таковыми. Пользуясь услугами компании,
управляющей недвижимостью, вы можете в любое время, когда пожелаете,
придти и сказать «управляйте моей недвижимостью, предназначенной для
сдачи в аренду, вместо меня», и они будут. Тогда как поглощающие
открытые акционерные компании в данном контексте чрезвычайно
переменчивы. Иногда они склонны купить, и переплатят огромную сумму; а
порой им это не интересно. Они как компании, управляющие
недвижимостью, которыми руководят сумасброды. Или точнее, Господин
Рынок Бенджамина Грэма (Benjamin Graham's Mr. Market).

Итак, пока поглощающие открытые акционерные компании лишь иногда ведут
себя как управляющие компании с объединенным страховым фондом, вам
понадобится несколько лет для того, чтоб сложились благоприятные
обстоятельства. Если вы будете ждать долго (скажем, лет пять), вполне
возможно, что вы попадете в тот период, когда некая поглощающая
открытая акционерная компания загорится желанием вас купить. Но сами
вы выбрать момент, когда этому случиться, не можете.

Вы не можете рассчитывать на то, что ваши инвесторы будут тащить вас
на протяжении долгого времени, в течение которого вам, возможно,
придется выжидать. Ваша компания должна зарабатывать деньги. Мнения
расходятся по поводу того, как рано надо сконцентрироваться на этом.
Джо Краус говорит, необходимо сразу же постараться взимать
соответствующую плату с клиентов. И все же некоторые из наиболее
успешных недавно созданных фирм, включая Google, в первое время
игнорировали доходы, а все внимание сконцентрировали исключительно на
развитии. Ответ, вероятно, зависит от типа компании, которую вы
создаете. Я могу представить некоторые, в которых для обеспечения
продаж хорошо было бы использовать эвристический подход в отношении
дизайна продукта, что для других, напротив, оказалось бы отвлекающим
фактором. Критерием, вероятно, является то, насколько это помогает вам
понять ваших пользователей.

Вы можете выбирать, какая стратегия в отношении доходов, по вашему
мнению, лучше всего подходит для создаваемой вами компании, при
условии, что вы имеете прибыль. Если вы имеете прибыль, вы получите,
по крайней мере, среднее от рынка приобретений, на котором открытые
акционерные компании на самом деле ведут себя как управляющие компании
с объединенным страховым фондом.

Девид не делает ошибку, говоря, что вы должны создать компанию, чтобы
жить на доходы, приносимые ею. Ошибка, если вы считаете, что это
как-то противоречит тому, чтобы создать компанию и продать ее. На
самом деле, для большинства людей последнее является просто
оптимальным выходом из первого.

\end{document}
