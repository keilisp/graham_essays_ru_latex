\documentclass[ebook,12pt,oneside,openany]{memoir}
\usepackage[utf8x]{inputenc} \usepackage[russian]{babel}
\usepackage[papersize={90mm,120mm}, margin=2mm]{geometry}
\sloppy
\usepackage{url} \title{Можно ли купить Кремниевую долину? Возможно}
\author{Пол Грэм} \date{}
\begin{document}
\maketitle

Многие города, глядя на Кремниевую долину, задаются вопросом: «Как нам
создать что-нибудь похожее у себя?» Естественный способ это сделать —
основать первоклассный университет в том месте, где захотели бы жить
богачи. Так случилось с Кремниевой долиной. Но можно ли ускорить этот
процесс, финансируя стартапы?

Возможно. Давайте подумаем, что для этого потребуется.

Первое, что нужно понять, — поддержка стартапов отличается от проблемы
поддержки стартапов в конкретном городе. Последнее гораздо дороже.

Люди иногда думают, что могут исправить ситуацию со стартапами в своём
городе путём основания своего Y Combinator'a, но на деле это будет
иметь эффект, близкий к нулю. Я знаю, потому что он имел почти нулевой
эффект в Бостоне, где мы находились около полугода. Мы спонсировали
людей со всей страны (на самом деле, мира), и после этого они шли
туда, где могли бы получить еще больше финансовой поддержки, проще
говоря, в Кремниевую долину.

Первоначальное спонсирование — это не региональный бизнес, потому что
на этой стадии стартапы мобильны. Это просто несколько человек с
ноутбуками. [1]

Если хотите поддержать стартапы в определенном городе, вы должны
спонсировать проекты, которые не покинут город. Есть два пути решения:
иметь законы, которые запретят им уезжать, либо спонсировать до тех
пор, пока они не осядут полноценно. Первый подход ошибочен, это просто
фильтр для плохих стартапов. Если ваши правила навязывают
разработчикам то, чего они делать не хотят, то только отчаянный
согласится на ваши деньги.

Хорошие стартапы уедут в города с лучшим финансированием. Они не
согласятся остаться при последующей необходимости инвестиций.
Единственный способ оставить их — это дать достаточно денег, чтобы им
не пришлось переезжать.

Сколько для этого нужно? Если вы хотите, чтобы стартапы не покидали
ваш город, вам придётся дать им достаточно, чтобы они не искушались
предложением венчурных капиталистов Кремниевой долины, которое
потребовало бы переезда. Стартап может отказаться от такого
предложения, если он вырос до стадии, когда он (а) укоренился в вашем
городе и/или (б) настолько успешен, что венчурные капиталисты
инвестировали бы их, даже если стартап не переедет.

Сколько будет стоить вырастить стартап до такой стадии? Как минимум,
несколько сотен тысяч долларов. Кажется, Wufoo укрепились в Тампе за
\$118k, но это крайний случай. В среднем, потребовалось бы не менее
полумиллиона.

Итак, если идея вырастить местную кремниевую долину, давая стартапам
по 15-20 тысяч долларов каждому, как Y Combinator, кажется слишком
хорошей, чтобы быть правдой, то это потому, что так и есть. Чтобы
заставить их оставаться неподалёку, вам придётся дать им в 20 раз
больше, не менее того.

Однако даже это интересная перспектива. Предположим для верности, что
это стоило бы миллион долларов на стартап. Если вы смогли бы удержать
стартапы в вашем городе за миллион на каждый, тогда за миллиард
долларов вы бы могли собрать тысячу стартапов. Это, наверное, не
вывело бы вас вперёд самой Кремниевой долины, но у вас могло бы быть
второе место.

По цене футбольного стадиона, любой город с приличным уровнем жизни
мог бы превратиться в один из крупнейших центров стартапов в мире.

Более того, это не заняло бы много времени. Возможно, вам удалось бы
сделать это за пять лет. За срок полномочий мэра. И со временем было
бы проще, потому что чем больше стартапов в городе, тем меньше нужно
для привлечения новых. Когда у вас в городе была бы тысяча стартапов,
венчурные капиталисты не старались бы изо всех сил переманить их в
Кремниевую долину; вместо этого они бы открыли местные офисы. Тогда вы
были бы в хорошей форме. Вы бы начали самоподдерживающуюся цепную
реакцию, вроде той, что движет Кремниевой долиной.

Но тут появляются трудности. Вы должны отбирать стартапы. А как вы
будете это делать? Умение выбирать стартапы — это редкий и ценный
навык, и нанять людей, обладающих им, не так легко. И навык этот
настолько сложно оценить, что даже если администрация захочет их
нанять, они выберут не тех кого нужно.

Например, город даст деньги венчурным капиталистам, чтобы те
организовали местный филиал. Но только плохие специалисты согласятся
на это. Такие не будут казаться плохими администрации города. Они
будут казаться впечатляющими. Но они будут хороши во всем кроме отбора
стартапов. Это их главная проблема. Все венчурные капиталисты выглядят
результативными для ограниченного числа партнеров. Различие между
плохими и хорошими специалистами проявляется в другой половине их
работы: выборе стартапов и советах им. [2]

На самом деле вам нужна группа инвесторов-ангелов — людей, которые
вкладывают деньги, заработанные на своих стартапах. Но, к сожалению,
тут возникает проблема яйца и курицы. Если ваш город не является
платформой для стартапов, то там нет людей разбогатевших на стартапах.
И я не вижу способов привлечь таких людей из других городов. Нет такой
особой причины, по которой бы они решили переехать.

Впрочем, городская администрация может выбрать стартапы "используя"
данные неместных инвесторов. Будет довольно просто составить список
наиболее успешных ангелов и найти стартапы в которые они бы хотели
вложить свои деньги. Если город предложит таким проектам по миллиону
долларов при условии переезда, многие находящиеся на ранних стадиях
работы могут согласиться.

План звучит довольно абсурдно, но возможно это самый эффективный
способ для отбора хороших статапов.

Конечно разработчикам стартапов будет неприятно покидать своих
первоначальных инвесторов. Но с другой стороны, дополнительный миллион
долларов будет им куда полезнее.

Выживут ли переехавшие стартапы? Вполне возможно. Единственный способ
узнать это - попробовать. Это будет довольно дешевым экспериментом,
как "расходы на гражданское население". Возьмите 30 стартапов которые
недавно проинвестировали известные ангелы, дайте каждому по миллиону
долларов если те согласятся переехать, и через год посмотрите на
результат. Если они будут успешно развиваться, вы можете попытаться
пригласить еще больше разработчиков стартапов.

Не усердствуйте с юридическими условиями договора по которым они будут
иметь возможность вас покинуть. Заключите что-то вроде джентльменского
соглашения.

Не пытайтесь удешевить этот процесс и выбрать только 10 стартапов для
экспреимента. Если сделать это с малым количеством проектов то это
будет гарантировать провал. Стартапы должны быть в среде других
стартапов. 30 будет достаточно чтобы создать что-то вроде сообщества.

Не пытайтесь заставить их работать в отремонтированном складе который
вы называете "инкубатором". Реальные стартапы предпочитают свое
собственное рабочее пространство.

На деле, не налагайте никак ограничений на стартапы вообще. Стартаперы
по сути хакеры, а они руководствуются больше соглашениями чем
правилами. Если они рукопожатием заключили с вами сделку, они ее
выполнят. Но покажите им замок и их первая мысль будет - как его
взломать.

Интересно, 30-стартапный эксперимент может устроить достаточно богатый
житель. И какое же давление он может оказать на город, если все
получится.[4]

Должен ли город брать акции в обмен на деньги? В принцепе, они имеют
на это право, но как они определят стоимость акции стартапов? Не
получится просто установить всем одинаковую стоимость: для некоторых
она будет слишком низкой (и они бы от вас отказались), а для других —
слишком высокой (и ухудшило бы их работу впоследствии). А так как мы
предполагаем, что мы не можем выбирать стартапы, мы также
предполагаем, что не можем их оценивать, потому что это практически
одно и то же.

Ещё одна причина не брать акции в стартапах — стартапы часто
ввязываются в сомнительные дела. Как и сформировавшиеся компании, но
последних не винят за это. Если кто-то убивает кого-то после встречи
на Facebook, пресса будет писать об этом событии, как о связанном с
Facebook. Если кто-то кого-то убивает после встречи в супермаркете,
пресса будет писать про убийство. В общем, поймите, если вы
инвестируете стартапы, они могут создать что-нибудь, что будет
использоваться для порнографии или для файлообмена или для выражения
немодных мнений. Вам, наверное, стоило бы спонсировать такой проект
вместе с вашими политическими противниками, так чтобы они не могли
использовать то, что делают стартапы, как клюшку, которой можно вас
побить.

Хотя было бы слишком политически сдержанно просто дать денги
стартапам. Так что лучший план — превратить их в конвертируемый
кредит, но который не был бы возвращён кроме как в крупной сделке —
например, за 20 миллионов долларов.

Насколько успешной окажется данная схема зависит от конкретного
города. Есть города, например Портлэнд, который легко будет сделать
долиной стартапов, и другие, как Детроит, где это будет задачей
запредельной сложности. Так что будьте честны перед собой, насколько
реально это для вашего города, перед тем как пытаться.

Легче всего оценить так, насколько близки условия к жизни в
Сан-Франциско. У вас хороший климат? Люди живут в центре или
преимущественно по окраинам? Город свободный и терпимый, или скорее
отражающий традиционные ценности? Есть ли хорошие университеты? Есть
ли соседи к которым можно сходить? Даже гики будут чувствовать себя
как дома? Если ответ на все вопросы положительный, то вы не только
сможете провернуть такой план, но и потратить сможете даже меньше
миллиона на каждый стартап.

Я понимаю что шанс того что какой-нибудь город попытается провернуть
такое предприятие очень мал. Я просто хотел объяснить насколько
трудным это окажется, если кто-то попытается. Насколько будет трудным
основать Кремниеву долину? Поражает мысль что этого могут добиться
очень многие города. Хотя все деньги они потратят на постройку
стадиона, теперь их хотя бы спросят: почему вы решили так, вместо того
чтобы стать серьезным подспорьем Кремниевой Долине?

Примечания

[1] Вот что обнаруживают люди, открывающие предположительно местную
фирму: (а) их кандидаты приезжают отовсюду, не только с окрестностей,
и (б) местные стартапы также нанимаются к другим начинающим фирмам. В
конце концов, кандидаты распределятся по качеству, а не по
местоположению.

[2] Что интересно, плохие венчурные капиталисты терпят неудачу, выбрав
стартап, управляемый людьми, похожими на них самих — хорошими
демонстраторами, но пустыми. Тот самый случай, когда фальшивка
фальшивкой погоняет. И пока все участники внушают доверие, партнёры,
инвестирующие в такой капитал, не будут в курсе дел, пока не оценят
итоги.

[3] Подозреваю, ни даже тогда, когда вы станете налоговой гаванью. Это
привлекло бы некоторых богачей, но не таких, которые бы стали хорошими
ангелами-инвесторами в стартапах.

[4] Спасибо Майклу Кинану за то, что обратил на это внимание.

\end{document}
