\documentclass[ebook,12pt,oneside,openany]{memoir}
\usepackage[utf8x]{inputenc} \usepackage[russian]{babel}
\usepackage[papersize={90mm,120mm}, margin=2mm]{geometry}
\sloppy
\usepackage{url} \title{Как выражать несогласие} \author{Пол Грэм}
\date{}
\begin{document}
\maketitle

Сеть превращает писательство в общение. Двадцать лет назад писатели
писали, а читатели читали. Сеть дает возможность читателям отвечать —
что все больше и больше они и делают — в комментариях, на форумах и в
собственных блогах.

Часто ответы выражают несогласие и это вполне понятно. Согласие менее
мотивирует людей, чем несогласие. И если вы согласны, то вы мало что
можете добавить. Вы могли бы развить что-то сказанное автором, но он,
вероятно, уже рассмотрел наиболее интересные моменты. Когда вы не
соглашаетесь -- вы вторгаетесь на территорию, которую он, возможно, не
исследовал.

В результате получается, что несогласия выражается больше, чем
согласия, особенно если считать по количеству слов. Это не значит, что
люди стали злее. Это объясняется всего лишь изменением структуры
нашего общения. Но хотя неверно, что злость заставляет людей не
соглашаться больше, тем не менее существует опасность, что, наоборот,
несогласия могут разозлить людей. В особенности в онлайне, где легко
говорить такие вещи, которые вы никогда не сказали бы лицом к лицу.

Так что, поскольку мы все собираемся не соглашаться всё больше и
больше, нам надо приложить усилия к тому, чтобы делать это правильно.
А что значит -- не соглашаться правильно? Большинство читателей легко
могут отличить бессмысленную ругань от тщательно обоснованного
опровержения, но я считаю, что было бы полезно перечислить все
промежуточные стадии. Поэтому здесь я пытаюсь описать некую иерархию
несогласия.

Уровень 0. Поношение автора

Это самый низкий уровень в иерархии несогласия и, возможно, наиболее
распространённый. Мы все видели комментарии типа:

кг/ам!!!!!!!!!!!!

Важно понимать, что и более чётко сформулированное поношение автора
имеет не больше значения. Комментарий типа:

Автор — самовлюблённый дилетант.

в действительности имеет не больше смысла, чем банальное "кг/ам".

Уровень 1. Переход на личности

Переход на личности не так прост, как обычное поношение автора. Он
может даже иметь некоторый смысл. Например, если сенатор написал
статью, в которой говорит, что зарплаты сенаторов должны быть
увеличены, некто может прокомментировать:

Конечно он будет так говорить, он же сам сенатор.

Такой комментарий не опровергнет аргументов автора, но он, по крайней
мере, имеет отношение к предмету обсуждения. Но он всё равно остаётся
достаточно слабой формой несогласия. Если в аргументах сенатора что-то
не так, вам нужно указать — что именно, если же аргументы верны, то
какая разница, что тот, кто их высказал, сам сенатор?

Утверждение, что автор недостаточно компетентен, чтобы писать о
чём-либо, — это вариант перехода на личности, к тому же имеющий мало
смысла, потому что довольно часто хорошие идеи высказываются людьми,
не имеющими прямого отношения к обсуждаемой теме. Вопрос по-прежнему
остаётся в том, прав автор или нет. Если его некомпетентность привела
к тому, что он допускает ошибки, — укажите на них. Если же нет, то
степень отношения автора к теме не имеет значения.

Уровень 2. Реакция на авторский тон

На следующием уровне иерархии мы начинаем наблюдать уже реакцию на
содержимое статьи, а не на самого автора. Самая низшая форма такой
реакции — это несогласие с тоном автора. Например:

Удивительно, в каком надменном тоне автор отрицает теорию
искусственного сотворения мира.

Такая реакция остаётся достаточно слабой формой несогласия, хотя она и
лучше, чем атаки на самого автора. Гораздо больше значения имеет тот
факт, прав ли автор или нет, чем то, в каком тоне он высказывает свои
суждения. Особенно в случаях, когда сложно судить о тоне. Кто-то,
принимающий тему обсуждения близко к сердцу, может легко обидеться на
тон автора, в то время как остальные сочтут этот тон нейтральным.

Итак, если ваше единственное возражение это критика тона автора, вам
мало что есть сказать. Что если автор легкомысленен, но прав? Это
лучше, чем если он был бы серьёзен, но неправ. И если автор в чём-то
неправ, — укажите в чём именно.

Уровень 3. Отрицание

На этом уровне мы, наконец, переходим к ответам на то, что было
сказано, а не на то, кем или как. Низшая форма таких ответов — это
просто заявление противоположного мнения без каких-либо существенных
аргументов.

Часто такие ответы комбинируются с ответами второго уровня, например:

Удивительно, в каком надменном тоне автор отрицает теорию
искусственного сотворения мира. Эта теория — настоящая научная теория.

Иногда отрицание может иметь некоторый вес. Иногда просто высказывание
противоположного утверждения оказывается достаточным, чтобы убедиться,
что оно правильно. Но обычно требуются какие-то аргументы.

Уровень 4. Контраргументация

На четвёртом уровне мы достигли первой формы убеждающего несогласия:
контраргументации. На формы несогласия, рассмотренные ранее, как
правило, можно не обращать внимания, поскольку они ничего не
доказывают. Контраргументация кое-что может доказать. Проблема состоит
в том, что не очень понятно, что именно она доказывает.

Контраргументация — это отрицание с объяснением и/или с фактами. Когда
она применяется по отношению к первоначальным аргументам, она бывает
довольно убедительной. Но, к несчастью, довольно часто
контраргументация бывает направлена на что-то слегка другое. Обычно,
когда два человека страстно спорят о чём-то, на самом деле они спорят
о двух разных вещах. Иногда они даже соглашаются друг с другом, но,
будучи увлечены своим спором, даже не понимают с чем они соглашаются.

Причина обсуждения чего-то, отличного от того, что было затронуто
автором, иногда может иметь под собой основания: в том случае, когда
вы считаете, что упускается суть проблемы. Но и в этом случае, вам
нужно явно отметить этот факт.

Уровень 5. Опровержение

Самая убедительная форма несогласия — опровержение. Она же самая
редкая, потому что самая трудоёмкая. В самом деле, иерархия форм
несогласия образует что-то вроде пирамиды: чем выше вы по ней
поднимаетесь, тем меньше примеров встречаете.

Чтобы опровергнуть кого-то вам, вероятно, требуется его процитировать.
Вы должны найти тот самый отрывок в тексте, с которым вы не согласны,
который, как вы считаете, и содержит ложное утверждение, а затем
объяснить, почему оно ложное. Если же вы не можете найти такой
отрывок, возможно вы спорите с кем-то, кого вы сами вообразили.

Хотя опровержение обычно влечёт за собой цитирование, само по себе
наличие цитирования не обязательно подразумевает опровержение.
Некоторые комментаторы цитируют часть текста, с которым они не
согласны, чтобы создать видимость разумного опровержения, а затем
пишут ответ в форме уровня 3 или даже 0.

Уровень 6. Опровержение основной идеи

Действенность опровержения зависит от того, что вы опровергаете.
Наиболее мощная форма несогласия это опровержение чьей-то основной
идеи.

Даже на уровне 5 мы иногда наблюдаем умышленную подтасовку, когда
кто-то выбирает малозначащие утверждения в тексте и опровергает их.
Иногда способ, которым это делается, превращает такое опровержение
скорее в изощрённую форму перехода на личности, чем в реальное
опровержение. Как пример можно назвать указание на грамматические
ошибки или незначительные огрехи в именах или числах. Во всех случаях,
кроме тех, когда аргумент оппонента действительно зависит от подобных
вещей, единственная цель таких замечаний -- это дискредитация
оппонента.

Истинное опровержение требует опровержения центральной идеи, или, по
крайней мере, её части. А это, в свою очередь, подразумевает, что
требуется явно обозначить основную идею. Действительно эффективное
опровержение могло бы выглядеть так:

Основная идея автора похоже заключается в следующем: ... Он, в
частности пишет:

<цитата>

Но это неверно в силу следующих причин....

Цитата, которую вы приводите в качестве неправильной, необязательно
должна быть именно тем текстом, который написал автор. Допускается
опровергать что-то, на чём базируется точка зрения автора.

Что это значит

Итак у нас есть система классификации форм несогласия. Где она нам
может пригодиться? Во-первых, следует понимать, что эта иерархия
ничего не говорит о том, кто на самом деле прав. Эти уровни описывают
лишь формы утверждений и ничего не говорят о их правильности.
Комментарий уровня 6 вполне может быть полностью ошибочным.

Но, хотя эти уровни не устанавливают нижнюю границу убедительности
комментариев, они устанавливают верхнюю границу. Ответ уровня 6 вполне
может быть неубедительным, но ответы 2 или более низких уровней
неубедительны всегда.

Наиболее очевидное преимущество классификации форм несогласия состоит
в том, что она поможет читателям оценивать, что они читают. В
частности, она помогает выявлять тщательно замаскированные ложные
аргументы. Красноречивый писатель или оратор может создать впечатление
победы над оппонентом просто используя убедительные слова. В
реальности же, они просто маскируют качественную демагогию.
Классифицируя различные уровни несогласия, мы даём пытливому читателю
инструмент для обнаружения подобных приёмов.

Классификация может помочь и самим авторам. Как правило,
интеллектуальная нечестность -- не намеренная. Кто-то, спорящий с
тоном высказывания, с которым он не согласен, на самом деле может
верить, что он в действительно говорит что-то стоящее. Взгляд со
стороны и помещение своей позиции в иерархию несогласия может дать
толчок к тому, чтобы он попытался сформулировать контраргументацию или
опровержение.

Но самая большая польза от правильных форм несогласия не столько в
том, что они делают обсуждение более качественным, сколько в том, что
они делают людей, использующих их, счастливее. Если вы поизучаете
обсуждения, вы можете обнаружить, что намного больше идиотских
комментариев уровня 1, чем уровня 6. Когда вам действительно есть что
сказать, вы не должны выглядеть нелепо. Вам и самим этого не
захочется. Если есть что сказать, слабость формы, в которую облекается
мысль, только мешает.

Если движение вверх по иерархии делает людей менее убогими, оно делает
большинство из них счастливее. Большинство людей в действительности не
любят выглядеть идиотами, но они не знают, как поступать по-другому.

\end{document}
