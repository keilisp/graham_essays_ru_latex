\documentclass[ebook,12pt,oneside,openany]{memoir}
\usepackage[utf8x]{inputenc} \usepackage[russian]{babel}
\usepackage[papersize={90mm,120mm}, margin=2mm]{geometry}
\sloppy
\usepackage{url} \title{Копируй то, что нравится} \author{Пол Грэм}
\date{}
\begin{document}
\maketitle

В средней школе я потратил много времени, подражая плохим писателям.
На уроках английского мы изучали в основном художественную прозу, и я
считал, что это – наивысшая форма письма. Ошибка №1. Казалось, что
больше всего почитали те истории, в которых люди замысловатым образом
страдали. Что-нибудь весёлое или увлекательное было, ipso facto,
подозрительным, если не было достаточно старым, чтобы быть едва
понятным (как Шекспир или Чоусэр). Ошибка №2. Идеальным носителем
казался короткий рассказ, жанр, который, как я позже узнал, имел
недолгий срок жизни, грубо говоря, совпавший с пиком журнальных
публикаций. Но размер делал их удобными для чтения на уроках, поэтому
мы читали их множество, и создавалось впечатление, что жанр процветал.
Ошибка №3. Так как они были такими короткими, в них ничего не должно
было происходить; можно было просто показать случайный срез жизни, и
это считалось продвинутым. Ошибка №4. В результате я написал много
историй, в которых не происходило ничего, кроме того, что кто-то был
как-то, по-видимому, глубоко, несчастен.

Большую часть учёбы в колледже я специализировался на философии. Меня
впечатляли работы, публиковавшиеся в философских журналах. Они были
так хорошо написаны, а их тон был захватывающим (иначе говоря –
бессистемным и буферопереполняюще-техническим). Кто-нибудь шёл по
улице, и вдруг «модальность в качестве модальности» набрасывалась на
него. Я никогда толком не понимал этих работ, но решил, что разберусь
с этим позже, когда будет время перечитать их тщательнее. А пока что я
пытался подражать им. Это была, как я вижу сейчас, обречённая затея,
потому что они на самом деле не говорили ни о чём. Ни один философ,
например, никогда не отрицал другого, потому что никто не сказал
ничего достаточно определённого, чтобы отрицать это. Стоит ли
говорить, что мои подражания тоже не говорили ни о чём.

В магистратуре я всё равно продолжал транжирить время, имитируя не то.
Тогда был такой модный тип программ под названием «экспертная
система», ядром которого было что-то называющееся «модуль принятия
решений». Я взглянул на то, что делали эти вещи, и подумал: «Я мог бы
написать это за тысячу строк кода». А видные профессора писали целые
книги об этих системах, новые компании продавали одну копию программы
по цене годовой зарплаты. «Какая перспектива!» – подумал я – «эти
сложные вещи кажутся мне простыми. Должно быть, я очень
сообразительный». Неправильно. Это была просто временная прихоть.
Книги об экспертных системах, написанные профессорами, сейчас никому
не нужны. Они даже не вели ни к чему интересному. А покупатели платили
за них так много, потому что многие из них были теми же
правительственными агентствами, которые платили тысячи за отвёртки и
стульчаки.

Как избежать копирования неправильных вещей? Копируйте только то, что
вам интуитивно нравится. Это спасло бы меня во всех трёх случаях. Мне
не нравились эти короткие рассказы, которые мы читали на Английском. Я
ничего не узнал из философских работ. Я сам не пользовался экспертными
системами. Я считал эти вещи хорошими, потому что ими восторгались.

Отделить вещи, которые тебе нравятся, от тех, которые просто
производят впечатление, может быть трудно. Есть одна хитрость –
игнорировать подачу. Когда я вижу картину, которая впечатляюще
выставлена в музее, я спрашиваю себя: сколько бы я заплатил за неё,
если бы нашёл её на гаражной распродаже, грязной и без рамки, и совсем
не зная, кто написал её? Если вы походите по музею, проводя этот
эксперимент, вы заметите, что это даёт по-настоящему изумительный
результат. Не игнорируйте это наблюдение только потому, что оно
крайнее.

Ещё один способ выяснить, что вам нравится – посмотреть на то, что вам
нравится как запретное удовольствие. Люди, особенно молодые и
амбициозные, любят многие вещи за то, что считают достоинством любить
их. 99\% людей, читающих Улисс
\url{http://www.ireland.ru/dublin/James_Joyce/introduction.html} , во
время чтения думают: «Я читаю Улисс». А запретное удовольствие – по
крайней мере, настоящее. Что вы читаете, когда не чувствуете себя
целомудренным? Какого рода книгу вы читаете и жалеете, что осталась
только половина книги, а не впечатляетесь тем, что уже преодолели
полпути? Вот что вам на самом деле нравится.

Даже когда вы находите неподдельно хорошие вещи, которые можно
копировать, есть ещё один подводный камень, которого нужно избегать.
Будьте осторожны и копируйте то, что делает их хорошими, а не их
недостатки. Легко скатиться на имитирование недостатков, потому что их
проще видеть и, конечно, проще копировать. Например, большинство
художников 18 и 19 веков рисовали буроватыми красками. Они подражали
великим художникам Возрождения, чьи картины к тому времени потемнели
от грязи. С тех пор эти картины были очищены, открыв сверкающие
краски; а их подражатели, конечно, так и остались бурыми.

Так совпало, что именно рисование вылечило меня от копирования не
того. На середине магистратуры я решил, что хочу попробовать быть
художником, и мир искусства был настолько явно испорченным, что было
очевидно, что он держится на доверчивости. Те же люди заставили
профессоров философии казаться скрупулёзными, как математики. Выбор –
либо хорошо работать, либо быть инсайдером – стал настолько очевидным,
что мне пришлось увидеть различие. Он есть, до некоторой степени,
почти в каждой сфере, но до того момента мне удавалось избегать
встречи с ним.

Это было одной из самых ценных вещей, которым я научился в рисовании:
нужно выяснить для себя самого, что хорошее, а что нет. Нельзя
полагаться на авторитеты, про хорошее они солгут.

\end{document}
