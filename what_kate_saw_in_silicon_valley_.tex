\documentclass[ebook,12pt,oneside,openany]{memoir}
\usepackage[utf8x]{inputenc} \usepackage[russian]{babel}
\usepackage[papersize={90mm,120mm}, margin=2mm]{geometry}
\sloppy
\usepackage{url} \title{Что увидела Кейт в силиконовой долине}
\author{Пол Грэм} \date{}
\begin{document}
\maketitle

Недавно мы обратились к ней с просьбой помочь запустить YC когда она
не была занята. Несмотря на то, что она много слышала о YC с самого
его начала, ей все же пришлось полностью вникнуть в него в течение
последних 9 месяцев.

Для меня, как человека, давно имеющего дело со стартапами, такой
подход к делу кажется вполне нормальным, поэтому мне было крайне
любопытно, что же впечатлило её больше всего.

1. Насколько много неудачных стартапов. Кейт знала, что в принципе,
стартапы сами по себе очень рискованные, но она была удивлена, что
угроза провала настолько серьезна, причем не только для небольших
стартапов, но и для крупных, серьезных проектов, основатели которых
выступали на вечерах YC.


2. Насколько сильно меняется концепция стартапа. Обычно, к
предпросмотру половина стартапов совсем не похожа на то, что
планировали получить в начале. Мы приветствуем это. Работа со
стартапом сродни науке – вы должны добраться до истины в любом случае,
какой бы она ни была и куда бы она не привела. В всех остальных
случаях люди обычно начинают какое-то дело лишь после того, как
определились с тем, что хотят иметь, и, начав дело, уже придерживаются
изначального пути, даже если он ошибочный. 3. Как мало денег может
потребоваться для начала стартапа. В мире Кейт все имеет свою цену, и
весьма немалую. Инвестиции в начало стартапа сравнимы со стоимостью
косметического ремонта в ванной.


4. Фанатизм учредителей. Кейт охарактеризовала их именно так. Я
согласен с ней, но я никогда не задумывался о том, что эта черта
характера совсем не ценится в других областях, до тех пор пока она не
упомянула об этом. Если в организации кого-то назовут
«неорганизованным/больным//фанатиком», вряд ли это можно воспринимать
как комплимент.

Что это значит? Фанатик обычно умудряется быть одновременно
агрессивным и целеустремленным, являясь ярым сторонником известного
высказывания – «цель оправдывает средства». По крайней мере, нам
хочется быть такими в любом виде деятельности, ИМХО. Если ты не
агрессивен, то вряд ли ты создаешь что-то новое, а
заносчивость/самонадеянность/высокое положение вообще никого до добра
не доводили.


5. Насколько силиконовая долина является технически насыщенной.
"Похоже, все здесь в теме". Высказывание не совсем корректное, но
разница между Силиконовой долиной и любым другим местом весьма
ощутима. Поневоле приходится разговаривать полушепотом, потому как
велика вероятность того, что коллега за соседним столом знает тех
людей, которых вы обсуждаете. Я никогда не делаю так в Бостоне.
Обратная (позитивная) сторона медали заключается в том, что человек за
соседним столом скорее всего вам может помочь.


6. Насколько докладчики на YC последовательны и логичны в своих
советах. Вообще-то, я тоже это заметил. Я всегда беспокоюсь о том, что
рассказ докладчика о стартапах поставит нас в затруднительное
положение, но происходит это на удивление редко.

Когда я спросил ее о том, что ей запомнилось из того что говорили
докладчики, она ответила: наилучший способ достичь успеха, это быстрый
запуск, получить отзывы пользователей и отреагировать соответствующим
образом; стартапы требуют гибкости, потому как они напоминают катание
на американских горках, а также то, что большинство инвесторов далеки
от IT-технологий. Я был впечатлен, насколько сильно и единогласно
докладчики поддерживают принцип быстрого запуска и итерации. 10 лет
назад ситуация была прямо противоположная, но сейчас это общепринятая
практика.


7. Насколько неприметно выглядят основатели успешных стартапов.
Большинство известных основателей в Силиконовой Долине вряд ли
привлекут ваше внимание на улице. И дело не только в одежде или
внешнем виде. Нет никакой ауры силы и власти вокруг них. «Они ни на
кого не пытаются произвести впечатление.» Забавно, но хотя Кейт никак
не могла определить успешных основателей, она запросто могла узнать
инвесторов, как по одежде, так и по их манере поведения.


8. Насколько важно учредителям/основателям иметь рядом человека, у
которого можно попросить совет. Без совета «они бы просто потерялись».
К счастью, есть много людей, способные помочь им. В YC существует
хорошая традиция помогать другим стартапам, финансируемым YC. Не мы
изобрели эту идею: это всего лишь более насыщенная форма общепринятой
в Силиконовой долине традиции.


9. Насколько уникальной задачей является стартап. Архитекторы
постоянно общаются с другими людьми, а технологический процесс
стартапа содержит в себе большие этапы непрерывной работы. У вас есть
все необходимое для работы.


Преобразовав этот список, мы получим картину «нормального» мира. В нем
люди много общаются друг с другом, не спеша, плодотворно работают над
консервативными дорогими проектами, чьи задачи и цели определены
заранее, а также аккуратно подстраивающиеся под свое положение в
иерархической лестнице. Это самое точное описание дней давно минувших.
Поэтому культура стартапов не отличается от любой другой суб-культуры,
но при этом является некоей лакмусовой бумажкой, отражающей последние
новшества.

\end{document}
