\documentclass[ebook,12pt,oneside,openany]{memoir}
\usepackage[utf8x]{inputenc} \usepackage[russian]{babel}
\usepackage[papersize={90mm,120mm}, margin=2mm]{geometry}
\sloppy
\usepackage{url} \title{Жизнь действительно очень коротка} \author{Пол
  Грэм} \date{}

\begin{document}
\maketitle

Как все мы знаем, жизнь коротка. Я часто размышлял об этом, ещё когда
был ребёнком. Действительно ли жизнь так уж коротка, или же мы просто
сожалеем о том, что она конечна? Казалось бы нам, что жизнь так же
коротка, если бы мы жили в десять раз дольше? \newline

Я не нашёл способа ответить на эти вопросы и прекратил об этом
раздумывать. Потом у меня появились дети. Это дало мне возможность
ответить на беспокоившие меня вопросы и выяснить, что жизнь на самом
деле очень коротка. \newline

Дети показали мне, как «сконвертировать» непрерывный поток времени в
набор отрезков ограниченной длины. Вы можете провести только 52 пары
выходных с вашим двухлетним ребенком. Магия Рождества длится для детей
с трёх до десяти лет — это значит, что вы увидите своего ребёнка в
таком настроении не больше восьми раз. Сложно сказать, много или мало
того, что длится непрерывно — как, например, время, — но все мы
понимаем, что 8 — это не слишком большое число. Например, у вас есть
восемь орехов или восемь книг — это, безусловно, ограниченное число. \newline

Итак, жизнь действительно очень коротка. Однако есть ли хоть какая-то
разница? \newline

Для меня — есть. Это значит, что аргументы вроде «Жизнь слишком
коротка для Х» теперь приобретают большую силу. Это не просто фигура
речи. Это не синоним для того, что вам просто надоело делать. Если вы
поймаете себя на мысли, что жизнь слишком коротка для чего-либо, вам
следует по возможности выкинуть это из своей жизни. \newline

Когда я спрашиваю себя, для чего же слишком коротка моя жизнь, в
голове всплывает одно слово — для всякой ерунды. Я понимаю, что это
кажется тавтологией. Фактически, ерунду можно определить как нечто,
для чего жизнь слишком коротка. Однако кое-что всё-таки отличает
ерунду от остального — её «фальшивый» характер. Это фастфуд среди
пищи, которую даёт нам опыт. \newline

Спросите себя, тратите ли вы время на ерунду — ведь вы уже знаете
ответ. Ненужные встречи, бессмысленные споры, бюрократия, обсуждение
чужих ошибок, дорожные пробки, — вызывающие привыкание, но не
приносящие никакой пользы занятия. \newline

Такие вещи проникают в вашу жизнь двумя путями: вас либо заставляют
этим заниматься, либо вы верите, что это действительно приносит
пользу. В какой-то степени вы вынуждены мириться с тем, что вам
навязывают обстоятельства. Вам нужны деньги, а их зарабатывание в
основном заключается в исполнении вереницы поручений. Закон спроса и
предложения гарантирует: чем лучше вознаграждается тот или иной вид
работы, тем больше людей будут ей заниматься — за меньшие деньги. \newline

Однако возможно, что вам навязывают гораздо меньше, чем вам кажется.
Всегда были и будут люди, которые бросают рутинную работу и уезжают
туда, где возможностей меньше, но жизнь кажется более «натуральной».
Это становится всё более распространённой практикой. \newline

Вы можете поступить так же, но в меньших масштабах — для этого не
обязательно куда-то переезжать. Количество времени, которое вам
придётся потратить на бесполезные вещи, варьируется в зависимости от
работодателя. Большинство крупных и некоторые мелкие организации
погружены в такую работу по уши. Но если вы осознанно расставите свои
авторитеты, и избавление от бесполезных занятий окажется в вашей
системе ценностей выше денег или престижа, возможно, вам удастся найти
компанию, где вы будете тратить меньше времени на ерунду. \newline

Если вы фрилансер или руководите небольшой компанией, вы можете
сделать то же самое на клиентском уровне. Откажитесь от «токсичных»
клиентов — и вы избавитесь от большей части бесполезной работы. Это
нечто большее, чем просто уменьшение дохода. \newline

Вам действительно навязывают некоторое количество бесполезных
действий. Однако большая часть такой работы проникает в вашу жизнь по
вашей вине — хотя вам и кажется, что вам её навязывают. Что ещё
неприятнее, такую бесполезную работу — которую вы выбираете сами, —
может быть гораздо сложнее выкинуть из своей жизни. \newline

Пример, который знаком многим — споры в интернете. Когда кто-то не
соглашается с вашим мнением, он в каком-то смысле вас атакует — иногда
довольно открыто. Естественная реакция для человека — уход в оборону.
Это инстинкт. Но этот инстинкт, как и многие другие, не предназначен
для того мира, в котором мы теперь живём. Это может показаться
нелогичным, но зачастую лучше не защищаться. Это отнимает мгновения
вашей жизни. \newline

Существует множество вещей, которые гораздо более опасны, чем споры в
интернете. Одним из побочных эффектов технического прогресса является
то, что вещи, которые нам нравятся, становятся всё более
затягивающими. Нам всё чаще приходится делать сознательные усилия и
спрашивать себя: «Действительно ли это то, как я хотел бы потратить
своё время?» — чтобы отказаться от ненужных занятий. \newline

Нужно активно искать вещи, которые действительно важны — так же
активно, как и пытаться избавиться от бесполезных дел. Для разных
людей важны разные вещи, и задача состоит в том, чтобы выяснить, что
важно для вас. Некоторым повезло больше остальных — они ещё в раннем
возрасте выяснили, что любят математику, заботиться о животных или
писать заметки или книги. Затем они нашли способ посвящать этим
занятиям большую часть своего времени. Но большинство людей только
учатся отличать вещи, которые действительно имеют для них значение, от
всего остального. \newline

Молодым оказывается особенно сложно — из-за того, что они попадают во
множество искусственных ситуаций, которые меняют их представления о
важном. В школе самым важным ученику кажется то, что о нём думают
сверстники. Но спросите любого взрослого, какова была его самая
большая ошибка в юности, — и он, скорее всего, ответит, что его
слишком заботило мнение окружающих. \newline

Один из способов отличить нечто важное от всего остального — спросить
себя, будет ли это так же важно через несколько лет. График «важности»
бесполезных вещей, как правило, имеет острый пик. Именно так ерунда и
обманывает нас. Она колет наше сознание этими «пиками», как булавками. \newline

Вещи, которые действительно имеют значение, иногда сложно назвать
«важными». Выпить кофе с друзьями — вот что имеет значение. Вы никогда
не скажете, что это была пустая трата времени. \newline

Маленькие дети помогают вам тратить время на важные вещи: на них
самих. Они хватают вас за рукав, заставляя отвлечься от своего
смартфона, и спрашивают: «Ты со мной поиграешь?». \newline

Жизнь коротка — и, скорее всего, понимание этого застанет вас
врасплох. Вы принимаете некоторые вещи как должное, а они уходят или
заканчиваются. Вы полагаете, что всегда успеете написать книгу,
подняться на гору или сделать что-то ещё — а потом понимаете, что
дверь уже закрыта. \newline

Хуже всего, когда другие люди умирают. Их жизнь тоже слишком коротка.
Мне бы хотелось провести больше времени с моей покойной матерью. Я жил
так, будто она могла быть со мной вечно. Она только поддерживала эту
иллюзию с типичным для неё молчанием. Но это была лишь иллюзия.
Множество людей совершают те же ошибки, что и я. \newline

Верный способ избежать сюрпризов — постоянно быть к ним готовым.
Раньше, когда сама жизнь была опаснее, люди были постоянно готовы к
смерти — до такой степени, что сейчас это кажется пугающим. Мне
кажется, это не совсем правильно — постоянно напоминать себе о том,
что смерть со своей косой парит где-то за вашей спиной. Возможно,
лучше взглянуть на проблему с другой стороны и развивать в себе
привычку не откладывать до завтра. Не нужно ждать подходящего времени,
чтобы забраться на гору, написать книгу или приехать в гости к своим
родителям. И не нужно напоминать себе о том, почему вам не нужно
чего-то ждать. Просто не ждите. \newline

Когда человеку чего-то не хватает, он работает в двух направлениях:
пытается сохранить то, что у него уже есть, и получить ещё больше.
Такой подход работает и в этом случае. \newline

То, как вы живёте, влияет на продолжительность вашей жизни. Многие из
нас могли бы сделать свою жизнь лучше — и я в том числе. \newline

Постарайтесь обратить внимание на время, которое у вас есть. Очень
просто погрузиться в непрерывный поток дней и перестать обращать
внимание на то, что происходит вокруг, занимаясь исполнением
бесконечных поручений и тревожась о мелочах. \newline

Попробуйте «замедлить» время. Мне в этом помогли дети. Дети дарят
множество прекрасных моментов, на которых невозможно не обратить
внимание. \newline

Ещё одна вещь, которая помогает это прочувствовать, — сожаления об
упущенном опыте. Я грущу о своей матери не только потому, что скучаю
по ней, но и потому, что раздумываю о том, сколько всего мы не сделали
вместе и сколько могли бы сделать. Моему старшему сыну скоро
исполнится семь лет. Конечно, я скучаю по нему трёхлетнему, но у меня
не осталось сожалений о том, что мы чего-то не успели. Нам повезло
провести это время так хорошо, как только могут проводить время отец
со своим трёхлетним сыном. \newline

Неустанно отбрасывайте бесполезное, не ждите подходящего момента,
чтобы сделать нечто действительно важное и наслаждайтесь временем,
которое у вас есть. Жизнь слишком коротка. \newline

\end{document}
