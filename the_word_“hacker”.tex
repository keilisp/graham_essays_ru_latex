\documentclass[ebook,12pt,oneside,openany]{memoir}
\usepackage[utf8x]{inputenc} \usepackage[russian]{babel}
\usepackage[papersize={90mm,120mm}, margin=2mm]{geometry}
\sloppy
\usepackage{url} \title{Слово «хакер»} \author{Пол Грэм} \date{}
\begin{document}
\maketitle

В СМИ слово «хакер» используют для описания того, кто взламывает
компьютеры. В среде разработчиков программного обеспечения это слово
означает искусного программиста. Но между этими двумя понятиями
существует связь. Для программистов «хакер» ассоциируется с
мастерством в самом буквальном смысле слова: некто, кто может
заставить компьютер делать то, что ему заблагорассудится, хочет того
сам компьютер или нет.

В дополнение к этой путанице укажем, что существительное «hack»
(англ.) также имеет два значения. Это слово может быть как
комплиментом, так и оскорблением. Слово «hack» используется для
описания ужасного результата вашей деятельности. Но когда вы
совершаете что-то так талантливо, что каким-то образом побеждаете
систему, то это также называют «hack». Данное слово употребляется чаще
всего в первом значении, а не во втором, вероятно, потому, что плохие
решения преобладают над блестящими.

Хотите — верьте, хотите — нет, но эти два значения слова «hack» также
связаны. У плохих и оригинальных решений есть нечто общее: они оба
идут вразрез с правилами. И существует постепенный переход от
нарушения правил, граничащего с безобразностью (использование клейкой
ленты для присоединения чего-либо к вашему велосипеду), к такому
нарушению правил, что сродни блестяще оригинальному (отказ от
Евклидового пространства).



Хакинг зародился еще до появления компьютеров. Работая над
«Манхэттенским проектом», Ричард Фейнман развлекался, бывало, тем, что
взламывал сейфы с секретными документами. Эта тенденция продолжается и
по сей день. Когда мы учились в магистратуре, у одного моего друга
хакера, который слишком много времени провел около MIT, был свой
собственный набор «отмычек» (сейчас он управляет хеджевым фондом, т.е.
занимается деятельностью, которая не так уж и далека от его предыдущих
увлечений).

Иногда, сложно объяснить властям, почему кому-то может захотеться
совершать такие вещи. У еще одного моего друга были проблемы с
правительством из-за взлома компьютеров. И только совсем недавно это
объявили преступлением, а ФБР обнаружила, что их обычные методы
ведения расследований не работают. Очевидно, что расследование полиции
начинается с мотива. Типичных причин преступлений немного: наркотики,
деньги, секс, месть. Любознательность не входила у ФБР в список
мотивов. На самом деле, сама идея казалась им чуждой.

Приближенных к власти людей раздражает общий принцип неповиновения
хакеров. Но это неповиновение является побочным продуктом синтеза тех
качеств, которые делают их искусными программистами. Они могут
посмеиваться над исполнительным директором, когда тот выражается
универсальной корпоративной речью, и они также смеются над кем-то
другим, кто утверждает, что у данной конкретной задачи нет решения.
Станете подавлять одно, и вы будете сдерживать другое.

Иногда это относится и к самому принципу. Порой юные программисты
замечают чудаковатости выдающихся хакеров и решают перенять некоторые
из них, чтобы казаться умнее. Фальшивка вызывает не только
раздражение; на самом деле обидчивость и вспыльчивость этих позеров
может замедлить инновационный процесс.

Но даже учитывая их раздражающие эксцентричные качества, принцип
неповиновения хакеров — это чистая победа. И хотелось бы, чтобы
преимущества такого принципа находили бОльшую поддержку.

Я, например, полагаю, что в Голливуде люди просто напросто озадачены
отношением хакеров к авторским правам. Это вечная тема горячих
дискуссий на Slashdot. Но почему людям, которые программируют
компьютеры, следует так переживать по поводу авторских прав, в
конце-то концов?!

Частично потому, что некоторые компании используют механизмы для
предотвращения копирования. Покажите любому хакеру замок, как первой
его мыслью будет, а как его взломать. Но есть причина посерьезнее,
почему хакеры встревожены такими инструментами как авторские права и
патенты. Все более и более агрессивные меры для защиты
«интеллектуальной собственности» они воспринимают как угрозу
интеллектуальной свободе, которая нужна им для выполнения своей
работы. И тут они правы.

Именно заглядывая в нутро действующих технологий, хакеры вырабатывают
идеи для последующих поколений. «Нет уж, спасибо, — возможно, скажут
интеллектуальные домовладельцы, — нам не нужна никакая посторонняя
помощь». Но они ошибаются. Последующие поколения компьютерных
технологий часто — возможно, даже чаще, чем наоборот — разрабатываются
не профессионалами, а людьми со стороны.


В 1977 году никто не сомневался в том, что некая группа в IBM
разрабатывает то, что, как они ожидали, будет следующим поколением
компьютеров для бизнеса. Они ошиблись. Следующее поколение компьютеров
для бизнеса разрабатывалось по совершенно другому пути двумя
длинноволосыми парнями с одинаковым именем Стив в гараже в Лос
Альтосе. В примерно это же время власть имущие объединились для
разработки официальной версии операционной системы следующего
поколения, Multics. Но двое парней, которые считали Multics чрезмерно
сложной, затаились и написали свою. Они дали ей имя с шутливой
отсылкой к Multics — Unix.

Недавние новые законы об интеллектуальной собственности налагают
беспрецедентные ограничения на тот вид любопытства, который приводит к
новым идеям. Раньше конкуренты могли воспользоваться патентом, чтобы
предотвратить продажу вами копии того, что они создали, но им бы не
удалось остановить вас от изучения какой-то одной части продукта с
целью узнать, как там все работает. Новейшие законы объявили это
преступлением. Как же мы должны разрабатывать новые технологии, если
мы не можем изучать действующие, чтобы понять, как их улучшить?

По иронии судьбы, на себя это взяли хакеры. Всему виной компьютеры.
Системы управления внутри машин обычно строятся на основе физических
принципов работы: приводы, рычаги, кулачковые механизмы. Все в большей
и большей степени принцип работы (а, следовательно, и ценность)
продуктов заключается в их программах. И под этим я подразумеваю
программы в общем смысле, т.е. данные. Песня на пластинке физически
является углублениями в пластике. Песня на диске iPod всего лишь
хранится на нем.

Данные по определению легко копировать. А Интернет позволяет их легко
распространять. Поэтому не удивительно, что компании боятся. Но, как
это часто случается, страх затуманил им разум. Правительство
отреагировало суровыми законами защиты интеллектуальной собственности.
Возможно, их задумали во благо. Но им и в голову может не прийти, что
такие законы принесут больше вреда, чем пользы.

Почему программисты так неистово выступают против этих законов? Будь я
на месте законодателя, я бы заинтересовался этим загадочным явлением —
по той же самой причине, как если бы я был фермером и внезапно услышал
бы однажды ночью громкое кудахтанье из курятника, я бы вышел
посмотреть, что там такое. Хакеры не глупы, а единодушие — большая
редкость в этом мире. Поэтому, если они все кудахчут, возможно, что-то
идет не так.

Может ли быть так, что такие законы, даже направленные на защиту
Америки, в действительности нанесут ей вред? Подумайте об этом. Есть
что-то слишком американское во взломах сейфов Фейнманом во время его
работы над проектом «Манхэттен». Сложно представить, чтобы власти
отнеслись к такого рода вещам с юмором в те времена в Германии. И
возможно, это не совпадение.

Хакеры непослушны. В этом вся суть хакинга. И это также является сутью
«американности». Не случайно Кремниевая долина находится в Америке, а
не во Франции, Германии, Англии или Японии. В тех странах люди не
выходят за очерченные рамки.

Я прожил некоторое время во Флоренции. Но после нескольких месяцев
пребывания там я понял, что то, что я неосознанно надеялся там найти,
находилось в месте, которое я только что покинул. Причина, по которой
знаменита Флоренция, заключается в том, что в 1450 году она была
Нью-Йорком. В 1450 году ее наполняли непокорные и амбициозные люди,
такие, каких сейчас вы встретите в Америке. (Поэтому я и вернулся в
Америку).

Довольно существенным преимуществом Америки является то, что она
обладает благоприятной средой для нужного вида непокорности, т.е. это
родной дом не только для умных, но и для дерзких. А хакеры — это
наглые всезнайки, без исключений. Если бы у нас был государственный
праздник, это был бы день первого апреля. О нашей работе многое
говорит тот факт, что мы используем одно и то же слово как для
блестящих, так и для жутко убогих решений. Когда мы размышляем, мы не
всегда уверены на
100% какого рода будет это решение. Но как только в нем появляется некоторая неправильность нужного характера, то это многообещающий знак. Странно, что люди воспринимают программирование как точную и систематическую деятельность. Это компьютеры точные и систематические, а хакинг — это то, чем вы занимаетесь непринужденно и с улыбкой.

В нашем мире некоторые из наиболее типичных решений не далеко ушли от
предмета острых шуток. Несомненно, IBM довольно сильно удивилась
последствиям дела о DOS лицензировании, прямо как и предполагаемый
«конкурент», когда Михаэль Рабин разобрался с проблемой путем ее
переопределения в такую, которая уже проще подавалась решению.

Циничные умники вынуждены развивать в себе четкое ощущение того,
сколько они могут унести. И не так давно хакеры почувствовали
изменение в окружающей среде, о чем на hackerlines отзывались довольно
неодобрительно.

Для хакеров недавнее ограничение гражданских свобод показалось
особенно угрожащим, что также введет в заблуждение людей несведущих.
Почему нам следует так сильно переживать за гражданские свободы?
Почему в большей степени это свойственно именно программистам, а не
стоматологам, морякам или ландшафтным дизайнерам?

Позвольте описать ситуацияю в терминах, которые одобрил бы любой
правительственный чиновник. Гражданские свободы — это не только повод
для гордости и необычная американская традиция. Гражданские права
обогощают страны. Если вы нарисуете график, отражающий зависимость ВВП
на душу населения от гражданских свобод, то заметили бы четкую
закономерность. Неужели гражданские свободы и есть причина, а не
только следствие? Полагаю, что так оно и есть. Что общество, в котором
люди могут делать и говорить, что хотят, также будет становиться и
таким, где побеждают наиболее действенные решения, а не те, что
спонсируются самыми влиятельными людьми. Авторитарные государства
становятся коррумпированными; корумпированные страны превращаются в
бедные, а те, в свою очередь, — в слабые. Такое ощущение, что наряду с
налоговыми поступлениями кривая Лаффера существует и относительно
правительственной власти. По крайней мере, это кажется вполне
возможным, так что глупо было бы проводить эксперимент и проверять. В
отличие от высоких налоговых ставок, вы не можете аннулировать
тоталитаризм, если он окажется ошибкой.

Вот что беспокоит хакеров. Правительственный шпионаж за людьми не
является прямой причиной того, что программисты пишут код хуже. В
конечном итоге это просто приведет к такому миру, в котором будут
преобладать плохие идеи. И потому, что для хакеров это так важно, они
особенно чувствительны к таким вещам. Они ощущают приближение
тоталитаризма издалека, словно животные, которые чувствуют приближение
грозы.

Было бы нелепо если бы, как того боятся хакеры, недавние меры,
направленные на защиту национальной безопасности и интеллектуальной
собственности, оказались бы боевой ракетой, нацеленной прямо на
причину успеха Америки. И это было бы не в первый раз, когда меры,
принятые в панической обстановке, имели бы эффект, отличный от
предполагаемого.

Существует такая вещь как «американность». Если вы захотите это
усвоить, то ничто не заменит вам опыт проживания заграницей. А если вы
захотите узнать, взрастит ли нечто это качество или подавит его, будет
сложно найти фокус-группу лучше хакеров, потому что из тех, кого я
знаю, они ближе всех приблизились к воплощению данного качества.
Возможно, даже ближе, чем люди, управляющие нашей страной, которые за
своими речами о патриотизме напоминают мне больше кардинала Ришелье
или его последователя Мазарини, а не Томаса Джефферсона или Джорджа
Вашингтона.

Когда вы читаете, что отцам-основателям приходится говорить в свое
оправдание, то их речи больше похожи на то, как говорят хакеры. «Дух
сопротивления правительству, — как писал Джефферсон, — так ценен в
определенных случаях, что мне бы хотелось, чтобы ему никогда не давали
погаснуть».

Представляете, если бы американский президент сказал бы такое сегодня.
Как и замечания откровенной старой бабушки, афоризмы отцов-основателей
приводили в замешательство поколения их менее уверенных наследников.
Они напоминают нам о том, откуда мы родом. Они напоминают нам о том,
что именно люди нарушают правила, которые являются источником
богатства и власти Америки.

Люди, обладающие достаточными полномочиями для установки правил,
обладают естественным желанием, чтобы этим правилам подчинялись. Но
будьте поосторожнее со своими просьбами. Они могут быть удовлетворены.

\end{document}
