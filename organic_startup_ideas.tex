\documentclass[ebook,12pt,oneside,openany]{memoir}
\usepackage[utf8x]{inputenc} \usepackage[russian]{babel}
\usepackage[papersize={90mm,120mm}, margin=2mm]{geometry}
\sloppy
\usepackage{url} \title{Идеи для «органического» стартапа} \author{Пол
  Грэм} \date{}
\begin{document}
\maketitle

Лучший способ придумать идею стартапа это задать себе вопрос: «Что вы
хотите, чтобы кто-то сделал для Вас?»

Есть два типа стартапов: те, которые органично исходят из вашей
собственной жизни, назову их оригинальными, и те, которые вы
придумаете, считая что они необходимы для определенного класса
пользователей, но не вам лично.

Apple — это пример первого типа стартапов. Apple появились благодаря
тому, что Стив Возняк захотел для себя персональный компьютер. В
отличие от большинства людей, которые хотели компьютер, он мог его
разработать один, что собственно он и сделал. А так как много других
людей, захотели то же самое, то Apple смогла сделать и продать их
достаточно, чтобы раскрутить компанию. Они кстати, по-прежнему
опираются на этот принцип, даже сегодня, между прочим, IPhone является
телефоном который хочет иметь Стив Джобс.

Наш собственный запуск Viaweb, был вторым типом стартапа. Мы сделали
программное обеспечения для создания интернет-магазинов. Нам не нужна
была эта программа для себя. Мы не были маркетологами. Мы даже не
знали, когда мы начинали, то что наши пользователи называют «прямым
маркетингом». Но мы были достаточно взрослыми, когда мы создали
компанию (мне было 30, а Роберту Моррису было 29), поэтому мы уже
видели достаточно много других продуктов, чтобы знать потребности
пользователей, и определить какой требуется тип программного
обеспечения.

Между этими двумя типами идей нет резкой грани, но самыми успешными
стартапами становятся, как показывает практика, стартапы типа Apple, а
не типа Viaweb. Когда Билл Гейтс создавал Бейсик для Альтаир, то он
делал это прежде всего для себя, так же как Ларри и Сергей, когда они
писали свою первую версию Google.

Оригинальные идеи, как правило очень интересны и продуктивны, но
особенно хорошими они получаются, у молодых авторов. Стартапы второго
вида требуют опыта, чтобы предсказать, что другие люди захотят
использовать. Самыми худшими идеями, как показывает наша практика,
получаются стартапы, когда молодежь пытается сделать вещи, которые как
они думают, потребуются другим людям.

Так что, если вы хотите основать стартап, и еще не знаете, что вы
собираетесь делать, я призываю вас, чтобы вначале вы сосредоточились
на оригинальных идеях. Что отсутствует или недостает вам в вашей
повседневной жизни? Иногда, если вы зададите этот вопрос себе, вы
получите моментальный ответ. Должно быть, Бейсика так не хватало Биллу
Гейтсу, что он занялся воплощать свою потребность.

Возможно, вам придется взглянуть на свой мир снаружи, чтобы увидеть,
то что кажущиеся вам привычные и обыденные вещи могут быть кардинально
изменены.Всегда есть великолепные идеи, которые находятся у нас под
носом. В принципе, такова история появления Facebook. Теперь, конечно,
всем ясно, что такая идея должна была воплотиться в Интернет.

Есть идеи, которые стали очевидными после своего запуска. Точно также
как и Facebook в 2004 году: оригинальная идея стартапа, вначале может
быть не похожа на конечный результат. Теперь мы знаем, идея Facebook
была очень успешной, но что он представлял из себя в 2004 году — ввод
профиля студентов, статус онлайн, не так уж и много, чтобы вырос
мощный стартап. И в самом деле, изначально это была не идея стартапа.
Этой зимой Марк сказал, что не пытался создать компанию, когда он
написал первую версию Facebook. Это был просто небольшой проект. Также
было и когда Стив Возняк в Apple начинал работать. Он не думал, что он
создает компанию. Если бы эти парни думали, что они начинают создавать
компанию, они, возможно, бы соблазнились сделать что-то более
«серьезное», и это было бы ошибкой.

Если вы хотите сделать стартап, начав с оригинальной идеи, я призываю
вас уделять больше внимания самой сути идеи, и в меньшей степени всему
остальному. Просто исправьте то, что на ваш взгляд не так работает,
независимо от того что из этого может получится. И если вы, невзирая
на кажущиеся трудности, в конечном итоге сделаете то, что окажется
ценностью для многих людей, вы внезапно увидите, что у вас появилась
своя компания и свой бизнес.

Не расстраивайтесь, если то, что вы сделаете изначально, другие люди
воспримут как простое развлечение. На самом деле, это хороший знак.
Так происходило почти со всеми отличными идеями. Первые
микрокомпьютеры использовались в качестве игрушки. И первые самолеты,
и первые автомобили. На данный момент, когда кто-то приходит к нам с
чем-то, что нравится пользователям, но что мы могли бы раскритиковать,
назвав пустой игрушкой, мы начинаем глубже задумываться и
присматриваться к проекту для инвестирования.

Хотя молодежь всегда находится в невыгодном положении, когда приходит
с идеями, но она является наилучшим источником оригинальных идей,
потому что молодежь на переднем крае технологий. Они используют все
новое. Они находят новые идеи, и новые способы использования последних
достижений науки, так почему бы и нет?

Нет ничего более ценного, чем неудовлетворенные потребности. Если вы
найдете что-то, что вы можете улучшить для многих людей, значит вы
нашли свою золотую жилу. Как и в случае с настоящим золотым рудником,
все равно придется потрудиться, чтобы получить золото из него. Самое
главное у вас есть идея — а это самая сложная часть стартапа.

\end{document}
