\documentclass[ebook,12pt,oneside,openany]{memoir}
\usepackage[utf8x]{inputenc} \usepackage[russian]{babel}
\usepackage[papersize={90mm,120mm}, margin=2mm]{geometry}
\sloppy
\usepackage{url} \title{Стартап на дошираке} \author{Пол Грэм} \date{}
\begin{document}
\maketitle

Выражение «ramen profitable» («компания, прибыли которой хватает
только на доширак») приобрело широкую популярность у нас, и я хотел бы
пояснить, что именно оно означает.

Доширак-прибыльной называют компанию-стартап, чей доход позволяет
покрыть расходы на проживание ее учредителей. Традиционно стартапы
стремились к иной форме прибыльности. Обычная прибыльность возникает,
когда начинают окупаться большие затраты, тогда как
доширак-прибыльность экономит ваше время. [1]

Раньше стартап становился прибыльным только после вложения довольно
крупной суммы денег. Так, компания, производящая компьютерное
оборудование, могла затратить 50 млн. долл. США в течение 5 лет и не
принести доход. Но со временем прибыль компании составляла 50 млн.
долл. США в год. Такая прибыльность означала, что стартап успешен.

С доширак-прибыльностью ситуация иная: прибыльным считается стартап,
который начинает приносить доход уже через два месяца. И хотя прибыль
составляет лишь 3000 долл. США в месяц, ее оказывается достаточно, так
как единственными работниками являются пара 25-летних учредителей,
которым много на жизнь и не нужно. Тем не менее, прибыль 3000 долл.
США в месяц не означает успех компании. Но у такой компании есть общее
с прибыльной в традиционном понимании этого слова: не нужно искать
деньги, чтобы выжить.

Идея доширак-прибыльности мало кому известна, потому что она стала
осуществимой лишь недавно. И она остается невозможной для многих
стартапов, например биотехнических, но вполне подходит для
малобюджетных компаний, выпускающих программное обеспечение. Ведь для
многих из них единственными издержками производства являются расходы
их учредителей.

Важной особенностью этого типа прибыльности является независимость от
инвесторов. В обычной ситуации стартап, не приносящий выгоду, вынужден
либо привлекать больше инвестиций, либо закрываться. Но если он
работает по принципу доширак-прибыльности, то лишен этого мучительного
выбора. Инвестиции привлекать можно, но это необязательно.

***

У отсутствия потребности в деньгах очевидное преимущество – в том, что
можно работать на более выгодных условиях. Зная, что стартапу нужны
деньги, инвесторы могут пользоваться положением. Некоторые могут
намеренно медлить со сделкой, понимая, что раз учредитель на мели, то
станет более уступчивым.

Также существует три менее очевидных преимущества
доширак-прибыльности. Одно из них — стартап становится более интересен
инвесторам. Если он получает доход, каким бы мелким он ни был, это
значит, что (а) удалось найти хоть какого-то инвестора, (б) учредители
серьезно намерены создать то, что востребовано, и (в) они достаточно
дисциплинированы, чтобы удерживать расходы на низком уровне.

Это обнадеживает инвесторов, так как три их основных сомнения теперь
развеяны. Обычно они спонсируют компании с толковыми учредителями и
большим рынком сбыта, но, тем не менее, терпят неудачу. Такие компании
становятся банкротами, как правило, из-за того, что (а) нет
покупателей для их продукции, например потому, что продукт было
слишком сложно продать или рынок был не готов, (б) учредители не
уделяли внимание потребностям потребителей, занимаясь решением не тех
задач, или (в) компания растратила средства до получения прибыли. Если
стартап достигает доширак-прибыльности, то уже смог избежать этих
ошибок.

Еще одним преимуществом доширак-прибыльности является моральное
состояние сотрудников. Основывая компанию, учредитель представляет
скорее ее теоретически. Хотя юридически это компания, нет ощущения,
что ее можно так назвать. Лишь когда начнут выплачиваться существенные
суммы денег, компания сможет считаться настоящей. И собственные
расходы учредителей будут свидетельствовать об этом, так как с этого
момента положение вещей изменится. Больше не нужно будет думать только
о выживании.

На данном этапе для стартапа крайне важен подъем духа, так как тяжела
именно моральная сторона управления компанией. Стартапов по-прежнему
не так много. Что же останавливает людей? Финансовые риски? Многие
люди в возрасте 25 лет все равно не имеют сбережений. Длинный рабочий
день? Многие перерабатывают и на обычных должностях. Удерживает людей
от основания стартапа страх ответственности. И этот страх вовсе не
безрассудный: его действительно сложно побороть. Избавляясь от этой
тяжести, учредители повышают шансы компании на выживание.

Шансы на успех стартапа, достигающего доширак-прибыльности, очень
высоки. Это довольно интересно, если принять во внимание следующее
бимодальное распределение результатов в стартапах: либо банкротство,
либо процветание.

Четвертое преимущество доширак-прибыльности наименее очевидное, но,
вероятно, наиболее важное. Не нужно прерывать работу в компании на
поиск денег.

Привлечение инвестиций очень отвлекает. Учредителю повезло, если его
производительность составляет треть от прежней. И так может
продолжаться месяц за месяцем.

До этого года я точно не понимал (или, скорее, не помнил), почему
поиск денег так отвлекает. Я заметил, что стартапы, которые мы
финансировали, тормозили в своем развитии, когда начинали искать
инвестиции. Но я не понимал почему именно, пока это не случилось с YC.
Для нас это не составило большого труда – первые же инвесторы ответили
мне согласием. Но затем несколько месяцев ушло проработку деталей, и
за это время я практически не успел сделать какую-либо настоящую
работу. Почему? Потому что я постоянно думал о деньгах.

В каждый период времени для стартапа наиболее актуальна какая-то одна
проблема. Это то, о чем вы думаете перед сном ночью и в душе утром. И
если вы начинаете искать деньги, это становится той проблемой, о
которой вы думаете. Вы принимаете душ лишь один раз утром, и если в
это время вы думаете об инвесторах, значит вы не думаете о продукте.

Тогда как при возможности выбора момента для поиска инвестиций,
учредитель скорее займется этим, освободившись от других задач. И,
вероятно, ему удастся ускорить процесс завершения переговоров и даже
освободиться от постоянных мыслей о нем, – так как ему не будет
принципиально важно, когда они завершатся.

***

Определение «доширак-прибыльная компания» можно понимать буквально.
Такой прибыли, как правило, не хватает, чтобы самостоятельно
«протолкнуть» стартап, то есть обойтись без помощи инвесторов. Как
показывает опыт, такое случается крайне редко. Немногим стартапам
удалось преуспеть без вложений инвесторов. Возможно, когда стартапы
подешевеют, то случаи успеха без инвесторов будут происходить чаще. С
другой стороны, деньги требуют инвестирования. И если стартапы
нуждаются в них меньше, то они охотнее возьмут их на более выгодных
для себя условиях. Именно это приведет к равновесию. [2]

Также доширак-прибыльный стартап не следует идее Джо Крауса о том, что
проводя бета-тестирование продукта, следует тестировать и
бизнес-модель. Он уверен, что люди должны платить вам с самого начала.
Я считаю, что это накладывает слишком большие ограничения. Facebook
достиг успеха и без этого. Они не стремились заработать сразу. Такое
желание могло бы быть для компании губительным. Тем не менее, идея Джо
может пригодиться многим стартапам. Когда я вижу, что учредитель в
растерянности, я иногда советую ему сделать что-то, за что покупатель
смог бы заплатить. Я рассчитываю, что такое условие подтолкнет их к
действию.

Разница между идеей Джо и доширак-прибыльностью в том, что компания,
считающаяся доширак-прибыльной, не обязана сразу зарабатывать деньги
тем же способом, каким будет зарабатывать их в итоге. Она просто
должна зарабатывать. Самый известный из таких стартапов – Google,
который первые деньги заработал на продаже таким сайтам как Yahoo
лицензий на свои поисковые технологии.

Есть ли недостатки у доширак-прибыльности? Вероятно самая большая
опасность в том, что вы можете стать консалтинговой фирмой. Стартапы
должны быть производственными компаниями, то есть создавать единый
продукт для всеобщего пользования. Отличительная черта стартапов в
том, что они очень быстро растут, а консалтинговой компании, в отличие
от производственной, не под силу масштабный рост. [3] С другой
стороны, для консалтинговой фирмы не составит труда зарабатывать 3000
долл. США в месяц. В сущности, это низкая ставка для программирования
по контракту. Таким образом, у учредителей может появиться желание
перейти в консалтинг и убеждать себя в том, что у них
доширак-прибыльный стартап, когда, по сути, он вообще не является
стартапом.

Возможно, в первое время учредителям придется заниматься консалтингом.
Поначалу всем стартапам приходится делать довольно странные вещи. Но
помните, что доширак-прибыльность — это не ваша цель. Цель стартапа —
вырасти как можно больше, а доширак-прибыльность — это хитрость,
позволяющая не погибнуть на пути к этой цели.

Примечания

[1] «Доширак» из выражения «доширак-прибыльный» означает доступную и
дешевую лапшу быстрого приготовления.

Пожалуйста, не рассматривайте это выражение буквально. Питаться лапшой
доширак каждый день вредно для здоровья. Куда более полезны рис и
фасоль. Советую начать с вложения денег в рисоварку.

Рис с фасолью (2 порции)

масло оливковое или сливочное лук репчатый овощи свежие по вкусу 3
зубца чеснока 330 г фасоли консервированной белой, черной или
обыкновенной 1 кубик бульона Knorr говяжьего или овощного перец
свежемолотый тмин молотый рис, предпочтительно коричневый


Положите рис в рисоварку. Добавьте воды как указано на упаковке (или 2
стакана воды на стакан риса). Включите рисоварку и забудьте о ней.

Порежьте лук и другие овощи и пожарьте на масле на слабом огне, пока
лук не станет золотистым. Добавьте порезанный чеснок, перец, тмин, еще
немного масла и помешайте. Продолжайте жарить на слабом огне еще 2-3
минуты, затем добавьте фасоль (не сливайте сок) и помешайте. Добавьте
бульонный кубик, накройте крышкой и готовьте на слабом огне еще не
менее 10 минут. Периодически помешивайте, чтобы ингредиенты не
склеились.

Экономьте, покупая фасоль в огромных банках по сниженным ценам. Специи
тоже стоят дешевле, если покупать их пачками. А в индийской лавке
можно купить мешок тмина по цене баночки из супермаркета.

[2] Если влияние перейдет от инвесторов к учредителям, возможен рост
венчурных предприятий. Я думаю, инвесторы допускают большую ошибку,
так грубо обращаясь с учредителями. Если бы их заставили прекратить
это, все венчурные предприятия смогли бы работать лучше, и мы бы стали
свидетелями роста торговли, который обычно возникает при отмене
ограничительных законов.

Инвесторы – самая сильная головная боль для учредителей. Если бы они
не приносили столько проблем, быть учредителем было бы намного
приятнее. А если бы было приятно быть учредителем, многие бы ими
становились.

[3] Существует возможность роста стартапа из маленькой консалтинговой
фирмы в нечто более крупное. Но в этом случае он станет именно
производственной компанией.

\end{document}
