\documentclass[ebook,12pt,oneside,openany]{memoir}
\usepackage[utf8x]{inputenc} \usepackage[russian]{babel}
\usepackage[papersize={90mm,120mm}, margin=2mm]{geometry}
\sloppy
\usepackage{url} \title{Почему «парикмахерская» не может быть
  стартапом. Часть 1} \author{Пол Грэм} \date{}
\begin{document}
\maketitle

Стартап – компания, созданная для быстрого роста. Это означает, что не
все основанные «с нуля» компании могут называть себя стартапами. Не
имеет значения сфера работы, технологии, привлечение инвестиций,
продукт на выходе – рост является основой всех основ, и все, что
связано с понятием «стартап», так или иначе связано с ростом в первую
очередь. Пожалуй, это самая главная мысль, которую необходимо усвоить,
прежде, чем открывать стартап. Если будет рост, все остальное
обязательно встанет на свои места. Он должен быть вашим компасом и,
одновременно, определяющим фактором при принятии решений.

Для того, чтобы понять разницу между различными новыми бизнесами,
рассмотрим главное отличие, которое часто упускается из виду.
Ежедневно по всему миру открываются миллионы новых компаний.
Большинство из них представляют сектор услуг – рестораны,
парикмахерские и т.д. – и не является стартапами (исключения, как мы
знаем, бывают везде, но все же): парикмахерская не создана для
ускоренного роста, в то время, как поисковые системы только ради этого
и существуют. У стартапов попросту другое предназначение, нежели чем у
обычного малого бизнеса. Чтобы прочувствовать разницу, подумайте о
предназначении саженцев красного дерева в сравнении с ростками бобов?!

Если бы все компании имели схожую судьбу, а их успешность зависела от
затраченных усилий и везения основателя, мы бы не нуждались в
отдельной категории для «особенных компаний». Мы могли бы просто
говорить о более успешных и менее успешных бизнесах, однако же
стартапы отличаются от подобных компаний в своей базовой структуре,
если хотите – «ДНК». Google – это не какая-нибудь «парикмахерская»,
владелец которой был более трудолюбив, чем все остальные. Google
изначально был другим…

Для быстрого роста стартап должен выдавать продукт, который
обязательно будет востребован на мировом рынке. Это еще одно отличие
Google от стандартной «парикмахерской». Быстрый рост означает а)
производство того, чего желают очень многие, б) возможность обслужить
всех клиентов, вне зависимости от расстояния, времени и т.д. На стадии
«а» успешно функционируют все «парикмахерские» (каждому из нас время
от времени требуется стрижка). Проблема же кроется в другом: бизнес
«парикмахерской», как правило, «привязан» к местности, а значит люди
из отдаленных райнов, скорее всего, не поедут за новой прической
далеко.

*Чтобы не вводить в заблуждение, поясним: речь идет не о большом
количестве пользователей услуги, а о большом рынке (количество
покупателей, умноженное на их покупательскую способность). Самое
главное здесь – не попасться в ловушку: высокая покупательская
способность ограниченно малого количества «покупателей» превратит вас
из стартапа в консультационную фирму.

В то же время написание программного софта, без труда справляющегося с
условием «б», может столкнуться с проблемой «а»: даже открытый доступ
к программе по изучению Тибетского языка в Венгрии может
нивелироваться малым спросом на такой продукт.

Большие компании жестко ограничены условиями «а» и «б». Сильные и
перспективные стартапы – нет.

Ограничения, существующие у традиционного бизнеса, как бы это ни было
парадоксально, охраняют его. Возвращаясь к примеру с парикмахерской:
открывая ее, вы, скорее всего, будете конкурировать лишь с
парикмахерскими в вашем округе. Поисковая система же будет
конкурировать со всем миром!

Секрет успеха прост: затевая стартап, думайте о чем-то, ранее не
существовавшем, потому что, выходя на глобальный рынок, вы должны
реализовывать ту идею, которая никому в голову не пришла раньше вас.
Иногда кажется, что для поиска такой идеи нужна осознанная, долгая,
целенаправленная работа мозга, однако, как показывает практика, на
самом деле все происходит совсем не так: основатели успешным стартапов
просто-напросто «другие» люди с «другим» образом мышления, поэтому в
тех вещах, которые обычному человеку кажутся очевидными, стартапер
находит что-то свое, то, что все «проморгали» — the next big thing.
Забавным является тот факт, что судьбоносные решения, как правило,
принимаются руководителями стартапов подсознательно.

Еще одно отличие успешных основателей стартапов заключается в том, что
они способны ответить на очень разные по своей сути вопросы: эдакая
комбинация технологической грамотности и навыка с возможностью вовремя
и уместно применить эту грамотность для решения широкого спектра
задач. Вчерашняя «плохая» идея сегодня может оказаться гениальной
просто потому, что технологический мир меняется слишком стремительно,
а с ним меняются и потребности и возникают новые проблемы. Например:
молодой Стивен Возняк хотел персональный компьютер, что для 1975 года
было очень неожиданным желанием. На помощь ему пришла технологическая
грамотность, потому что Стив, не только хотел ПК, он еще и знал, КАК
его сделать. Решенная им задача в итоге вылилась в продукт, который
Apple сегодня продает миллионам людей на планете. Плюс ко всему, Стив
опередил своих современников: к тому моменту, как многие осознали,
насколько привлекателен рынок компьютеров, компания Apple уже во всю
оперировала на нем, захватит значительную долю.

Та же история случилась и с Google: Лари Пэйжд и Сергей Брин хотели
исследовать Интернет. В отличие от большинства специалистов, у них
хватало компетенции и знаний для того, чтобы оценить все
несовершенство существующих «поисковиков», а также возможности для их
улучшения. За следующие несколько лет проблема Пэйджа и Грина стала
всеобщей, и уже каждый, даже не самый требовательный пользователь,
невооруженным взглядом видел слабые места поисковых систем. Однако,
Google в этом вопросе уже слишком оторвался от только-только
появлявшихся конкурентов.

На этих примерах можно проследить одну из взаимосвязей между идеей для
стартапа и существующими технологиями: резкие изменения в определенной
области с одной стороны обнажают проблемы, а с другой, дают широкие
возможности для их решения! Отсюда же вытекает и другая связь: в
процессе решения новых проблем, стартапы создают новые технологии, что
замыкает круг постоянного развития.

Насколько стремительно должна развиваться компания для того, чтобы
быть стартапом? На этот вопрос не существует однозначного ответа.
Скорее его стоит переформулировать таким образом: «Насколько
стремительно развивается УСПЕШНЫЙ стартап». Для основателя бизнес это,
скорее, философский вопрос, равнозначный вопросу «На правильном ли я
пути?»

По мнению Пола Грэма, развитие успешного стартапа, как правило, имеет
3 стадии: Стадия зародыша и медленного роста, когда основатель
пытается понять, в каком направлении двигаться Стадия стремительного
роста, когда основатель, наконец, находит и создает продукт,
востребованный многими людьми Стадия перехода от успешного стартапа к
крупной компании, когда развитие компании замедляется из-за
ограничений рынка


Если изображать эти стадии в виде графика, то угол наклона кривой –
это как раз-таки темп роста. Это основной показатель стартапа. Не
просчитав темп роста своей компании, вы никогда не поймете, на
правильном ли вы пути.

Когда общаешься со стартапами и задаешь им вопрос о темпах роста, они
как правило отвечают что-то типа «сто новых клиентов в неделю». Это
неправильно, потому что рост не измеряется абсолютным числом, а
является пропорцией роста числа новых клиентов к уже существующим!
Если вы имеете стабильное абсолютное одинаковое число новых клиентов
от месяца к месяцу, время бить тревогу – вы не развиваетесь!

Y Combinator предлагает своим стартапам измерять рост в недельном
цикле: во-первых, потому что у них есть не так много времени до Demo
Day, а, во-вторых, стартапам нужен как можно боле частый фидбэк, чтобы
иметь возможность корректировать курс развития. Если говорить о
цифрах, то 5-7\% роста – это хороший результат. Если процент
достигается отметки в 10, то у вас впечатляющий прогресс, если рост
составляет всего 1\% — скорее всего, вы просто пока не определились,
куда вы движетесь. Как правило, лучший эквивалент показателю роста –
объем выручки. На втором месте – число активных
пользователей/клиентов.

Обычно инкубаторы и акселераторы предлагают стартапам проанализировать
свои возможности и выбрать тот темп роста, который они смогут
стабильно поддерживать каждую неделю. Ключевое слово здесь –
«стабильно».

Фокусирование внимания на достижении поставленных целей по темпам
роста за неделю, в целом, положительно влияет на развитие стартапа и
сужает круг проблем от многогранных и сложных к одной-единой. Вы
делаете выбор, исходя только из достижения показателей, все, что дает
вам рост, является искомым решением проблемы. Нужно ли Вам ехать на
конференцию? Стоит ли нанимать еще одного разработчика? Уделять ли
больше внимания маркетингу? Искать ли много мелких клиентов, или
сосредоточить все усилия на одном крупном? Делайте все, что приведет к
росту!

Недельная оценка роста совсем не означает, что вы не должны оценивать
более долгосрочную перспективу. Однажды, не достигнув поставленной
цели и испытав чувство разочарования, вы начинаете совершать действия
в расчете на будущий успех. Так, например, новый программист вряд ли
моментально поможет вам достичь краткосрочных успехов. Однако, он же,
возможно, станет залогом вашего успеха в будущем, через месяц, внедрив
новый продукт, оптимизировав программный код и т.д. Существует,
правда, возможность попасть в хитрую ловушку и, слишком сильно думая о
завтрашнем дне, упустить ситуация дня текущего.

Теоретически, такая политика «подъема в гору», когда вы
целенаправленно достигаете намеченных показателей, может привести к
итоговой стагнации, но на практике такое ни разу не случалось. В 10
случаях из 10, какое-либо действие всегда лучше бездействия: там, где
многие тратят дни на детализацию стратегии, успешные предприниматели
действуют наугад, на интуиции, и попадают в самое яблочко!

Немного о цифрах: нетрудно посчитать, что стартап, который растет на
1\% в неделю, вырастет в 1,7 раза за год, в том время как 5\% роста в
неделю дадут вам 12,6-кратный рост в год. В первом случае, зарабатывая
\$1000 в неделю, через 4 года вы будете зарабатывать \$7900. Во втором
начальный доход в ту же \$1000 приведет к гораздо более внушительным
\$25 миллионам в неделю через 4 года! Впечатляющие цифры, не правда
ли?

Темп роста, а также риски, играют определяющую роль в оценке
привлекательности стартапа. Вдумайтесь: при ожидаемом доходе в \$100
млн., однопроцентный шанс на успех все еще эквивалентен \$1 млн! А у
группы инициативных, образованных и способных предпринимателей шансы
на успех явно превышают 1\%. Становится ясным, почему сегодня многие
запускают собственные проекты (в то время, как собственники бизнеса из
другой категории сокрушаются «почему стартапы постоянно в центре
внимания?!»).

Обобщая вышесказанное: процесс постоянного следования намеченному
плану роста помогает идентифицировать, а иногда даже и модифицировать
стартап. По сути, это и является главным «ограничителем», в широком
смысле слова, который направляет стартап в правильное русло. Проводя
параллель с обычным бизнесом, можно привести пример, как
географическое положение определяет стиль и публику того или иного
бара или ресторана.

Как говорил Ричард Фейнман, один из крупнейших физиков ХХ столетия:
«Воображение природы всегда было богаче, чем воображением человека,
поэтому, следуя за истинным знанием, можно открыть вещи, гораздо более
удивительные в реальности, чем в любом из воображаемых миров». Темп
роста, в данном случае – это как раз то «истинное знание» для
успешного стартапа!

Прочные связи

Почему инвесторы так любят стартапы? Зачем они вкладывают свои кровные
деньги в инстаграмы, вместо крупного бизнеса с большими оборотами?
Причины различны.

Показателем инвестиционной привлекательности в данном случае выступают
риски. Стартапы заслуживают внимание инвесторов хотя бы потому, что,
будучи авантюрой, в случае успеха возвращают вложенные ресурсы в
многократном размере. Однако, это не единственная причина.

Второй аспект – это принцип работы инвесторов: они получают большие
премиальные в том случае, когда возвращают вложенные деньги. Как вы
понимаете, шанс пополнить свой баланс в случае успешного выхода
стартапа на IPO слишком соблазнителен. Обогащая себя, стартап
обогащает инвестора: все сделки довольно прозрачны, а значит легко
отслеживаемы – у основателя стартапа мало шансов скрыть свои доходы и
свою прибыль.

Теперь давайте разберемся, зачем стартапу нужны деньги инвестора?
Конечно, это, опять же, рост. Связь между ростом и хорошей идеей очень
сильна. Если у вас хорошая (необязательно масштабируемая) идея, но нет
стремительного роста, конкуренты оставят вас позади. Медленный рост –
самоубийство в стартап-бизнесе.

Любая компания нуждается в капитале для начала. Иногда, стартапы
привлекают инвестиции, даже не имея острой необходимости. Кому-то
продажа доли в своей компании по цене, заведомо меньшей, чем в будущем
(в случае успеха), может показаться глупой, однако такой шаг не менее
оправдан, чем страхование собственной жизни. Все очень просто:
дополнительные средства, полученные в результате привлечения
инвестиций, помогают ускорить рост.

На самом деле, инвесторы нуждаются в стартапах не меньше, а иногда
даже больше, чем стартапы в инвесторах: развиваясь на собственные
деньги, стартап хоть и ставит себя под угрозу и замедляет рост,
однако, все же, имеет шанс выжить. В свою очередь инвестор, не
вкладывая в потенциально успешные стартапы, просто выходит из бизнеса!
Данный факт приводит нас к следующему выводу: любой мало-мальски
успешный стартап может привлечь большие суммы от инвесторов на
заведомо выгодных условиях.

Помимо инвестиций, успешные стартапы часто получают предложения о
полном выкупе долей другой компанией. Почему? Что приводит к подобным
мыслям руководителей элитного эшелона крупного бизнеса?

И тут ответ находится на поверхности: доля в быстро растущей компании
является лакомым куском: когда eBay выкупил PayPal, было ясно, что
взамен на деньги компания получит контроль над сделками, а значит, и
над прибылью. Кроме того, немаловажной причиной покупки стартапов
является страх конкуренции. Стартап, расширяясь, постепенно начинается
претендовать на рыночные доли крупных компаний. Даже если компания не
боится самого стартапа и его продуктов, топ-менеджмент явно опасается
того, как конкуренты могут применить их технологические новинки, в
случае поглощения. Именно поэтому зачастую компании платят за выкуп
стартапов суммы даже большие их реальной стоимости.

Верить или не верить Полу Грэму — это ваше личное дело, но в мире
каждая его статья становится фактически стандартом, удачи вам с ростом
ваших стартапов!

\end{document}
