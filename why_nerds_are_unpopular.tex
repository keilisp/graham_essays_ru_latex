\documentclass[ebook,12pt,oneside,openany]{memoir}
\usepackage[utf8x]{inputenc} \usepackage[russian]{babel}
\usepackage[papersize={90mm,120mm}, margin=2mm]{geometry}
\sloppy
\usepackage{url} \title{За что не любят ботанов} \author{Пол Грэм}
\date{}
\begin{document}
\maketitle
Когда я учился в старших классах школы, мы с моим другом Ричем сделали
карту школьных обеденных столов, на которой ранжировали их по
популярности. Это было несложно сделать, так как дети обедали только с
теми детьми, у которых был тот же уровень популярности. Мы ранжировали
столы от A до E. За столами класса А сидели футболисты,
девушки-чирлидеры и так далее. Столами класса E довольствовались дети
с синдромом Дауна лёгкой степени, которых мы на тогдашнем жаргоне
называли тормозами.

Мы с другом сидели за столом класса D, ниже которого можно было
упасть, только имея физические недостатки. От нас не требовалось
особой объективности, чтобы классифицировать себя как «D». Мы не могли
отнести себя к другому классу – каждый школьник, включая нас, отлично
знал, кто и насколько популярен.

Мои акции потихоньку росли в процессе обучения в институте. Наконец-то
пришла половая зрелость; я стал неплохим игроком в футбол; я начал
издавать скандальную подпольную газету. Так что я увидел значительную
часть ландшафта популярности.

Я знаю много людей, которые были ботанами в школе, и они все
рассказывают одну историю: есть сильная корреляция между «быть
сообразительным» и «быть ботаном», и есть ещё более сильная обратная
корреляция между «быть ботаном» и «быть популярным». Похоже на то, что
сообразительность делает тебя непопулярным.

Почему так происходит? Для тех, кто учится сейчас в школе, этот вопрос
может выглядеть нелепым. Голые факты настолько убедительны, что может
показаться странным вообразить, что может быть по-другому. Однако же
может. Сообразительность не делает вас изгоем в начальной школе. Точно
так же сообразительность не приносит вам вреда в реальной жизни. Кроме
того, насколько я знаю, проблема не является настолько острой в других
странах. Но в типичной американской школе, жизнь умного
старшеклассника весьма тяжела. Почему?


Ключ к этой загадке можно найти, если слегка перефразировать вопрос.
Почему умные школьники не делают себя популярными? Если они такие
умные, почему они не могут понять, как устроена популярность, и
справиться с системой также просто, как они справляются с
контрольными?

Распространено мнение, что это невозможно, так как другие дети
завидуют своим умным одноклассникам. То есть, причина непопулярности
ботанов кроется в зависти одноклассников к их уму. Следовательно, они
никак не могут сделаться популярными. Хотел бы я чтобы было так! Если
другие старшеклассники и завидовали мне, они это тщательно скрывали.
И, в любом случае, если бы они завидовали моему уму, то я бы нравился
девочкам. Девочки любят парней, которым завидуют другие парни.

В школах, в которых я учился, ум не являлся чем-то важным. Дети не
восхищались им и не презирали его. При прочих равных, школьники
предпочли бы быть скорее с умными, чем с глупыми, но интеллект
котировался гораздо ниже чем, например, внешний вид, обаяние или
физическая сила.

Но если интеллект сам по себе не является значащим фактором в вопросе
популярности, почему же умные школьники так последовательно
непопулярны? Ответ, я думаю, заключается в том, что они на самом деле
не хотят становиться популярными.

Если бы кто-нибудь сказал мне это в то время, я бы рассмеялся ему в
лицо. Школьная непопулярность делает детей несчастными, и некоторые из
них настолько несчастны, что даже решаются на самоубийство. Сказать
мне тогда, что я не хочу быть популярным – это было бы как сказать
умирающему от жажды в пустыне, что он не хочет выпить стакан воды.
Разумеется, я хотел быть популярным.

Но на самом деле я не хотел этого, недостаточно хотел. Было кое-что
ещё, чего я хотел сильнее, чем быть популярным. Я хотел быть умным. Не
просто хорошо учиться в школе, хотя и это имело значение, а
проектировать красивые ракеты, или хорошо писать, или разбираться в
программировании. В общем, делать большие вещи.

За всё время учёбы я ни разу не пытался отделить мои желания друг от
друга, и взвесить – что для меня важнее. Если бы я так сделал, я бы
понял, что желание быть умным было для меня важнее всего. Если бы
кто-нибудь предложил мне сделку – стать самым популярным ребёнком в
школе, но за это получить средний ум (можете смеяться надо мной), я бы
не согласился.

Как бы ни страдали ботаны от своей непопулярности, я не думаю, что
многие из них согласились бы на такой обмен. Для ботана тяжело даже
думать о том, чтобы получить обыкновенные мозги. А вот большинство
других детей согласились бы. Для половины из них это был бы шаг
наверх. Даже для тех, кто находился в восьмидесятом процентиле
(предполагая, как принято делать, что ум – это скаляр), кто бы не
отдал тридцать пунктов в обмен на всеобщую любовь и восхищение?

И это, я думаю, корень проблемы. Ботаны гонятся за двумя кроликами.
Они, несомненно, хотят быть популярными, но еще сильнее они хотят быть
умными. А популярность – это не та вещь, которой ты можешь достичь, не
прикладывая усилий, особенно в атмосфере яростного соперничества
американской средней школы.


Альберти, которого можно считать архетипом Человека Ренессанса, писал,
что «любое искусство, каким бы оно ни было незначительным, требовало
абсолютного посвящения от того, кто хотел достичь в нём высот». Я
удивлюсь, если узнаю, что кто-то в мире добивается чего-то сильнее,
чем американские старшеклассники добиваются популярности. Морские
десантники и врачи-нейрохирурги просто лентяи по сравнению с ними. Они
изредка берут отпуска, кое-кто даже заводит хобби. Американский
тинэйджер добивается популярности каждый час бодрствования, 365 дней в
году.

Я не хочу сказать, что они делают это сознательно. Конечно, некоторые
из них – настоящие маленькие Макиавелли, но большинство искренне
пытается соответствовать обществу.

Например, подростки придают огромное значение одежде. Они не думают,
что одеваются для того, чтобы стать популярными. Они одеваются, чтобы
хорошо выглядеть. Но для кого? Для других детей. Мнение других детей
является определяющим, и не только в том, что касается одежды, но и
практически в любых вещах, вплоть до походки. И каждое усилие, которое
они предпринимают, чтобы поступать «правильно», это также, сознательно
или нет, и усилие, чтобы стать более популярным.

Ботаны этого не понимают. Они не понимают, что достижение популярности
требует усилий. В общем, люди, которые находятся за пределами этого
круга, не осознают степень, в которой успех зависит от постоянных
(пусть часто и неосознаваемых) усилий. Например, принято считать, что
способность рисовать – это своего рода врождённое свойство, вроде
высокого роста. На самом деле, большинство людей, которые «могут
рисовать», любят рисовать, и провели много часов, делая это; вот
почему они хорошо рисуют. Так и популярность – это не просто качество,
которым ты обладаешь или не обладаешь, а нечто, чего ты самостоятельно
добиваешься упорным трудом.

Главная причина, по которой ботаны непопулярны, это то, что у них есть
другие интересы. Им интересны книги и природа, а не мода и вечеринки.
Они похожи на человека, который пытается играть в футбол, держа на
голове наполненный стакан с водой. Другие игроки, которые могут
сосредоточить своё внимание на футболе, легко у них выигрывают, и
удивляются, почему ботаны такие неумелые.

Даже если бы ботаны старались стать популярными так же сильно, как и
другие дети, это требовало бы от них больше усилий, чем от других.
Популярные дети научились быть популярными, и хотеть быть популярными,
также как ботаны научились быть умными и хотеть быть умными: от своих
родителей. Пока ботаны тренировались находить верные ответы,
популярные дети тренировались быть приятными.


До этого момента я уклонялся от обсуждения разницы между умными и
ботанами, используя эти слова как синонимы. На самом деле, эти слова
взаимозаменяемы только в школьном контексте. Ботан – это некто, кто
недостаточно социально адаптирован. Но «недостаточно» зависит от того,
где ты находишься. В типичной американской школе стандарты «крутости»
настолько высоки (или, по крайней мере, настолько своеобразны), что
тебе не нужно быть особенно неуклюжим, чтобы выглядеть неуклюжим в
сравнении с другими.

Мало умных детей могут позволить себе не уделять достаточно внимания
работе над популярностью. Если им не повезло быть хорошо физически
развитыми от природы, либо иметь популярных братьев или сестёр, они
имеют склонность к тому, чтобы стать ботанами. И в этом заключается
причина, по которой жизнь умных людей наиболее тяжела между, скажем,
одиннадцатью и шестнадцатью годами. Жизнь в этом возрасте зависит от
популярности гораздо сильнее, чем до или после него.

До этого возраста главную роль в жизни детей играют их родители, а не
другие дети. Конечно, детей волнует то, что думают их сверстники в
начальной школе, но это не является для них настолько безумно важным,
как в старших классах.

Около одиннадцати дети начинают воспринимать свою семью, как дневную
работу. Они создают новый мир вокруг себя, и их волнует положение в
этом мире, а не положение в своей семье. На самом деле, неприятности в
семье могут даже добавить им очков в том мире, которых их
действительно волнует.

Проблема заключается в том, что мир, который создают вокруг себя эти
дети, очень жесток. Если вы предоставите стае одинадцатилетних детей
полную свободу, в итоге вы получите Повелителя Мух. Как и большинство
американских детей, я читал эту книгу в школе. По-видимому, это не
было случайностью. По-видимому, кто-то хотел указать нам, что мы были
дикарями, и что мы создали сами себе жестокий и глупый мир. Но это
было слишком сложным для меня. Хотя книга и выглядела совершенно
правдоподобной, я не понял подтекста. Я бы хотел, чтобы мне сказали
прямым текстом, что мы были дикарями, а наш мир был глупым.


Ботаны находят свою непопулярность терпимой, если она всего лишь
приводит к их игнорированию. К сожалению, быть непопулярным в школе –
это значит быть активно преследуемым.

Почему? Ещё раз, любой школьник найдёт этот вопрос нелепым. Как же
может быть иначе? Однако может. Взрослые, как правило, не мучают
ботанов. Почему же так поступают подростки?

Возможно по той причине, что подростки – ещё наполовину дети, а многие
дети по своей сути жестоки. Некоторые пытают ботанов по той же
причине, по которой отрывают ноги у пауков. Перед тем, как у тебя
появится совесть, пытки – весьма занимательное дело.

Другая причина, по которой дети преследуют ботанов – это чтобы
чувствовать себя лучше. Когда ты находишься в воде, ты можешь
приподняться, опустив воду вниз. Так и в социальной иерархии, люди,
которые не уверены в своей позиции, попытаются укрепить её, жестоко
обращаясь с теми, кого они полагают стоящими ниже. Я читал, что это –
причина, по которой бедные белые в США – это наиболее враждебная
чёрным группа.

Но я думаю, что главная причина, по которой другие дети преследуют
ботанов, заключается в том, что это часть механизма популярности.
Популярность только частично зависит от личной привлекательности.
Гораздо больше она зависит от альянсов. Чтобы быть популярным, тебе
постоянно нужно совершать поступки, которые подвинут тебя ближе к
другим популярным людям, а ничего не сплачивает людей лучше, чем общий
враг.

Как политик, который хочет отвлечь внимание избирателей от плохих
новостей дома, вы можете создать врага, даже если его реально нет в
наличии. Отделяя от стада ботана и преследуя его, группа детей,
которая стоит выше в иерархии, создаёт связи друг с другом. Атака
чужого делает их всех своими. Это причина, по которой самые страшные
издевательства – дело рук группы. Спросите любого ботана: гораздо хуже
когда тебя мучает стайка подростков, чем когда тебя мучает один
подросток, даже если он и садист.

Если это может служить утешением для ботанов, в этом нет ничего
личного. Группа детей, которая собралась вместе, чтобы поиздеваться
над вами, делает то же самое, и по той же причине, что и группа людей,
которая собралась, чтобы поохотиться. На самом деле они вовсе не
ненавидят вас. Просто им нужно кого-то преследовать.

Так как ботаны находятся внизу школьной иерархии, они являются
безопасной мишенью для всей школы. Если я правильно помню, самые
популярные ребята не мучают ботанов; им не нужно опускаться до этого.
Большая часть мучений идёт от детей, которые стоят ниже, от нервных
середнячков.

Проблема в том, что этих середнячков много. Популярность
распределяется не как пирамида, а как два конуса, соединённых
основаниями. Группа с наименьшей популярностью – довольно невелика. (Я
полагаю, что мы были единственным столом D на карте нашей столовой).
Так что людей, которые хотят поиздеваться над ботанами, больше, чем
самих ботанов.

Отмежевание от непопулярных детей приносит очки, приближение к
непопулярным детям убавляет очки. Одна моя знакомая сказала, что в
старших классах ей нравились ботаны, но она боялась заговаривать с
ними, так как другие девочки подняли бы её на смех. Непопулярность –
это заразная болезнь; дети, слишком хорошие для того, чтобы мучать
ботанов, всё равно будут их игнорировать для самозащиты.

Неудивительно, что умные ребята, как правило, несчастливы в старших
классах. Другие интересы оставляют им мало времени работать над
популярностью, а пока популярность похожа на игру с нулевой суммой,
это делает их мишенью для всей школы. Забавно, что кошмарный сценарий
воспроизводится безо всякого сознательного злого умысла, а просто
логически следует из ситуации.


Для меня худшим участком были старшие классы, когда культура
подростков была новой и жестокой, а специализация, которая позже
постепенно отделит умных, только началась. Почти каждый, с кем я
говорил, соглашался – надир, самая низкая точка, где-то между
одиннадцатью и четырнадцатью.

В нашей школе это был восьмой класс, мне тогда было двенадцать и
тринадцать лет. В тот год была небольшая сенсация, когда одна из
учительниц подслушала стайку школьниц, которые ждали школьного
автобуса, и была шокирована услышанным. На следующий день учительница
красноречиво убеждала весь класс не быть такими жестокими друг к
другу.

Это не дало никакого заметного эффекта. Но что меня поразило, так это
то, что учительница была удивлена. Вы хотите сказать, что она не
знала, какие вещи дети говорят друг другу? Вы хотите сказать, что это
не является чем-то нормальным?

Важно понимать, что взрослые не знают, как дети обращаются друг с
другом. Теоретически взрослые знают, что дети очень жестоки по
отношению друг к другу, также как мы теоретически знаем, что в странах
третьего мира пытают людей. Но взрослые не любят думать об этом
неприятном факте, и взрослые не видят доказательств жестокого
обращения, пока не начинают их целенаправленно искать.

Учителя государственных общеобразовательных школ занимают примерно ту
же позицию, что и тюремные надсмотрщики. Главная задача надсмотрщиков
– это удержать заключённых на их местах. Им также нужно их кормить, и,
по-возможности, следить, чтобы заключённые не убивали друг друга.
Надсмотрщики могут сделать немногое сверх этого, поэтому они вынуждены
разрешить заключённым создать ту социальную организацию, которую те
хотят. Из того, что я читал, я понял, что общество, которое создают
заключённые, извращённое, дикое и всюду проникающее, и мало радости
быть на его дне.

В общих чертах, именно это и было в школах, в которые я ходил.
Наиболее важной вещью была посещаемость. Пока вы находились в школе,
администрация кормила вас, предотвращала публичные проявления
жестокости и делала попытки вас чему-нибудь научить. Но администрация
могла сделать не слишком много сверх этого. Как и тюремные
надзиратели, школьные учителя по большей части предоставляли нас самим
себе. И, как и заключённые, мы создали себе варварскую культуру.


Почему реальный мир более гостеприимен к ботанам? Может показаться,
что ответ прост, и заключается в том, что он населён
совершеннолетними, которые слишком взрослые, чтобы издеваться друг над
другом. Но я не думаю, что причина в этом. Взрослые в тюрьме постоянно
мучают друг друга. И точно также, очевидно, поступают богатые
домохозяйки – в некоторых районах Манхеттена жизнь для женщины
является продолжением школы, со всеми её мелочными интригами.

Я думаю, что важное качество настоящего мира заключается не в том, что
он населён взрослыми, а в том, что он очень большой, и в том, что
вещи, которые ты делаешь, приносят настоящие результаты.

Это то, чего не хватает школе, тюрьме и домохозяйкам. Обитатели всех
этих миров заперты в своих маленьких пузырях, где всё, что бы они ни
делали, имеет не более чем локальный эффект. Естественно, эти
сообщества деградируют в дикость. Им больше ничего не остаётся делать.

Когда поступки, которые ты совершаешь, приносят реальный результат,
далеко не достаточно быть просто приятным. Становится важным находить
верные ответы, и ботаны имеют тут значительное преимущество. Сразу же
приходит на ум Билл Гейтс. Несмотря на то, что он знаменит своим
неумением вести себя в обществе, он сумел найти верные ответы, по
крайней мере, если измерять их в деньгах.

Другая вещь, которая отличает реальный мир, заключается в том, что он
гораздо больше. В достаточно большом озере даже самые маленькие
меньшинства могут набрать критическую массу, если они соберутся
вместе. В реальном мире ботаны собираются в определённых местах и
формируют их собственное общество, где интеллект является самой важной
вещью. Иногда поток даже меняет направление: иногда, в университетах,
ботаны сознательно подчёркивают свою неуклюжесть, чтобы выглядеть
умнее. Джон Нэш так восхищался Норбертом Винером, что перенял его
привычку касаться рукой стены во время движения по коридору.


Как тринадцатилетний подросток, я имел опыт общения только с тем
миром, который меня окружал. Я думал, что извращённый маленький мирок,
в котором мы жили, это и есть мир. Мир казался жестоким и скучным, и я
не был уверен, что из этого хуже.

Так как я не соответствовал этому миру, я думал, что какая-то проблема
во мне. Я не понимал, что ботаны не соответствуют этому миру по той
причине, что они находятся на шаг впереди. Мы уже думали о вещах,
которые имеют значение в реальном мире, вместо того, чтобы проводить
наше время, играя в обременительную и бессмысленную игру, как другие.

Мы были немного похожи на взрослого, которого внезапно засунули
обратно в старшие классы школы. Он не знал бы, какую одежду надо
носить, какую музыку слушать и какой жаргон использовать. Он бы
выглядел среди детей как полный чужак. Но этот взрослый знал бы
достаточно, чтобы не заботиться о том, что думают другие дети. У нас
же не было такой самоуверенности.

Много людей, похоже, думают, что для умных детей полезно побыть среди
«нормальных» в течение этого периода их жизни. Возможно. Но есть как
минимум несколько случаев, когда причина несоответствия ботанов
детскому мирку заключается в том, что остальные ведут себя как
сумасшедшие. Я помню, как сидел среди публики во время выступления
девочек-чирлидеров, наблюдая как девочки бросают изображения
команды-противника на трибуны, чтобы там их порвали в клочья. Я
чувствовал себя как естествоиспытатель во время эксцентричного ритуала
какого-то дикого племени.


Если бы я мог вернуться в прошлое, и дать несколько советов себе
тринадцатилетнему, главное, чтобы я ему сказал – это поднять голову и
оглядеться. В то время я не осознавал, что весь мир, в котором мы
жили, был так же фальшив, как китаец, переодевшийся белым. Не просто
школа, но и всё предместье. Зачем люди едут в предместья? Чтобы
заводить детей! Так что неудивительно, что всё вокруг казалось скучным
и стерильным. Целый город был гигантским детским садом – искусственный
город, созданный для разведения детей.

Там где я вырос, у меня было чувство, что нечего делать и некуда идти.
Ничего не случалось. Предместья сознательно проектируют так, чтобы
свести к минимуму влияние внешнего мира, так как внешний мир содержит
вещи, которые могут подвергнуть опасности детей.

Ну а школы представляли собой небольшие загоны внутри этого фальшивого
мира. Официальное предназначение школ – учить детей. На самом деле, их
основное предназначение – это держать детей запертыми в одном месте в
течение значительной части дня, чтобы взрослые могли заниматься своими
делами. И я это понимаю – в современном индустриальном обществе дети,
бегающие по офисам, были бы стихийным бедствием.

Что раздражает меня на самом деле – это не то, что детей держат в
тюрьмах, а то, что (а) детям не говорят об этом, и (б) в этих тюрьмах,
как правило, устанавливают порядки сами заключённые. Детей посылают на
шесть лет запоминать бессмысленные факты в мир, которым правит каста
гигантов, бегающая за продолговатым коричневым мячом, как если бы это
было самой естественной вещью в мире. А если дети пытаются уклониться
от этого сюрреалистичного коктейля, их называют плохо
приспособленными.


Жизнь в этом искривлённом мире полна стресса для детей. И не только
для ботанов. Как и на любой войне, здесь получают раны даже
победители.

Взрослые не могут не понимать, что детей мучают в школах. Почему же
они ничего с этим не делают? Потому что они возлагают вину на
переходный возраст. Причина, по которой дети так несчастны, говорят
себе взрослые, заключается в том, что ужасные новые химикалии,
гормоны, бурлят в крови подростков и всё портят. С самой системой всё
в порядке – подростки неминуемо будут несчастными в этом возрасте.

Эта идея настолько распространена, что даже дети верят в неё, что,
надо полагать, не особо им помогает. Тот, кто думает, что у него
больные ноги, не будет останавливаться, чтобы проверить – а не надел
ли он ботинки слишком маленького размера.

У меня не вызывает доверия теория, что с тринадцатилетними подростками
действительно что-то не в порядке. Если причина в физиологии – так
должно было быть всегда и везде. Но являлись ли все монгольские
кочевники нигилистами в 13 лет? Я прочёл немало книг по истории, но я
нигде не видел ссылки на этот «универсальный» факт до двадцатого
столетия. Подростки-подмастерья в эпоху Ренессанса выглядят радостными
и энергичными. Разумеется, они подшучивают друг над другом и дерутся
(Микеланджело сломал себе нос в драке), но они не выглядят
сумасшедшими.

Насколько я могу судить, концепция гормонального безумия подростков –
ровесник предместий. Я не думаю, что это – совпадение. Я думаю, что
тинейджеров доводит до безумия жизнь, которую их заставляют вести.
Подростки-подмастерья в эпоху Ренессанса были служебными собаками.
Современные тинейджеры – невротичные болонки. Их безумие – это безумие
полной бессмысленности.


Когда я учился в школе, самоубийство было постоянной темой разговоров
среди умных детей. Никто из тех, кого я знал, не покончил жизнь
самоубийством, но некоторые собирались, а кое-кто, возможно, и
попытался. Как и другие подростки, мы любили всё драматичное, а суицид
казался очень драматичным. Но главной причиной было то, что наша жизнь
в то время была действительно жалкой.

Издевательство сверстников было только частью проблемы. Другая
проблема, и, возможно, даже ещё худшая, заключалась в том, что мы не
могли делать ничего настоящего. Люди любят работать, и в большей части
мира твоя работа – это твоя отличительная черта. А вся работа, которую
мы делали или была бессмысленной, или казалась таковой в то время.

В лучшем случае, это была практика перед настоящей работой, который мы
могли заняться в далёком будущем, хотя мы даже и не знали в то время,
в чём именно мы тренируемся. Но чаще это был просто случайный набор
обручей, через которые нужно было прыгнуть, слова без контекста,
придуманные специально для тестирования. (Тремя главными причинами
Гражданской Войны были… Тест: Перечислите три главных причины
Гражданской Войны).

И не было никакого способа выбрать. Взрослые договорились между собой,
что школа – это дорога в колледж. Единственным способом выбраться из
этой пустой жизни было подчиниться этому решению.


Раньше подростки играли гораздо более важную роль в обществе. В
доиндустриальные времена, они все были подмастерьями того или иного
сорта, в магазинах, или на фермах, или на военных судах. Их не
собирали вместе, чтобы они создавали собственные сообщества. Они были
младшими членами взрослых сообществ.

Тинейджеры сильнее уважали взрослых, так как взрослые были экспертами
в умениях, которыми подростки пытались овладеть. Сейчас большинство
детей слабо представляют, чем занимаются их родители в своих офисах, и
не видят связи (связь действительно слабая) между учёбой в школе и
работой, которую они будут выполнять, когда станут взрослыми.

И, так как подростки больше уважали взрослых, взрослые, в свою
очередь, могли найти применение подросткам. После тренировки в течение
нескольких лет, подмастерье становился реальным помощником. Даже самый
юный подмастерье мог носить письма или чистить мастерскую.

Сейчас взрослые не могут найти применения подросткам. Подростки были
бы некстати в офисе или на заводе. Так что взрослые сдают подростков в
школу по пути на работу, примерно так же, как собаку сажают на цепь,
уезжая на выходные.

Почему так произошло? Причина та же, что и у многих других проблем –
специализация. По мере того, как работа становилась всё более
специализированной, нам приходилось всё дольше и дольше к ней
готовиться. Дети в доиндустриальные времена начинали работать самое
позднее в 14 лет; дети на фермах, где жила основная масса людей,
начинали гораздо раньше. Сейчас дети, которые идут в колледж, не
начинают работать на полный рабочий день до 21 или 22. Вы можете
видеть, как кандидаты наук учатся до тридцати лет.

Труд подростков сейчас бесполезен, если не считать дешёвого труда в
отраслях типа фастфуда, который, кстати, эксплуатирует этот факт.
Практически в любой другой работе подростки будут только мешать. Но
они также и слишком юны, чтобы оставлять их без присмотра. Кто-то
должен присматривать за ними, и проще всего сделать это, собрав их в
одно место. Тогда несколько взрослых смогут присматривать сразу за
всеми.

Если остановиться на этом, мы получим натуральную тюрьму, хотя и не на
полный день. Проблема в том, что множество школ на этом и
останавливается. Изначальное предназначение школ – обучение детей. Но
у школ нет механизма обратной связи, чтобы делать это хорошо. И
большинство школ учат детей настолько плохо, что даже умные дети не
воспринимают учёбу серьёзно. Большую часть времени и учителя и ученики
просто имитируют полезную деятельность.

В моей школе на уроках французского предполагалось, что мы будем
читать «Отверженных» Гюго. Я не думаю, чтобы хоть один из нас знал
французский достаточно хорошо, чтобы продраться сквозь эту гигантскую
книгу. Как и другие мои одноклассники, я просто ознакомился с кратким
содержанием «Отверженных» в Cliff's Notes – специальной книжке для
школьников, где кратко пересказывается содержание книг, входящих в
школьную программу. Когда нам дали тест на «Отверженных», я заметил,
что вопросы звучат странно. Они были полны длинных слов, которые наш
учитель никогда не использовал. Откуда пришли эти вопросы? Как
выяснилось, из тех же Cliff's Notes. Учитель тоже не читал оригинала.
Мы все только делали вид, что учим французский.

Разумеется, в государственных школах иногда встречаются великолепные
учителя. Энергия и воображение моего учителя в четвёртом классе,
мистера Михалко, до сих пор, спустя тридцать лет, заставляют его
учеников вспоминать этот год. Но такие учителя как он – это единичные
энтузиасты, плывущие против течения. Они не могут исправить систему.


Почти в любой группе людей Вы найдёте иерархию. Когда группа взрослых
собирается в реальном мире, она, как правило, собирается для каких-то
целей, и лидерами становятся те, кто лучше всех этим целям
соответствуют. Проблема в том, что у большинства школ нет никакой
цели. Но иерархия быть должна. И дети создают себе иерархию из ничего.

У нас есть выражение, чтобы описать, что случается, когда ранжирование
создаётся без какого-либо значащего критерия. В таких случаях мы
говорим, что ситуация вырождается в состязание за популярность. Именно
это и происходит в большинстве американских школ. Вместо того, чтобы
зависеть от какого-нибудь реального критерия, ранг особи зависит, в
основном, только от способности этой особи увеличивать свой ранг.
Примерно как при дворе Людовика XIV. Внешнего врага нет, так что дети
становятся врагами друг другу.

Когда же есть какой-нибудь внешний критерий умения, нахождение внизу
иерархии не является болезненным. Новичок в футбольной команде не
возмущается мастерством ветерана; он надеется однажды стать таким же
как он, и рад шансу поучиться у него. Ветеран может, в свою очередь,
испытывать чувство noblesse oblige (положение обязывает). И, самое
главное, статус ветерана зависит от того, насколько он хорош в
сражении с оппонентами, а не от того, насколько хорошо он умеет
задвигать вниз своих товарищей.

Дворовая иерархия устроена совершенно по-другому. Этот тип общества
унижает любого, кто входит в него. Здесь нет ни восхищения снизу, ни
noblesse oblige на вершине. Убивай или будь убитым.

Это тот сорт общества, который создаётся в старших классах
американских школ. И это происходит, так как у этих школ нет реального
предназначения, кроме как содержать детей в одном месте на протяжении
определённого числа часов ежедневно. Чего я не понимал тогда, и что я
понял совсем недавно, так это то, что оба близнеца-кошмара школьной
жизни, жестокость и скука, имеют одну и ту же причину.


Бездарность американских государственных средних школ влечёт за собой
худшие последствия, чем просто превращение шести школьных лет в
несчастные годы. Эта бездарность порождает непокорность, которая
отвращает школьников от предметов, которые они, как предполагается,
должны учить.

Как, вероятно, и многим другим ботанам мне пришлось прожить много лет
после окончания института, прежде чем я сумел заставить себя прочесть
что-нибудь из школьной программы. И я потерял больше, чем книги. Я
потерял веру в такие слова как «личность» и «честность», так как я
подвергся в школе обману со стороны взрослых. Взрослые использовали
эти слова в единственном смысле – покорность. Дети, которых хвалили за
эти качества, были в лучшем случае упорными тупицами, а в худшем –
поверхностными болтунами. Если это называлось «личностью» и
«честностью» – я не хотел иметь с этими понятиями ничего общего.

Словом, которое вызывало во мне больше всего непонимания, было слово
«такт». Взрослые использовали его как аналог слов «держи свой рот
закрытым». Я полагал, что это слово означает что-то вроде «немой» или
«молчаливый». Я поклялся себе, что никогда не буду тактичным; им не
удастся меня заткнуть. На самом деле, слово такт имеет тот же корень,
что и слово «тактильный», и его значение – «делать ловкое
прикосновение». Тактичный – это противоположный застенчивому. Я не
думаю, что я понимал это до того времени, как пошёл в колледж.


Ботаны не единственные неудачники в этих крысиных бегах на
популярность. Ботаны непопулярны, так как они застенчивы. Но есть и
другие дети, которые сознательно отказываются от участия в этих бегах,
так как сам процесс вызывает у них отвращение.

Подростки, даже бунтари, не любят быть одни, поэтому когда дети
выбирают выход из системы, они стараются сделать это в группе. В
школах, в которые я ходил, бунт заключался в использовании наркотиков,
особенно марихуаны. Дети из этих групп носили чёрные футболки с
изображениями популярных певцов, и назывались «фриками».

Фрики и ботаны были союзниками, и между ними было много общего. Фрики
были, как правило, умнее других детей, хотя демонстративное
пренебрежение учёбой было важной составляющей их культуры. Я был
скорее ботаном, но у меня было много друзей фриков.

Они использовали наркотики, чтобы создавать свои социальные связи, по
крайней мере, когда начинали их употреблять. Наркотики – это было то,
что можно делать вместе, и, так как наркотики были нелегальными, их
употребление было общим символом бунта.

Я не хочу сказать, что плохие школы – это единственная причина, по
которой подростки имеют проблемы с наркотиками. Через какое-то время
после начала употребления, наркотики вызывают тягу сами по себе. Нет
сомнений, что некоторые фрики использовали наркотики, чтобы уйти от
проблем, не связанных со школой – домашних проблем, например. Но, по
крайней мере, в моей школе, причиной, по которой большинство детей
начинали использовать наркотики, был бунт. Четырнадцатилетние не
начинают курить травку в надежде, что она поможет им забыть их
проблемы. Они начинают курить травку чтобы присоединиться к другой
компании.

Плохое руководство порождает бунт; это не новая идея. Но власти ведут
себя так, как будто наркотики являются причиной всех проблем сами по
себе.


Настоящая проблема – это пустота школьной жизни. Мы не найдём решения,
пока взрослые не осознают этого. Взрослые, которые могут понять это
первыми – это те, кто сами были ботанами в школах. Вы хотите, чтобы
ваши дети были также несчастны в восьмом классе как вы? Я не хочу. Но
можем ли мы что-нибудь сделать, чтобы исправить положение? Почти
наверняка. В текущей системе нет ничего, что нельзя было бы поменять.
Эта система сложилась под влиянием случайных обстоятельств.

Взрослые, однако, занятые люди. Показать информацию о школьных «играх»
– это одно. Бороться с образовательной бюрократией – совсем другое.
Возможно, совсем немногие обладают достаточной энергией, чтобы
попытаться что-то изменить. Я полагаю, что самое сложное – это
осознать, что ты можешь это сделать.

Ботанам в школах по-прежнему тяжело дышать. Может быть однажды армия
тяжеловооружённых взрослых прилетит на вертолётах, чтобы спасти тебя,
но они, надо полагать, не прилетят в этом месяце. Если ботаны хотят
улучшить свою жизнь прямо сейчас, они должны заняться этим сами.

Простое понимание ситуации, в которой они находятся, должно сделать её
менее болезненной. Ботаны – не неудачники. Они просто играют в другую
игру, и эта игра гораздо ближе к той, в которую играют в большом мире.
Взрослые знают это. Сейчас тяжело найти успешных взрослых, которые не
заявляли бы, что в школе были ботанами.

Для ботанов также важно понять, что школа – это не жизнь. Школа – это
странная, искусственная вещь, наполовину стерильная и наполовину
дикая. Она всеобъемлюща, как жизнь, но она – не настоящая. Школа – это
временно, и если Вы посмотрите, Вы сможете заглянуть за неё, даже если
Вы по-прежнему находитесь в ней.

Если жизнь кажется ужасной для подростков, это не потому, что ваши
гормоны превратились в монстров (как думают родители), и не потому,
что жизнь и в самом деле ужасна (как думаете вы). Это потому что
взрослые, которые не могут использовать вас в экономике, оставили вас,
чтобы вы провели несколько лет собранные вместе, безо всяких реальных
дел. Любое общество такого типа ужасно. И вам не надо смотреть глубже,
чтобы понять, почему тинейджеры несчастны.

Я сказал несколько жестоких вещей в этом эссе, но главный тезис
оптимистичен – проблемы, которые перед нами стоят, не являются
неразрешимыми. Подростки не являются несчастными чудовищами по своей
сути. И это должно быть ободряющей новостью и для детей, и для
взрослых.

\end{document}
