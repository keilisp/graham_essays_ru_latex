\documentclass[ebook,12pt,oneside,openany]{memoir}
\usepackage[utf8x]{inputenc} \usepackage[russian]{babel}
\usepackage[papersize={90mm,120mm}, margin=2mm]{geometry}
\sloppy
\usepackage{url} \title{Что такое жизнь настоящего стартапера}
\author{Пол Грэм} \date{}
\begin{document}
\maketitle

Я не был уверен, о чем мне лучше поговорить в выступлении на Startup
School, так что решил спросить об этом создателей тех стартапов, что
мы профинансировали. О чем я еще не написал?

У меня необычное положение: я могу проверять на практике то, что пишу
о стартапах. Надеюсь, мои тексты по другим темам тоже нормальные, но
проверить это невозможно. А то, что я пишу о стартапах, тестируют на
себя порядка 70 человек каждые полгода.

Поэтому я отправил всем предпринимателям, с какими работал, письмо с
просьбой рассказать, что их больше всего поразило в их стартапе.
Другой смысл этого вопроса – что я сделал не так: ведь если бы я все
объяснил достаточно хорошо, их ничто не должно было удивить.

Я горжусь тем, что один ответ звучал так: «Больше всего меня удивило
то, что все было довольно-таки предсказуемо!»

Но плохая новость в том, что я получил около сотни ответов с перечнем
сюрпризов, на которые наткнулись предприниматели. В этих ответах видны
четкие паттерны: поразительно, как часто людей удивляет буквально одно
и то же. Вот самые главные темы.

1. Осторожнее с сооснователями!

Об этом сюрпризе говорили большинство предпринимателей. Я видел два
типа ответов: что нужно осторожно выбирать сооснователей, и что нужно
много трудиться над поддержанием отношений.

Люди хотели бы обращать больше внимания при поиске «кофаундеров» на их
характер и приверженность делу, а не на их способности. И чаще всего
так говорили те, чьи стартапы провалились. Урок: не берите
сооснователей, которые вас подставят.

Вот типичный ответ: «Никогда не узнаешь, каков человек на самом деле,
пока не поработаешь с ним в стартапе».

Характер очень важен, так как в стартапе он подвергается более суровой
проверке, чем в других ситуациях. Один предприниматель прямо сказал,
что отношения между соучредителями важнее их навыков: «Я бы лучше взял
в партнеры друга, чем более эффективного незнакомца. Стартапы
настолько трудны и эмоциональны, что тесные связи, эмоциональная и
социальная поддержка, которую дает дружба, перевешивают потерю
эффективности».

Мы в Y Combinator давно усвоили этот урок: если вы посмотрите на нашу
анкету, там больше вопросов о готовности работать и об отношениях
между основателями, чем об их навыках.

Основатели успешных стартапов меньше говорили о выборе партнеров и
больше – о том, как непросто им было поддерживать дальнейшие
отношения:

«Что меня удивило – это как отношения между стартаперами превращаются
из дружбы практически в брак. Мы с моим сооснователем сначала были
просто друзьями, потом мы начали видеться постоянно, браниться из-за
денег и уборки. И стартап был нашим детищем. Однажды я подытожил все
это: мы как будто женаты, только не трахаемся».

Несколько из опрошенных мной людей использовали это слово – «женаты».
Это гораздо более тесные отношения, чем обычно видишь между коллегами
– отчасти потому, что стресс гораздо больше, а отчасти – потому что
основатели и есть вся компания. Так что эти отношения должны быть
построены из высококачественного материала и должны тщательно
поддерживаться. Это основа всего.

2. Стартап отбирает твою жизнь

Отношения между основателями и самой компанией тоже весьма
интенсивные. Управлять стартапом – это не то же самое, что работать
или учиться где-то: ты никогда не останавливаешься. Это настолько
чуждый для большинства людей опыт, что они не понимают этого, пока не
ощутят на своей шкуре:

«Я не осознавал, что буду тратить практически каждую минуту дня на
работу в стартапе или на размышления о нем. Когда ты создаешь свою
компанию, начинается совсем другая жизнь».

Дело осложняет быстрый темп жизни в стартапах. Кажется, время
замедляется:

«Думаю, самое удивительное для меня было то, как меняется
представление о времени. Работая в стартапе, я видел, как время прямо
растягивается: месяц – это очень долго».

Если все здорово, то это полное погружение может быть даже
увлекательным:

«Удивительно, насколько тебя поглощает стартап: ты думаешь о нем днем
и ночью, но никогда это не кажется «работой»».

Но я должен сказать, что эта цитата принадлежит человеку, который
получил от нас финансирование этим летом. Возможно, через пару лет он
не будет высказываться так бодро.

3. Это просто «американские горки»

Массу людей это сильно удивило. Взлеты и падения оказались более
экстремальными, чем то, к чему они готовились.

В стартапе все кажется то прекрасным, то безнадежным. И этот настрой
может меняться буквально каждые пару часов:

«Эмоциональные подъемы и падения были для меня главным сюрпризом. В
один день нам кажется, что мы – новый Google, мы грезим о том, как
будем покупать острова; на следующий день мы взвешиваем, как
рассказать своим любимым о нашем полном провале; и так далее, и так
далее».

Естественно, неприятная часть – это падения. Для многих основателей
это и было главным сюрпризом:

«Как трудно поддерживать у всех мотивацию в трудные дни и недели –
т.е. на каком дне мы можем оказаться».

Проходит время, и если у вас нет значительных успехов, которые могли
бы вас подбодрить, это истощает:

«Ваш главный совет основателям – «просто не умирайте». Но энергия,
чтобы поддерживать компанию на плаву в счет будущих успехов, не
бесплатна – она выкачивается из самих основателей».

Есть предел тому, что вы можете вынести. Если вы дошли до черты, за
которой работать уже невозможно, это не конец света. Многие знаменитые
предприниматели по дороге к успеху натыкались на неудачи.

4. Это бывает весело

Есть и хорошие новости: подъем – это действительно подъем, и еще
какой:

«Я думаю, вы умолчали о том, насколько весело делать стартап. Я больше
удовлетворен своей работой, чем практически все мои друзья, не
создавшие своих компаний».

Больше всего нравится свобода:

«Я поражен тем, насколько приятнее работать над чем-то сложным и
творческим, чем-то, во что я верю, а не теми делами, что я занимался,
работая по найму. Я знал, что будет лучше – но был удивлен тем,
насколько лучше».

Но честно говоря, если я и вводил людей в заблуждение, то не спешу это
исправлять. Лучше бы все думали, что сделать стартап – это суровое,
тяжелое дело, чем начинать это, думая, что будет сплошное веселье, а
через несколько месяцев говорить: «Вот это, вы считаете, весело? Вы
что, издеваетесь?»

И по правде, для большинства людей весело это не будет. Принимая
заявки, мы очень стараемся отсеять людей, которым стартаперство не
понравится – это и в их интересах, и в наших.

Лучше всего сказать так: начать стартап – это весело в том смысле, в
каком может быть веселым курс выживания, если вам такое нравится. А
если нет – то совсем не весело.

5. Главное – настойчивость

Большинство основателей были удивлены тем, насколько важна в стартапах
настойчивость. Это был и позитивный, и негативный сюрприз. Одних
удивляло, насколько настойчивыми надо быть:

«Все говорят, каким целеустремленным и твердым надо быть, но по ходу
дела я понял, что это еще мягко сказано».

А других – что одной настойчивости хватает, чтобы преодолеть
препятствия:

«Если вы настойчивы, даже те проблемы, которые, кажется, вне вашего
контроля (скажем, иммиграция) как-то сами собой решаются».

Несколько основателей высказались еще прямее: «Я снова и снова
удивлялся, насколько настойчивость важнее интеллекта в чистом виде».

И это касается не только интеллекта, но и вообще способностей. И вот
поэтому так много людей говорят, что в выборе партнера важнее его
характер.

6. Думайте на долгую перспективу

Настойчивость нужна, потому что все отнимает больше времени, чем
ожидаешь вначале. Многих людей удивляло именно это:

«Я постоянно удивляюсь, как все затягивается. Если ваш продукт не
добьется взрывного роста, а это под силу очень немногим продуктам, то
все от разработки до сделок (особенно сделки) занимает в 2-3 раза
дольше, чем я всегда думал».

Отчасти предприниматели удивлены потому, что они сами работают быстро
и от других ждут того же. И когда стартап сталкивается с более
бюрократической структурой – большой компанией или венчурным фондом, –
возникает колоссальный стресс. Вот почему привлечение средств и
корпоративный рынок убивают и истязают столько стартапов.

Но я думаю, что главная причина удивления – излишняя уверенность. Люди
думают, что они мгновенно станут успешны, как YouTube или Facebook.
Говоришь им, что лишь у 1 из 100 успешных стартапов бывает такая
траектория, а они все думают: «Ну мы-то станем этим одним».

Может, они прислушаются к одному из успешных основателей:

«Главное, чего я не понимал, прежде чем занялся этим – что упорство
всему голова. У подавляющего большинства успешных стартапов
путешествие очень долгое, минимум 3 года, а то и больше пяти».

Когда думаешь на долгую перспективу, в этом есть плюсы. И не так
страшно, что приходится ждать дольше, чем следовало бы. Если вы
работаете терпеливо, это не такой уж стресс, а работа получается
лучше:

«Поскольку мы не напрягаемся по этому поводу, гораздо легче получать
удовольствие от того, что мы делаем. Больше нет этой некомфортной,
нервной энергии, порождаемой отчаянным желанием не проиграть. Мы можем
сосредоточиться на том, что правильнее всего для компании, для
продукта, сотрудников и клиентов».

Вот почему все становится гораздо лучше, когда вы достигаете
минимальной прибыльности. Вы просто переключаетесь в другой режим
работы.

7. Много мелочей

Мы часто подчеркиваем, как редко стартапы побеждают лишь благодаря
тому, что набрели на какую-то волшебную идею. Думаю, теперь до
предпринимателей это уже дошло. Но многие были удивлены, что это
применимо и к внутренним делам стартапов. Нужно заниматься кучей
разных дел:

«Это больше рутина, чем гламур. В любой произвольно выбранный момент
времени я, скорее всего, отслеживал странный баг при загрузке в
шведской версии Swedish Windows или глюк в финансовой модели,
возникший прямо перед советом директоров, а не наслаждался
стратегическими озарениями».

Большинство основателей-программистов с радостью бы все свое рабочее
время программировали. Но этого не выйдет – иначе их ждет провал. И
это касается даже самого программирования – редко когда находится
гениальный кусок кода, который гарантирует успех. А даже если и
находится, вы узнаете это лишь потом.

Так что самая лучшая стратегия – пробовать много и разного. Обычно не
советуют складывать все яйца в одну корзину, даже если вы знаете,
какая корзина лучше. В стартапе вы не знаете и этого.

8. Начните с минимума

Многие предприниматели говорили, как важно запуститься с самым
простейшим продуктом. Сейчас все уже знают, что надо выпускать продукт
быстро и все время его совершенствовать. Но все равно многие
обжигаются, пытаясь обойти это требование. Почему же на первую версию
уходит так много времени? Обычно дело в гордыне: не хочется выпускать
продукт, который может быть лучше. И в волнении: что скажут люди? Но
это нужно преодолеть:

«Если вы делаете что-то на первый взгляд «простое», это не значит, что
ваше дело лишено смысла, ценности или не обосновано».

Не волнуйтесь о том, что скажут люди. Если ваша первая версия
настолько впечатляет, что троллям даже не нашлось над чем
поиздеваться, значит, вы слишком долго ждали запуска.

Слишком много разработки – это яд. Это не дополнительная работа ради
новых заслуг: это все равно что лгать, а потом все время помнить об
этой лжи, чтобы ненароком ее не опровергнуть.

9. Привлекайте пользователей

Разработка продукта – это разговор с пользователем, который реально
начинается только после запуска. А до запуска вы играете роль
полицейского, который еще не успел показать жертве преступления
фоторобот подозреваемого.

Запускать продукт быстро так важно, что лучше думать о своей первой
версии не как о продукте, а как о трюке, завлекающем пользователей и
заставляющем их говорить о вас:

«Я научился думать о первых шагах стартапа как о гигантском
эксперименте. Все продукты должны считаться экспериментами, и те, у
которых есть реальный рынок, очень быстро показывают многообещающие
результаты».

«Когда вы позволите клиентам рассказать вам, чего же они на самом деле
хотят, почти наверняка обнаружатся поразительные детали о том, что они
считают ценным, и за что они готовы платить».

10. Меняйте свою основную идею

Чтобы общение с пользователями принесло пользу, вы должны быть готовы
поменять свою базовую идею. Мы всегда поощряли стартаперов видеть в их
идее гипотезу, а не готовый план. И все равно они удивляются, как это
полезно:

«Обычно если ты жалуешься на сложности, советуют работать еще
усерднее. В стартапе, думаю, надо искать проблемы, которые вам легко
решить».

А чистое упорство без какой-либо гибкости может принести вам лишь
посредственный результат. Лучше мчаться вперед, но при этом успевать
поворачивать, чтобы найти самый перспективный путь.

Люди не понимают, как сложно оценивать идеи стартапов, особенно если
это их собственные идеи. Опытные предприниматели учатся быть открытыми
ко всему:

«Теперь я больше не смеюсь над идеями, потому что понял, как плохо мне
раньше удавалось разобрать, хороши они или плохи».

Никогда не знаешь, что сработает. Надо лишь делать то, что в данный
момент кажется наилучшим. Мы поступаем так и в Y Combinator. Мы все
еще не знаем, работает это или нет, но гипотеза, похоже, вменяемая.

11. Не думайте о конкурентах

Когда вам кажется, что вы придумали отличную идею, это все равно что
нечистая совесть. Стоит кому-то посмотреть на вас странным взглядом, и
вы тут же думаете: «О боже, они все знают».

Эта тревога почти всегда ложная:

«Компании, которые сперва казались конкурентами или угрозами, обычно
переставали быть таковыми, как только к ним внимательно присмотришься.
Даже если они работают в той же нише, у них другая цель».

Люди слишком нервно реагируют на конкурентов, потому что переоценивают
важность идей. Если бы идеи были главным секретом успеха, то
конкурент, нашедший такую же идею, был бы реальной угрозой. Но обычно
самое главное – исполнение:

«Все страхи, которые вызывает появление нового конкурента, через
считанные недели забываются. В конечном счете дело всегда в том, каков
ваш собственный продукт и подход к рынку».

Это верно даже тогда, когда конкурентам удалось привлечь к себе много
внимания:

«Конкуренты, которые купаются в отзывах блоггеров – это не настоящие
победители, они могут быстро выйти из игры. Вам нужны не отзывы, а
клиенты».

12. Клиентов найти трудно

Многие предприниматели жаловались, как трудно искать клиентов. Когда
это не удается, трудно сказать, в чем проблема – в том, что вас плохо
знают, или что у вас плохой продукт. Даже хорошие продукты натыкаются
на издержки перехода или интеграции. Самая острая критика в адрес YC
прозвучала от предпринимателя, который сказал, что мы недостаточно
говорим о привлечении клиентов:

«YC проповедует лозунг «делайте то, что люди захотят иметь» как
инженерную задачу: бесконечный поток функций, пока не удовлетворишь
достаточно людей, и пока твое решение не взлетит. Но очень мало
внимания к тому, во что обходится привлечение клиентов».

Может, и так. Если вы выпускаете продукт, где главный вызов –
технический, можно положиться на отзывы и рекомендации, как это сделал
Google. Но в других стартапах главное – не функции, а умение заключать
сделки и маркетинг.

13. Сделки: ожидайте худшего

Сделки срываются. Это константа стартап-мира. Стартапы не имеют
власти, а хорошие идеи часто кажутся неправильными. Поэтому все
нервничают, закрывая сделки с вами, и вы с этим ничего не можете
поделать.

Это особенно касается инвесторов:

«Сейчас понятно, что было бы гораздо лучше, если бы мы действовали,
исходя из гипотезы, что мы вообще не получим дополнительных внешних
инвестиций. Тогда бы мы раньше сосредоточились на источниках доходов».

Мой совет – нужен пессимизм. Исходите из того, что вы не получите
денег, и если кто-нибудь предложит вам хоть сколько-нибудь, считайте,
что больше вы не получите.

Почему основатели меня игнорируют? В основном потому, что они по
природе оптимисты. Но ошибка случается, когда вы оптимистичны в
отношении вещей, которые не контролируете. Ваша способность создать
нечто великое и должна вызывать оптимизм. Но если этот оптимизм
вызывают крупные компании и инвесторы – вы напрашиваетесь на проблемы.

14. Инвесторы ничего не понимают

Многие предприниматели рассказывали о своем удивлении от бестолковости
инвесторов:

«Они даже не знают, во что инвестировали. Я видел инвесторов, которые
вложились в одно устройство, и когда я попросил их продемонстрировать
его, они не знали, как его включить».

Почему бестолковость венчурных инвесторов так удивляет? Думаю, потому,
что они кажутся такими внушительными и крутыми. Такова их профессия.
Нельзя стать венчурным инвестором, не убедив управляющих активами
доверить тебе сотни миллионов долларов. А как это сделать? Надо
выглядеть уверенным и делать вид, что ты разбираешься в технологиях.

15. Приходится играть в игры

Инвесторы не умеют оценивать предпринимателей, поэтому вам приходится
продавать себя им усерднее, чем хотелось бы. Один предприниматель
сказал, что больше всего его поразило, насколько инвесторов впечатляет
притворная уверенность стартаперов.

Я думал, это я цинично отношусь к венчурным инвесторам, но
предприниматели оказались гораздо более циничны: «Во многом то, чем
занимаются основатели стартапов – это показуха. Это работает».

Это тот же феномен, что мы видели выше. Венчурные инвесторы получают
деньги, демонстрируя уверенность своим партнерам, а основатели
получают деньги, демонстрируя уверенность венчурным инвесторам.

16. Удача – это очень важно

«Я не осознавал, какую большую роль играет удача, и как многое нам
неподвластно».

Если вы вспомните знаменитые стартапы, совершенно ясно, насколько
важна удача. Где была бы Microsoft, если бы IBM настояла на
эксклюзивной лицензии на DOS?

«Когда мы создавали наш стартап, я проникся мечтой стартапера – что
это игра для умелых. В каком-то смысле так и есть. Навыки – это ценно.
Как и адская целеустремленность. Но удачливость – критический
ингредиент».

Об удаче говорят не те, чьи стартапы провалились. Основатели, которые
быстро терпят поражение, склонны винить самих себя. А основатели,
которые быстро преуспевают, обычно не осознают, как им повезло. А вот
те, кто в середине, видят, насколько важна удача.

17. Ценность сообщества

Многие предприниматели сказали, что больше всего их удивило в
стартаперстве, как ценно сообщество: «Большое преимущество – жить в
Кремниевой долине, где нельзя пройти мимо всех продвинутых технологий
и новостей стартапов, где ты постоянно натыкаешься на полезных людей».

Что больше всего их удивляет – это всеобщая благожелательность,
«готовность людей помогать нам. Даже люди, для которых в этом не было
выгоды, делали все, чтобы помочь нашему стартапу добиться успеха».

Это одна из причин, почему мне нравится быть частью этого мира.
Творить богатство – это не игра с нулевой суммой; чтобы выиграть, не
нужно наносить удары в спину.

18. Никакого уважения

Стартаперы говорили об удивительной вещи, о которой я забыл: за
пределами стартап-мира они не пользуются уважением:

«Общаясь с людьми, я заметил, что меня гораздо больше уважают, когда я
говорю, что работал над Microsoft Office, чем когда говорю, что
работал над маленьким стартапом Х, о котором люди не слышали».

Отчасти дело в том, что остальной мир просто не понимает стартапы, а
отчасти потому, что большинство хороших стартап-идей кажутся плохими:

«Если рассказать о своей идее случайному человеку, в 95\% случаев он
инстинктивно подумает, что тебя ждет провал и ты зря тратишь время
(хотя вряд ли он скажет это прямо)».

Увы, это распространяется даже на личную жизнь:

«Меня поразило, что создание стартапа не дает тебе дополнительных
очков в глазах женщин».

Я об этом знал, но позабыл.

19. С ростом все меняется

Последний большой сюрприз в этом списке – как все меняется, когда
начинается рост. Главная перемена – теперь программировать доводится
еще меньше:

«Задачи твоей должности технического директора или CEO каждые
полгода-год полностью меняются. Меньше программирования, больше
управления/планирования/работы над оргструктурой, поиска людей,
устранения проблем, да и вообще подготовки к тому, что должно
случиться через несколько месяцев».

В частности, теперь приходится иметь дело с сотрудниками, а у них
часто бывают очень разные мотивы. К счастью, стресса становится
гораздо меньше, когда достигаешь крейсерской высоты:

«Если вспомнить самое начало, то сейчас 75\% стресса больше нет.
Теперь управлять бизнесом гораздо приятнее. Мы больше уверены в себе.
Мы более терпеливы. Мы меньше ссоримся. Мы больше спим».

Хотелось бы, чтобы так было во всех успешных стартапах, но 75\% –
все-таки очень оптимистично.

Так в чем тут дело?

Люди не понимают, насколько особенное это дело – стартаперство, пока
не займутся этим. Почему? Надо понять, в чем оно особенное, от чего
оно отличается. Когда формулируешь так, ответ очевиден: оно отличается
от работы по найму. Для всех людей главная модель работы – это работа
по найму. Даже если вы сами не работали по найму, так, вероятно,
работали ваши родители и практически все остальные взрослые, с кем вы
знакомы.

Подсознательно все ждут, что работа в стартапе будет примерно такой
же, как по найму, и это объясняет большую часть сюрпризов. И в том
числе объясняет, почему хорошие моменты кажутся столь прекрасными:
большинство людей просто не могут представить себе такой свободы.

Вряд ли можно преодолеть столь укорененное в сознании представление о
работе. Поэтому лучше всего просто осознавать это. Создавая стартап,
вы подумаете: «Все говорят, что это настоящий экстрим». А затем,
наверное, подумаете: «Но не верится, что это так ужасно». Если хотите
избежать сюрпризов, то следующей вашей мыслью должно стать: «И мне не
верится в это потому, что моя модель работы – это работа по найму».

\end{document}
