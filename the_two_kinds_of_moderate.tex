\documentclass[ebook,12pt,oneside,openany]{memoir}
\usepackage[utf8x]{inputenc} \usepackage[russian]{babel}
\usepackage[papersize={90mm,120mm}, margin=2mm]{geometry}
\sloppy
\usepackage{url} \title{О нейтральности политических взглядов и
  независимости мышления} \author{Пол Грэм} \date{}
\begin{document}
\maketitle

Существует два вида политической умеренности: осознанная и
произвольная. Сторонники осознанной умеренности – это перебежчики,
сознательно выбирающие свою позицию между крайностям правой и левой
сторон. В свою очередь те, чьи взгляды являются произвольно умеренными
оказываются в середине, так как они рассматривают каждый вопрос по
отдельности, а крайние правые или левые взгляды для них одинаково
неверны.

Вы можете отличить тех, чья умеренность сознательна от тех, для кого
это вопрос случая. Если взять шкалу, в которой крайнее левое мнение по
какому-либо вопросу равно 0, а крайнее правое – 100, то в случае с
осознанной умеренностью оценка мнения людей по каждому из вопросов
будет примерно равна 50. У людей, которые не задумывались об
умеренности своих взглядов, оценки будут разбросаны по разным областям
условной шкалы, но среднее значение оценки будет стремиться к 50.

Люди с осознанной умеренностью схожи с крайне левыми и крайне правыми
в том, что в некотором смысле их мнения не являются их собственными.
Определяющим качеством идеолога (как левого, так и правого) является
цельность его мнения. Люди, для которых умеренность является
сознательной позицией не принимают отдельных решений по разным
вопросам. Их мнение о налогообложении можно предсказать на основе
отношения к однополым бракам. И хотя может показаться, что такие люди
являются противоположностью идеологов, их убеждения (хотя в этом
случае вернее сказать “их позиции по различным вопросам”) также цельны
и последовательны. Если усредненное мнение сместится влево или вправо,
то соответственно сместится и мнение людей с осознанной умеренностью
взглядов. В противном случае их взгляды перестанут быть умеренными.

В свою очередь люди, умеренность которых произвольна, выбирают не
только ответы, но и сами вопросы. Они могут не придавать значения тем
проблемам, которые очень важны для сторонников левых или правых идей.
Таким образом вы можете оценить взгляды человека с произвольной
умеренностью по пересечению тех вопросов, которые важны и для него, и
для тех, кто придерживается левой или правой стороны (хотя иногда это
пересечение может быть очень мало).

Фраза «если ты не с нами, ты против нас» не просто является
риторической манипуляцией, зачастую она просто неверна.

Людей с умеренными взглядами зачастую высмеивают как трусов, особенно
приверженцы левых идеологий. И хотя возможно и правильно считать
трусами людей, которые умышленно придерживаются умеренных взглядов,
больше всего храбрости необходимо для того чтобы не скрывать свою
произвольную умеренность, потому что вам будут предъявлять претензии и
справа, и слева, а возможности быть членом какой-то большей группы,
которая может оказать поддержку, нет.

Почти все самые впечатляющие люди, которых я знаю, придерживаются
произвольной умеренности в своих взглядах. Если бы я был знаком с
большим числом профессиональных спортсменов или людей в индустрии
развлечений, то мой опыт мог бы отличаться. Приверженность к правой
или левой стороне не влияет на скорость вашего бега или то, насколько
хорошо вы поете. Но тот, кто работает с идеями, должен иметь
независимый ум, чтобы справляться со своей работой хорошо.

Если говорить более точно, вы должны подходить с независимым мышлением
к тем идеям, с которыми работаете. Вы можете очень строго следовать
какой-либо политической доктрине и при этом быть хорошим математиком.
В XX веке многие хорошие люди были марксистами – просто никто не
разобрался в том, что марксизм в себя включает. Но если идеи, которые
вы используете в своей работе, пересекаются с политикой вашего
времени, то у вас есть два варианта: придерживаться произвольной
умеренности или быть посредственностью.

Заметки

[1] Теоретически возможно, что одна сторона полностью права, а другая
– неправа полностью. Действительно, идеологи всегда должны верить, что
так оно и есть. Но в истории такое случалось редко.

[2] Почему-то приверженцы крайне правых идей склонны игнорировать
людей с умеренными взглядами, а не презирать их как отступников. Я не
уверен почему. Возможно это значит, что крайняя правая сторона менее
идеологична чем крайняя левая. А может быть они более уверены в себе,
смиренны или дезорганизованы. Я не знаю.

[3] Если у вас есть мнение, которое считается ересью, то вы не обязаны
выражать его открыто. Возможно вам будет проще его сохранить, если вы
этого не сделаете.

Люди, которым я благодарен за прочтение черновиков этого текста: Остен
Оллред, Тревор Блэквелл, Патрик Коллисон, Джессика Ливингстон, Амджад
Масад, Райан Петерсен и Харж Таггар.

\end{document}
