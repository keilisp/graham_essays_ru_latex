\documentclass[ebook,12pt,oneside,openany]{memoir}
\usepackage[utf8x]{inputenc} \usepackage[russian]{babel}
\usepackage[papersize={90mm,120mm}, margin=2mm]{geometry}
\sloppy
\usepackage{url} \title{Сила маргинального} \author{Пол Грэм} \date{}
\begin{document}
\maketitle

Пару лет назад мы с моим другом Тревором отправились посмотреть на
гараж Apple.

Пока мы там стояли, он сказал, что как человек, выросший в Саскачеване
(1), поражён самоотверженностью, которой должны были обладать Джобс и
Возняк, чтобы работать в гараже. «Должно быть, эти ребята страшно
мёрзли!”

Здесь кроется один из тайных плюсов Калифорнии: мягкий климат = много
маргинального пространства.

Холодные места лишены этого преимущества. Там чётче линии между
“внутри” и “снаружи”, и только проекты, санкционированные официально —
организациями, родителями, жёнами, или, по крайней мере, самим собой —
получают надлежащее пространство “внутри”.

Это вызывает активацию энергии для новых идей. Вы не можете просто
повозиться. Вы должны оправдать.

Некоторые из самых известных компаний Кремниевой долины начинались в
гаражах: Хьюлетт-Паккард (Hewlett-Packard) в 1938, Apple в 1976,
Google в 1998. В случае Apple, история про гараж имеет оттенок
городской легенды.

Воз (Стив Возняк) говорит, что они делали в гараже только сборку
нескольких компьютеров, а весь дизайн Apple I и Apple II был сделан в
его квартире или в его офисе в HP [1]. Что было явно слишком
маргинально даже для пиарщиков Apple.

По общепринятым меркам, Джобс и Возняк тоже были маргиналами (2).
Очевидно, что они были умны, но они не могли выглядеть хорошо на
бумаге. На тот момент оба были студентами, бросившими университет, с
разницей в 3 учебных года. И, в придачу, оба были хиппи.

Их предыдущий опыт в бизнесе состоял из создания «синих коробочек» для
взлома телефонных сетей — бизнес редкого сочетания незаконности и
убыточности.

1. Аутсайдеры Сегодня стартап, работающий в гараже в Кремниевой
долине, чувствовал бы себя частью высокой традиции — как поэт в своей
каморке. Или художник, который не может позволить себе отапливать свою
студию, и из-за этого должен носить берет в помещении.

Но в 1976 году работа в гараже не казалась такой крутой. Тогда мир ещё
не успел осознать, что запуск компьютерной компании попадает в одну
категорию с писателями и художниками.

Так продолжалось довольно долго. Резкое падение стоимости
оборудования, которое сделало аутсайдеров конкурентоспособными,
случилось всего пару лет назад.

В 1976 все смотрели на компании, работающие в гараже, сверху вниз,
включая их основателей. Первое, что сделал Стви Джобс, получив немного
денег — снял офисное помещение. Он хотел, чтобы Apple выглядела
настоящей компанией.

Уже тогда у Джобса и Возняка было кое-что, о чём многим реальным
компаниям приходилось только мечтать: сказочно хорошо разработанный
продукт.

Вы, должно быть, подумали, что им стоило быть более уверенными. Но я
разговаривал с многими основателями стартапов, и так происходит
всегда. Они создали нечто, что изменит мир, и при этом беспокоились о
такой ерунде, как отсутствие хороших визитных карточек.

Этот парадокс я и хочу исследовать: мы часто обязаны появлению новых
великих вещей маргиналам. Но на людей, которые делают эти вещи, все
смотрят свысока — и даже они сами.

Это старая идея, что всё новое приходит с окраин. Я хочу изучить её
внутреннюю структуру. Почему великие идеи приходят с окраин? Идеи
какого типа? Можем ли мы что-нибудь сделать для стимуляции этого
процесса?

2. Инсайдеры Одна из причин, что так много хороших идей приходит с
окраин — просто факт, что аутсайдеров больше. Аутсайдеров и должно
быть больше, чем инсайдеров, если статус инсайдера что-то значит.

Если число аутсайдеров огромно, то всегда будет казаться, что от них
приходит много идей, даже если в пересчёте на душу населения это очень
низкий процент.

Но, думаю, это далеко не единственная причина. У статуса инсайдера
есть реальные недостатки, и в некоторых видах работы они могут
перевешивать преимущества.

Представьте, например, что произойдет, если правительство решит
заказать кому-нибудь написание официального Великого Американского
Романа.

Во-первых, начнётся великая идеологическая перебранка, кого выбирать.
Большинство лучших писателей тут же будут исключены, как уязвившие
одну или другую сторону.

Умные откажутся от такой работы, и останется только несколько
претендентов с амбициями самого низкого пошиба. Комитет выберет
кого-нибудь, находящегося на пике своей карьеры — то есть кого-то, чьи
лучшие работы уже позади — и отдаст проект ему, попутно обильно
раздавая советы, как книга должна будет показать в позитивном ключе
силу и разнообразие американского народа, и всё в том же духе.

Потом писатель-горемыка сядет за работу с огромным весом
ответственности на плечах. Не желая провалить столь публичный заказ,
он перестрахуется.

Книга должна вызывать уважение, и верный способ это устроить —
написать трагедию. Аудиторию нужно заставлять смеяться, но если
написать про убийство людей, вас воспримут всерьёз. Как всем известно,
Америка + Трагедия = Гражданская война, так что она и станет темой
книги.

Лучше придерживаться стандартной версии, что предметом Гражданской
войны было рабство, иначе люди придут в замешательство; плюс это шанс
показать много силы и разнообразия (diversity).

Спустя 12 лет книга, наконец, будет окончена, и будет представлять
собой 900 страниц стилизации популярных романов — “Унесённые ветром”
Маргарет Митчелл плюс “Корни” Алека Хейли.

Размах и известность книги сделают её хитом продаж на несколько
месяцев — до тех пор, пока её не вытеснит биография какого-нибудь
завсегдатая ток-шоу. Книга будет экранизирована и забыта всеми, кроме
самых въедливых рецензентов, среди которых она станет притчей во
языцех, как группа Milli Vanilli или фильм Battlefield Earth.

Возможно, я немного увлёкся этим примером. И всё же, разве не так
обычно бывает с подобными проектами? Правительство знает лучше, чем
как попасть в литературный бизнес, что и в других областях, где оно
обладает естественной монополией — ядерные отходы, авианосцы, смена
режима — найдётся множество проектов, изоморфных моему примеру. Многие
из которых были даже менее успешными.

Мой маленький мысленный эксперимент описывает несколько недостатков
инсайдерских проектов: выбор неправильных людей, чрезмерный объём,
неспособность принимать риски, потребность казаться серьезным, груз
ответственности и ожиданий, сила личных корыстных интересов, аморфная
аудитория, которой не хватает вкуса и проницательности.

И, возможно, самая опасная тенденция такой работы: она становится
больше обязанностью, чем удовольствием.

3. Тесты Мир, состоящий из инсайдеров и аутсайдеров, подразумевает
наличие какого-то критерия отличия одних от других. И беда большинства
тестов для отбора элиты в том, что есть 2 способа пройти их: быть
крутым в том, что тесты пытаются измерить, или быть крутым во взломе
тестов.

Итак, первый вопрос, который тут стоит задать — насколько честен тест,
потому что именно он определяет, что значит быть аутсайдером. Тест
обозначает, насколько можно доверять своим инстинктам, когда вы не
согласны с признанными авторитетами, стоит ли выбирать стандартные
пути, чтобы стать одним из них, и, возможно, хотите ли вы работать в
этой сфере вообще.

Наименее подвержены взлому тесты, которые представляют собой единые
последовательные стандарты качества, создатели которых по-настоящему
заботятся об их целостности.

Например, тесты для поступления на программы PhD в точных науках
довольно точны. Кого отберут преподаватели, те и будут их аспирантами,
поэтому они очень стараются выбирать хорошо, и у них достаточно
данных, чтобы это сделать. В то время, как тесты бакалавриата кажутся
гораздо более подверженными взлому.

Один из способов определить, имеет ли взятая область последовательные
стандарты — посмотреть на зону перекрытия между ведущими практиками
взятой области и преподавателями этого предмета в университетах.

Вверху шкалы вы обнаружите физику и математику, где почти все
преподаватели являются одними из лучших специалистов-практиков.

В середине будут медицина, право, история, архитектура и computer
science, где большой процент практиков среди учителей.

Внизу окажутся бизнес, литература и изобразительное искусство, где
преподаватели и практики практически не перекрываются. Эта часть шкалы
и порождает такие фразы, как «Кто не умеет, тот учит».

Кстати, эта шкала может помочь определиться, какие предметы стоит
изучать в университете. Когда я был студентом, мне казалось, что
основное правило — изучать то, что больше всего интересно.

Но в ретроспективе я думаю, что, пожалуй, лучше изучать что-то
умеренно интересное с кем-то, кто хорошо в этом разбирается, чем нечто
очень интересное с тем, кто разбирается в предмете не очень хорошо.

Часто можно услышать мнение, что не стоит выбирать бизнес в качестве
основного предмета в университете. Но, на самом деле, это просто
частный случай более общего правила: не учиться у тех, кто плохо знает
свой предмет.

Насколько вам стоит беспокоиться о вашем положении аутсайдера, зависит
от качества инсайдеров.

Если вы любитель математики и думаете, что решили одну из известных
нерешённых задач — лучше сделайте шаг назад и перепроверьте.

Когда я был в аспирантуре, мой друг с математического факультета
работал, отвечая людям, присылающим доказательство последней теоремы
Ферма и подобных известных задач.

Со стороны казалось, что он видел эти письма не источник ценных
советов, а, скорее, как обслуживание горячей линии психического
здоровья. Но когда вещи, которые вы пишете, отличаются от того, что
интересует английских профессоров — это не обязательно проблема.

4. Антитесты Везде, где способ отбора имеет дефекты, большинство
хороших людей будут оказываться аутсайдерами.

Например, в искусстве образ художника, как бедного, непонятого гения —
не просто одно из возможных изображений великого художника. Это
стандартный образ.

Я не говорю, что он верный, но показательно, насколько хорошо он
прижился. Невозможно представить, чтобы такой образ прижился в
математике или медицине. [2]

Если тест повреждён, он становится антитестом, который вместо отбора
людей отсеивает их, заставляя их делать вещи, которые станут делать
только неправильные люди.

Популярность в школе — хороший пример такого антитеста. Есть много
аналогов и во взрослом мире. Например, подъём по лестнице
корпоративной иерархии среднестатистической крупной компании требует
внимания к политике, которое мало кто из думающих людей может
сохранять. [3]

Кто-то вроде Билла Гейтса может вырастить крупную компанию. Но трудно
представить, чтобы он поднялся по карьерной лестнице в General
Electric — или, как ни странно, в Microsoft.

Если подумать, это кажется странным, потому что школы властелинов
файлов и компании с бюрократией — опции по умолчанию. Вероятно, есть
много людей, которые ходят от одного антитеста к другому, и так
никогда и не осознают, что мир в целом не работает подобным образом.

Думаю, это одна из причин, почему стартапы так часто ошарашивают
крупные компании. Люди из больших компаний не осознают, до какой
степени они живут в среде, представляющей из себя один большой
непрерывный антитест с негативной селекцией.

Если вы аутсайдер — очевидно, что ваши наибольшие шансы обыграть
инсайдеров лежат в тех сферах, где антитесты отбирают плохую элиту. Но
вот загвоздка: если тесты испорчены, ваша победа тоже не будет
признана — по крайней мере, не при вашей жизни.

Возможно, вы чувствуете, что вам и не нужно признание. Но история
учит, что работать в областях, где распространены антитесты — опасно.
Вы можете переиграть инсайдеров, но при этом не делать работу
настолько хорошо, как вы бы могли (в абсолютном масштабе), окажись вы
в среде с честным отбором.

К примеру, в 1 половине 18 века стандарты искусства были почти так же
повреждены, как и сейчас. Это была эпоха тех самых пышных
идеализированных портретов графинь со своими болонками.

Шарден решил пройти мимо всего этого и рисовать обычные вещи так, как
он их видел. Сегодня он считается лучшим художником того периода — и
всё-таки не равным Леонардо, Беллини или Мемлингу, у которых было
дополнительное преимущество в виде поощрения честных стандартов.

Тем не менее, участие в антитесте может иметь смысл, если за ним
следует другой, честный тест.

Например, имеет смысл конкурировать с компанией, которая может
потратить на маркетинг больше, чем вы, чтобы дотянуть до следующего
раунда, когда клиенты начнут сравнивать ваши продукты.

Аналогично, не стоит переживать из-за таких сравнительно нечестных
тестов, как вступительные экзамены в университет, потому что сразу за
ними идут менее подверженные взлому тесты. [4]

5. Риск У позиции аутсайдера есть преимущества даже в областях с
честными тестами. Самое очевидное: аутсайдерам нечего терять. Они
могут делать рискованные вещи. Если даже они потерпят неудачу — ну и
что? Никто этого даже не заметит.

Звёзды, напротив, отягощены своей знаменитостью. Высокое положение —
как костюм: он впечатляет не тех людей, и это ограничивает владельца.

Аутсайдеры должны осознавать своё преимущество в этом вопросе.
Способность принимать риски является чрезвычайно ценной. Все слишком
высоко ценят безопасность — и малоизвестные, и знаменитые. Никто не
хочет выглядеть дураком.

Но иметь возможность рисковать очень полезно. Если большинство ваших
идей не глупы, то вы, наверное, слишком консервативны. Вы не
улавливаете сути проблемы.

Как сказал Лорд Актон, талант нужно судить по самым лучшим
достижениям, а личность — по самым худшим проявлениям.

Например, если вы напишите 1 великую книгу и 10 плохих, вы по-прежнему
будете считаться великим писателем — или, по крайней мере, лучшим, чем
тот, кто написал 11 просто хороших книг.

А если вы большую часть времени тихий, законопослушный гражданин, но
иногда режете кого-то и хороните на заднем дворе, то вы — плохой
парень.

Почти все совершают ошибку, относясь к идеям так, как если бы они были
показателями характера, а не таланта — как будто глупая идея делает
вас глупцом.

Огромное наследие традиций советует нам перестраховываться.

Даже глупец, когда молчит, мудрецом засчитается



— говорит Ветхий Завет (притчи 17:28).

Возможно, это был подходящий совет для козопаса из Палестины
бронзового века. Там консерватизм был на повестке дня. Но времена
изменились. Может, и целесообразно придерживаться Ветхого Завета в
политических вопросах, но материальный мир гораздо многообразнее.

Традиция теряет роль путеводного огня не потому, что всё стало быстро
меняться, а потому, что пространство возможностей настолько велико.
Чем сложнее становится мир, тем большую ценность приобретает
готовность выглядеть дураком.

6. Делегирование И ещё, чем успешнее люди становятся, тем больше на
них обрушиваются за ошибки и провалы. Или даже просто если кому-то
начинает казаться, что они их допускают.

В этом отношении, как и во многих других, именитые люди оказываются
пленниками собственного успеха. Поэтому, лучший способ понять
преимущества позиции аутсайдера может состоять в том, чтобы
присмотреться к недостаткам положения инсайдера.

Если вы спросите именитых людей, что плохого есть в их жизни, то
первым делом они пожалуются на нехватку времени.

Мой друг занимает довольно высокий пост в Google. Он начал работать в
Google задолго до того, как они стали публичной компанией. Другими
словами, он теперь достаточно богат, чтобы не работать вообще.

Я спросил его, мог бы ли он сейчас терпеть неудобства постоянной
работы, с которыми ему больше нет нужды мириться. И он ответил, что,
на самом деле, не было никаких особых неудобств, кроме — и тут его
взгляд стал печальным, пока он это говорил — он получал так много
имейлов.

Известные люди чувствуют себя так, будто каждый хочет откусить от них
кусочек. Эта проблема настолько широко распространена, что многие
притворяются известными, имитируя перегруженность.

Жизнь знаменитости становится запланированной, и это не очень хорошо
для мышления.

Одно из огромных преимуществ жизни изгоя — длительные, непрерывающиеся
блоки времени. Вот что я помню об аспирантуре: кажущиеся бесконечными
запасы времени, которые я провёл в переживаниях о (но не за
написанием) своей диссертации.

Неизвестность — это как здоровая еда: неприятно, но полезно для вас. А
известность имеет тенденцию быть похожей на алкоголь. Когда брожение
достигает определенной концентрации, она убивает дрожжи, породившие
её.

В целом, знаменитости реагируют на дефицит времени превращением в
менеджеров. У них просто нет времени работать. Они окружены младшими
сотрудниками, которым они должны помогать или контролировать.

Делегировать этим людям свою работу — очевидное решение. Некоторые
хорошие вещи случаются именно так. Но есть проблемы, которые не
решаются делегированием — проблемы, для работы над которыми лучше
держать всё в одной голове.

Например, недавно выяснилось, что известный художник по стеклу Дэйл
Чихули (Dale Chihuly) на самом деле не выдувал стекло уже 27 лет. У
него есть помощники, которые выполняют работу за него.

Но один из самых ценных источников идей в изобразительном искусстве —
сопротивление материала. Вот почему картины маслом так отличаются от
акварели.

В теории, можно сделать любой отпечаток на любом материале; но на
практике материал направляет вас. И если вы больше не делаете работу
самостоятельно, вы прекращаете учиться на ней.

Так что, если вы так хотите превзойти знаменитостей, что начинаете
делегировать, один из способов их сделать — воспользоваться
преимуществом прямого контакта с материалом.

В искусстве очевидно, как это сделать: выдувать своё стекло,
монтировать собственные фильмы, ставить свои пьесы. И в процессе
обращать пристальное внимание на интересные случайности и новые идеи,
приходящие в голову на лету.

Эту тактику можно приложить к любой работе: если ты аутсайтер — не
будь во власти планов. Часто планирование — это просто слабое место
тех, кто вынужден делегировать.

Существует ли общее правило для нахождения проблем, которые лучше
решать в одной голове? Ну, вы можете сами их создать, взяв любой
проект, который обычно выполняется несколькими людьми, и попытавшись
сделать его самостоятельно.

Работа Возняка — классический пример такого подхода: он всё делал сам
— и аппаратное, и программное обеспечение, и результат был чудесным.
Он утверждает, что в Apple II не было найдено ни одного бага — ни в
аппаратной, ни в программной части.

Другой способ найти хорошие проблемы для решения в одну голову —
сосредоточиться на канавках между окошками плитки шоколада: швах между
задачами, разделёнными между несколькими людьми.

Если вы хотите избежать делегирования — сосредоточьтесь на
вертикальном срезе: например, быть одновременно писателем и
редактором, или одновременно проектировать и строить здания.

Одна особенно хорошая шоколадная канавка пролегает между инструментами
и вещами, которые делаются с их помощью.

Например, языки программирования и приложения обычно пишутся разными
людьми, и именно этому факту мы обязаны худшими недостатками языков
программирования.

Я думаю, что каждый язык должен разрабатываться параллельно с
разработкой большого приложения, написанного на нём, как это было с C
и Unix.

Методы конкуренции с делегированием хорошо приложимы к бизнесу,
поскольку процесс делегирования для него эндемичен.

Вместо того, чтобы избегать процесса делегирования, как издержки
устаревания, многие компании принимают его за признак зрелости.

В крупных компаниях программное обеспечение часто проектируется,
реализуется и продаётся тремя отдельными типами сотрудников. В
стартапе все 3 функции может выполнять 1 человек.

И, хотя модель “одной головы” кажется куда более стрессовой, это одна
из причин, по которой стартапы выигрывают у компаний: потребности
пользователей и способы их удовлетворения находятся в одной голове.

7. Фокус Сам навык инсайдерства может быть недостатком. Тенденция
такова, что как только кто-то становится хорош в каком-либо деле, он
начинает тратить на него всё своё время.

На самом деле, такой сфокусированный подход очень ценен. Большую часть
экспертных навыков составляет как раз способность игнорировать ложные
следы. Но жёсткий фокус имеет и недостатки: вы больше не учитесь в
других областях, и при появлении новых подходов можете оказаться
последним, кто их заметит.

Для аутсайдеров это означает, что у них есть 2 способа выиграть.
Первый — работать одновременно над разными вещами.

Раз вы не можете (пока) получить максимальную пользу от узкого фокуса,
вы можете бросить сети шире и извлечь выгоду из сходства между разными
областями работы.

Как и в случае конкуренции с делегированием путём работы над более
длинными вертикальными срезами, вы можете конкурировать с узкой
специализацией, работая над большими горизонтальными срезами. Пример —
вы и пишете, и иллюстрируете книгу самостоятельно.

Второй способ конкурировать с фокусом — найти что-то, что он упускает.
А именно — новые вещи. Если вы пока ни в чём не хороши — задумайтесь о
работе над чем-то настолько новым, что вряд ли это придёт в голову
другим.

Это что-то не будет престижным, поскольку его ещё никто не умеет
делать хорошо. Зато вы можете полностью занять эту нишу.

Потенциал новых направлений обычно занижается — именно потому, что
никто ещё не исследовал их возможности. Пока Дюрер не начал делать
гравюры, никто не воспринимал их всерьёз.

Гравировка предназначалась для создания маленьких религиозных
изображений — в основном, карточек святых 15 века. Попытки делать
шедевры в этой технике, наверно, производили на современников Дюрера
примерно такой же эффект, как на современного человека — создание
шедевров в технике комиксов.

В компьютерном мире мы получаем не новые инструменты, а новые
платформы: микрокомпьютер, микропроцессор, веб-приложения. Поначалу
они всегда игнорируются, как непригодные для реальной работы.

Но кто-нибудь всё равно решает их попробовать, несмотря ни на что, и
обнаруживает, что они приносят больше пользы, чем все ожидали. Так что
если вы услышите, что люди говорят о новой платформе: “Да, это
популярно и дёшево, но ещё сыро для реальной работы” — хватайтесь за
неё.

Как правило, инсайдеры не только чувствуют себя комфортно, работая в
заданных рамках, но и кровно заинтересованы в том, чтобы сохранять их
до бесконечности. Профессор, сделавший репутацию на открытии некой
новой идеи, вряд ли откроет её замену.

Это особенно верно для компаний, у которых есть не только мастерство и
гордость, задающие их статус кво, но и деньги. Ахиллесова пята
успешных компаний — их неспособность пожирать себя (конкурировать с
собой) (cannibalize). (3)

Многие инновации состоят из замены чего-либо более дешёвой
альтернативой, но компании просто не хотят видеть путь, который
приводит, в том числе, к сокращению существующего источника дохода.

Таким образом, если вы — аутсайдер, вы должны активно искать
проекты-белые вороны. Вместо того, чтобы работать над вещами, которые
сделали престижными инсайдеры, работайте над вещами, которые могут
похитить этот престиж.

По-настоящему сочные новые подходы — не те, которые инсайдеры
отвергают, как невозможные, а те, которые они игнорируют, как
несолидные.

Например, создав Apple II, Возняк сначала предложил его своему
работодателю — НР. Они отказались. Одна из причин — Возняк использовал
телевизор в качестве монитора Apple II для экономии, а HP посчитали,
что не могут производить нечто настолько низкопробное.

8. Меньше значит больше Возняк взял телевизор в качестве монитора по
той простой причине, что не мог себе позволить монитор.

Аутсайдеры не просто свободны, а вынуждены делать более дешёвые и
легковесные вещи. И то, и другое — хорошая ставка на рост: дешёвые
вещи распространяются быстрее, а облегчённые вещи быстрее
эволюционируют.

С другой стороны, именитые люди практически вынуждены работать в
крупных масштабах.

Вместо садовых сараев они обязаны разрабатывать огромные
художественные музеи. Отчасти потому, что у них есть такая
возможность: как и наш гипотетический писатель, они польщены этим
шансом.

Так же они знают, что крупные проекты впечатлят аудиторию просто по по
факту их огромного объёма. Симпатичный садовый сарай легко
проигнорировать; некоторые могут даже посмеяться над ним.

Но вам не удастся похихикать над гигантским музеем, независимо от
того, насколько он вам не понравился. И, в конце концов, есть все эти
люди, работающие на знаменитостей, и надо выбирать проекты так, чтобы
обеспечить им занятость.

Аутсайдеры свободны от всего этого. Они могут работать над маленькими
вещами, и в этом есть что-то очень приятное.

Небольшие вещи могут быть совершенными. С большими всегда что-то не
так. Но в маленьких вещах есть некая маги, которая выходит за рамки
рациональных объяснений. Все дети это знают. У небольших вещей больше
индивидуальности.

К тому же, делать их гораздо веселее. Вы можете делать, что хотите;
вам не нужно удовлетворять комиссии.

И, пожалуй, самое главное: небольшие вещи можно сделать быстро.
Перспектива увидеть готовый проект висит в воздухе, как запах
готовящегося ужина. Если вы работаете быстро — может быть, вы сможете
закончить уже сегодня вечером.

Работа над маленькими вещами — ещё и отличный способ учиться. Самые
важные виды обучения происходят одновременно с проектом («В следующий
раз я не буду…»). Чем короче ваш цикл работы с проектом, тем быстрее
ваше развитие.

Простые материалы обладают тем же шармом, что и небольшой масштаб.
Кроме того, они содержат вызов сделать больше меньшими средствами.

При упоминании этой игры уши каждого дизайнера навостряются — ведь это
игра, в которой нельзя проиграть. Это как играть за JV-сборную: даже
если вы сыграли вничью — вы выиграли. (4)

Так что, как ни парадоксально, бывают случаи, когда меньшее количество
ресурсов приводит к лучшим результатам, поскольку удовольствие
дизайнеров от собственной находчивости более чем компенсирует их
нехватку. [5]

Так что, если вы аутсайдер, воспользуйтесь возможностью делать
маленькие и недорогие вещи. Культивируйте простоту и радость от такой
работы — когда-нибудь вы будете по ней скучать.

9. Ответственность Когда вы станете стары и имениты, по чему вы будете
скучать в том времени, когда были молоды и ещё не определились?
Кажется, больше всего люди скучают по отсутствию обязанностей.

Ответственность — профессиональная болезнь известных людей. В
принципе, её можно избежать — примерно так же, как можно не толстеть с
возрастом. Но мало у кого это получается.

Иногда я подозреваю, что ответственность — это западня, и что наиболее
добродетельным путём было бы уклониться от неё. Но ответственность,
конечно, ограничивает.

Если вы аутсайдер — вы, конечно, тоже очень несвободны. Например, вам
не хватает денег. Но это ограничение другого рода.

Как вас ограничивает ответственность? Самое худшее в ответственности —
что она позволяет не фокусироваться на реальной работе.

Так же, как самые опасные формы прокрастинации — те, что похожи на
работу, обязанности опасны не только тем, что могут занять целый день,
но и тем, что не подают при этом сигналов к действию типа тех, что
возникают, если вы провели весь день, сидя на скамейке в парке.

Большая часть дискомфорта положения аутсайдера связана с осознанием
собственной прокрастинации. Но, это, на самом деле, хорошо. По крайней
мере, так вы настолько близки к работе, что её запах делает вас
голодным.

Если вы аутсайдер, вы находитесь всего в одном шаге от того, чтобы
сделать дела. На самом деле, это — огромный шаг. Шаг, который
большинство людей, похоже, никогда не делают. Но это только первый
шаг.

Если вам удастся призвать энергию для начала работы, вы сможете
работать над проектами с интенсивностью (в обоих смыслах), которая
едва ли доступна инсайдерам.

Работа инсайдеров быстро превращается в ответственность, обременённую
обязанностями и ожиданиями. Для них работа больше никогда не бывает
такой чистой и первозданной, как в период их молодости.

Работать, как собака, которую ведут на прогулку, вместо того, чтобы
быть волом, запряжённым в плуг. Вот по чему они скучают.

10. Аудитория Многие аутсайдеры совершают ошибку, делая всё наоборот:
они так восхищаются именитыми, что копируют даже их недостатки.
Копирование — хороший способ учиться, но важно копировать правильные
вещи.

Когда я был студентом, я подражал пафосной дикции известных
профессоров. Но знаменитыми их делала не дикция — скорее, она было
недостатком, которому их именитость позволяла им потакать.

Имитировать этот недостаток — примерно как притворяться, что у вас
подагра, чтобы казаться богатым.

Половина отличительных качеств выдающихся людей — на самом деле
недостатки. Имитировать их — не только пустая трата времени. Это ещё и
выставит вас идиотом перед теми, кто хорошо знает ваши повадки.

Каковы настоящие преимущества положения инсайдера? Наибольшее —
аудитория.

Аутсайдерам часто кажется, что наибольшее преимущество инсайдеров —
деньги. Что у них есть ресурсы чтобы делать то, что они хотят.

Но люди, которые наследуют деньги, так и делают, и не похоже, чтобы им
это помогало. По крайней мере, не так сильно, как аудитория. Для
морального духа полезно знать, что люди хотят видеть то, что вы
делаете: это вытаскивает из вас работу.

Если я прав в том, что аудитория — определяющее преимущество
инсайдеров, то мы живем в удивительное время, потому что всего за
последние 10 лет интернет сделал аудиторию намного более текучей.

Аутсайдерам больше нет нужды довольствоваться аудиторией из нескольких
доверенных умных друзей. Сейчас, благодаря интернету, они могут начать
растить свою собственную аудиторию.

Это отличная новость для маргиналов, которые сохраняют преимущества
аутсайдеров и одновременно всё больше оттягивают на себя преимущества,
которые до недавнего времени были уделом элиты.

Несмотря на то, что интернет существует уже более десяти лет (на 2006
год), я думаю, мы всё ещё только начинаем видеть его эффекты
демократизации. Аутсайдеры всё ещё учатся тому, как присваивать
аудитории. Но, что более важно, аудитории до сих пор учатся быть
украденными.

Они ещё только начинают осознавать, насколько глубже блогеры могут
копать, чем журналисты. Насколько интереснее может быть демократичный
новостной сайт, чем главная страница, контролируемая редакторами, и
насколько смешнее массово производимых ситкомов может быть компания
детей с вебкамерами.

Крупные медиакомпании не должны беспокоиться о том, что люди будут
размещать их авторские материалы на YouTube. Они должны беспокоиться,
что люди будут создавать свои собственные видео на YouTube, и зрители
будут смотреть их вместо роликов от компаний.

11. Хакинг Если бы мне пришлось выразить всю суть маргинальности в
одном предложении, это было бы: просто пробуйте хакать что-то вместе.
Эта фраза стоит за многими линиями повествования, которые я здесь
затрагиваю.

Совместный хакинг означает вместе решать, что делать, по мере делания,
в отличие от подчиненного воплощения замыслов своего босса.

Это подразумевает, что результат будет не слишком красив, потому что
будет сделан быстро и из неправильных материалов. Он может работать,
но не будет той вещью, на которой именитые хотят поставить своё имя.

Хакнуть что-то вместе означает создать нечто, что едва ли решает
проблему. Или, может быть, не решает её совсем, но решает другую,
открытую вами по пути. Это нормально, потому что главная ценность
начальной версии — не она сама, а то, к чему она ведёт.

Инсайдеры, которые не смеют ходить по луже грязи в их красивой одежде,
никогда не окажутся на твёрдой земле по ту сторону.

Слово «пробовать» — особенно ценный компонент. Здесь я не согласен с
Йодой, который говорит: “Не надо пытаться. Делай или не делай”. Для
попыток есть место.

Это подразумевает, что нет никаких наказаний за провалы. Вами движет
любопытство, а не долг. Это значит, что ветер прокрастинации дует в
ваши паруса: вместо того, чтобы избегать работу, вы будете работать,
чтобы избежать другую работу.

И пока вы так поступаете, вы будете в лучшем настроении. Чем больше
работа зависит от воображения, тем больше значения имеет настроение,
поскольку у большинства людей больше идей, когда они счастливы.

Если бы я мог вернуться назад в то время, когда мне было 20+ и
пережить его снова, я бы больше занимался совместным хакингом: просто
пробовал бы вместе “взламывать” разные вещи.

Как и многие люди того возраста, я провел много времени, беспокоясь о
том, что я должен делать. Я также провёл некоторое время, пытаясь
создавать и строить вещи. Мне стоило тратить меньше времени на
беспокойство и больше на созидание.

Если вы не знаете, что делать, сделайте хоть что-то!

Рэймонд Чандлер дал совет писателям триллеров: «Когда сомневаетесь,
пишите о мужчине, который появляется в дверях с пистолетом в руке».

Он и сам следовал своему совету. Судя по его книгам, сомневался он
часто. И, хотя результат часто получался попсовым, он никогда не был
скучным.

В жизни, как и в книгах, важность действия часто остаётся
недооцененной.

К счастью, количество вещей, которые вы можете просто хакать вместе,
продолжает расти. 50 лет назад люди поражались, например, тому, что
можно взять и сделать вместе кино. Теперь вы можете совместно хакнуть
даже дистрибуцию. Просто делайте что-то и размещайте в интернете.

12. Неправильное Если вы действительно хотите сорвать куш, то лучшее,
на чём стоит фокусироваться — край окраины: территории, недавно
захваченные у инсайдеров.

Здесь вы найдёте самые сочные проекты, которые всё ещё не сделаны —
из-за того, что они показались слишком рискованными, или просто
потому, что о них было осведомлено слишком мало инсайдеров, чьих сил
не хватило на то, чтобы всё изучить.

Вот почему теперь я провожу большую часть своего времени за написанием
эссе. Раньше написание эссе существовало только для тех, кто мог их
опубликовать. В принципе, вы могли писать их и просто показывать своим
друзьям; но на практике это не работало. [6]

Эссеист нуждается в сопротивлении аудитории, как гравёр в
сопротивлении материала основы.

Еще пару лет назад написание эссе не выходило за рамки типичной
инсайдерской игры. Отраслевым экспертам дозволялось публиковать эссе в
их областях, но пул людей, которым было позволено писать на общие
темы, составлял порядка 8 человек, которые попали в правильные партии
в Нью-Йорке.

Теперь эту территорию захватила реконкиста, обнаружив, что она
практически не возделана (что неудивительно). Так много эссе ещё не
написано. Скорее всего, они окажутся самыми неформальными: инсайдеры
значительно исчерпали темы материнства и яблочных пирогов.

Это подводит меня к моему последнему предложению: методике
определения, на верном ли пути вы находитесь.

Вы на верном пути, если люди жалуются, что вы неквалифицированы, или
что вы сделали что-то неподобающее. Если люди жалуются — значит, вы
что-то делаете вместо того, чтобы сидеть, и это — первый шаг. И, если
они привели столь пустотелые виды жалоб, это значит, что вы, похоже,
сделали нечто хорошее.

Если вы что-то делаете и люди жалуются, что оно не работает — это
проблема. Но если самое худшее, что они могут вам сказать — это
напомнить о вашем статусе аутсайдера — подразумевается, что во всех
остальных отношениях вы преуспели.

Указания на недостаток квалификации так же отчаянны и безнадёжны, как
прибегание к расистским оскорблениям. Это просто законно выглядящий
способ сказать: “Такие, как вы, нам тут не нравятся”.

Но самое лучшее — это когда люди называют то, что вы делаете,
неправильным. Я слышал это слово всю свою жизнь и я только недавно
понял, что на самом деле это звук путеводного маяка. «Неправильное» —
это ноль критики. Это всего лишь прилагательное от «Мне не нравится».

Так что эта оценка, я думаю, должна быть высшей целью для маргиналов.
Быть неправильным. Когда вы слышите, как люди это говорят, вы —
золото. А они, между прочим, банкроты.

Примечания [1] Факты о ранней истории Apple взяты из интервью Джессики
Ливингстон со Стивом Возняком для книги Founders at Work

[2] Как правило, популярные образы отстают от реальности на десятки
лет. Сейчас непонятный художник — это уже не курящий сигарету за
сигаретой пьяница, изливающий душу на большие беспорядочные холсты,
при взгляде на которые обыватели говорят, что “это не искусство”,
потому что на них не изображено ничего конкретного.

Сегодняшние обыватели выучили, что всё, что висит на стенах — это
искусство. Теперь непонятый художник — это пьющий кофе
веган-карикатурист, при взгляде на работы которого они говорят “это не
искусство”, потому что оно выглядит как то, что они видели в
воскресной газете.

[3] На самом деле, это довольно точное определение политики: то, от
чего зависит рейтинг в отсутствие объективных тестов.

[4] В школе вас заставляют поверить, что всё ваше будущее зависит от
того, в какой университет вы поступите. Но, как выясняется, так вы
лишь можете купить себе пару лет. К 25 годам люди, которые стоят того,
чтобы их впечатлять, будут судить о вас больше по тому, что вы
сделали, чем по тому, в какую школу вы ходили.

[5] Менеджерам, должно быть, интересно: как можно заставить такое чудо
случиться? Как я могу заставить людей, работающих на меня, делать
больше с меньшими затратами? К сожалению, это ограничение должно быть
добровольным. Если вы должны делать больше меньшими средствами, то это
вас морят голодом, а не вы виртуозно едите.

[6] Без перспектив публикации, ближайший пункт на пути к написанию
эссе, до которого доходят люди — ведение дневника. Я обнаружил, что
никогда не вникаю в темы так глубоко, как при написании хорошего эссе.
Как видно из названия “дневник”, вам не нужно возвращаться назад и
переписывать записи дневника снова и снова на протяжении двух недель.

Примечания переводчика (1) Саскачеван (Saskatchewan) — провинция на
юге центральной части Канады в центре Северной Америки, для которой
характерны резкие перепады погоды и температур.

(2) Маргинальность (позднелатинское marginalis — находящийся на краю)
— социологическое понятие, обозначающее промежуточность,
«пограничность» положения человека между какими-либо социальными
группами. Маргиналом можно назвать человека, вышедшего из одной
социальной группы и не принятого в другую, скитающегося между
группами.

(3) “cannibalize” — маркетинговая тактика, при которой одна и та же
фирма выпускает на рынок новую марку товара той же товарной категории,
в которой у неё уже есть зарекомендовавший себя на рынке товар, тем
самым уменьшая спрос на него в пользу нового товара (обычно приводит к
росту прибылей компании)

(4) JV (Junior varsity team) — команда, за которую выступают игроки,
не являющиеся основными в проводимых соревнованиях. Она состоит из
резервистов. Основные игроки соревнований играют за команду VT
(Varsity team) — основной состав.

\end{document}
