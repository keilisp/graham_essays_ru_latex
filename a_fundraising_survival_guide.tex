\documentclass[ebook,12pt,oneside,openany]{memoir}
\usepackage[utf8x]{inputenc} \usepackage[russian]{babel}
\usepackage[papersize={90mm,120mm}, margin=2mm]{geometry}
\sloppy
\usepackage{url} \title{Руководство по выживанию в процессе поиска
  инвесторов} \author{Пол Грэм} \date{}
\begin{document}
\maketitle

Поиск источника финансирования – это вторая самая сложная часть в
создании стартапа. Самое трудное – это создавать то, что нужно людям:
многие стартапы умирают, потому что не делают это. Но вторая самая
большая причина краха этих фирм заключается, вероятно, в трудности,
связанной с поиском источника финансирования. Поиск инвестора – вещь
брутальная.


Одна из причин такой брутальности заключается в жестокости самих
рынков. Люди, которые большую часть своей жизни провели в школах или
крупных компаниях, возможно, не сталкивались с этим. Преподаватели и
боссы обычно чувствуют некоторую ответственность за вас; если вы
сделаете титаническое усилие, но все же потерпите неудачу, они простят
вам ее. Рынки меньше всего склонны к тому, чтобы прощать. Клиентов
волнует не то, насколько усердно вы работаете, а лишь то, решили ли вы
их проблемы.

Инвесторы оценивают стартапы так, как покупатели оценивают товар, а не
как боссы своих служащих. И если вы изо всех сил постараетесь и все же
потерпите неудачу, может быть, они будут инвестировать ваш следующий
стартап, но не этот.

Однако поиск инвесторов намного сложнее, чем продажа товара
покупателям, потому что их совсем немного. Это совсем не похоже на
эффективный рынок. Вряд ли наберется 10 человек, которые были бы
заинтересованы в вас; трудно взаимодействовать с большим числом. И
поскольку для этой сферы характерна случайность, отношение одного
какого-то инвестора может действительно повлиять на вас.

Проблема номер 3: инвесторы – крайне случайный фактор. Все инвесторы,
включая нас, обычно некомпетентны. Нам постоянно приходится принимать
решения относительно вещей, в которых мы не разбираемся, и мы
ошибаемся чаще, чем бываем правы.

И все же, на карту ставится много. Суммы, инвестируемые разными
инвесторами, варьируются от пяти тысяч долларов до пятидесяти
миллионов, сумма обычно кажется большой для инвестора, к какому бы
типу инвесторов он не принадлежал. Любые решения по инвестициям
являются крупными решениями.

В связи с ситуацией, когда инвесторам приходится принимать крупные
решения относительно вещей, в которых они не разбираются, они имеют
тенденцию становиться ненадежными. Венчурные инвесторы печально
известны тем, что обманывают учредителей. Некоторые из наиболее
беспринципных делают это намеренно. Но даже поведение большинства тех
инвесторов, которые руководствуются благими намерениями, порою кажется
бредовым в повседневной жизни. Сегодня они горят энтузиазмом, и,
кажется, готовы немедленно выписать вам чек; а на следующий день они
даже не ответят на ваш телефонный звонок. Они не обманывают вас. Они
просто не могут принять решение. [1]

Все было бы не так плохо, но эти сильно колеблющиеся инвесторы связаны
друг с другом. Инвесторы стартапов знают друг друга и (хотя они
неохотно в этом признаются) их мнение о вас в основном базируется на
мнении других. [2] Поговорим о способе для создания нестабильной
системы. Вы получаете из застоя нечто противоположное тому, что обычно
порождает баланс страха/жадности на рынке. Никто не заинтересован в
стартапе, это «сделка», и всем остальным она очень не по нраву.

Вот такой неэффективный рынок вы имеете, потому что очень мало игроков
возмущается тем фактом, что они практически лишены независимости. В
результате, возникает система, подобная некоему примитивному
многоклеточному морскому животному, когда вы раздражаете одну из его
конечностей, и оно все целиком сильно сжимается.

Работа Y Combinator заключается в том, чтобы отрегулировать это. Мы
пытаемся увеличить число инвесторов точно так же, как мы увеличиваем
число стартапов. Мы надеемся, что когда число и тех и других
возрастет, мы будем иметь нечто, похожее на эффективный рынок. Как t
стремится к бесконечности, так Demo Day стремится к аукционным торгам.

К сожалению, t все еще далеко от бесконечности. Чем же занимается
стартап сейчас, в этом несовершенном мире, в котором мы живем? Самое
важное — не допустить, чтобы поиски инвестора сломили ваш моральный
дух. Стартапы живут или умирают в зависимости от морального состояния.
Если вы позволите трудностям, связанным с поиском инвестора, подорвать
ваш моральный дух, это станет самореализующимся прогнозом.

Самообеспечение (= Консалтинг)

Кое-кто из потенциальных учредителей, возможно, сейчас задумался о
том, зачем вообще иметь дело с инвесторами? Если поиск инвесторов
требует таких больших усилий, для чего заниматься этим?

Один ответ на эти вопросы очевиден: потому что вам нужны деньги, чтобы
жить. В принципе, прекрасная идея финансировать стартап на доходы,
приносимые им самим, но вы не сможете в один миг создать круг
клиентов. Чем бы вы не занимались, вам необходимо продать определенное
количество вашей продукции для того, чтобы начать покрывать расходы.
Чтобы увеличить продажи до такого уровня, потребуется некоторое
количество времени, и сколько времени потребуется трудно предсказать
до тех пор, пока вы не начнете.

Мы не могли, например, иметь самообеспеченный Viaweb. У нас были
достаточно высокие цены на наши программы – около 140 долларов на
одного покупателя в месяц – но прошел почти год прежде, чем наши
доходы стали покрывать хотя бы частично наши расходы. У нас не было
достаточно накопленных денег, чтобы жить на них в течение года.

Если вы внимательно присмотритесь к «самообеспечиваемым» компаниям, то
увидите, что некоторые из них на самом деле финансировались их
учредителями посредством сбережений или стабильной работы, остальным
же или (а) действительно улыбнулась удача, что по требованию трудно
осуществить, или же (б) они начинали как консалтинговые компании и
постепенно трансформировались в продуктовые компании.

Консалтинговая компания – это единственный вариант, на который вы
можете рассчитывать. Но такая компания далека от того, чтобы иметь
свободные деньги. Это, возможно, не так мучительно, как искать деньги
у инвесторов, однако же мучение растягивается на более долгий период.
Вероятно, годы. А для многих стартапов такая задержка может оказаться
фатальной. Если вы работаете над чем-то, до чего никто больше не смог
додуматься, вы можете не спешить. Джошуа Шактер (Joshua Schachter),
работая на Уолл-стрит, в нерабочее время создал Delicious. И с этого
он начал, потому что никому не пришло в голову, что это была хорошая
идея. Но если вы создаете что-то такое явно необходимое как программы
для онлайновых магазинов в тот же период времени, что и Viaweb, и вы
работаете над этим, выкраивая дополнительное время, тогда как большую
часть своего времени тратите на работу заказчика, то вы не в очень
выгодном положении.

В принципе, самообеспечение, звучит здорово, но очевидно это
единственная благодатная почва, на которой возникают жизнеспособные
стартапы. Тот простой факт, что стартапы с самообеспечением имеют
тенденцию становится знаменитыми, должен был бы насторожить. Если это
так хорошо работает, то это должно стать нормой.

Самообеспечение, возможно, более легкий способ, потому что так дешевле
создать компанию. Но я не думаю, что мы когда-нибудь достигнем того,
что большинство стартапов смогут обходиться без внешнего
финансирования. Технологии имеют тенденцию значительно дешеветь, но
жизненные расходы дешевле не становятся.

Ну и в итоге, вы можете выбирать: кратковременное, но сильное мучение,
связанное с поиском инвестора или же хроническую муку, рождаемую
консалтинговой компанией. Если сравнить эти виды мучений, то лучшим
выбором будет поиск инвестора, потому что новая технология обычно
больше ценится сейчас, чем потом.

И хотя для многих стартапов поиск инвестора будет наименьшим злом, все
же он остается достаточно большим злом, настолько большим, что легко
может убить вас. Но не совсем в том смысле, что если вы не сможете
найти источник финансирования, то вам, возможно, придется закрыть
компанию, а в том смысле, что процесс поиска денег сам по себе может
убить вас.

Чтобы пережить его, вам нужен набор методов по большей части
ортогональных тем, что используются для убеждения инвесторов, как
скалолазам необходимо знать методы выживания, которые ортогональны
тем, что используются при физическом восхождении на горы и спуске с
них.

1. Избавляйтесь от завышенных ожиданий Причина, по которой поиск
источника финансирования разрушает моральный дух многих стартаперов не
просто в том, что это тяжелая задача, а в том, что она намного
тяжелее, чем они ожидали. Вас убивает разочарование. И чем ниже порог
ваших ожиданий, тем труднее разочаровываться.

Стартаперы имеют тенденцию быть оптимистами. Оптимизм работает для
технологии, по крайней мере, какое-то время, но это неправильный
подход к поиску источника финансирования. Лучше всего предполагать,
что инвесторы всегда могут вас подвести. Поглощающие открытые
акционерные компании тоже, если мы нацелились на таковые. В YC одна из
наших вспомогательных мантр звучит так «Сделки, как правило,
проваливаются». Не важно, что за сделка у вас в процессе, допускайте,
что она может провалиться. Прогнозирующая сила этого простого правила
поразительна.

По мере развития сделки, вы начинаете верить, что она осуществится, а
затем начинаете зависеть от мысли о ее реализации. Вы должны
противостоять этому. Держитесь изо всех сил. Потому что это именно то,
что сломит вас. У сделок нет той траектории развития, которая присуща
большинству других видов человеческих взаимодействий, где совместные
планы обретают все большую стабильность в линейной зависимости от
времени. Сделки могут не состояться в самый последний момент. Зачастую
представители другой стороны на самом деле до последнего момента не
понимают, что же они хотят. Поэтому вы не можете использовать свои
непосредственные обыденные знания в качестве руководства в отношении
совместных планов. Когда речь идет о сделках, вы должны сознательно
отключить эти знания и настроится на патологически циничное отношение.

Звучит просто, но сделать это намного труднее. Очень лестно, когда
видные инвесторы, кажется, заинтересованы финансировать вас. И легко
начать верить в то, что процесс поиска денежных средств будет быстрым
и прямолинейным. Однако так почти никогда не бывает.

2. Продолжайте работу над своим стартапом Прозвучит банально, если я
скажу, что вы должны работать над вашим стартапом во время поиска
инвестора. В реальности делать это трудно. Большинство стартаперов не
справляются с этим.

Поиски финансирования загадочным образом поглощают все ваше внимание.
Даже если у вас всего одна встреча с инвесторами в день, каким-то
непостижимым образом эта единственная встреча занимает у вас весь этот
день. Это не просто время, которое необходимо для самой встречи, но
это время нужное, чтобы добраться туда и вернуться, это время на то,
чтобы подготовится к ней, и время на то, чтобы обдумать все после нее.

Самый лучший способ, чтобы преодолеть отвлекающий фактор, связанный с
встречей с инвесторами, — это, вероятно, поделить обязанности в
компании: выбрать одного из учредителей для встречи с инвесторами, а
другие в это время будут заниматься работой над развитием компании.
Это работает лучше, когда в стартапе не 2 учредителя, а 3, а еще
лучше, когда руководитель компании не является одновременно ведущим
разработчиком. В самом лучшем случае, компания продвигается вперед в
полсилы.

Однако, это в самом лучшем случае. Гораздо чаще во время сбора средств
компания топчется на месте. И это опасно по многим причинам. Поиск
источника финансирования всегда занимает больше времени, чем вы
ожидаете. То, что, по вашему мнению, могло бы занять 2 недели перерыва
в работе, оборачивается в реальности 4 месяцами. Это может
деморализовать. И что еще хуже, это может сделать вас менее
привлекательными для инвесторов. Они хотят вкладывать деньги в
динамичные компании. Компания же, которая в течение 4 месяцев не
сделала ничего нового, не производит впечатления динамичной, и поэтому
они начинают терять к ней интерес. Инвесторы редко понимают, что,
когда они теряют интерес к стартапу, это в основном их реакция на тот
вред, который был причинен стартапу их собственной нерешительностью.

Решение: на первом месте должен быть стартап. Назначайте встречи с
инвесторами в свободное время в графике вашей работы по развитию
компании, вместо того, чтобы заниматься ее развитием в моменты между
встречами с инвесторами. Если вы будете двигать вашу компанию вперед –
выпуская новые материалы, увеличивая трафик, заключая договоры,
создавая условия, чтобы о ней писали – встречи с инвесторами станут
более продуктивными. И не просто потому, что ваш стартап будет
казаться более живым, но также потому, что это лучше для вашего
морального духа, по которому, главным образом, инвесторы оценивают
вас.

3. Будьте осторожны По мере ухудшения ситуации, оптимальной стратегией
будет осторожность. Когда все идет хорошо, вы можете рисковать; когда
все плохо, вы должны играть наверняка.

Я советую воспринимать поиск инвесторов, как процесс, который всегда
идет плохо. Причина в том, что между вашей способностью к самообману и
ужасно нестабильной природой системы, с которой вам приходится иметь
дело, находится то, что уже намного хуже, чем кажется, или же очень
легко может стать хуже.

Большинству учредителей стартапов, которых мы финансируем, я говорю,
что, если кто-то с хорошей репутацией предлагает финансировать вас на
разумных условиях, принимайте это предложение. Было несколько
стартаперов, проигнорировавших этот совет, но которые вышли из
трудного положения – это были учредители стартапов, которые отклонили
хорошее предложение в надежде получить что-то лучше, и они
действительно получили. Но в той же самой ситуации я бы снова дал тот
же самый совет. Кто знает, сколько пуль в пистолете, который
используют для игры в русскую рулетку?

Вывод: если вам кажется, что инвестор заинтересовался вами, не
позволяйте ему бездействовать. Вы не можете предполагать, что тот, кто
заинтересован в инвестировании вашей компании, будет заинтересован и
дальше. На самом деле, вы даже не можете с уверенностью сказать (и они
не могут с уверенностью сказать), что они действительно заинтересованы
до тех пор, пока вы не попробуете обратить этот интерес в деньги. Если
у вас есть прекрасный шанс, то или договаривайтесь с ними сейчас же
или оставьте их. И пока у вас нет достаточно денег, то выход один:
заключайте сделку сейчас же.

Стартапы выигрывают не тем, что получают большие инвестиционные
раунды, а тем, что производят великолепные продукты. Так что
прекращайте сбор средств и принимайтесь снова за работу.

4. Будьте гибкими Есть два вопроса, которые вам зададут венчурные
инвесторы, на которые вы не должны отвечать: «С кем вы еще говорили?»
и «Какие средства вы хотели бы собрать?»

Венчурные инвесторы не ждут от вас ответа на первый вопрос. Они
спрашивают на всякий случай. [3] Но, кажется, они действительно ждут
ответа на второй. Но я не считаю, что вы должны называть сумму. Не для
того, чтобы обмануть их, а потому что у вас нет четкого представления
о сумме, которую вам нужно собрать.

Традиция, предписывающая условие, что стартапы нуждаются в четко
обозначенной сумме, является устаревшей, идущей еще с той поры, когда
стартапы были более дорогостоящими. Очевидно, что компания, которой
нужно построить завод или нанять 50 человек, нуждается в том, чтобы
собрать определенную минимальную сумму. Но лишь немногие стартапы,
занимающиеся технологиями, находятся в настоящее время в таком
положении.

Мы рекомендуем учредителям стартапов говорить своим инвесторам, что
есть несколько путей, по которым они могут развиваться в зависимости
от того, сколько денег они могут собрать. Такая небольшая сумма как
\$50k могла бы быть достаточной для учредителей, чтобы питаться и
платить арендную плату в течение года. Пара сотен тысяч позволила бы
им приобрести офисное помещение и нанять несколько умных работников,
которых они знают со школы. Пара миллионов позволила бы им устроить
настоящий прорыв. Послание (и не только послание, но и факт) должно
быть таким: что бы ни случилось, мы достигнем успеха. Просто, если
денег будет больше, мы сделаем это быстрее.

Если вы проводите первоначальное размещение ангельских инвестиций, то
их объем может варьироваться по ходу дела. Вообще, даже лучше
первоначально заявить малый объем, а затем расширять его по мере
необходимости, чем, если бы вы изначально заявили большой объем с
риском потерять уже имеющихся инвесторов, в случае если вам не удастся
привлечь заявленные средства в полном объеме. Может быть вам даже
лучше использовать размещение с “плавающим закрытием”, когда объем
размещения заранее не определен, а вместо этого вы продаете акции
инвесторам по одной в тот момент, когда они выражают согласие. Такая
процедура помогает избежать задержек, поскольку вы можете начать, как
только первый инвестор готов совершить покупку. [4]

5. Будьте независимы Стартап, имеющий пару учредителей в возрасте 20 с
небольшим лет, может иметь такие низкие расходы, что он может быть
прибыльным при такой небольшой сумме как \$2000 в месяц. Это ничто, по
сравнению с корпоративными доходами, но эффект, производимый на ваш
моральный дух и на вашу позицию в переговорах, потрясающий. В YC мы
используем фразу «на доширак хватает» (”ramen profitable”) для
описания ситуации, когда вы зарабатываете достаточно, чтобы оплачивать
ваши жизненные расходы. Как только у вас складывается такая ситуация,
все меняется. Может, вам все еще и нужно увеличить сумму инвестиций,
но в этом месяце вы в этом не нуждаетесь.

Когда вы создаете стартап, вы не можете планировать, сколько времени
уйдет на то, чтобы он стал прибыльным. Но когда вы обнаруживаете себя
в положении, где еще одно небольшое усилие, потраченное на продажи,
приведет вас к порогу «ramen profitable», сделайте его.

Инвесторам нравится, когда вы можете себя прокормить. Это указывает на
то, что вы подумали о том, как зарабатывать деньги, а не просто
работали над техническими проблемами, вызывающими у вас изумление; это
указывает также на то, что вы достаточно дисциплинированы и
удерживаете свои расходы на низком уровне; и более того, это указывает
на то, что вы не нуждаетесь в инвесторах.

Для инвесторов нет ничего привлекательнее, чем стартап, который,
похоже, становится преуспевающим даже без их участия. Инвесторам
нравится тот факт, что они могут помочь стартапу, но им не нравятся
стартапы, которые умерли бы без их помощи.

В YC мы тратим много времени, пытаясь прогнозировать, как будут
развиваться стартапы, которые мы финансируем, потому что мы пытаемся
научиться выбирать победителей. К настоящему моменту мы проследили
траектории развития такого большого количества стартапов, что стали
лучше в наших прогнозах. И когда мы говорим о стартапах, которые, по
нашему мнению, должны преуспевать, мы обычно говорим «О, эти ребята
могут позаботиться о себе. С ними все будет в порядке». Мы не говорим
«эти ребята действительно умницы» или «эти ребята работают над
прекрасной идеей». [5] Когда мы прогнозируем хорошие результаты для
стартапов, то мотивируем это такими аргументами как жесткость,
приспособляемость, решительность. И в какой-то мере мы правы,
поскольку именно эти качества вам нужны для победы.

Инвесторы знают об этом, по крайней мере, на уровне подсознания.
Причина, по которой им нравится то, что вы не нуждаетесь в них, не
просто в том, что им нравится то, что им не принадлежит, но потому что
именно это качество делает учредителей преуспевающими.

У Сэма Альтмана (Sam Altman) это есть. Вы можете сбросить его на
парашюте на остров, кишащий каннибалами, и, вернувшись через 5 лет, вы
обнаружите, что он король. Если вы Сэм Альтам, вам не надо быть
преуспевающими, чтобы внушить инвесторам, что с ними или без них, вы
все равно добьетесь успеха. (Он не был, но он добился). Не у каждого
есть такая способность к совершению сделок, как у Сэма. У меня самого
нет. Но если и у вас нет, позвольте вашим цифрам говорить за вас.

6. Не воспринимайте отказ как что-то личное Получив отказ от
инвесторов, вы можете начать сомневаться в себе. В конце концов, у них
больше опыта, чем у вас. И если они полагают, что ваш стартап
нежизнеспособный, уж не правы ли они?

Может да, а может и нет. Надо знать, как справляться с отказом. Вы не
можете просто игнорировать отказ. Он вполне может что-то означать. Но
вам не стоит также автоматически терять присутствие духа.

Чтобы понять, что означает отказ, сначала вы должны понять, насколько
это распространенная вещь. По статистике, средний венчурный инвестор –
это устройство для отказов. Девид Хорник (Davod Hornik), партнер
August, сказал мне следующее:

«Для меня эти цифры таковы: получено и прочитано от 500 до 800 планов,
проведено первоначальных одночасовых встреч от 50 до 100, около 20
компаний меня заинтересовало, и в год заключается от 1 до 2 сделок.
Так что, у вас нет шансов. Вы можете быть прекрасным предпринимателем,
работающим над интересными вещами, и т.д., и все равно маловероятно,
что вы получите финансирование».

Это меньше распространяется на «инвесторов-ангелов», но венчурные
инвесторы отказывают практически каждому. Структура их бизнеса такова,
что партнер делает не более 2 инвестиций в год, не важно, сколько
основателей хороших стартапов обращаются к нему.

В дополнение к отсутствию каких бы то ни было шансов, как я уже
упоминал, средний инвестор мало что понимает в стартапах. Судить о
стартапах труднее, чем о чем-то другом, потому что лучшие идеи
стратапов имеют тенденцию не вызывать доверие. Хорошая идея стартапа
должна быть не просто хорошей, она должна быть новой. И то, что она и
хорошая, и новая, большинству людей, возможно, кажется плохим, или же
кто-то уже занимается ею, и потому она уже не совсем новая.

Поэтому о стартапах судить труднее, чем о многом другом. Вы должны
быть своего рода интеллектуальной «белой вороной», чтобы стать
инвестором хорошего стартапа. В этом проблема для венчурных
инвесторов, большинство из которых не блещут богатым воображением. [6]
Ангелы в большей мере ценят оригинальные идеи, потому что большинство
из них сами являются учредителями.

Итак, когда вы получаете отказ, используйте данные, которые заложены в
нем, а не те, которых в нем нет. Если инвестор конкретизирует причины,
по которым отказывается инвестировать ваш стартап, внимательно
присмотритесь к своему стартапу и спросите себя, обоснованы ли они.
Если они представляют реальную проблему, устраните их. Но не надо
просто верить на слово. Предполагается, что вы эксперт в этой области;
так что вам и решать.

Впрочем, необязательно, что отказ скажет вам что-то о вашем стартапе,
он может означать, что есть что-то, что можно улучшить. Выясните, что
не работает и измените это. Не надо думать, что «инвесторы глупы».
Зачастую они не разбираются, но вы должны точно выяснить, что именно
отталкивает их.

Не позволяйте отказам копиться в виде удручающей неопределенной кучи.
И вместо того, чтобы думать «мы никому не нравимся», сортируйте и
анализируйте их, и затем вы будете точно знать какая проблема у вас
самая большая и что делать, чтобы устранить ее.

7. Будьте готовы трансформироваться в консалтинговую компанию (если
сложилась соответствующая ситуация) Консалтинг, как я уже говорил,
опасный путь для финансирования стартапа. Но это лучше, чем умереть.
Это немного напоминает анаэробное дыхание: не является оптимальным
решением на долгий период, но может спасти вас от непосредственной
угрозы. Если вы вообще не способны заниматься сбором средств,
привлекая инвесторов, то способность трансформироваться в консалтинг
может спасти вас.

Для одних стартапов это работает лучше, чем для других. Для Google,
скажем, это было бы не совсем естественно, но если ваша компания
делала программы для создания веб-сайтов, вы достаточно грациозно
могли бы деградировать до консалтинга, создавая сайты для заказчиков с
помощью своих программ.

Пока вы будете осторожны и не утонете навечно в консалтинге, такой
выход будет даже выгодным для вас. Вы бы лучше понимали своих
пользователей, если вы бы использовали свои программы для них. Плюс к
этому, в качестве консалтинговой компании, вы могли бы заполучить
широко известных пользователей, использующих ваши программы, которых
вы не заполучили бы, будучи продуктовой компанией.

В Viaweb первоначально мы были вынуждены работать как консалтинговая
компания, поскольку нам настолько отчаянно нужны были пользователи,
что мы предложили бы создать сайты торговцев для них, если б они
подписались на это. Но мы за это не брали с них денег, потому что мы
не хотели, чтобы они начали обращаться с нами как с настоящими
консультантами, и вызывали бы нас каждый раз, когда им хотелось что-то
изменить на их сайте. Мы знали, что должны оставаться продуктовой
компанией, потому что только это имеет значение.

8. Избегайте неопытных инвесторов Хотя начинающие инвесторы, как
кажется, не таят в себе никакой угрозы, они могут оказаться самыми
опасными, потому что они сильно нервничают. Особенно в соотношении с
суммой, инвестированной ими. Получить сумму \$20,000 у начинающего
ангела может оказаться намного более сложной задачей, чем получить \$2
миллиона из фонда венчурного инвестора.

Их юристы в целом также неопытные. Но если инвесторы могут допустить,
что они чего-то не знают, то их юристы – нет. Один из YC стартапов
обговаривал условия с ангелом для небольшого раунда, чтобы только
получить договор на 70 страницах от его юриста. И поскольку юрист
никогда бы не признался перед своим клиентом, что он облажался, он
вынужден был настаивать на сохранении драконовских условий в договоре;
таким образом, сделка провалилась.

Конечно, кто-то должен брать деньги и у инвесторов-новичков, иначе
опытные никогда не появятся. Но если вы это делаете, то или (а) сами
продвигайте процесс, включая оформление документации или (б)
используйте их только для пополнения более крупного раунда под чьим-то
еще руководством.

9. Оценивайте свои позиции Самая большая опасность, связанная с
инвесторами – это их неуверенность. Самый худший сценарий ситуации –
это медлительность, из-за которой вы получаете отказ после
многочисленных встреч на протяжении нескольких месяцев. Отказы
инвесторов подобны недостаткам в дизайне: они неизбежны, но обходятся
вам дешевле, если вы быстро их обнаруживаете.

Таким образом, ведя переговоры с инвесторами, постоянно оценивайте
свои позиции. Насколько реально, что они предложат вам предварительное
соглашение? В чем, прежде всего, их необходимо убедить? Вам не надо
задавать эти вопросы прямо – это может вызвать раздражение – но вы
постоянно должны собирать о них информацию.

Инвесторы имеют тенденцию избегать связывать себя обязательствами,
хотя бы до того момента, когда вы подтолкнете их к этому. В их
интересах собрать максимум информации на стадии, когда принимается
минимальное число решений. Самое лучшее из всего, что может заставить
их действовать – это, конечно же, конкурирующие инвесторы. Но вы тоже
можете приложить некоторое усилие, сфокусировав обсуждение:
поинтересоваться, на какие конкретные вопросы они хотели бы получить
ответы, чтобы принять решение, и затем ответить на эти вопросы. Если
вы преодолеваете препятствие за препятствием, а они продолжают
возводить новые, делайте вывод, что, в конце концов, они собираются
уклониться.

Вы должны быть аккуратны в сборе информации о намерениях инвесторов.
Иначе их желание ввести вас в заблуждение соединится с вашим
собственным желанием впасть в заблуждение, и в итоге вы получите
совершенно неверные представления.

Используйте информацию для оценки вашей стратегии. Вам, вероятно,
следует провести переговоры с несколькими инвесторами. Сосредоточьтесь
на тех, кто, скорее всего, согласится. Оценивать потенциального
инвестора надо на основании двух факторов: насколько это хорошо, если
они согласятся, и какова вероятность, что они согласятся. В основном
сосредоточьтесь на втором факторе. Частично потому что самым важным
качеством инвестора является инвестирование. А еще потому что, как я
уже говорил, самую большую часть мнения инвесторов о вас составляют
мнения других инвесторов. Если вы ведете переговоры с несколькими
инвесторами, и вам удалось одного из них склонить к согласию, другие
проявят намного больше заинтересованности в вас. Итак,
сосредоточиваясь на тех, кто горит желанием, вы не жертвуете теми, кто
индифферентен; убедить заинтересованных инвесторов – самый лучший
способ убедить тех, кто не горит желанием.

Будущее Я полон надежды, что ситуация не будет такой затруднительной
всегда. Я надеюсь, что когда стартапы станут дешевле и число
инвесторов увеличится, сбор средств станет пусть не легким, но, по
крайней мере, прямолинейным.

А пока, неравномерность процесса инвестирования предлагает большие
возможности. Большинство инвесторов не имеют представления, насколько
они опасны. Они бы сильно удивились, если бы услышали, что попытка
получить у них финансирование является чем-то, к чему надо относиться
как к угрозе для жизни компании. Они просто думают, что им надо
немного больше информации для принятия решения. Они не осознают, что
есть еще 10 других инвесторов, которым также нужно еще немного
информации, и что процесс переговоров с ними, затягивающийся на
месяцы, тормозит работу стартапа.

Поскольку инвесторы не понимают, сколько стоит иметь с ними дело, они
не осознают, какое пространство открывается для потенциального
конкурента, чтобы подсечь их. Я из собственного опыта знаю, насколько
быстрее инвесторы могли бы принимать решения, потому что мы снизили
это время до 20 минут (5 минут на чтение заявки плюс 10 минут на
интервью плюс 5 минут на обсуждение). Если бы вы инвестировали больше
денег, вы бы хотели, чтобы времени было больше, конечно. Но если мы
можем решить за 20 минут, неужели кому-то надо больше пары дней?

Такие возможности не остаются не использованными веками даже в такой
реакционной сфере, как венчурный капитал. Итак, или существующие ныне
инвесторы начнут быстрее принимать решения, или появятся новые
инвесторы, которые станут это делать.

Ну, а пока учредители должны относиться к сбору средств как к опасному
процессу. К счастью, я прямо здесь могу обозначить самую большую
опасность. Это удивление. Стартаперы преуменьшают трудность поиска
инвестора, они проходят все начальные стадии, и когда они начинают
сбор средств, они с удивлением обнаруживают, что это трудно, теряют
присутствие духа и сдаются. Поэтому я говорю вам сразу: искать
инвесторов трудно.

Примечания [1] Когда инвесторы не могут принять решение, они описывают
это иногда как свойство самих стартапов. «Вы слишком рано появились»,
— говорят он иногда. Но если бы их на машине времени перенести в тот
момент, когда был создан Google, кто из них не предложил бы
инвестировать любую сумму, предложенную учредителями? Часом позже –
это не слишком рано, если это стоящий стартап. «Слишком рано» означает
всего лишь, что «мы не можем пока вычислить, будете ли вы иметь
успех».

[2] Инвесторы влияют друг на друга, как прямо, так и опосредованно.
Прямое влияние оказывается посредством «шума» вокруг только что
возникшего стартапа. Но также они влияют друг на друга опосредованно,
через учредителей. Когда многие инвесторы заинтересованы в вас, это
увеличивает вашу уверенность в себе и делает вас намного
привлекательнее для инвесторов.

Ни один венчурный капиталист не признает, что на его мнение может
повлиять “шум”. И некоторые из них искренни в этом заявлении. Но тех,
кто может сказать, что на него не влияет уверенность в себе, мало.

[3] Оптимальный путь для ответа на первый вопрос – это сказать, что
было бы некорректно называть имена, одновременно подразумевая, что вы
ведете переговоры с группой других венчурных инвесторов, которые на
грани того, чтобы заключить с вами предварительное соглашение. Если вы
принадлежите к тому типу людей, кто знает, как это делать, то делайте.
Если же нет, то даже не пытайтесь. Ничто так не раздражает венчурных
капиталистов, как неуклюжие попытки манипулировать ими.

[4] Недостаток увеличения инвестирования по ходу дела в том, что
оценка фиксируется на старте, поэтому, если вы вдруг испытаете
внезапный всплеск интереса, вам, возможно, придется решать между
отказом некоторым инвесторам и продажей большей части компании, чем вы
намеревались. Однако иметь такую проблему неплохо.

[5] Я бы не сказал, что умственные способности не имеют значения для
стартапов. Мы просто сравниваем со стартаперами YC, у которых они уже
выше некоторого порога.

[6] Но не все. Хотя многие венчурные инвесторы и похожи в глубине
души, наиболее успешные имеют тенденцию отличаться. Немного странно
то, что самые лучшие из венчурных капиталистов очень мало похожи на
таковых.



\end{document}
