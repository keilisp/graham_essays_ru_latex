\documentclass[ebook,12pt,oneside,openany]{memoir}
\usepackage[utf8x]{inputenc} \usepackage[russian]{babel}
\usepackage[papersize={90mm,120mm}, margin=2mm]{geometry}
\sloppy
\usepackage{url} \title{Как стать кремниевой долиной} \author{Пол
  Грэм} \date{}
\begin{document}
\maketitle

Можно ли создать еще одну кремниевую долину, или она может быть одна и
только одна?

Неудивительно, что создать кремниевую долину в других странах должно
быть непросто, потому что даже в США это возможно далеко не везде. Но
что нужно, чтобы построить кремниевую долину даже в США?

Необходимы правильные люди. Если вы сможете перевезти десять тысяч
правильных людей из Кремниевой Долины в Буффало, то Буффало станет
Кремниевой Долиной [1]. В этом заключается принципиальное отличие от
прошлого. Еще несколько десятилетий назад местоположение города
полностью определялось географией. Все великие города основаны на
водных путях, потому что города зарабатывали торговлей, а торговать
можно было только по воде. Сейчас вы можете создать великий город где
угодно, если сможете переселить туда правильных людей. Поэтому вопрос
о создании кремниевой долины заключается в том, кто такие правильные
люди и как сделать так, чтобы они переехали. Два типа людей Я считаю,
что необходимо лишь два типа людей, чтобы создать технологический
центр: богатые люди и ботаники. Они являются ключевыми компонентами в
реакции в которой рождаются стартапы, потому что только они
присутствуют при их рождении. Все остальные переедут позже.
Невооруженным глазом видно, что в США города становятся
технологическими центрами тогда и только тогда, когда есть и
обеспеченные люди, и ботаники. Случается, например, что некоторые
стартапы рождаются в Майями, потому что там живет куча обеспеченных
людей, но там почти нет ботаников. Ботаники не любят подобные места. В
то же время у Питтсбурга противоположная проблема: куча ботаников, но
нет богатых людей. Известно, что лучшие американские факультеты
компьютерных наук - в Массачусетском технологическом институте (MIT),
Стэнфорде, Беркли и Карнеги-Меллоне. Из MIT получилась Route 128. Из
Стэнфорда и Беркли - Кремниевая долина. А Карнеги-Меллон? Здесь у нас
пробел. Далее по списку, унивеситет Вашингтона способствовал созданию
высокотехнологичного сообщества в Сиэтле, а университет Техаса в
Остине сделал то же самое в Остине. Но что случилось в Питтсбурге? А в
Итаке, где расположен Корнелл, который тоже есть в этом списке? Я
вырос в Питтсбурге и учился в колледже Корнелла, поэтому я знаю ответ
на оба вопроса. Погода ужасна, особенно зимой, и в городе нет ничего,
что могло бы это компенсировать, как в Бостоне. Богатые люди не хотят
жить в Питтсбурге или в Итаке. Так что хотя там и есть куча ботаников,
которые могли бы основать стартап, в них некому инвестировать. Не
чиновники Действительно ли для этого необходимы обеспеченные люди? А
разве правительство не может инвестировать в ботаников? Нет, не может.
Люди, инвестирующие в стартапы, - это особенный вид богатых людей. У
них у самих достаточно опыта в высокотехнологичном бизнесе. Это,
во-первых, помогает им выбирать правильные стартапы, а во-вторых,
означает, что они могут помочь не только деньгами, но также советами и
связями. То, что они лично заинтересованы в результате, заставляет их
очень внимательно относиться к вопросу. Чиновники по своей природе
являются антиподами людей, которые инвестируют в стартапы. Сама идея
таких инвестиций кажется им смешной. Это как если бы математики
издавали Vogue или, точнее, как если бы издатели Vogue взялись за
математический журнал. [2] И в самом деле, большую часть вещей,
которые делают бюрократы, они делают плохо. Просто мы обычно этого не
замечаем, потому что единственные их конкуренты - это другие
бюрократы. Однако в роли инвесторов стартапов им бы пришлось
соревноваться с профессионалами, обладающими намного большим опытом и
мотивацией. Даже корпорации, имеющие в своем составе венчурные группы,
обычно запрещают им принимать независимые инвестиционные решения.
Большинству разрешается лишь инвестировать на пару с респектабельными
частными венчурными фондами, которые выступают в роли основных
инвесторов. Не здания Если вы поедете в Кремниевую Долину, то вы
увидите лишь здания. Но Долина состоит не из зданий, а из людей. Я
как-то читал про попытки основать технологические парки в других
местах, как если бы основой Кремниевой Долины были офисные здания.
Статья про Sophia Antipolis хвасталась, что среди компаний там были
Cisco, Compaq, IBM, NCR и Nortel. Французы, что, не понимают, что эти
компании - не стартапы? Из офисных комплексов для высокотехнологичных
компаний не получится кремниевой долины, потому что ключевой момент в
жизни стартапов наступает еще до того, как им потребуются офисы.
Ключевой момент - это когда три парня начинают работать вне своего
дома. Там, где стартап получит финансирование, он и останется.
Основное достоинство Кремниевой Долины не в том, что там располагаются
офисы Intel, Apple или Google, а в том, что они были там основаны. Так
что если вы хотите, чтобы у вас была новая Кремниевая Долина, то у вас
должны быть два или три парня, которые сидя за кухонным столом решают
основать стартап. И для этого вам потребуются такие люди. Университеты
Хорошая новость заключается в том, что все, что необходимо - это люди.
Если вы сможете привлечь критическую массу ботаников и инвесторов в то
или иное место, вы сможете создать вторую Кремнивую Долину. Обе эти
группы очень мобильны, поэтому они переедут туда, где хорошо живется.
А где им хорошо живется? Ботаники любят других ботаников. Умные люди
тянутся к другим умным людям и, в частности, в великие унивеситеты.
Теоретически, могут быть и другие способы привлечь их, но университеты
пока являются незаменимыми. В США технологические центры не существуют
вне университетов или, по крайней мере, первоклассных факультетов
компьютерных наук. Поэтому если вы хотите создать вторую кремниевую
долину, то вам необходим только университет, но один из лучших в мире.
Он должен быть достаточно хорош, чтобы притягивать лучших за тысячи
километров. Это значит, что сила его притяжения должна быть сравнима с
притяжением MIT и Стэнфорда. Это выглядит довольно сложно, хотя на
самом деле может оказаться совсем простым делом. Мои друзья из
профессуры, выбирая новое место работы, ориентируются прежде всего на
уровень будущих коллег. Профессоров привлекают хорошие коллеги.
Поэтому если вы сможете разом набрать значительное количество
выдающихся молодых исследователей, вы можете создать первоклассный
университет на пустом месте. Для этого вам потребуется на удивление
мало денег. Если вы единоразово заплатите при приеме на работу 200
людям бонус в размере \$3 000 000, то вы получите коллектив, который
сможет тягаться с лучшей мировой профессурой. Начиная с этого момента
цепная реакция должна стать самоподдерживающейся. Поэтому сколько бы
ни стоило создать посредственный университет, за дополнительные
полмиллиарда или около того вы получите выдающийся университет. [3]
Индивидуальность Однако, просто выдающегося университета недостаточно,
чтобы создать кремниевую долину. Университет - это всего лишь зерно.
Оно должно быть посажено в подходящую почву, иначе оно не прорастет.
Посадите его в неподходящем месте и вы получите Карнеги-Меллон. Чтобы
стартапы начали появляться как грибы после дождя, университет должен
быть расположен в городе, который имеет другие достоинства, кроме
университета. Это должно быть место, где инвесторам хочется жить, а
студентам - остаться после выпуска. И инвесторы, и студенты любят
примерно одно и то же, потому что большинство людей, инвестирующих в
стартапы, сами ботаники. Так что же нужно ботанику от города? Их вкусы
не сильно отличаются от вкусов других людей, потому что многие города,
которые им нравятся, также нравятся и туристам: Сан-Франциско, Бостон,
Сиэтл. Но их вкусы все же не совпадают со вкусами большинства, потому
что в других местах, почитаемых туристами, таких как Нью-Йорк, Лос
Анжелес или Лас Вегас, ботаникам жить не хочется. В последнее время
много было написано про "творческий класс". Идея, кажется, заключается
в том, что раз богатство все больше обретается благодаря идеям, города
будут процветать только если они смогут привлечь тех, у кого есть эти
идеи. Разумеется, это так: это было, в самом деле, основой процветания
Амстердама 400 лет назад. Вкусы ботаников во многом совпадают со
вкусами творческого класса. Например, они любят хорошо сохранившиеся
старые районы, а не однотипные новостройки на окраине; также им
нравятся местные частные магазины и рестораны, а не национальные сети.
Как и другие представители творческого класса, они хотят жить в месте,
обладающем индивидуальностью. Но что такое индивидуальность? Я думаю,
что это ощущение, что каждое знание было создано отдельным коллективом
людей. Город, обладающий этой индивидуальностью не производит
впечатление только сошедшего с конвейера. Поэтому если вы хотите
создать центр для стартапов или, в общем, город, который бы привлекал
творческий класс, вы должны запретить масштабные стройки. Когда
большой участок застраивается одной организацией, это сразу видно. [4]
Большинство городов, обладающих индивидуальностью достаточно старые,
но это не обязательно. У старых городов есть два преимущества: они
плотнее застроены, потому что были спланированы до изобретения машин,
а также они разнообразнее, потому что они строились по зданию за раз.
И сейчас можно сделать то же самое, надо просто законодательно
обеспечить плотную застройку и запретить масштабные проекты. Как
следствие, необходимо не позволять вмешиваться самому большому
застройщику: правительству. Правительство, которое спрашивает "Как мы
можем построить кремниевую долину?", вероятно постигнет неудача только
из-за того, как оно формулирует вопрос. Вы не строите кремниевую
долину, вы способствуете ее росту. Ботаники Если вы хотите привлечь
ботаников, вам потребуется больше, чем просто город, обладающий
индивидуальностью. Нужен город, обладающий правильной
индивидуальностью. Ботаники - это особая часть творческого класса, с
особыми вкусами. Это очень хорошо заметно в Нью-Йорке, который
привлекает огромное количество творческих людей, но мало ботаников.
[5] Что нравится ботаникам, так это города, где люди ходят и
улыбаются. Сюда не относится Лос Анжелес, где никто вообще не ходит, а
также Нью-Йорк, где люди ходят, но не улыбаются. Когда я получал
последипломное образование в Бостоне, ко мне приехала знакомая из
Нью-Йорка. Когда мы ехали в аэропорт, она спросила, почему все
улыбаются. Я посмотрел и увидел, что никто не улыбается. Просто если
сравнить с теми выражениями лиц, которые она привыкла видеть, можно
было подумать, что они и правда улыбаются. Если вы жили в Нью-Йорке,
то вы знаете, откуда эти выражения на лицах. Это то место, где духу
может быть хорошо, а телу - очень плохо. Людям не столько нравится
жить там, сколько им приходится терпеть ради развлечений. А если вы
падки на определенные виды развлечений, то Нью-Йорк вне конкуренции.
Это центр гламура, магнит для всех тех, кто гонится за коротким
периодом расцвета, стилем и славой. Ботаников не заботит гламур,
поэтому притягательность Нью-Йорка для них - загадка. Люди, которым
нравится Нью-Йорк выложат состояние за маленькую, темную, шумную
квартиру, чтобы жить в городе с реально клевыми людьми. Ботаник видит
в этом только одну сторону: заплатить состояние за маленькую, темную,
шумную квартиру. Ботаники согласны переплачивать за то, чтобы жить в
городе с действительно умными людьми, но за это не надо так много
платить. Это закон спроса и предложения: гламур востребован, поэтому
приходится раскошеливаться. Большинство ботаников предпочитает более
тихие удовольствия. Им нравятся кафе, а не клубы, книжные магазины, а
не бутики, прогулки, а не дискотеки, солнечный свет, а не небоскребы.
Рай для ботаников - это Беркли или Боулдер. Молодежь Стартапы
создаются молодыми ботаниками, поэтому именно на них должен
ориентироваться город. Дух молодости чувствуется во всех американских
городах, в которых рождаются стартапы. Это не значит, что города
должны быть новые. План города Кэмбриджа - один из старейших в
Америке, но там чувствуется дух молодости, потому что он полон
студентов. Чего быть не должно, если вы создаете кремниевую долину,
так это огромного количества уже проживающего там инертного населения.
Попытка развернуть развитие угасающего индустриального города,
например, Детройта или Филадельфии, путем стимулирования развития
стартапов, обречена на провал. Эти города слишком долго двигались в
неверном направлении. Гораздо лучше начать с чистого листа в маленьком
городе. Или, еще лучше, в городе, в который уже стекается молодежь.
Район залива Сан-Франциско притягивал молодых и оптимистичных людей на
протяжении десятилетий до того, как он начал ассоциироваться с
высокими технологиями. Это было место, куда приезжали в поисках
нового. Также, Залив стал синонимом калифорнийского сумасшествия, это
сохраняется и по сей день. Если вы хотите сделать что-то модным,
например, распространить новый способ фокусирования чьей-то "энергии"
или "открыть" новую категорию продуктов, которые не стоит есть, то
Залив будет прекрасным местом. (Сколько людей захотят иметь компьютер
дома? Что, еще один поисковик?) В этом заключается связь между
технологиями и либерализмом. Все без исключения высокотехнологичные
города в США также являются наиболее либеральными. Однако это
происходит не потому, что либералы умнее, а потому что либеральные
города терпимо относятся к необычным идеям, а у умных людей по
определению появляются только такие идеи. Точно так же, город, который
хвалят за "устойчивость к новомодным веяниям" или который поддерживает
"традиционные ценности", может быть прекрасным городом для жизни, но
он никогда не станет колыбелью для стартапов. В ходе президентских
выборов 2004 г., хотя они и привели к катастрофичным результатам в
ином отношении, обрисовалась карта таких мест по округам. [6] Чтобы
привлекать молодежь, центр города должен быть нетронут. В большинстве
американских городов центр заброшен и рост города, если таковой вообще
наблюдается, происходит за счет окраин. Большинство американских
городов просто вывернуты наизнанку, но таких городов нет среди тех, в
которых рождаются стартапы: Сан Франциско, Бостон, Сиэтл. У них у всех
нетронутый центр. [7] Мне кажется, что в городе с мертвым центром
стартапы не могут рождаться. Молодежь не хочет жить на окраине. В США
есть два города, которые, по моему мнению, имеют наибольшие шансы
превратиться в новые кремниевые долины: Боулдер и Портлэнд. В обоих
городах кипит та жизнь, которая привлекает молодежь. Если эти города
хотят стать кремниевыми долинами, то все, что им осталось сделать, -
это создать великий университет. Но место, в котором можно быть
чудаковатым в поисках нового - это как раз то, что вы хотите увидеть в
центре стартапов. Потому что экономически именно это и есть стартапы.
Большинство хороших стартап-идей казались немного сумасшедшими - если
бы они были очевидно хорошими, кто-нибудь бы их уже реализовал. Время
Великий университет рядом с привлекательным городом. Это все, что
нужно? Этого хватило, чтобы создать Кремниевую Долину. Ее история
приводит нас к Уильяму Шокли (William Shockley), одному из
изобретателей транзистора. Его исследования в Bell Labs принесли ему
Нобелевскую премию, но когда он захотел основать собственную компанию
в 1956 г., то он переехал для этого в Пало-Альто (Palo Alto). Тогда
это казалось необычным. Но почему же он это сделал? Потому что он там
вырос и он помнил, как там хорошо. Сейчас Пало-Альто - это пригород,
но тогда это был очаровательный университетский городок с прекрасной
погодой и всего часом езды до Сан-Франциско. Все компании, которые
заправляют в Кремниевой Долине, тем или иным способом вышли из
Shockley Semiconductor. Шокли был тяжелым в общении человеком и в 1957
г. его команда - "предательская восьмерка" - оставила его, чтобы
создать новую компанию, Fairchild Semiconductor. Среди них были Гордон
Мур (Gordon Moore) и Роберт Нойс (Robert Noyce), которые позже
основали Intel и Юджин Кляйнер (Eugene Kleiner), который основал
венчурный фонд Kleiner Perkins. Сорок два года спустя Kleiner Perkins
инвестировал в Google, а партнером, контролирующим сделку, был Джон
Дорр (John Doerr), который приехал в Кремниевую Долину в 1974 г.,
чтобы работать в компании Intel. Так что хотя многие из новых компаний
Кремниевой Долины ничего не делают из кремния, их корни уходят к
Шокли. И в этом есть урок: стартапы рождают стартапы. Люди, которые
работают в стартапах, потом основывают уже свои стартапы. Люди,
которые обогащаются на стартапах, потом инвестируют в новые стартапы.
Я подозреваю, что это единственный естественный способ создать центр,
в котором бы рождались стартапы, потому что это единственный способ
вырастить тот человеческий капитал, который необходим. Из этого
следует два важных вывода. Во-первых, необходимо время, чтобы
вырастить кремниевую долину. Университет можно построить за пару лет,
но сообщество людей, создающих стартапы, должно вырости естественным
путем. Временной цикл ограничен временем, необходимым для создания
успешной компании, т.е. примерно пятью годами. Во-вторых, гипотеза
естественного роста подразумевает, что не может быть какого-то «вроде
бы подходящего» места для стартапов. У вас или идет
самоподдерживающаяся реакция, или нет. Это видно невооруженным глазом:
города или являются пригодными для стартапов, или нет, третьего не
дано. Чикаго - это третий по размерам мегаполис в США, но как колыбель
для стартапов он сильно проигрывает Сиэтлу, который на пятнадцатом
месте.

Хорошие новости заключаются в том, что для начала требуется немного.
Shockley Semiconductor, пусть это была и не самая успешная компания,
была достаточно большой. Она привлекла критическую массу экспертов в
области важных новейших технологий в место, которое им настолько
понравилось, что они остались.

Конкуренция

Конечно, претенденты на звание кремниевой долины сталкиваются с
проблемой, которой не было у первой Кремниевой Долины: они должны с
ней конкурировать. Возможно ли это? Может быть.

Одно из самых больших преимущества Кремниевой Долины - это ее
венчурные фонды. Это не играло важной роли во времена компании
Shockley, потому что тогда венчурных фондов не было. На самом деле,
Shockley Semiconductor и Fairchild Semiconductor не были стартапами в
современном понимании этого слова. Они были дочерними компаниями
Beckman Instruments и Fairchild Camera and Instrument, соответственно.
По-видимому, эти компании решили создать дочерние компании там, где
экспертам хотелось жить.

Венчурные инвесторы, однако, предпочитают финансировать стартапы,
которые находятся в часе езды. В частности, стартап, расположенный
неподалеку, проще заметить. Но даже когда они находят стартапы в
других городах, они предпочитают, чтобы стартапы переехали поближе.
Они не хотят путешествовать, чтобы поучаствовать в заседаниях и в
любом случае шансы на успех выше в тех местах, где стартапы
традиционно развиваются.

Централизующее свойство венчурных фирм имеет два положительных
эффекта: венчурный капитал притягивает стартапы, а те, в свою очередь,
притягивают еще больше стартапов путем поглощений. И хотя первый
эффект представляется незначительным, потому что основать компанию
стоит сегодня очень дешево, второй эффект силен как никогда. Три из
наиболее впечатляющих компаний эпохи Web 2.0 были основаны вне мест,
где обычно появляются стартапы, но две из них уже были поглощены
другими компаниями.

Такие централизующие силы затрудняют появление новых кремниевых долин,
но ни в коем случае не делают это невозможным. В конце концов все
зависит от основателей. Стартап с лучшими основателями будет лучше
стартапа, который получит финансирование от известных венчурных
капиталистов, а стартап, который успешно начал работу, никогда не
будет переезжать. Поэтому город, который сможет удержать правильных
людей, сможет противостоять Кремниевой Долине или даже превзойти ее.

Несмотря на всю ее мощь, Кремниевая Долина имеет одно слабое звено:
тот рай, который Шокли основал в 1956 г. сейчас представляет из себя
гигансткую парковку. Сан-Франциско и Беркли прекрасны, но они
расположены в 40 милях друг от друга. Кремниевая Долина - это по сути
разрастающийся пригород, перемалывающий людские души. Там шикарная
погода, которая дает ему преимущество перед другими американскими
пригородами, перемалывающими людские души. Но конкурент, который
сможет избежать такого расползания, будет иметь настоящее
преимущество. Все, что нужно городу - это быть таким, чтобы следующая
"предательская восьмерка" приехала и сказала: "Я хочу здесь остаться"
и этого будет достаточно, чтобы цепная реакция началась.



Примечания

[1] Интересно отметить, насколько мало может быть это число. Я
подозреваю, что пяти сотен было бы достаточно, даже если бы они
приехали без вещей. Может быть хватило бы даже тридцати, если бы я сам
выбирал их, чтобы превратить Буффало в настоящую колыбель для
стартапов.

[2] Чиновники более-менее прилично справляются с выделением
финансирования на исследования, но только потому, что как и дочерний
венчурный фонд, они отдают большую часть работы по выбору
специалистам. Известный среди своих коллег профессор из знаменитого
университета получит финансирование, по большому счету, вне
зависимости от того, подо что он просит деньги. Для стартапов эта
схема не работает, т.к. их основатели не получают денег от организаций
и зачастую неизвестны широкой публике.

[3] Это все надо делать за один раз или хотя бы по одному факультету
за раз, потому что люди склонны приезжать туда, где уже живут их
друзья. Начать будет проще с чистого листа, чем с преобразования
существующего университета, иначе придется потратить много сил на
преодоление разногласий.

[4] Гипотеза: любой план, подразумевающий значительную реконструкцию
или полное разрушение множества невзаимосвязанных зданий в рамках
одного проекта - это чистые потери в индивидуальности города, за
исключением случая преобразования знаний, которые ранее не занимали
люди, например, складов.

[5] Несколько стартапов были основаны в Нью-Йорке, но в расчете на
одного жителя их на порядок меньше, чем в Бостоне, кроме того,
большинство из них работает в областях, не свойственных ботаникам,
например в области финансов или СМИ.

[6] Некоторые консервативные округа не подчиняются этому правилу
(отражая оставшуюся мощь аппарата демократической партии), но нет
демократических округов, которые бы не подчинялись ему. Можете смело
вычеркнуть все республиканские округа.

[7] Некоторые эксперты по "обновлению городов" пытались уничтожить
Бостон в начале 1960-ых, превратив пространство вокруг здания
городской мэрии в унылый пустырь, но большинство прилегающих округов
смогло справиться с этим.

\end{document}
