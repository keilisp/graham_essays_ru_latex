\documentclass[ebook,12pt,oneside,openany]{memoir}
\usepackage[utf8x]{inputenc} \usepackage[russian]{babel}
\usepackage[papersize={90mm,120mm}, margin=2mm]{geometry}
\sloppy
\usepackage{url} \title{Что бизнес мог бы взять от свободного ПО}
\author{Пол Грэм} \date{}
\begin{document}
\maketitle

В последнее время компании обращают все больше внимания на свободное
ПО. Десять лет назад угроза распространения монополии компании
Microsoft на рынок серверного программного обеспечения казалась
реальной. Теперь, похоже, можно сказать, что свободное ПО
предотвратило такое развитие событий. Недавние опросы показывают, что
52\% компаний мигрируют с серверов, использующих Windows, на Linux.

Особенно важно, что это за компании входят в эти 52 процента. В наше
время человеку, который предлагает использовать Windows на серверах,
придется объяснить, что такого он знает о серверах, чего не знают в
компаниях Google, Yahoo и Amazon.

Однако самое важное, что должен осознать бизнес, касается не Linux или
Firefox, а сил, которые привели к созданию этого ПО. В конце концов,
они повлияют далеко не только на то, какими программами мы пользуемся.

Мы сможем разобраться в силах, движущих свободное ПО, отталкиваясь от
феноменов программ с открытыми исходниками и блогов [сетевых
дневников]. Вы могли заметить, что у тех и других много общего.

Люди ведут сетевые дневники также как и программисты, пишущие
свободные программы, бесплатно, для развлечения. Как и создатели
свободного ПО, блоггеры конкурируют с людьми, работающими за деньги, и
часто выигрывают. Метод определения лучшего все тот же: естественный
отбор. Компании обеспечивают качество продукции заставляя работников
следовать правилам, которые не дают им завалить всё дело. Однако
ничего такого не требуется, если потребители могут взаимодействовать
друг с другом. Люди создают то, что им хочется; удачные решения
развиваются, остальные отмирают. В обоих случаях поддержка
потребителей идет на пользу лучшей работе.

Еще одна черта, общая для свободного ПО и блоггинга - использование
Веба. Люди и раньше с удовольствием занимались творчеством для
собственного удовольствия, но до Веба им было трудно добраться до
пользователей, а также сообща работать над общей задачей.

Любители Мне кажется, что наиболее важный из новых принципов, который
бизнес должен взять на вооружение - тот, что люди гораздо лучше
работают над задачами, которые им интересны. Очевидно, что это не
ново. Так почему же я утверждаю, что бизнес должен учитывать это? Дело
в том, что нынешняя структура бизнеса не предполагает обеспечение
сотрудников интересной работой.

Современный бизнес по-прежнему следует старой модели, которая
отражается французким словом "travailler" - "работать". "Travail",
родственное английское слово, переводится как "пытка". (Слово travail
произошло от латинского trepalium, инструмента для пыток, который
содержал три шеста. Я не знаю, как использовались эти шесты. Слово
"travel" имеет тот же корень.)

Но это еще не все. По мере того, как общество богатеет, его
представления о работе становятся похожи на идеи о диете. Как мы
теперь знаем, полезнее всех питались бедные крестьяне, жившие много
лет назад. Как и обильная трапеза, безделье оказывается желанным
только когда его не хватает. Я думаю, что все мы созданы для работы,
так же как нашим организмам необходимо некоторое количество клетчатки
в рационе. И мы чувствуем себя неважно, когда нам ее недостаточно.

Существует отдельное слово, для обозначения людей, которые работают из
любви к работе - "любители". В наше время это слово приобрело
негативную окраску и мы не замечаем его этимологию, хотя она просто
бросается в глаза. В английском языке слово "amateur" ("любитель")
изначально использовалось для поощрения, похвалы. ["amateur"
происходит от латинского слова "amare", любить] Но двадцатый век был
временем профессионалов, которыми любители не были по-определению.

Пример свободного ПО показывает, что люди, трудящиеся бесплатно, для
собственного удовольствия, часто работают лучше профессионалов,
работающих за деньги. Для бизнес-сообщества это оказалось
неожиданностью. Пользователи меняют веб-браузер Internet Explorer на
Firefox не потому, что собираются модифицировать исходный текст
браузера. Firefox просто лучше.

И нельзя сказать, что в Microsoft не стараются. Там знают, что
сохранение контроля над рынком браузеров есть одно из необходимых
условий сохранения монополии. Однако тут Microsoft сталкивается с той
же проблемой, что и в случае с операционными системами: нет таких
денег, за которые кто-то будет работать лучше, чем энтузиасты
бесплатно.

Подозреваю, что профессионализм всегда ценили незаслуженно высоко - не
только буквально, в смысле оплаты, но и в смысле таких характерных
атрибутов как формальность и недоступность. Это показалось бы
невероятным, скажем, в семидесятых годах, но сейчас мне кажется, что
профессионализм был в основном традицией или модой, сложившейся в
обществе в двадцатом веке под влиянием обстоятельств.

Одним из наиболее важных обстоятельств было существование так
называемых "каналов". Этот термин используется применительно и к
продуктам, и к информации: существуют каналы сбыта, каналы телевидения
и радио.

Вследствие узкой специализации профессионалы выглядели гораздо
серьезнее любителей. Количество рабочих мест на рынке всегда было
ограничено, например, для профессиональных журналистов. Возникавшая в
результате дефицита конкуренция поддерживала качество работы среднего
журналиста на достаточно высоком уровне. И хотя любой мог
разглагольствовать в баре по поводу текущих событий, на фоне
профессионального журналиста казался идиотом.

В Вебе барьер для выражения собственного мнения ниже. Вам не придется
покупать выпивку, и в Веб, в отличие от пивных баров, пускают даже
детей. Миллионы людей пишут в онлайне, и уровень в среднем оказывается
не слишком высоким. Такое положение вещей привело некоторых издателей
к мысли, что блоги не представляют опасности для их бизнеса, что блоги
- лишь веяние моды.

В действительности поветрие - только само слово "блог", по крайней
мере в том смысле, который обычно используется в печати. По сути, они
называют блоггером каждого, кто публикует что-либо в Сети, а не
человека, который пишет в формате блога. Эта неточность перерастет в
проблему, когда Веб станет основной средой для публикаций. Поэтому я
хотел бы заранее предложить другое слово для людей, публикующих
онлайн. Как вам слово "автор"?

В печатных изданиях, которые не обращают внимания на
интернет-публикации из-за невысокого среднего качества, не осознают
один важный момент: никто не читает средний блог. В прежнем мире
"каналов" имелся смысл говорить о среднем качестве, поскольку та или
иная информация существовала только в пределах "каналов", и особой
альтернативы не было. Но сейчас вы можете выбирать из миллионов статей
независимых авторов. И выходит, что традиционные СМИ конкурируют не со
средними, а с лучшими сетевыми авторами. И, подобно Microsoft, они
проигрывают.

Я сужу об этом по собственному опыту читателя. Несмотря на то, что
большая часть печатных СМИ уже присутствуют в Сети, на каждую статью в
газете или журнале, которые я прочитываю, приходится две-три статьи,
опубликованные на частных сайтах.

Например, на интересные статьи New York Times я попадаю не через их
главную страницу. В основном я читаю статьи, подготовленные
агрегаторами новостей вроде Google News, Slashdot или Del.icio.us.
Агрегаторы демонстрируют, насколько они эффективнее по сравнению с
"каналами". Главная страница New York Times содержит статьи,
написанные сотрудниками New York Times. Del.icio.us содержит список
интересных статей. Только теперь вы можете посмотреть на два эти
списка и обнаружить, насколько мало они пересекаются.

Большинство статей в печатных СМИ скучны. Например, президент США
обнаружил, что большинство избирателей считают, что вторжение в Ирак
было ошибкой, и вот он выступает с обращением к нации, чтобы
обеспечить поддержку. Что в этом интересного? Я не слышал этого
выступления, но вероятнее всего заранее знаю, о чем оно было. Такого
рода речь в буквальном смысле не новость: там нет ничего нового. (Было
бы большей, в прямом смысле, новостью, если бы на пресс-конференции
Президенту задавали бы неподготовленные заранее вопросы.)

Как правило, в новостях нет ничего нового, за исключением имен и мест.
В новостях обычно говорится о событиях, которые не должны были бы
случится. Похищен ребенок, прошел ураган, затонул паром, кого-то
покусала акула, упал частный самолет. Что же бы узнаем о мире из этих
новостей? Абсолютно ничего. Они подобны погрешностям измерения,
выпадающим за область типичных значений. То, что делает эти новости
привлекательным, делает их несущественным.

Также как и в случае с программированием, когда из рук профессионалов
выходит мусор, неудивительно, что любители могут добиться лучшего.
Существуя в "канале", находясь в зависимости от сложившейся там
олигополии, вы обретаете нехорошие привычки, которые трудно преодолеть
в ситуации внезапно появившейся конкуренции.

Открытое ПО и блоги зачастую делаются людьми, работающими дома. Это не
кажется странным, а должно бы. Неплохой аналогией был бы самодельный
самолёт, который способен сбить F-18. Компании тратят миллионы на
офисные здания ради одной единственной цели - создать место для
работы. И все же, люди, трудящиеся у себя на дому, который вовсе не
предназначен для работы, в конце концов, оказываются продуктивнее.

Это подтверждает то, о чем многие из нас догадывались. Типичный офис
не годится, для того чтобы делать дело. Многие из качеств, делающие
офис не лучшим для работы местом, мы привыкли считать отличительными
чертами профессионализма. Предполагается, что стерильность офисов
символизирует эффективность. Однако символизировать эффективность
вовсе не означает действительно быть эффективным.

Атмосфера среднего офиса и продуктивность работы в нем имеют не больше
общего, чем нарисованные на автомобиле языки пламени и реальная
скорость машины. Люди в унылом и безрадостном офисе и работают
соответственно.

В начинающих компаниях - стартапах - все иначе. Чаще всего стартап
начинается в квартире основателя. Вместо серой офисной мебели в
стартапах используют разномастные подержанные столы и стулья. Там
работают не по расписанию и носят самую обыкновенную одежду. Там
гуляют по Сети, не беспокоясь о том, уместно ли это в рабочее время.
Вместо фальшивой офисной болтовни там хулиганский юмор. Но, вероятно,
компания никогда не будет продуктивнее, чем на этапе стартапа.

Возможно, это не совпадение. Может быть, от некоторых аспектов
профессионализма больше вреда, чем пользы.

Более всего меня в офисах деморализует необходимость присутствия там в
определенное время. По сути, лишь немногим сотрудникам действительно
требуется быть в офисе по расписанию. Основная причина, по которой
большинство офисных служащих обязаны работать по часам, заключается в
том, что компания не может оценить их продуктивность.

Основной смысл работы по расписанию таков: если нельзя заставить людей
работать, то, по крайней мере, нужно помешать им развлекаться. Если
обязать сотрудников отбывать определенное время и не позволять им
заниматься посторонними вещами, то тем ничего не останется делать,
кроме как работать. В теории. На практике люди просиживают часы "на
нейтральной полосе", не работая и не развлекаясь.

Если бы можно было оценить, сколько люди работают, многим компаниям не
понадобился бы фиксированный рабочий день. Можно было бы просто
сказать сотрудникам: "Вот вам работа. Выполняйте ее, где и когда
хотите. Если эта работа потребует контактов с коллегами, можете
встретиться в офисе. Остальное нас не волнует".

Это похоже на утопию, но именно так мы говорили людям, работавшим в
нашей компании. У нас не было фиксированного рабочего дня. Я никогда
не появлялся на работе раньше одиннадцати утра. Но мы и не занимались
благотворительностью. У нас были такие правила: если работаешь в нашей
компании, работа должна быть сделана. Присутствием на рабочем месте
никого не обмануть.

Недостаток рабочего дня не только в том, что он разлагает. Cотрудники,
изображающие деятельность на рабочем месте, мешают тем, кто
действительно работает. Я убежден, что именно из-за системы учета
рабочего времени в больших организациях проводится так много
совещаний. В пересчете на одного сотрудника достижения больших
организаций мизерны. А ведь все эти люди сидят в офисе, как минимум,
по восемь часов в день. Когда на входе столько потраченного времени, а
на выходе такие скромные результаты, значит где-то "утечка". Совещания
- один из основных способов убивать время впустую.

Однажды я год трудился на обыкновенной работе: полный рабочий день с
9.00 до 17.00 в офисе. Я хорошо помню это странное чувство уюта,
сопровождающее совещания. Тогда я впервые получал деньги за
программирование. Это казалось потрясающим, как если бы на моем
рабочем столе стояла машинка, выплёвывающая долларовую банкноту каждые
две минуты независимо от того, чем я занимаюсь, даже если я провел эти
две минуты в туалете. Поскольку эта воображаемая машинка
функционировала беспрерывно, я постоянно чувствовал, что должен
работать. На этом фоне совещания казались настоящим отдыхом. Они
считались работой, такой же, как программирование, но были куда проще.
Достаточно было просто сидеть в кресле и делать вид, что внимательно
слушаешь.

Заседания - это коллективное снотворное. В меньшей степени этим
страдает электронная почта. Помимо прямых затрат времени, заседания
разбивают рабочий день на фрагменты, слишком маленькие, для того чтобы
их толком использовать.

Чтобы увидеть, насколько вы зависимы от какой-то привычки, попробуйте
на время от нее отказаться. Большим компаниям я предлагаю поставить
следующий эксперимент: выделите один день в неделю без заседаний. В
этот день каждый должен сидеть за своим столом и работать над
задачами, не требующими общения с коллегами. Некоторые виды
деятельности требуют обсуждений, но я уверен, что многие сотрудники
смогут занять восемь часов какой-либо автономной работой. Вы можете
назвать такой день "Рабочим Днем". Проблема с мнимой работой в том,
что она часто выглядит лучше, чем настоящая работа. Когда я пишу
статью или программирую, я трачу на размышления не меньше времени, чем
на набор текста на компьютере. Половину времени я сижу за чашкой чая
или слоняюсь вокруг. Это самая важная часть работы, поскольку в такие
моменты и появляются идеи. И все же, я бы чувствовал себя неловко,
находясь в это время в офисе, где все остальные кажутся занятыми.

Трудно бывает осознать, насколько плоха та или иная система, пока ее
не с чем сравнить. Еще и поэтому свободное ПО, а иногда даже сетевые
дневники так важны - они показывают нам, что такое настоящая работа.

Сейчас мы (компания Y Combinator) спонсируем восемь стартапов. Один
мой друг спросил меня, где работают все эти люди, и удивился, когда я
ответил, что мы предложили им работать дома. Дело не в том, что мы
хотим сэкономить деньги. Мы поступили так, потому что хотим, чтобы
программы, которые они напишут, оказались хорошими. Работа в
неформальной обстановке - одна из вещей, которые неумышленно, но
совершенно правильно делаются в стартапах. Как только вы откроете
офис, работа начнет отдаляться от жизни.

Отдаление работы от остальной деятельности человека - один из
принципов профессионализма. Я уверен, что это неправильно.

И наконец, как мы можем видеть на примере свободного ПО и блогов,
новые идеи могут появляться внизу и постепенно подниматься кверху.
Создание свободного ПО, как и блогов, происходит снизу вверх: люди
занимаются своим любимым делом, а наилучшие результаты начинают
использоваться сообществом.

Неправда ли, мы где-то уже это слышали? Ведь это принцип рыночной
экономики. Забавно, что хотя свободное ПО и блоги создаются бесплатно,
процесс их создания можно описать терминами рыночной экономики, а
большинство компаний при всех разговорах об открытых рынках внутренне
устроены как коммунистические государства.

Существуют два соображения, влияющие на результат созидательной
деятельности: выбор задачи и способ управлением качеством. Во времена
"каналов" и то, и другое спускалось сверху. Например, редактор газеты
заказывал журналистам статьи на определенные темы, а затем
редактировал и публиковал их в газете.

Существование свободного ПО и блогов показывает, что выбор задачи и
обеспечение качества продукции возможны и внизу. И то, и другое
оказывается не просто приемлемо, но лучше. Например, свободное ПО
более надежно в следствиe того, что любой пользователь может найти и
исправить ошибку в исходных текстах программ.

То же касается публикаций. Когда моя книга "Hackers \& Painters"
готовилась к печати, я почувствовал, что особенно беспокоюсь за эссе,
которые не были ранее опубликованы в онлайне. После того как материал
прочитывали несколько тысяч пользователей Сети, я был достаточно
уверен в нем. Непроверенные таким образом статьи внушали на порядок
больше беспокойства. Я чувствовал себя, как будто я выпускаю
программу, не протестировав ее.

До недавних пор это была общая проблема всех публикаций. Хорошо, если
статью удавалось пропустить через десяток рецензентов до печати. Я
успел настолько привыкнуть к публикациям в онлайне, что традиционный
способ кажется теперь верхом ненадежности, подобно навигации по
компасу и логу в сравнении с GPS.

Еще одно преимущество онлайновых публикаций заключается в том, что вы
можете писать, о чем хотите и когда хотите. В этом году у меня
появилась статья, которая казалась подходящей для журнала, и я
отправил ее знакомому издателю. Пока ждал ответа, я с удивлением
поймал себя на том, что хочу, чтобы журнал отказался от этой статьи.
Тогда бы я смог немедленно выложить ее в Сеть. Если бы они приняли
статью, в ближайшие несколько месяцев никто бы ее не прочитал, а тем
временем мне бы пришлось бороться за каждое слово с выпускающим
редактором этого журнала. [Они приняли статью, но я так долго не мог
отправить им окончательный вариант, что раздел журнала, где должна
была быть напечатана эта статья, закрылся в связи с реорганизацией.]

Многие люди могли и хотели бы создавать выдающееся продукты для
компании, в которой они работают. Но, как правило, менеджмент не
поддерживает их. Многие из нас слышали истории о том, как такой-то
сотрудник предлагал руководству заняться перспективной проблемой и
заработать деньги для компании, но компания отказывала. Наиболее
известным примером был бы, вероятно, Стив Возняк, который предлагал
заняться производством микрокомпьютеров компании HP, в которой
работал. Ему тогда отказали. Масштаб этой ошибки сравним с согласием
IBM на неэксклюзивную лицензию на DOS. Я думаю, такие вещи происходят
постоянно. Мы обычно не знаем о них, поскольку, чтобы доказать свою
правоту, вам пришлось бы уволиться и организовать собственную
компанию, как это сделал Возняк.

Стартапы

Итак, свободное ПО и блоггинг могли бы показать бизнесу, что:

1. Люди лучше работают над задачами, которые им интересны 2. Типичная
атмосфера офиса чрезвычайно непродуктивна 3. Организация работы снизу
вверх часто оказывается лучше, чем сверху вниз

Я уже вижу менеджеров, вопиющих: "Что предлагает этот парень? Что
толку от того, что мои программисты будут работать лучше над своими
собственными проектами у себя дома? Мне они нужны здесь, для работы
над версией 3.2 нашей программы! А иначе мы никогда не выпустим это
ПО!"

Так оно и есть, этот конкретный менеджер едва ли выиграет от
использования принципов, обсуждающихся в данной статье. Не думаю, что
любой бизнес сможет внедрить все эти новые методы работы. Можно
рассматривать их как конкурентное преимущество в процессе
естественного отбора. Я не утверждаю, что в результате компании станут
лучше, лишь намекаю, что самые неповоротливые закроются.

Так как же будет выглядеть бизнес, который усвоил методологию
свободного ПО и блоггинга? Мне кажется, что главная преграда, мешающая
увидеть будущее бизнеса - предположение о том, что люди, работающие на
вас, обязательно должны быть сотрудниками вашей компании. Традиционная
модель такова: компания платит сотрудникам зарплату в надежде, что те
сделают для компании что-нибудь, что можно затем продать с выгодой.
Однако возможны другие формы сотрудничества. Вместо того чтобы платить
человеку зарплату, почему бы не выплатить ему те же деньги в форме
инвестиций. Тогда вместо работы в вашем офисе над вашими задачами он
работал бы, где ему хочется, над собственным проектом.

Поскольку немногим знакома иная система, мы не представляем, насколько
эффективнее можно было бы работать вне традиционной модели
"босс-подчиненный". Такого рода комплексы удивительно устойчивы.
Традиционная модель трудовых отношений базируется на наших врожденных
инстинктах создания иерархического сообщества. [Слово "босс"
происходит от голланского "baas", что означает "хозяин".]

Мне не нравится находиться ни наверху, ни внизу служебной лестницы. Я
хочу работать на благо клиента, но указания начальника выполнял бы
безо всякого энтузиазма. Позиция руководителя также крайне
разочаровывает. Иногда бывает проще выполнить работу самому, чем
заставить подчиненных. Я готов сделать все, что угодно, только бы не
сдавать или принимать переаттестацию.

Помимо неэффективной основы, система найма на работу за многие годы
накопила множество комплексов и запретов. Список запрещенных вопросов
при приеме на работу так разросся, что проще считать его бесконечным.
В офисе вы будете опасаться сказать или сделать что-нибудь, что может
сделать компанию жертвой судебного иска. И упаси вас Бог уволить
кого-либо.

То, что трудовые отношения не ограничиваются экономическими
соображениями, лучше всего видно на примере компаний, на которые
подали в суд за увольнения работников. В случае чисто экономических
отношений вы свободны делать все что угодно. Например, в какой-то
момент вы можете решить сменить поставщика стальных труб и вы не
обязаны никому объяснять причину. Трудовое законодательство вносит
что-то вроде родительской опеки, которой нет в отношениях между
равными.

Большая часть ограничений, накладываемых законом на работодателей,
призваны защитить работников. Но действие всегда вызывает
противодействие. Компании окружают сотрудников подобием родительской
опеки, в следствии которой работники оказываются в роли детей. И это
не идет им на пользу.

Если вы окажетесь в каком-нибудь достаточно крупном городе,
понаблюдайте за служащими центрального почтового отделения. У них
можно заметить ту же скрытую обиду, негодование и неудовлетворенность,
что и у детей, которых взрослые заставляют заниматься неинтересным
делом. Их профсоюз добился увеличения зарплаты и улучшения условий
труда, которым позавидовали бы предыдущие поколения почтовых служащих.
И все же это не приносит им особой радости. Находиться под чьей-то
опекой всегда неприятно, как бы удобны ни были условия. Спросите у
любого подростка.

Я вижу недостатки традиционных трудовых отношений, поскольку я побывал
по обе стороны лучшей системы - инвестор-владелец компании. Конечно, и
такая система не идеальна. Когда я работал в стартапе, беспокойства
по-поводу инвесторов мешали мне спать. А сейчас я сам инвестор,
проблемы наших стартапов также не дают мне покоя. Все беспокойства о
проблеме, которую вы пытаетесь решить, по-прежнему мучают вас. И все
же стресс не так силен, если к нему не добавляются негодование и
обида.

Я имел несчастье участвовать в эксперименте по проверке этой теории.
После того как компания Yahoo купила наш стартап, я начал работать на
Yahoo. Я занимался точно тем же, чем и раньше, но теперь под чутким
руководством. К моему изумлению, я начал вести себя как ребенок.
Ситуация затронула рычаги, о существовании которых я давно забыл.

Большое преимущество инвестиционной модели над наемной кроется в том,
что, как показывают примеры свободного ПО и блоггинга, люди,
работающие над такими проектами, гораздо продуктивнее. Стартап, по
сути, есть собственность владельца в двух отношениях, каждое из
которых важно: это результат его творчества и продукт экономической
деятельности.

Компания Google - это пример большой компании, находящейся в гармонии
с силами, о которых написано это эссе. В Google попытались сделать
офисы менее похожими на обычную ферму кубиков. Выдающимся сотрудникам
предлагаются значительные пакеты акций компании, чтобы стимулировать
заинтересованность в успехах компании. Они даже разрешают инженерам
тратить 20\% рабочего времени на работу над собственными проектами.

Почему бы не позволить людям тратить все свое время на собственные
проекты, а вместо того чтобы пытаться вычислить стоимость их работы,
заплатить им, исходя из реальной рыночной стоимости их продукта?
Невозможно? Ведь это то, чем занимаются венчурные капиталисты.

Так что же, я предлагаю каждому уволиться с работы и организовать
стартап? Конечно, нет. На самом деле людей, способных на такой шаг,
больше чем людей, такой шаг совершивших. Сейчас даже самые смышленые
студенты заканчивают университет с намерением найти работу. В
действительности, все, что им нужно, это создать что-нибудь полезное.
Наемная работа - один из способов достижения цели. Более амбициозные
люди, вероятно, добьются большего, получая деньги не от работодателя,
а от инвестора.

Инженеры склонны считать, что занятие бизнесом - удел людей с дипломом
MBA. Но управление бизнесом - это не то, чем занимаются в стартапе.
Там занимаются созданием бизнеса, первая фаза которого - создание
продукта. Это сложнейшая часть процесса. Гораздо труднее создать
что-либо , что будет востребовано обществом, чем наладить продажу
готового продукта.

Люди опасаются открывать стартап из-за возможных рисков. Семейный
человек с непогашенным долгом за квартиру несколько раз подумает,
перед тем как ввязываться в такое рискованное дело. Но у молодых
инженеров, как правило, еще нет ни детей, ни долгов.

Как мы видим на примерах свободного ПО и блоггинга, от работы в
стартапе вы получите гораздо больше удовольствия. Вы будете работать
над собственным детищем, а не приходить в какой-то офис и выполнять
то, что решит начальник. Собственная компания может доставлять больше
хлопот, но эти хлопоты не будут так болезненны.

В перспективе, главным достижением сил, движущих создание свободного
ПО и блогов, возможно, окажется демонтаж старых, отеческих отношений
работника с работодателем и замена их экономическим партнерством
равных.

\end{document}
