\documentclass[ebook,12pt,oneside,openany]{memoir}
\usepackage[utf8x]{inputenc} \usepackage[russian]{babel}
\usepackage[papersize={90mm,120mm}, margin=2mm]{geometry}
\sloppy
\usepackage{url} \title{Теория навязчивых идей} \author{Пол Грэм}
\date{}
\begin{document}
\maketitle

Всем известно: чтобы совершить что-то выдающееся, нужны как природные
способности, так и решительность. Но есть и третий компонент, который
не так хорошо понят: страстный интерес к определенной теме. \newline

Чтобы объяснить этот момент, мне придется пожертвовать своей
репутацией среди какой-то группы людей, и я выбрал коллекционеров
автобусных билетов. Есть люди, которые собирают старые автобусные
билеты. Как и многие коллекционеры, они одержимы интересом к
мельчайшим деталям того, что собирают. Они умеют находить различия
между разными типами билетов, которые нам всем трудно запомнить.
Потому что нам все равно. Какой смысл тратить столько времени на
размышления о старых автобусных билетах? \newline

Что приводит нас ко второй особенности этого вида одержимости: в ней
нет никакого смысла. Любовь коллекционера автобусных билетов
бескорыстна. Они делают это не для того, чтобы произвести на нас
впечатление или разбогатеть, а ради самих себя. \newline

Если посмотреть на жизнь людей, которые совершили что-то выдающееся,
можно увидеть последовательную картину. Они часто начинают с
одержимости в духе коллекционера автобусных билетов к чему-то, что
кажется бессмысленным большинству их современников. Одна из самых
ярких черт книги Дарвина о путешествии на «Бигле» — это его глубокий
интерес к естествознанию. Его любопытство кажется бесконечным. То же
самое можно сказать про Рамануджана, часами размышляющего над тем, что
происходит с рядами чисел. \newline

Ошибочно думать, что они «закладывают основу» для последующих
открытий. В этой метафоре слишком много стремления. Как и
коллекционеры автобусных билетов, они делали это, потому что им это
нравилось. \newline

Но есть разница между Рамануджаном и коллекционерами билетов. Числа
имеют значение, а билеты на автобус — нет. \newline

Если бы мне пришлось уложить рецепт гениальности в одно предложение,
то я написал бы так: бескорыстная одержимость чем-то важным. \newline

Не забыл ли я о двух других компонентах? Меньше, чем вы думаете.
Страстный интерес к теме — это одновременно и показатель способностей,
и замена решимости. Если у вас нет достаточных математических
способностей, ряды не будут вам интересны. А когда вы одержимы чем-то
интересным, не требуется много решимости: вам не нужно слишком сильно
давить на себя, когда вас влечет любопытство. \newline

Страстный интерес даже принесет вам удачу, насколько это возможно.
Удача, как сказал Пастер, благосклонна к подготовленным умам, и если
чем и характеризуется одержимый разум, так это подготовленностью. \newline

Бескорыстие — это наиболее важная особенность этого вида одержимости.
Не только потому, что это фильтр для серьезности, но потому что оно
помогает вам открывать новые идеи. \newline

Пути, ведущие к новым идеям, выглядят бесперспективными. Если бы они
выглядели многообещающими, другие люди уже изучили бы их. Как люди,
которые совершают что-то выдающееся, находят пути, которые пропускают
другие? Популярное объяснение гласит, что у них просто лучшее видение:
они настолько талантливы, что видят пути, которые другие не замечают.
Но если вы посмотрите на то, как были сделаны великие открытия, то
заметите, что это произошло иначе. Дарвин обращал больше внимания на
отдельные виды, чем другие люди, не потому что видел, что это приведет
к великим открытиям, а они не видели. Он просто очень, очень
интересовался такими вещами. \newline

Дарвин просто не мог перестать заниматься этим. Рамануджан тоже не
мог. Они открыли новые пути не потому, что эти пути казались
многообещающими, а потому что ничего не могли с собой поделать. Вот
почему они пошли по пути, который амбициозный человек просто
проигнорировал бы. \newline

Какой разумный человек решит, что для написания великих романов нужно
потратить несколько лет на создание воображаемого эльфийского языка,
как Толкиен, или побывать в каждом доме на юго-западе Британии, как
Троллоп? Никто, включая Толкиена и Троллопа. \newline

Теория автобусных билетов похожа на знаменитое определение
гениальности как бесконечной способности к титаническому труду
(Карлайл). Но есть два отличия. Теория автобусных билетов дает понять,
что источник этой бесконечной способности к титаническому труду — не
бесконечное усердие, как, по-видимому, предполагал Карлайл, а своего
рода бесконечный интерес, который есть у коллекционеров. Также к этому
добавляется бесконечная способность трудиться над тем, что имеет
значение. \newline

А что имеет значение? Никогда нельзя быть уверенным. Именно потому,
что никто не может заранее сказать, какая дорожка приведет к новым
идеям в том, что вас интересует. \newline

Но есть кое-какие эвристические правила, которые можно использовать,
чтобы угадать, может ли навязчивая идея стать тем, что имеет значение.
Например, более многообещающая ситуация — если вы создаете что-то, а
не просто потребляете то, что создает кто-то другой. Или если вас
интересует что-то сложное, особенно если другие люди разбираются в
этом хуже, чем вы. Более многообещающие идеи встречаются у талантливых
людей. Когда талантливые люди начинают интересоваться случайными
вещами, это не случайно. \newline

Но вы никогда не можете быть уверены. Вот интересная идея, которая
также вызывает тревогу, если она верна: возможно, чтобы совершить
что-то выдающееся, вам придется потратить много времени впустую. Во
многих различных областях вознаграждение пропорционально риску. Если
речь идет об открытиях, придется потратить много усилий на вещи,
которые окажутся именно такими бесперспективными, как и казались. \newline

Я не уверен, правда ли это. С одной стороны, кажется удивительно
трудным тратить впустую так много времени, пока вы усердно работаете
над чем-то интересным. Многое из того, что вы делаете, оказывается
полезным. Но с другой стороны, правило о соотношении риска и
вознаграждения настолько сильно, что кажется, оно действует везде, где
возникает риск. История Ньютона, по крайней мере, показывает, что
правило работает. Он известен своей особой навязчивой идеей, которая
оказалась беспрецедентно плодотворной: использование математики для
описания мира. Но у него были две другие навязчивые идеи — алхимия и
теология, — которые, похоже, оказались пустой тратой времени. В итоге
он остался в выигрыше. Его ставка на то, что мы сейчас называем
физикой, окупилась настолько хорошо, что с лихвой компенсировала две
другие. Но были ли эти две другие необходимы? Должен ли он был пойти
на большой риск, чтобы сделать такие важные открытия? Я не знаю. \newline

Вот еще более тревожная идея: бывает ли, что все ставки оказываются
неправильными? Вероятно, бывает довольно часто. Но мы не знаем, как
часто, потому что люди, сделавшие неудачные ставки, не становятся
знаменитыми. \newline

Дело не только в том, что результаты трудно предсказать. Они резко
меняются со временем. 1830 год был действительно хорошим временем для
увлечения естествознанием. Если бы Дарвин родился в 1709 году, а не в
1809, мы никогда бы о нем не услышали. \newline

Что можно сделать, столкнувшись с такой неопределенностью? Одно из
решений — подстраховаться, что в данном случае означает идти по
многообещающим путям вместо личных навязчивых идей. Но как и при любом
смягчении рисков, вознаграждение уменьшается. Если вы отказываетесь от
работы над тем, что вам нравится, чтобы идти по какому-то более
амбициозному пути, вы можете упустить что-то замечательное, что иначе
обнаружили бы. \newline

Другое решение — позволить себе интересоваться множеством разных
вещей. Вы не уменьшите свой потенциал, если будете переключаться между
одинаково искренними интересами. Но и здесь есть опасность: если у вас
слишком много разнообразных проектов, вы не можете углубиться ни в
один из них. \newline

Одна интересная вещь в теории автобусных билетов заключается в том,
что она может объяснить, почему разные люди преуспевают в разных видах
работы. Интерес распределен гораздо более неравномерно, чем
способность. Если естественные способности — это все, что вам нужно
для достижения важных результатов, и естественные способности
распределены равномерно, нужно придумать сложные теории, чтобы
объяснить искаженное распределение, которое мы видим среди тех, кто
действительно совершает что-то выдающееся в различных областях. Но
может быть, большая часть перекосов объясняется гораздо проще: разные
люди интересуются разными вещами. \newline

Теория автобусных билетов также объясняет, почему менее вероятно, что
люди совершат что-то выдающееся после рождения детей. Здесь интерес
конкурирует не только с внешними препятствиями, но и с другим
интересом — который у большинства людей чрезвычайно силен. Когда у вас
появляются дети, становится труднее найти время для работы, но это не
самое страшное. Главное, что вы этого не хотите. \newline

Но самое захватывающее следствие теории автобусных билетов состоит в
том, что она предлагает способы поощрения качественной работы. Если
рецепт гениальности — просто природные способности плюс усердная
работа, то все, что мы можем сделать, — надеяться, что у нас много
способностей, и работать изо всех сил. Но если важнейший компонент
гениальности — это интерес, то взращивая его, мы можем культивировать
гениальность. \newline

Например, теория автобусных билетов утверждает, что способ совершить
важную работу — немного расслабиться. Вместо того, чтобы стискивать
зубы и усердно заниматься тем, что большинство ваших коллег считают
многообещающим направлением, возможно, вам следует заняться чем-то
просто ради удовольствия. И если вы зашли в тупик, это может быть
вектором, за которым наступит перелом. \newline

Мне всегда нравился знаменитый двусмысленный вопрос Хэмминга: каковы
самые важные проблемы в вашей области и почему вы не работаете над
одной из них? Это отличный способ встряхнуться. Полезно спросить себя:
если бы можно было взять годовой отпуск, чтобы заняться чем-то не
обязательно важным, но действительно интересным, что это было бы? \newline

Теория автобусных билетов также предлагает способ избежать снижения
темпов, когда вы становитесь старше. Возможно, с возрастом у людей
появляется все меньше новых идей не просто потому, что они теряют свои
преимущества. Причина также может быть в том, что вы зарабатываете
авторитет и больше не можете возиться с ненадежными проектами, как в
молодости, когда никому не было до этого дела. \newline

Решение очевидно: оставайтесь безответственными. Будет трудно, потому
что очевидно случайные проекты, за которые вы возьметесь, чтобы
предотвратить спад, будут восприниматься окружающими как
доказательство этого. И вы сами не будете уверены, что они ошибаются.
Но по крайней мере, будет приятнее работать над тем, над чем вы
хотите. \newline

Мы даже можем выработать у детей привычку собирать интеллектуальные
автобусные билеты. Стандартный план в образовании состоит в том, чтобы
начинать с широкого фокуса, а затем постепенно углубляться в
конкретные предметы. Но со своими детьми я сделал прямо
противоположное. Я знаю, что могу рассчитывать на школу в широких
поверхностных знаниях, поэтому занимаюсь с ними глубокими. \newline

Когда что-то внезапно вызывает у них интерес, я призываю их быть
нелепыми, собирать автобусные билеты, идти вглубь. Я делаю это не
из-за теории автобусных билетов. Я делаю это потому, что хочу, чтобы
они почувствовали радость обучения, а они никогда не испытают этого,
если заставить их учиться. Это должно быть чем-то, что их интересует.
Я просто иду по пути наименьшего сопротивления, а глубина — лишь
побочный результат. Но пытаясь показать им радость обучения, я в
конечном итоге учу их смотреть вглубь. \newline

Окажет ли это какое-либо влияние? Понятия не имею. Но эта
неопределенность может быть самым интересным моментом из всех. Можно
так многому еще научиться, как достичь чего-то значимого. Какой бы
старой ни чувствовала себя человеческая цивилизация, на самом деле она
еще очень молода, если мы до сих пор не научились такой
основополагающей вещи. Дух захватывает, сколько открытий об открытиях
еще можно сделать. Если, конечно, вам это интересно. 

\end{document}
