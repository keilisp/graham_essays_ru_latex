\documentclass[ebook,12pt,oneside,openany]{memoir}
\usepackage[utf8x]{inputenc} \usepackage[russian]{babel}
\usepackage[papersize={90mm,120mm}, margin=2mm]{geometry}
\sloppy
\usepackage{url} \title{Что решают языки программирования} \author{Пол
  Грэм} \date{}
\begin{document}
\maketitle

Кевин Келлехер (Kevin Kelleher) предложил интересный способ сравнивать
языки программирования: описать каждый в терминах решаемой им задачи.
Удивительно, как хорошо языки могут быть описанны таким образом.

Algol: Ассемблер слишком низкоуровнен.

- Assembly language is too low-level.

Pascal: В Анголе недостаточно типов данных.

- Algol doesn't have enough data types.

Modula: Pascal слишком хлипок для системного программирования.

- Pascal is too wimpy for systems programming.

Simula: Ангол недостаточно хорош для симуляций.

- Algol isn't good enough at simulations.

Smalltalk: В Simula не всё является объектами.

- Not everything in Simula is an object.

Fortran: Ассемблер слишком низкоуровнен.

- Assembly language is too low-level.

Cobol: Fortran страшен.

- Fortran is scary.

PL/1: В Fortran недостаточно типов данных.

- Fortran doesn't have enough data types.

Ada: В каждом существующем языке чего-нибудь да не хватает.

- Every existing language is missing something.

Basic: Fortran страшен.

- Fortran is scary.

APL: Fortran недостаточно хорош для манипуляций массивами.

- Fortran isn't good enough at manipulating arrays.

J: APL требует собственной раскладки клавиатуры.

- APL requires its own character set.

C: Ассемблер слишком низкоуровнен.

- Assemby language is too low-level.

C++: C слишком низкоуровнен.

- C is too low-level.

Java: C++ is a kludge. Microsoft is going to crush us.

- C++ это вариантная запись (kludge). И Мелкомягкие собираются
сокрушить нас.

C\#: Java контролируется Sun.

- Java is controlled by Sun.

Lisp: Машины Тюринга это неудобный способ для описания вычислений.

- Turing Machines are an awkward way to describe computation.

Scheme: MacList -- вариантная запись.

- MacLisp is a kludge.

T: В Scheme нет библиотек.

- Scheme has no libraries.

Common Lisp: Развелось слишком много диалектов Lisp.

- There are too many dialects of Lisp.

Dylan: В Scheme нет библиотек, а синтаксис Lisp страшен.

- Scheme has no libraries, and Lisp syntax is scary.

Perl: Языки оболочек scripts/awk/sed недостаточно похожи на языки
программирования.

- Shell scripts/awk/sed are not enough like programming languages.

Python: Perl -- вариантная запись.

- Perl is a kludge.

Ruby: Perl -- вариантная запись, а синтаксис Lisp страшен.

- Perl is a kludge, and Lisp syntax is scary.

Prolog: Программирование недостаточно похоже на логические
размышления.

- Programming is not enough like logic.

\end{document}
