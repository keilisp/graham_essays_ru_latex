\documentclass[ebook,12pt,oneside,openany]{memoir}
\usepackage[utf8x]{inputenc} \usepackage[russian]{babel}
\usepackage[papersize={90mm,120mm}, margin=2mm]{geometry}
\sloppy
\usepackage{url} \title{Сделано в США} \author{Пол Грэм} \date{}
\begin{document}
\maketitle

Несколько лет назад одна моя итальянская подруга путешествовала на
поезде из Бостона в Провиденс. На тот момент, она провела в Америке
только пару недель и видела немного. По приезду она выглядела
удивленной. "Так уродливо!".

Люди из других богатых стран с трудом представляют убожество
рукотворных кусочков Америки. В книгах о путешествиях им показывают в
основном природу: Большой Каньон, сплав на плотах, лошадей в поле.
Если вы видите изображения рукотворных объектов, то это будет или
панорамный вид Нью-Йорка, снятый с благоразумного расстояния, или же
тщательно выбранное изображение прибрежного городка в Мэне.

Как такое возможно, должно быть, удивляются приезжие. Как может
богатейшая страна выглядеть таким образом?

Как это не странно, но это может быть и неслучайно. Американцы хороши
в одних вещах, и плохи в остальном. Мы хороши в создании фильмов и
программного обеспечения, и плохи в создании автомобилей и городов. Я
думаю, что мы хороши в том в чём мы хороши по той же самой причине,
почему мы плохи в том в чём мы плохи. Мы нетерпеливы. В Америке, если
вы хотите что-то сделать, вы не беспокоитесь о том, что это может
выйти плохо, или нарушить деликатное социальное равновесие, или о том
что люди подумают о вас, будто вы пытаетесь прыгнуть выше головы. Если
вы хотите что-то сделать, то, как говорит Nike, "просто делайте это".

Это хорошо работает в одних областях и плохо в других. Я догадываюсь,
что это хорошо работает с кино и софтом из-за того, что это -
раздолбайские процессы. "Систематичность" это последнее слово для
описания того, как хороший программист пишет софт. Код это не то, что
тщательно собирают после всеобъемлющего планирования, как пирамиды.
Это что-то во что бросаются сломя голову, трудятся быстро и постоянно
изменяют своё мнение, как при создании наброска мелом.

С программным обеспечением, как бы парадоксально это не звучало,
хороший уровень мастерства предполагает быструю работу. Если вы
работаете медленно и методично, то в конце-концов вы получаете просто
хорошую реализацию своей первоначальной ошибочной идеи. Медленная и
методичная работа - это преждевременная оптимизация. Лучше всего
быстро разработать прототип, и посмотреть какие новые идеи он даёт.

Создание фильмов происходит так же как и создание программного
обеспечения. Каждый фильм это Франкенштейн, полный несовершенств и
обычно достаточно далёк от первоначального виденья. Но интересен, и
закончен довольно быстро.

Я думаю, мы достигли многого таким образом с фильмами и программным
обеспечением из-за того что это податливый материал. Смелость
вознаграждается. И если в последнюю минуту две части не полностью
подходят друг к другу, то можно придумать какой-нибудь хак, по крайней
мере, облегчающий проблему.

Но это не так с машинами и городами. Они слишком материальны. Если бы
авто-промышленность работала как софтостроение или киноиндустрия, то
вы бы попытались бы обойти конкурентов производя автомобиль весящий
всего пятьдесят фунтов, или складывающийся до размеров мотоцикла когда
вы хотите припарковать его. Но у материальных продуктов больше
ограничений. Вы не можете выиграть значительными нововведениями
столько же сколько хорошим вкусом и вниманием к деталям.

Неприятность заключается в том, что само слово "вкус" звучит слегка
нелепо для американского уха. Оно кажется претенциозным, или
легкомысленным, или даже женственным. Голубые статоры!!(Blue staters)
считают, что он "субъективен", а красные статоры!! - что он для
неженок. Так что, любой кто действительно заботится о дизайне в
Америке будет плыть против ветра.

Двадцать лет назад мы постоянно слышали, что главная проблема
авто-индустрии США заключается в рабочих. Сейчас, когда японские
компании делают автомобили в США, этого больше не слышно. Проблема
американских автомобилей заключается в плохом дизайне. Это можно
увидеть просто глядя на них.

Все эти дополнительные куски металла на AMC Matador не были добавлены
рабочими. Проблема этого автомобиля, как и с остальными американскими
автомобилями сегодня, заключается в том, что он был разработан людьми
из маркетинга, вместо дизайнеров.

Почему японцы делают лучшие машины чем делаем мы? Некоторые скажут,
что это из-за того что их культура поощряет сотрудничество. Частично
это так. Но в данном случае, более соответствующим данному вопросу
кажется то, что их культура ценит дизайн и мастерство.

Веками японцы делали более прекрасные вещи, чем то, чем мы владеем на
западе. Когда вы смотрите на меч, сделанный ими в 1200, то вы просто
не можите поверить в то, что дата на ярлычке верна. Наверное их машины
лучше чем наши по той же причине, почему и их столярное ремесло всегда
было лучше. Они одержимы идеей делать вещи хорошо.

Но не мы. Когда мы в Америке что-то делаем - наша цель просто
выполнить работу. Как только мы этого добиваемся мы идём по одному из
двух путей. Мы можем остановиться на этом, и получить нечто грубое и
пригодное к эксплуатации, как Vice-grip. Или же, мы можем улучшать
это, что обычно означает украшение бессмысленным орнаментом. Когда мы
хотим сделать авто "лучше", мы прилепляем хвостовые плавники на неё,
или делаем её длиннее, или уменьшаем окна, в зависимости от
современной моды.

Тоже самое относится и к домам. В Америке вы можете владеть или
непрочной коробкой, собранной из двух-четырёх частей и со стенами из
фанеры(drywall), или McMansion - непрочной коробкой, собранной из
двух-четырёх частей и со стенами из фанеры(drywall), но больше,
выглядящим более внушительно, и полным дорогой обстановкой. Богатые
люди не получают лучший дизайн или лучшее качество; они получают более
крупную, более заметную версию стандартного дома.

Мы не особо ценим дизайн или мастерство в этом. Нам нравится скорость,
и мы желаем делать что-нибудь уродливо для того чтобы сделать это
быстро. В некоторых областях, таких как программное обеспечение или
кино, - это чистая победа.

Но не только фильмы и программы являются податливыми средами. В таких
областях бизнеса, дизайнеры(хотя, в основном их так не называют)
обладают большей властью. Компании создающие ПО, по крайней мере
удачливые, в основном управляются программистами. В киноиндустрии,
несмотря на то, что продюссер может указывать режисёру, режисёр
контролирует большую часть того, что появится на экране. И таким
образом американское ПО и фильмы и японские автомобили, схожи в том,
что руководство заботится о дизайне - первые из-за того что дизайнеры
руководят, а вторые из-за того что вся культура озабочена дизайн.

Я думаю, что большая часть японских руководителей ужаснётся от идеи
сделать плохой автомобиль. В то время, как большая часть американских
руководителей до сих пор верит в то, что в автомобиле наиболее важно
то, какой образ он выражает. -- Сделать хороший автомобиль? -- Что
значит "хороший"? -- Это так субъективно. Если вы хотите узнать как
спроектировать хороший автомобиль - спросите у фокус-группы.

Вместо того чтобы полагаться на свой внутренний дизайн-компас(как это
делал Генри Форд), американские автомобильные компании пытаются
заставить людей из маркетинга выяснить чего хотят потребители. Но это
не работает. Доля рынка американских автомобилей продолжает
сокращаться. И причина в том, что потребитель не хочет того, что, как
он думает, он хочет.

Позволяя фокус-группам разрабатывать ваши автомобили за вас, вы
получаете только краткосрочное преимущество. В долгосрочном плане,
такой подход проигрывает качественному дизайну. Фокусная группа может
сказать, что им хочется показную фишку-однодневку, но ещё больше они
хотят изобразить изощрённых покупателей, и они, хотя и меньшинство из
них, действительно заботятся о хорошем дизайне. В конце концов
сутенёры и наркоторговцы заметили, что врачи и юристы переключились с
Кадилака на Лексус, и сделали то же самое.

Apple является интересным примером подхода обратного основной
американской традиции. Если вы хотите приобрести милый CD плейер, то
вы скорее всего купите японский. Но если вы хотите купить MP3 плейер,
то скорее всего, приобретёте iPod. Что произошло? Почему Sony не
доминирует на рынке MP3 плейеров? Из-за того что Apple сейчас в
бизнессе потребительской электроники, и в отличее от других
американских компаний, они одержимы хорошим дизайном. Или, более
точно, их CEO одержим.

Я только что достал один iPod, и он не просто приятен. Он удивительно
приятен. Для того чтобы удивить меня, он должен был удовлетворять те
ожидания о которых я даже не знал, что у меня они есть. Ни одна фокус
группа не обнаружит этого. Только великие дизайнеры могут.

Машины это не худшее, что производится в Америке. Наши города, а
вернее exurbs!! - пример того, где модель "просто делай это" терпит
неудачу наиболее явно. Если масштаб работы застройщика достаточно
велик, если он возводит целый город, рыночные силы будут вынуждать его
возводить города не являющиеся отстоем. Но если они строят зараз
только пару офисных зданий или улицу в пригороде, то результат
настолько подавляющий, что обитатели рассматривают полет в Европу на
пару недель для того чтобы просто пожить в условиях обычных для
местных жителей, как отличное лечение.

Но модель "просто делай это" имеет свои преимущества. Она в чистую
выигрывает в создании богатства и технических инноваций (которые
являются практически одним и тем же). Я предполагаю, что причина в
скорости. Сложно создать богатство создавая обычные вещи. Настоящая
ценность в новых вещах, и если вы хотите быть первым в создании
чего-либо, то вам стоит работать быстро. Хорошо это или плохо, модель
"просто делая это" - быстра, являетесь ли вы Деном Бриклином, пишущим
прототип VisiCalc за одни выходные, или застройщиком строящим квартал
дрянных жилых домов за месяц.

Если мне пришлось выбирать между моделью "просто-делай-это" и
тщательной моделью, то я, вероятно выбрал-бы
модель-"просто-делай-это". Но должны ли мы выбирать? Не можем ли мы
идти двумя путями? Могут ли американцы жить в приятных домах без
подрыва нетерпеливого духа индивидуальности, делающего нас настолько
сильными в создании ПО? Могут ли другие страны добавить больше
индивидуализма в свои технологические компании и исследовательские
лаборатории?(without having it metastasize as strip malls?) Я
оптимист. Трудно сказать о других странах, но, по крайней мере, в США
мы можем обладать и тем и другим.

Apple является вдохновляющим примером. Они умудрились сохранить
достаточно нетерпеливого, хакерского духа, необходимого для написания
софта. Но когда вы берёте настольный компьютер Apple он не выглядит
американским. Он слишком совершенен. Он выглядит так, как будто он был
сделан шведской или японской компанией.

Во многих технологиях версия 2 имеет большее разрешение. Но почему не
в дизайне? Я думаю, что постепенно мы будем видеть как национальные
черты будут сменяться профессиональными: хакеры в Японии смогут вести
себя с преднамеренностью, не выглядящей японской, а в Америке продукты
будут разрабатываться с упором на вкус, который сейчас не показался бы
американским. Возможно, наиболее удачливыми странами, в будущем будут
наиболее желающие игнорировать национальные черты, и выполнять каждый
тип работ способом, который работает лучше всего. Готовтесь к гонке!

Замечания

[1] Японские города тоже неприглядны, но по другой причине. Япония
склонна к землетрясениям, так что, здания традиционно считались чем-то
временным; в Японии нет великих традиций планирования городов,
подобных унаследованным Европой от Рима. Другая причина - печально
известные высокой коррупцией отношения между правительством и
строительными компаниями.

\end{document}
