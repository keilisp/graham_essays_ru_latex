\documentclass[ebook,12pt,oneside,openany]{memoir}
\usepackage[utf8x]{inputenc} \usepackage[russian]{babel}
\usepackage[papersize={90mm,120mm}, margin=2mm]{geometry}
\sloppy
\usepackage{url} \title{Города и амбиции} \author{Пол Грэм} \date{}
\begin{document}
\maketitle

Великие города привлекают людей с амбициями. Это можно почувствовать,
гуляя по городу. Сотней едва заметных способов город сообщает вам: вы
способны на большее; вам стоит приложить больше усилий.

Удивительно, какими разными могут быть эти ощущения. Нью-Йорк прежде
всего прочего говорит вам: нужно зарабатывать больше. Конечно, есть и
другие намеки. Нужно быть моднее. Нужно лучше выглядеть. Но самое
ясное послание, исходящее от Нью-Йорка – нужно быть богаче.

Что мне нравится в Бостоне (или скорее даже Кембридже) – что здесь
другой сигнал: нужно быть умнее. Нужно успевать прочесть все те книги,
что ты давно собирался.

Когда задаешься вопросом, какие же сигналы посылает город, иногда
получаешь поразительные ответы. Как бы в Кремниевой долине ни уважали
ум, Долина сообщает следующее: нужно быть могущественнее.

Это совсем не то ощущение, которое возникает в Нью-Йорке. Власть важна
и в Нью-Йорке, разумеется, но Нью-Йорк легко впечатлить миллиардом
долларов, даже если ты просто получил его в наследство. В Кремниевой
долине это не впечатлит никого, разве что нескольких продавцов
недвижимости. В Кремниевой долине важно, какой эффект ты оказываешь на
мир. Почему людей из Долины так волнуют персоны Ларри и Сергея? Не
потому, что они богачи, но потому, что они контролируют Google, а
Google влияет на жизнь практически каждого человека.

Насколько важны те сигналы, что посылает город? На практике ответ
ясен: весьма важны. Вы можете думать, что если бы у вас было
достаточно умственных сил, чтобы делать что-то прекрасное, то ваше
окружение не имело бы для вас значения. Место, где вы живете, должно
быть, мало что меняет. Но если посмотреть на исторические факты, то
оно меняет очень многое. Большинство людей, создавших что-то великое,
скучивались в нескольких местах, где этими великими делами занимались
постоянно.

Города – могущественная сила. Практически каждый итальянский живописец
15 века был из Флоренции, хотя Милан тогда был столь же крупным
городом. Жители Флоренции генетически не отличались от жителей Милана,
так что логично предположить, что в Милане тоже должен был родиться
художник со способностями Леонардо да Винчи. Но что тогда с ним
произошло?

Даже если человек, обладающий талантами Леонардо, не преодолел силу
своего окружения, неужели вы думаете, что вам это под силу?

Я так не думаю. Я человек упрямый, но ни за что не стал бы сражаться с
этой силой. Лучше использовать ее. Поэтому я много думал о том, где
мне жить.

Я всегда думал, что идеальное место – Беркли в Калифорнии. Но когда я,
наконец, переехал туда пару лет назад, оказалось, что все не так.
Сигнал, который посылает Беркли – нужно жить лучше. Жизнь в Беркли
очень цивилизованная. Наверное, это единственное место в Америке, где
люди из Северной Европы почувствуют себя как дома. Но там нет никаких
амбиций.

И на самом деле нет ничего удивительного, что такое приятное место
притягивает людей, которых больше всего волнует качество жизни. Люди,
которых встречаешь, скажем, в Кембридже (в Массачусетсе), попадают
туда не случайно. Они идут на жертвы, чтобы жить там. Там дорого и
порой неряшливо, погода часто не радует. Поэтому люди, живущие в
Кембридже – это люди, желающие жить там, где живут умнейшие, даже если
это грязный дорогой город с плохой погодой. Кембридж – это город, где
производят идеи, тогда как Нью-Йорк – это город финансов, а Кремниевая
долина – территория стартапов.


Когда обсуждаешь города в этом ключе, на самом деле говоришь о людях.
Долгое время города были просто скоплениями людей. Но сегодня многое
поменялось. Нью-Йорк – классический большой город. Но Кембридж – лишь
часть города, а Кремниевая долина – вообще не город.

Может быть, интернет изменит все еще больше. Может быть, когда-нибудь
самое важное ваше сообщество будет виртуальным, и будет не важно, где
вы живете в географическом смысле. Но я бы не стал на это полагаться.
Физический мир обладает очень большой пропускной способностью, и порой
города посылают сигналы очень малозаметными способами.

Город разговаривает с вами по большей части случайно – это то, что ты
видишь в окнах, это разговоры, которые ты случайно слышишь. Эти
сигналы не приходится специально искать, но и отключить их невозможно.
Моя подруга, которая переехала в Кремниевую долину в конце 1990-х,
сказала, что худшее там – это низкое качество подслушанных разговоров.
Тогда мне казалось, что это просто эксцентричное мнение. Конечно,
бывает интересно послушать чужие беседы, но неужели их качество так
важно, что ради этого ты готов сменить место проживания? Теперь я
понимаю, о чем она говорила: случайно подслушанные беседы говорят вам,
среди каких людей вы живете.


Каким бы целеустремленным человеком вы ни были, трудно не испытывать
влияние окружающих людей. И не то чтобы ты выполняешь все, чего от
тебя хочет город – но ты расстраиваешься, когда никого вокруг не
заботит то, что ты хочешь делать.

Между воодушевлением и разочарованием есть дисбаланс – как между
получением и потерей денег. Большинство людей переоценивают
отрицательные денежные величины: они работают гораздо усерднее, чтобы
не потерять доллар, чем чтобы заработать доллар. Точно так же, хотя
есть много людей, достаточно твердых, чтобы устоять перед каким-то
занятием, которое тебе навязывается лишь потому, что «все это делают»,
все же очень мало людей, настолько сильных, чтобы продолжать работать
над чем-то, на что всем вокруг наплевать.

Каждый город сосредоточен на каком-то определенном типе амбиций.
Кембридж – интеллектуальная столица не только потому, что там
сконцентрированы умные люди, но потому, что там нет ничего, что этих
людей заботило бы больше. В Нью-Йорке или Долине профессора – люди
второго сорта – пока они не учредят свой хедж-фонд или стартап.

Это подсказывает ответ на вопрос, который давно волнует нью-йоркцев:
сможет ли Нью-Йорк превратиться в центр стартапов, соперничающий с
Кремниевой долиной? Одна причина, почему вряд ли – то, что люди,
создающие стартапы в Нью-Йорке, будут чувствовать себя людьми второго
сорта. В Нью-Йорке людей восхищает нечто другое. И Нью-Йорк уже
проигрывает Кремниевой долине в игре богатства: соотношение
нью-йоркцев и калифорнийцев в списке Forbes 400 уже упало с 81:56 в
1982 году до 73:88 в 2007 году.


Не все города посылают сигналы, а лишь те, которые являются центрами
тех или иных амбиций. И не пожив в городе, бывает трудно сказать,
какого рода сигналы он испускает.

В Лос-Анджелесе главное, похоже, слава. Есть список людей, которые
пользуются сейчас наибольшим спросом, и главное достижение – попасть в
этот список или подружиться с кем-то из него. Ну а помимо этого
сигналы примерно такие же, как в Нью-Йорке, разве что с упором на
физическую привлекательность.

В Сан-Франциско сигналы, похоже, такие же, как в Беркли: нужно жить
лучше. Хотя это может измениться, если достаточно стартапов предпочтут
Сан-Франциско Кремниевой долине. В годы интернет-пузыря такие
предпочтения казались симптомом провала – все равно что покупать
дорогую офисную мебель. Даже сейчас стартапы, базирующиеся в
Сан-Франциско, вызывают у меня подозрения. Но если там поселится
достаточно хороших предпринимателей, центр притяжения переместится
туда.

Я не видел городов, равных Кембриджу по интеллектуальным амбициям.
Оксфорд и Кембридж (тот, что в Англии), похожи скорее на Итаку или
Ганновер: сигнал есть, но не такой сильный.

Париж когда-то был великим интеллектуальным центром. Но я попробовал
пожить там и увидел, что амбиции его жителей – не интеллектуального
плана. Сегодня Париж посылает сигнал: действуй стильно. Вообще-то мне
это понравилось. Париж – единственный город, где я жил, где люди
искренне интересуются искусством.

И вот еще один городской сигнал, что я уловил: в Лондоне все еще
(чуть-чуть) можно почувствовать побуждение быть более аристократичным.
Если прислушаться, подобное можно услышать и в Париже, Нью-Йорке или
Бостоне. Но в других местах этот сигнал очень слаб и неустойчив.


Вот полный список сигналов, что я уловил в разных городах: богатство,
стиль, мода, физическая привлекательность, слава, политическая власть,
экономическая власть, интеллект, классовое превосходство и качество
жизни.

От этого списка мне немного нехорошо. Я всегда считал амбиции делом
хорошим, но теперь я понимаю: это потому, что я всегда подразумевал
под этим амбиции в тех сферах, которые волнуют меня самого. Когда
перечисляешь все, по поводу чего возникают амбиции у амбициозных
людей, это выглядит не так симпатично.


Обязательно ли всякому, кто хочет добиваться отличных результатов,
жить в большом городе? Нет. Все большие города внушают определенные
амбиции, но для некоторых видов занятости все, что нужно – это горстка
талантливых коллег.

Города дают аудиторию и воронку, втягивающую равных вам людей. Это не
так важно в математике или физике, где вся аудитория – это ваши
коллеги, и все, что вам нужно – это факультет с правильными коллегами.
В таких областях, как искусство, журналистика или технология, важна
более общая среда. Там лучшие профессионалы не сосредоточены в
нескольких топовых факультетах и лабораториях. В этих более
хаотических областях выгоднее всего работать в большом городе: нужно
чувствовать все это поощрение, исходящее от окружающих и искать
коллег.

Вовсе не обязательно жить в большом городе всю жизнь, чтобы его
механизмы сработали на вас. Ключевые годы – это начало и середина
вашей карьеры. И конечно, не обязательно родиться и взрослеть в
большом городе. И даже не обязательно учиться в университете в таком
городе.

Но когда переходишь к следующему, более сложному шагу, лучше всего
находиться в месте, где ты можешь найти коллег и поддержку. Когда
найдешь и то, и другое, можно и уезжать. Возьмите импрессионистов: они
рождались и умирали по всей Франции, но определяющими для них были
годы, проведенные вместе в Париже.


Если вы еще не уверены, чем хотите заниматься, и где главный центр для
тех, кто занимается этим, то эффективнее всего, наверное, попробовать
пожить в нескольких городах, пока вы еще молоды. Никогда не знаешь,
какой сигнал подает город, пока не поживешь там. Иногда предположения
оказываются неверными: я переехал во Флоренцию в 25 лет, думая, что
увижу центр искусства, но оказалось, что я опоздал на 450 лет.

Даже когда город по-прежнему бурлит амбициями, невозможно знать точно,
будет ли резонировать его сигнал с вашими стремлениями, пока вы его не
услышите. Когда я переехал в Нью-Йорк, я сперва был восхищен. Это
волнующее место. И потребовалось немало времени, чтобы понять, что я
не совсем такой, как тамошние жители.

Некоторые люди уже в 16 лет знают, чем они будут заниматься. Но у
самых амбициозных молодых людей само наличие амбиции, кажется,
предшествует пониманию, по поводу чего эти амбиции возникают. Они
знают, что хотят сделать нечто великое. Но еще не решили, станут ли
рок-звездами или нейрохирургами. В этом нет ничего плохого. Но это
значит, что если и у вас амбиции этого самого распространенного типа,
то вам придется искать место для жизни методом проб и ошибок.
Наверное, вам придется найти город, где вы чувствуете себя как дома,
чтобы понять, какого же рода ваши амбиции.

\end{document}
