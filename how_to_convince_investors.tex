\documentclass[ebook,12pt,oneside,openany]{memoir}
\usepackage[utf8x]{inputenc} \usepackage[russian]{babel}
\usepackage[papersize={90mm,120mm}, margin=2mm]{geometry}
\sloppy
\usepackage{url} \title{Как убедить инвесторов} \author{Пол Грэм}
\date{}
\begin{document}
\maketitle

Травмы при поднятии тяжестей обычно происходят от того, что люди
напрягают мышцы спины, вместо мышц ног. Неопытные основатели совершают
похожую ошибку, когда пытаются убедить инвесторов своим питчем.
Большинству из них пошло бы на пользу, если бы сам стартап говорил за
себя, если бы они начали с попытки понять, почему в их стартап имеет
смысл инвестировать, а потом просто объяснили бы это инвесторам.

Инвесторы ищут стартапы, которые станут очень успешными. Легко
сказать! В стартапах, как и в некоторых других областях, распределение
стартапов по степени их успешности подчиняется степенному закону,
только в стартапах кривая распределения ещё круче. Очень успешные
стартапы настолько успешны, что затмевают собою все остальные. А
поскольку число крупных успехов в год можно по пальцем пересчитать
(есть мнение, что их в среднем 15 успешных стартапов в год), то
инвесторы вообще смотрят на стартапы в черно-белых очках – либо у вас
есть шанс стать одним из 15, либо нет. (Есть ещё бизнес-ангелы,
которым интересны стартапы средней степени успешности, но и от
крупного успеха они не откажутся.)

Как сделать так, чтобы Ваш стартап воспринимался как один из таких 15
очень успешных стартапов? Вам понадобятся 3 ингредиента: напористые
основатели крупный рынок текущие достижения.


Напористые основатели

Наиболее важный ингредиент – это основатели. Большинство инвесторов в
первые минуты решает, победители вы или лузеры, и как только это
мнение сформировалось, его уже трудно изменить. Любому стартапу
присущи как благоприятные факторы для инвестирования, так и
неблагоприятные. Если инвесторы думают о вас как о победителях, то они
придают больший вес благоприятным факторам. Если наоборот, то
наоборот. Например, у вашего стартапа крупный рынок, но каждая продажа
занимает длительное время. Если Вы понравились инвесторам, они захотят
инвестировать, поскольку это крупный рынок, а если нет, то они не
захотят из-за длинного цикла продаж.

Они не пытаются этим ввести вас в заблуждение, просто они сами порой
не понимают, почему им не приглянулся тот или иной стартап. Если вы
выглядите победителями, им больше понравится ваша идея. Инвесторы тоже
люди и им присущи все те же слабости, что и нам. Идея, конечно тоже
имеет значение. Идея — она как масло, которое подливается в огонь,
зажжённый хорошим расположением к основателям. Стоит только вам
понравиться инвесторам, им сразу же начнёт нравиться ваша идея: «да, и
вы могли бы ещё добавить фичу x.» (В то время как, если бы вы им не
приглянулись, они бы спросили вас: «да, но почему у вас нет фичи x?»)

Самое главное в убеждении инвесторов – это казаться напористыми.
Напористый человек – это тот, который всегда добивается своего,
независимо от обстоятельств. «Напористый» по смыслу близок к
«уверенный», за тем лишь исключением, что можно быть уверенным и
неправым одновременно. Напористые люди обоснованно уверенны в себе.

Немногим людям удается казаться напористыми — некоторым потому что они
и в самом деле напористые, другим потому что они умеют казаться. Но
большинству основателей, включая и тех, которые построили очень
успешные компании, не очень хорошо удается казаться напористыми в тот
самый первый раз, когда они приходят за инвестициями. Что же им
делать?

Чего совершенно точно не стоит делать, так это вести себя вызывающе,
как это иногда делают опытные основатели. Возможно инвесторы подчас не
так хороши в своих оценках технологий, но неуверенность они за версту
чуют. Если будете пытаться казаться кем-то, кем не являетесь, то
обнаружите себя там, где не хотели быть: с одной стороны вы
перестанете быть искренними, а с другой никогда не станете
убедительными.

Правда

Путь к напористости лежит через честность. Степень напористости,
которую Вы внушаете не постоянна. Она колеблется в зависимости от
того, что Вы говорите. Большиство людей выглядят уверенно, когда
говорят, что «один плюс один равно два», поскольку они знают, что это
правда. Хотя, сильно неуверенных в себе можно оставить в недоумении,
если даже «1+1=2» инвестор встретит со скептицизмом. Магическая
способность людей, умеющих казаться напористыми заключается в том, что
они могут искренне сказать «мы будем зарабатывать миллиард долларов в
год». Но и вы можете сказать подобное, если не про миллиард в год, то
о чем-нибудь менее внушительном, если только сумеете убедить себя в
этом.

Вот в чем дело! Убедите себя в том, что ваш стартап стоит того, чтобы
в него инвестировали, и когда Вы будете объяснять это инвесторам, они
вам поверят. И когда я говорю «убедите себя», я не имею в виду
психологические трюки для повышения самооценки. Я имею в виду –
оцените реально, достоен ли ваш стартап инвестирования. И если нет, то
не пытайтесь получить инвестиции. Но если да, то когда Вы скажете
инвесторам, что в Ваш стартап стоит инвестировать – они Вам поверят.
Вам не нужно быть мастером презентаций, чтобы объяснить то, что Вы
понимаете и во что Вы искренне верите.

Чтобы оценить стоит ли инвестировать в Ваш стартап, Вам придётся уметь
разбираться в своей теме. Если не разбираетесь, то какой бы
убедительной ни была ваша речь, она будет казаться инвесторам
очередным примером эффекта Даннинга-Крюгера. Чем она зачастую и
является. Инвесторы быстро определят разбираетесь ли вы в теме по
тому, как вы отвечаете на их вопросы. Так что изучите свою отрасль.
Почему основатели так настойчиво пытаются убедить инвесторов в том, во
что не верят сами? Частично потому, что нас учат этому со школьного
возраста.

Когда мои друзья, Роберт Моррис и Тревор Блэквелл были аспирантами,
профессор сказал одному из их студентов в конце доклада: В какие из
этих умозаключений Вы действительно верите?

Одним из побочных эффектов нынешней системы образования является наша
способность говорить даже тогда, когда нам нечего сказать. Если вам
нужно написать 10-страничную работу, тогда вы растягиваете её на
десять страниц, даже если идей у вас всего на одну страницу (это
относится и к данной работе). Многие стартапы поступают так же, когда
ищут инвестиции. Когда приходит пора готовить презентацию, они берут
канву фиксированного размера и размазывают по ней масляное масло,
только очень тонким слоем.

Искать инвестиции надо не тогда, когда они вам нужны, или когда
наступил день встречи с инвесторами, а тогда, когда Вы сможете убедить
инвесторов, и не раньше. Если только Вы не отличный актер, Вам не
убедить инвесторов в том, в чём Вы сами не уверены. Они отлично
распознают туфту, даже если Вы производите её не намеренно. Не тратьте
своё время, лучше остановитесь, подумайте и организуйте свои мысли.
Чтобы убедить себя в том, что стартап стоит того, чтобы в него
инвестировать, Вам предстоит выяснить почему в него имеет смысл
инвестировать. Само размышление над этим придаст Вам уверенности и
сформирует в голове план достижения успеха.

Рынок

Вы наверно уже обратили внимание на то, как аккуратно я подбираю
слова. Я отличаю стартапы в которые стоит инвестировать, от стартапов,
которые добьются успеха. Никто не знает, добъется ли стартап успеха.
Для инвесторов это тоже хорошо, поскольку если бы Вы знали заранее
какой стартап преуспеет, текущая оценка стоимости стартапа была бы уже
его будущей оценкой, а значит получить прибыль инвестируя было бы
невозможно. Инвесторы знают, что любая инвестиция – это ставка, причем
в очень неравной игре.

Поэтому чтобы доказать, что в вас стоит инвестировать, Вам нужно
доказать не то, что Вы преуспеете, а то, что Ваши шансы достаточно
хороши. Что делает Ваши шансы достаточно хорошими? В дополнение к
напористым основателям, Вам нужен реалистичный способ получения
большого куска большого рынка. Основатели думают о стартапах как об
идеях, а инвесторы думают о них, как о рынках. Если существует X
платящих клиентов, которые заплатят в среднем \$Y в год за то, что мы
делаем, тогда общий адресуемый рынок (TAM, Total Addressable Market)
Вашей компании = \$XY. Инвесторы не ждут, что Вы все эти деньги
получите, но это потолок того, что Вы можете получить.

Ваш целевой рынок должен быть крупным и осваиваемым. Не обязательно
крупным сейчас. На самом деле, иногда лучше начать на маленьком рынке,
который либо превратится в крупный, либо из которого Вы можете
двинуться в крупный. Просто должна быть реалистичная
последовательность шагов, которая приведет к доминирующей позиции на
крупном рынке через несколько лет.

Степень ожидаемой инвесторами реалистичности зависит от стадии
стартапа. Трехмесячной компании на демо-дне достаточно
сформулированного эксперимента для проверки гипотезы, требующего
денег. В то время как двухлетней компании поднимающий Раунд А, нужно
показать, что эксперимент сработал.

Каждой крупной компании, слегка повезло, они поймали какую-то внешнюю
волну. Вам тоже нужно обнаружить некий тренд, обычно задавая себе
вопрос «Почему сейчас?» Если это такая уж отличная идея, то почему
никто её ещё не реализовал? Идеальный ответ на этот вопрос звучит так:
«до недавнего времени эта идея была плохой, но сейчас кое-что
изменилось, и пока никто этого не заметил».

Майкрософт, например, вообще не делал ставку на BASIC. Но к тому
времени микрокомпьютеры становились достаточно мощными, чтобы
поддерживать его. А их распространение стало той самой волной, которую
никто не мог предугадать до 1975 года.

Хочется думать, что Майкрософт предвидел эту волну заранее, но скорее
всего это не так. Любая успешная компания в самом начале была не
более, чем просто хорошей ставкой. Майкрософт оказался отличной
ставкой, но в начале это было совсем не очевидно. Примерно половина
компаний, в которые мы инвестировали, казались такой же хорошей
ставкой, какой казался в своё время Майкрософт. Что ещё нужно
инвестору!

Отказы

Если Ваш стартап достоен инвестиций настолько же, насколько Майкрософт
был достоен инвестиций в своё время, удастся ли Вам убедить
инвесторов? Не обязательно. Многие инвесторы отказали бы и
Майкрософту. Многие отказали Гуглу. Отказы ставят Вас в трудное
положение, поскольку другие инвесторы задают Вам вопрос: «кто ещё
инвестирует? » Что ответить, если вы уже ищете инвестиции приличное
время, а никто ещё не согласился?

Люди, которым удается казаться напористыми часто решают эту проблему,
говоря инвесторам, что несмотря на то, что никто пока не инвестировал,
некоторые собираются это сделать. Это весьма сомнительная тактика.
Хотя это и отстойно, что инвесторы меньше интересуются Вашим
стартапом, и больше тем, как на него смотрят другие инвесторы, но
перелукавить лукавого у Вас вряд ли выйдет. Это самая распространенная
ложь инвесторам, и нужно хорошо уметь врать, чтобы они в неё поверили.

Если вы не профессиональный переговорщик (да даже если и
профессиональный), лучшим решением будет объяснить почему инвесторы
отвергли Вас и почему они ошибаются. Если Вы знаете, что на верном
пути, тогда Вам известно почему инвесторы ошибаются, отвергая Вас.
Опытные инвесторы знают, что самые лучшие идеи также самые пугающие.
Они прекрасно помнят инвесторов, отвергнувших Гугл. Вместо того, чтобы
уклончиво и стыдливо говорить о том, что Вас отвергли (тем самым
признавая, что те были правы), Вам нужно честно рассказать, что могло
отпугнуть инвесторов. В лучшем случае Вы будете казаться более
уверенными и лучше представите этот аспект Вашего стартапа. В худшем
случае, сказав правду, Вы освободите себя от неприятных сюрпризов в
будущем, когда неприятная правда вскроется.

Эта стратегия сработает с лучшими инвесторами, с которыми непросто
блефовать и которые уверены, что большинство других инвесторов – плохо
запрограммированные клоны, которые из раза в раз выпускают самородки
из рук. Поскольку лучшие инвесторы намного умнее остальных, а лучшие
идеи изначально кажутся плохими, то вероятнее всего Вас отвергнет
большинство инвесторов, за исключением немногих самых лучших. Это
случилось например с Dropbox. Y Combinator был запущен в Бостоне, и
первые 3 года мы попеременно работали то в Бостоне, то в Долине.
Поскольку Бостонских инвесторов было мало и они были все такие робкие,
мы привозили Бостонский поток на второй демо-день в Долину. Dropbox
был в Бостонском потоке, а значит на него сначала смотрели Бостонские
инвесторы, и ни один из них не вложился в него. Очередная штука для
бэкапа и синхронизации, думали они. А через пару недель, Dropbox
получил Раунд A от Sequoia.

Другой способ

Таким образом основатели думают, что им нужно продать инвесторам
что-то очень неопределенное – убедить их, что стартап станет огромным.
А инвесторы ждут от них кое-что гораздо менее спекулятивное –
подтверждение того, что компания обладает признаками хорошей ставки.
Понимая это, Вы можете подойти к решению этой задачи принципиально
другим способом – вы сможете убедить себя, и потом убедить их.

И чтобы убедить их, Вы воспользуетесь не той расплывчато-напыщенной
маркетинговой речёвкой, которую заготовили сначала, а той
аргументацией, которой Вам удалось убедить самих себя.

Просто будьте кратки. Многие инвесторы используют это в качестве
теста, поясняя, что если Вы не можете рассказать сжато свои планы, то
Вы их не полностью понимаете. Но даже если у инвестора и нет такого
правила, то ему все равно будет неприятно выслушивать неясные
объяснения.

Итак, вот рецепт того, как впечатлить инвесторов, даже если Вы не
кажетесь такими уж напористыми: Сделайте что-то, во что имеет смысл
инвестировать. Поймите почему в это имеет смысл инвестировать.
Объясните это инвесторам.


Если Вы говорите правду, Вы будете выглядеть уверенными. У кого правда
тот и сильнее.
\end{document}
