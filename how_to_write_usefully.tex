\documentclass[ebook,12pt,oneside,openany]{memoir}
\usepackage[utf8x]{inputenc} \usepackage[russian]{babel}
\usepackage[papersize={90mm,120mm}, margin=2mm]{geometry}
\sloppy
\usepackage{url} \title{Как писать полезные тексты} \author{Пол Грэм}
\date{}
\begin{document}
\maketitle

Каким должно быть эссе? Многие сказали бы — убедительным. Так нас
учили… Но я думаю, что мы можем стремиться к чему-то более
амбициозному: эссе должно быть полезным.

Прежде всего эссе должно быть корректным. Но недостаточно просто быть
корректным. Легко сделать утверждение корректным, сделав его
расплывчатым. Это распространенный недостаток академического письма.
Даже если вы ничего не знаете о предмете, вы не ошибетесь, сказав, что
проблема сложная, что есть много факторов, которые следует учитывать,
что слишком простой взгляд на неё был бы ошибкой, и так далее.

Такие утверждения, несомненно, корректны, но они ничего не сообщают
читателю. Хороший текст содержит в себе сильные высказывания —
настолько сильные, что сделать их еще немного сильнее означало бы
превратить их в ложь.

Например, полезнее было бы сказать, что Пайкс-Пик находится рядом с
центром штата Колорадо, чем просто где-то в Колорадо. Но если я скажу,
что это точно в центре Колорадо, окажется, что я преувеличил, так как
на самом деле это немного восточнее середины.

Точность и корректность подобны противостоящим силам. Легко
удовлетворить одну, игнорируя другую. Поддержание академического стиля
— смело, но фальшиво, это риторика и демагогия. Полезное писательство
— смело и правдиво.

Здесь также есть 2 других момента: говорить людям что-то важное и то,
что, по крайней мере, некоторые из них еще не знают.

Говорить людям то, что они не знают, не всегда означает удивлять их.
Иногда это означает говорить им то, что они бессознательно знали, но
никогда не выражали при помощи слов. На самом деле это может дать
более ценные прозрения, потому что имеет тенденцию быть более
фундаментальными.

Давайте объединим все это вместе. Полезное эссе говорит людям что-то
правдивое и важное, то, что они еще не знали, и делает это настолько
однозначно, насколько это возможно.

Заметьте, что все это вопрос уровня. Например, вы не можете ожидать,
что идея будет нова для всех. Любое прозрение, которое было у вас,
вероятно также было хотя бы у одного из 7 миллиардов людей, живущих в
мире. Но на самом деле достаточно, если идея будет новой для большого
числа читателей.

То же самое касается правильности, важности и прямоты. По сути, эти
четыре компонента подобны числам, которые вы можете перемножить, чтобы
получить оценку полезности. Я признаю, что это неуклюжее упрощение, но
тем не менее — верное.

Как вы можете быть уверены, что то, что вы говорите, истинно, ново и
важно? Хотите, верьте, хотите нет, но для этого есть хитрость. Я
научился этому у своего друга Роберта Морриса, который боится говорить
глупости. Его трюк — ничего не говорить, пока он не уверен, что это
стоит услышать. Поэтому трудно получить его мнение, зато когда это
удаётся, оно обычно верное.

В смысле создания эссе это означает, что если вы пишете плохое
предложение, вы его не публикуете. Вы удаляете его и повторяете
попытку. Часто вы отказываетесь от целых веток повествования из
четырех или пяти абзацев. Иногда — от целого очерка.

Невозможно гарантировать, что каждая идея, пришедшая в голову, хороша,
но можно удостовериться, что хороша каждая опубликованная. Просто не
публикуйте те, что неудачны.

В науке это называется систематическая ошибка отбора и считается
помехой. Когда некоторая гипотеза приводит к неубедительным
результатам, в любом случае полагается донести их до людей. Но при
написании эссе отбор это естественный способ.

Моя стратегия: ослабить, а затем затянуть. Я пишу первый набросок эссе
быстро, пробуя все идеи. А затем провожу долгое время, тщательно его
переписывая.

Я никогда не пытался подсчитать, сколько раз я корректировал
(proofread) эссе, но я уверен, что есть предложения, которые я
прочитал 100 раз, прежде чем публиковать их. Когда я корректирую эссе,
обычно есть отрывки, которые выглядят раздражающими, иногда потому,
что они неуклюже написаны, а иногда потому, что я не уверен, что они
правдивы. Я раздражаюсь неосознанно, но после десятого чтения или
около того я говорю «Тьфу, вот этот кусок» каждый раз, когда
отлавливаю его. Они становятся как быки, которые кусают твой рукав,
когда ты проходишь мимо. Обычно не публикую эссе, пока они все не
исчезнут, — пока я не смогу прочитать всё без зацепок.

Иногда я пропускаю предложение, которое кажется неуклюжим, если не
могу придумать как его перефразировать, но я никогда сознательно не
пропущу предложение, которое кажется неправильным. И вы никогда не
должны это пропускать. Если предложение кажется неправильным, всё, что
вам нужно сделать, это спросить, почему это не так, и обычно сразу в
вашей голове возникает замена.

Здесь эссеисты имеют преимущество перед журналистами. У вас нет
крайнего срока. Вы можете работать над сочинением столько времени,
сколько вам нужно, чтобы все было правильно. Вам не нужно публиковать
эссе вообще, если вы не можете сделать его таким, как надо. Ошибки,
похоже, теряют мужество перед лицом врага с неограниченными ресурсами.
Ну, или это так чувствуется. Что на самом деле происходит, так это то,
что у тебя разные ожидания от себя самого. Ты как родитель, говорящий
ребенку: «Мы можем сидеть здесь всю ночь, пока ты не съешь свои
овощи». Есть только один нюанс — ты и есть этот ребенок.

Я не говорю писать вообще без ошибок. Например, я добавил условие (с)
в «Способ обнаружить предвзятости» после того, как читатели отметили,
что я его пропустил. Но на практике вы можете поймать почти все из
них.

Также есть трюк, чтобы донести важность. Он похож на тот, что я
предлагаю молодым основателям при выборе идей для стартапа: делайте
то, в чем лично вы нуждаетесь. Вы можете использовать себя как
посредника для читателя. Читатель совершенно не похож на вас. Пишите о
важных для себя темах, тогда они возможно покажутся важными
значительному количеству читателей.

Есть два фактора, которые определяют важность. Число людей, для
которых это имеет значение, умноженное на величину этой значимости.
Конечно, это не значит, что мы получаем прямоугольник. Скорее нечто
вроде рваного гребня, как график функции суммы Римана.

Чтобы достичь новизны, следует писать на те темы, на которые вы много
думали. В таком случае, вы можете стать проводником в эту область
также и для читателя. Всё, что вызвало удивление у вас — того, кто
много думал на эту тему — наверняка удивит и значительное число
читателей. И тут, чтобы убедиться в этом, также как с корректностью и
важностью, вы можете использовать технику Морриса. Не публикуйте эссе,
если из его написания вы не извлекли ничего нового для себя.

Чтобы оценить новизну, вы должны быть скромным — ведь осознавая
новшество идеи, вы осознаете, что не замечали его в прошлом.
Уверенность и скромность часто противопоставляются друг другу, но в
этом случае, как и во многих других, уверенность помогает вам
смириться. Когда вы знаете, что являетесь экспертом в какой-либо
области, вы можете смело признать, что поняли что-то, чего не знали
раньше, потому что будете уверены, что большинство людей скорее всего
тоже об этом не знали.

Четвертый компонент практического письма, убедительность, складывается
из 2 аспектов: четкий ход мысли и умелое использование оговорок. Эти
аспекты уравновешивают друг друга, как педаль газа и сцепление в
автомобилях с механической коробкой передач. По мере шлифования
формулировки идеи, вы соответственно корректируете и оговорки. Знайте,
вы можете не делать оговорок вообще, если говорите напрямик, как в
моем случае с четырьмя компонентами практического письма.
Неоднозначные тезисы также не должны вызывать никаких догадок.

В процессе доработки идеи вы стремитесь свести к минимуму количество
оговорок. Однако, крайне редко получается вовсе отказаться от них.
Иногда и не стоит этого делать, например, если это дополнительные
аргументы и полностью доработанная версия получается слишком длинной.

Некоторые считают, что оговорки делают письмо неубедительным.
Например, что вы никогда не должны начинать предложение в эссе со слов
“Я считаю”, потому что если вы пишите об этом, очевидно, что вы так
считаете. Действительно, утверждение “Я считаю, что х” звучит слабее,
чем просто “х”. Как раз поэтому вам и нужно это “Я считаю”. Это нужно
для выражения степени вашей уверенности.

Но оговорки не скаляры. Это не просто экспериментальная ошибка.
Существует 50 вещей, которые они могут выразить: насколько широко
что-то применимо, как вы это знаете, насколько вы счастливы, даже, как
это может быть сфальсифицировано. Я не собираюсь пытаться исследовать
структуру оговорки здесь. Это, вероятно, сложнее, чем вся тема о
написании с пользой. Вместо этого я просто дам вам практический совет:
не стоит недооценивать оговорку. Это важный навык сам по себе, а не
просто какой-то налог, который вы должны заплатить, чтобы избежать
ложных высказываний. Так что учитесь и используйте его в полном
объеме. Возможно, это не половина того, чтобы иметь хорошие идеи, но
это часть их наличия.

В эссе я стремлюсь к еще одному качеству: объяснять что-то как можно
проще. Но я не думаю, что это компонент полезности. Это скорее вопрос
для размышления читателям. И это практическая помощь в понимании
вещей; ошибка более очевидна, когда выражается простым языком. Но я
признаю, что основная причина, по которой я пишу, — это не ради
читателя или потому, что это помогает понять все правильно, а потому,
что это мешает мне использовать больше слов или более причудливые
слова, чем мне нужно. Это кажется не элегантным, как слишком длинная
программа.

Я понимаю, что некоторые люди пишут замысловато. Но если вы уверены,
что являетесь одним из них, то лучший совет — писать как можно проще.
Я верю, что формула, которую я вам дал, важность + новизна +
корректность + сила — это рецепт хорошего эссе. Но я должен
предупредить вас, что это также рецепт для того, чтобы сводить людей с
ума.

Корень проблемы — новизна. Когда вы говорите людям что-то, чего они не
знали, они не всегда благодарят вас за это. Иногда причина, по которой
люди чего-то не знают, заключается в том, что они не хотят этого
знать. Обычно потому, что это противоречит какой-то заветной вере. И
действительно, если вы ищете новые идеи, популярные, но ошибочные
убеждения являются хорошим местом для их поиска. Каждое
распространенное ошибочное убеждение создает вокруг себя мертвую зону
идей, которые относительно не исследованы, поскольку противоречат ей.
Компонент силы только ухудшает положение. Если есть что-то, что
раздражает людей больше, чем противоречие их заветным предубеждениям,
то это категорически противоречит им.

Плюс, если вы использовали технику Морриса, написанное вами будет
казаться довольно уверенным. Возможно, оскорбительно уверенным для
людей, которые не согласны с вами. Причина, по которой вы будете
казаться таким заключается в том, что вы уверены: вы схитрили,
публикуя только то, в чем вы не сомневаетесь. Людям, которые пытаются
с вами спорить, покажется, что вы никогда не признаете свою неправоту.
На самом деле вы постоянно признаете, что ошибаетесь. Вы просто
делаете это до публикации, а не после.

И если ваш текст настолько прост, насколько это возможно, это только
ухудшает ситуацию. Краткость — это умение командовать. Если вы
наблюдаете, как кто-то передает нежелательные новости с позиции
низкого качества, вы заметите, что они склонны использовать много
слов, чтобы смягчить удар. В то время как для того чтобы быть кратким
с кем-то означает быть с ним грубым.

Иногда это используется, чтобы ослабить категоричность высказывания.
Поставить «возможно» перед чем-то, в чем вы действительно уверены. Но
вы заметите, что когда авторы делают это, они обычно делают это
подмигивая читателю.

Я не очень люблю это делать. Глупо использовать ироничный тон для
всего эссе. Я думаю, что мы просто должны признать тот факт, что
элегантность и лаконичность-это два названия для одной и той же вещи.

Вы можете подумать, что если вы достаточно усердно работаете, чтобы
убедиться, что эссе написано правильно, оно будет неуязвимо. Это в
некотором роде правда. Оно будет неуязвимо для обоснованных атак. Но
на практике это слабое утешение.

На самом деле, компонент силы полезного написания сделает вас особенно
уязвимым для искажения информации. Если вы изложили свою идею со всей
уверенностью, не делая ее ложной, все, что вам нужно сделать, — это
слегка преувеличить то, что вы сказали, и теперь это ложь.

По большей части они даже не делают это сознательно. Одна из самых
удивительных вещей, которые вы обнаружите, если начнете писать эссе,
это то, что люди, не согласные с вами, редко не соглашаются с тем, что
вы на самом деле написали. Вместо этого они интерпретируют то, что вы
сказали, и не согласны с этим.

Для этого стоит попросить кого-то, кто поступает так, процитировать
конкретное предложение или отрывок, которые вы написали, которые они
считают ложными, и объяснить, почему. Я говорю «для этого стоит»,
потому что они никогда так не делают. Поэтому, хотя может показаться,
что это может привести к провалу в дискуссии, истина заключается в
том, что она никогда и не была на верном пути.

Следует ли вам явно предупреждать возможные неверные интерпретации?
Да, если они неверно истолкованы, то достаточно умный и
благонамеренный человек может сделать это. На самом деле иногда лучше
сказать что-то слегка вводящее в заблуждение, а затем добавить
поправку, чем пытаться сформулировать идею сразу правильно. Это может
быть более эффективным, а также может моделировать способ обнаружения
такой идеи.

Но я не думаю, что вы должны явно предупреждать намеренные неверные
интерпретации в структуре эссе. Эссе — это место встречи с честными
читателями. Вы же не хотите испортить свой дом, поставив решетки на
окна, чтобы защитить их от нечестных людей. Место для защиты от
преднамеренных неверных толкований находится в конечных примечаниях.
Но не думайте, что вы можете предсказать их все. Люди так же
изобретательны в том, чтобы исказить вас, когда вы говорите то, что
они не хотят слышать, как и в том, чтобы придумать рациональные
объяснения для того, что они хотят сделать, но знают, что не должны
этого делать.

Как и в любом другом деле, чтобы стать лучше в написании эссе, нужно
практиковаться. Но с чего начать? Теперь, когда мы рассмотрели
структуру полезного письма, мы можем перефразировать этот вопрос более
точно. Какое ограничение вы ослабляете изначально? Ответ таков: первый
компонент важности: количество людей, которым небезразлично то, что вы
пишете.

Если вы достаточно сузите тему, то, вероятно, сможете найти что-то, в
чем вы являетесь экспертом. Напиши об этом для начала. Если у вас есть
только десять читателей, которым не все равно, это прекрасно. Ты
помогаешь им, и ты пишешь. Позже вы можете расширить круг тем, о
которых пишете.

Другое ограничение, которое вы можете ослабить, немного удивляет:
публикация. Написание эссе вовсе не обязательно означает их
публикацию. Это может показаться странным сейчас, когда существует
тенденция публиковать каждую случайную мысль, но это сработало для
меня. Я писал то, что составляло эссе в тетрадях, около 15 лет. Я
никогда не публиковал ни одного из них и никогда не ожидал этого. Я
написал их как способ разобраться в происходящем. Но когда появилась
сеть, у меня было много практики.

Кстати, Стив Возняк сделал то же самое. В старших классах он для
забавы создавал компьютеры на бумаге. Он не мог их построить, потому
что не мог позволить себе купить компоненты. Но когда Intel запустила
4K DRAMs в 1975 году, он был готов.

А сколько еще эссе осталось написать? Ответ на этот вопрос, вероятно,
является самой невероятной мыслью, которую я понял о написании эссе.
Почти все из них осталось написать.

Хотя эссе — это старая форма, она не была усердно культивирована. В
печатную эпоху публикация была дорогой, и не было достаточного спроса
на эссе, чтобы публиковать их так много. Вы могли бы публиковать эссе,
если бы уже были хорошо известны тем, что пишете что — то другое,
например романы. Или вы можете написать рецензии на книги, которые вы
взяли на себя, чтобы выразить свои собственные идеи. Но на самом деле
прямого пути к тому, чтобы стать эссеистом, не было. А это означало,
что писалось очень мало эссе, а те, что писались, как правило, были
посвящены узкому кругу тем. Теперь, благодаря интернету, есть путь.
Любой желающий может опубликовать эссе в интернете. Вы начинаете в
неизвестности, возможно, но, по крайней мере, вы можете начать. Тебе
не нужно ничье разрешение.

Иногда случается, что та или иная область знания спокойно сидит
годами, пока какое-нибудь изменение не заставит ее взорваться.
Криптография сделала то же самое с теорией чисел. Интернет делает то
же самое с эссе.

Самое захватывающее — это не то, что нам еще многое предстоит
написать, а то, что нам еще многое предстоит открыть. Есть
определенные идеи, которые лучше всего раскрываются через написание
эссе. Если большинство эссе все еще не написаны, то большинство таких
идей все еще не открыты.

Примечания

[1] Поставьте перила на балконах, но не ставьте решетки на окнах.

[2] Даже сейчас я иногда пишу эссе, которые не предназначены для
публикации. Я написал несколько, чтобы выяснить, что должен делать Y
Combinator, и они были действительно полезны.

\end{document}
