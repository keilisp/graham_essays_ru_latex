\documentclass[ebook,12pt,oneside,openany]{memoir}
\usepackage[utf8x]{inputenc} \usepackage[russian]{babel}
\usepackage[papersize={90mm,120mm}, margin=2mm]{geometry}
\sloppy
\usepackage{url} \title{Почему не появляются новые Google} \author{Пол
  Грэм} \date{}
\begin{document}
\maketitle

Умар Хак (Umair Haque) недавно написал, что причина, по которой не
появляются новые Google, состоит в том, что большинство стартапов
покупаются до того, как они могли бы изменить мир.

Google, несмотря на серьезный интерес со стороны Microsoft и Yahoo --
что, должно быть, казалось выгодным в то время -- не продался. Google
мог бы стать всего лишь поисковым движком для Yahoo или MSN.

Почему же этого не случилось? Потому что у Google была чёткая и ясная
цель: твердая убежденность изменить мир к лучшему.

Звучит красиво, но это -- неправда. Основатели Google были готовы
продать свое детище на ранних этапах. Просто они хотели больше, чем
предлагали покупатели.

Так же было и с Facebook. Они бы продались, но Yahoo погасил их мечты,
предложив слишком мало.

Небольшой совет потенциальным покупателям: если стартап не соглашается
на ваше предложение, подумайте о том, что бы предложить больше, потому
что есть неплохие шансы, что чрезмерная цена, которую они просят
сейчас, в будущем покажется выгодной сделкой.

Мой опыт показывает, что стартапы, которые отвергают предложения о
покупке, часто преуспевают впоследствии. Как правило, хотя и не
всегда, потом появляются более выгодные предложения, а иногда компания
даже размещает свои акции на бирже.

Конечно, причина того, что стартапы успешно развиваются после того,
как они отвергают предложение о покупке, не обязательно состоит в том,
что предлагалась слишком низкая цена. Скорее причина в том, что
основатели, которым хватает духу отвергнуть выгодное предложение,
обычно сами по себе бывают успешными. В них именно такой дух, который
нужен стартапу.

Хотя я и уверен в том, что Ларри и Сергей действительно хотят изменить
мир, по крайней мере сейчас, причина, по которой Google выжил и стал
большой независимой компанией та же самая, по которой Facebook
остаётся независимым: покупатели недооценили их.

Корпорации, занимающиеся покупками компаний, ведут довольно странный
бизнес с этой точки зрения. Они постоянно пропускают выгодные сделки,
не понимая того, что отказ от сделки это самый надёжный индикатор
того, что стартап преуспеет, из всех индикаторов, которые вы могли бы
придумать.

Венчурные капиталисты

Итак, какова же причина того, что нет других Google? Забавно, причина
та же самая, по которой Google и Facebook остаются независимыми: парни
с деньгами недооценивают большинство новаторских стартапов.

Причина, по которой мы не видим новых Google не в том, что инвесторы
побуждают новые стартапы продаваться, а в том, что они даже не
инвестируют в них. Я многое узнал о венчурных капиталистах за
последние 3 года, работая над Y Combinator'ом, так как мы с ними
взаимодействовали довольно плотно. Больше всего меня удивило то,
насколько они консервативны. Об инвестиционных компаниях часто думают,
что они поддерживают самые дерзкие идеи. В действительности, только
немногие компании так поступают и даже они более консервативны в
реальности, чем вы могли бы предположить читая тексты на их сайтах.

Раньше венчурные капиталисты представлялись мне кем-то вроде пиратов:
отважные и беспринципные. При более близком знакомстве они оказались
более похожими на бюрократов. Они более честны, чем я думал (по
крайней мере, их лучшие представители), но менее отважны. Возможно
инвестиционная индустрия изменилась с тех пор. Возможно раньше
инвесторы были более отважными. Но мне кажется, что изменились не они,
а мир стартапов. Низкая стоимость запуска стартапа означает, что
средняя ""хорошая ставка"" -- это рискованная ставка, однако
большинство венчурных фирм продолжают работать так же, как если бы они
инвестировали в стартапы, производящие оборудование в 1985 году.

Говард Айкен (Howard Aiken) сказал: "Не беспокойтесь о людях, которые
воруют ваши идеи. Если ваши идеи хоть сколько нибудь хороши, вам
придётся вдалбливать их в других людей". У меня возникает подобное
чувство, когда я пытаюсь убедить венчурного капиталиста инвестировать
в стартапы, основанные Y Combinator'ом. Действительно новые идеи
вызывают у них ужас, только если основатели не компенсируют это своими
хорошими навыками продавцов.

Но наибольшую отдачу приносят именно смелые идеи. Любая действительно
хорошая новая идея большинству людей покажется плохой; иначе кто-то бы
уже сделал это. И тем не менее большинство венчурных капиталистов
подвержены "консенсусу", не только внутри своих фирм, но и во всём
своём сообществе. Самым большим фактором, который определит, что
почувствует венчурный капиталист по отношению к вашему стартапу,
является то, что чувствуют по отношению к нему другие венчурные
капиталисты. Сомневаюсь, что они осознают это, но такой алгоритм
гарантирует, что они упустят все лучшие идеи. Чем большему числу людей
понравится новая идея, тем меньше несогласных вы услышите.

Кто бы ни был будущим Google, скорее всего прямо сейчас какой-нибудь
инвестор говорит им, чтобы они пришли ещё раз, когда у них будет
больше шансов на успех.

Почему венчуры столь консервативны? Возможно тут играет роль сочетание
факторов. Консервативными их делает большой размер их инвестиций. К
тому же они инвестируют деньги других людей, что заставляет инвесторов
беспокоиться о том, что у них могут быть проблемы, если они рискнут и
начинание провалится. Кроме того, большинство из них финансисты, а не
технические специалисты, т.е. они не понимают, во что они инвестируют
в действительности.

Что дальше

Замечательная особенность рыночной экономики состоит в том, что
недальновидность означает возможности. В данном случае это именно так.
Существуют огромные, неоткрытые возможности в инвестировании
стартапов. Y Combinator финансирует стартапы в самом начале их
деятельности. Венчурные капиталисты стали бы финансировать их только
после того, как те начали приносить ощутимые результаты. Но между
этими двумя событиями существует довольно большой отрезок времени.

Существуют компании, которые могут дать 20 тысяч долларов стартапу,
который пока ещё ничего не сделал, и существуют компании, которые
могут дать 2 миллиона долларов стартапу, который уже добился успеха,
но существует дефицит инвесторов, которые могут дать 200 тысяч
долларов стартапу, выглядящему многообещающе, но с которым пока ещё не
всё ясно. На этой территории встречаются в основном "ангелы", частные
инвесторы -- такие люди как Энди Бехтолшейм (Andy Bechtolsheim),
который дал Google 100 тысяч долларов в то время, когда он уже
выглядел перспективным, но до прибыльной компании ему было ещё далеко.
Мне нравятся "ангелы", но их не так много, и инвестирование для них,
как правило, всего лишь часть их деятельности.

Поскольку запуск стартапов становится всё дешевле, эта неосвоенная
территория становится всё более и более ценной. В наши дни большое
количество стартапов не хотят заключения многомиллионых инвестиционных
сделок на первом этапе, в раунде А. Им не нужно столько денег и они не
хотят ввязываться в решение проблем, связанных с такими сделками.
Средний стартап, выходящий из Y Combinator, хочет получить 250-500
тысяч долларов. Но когда они идут к венчурным капиталистам, им
приходится запрашивать больше, потому что венчурам не интересно иметь
дело с такими небольшими суммами.

Венчурные капиталисты управляют деньгами. Они высматривают способы
заставить работать большие суммы. Но мир стартапов развивается вне их
текущей модели.

Стартапы стали дешевле. Это означает, что для них требуется меньше
денег, но это также значит, что их число растёт. Потому вы всё ещё
можете получить большую отдачу от крупных денежных вложений, просто
придётся распределять по большему количеству инвестируемых компаний.

Я пытался объяснить это инвестиционным компаниям. Вместо одной сделки
на 2 миллиона заключите пять сделок по 400 тысяч. Не означает ли это,
что нужно присутствовать слишком во многих советах директоров? Не
присутствуйте в их советах директоров. Не слишком ли много экспертизы
(due diligence) потребуется проводить? Проводите меньше. Если вы
инвестируете десятую часть стоимости, вам нужно быть уверенным в
успехе всего лишь на десятую часть.

Это выглядит очевидным. Но когда я предложил нескольким венчурным
компаниям выделить немного денег и определить одного партнёра, который
сделает много инвестиций меньшего размера, они отреагировали так, как
будто я предложил им вставить в нос кольца. Удивительно, как они
привязаны к своим стандартам инвестирования.

Эта область очень перспективна, и так или иначе она будет заполнена.
Или существующие венчурные компании изменятся и заполнят её или, что
более вероятно, появятся новые инвесторы. Будет отлично, когда это
произойдёт, потому что эти новые инвесторы, самой структурой своих
инвестиций будут вынуждены быть более смелыми, чем существующие
венчурные капиталисты. И в результате будут появляться много новых
Google. По крайней мере до тех пор, пока покупатели будут оставаться
такими недалекими, как сейчас.

Примечания

[1] Ещё один совет: Если вы действительно хотите получить отдачу, не
уничтожайте стартап после приобретения. Предоставьте учредителям
достаточно независимости, чтобы они могли вырастить приобретение в то,
чем оно должно стать.

\end{document}
