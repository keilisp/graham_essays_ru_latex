\documentclass[ebook,12pt,oneside,openany]{memoir}
\usepackage[utf8x]{inputenc} \usepackage[russian]{babel}
\usepackage[papersize={90mm,120mm}, margin=2mm]{geometry}
\sloppy
\usepackage{url} \title{Планшеты} \author{Пол Грэм} \date{}
\begin{document}
\maketitle

Недавно я размышлял: «Как же неудобно, что у нас нет одного общего
термина для таких устройств как Айфон, Айпэд и похожих штук,
работающих под Андройдом». Самое близкое из того, что есть это похоже
«мобильные устройства», но это, во-первых, относится и к любым сотовым
телефонам, а, во-вторых, в действительности не отражает важных
особенностей Айпэда.

Но после секундного замешательства, я понял, мы будем называть их
«интернет-планшетами» или просто «планшетами». Единственная причина,
по которой мы подумали о термине «мобильные устройства» в том, что
Айфон появился раньше Айпэда. Если бы сначала был Айпэд, мы бы никогда
не рассматривали Айфон в качестве телефона. Мы бы воспринимали его как
планшет, размер которого позволяет подносить его к уху.

— Как мы их будем называть, не до конца понятно :) Сейчас
распространены варианты "планшет", "интернет-планшет", "планшетный
компьютер". Есть еще такая сложность, что планшетом часто называют
"графический планшет", устройство отчасти другого толка, хотя очень
дорогие модели по смыслу схожи с планшетными компьютерами (рисуют
прямо по экрану). Я думаю со временем останется только слово
"планшет", жизнь покажет :) — design-monster

Айфон не столько телефон, сколько устройство, способное заменить
телефон. А это большая разница, потому что такой подход станет очень
распространенным. Множество, если не большинство, окружающих нас
предметов, выполняющих строго определенную задачу, будет заменено
приложениями на планшетных компьютерах.

Это уже очевидно в отношении GPS-устройств, музыкальных плейеров,
видео- и фотокамер. Но я думаю, люди удивятся тому, как много
предметов будет заменено. Мы финансируем один стартап, который
позволит избавиться от ключей. Тот факт, что вы можете менять размер
шрифта у текста, попросту означает что Айпэд успешно заменяет очки для
чтения. Я не удивлюсь, если в результате хитрых манипуляций со
встроенным в Айпэд акселерометром, вы даже сможете заменить напольные
весы.

Преимущества замены различных предметов на приложения в одном
единственном устройстве настолько огромны, что все, что может
превратиться в приложение станет таковым. Так что, хороший рецепт для
стартапа на ближайшие несколько лет – это поискать вокруг себя такие
предметы, которые пока еще не догадались заменить приложениями для
планшета.

В 1938 году Бакминстер Фуллер придумал термин «эфемеризация», чтобы
описать стремительную тенденцию к замене материальных,
физически-существующих механизмов на виртуальные, на то, что мы бы
сейчас назвали «программным обеспечением». Причина, по которой
планшеты заполонят весь мир, не только в том, что Стив Джобс и
компания являются волшебниками от промдизайна, но и в том, что за ними
стоит эта самая сила. Айфон и Айпэд открыли дверь для эфемеризации,
впустив ее во множество новых сферы. Ни один человек, изучавший
историю развития технологий, не осмелится недооценивать силу
тенденции, открытой Фуллером.

И меня беспокоит эта мощь, которую компания «Эппл» получит благодаря
силе, стоящей за ней. Я не хочу увидеть очередную эпоху
компании-монополиста, как это было с «Майкрософт» в 80-х и 90-х. Но
если эфемеризация та главная сила, которая сделает планшеты
популярными, то она же подсказывает как конкурировать с «Эппл» – нужно
создать платформу для приложений, которая была бы еще лучше.

Отличной идеей со стороны «Эппл» оказалось встроить в планшеты
акселерометр. Разработчики приложений начали его использовать
способами, которые даже не могли прийти в голову сотрудникам «Эппл».
Такова природа создаваемых платформ. Чем большими возможностями
обладает устройство, тем сложнее предсказать, как именно люди будут
его использовать. Поэтому производителям планшетов стоит задуматься:
«Что же еще мы может встроить вовнутрь?» И не обязательно на уровне
«железа», но и на уровне программ. Что еще мы можем предложить
разработчикам? Дайте этим парням палец, и они откусят руку по локоть!

\end{document}
