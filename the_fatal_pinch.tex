\documentclass[ebook,12pt,oneside,openany]{memoir}
\usepackage[utf8x]{inputenc} \usepackage[russian]{babel}
\usepackage[papersize={90mm,120mm}, margin=2mm]{geometry}
\sloppy
\usepackage{url} \title{Последняя капля} \author{Пол Грэм} \date{}
\begin{document}
\maketitle

Многие стартапы за пару месяцев до кончины проходят через такой этап,
когда при значительных средствах на счетах они тратят слишком много.
Выручка при этом либо незначительна, либо вообще отсутствует. У
компании есть, скажем, 6 месяцев на исправление ситуации. А если
называть вещи своими именами, то 6 месяцев до банкротства. И такие
стартапы рассчитывают избежать провала с помощью дополнительных
инвестиций. [1]

В этом-то и кроется ошибка.

Уверенность основателей в том, что инвестор захочет их дополнительно
профинансировать, вполне может оказаться иллюзией. Хотя убедить
инвестора было непросто и в первый раз, предприниматель все равно
рассчитывает на положительный ответ. Но во второй раз он столкнется с
неожиданными проблемами:

Сейчас компания тратит больше денег, чем при первоначальном
финансировании. К уже получившим инвестиции стартапам требования выше.
К компании теперь относятся как к провальной. Когда деньги выдавались
в первый раз, еще слишком рано было предсказывать успех или неудачу.
Теперь же самое время задуматься, и обычно все склоняются к «неудаче».
Логично — ведь результат вполне типичный для начинающей компании.

Я называю такую ситуацию «последней каплей». Не люблю навязывать
штампы, но такое название может немного отрезвить предпринимателя.

Ситуацию делает особенно опасной тот факт, что бизнесмены сами же все
усугубляют. Они слишком рассчитывают на дополнительное финансирование,
и поэтому не особенно стараются выйти в «плюс» самостоятельно. А это
еще больше снижает шансы на благополучный выход из кризиса.

“Чем меньше вы нуждаетесь в деньгах извне, тем легче их получить”.

Теперь вы знаете о проблеме последней капли, но что с этим делать?
Разумеется, кроме очевидного совета не допускать подобного.
Инвестиционная компания Y Combinator советует относиться к инвестициям
так, как будто это последнее вливание в ваш бизнес. Ведь сила
самоубеждения работает и в обратную сторону — чем меньше вы нуждаетесь
в деньгах извне, тем легче их получить.

Но что делать, если положении уже стало незавидным? Во-первых, еще раз
взвесьте вероятность дополнительных инвестиций. Сейчас я побуду
ясновидящим и предскажу, что вероятность равна нулю. [2]

Остается три варианта: можно закрыть компанию, увеличить прибыль, или
снизить расходы.

Если вы действительно уверены в провале, то компанию лучше свернуть.
Как минимум, сможете вернуть оставшиеся деньги и сэкономить силы на
бессмысленных попытках все исправить.

Finish/Start


На самом деле, действительно обреченные компании встречаются не часто
— я просто даю вам еще один шанс признать, что вы уже сдались.

Если закрывать компанию не хочется, то остается путь увеличения
прибыли или снижения издержек. Для большинства стартапов расходы =
сотрудники и снижение расходов = увольнения. [3] Решиться на
увольнения часто непросто, за одним исключением: вы и так знаете, что
с этим человеком нужно расстаться, но просто отказываетесь это
принять. Если это как раз тот случай, то сейчас самое время.

Если это сделает компанию прибыльной, или сэкономленные деньги
позволят улучшить ситуацию — вы избежали непосредственной опасности.

В противном случае снова появляются три варианта: уволить хороших
сотрудников, снизить некоторым или всем коллегам зарплату на какое-то
время либо увеличить доход.

Снижение зарплат нельзя назвать хорошим решением, и оно сработает
только когда дела не сильно плохи. Если выбранная бизнес-стратегия не
ведет к особенной прибыли, но компания сможет выжить при небольшом
снижении зарплат — можно попробовать. Иначе вы просто отложите
решение, и это будет очевидно для всех, кому вы урезали компенсацию.
[4]

Остается выбор из двух вариантов: увольнять хороших сотрудников или
зарабатывать больше денег. Пытаясь балансировать между ними, помните о
конечной цели построить успешную продуктовую компанию. Которая создает
одну вещь для использования миллионами людей.

Еще активнее подумать об увольнениях стоит, если причина сложившейся
ситуации в чрезмерно раздутом штате. Когда на старте нанимается 15
человек, и нет четкого понимания что вы создаете, то компания
обречена. Сначала нужно обязательно определиться с целью, а сделать
это с горсткой людей гораздо проще. К тому же эти 15 человек вполне
могут оказаться не теми, кто вам действительно нужен. И решением
вполне может быть сокращение штата с последующим переосмыслением целей
и путей развития. В конце концов, вы окажете медвежью услугу этим
людям, если разоритесь вместе с ними. Они все равно потеряют работу,
вместе с потраченным на обреченную компанию временем.

В случае, когда на вас работает лишь пара человек, лучше
сосредоточиться на способах увеличения прибыли. Предложение
зарабатывать больше может выглядеть легкомысленным — как будто
достаточно просто попросить денег у кого-то. Начинающий
предприниматель и так делает что может, чтобы продать побольше. Но я
не предлагаю продавать еще активнее, а просто рекомендую поискать
другие варианты дохода. Например, у вас целая команда пишет код, и
только один из них занимается продажами. Тогда попробуйте привлечь к
продажам всех. Чем вам поможет обилие кода, когда компания разорится?
Если код нужен для закрытия сделки — тогда вперед. Это и есть
привлечение программистов к продажам: необходимо в первую очередь
тратить силы на то, что принесет реальный доход в обозримом будущем.

IT man versus salesman


Еще способы заработать — увеличить ассортимент или попробовать выйти
на рынок аутсорсинга. Я говорю «попробовать», потому что путь от
создания продуктов до аутсорсинговых услуг долгий и непростой. Лучше
не заходить так далеко до тех пор, пока ваша продукция не начнет
пользоваться стабильным спросом. Но даже если ваш продукт еще не так
хорош, то программисты наверняка окажутся лучше тех, кого могли бы
нанять потенциальные заказчики. Или у команды может быть накоплен опыт
в каком-то малоизученном направлении. В общем, переход с вопроса
«хотите купить наш продукт?» на «что мы можем предложить такого, за
что вы готовы платить?» сделает заработок значительно проще.

На время придется стать безжалостным наемником, ведь вы стараетесь
спасти от гибели свою компанию. А значит, клиент должен платить много
и быстро. Но старайтесь избегать известных ловушек аутсорсинга.
Идеальной ситуация может быть только если вы создали продукт, четко
соответствующий потребностям и желаниям заказчика, что сделало бы его
продажу простой и предсказуемой. Но вы же не работаете с оплатой по
часам, а развиваете бизнес.

В лучшем случае, аутсорсинг превратится из необходимого для выживания
занятия в вещь-которая-не-масштабируется, определяющую вашу компанию.
Не ожидайте, что так и случится, но старайтесь не упускать маленькие
возможности с большими перспективами.

На частные заказы спрос достаточно велик, и вы наверняка найдете
подходящий для выживания путь на этой скользкой дорожке, если
обладаете хоть какой-то квалификацией. И я не просто так назвал путь
«скользким». Нескончаемые запросы клиентов всегда будут неумолимо
толкать вас вниз, подальше от первоначальной цели. Теперь, когда
выживание стало более вероятным, осталось выжить с минимальными
потерями и не сильно отвлечься от основной деятельности.

Хорошая новость в том, что большинство стартапов прошли через
аналогичный этап и стали успешными. Главное вовремя понять, что вы на
грани — и если дела стали плохи, то так оно и есть.

Примечания

[1] Есть категория компаний, которые не смогут много заработать за год
или два — результат их деятельности требует времени. Для них просто
замените «рост доходов» на «прогресс». Ваша компания к таким не
относится, если только подобное не обсуждалось с инвесторами сразу. Да
и не позавидуешь подобным организациям, ведь сложности с ликвидностью
ставят их в зависимость от инвесторов.

[2] Существует гипотетический вариант, что ваши инвесторы пойдут
навстречу и выделят еще денег. Скорее, вы просто решите, что они
пойдут навстречу — хотя они всего лишь упоминали такую возможность.
Единственный путь решения проблемы за оставшиеся 8 месяцев и менее —
постараться получить деньги прямо сейчас. Тогда вы либо получите
ожидаемое и сразу решите вопрос, либо перестанете пребывать в
заблуждениях касаемо намерений инвесторов.

[3] Само собой, стоит сейчас же избавиться от других крупных расходов,
помимо зарплат.

[4] Если, конечно, источник проблемы не в том, что вы платите слишком
большую зарплату себе. Урезание дохода предпринимателей обычно
способствует выходу компании в «плюс». Плохо только, что понимание
всего сказанного пришло к вам лишь после прочтения статьи.

\end{document}
