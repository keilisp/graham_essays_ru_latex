\documentclass[ebook,12pt,oneside,openany]{memoir}
\usepackage[utf8x]{inputenc} \usepackage[russian]{babel}
\usepackage[papersize={90mm,120mm}, margin=2mm]{geometry}
\sloppy
\usepackage{url} \title{Будущее финансирования стартапов} \author{Пол
  Грэм} \date{}
\begin{document}
\maketitle

Два года назад я писал о том, что назвал «огромной и неиспользуемой
возможностью в финансировании стартапов» — растущей пропасти между
венчурными фондами, чья бизнес-модель подразумевает инвестирование
больших денежных сумм единовременно, и большим классом стартапов,
которым нужно меньше денег, чем привыкли инвестировать фонды. Все чаще
новые проекты нуждаются лишь в паре сотен тысяч долларов, а не в паре
миллионов. [1]

Сейчас эта возможность используется намного активнее. Инвесторы
ворвались в нишу и осваивают её с двух направлений. Венчурные фонды
стали активнее инвестировать небольшие суммы, нежели год назад, и в то
же время в прошлом году наблюдалось резкое увеличение числа инвесторов
нового типа: «супер-ангелов», которые действуют как обычные
бизнес-ангелы (прим. Переводчика: частные венчурные инвесторы,
обеспечивающие финансовую поддержку компаний на ранних этапах
развития), но оперируют деньгами других людей, что сближает их с
венчурными фондами.

Хотя в нишу финансирования стартапов входят многие инвесторы, на рынке
все еще найдется место для новых. Распределение инвесторов должно
соответствовать распределению стартапов, которое изменяется по
экспоненте. Исходя из этого, среди инвесторов должно быть намного
больше тех, кто готов инвестировать десятки или сотни тысяч, нежели
миллионы. [2]

На самом деле, «ангелам», возможно, и выгодно появление в нише
большого количества конкурентов. В этом случае предприниматели станут
больше доверять «ангельским инвестициям», и, вероятно, предпочтут их в
будущем, даже если у них будет возможность попытаться привлечь серию
«А раундов» от венчурного фонда (прим. Переводчика: подробнее про
стадии венчурного инвестирования можно посмотреть тут). Сейчас одной
из причин, по которым стартапы предпочитают «А раунды», является
престиж. Однако, если «ангелы» станут более активными и известными
инвесторами на рынке, то однажды смогут конкурировать по авторитету с
венчурными фондами.

Естественно, престиж не является главной причиной, почему стартапы
предпочитают «А раунды». Получив серию «А раундов», стартап наверняка
привлечет больше внимания инвесторов, чем в случае инвестиции от
ангела. Поэтому, когда стартап выбирает между «А раундом» от хорошего
венчурного фонда и деньгами «ангела», я обычно советую выбрать первое.
[3]

Но я считаю, что, пока серии «А раундов» остаются на рынке, именно
венчурные фонды должны беспокоиться о «супер-ангелах», а не наоборот.
Несмотря на название, «супер-ангелы» в действительности являются
мини-венчурными фондами, и они явно нацеливаются на место уже
существующих фондов.

Думаю, история будет на их стороне. Ситуация похожа на те, при которых
стартапы и крупные компании выходят на новый рынок. Лишь только
онлайн-видео стало технически возможным, Youtube ворвался в
появившуюся нишу; в то же самое время крупные медиа-компании медленно
и неохотно приступали к освоению новых перспектив, движимые скорее
страхом, чем надеждой, и стремясь защитить свою сферу влияния, а не
создать что-то нужное пользователю. Это же справедливо для PayPal.
Паттерн повторяется вновь и вновь, обычно заканчиваясь победой
агрессоров. Для рынка инвестиций агрессорами являются «супер-ангелы».
Весь их бизнес строится на предоставлении ангельских инвестиций (точно
так же онлайн-видео было основным бизнесом для YouTube). А венчурные
фонды обычно вкладывают небольшие инвестиции лишь для создания
стартовой точки, с которой можно будет запустить поток сделок для
полноценных серий «А раундов». [4]

С другой стороны, финансирование стартапов — это очень необычный
бизнес. Почти вся прибыль концентрируется в нескольких наиболее
удачных проектах. И если «супер-ангел» не сумел найти и инвестировать
в них свои средства, то он окажется не у дел даже в том случае, если
вкладывал деньги во всех остальных участников.

Венчурные фонды Почему венчурные фонды не начнут использовать меньшие
серии «А раундов»? Камнем преткновения является участие фонда в
управлении каждым финансируемым им стартапом. Обычно при «А раунде»
инвестор, предоставивший финансирование, получает место в совете
директоров. Предположим, что средний стартап живет около 6 лет и
ответственный за сделку партнер может принимать участие в управлении
не более чем 12 стартапами одновременно. В такой ситуации
инвестиционный фонд может заключаться лишь две сделки в год для
каждого партнера.

Мне всегда казалось, что решение проблемы заключается в отказе фонда
от политики иметь место в совете директоров каждого стартапа. Не нужно
быть в совете директоров, чтобы помогать стартапу. Возможно, венчурные
фонды считают власть, получаемую с местом в совете директоров, некой
гарантией того, что их деньги не будут потрачены зря. Но проверяли ли
они эту теорию? Если только они не пытались снизить степень своего
участия в управлении стартапами и после не обнаружили снижение
прибыли, то можно сделать вывод, что они даже не пробовали изменить
ситуацию.

Я не утверждаю, что венчурные фонды не помогают стартапам. Хорошие
фонды помогают, и значительно. Я лишь пытаюсь сказать — чтобы оказать
существенную помощь, не обязательно иметь место в совете директоров.
[5]

Чем все это закончится? Некоторые венчурные фонды смогут
адаптироваться, увеличив количество небольших инвестиций. Не удивлюсь,
если, упорядочив процесс отбора кандидатов и уменьшив число мест в
советах директоров, фонды смогут заключать в 2-3 раза больше серий «А
раундов», не потеряв в качестве.

Другие венчурные фонды ограничатся лишь поверхностными изменениями.
Угроза не фатальна, а эти организации очень консервативны. Такие фонды
не будут жестоко изгнаны с рынка, но постепенно, сами того не понимая,
перейдут в другой бизнес. Эти фонды будут продолжать инвестировать и
даже называть это сериями «А раунда», но де факто это будут серии «Б
раунда». [6]

В этих раундах они уже не смогут рассчитывать на 25-40\% стартапа, как
сейчас. За исключением совсем плохих ситуаций у учредителей нет
необходимости продавать столь большую долю компании на поздних этапах
финансирования. Так как неадаптировавшиеся фонды будут предоставлять
финансирование на более позднем этапе, их прибыль с каждого успешного
стартапа соответствующим образом уменьшится. Впрочем, так же
уменьшится количество неудачных сделок. Из-за этого соотношение уровня
риска к прибыли может остаться таким же, либо даже улучшиться. Эти
фонды просто станут инвесторами другого, более консервативного типа.

Ангелы Сейчас в больших «ангельских инвестициях», которые все чаще
конкурируют с сериями «А раундов», инвесторы забирают не так много
капитала стартапа, как венчурные фонды. Фонды, пытающиеся
конкурировать с «ангелами» путем заключения большего количества
меньших сделок, наверняка обнаружат, что им придется соглашаться на
меньший объем капитала. И это является отличной новостью для
учредителей стартапов: они сохранят под своим контролем большую часть
капитала компании!

Условия предоставления «ангел-раундов» также станут менее жесткими. Не
просто менее жесткими, чем для серий «А раундов», но менее жесткими,
чем условия для «ангел-раундов», которые когда-либо были.

В будущем «Ангельские инвестиции» все реже будут ограничиваться
определенной суммой или иметь ведущего инвестора. Прежде привычным
порядком действий для любого стартапа был поиск одного ангела, который
бы выступил в качестве ведущего инвестора. С ведущим инвестором,
который предоставит некую (но не всю) часть денег, стартап
впоследствии будет обсуждать рыночную стоимость компании и размер
раунда. После этого стартап и ведущий инвестор кооперируются для
поиска остальных инвесторов.

Скорее, будущее «ангельских инвестиций» выглядит так: вместо
фиксированного размера раунда стартап будет искать деньги,
договариваясь с каждым инвестором по отдельности (т.н. «rolling
close»). Это будет продолжаться до тех пор, пока учредители не решат,
что денег достаточно. [7]

И хотя будет один, самый первый инвестор, и его помощь при поиске и
переговорах с инвесторами, конечно, будет приветствоваться, этот
инвестор не будет «ведущим» в старом смысле управления раундом. Теперь
стартап будет заниматься этим самостоятельно.

Останутся т.н. «ведущие инвесторы», которые возьмут на себя роль
помогать стартапу деловыми советами. Эти инвесторы так же могут делать
наибольшие инвестиции в стартап, но теперь они не всегда будут
единственной стороной для переговоров или теми, кто первым выписал
чек. Унифицированная документация покончит с необходимостью
договариваться о чем-то кроме рыночной оценки проекта. Да и этот
процесс она упростит.

Если несколько инвесторов будут отталкиваться от одной и той же
рыночной стоимости проекта, то это будет та сумма, которая убедит
первого из них выписать чек. Впрочем, нужно учитывать, отпугнет ли
такая стоимость других инвесторов. Не обязательно останавливаться на
однократной оценке стоимости стартапа. Стартапы все чаще привлекают
деньги с помощью конвертируемых облигаций, которые не дают долю в
компании, но определяют «предел оценки рыночной стоимости»: при
конвертировании долга в капитал (при последующей переоценке, либо
приватизации, смотря что случится раньше) такой инвестор получит долю
капитала со скидкой, определенной конвертируемой облигацией. Это очень
важное отличие, поскольку дает стартапу возможность сделать несколько
займов с разными ограничениями. На данный момент этот способ только
получает распространение, но я предполагаю, что он станет более
популярным. (прим. Переводчика: Подробнее о конвертируемых займах
можно прочесть тут)

Овцы Причина нынешних событий — для стартапов старые пути были
откровенно плохи. Ведущие инвесторы могли использовать и использовали
фиксированный размер раунда в качестве оправдания позиции, которую
ненавидят все учредители стартапов: «Я инвестирую, если инвестируют
другие». Большинство инвесторов не способны самостоятельно оценивать
стартапы — вместо этого они полагаются на мнение других инвесторов.
Если другие инвесторы участвуют, то будут и они, а если нет, то нет.
Основатели стартапов ненавидят такое отношение потому, что оно ведет к
патовой ситуации и задержке, а это последнее, что может позволить себе
стартап. Большинство инвесторов понимают, что такой образ действия
неэффективен, и очень немногие открыто признаются, что работают по
этой схеме. Наиболее изобретательные из инвесторов добиваются того же
эффекта, предлагая фиксированные раунды и предоставляя лишь часть
необходимой для финансирования суммы. Если стартап не сможет найти
недостающее, то они выходят из сделки. Как вообще можно преуспеть с
такой сделкой? Стартап будет недофинансирован!

В будущем инвесторы уже не смогут предлагать инвестиции со столь
неопределенными условиями, как участие других людей. Точнее, подобный
инвестор окажется последним в очереди. Стартап прибегнет к их деньгам
только в том случае, если необходимо дополнить уже почти
профинансированный раунд. А учитывая, что на «горячие» стартапы
инвесторов обычно больше, чем необходимо денег, то быть последним в
очереди означает, что они, скорее всего, упустят самые «горячие»
сделки. «Горячие» сделки и успешные стартапы — не одно и то же, но они
сильно коррелируют между собой. [8] Исходя из вышесказанного, не
инвестирующие самостоятельно инвесторы окажутся в проигрыше.

Наверняка инвесторы обнаружат, что, избавившись от этого костыля, они
станут справляться лучше. Погоня за привлекательными сделками не
заставляет инвестора выбирать тщательнее, но заставляет их лучше
относиться к своему выбору. Я видел рождение и угасание не одной
инвестиционной горячки и, насколько я могу судить, обычно они
случайны. [9] Если инвесторы не смогут больше рассчитывать на стадный
инстинкт, то будут вынуждены тщательнее анализировать каждый стартап
перед инвестированием. Вероятно, они будут удивлены, насколько хорошо
это работает.

Патовые ситуации — это не единственное неприятное последствие передачи
ведущему инвестору управления «ангельскими инвестициями». Нередко
инвесторы вступают в сговор с целью подтолкнуть рыночную стоимость
стартапа вниз. Или же раунд слишком долго собирает средства, поскольку
инвестор в поисках денег не обладает и десятой долей мотивации
учредителей стартапа.

Все чаще стартапы сами управляют своими «ангел-раундами». Пока лишь
некоторые зашли так далеко, но я думаю, что это достаточное основание,
чтобы объявить прежний способ мертвым, ведь эти некоторые — лучшие
стартапы. Они — те, кто может навязать инвестором свое видение раунда.
А если стартап, в который вы хотите инвестировать, работает по
определенному принципу, то какая разница, как работают остальные?

Traction Фактически, говорить, что «ангельские инвестиции» все больше
вытесняют серии «А раундов», будет заблуждением. В действительности,
раунды, находящиеся под контролем стартапов, начинают заменять раунды
под контролем инвесторов.

Это является примером очень важной общей тенденции, на которой с
самого начала был основан Y Combinator: учредители становятся все
более влиятельными в сравнении с инвесторами. Поэтому, если вы
захотите предсказать, как будет развиваться венчурное финансирование,
то просто спросите: «Каким его хотят видеть учредители стартапов?»
Всё, что не устраивало учредителей в процессе финансирования, будет
устранено одно за другим. [10]

Прибегнув к эвристике, я предскажу несколько вещей. Во-первых,
инвесторы не смогут более ждать, пока стартап «наберет обороты»,
прежде чем начнут вкладывать в него значительные суммы. Заранее сложно
предсказать, какой из стартапов «выстрелит». Поэтому сейчас стратегия
большинства инвесторов — ждать, пока стартап не преуспеет, а потом
быстро прийти с предложением. Стартапы тоже ненавидят подобный подход,
потому что он может создать патовую ситуацию, а также потому, что он
кажется жульничеством. Если вы многообещающий, но еще недостаточно
развившийся стартап, то большинство инвесторов будут вашими друзьями
на словах, но дел вы от них не дождетесь. Они будут громко заявлять,
что поддерживают вас, но деньги оставят при себе. Однако, когда
стартап начнет расти, они будут утверждать, что поддерживали вас все
это время и их просто ужасает мысль, что вы окажетесь настолько
неблагодарными, что оставите их за бортом вашего «раунда». В случае,
если учредители станут влиятельней, они смогут получить большую сумму
денег авансом от инвесторов.

(Худшее проявление такого подхода — финансирование траншами. В этом
случае инвестор осуществляет небольшую первоначальную инвестицию с
намерением вложить больше в будущем, но лишь в том случае, если
стартап будет успешен. На самом деле, этот подход дает инвестору
возможность свободного выхода из следующего раунда финансирования. И
инвестор воспользуется им, если найдет на рынке вариант получше.
Сделка, включающая в себя финансирование траншами, однозначно является
злоупотреблением. Такие сделки заключаются очень редко и в будущем их
будет все меньше.) [11]

Хоть инвесторы и не любят предсказывать, какой из стартапов будет
успешным, им придется делать это все чаще. Впрочем, не факт, что
текущая ситуация заставит их измениться. Возможно, они просто будут
заменены инвесторами с другим отношением к стартапам — инвесторами,
которые достаточно хорошо понимают стартапы, чтобы справиться с
проблемой предсказывания траектории проекта. Эти новые инвесторы
постепенно вытеснят тех дельцов, чьи умения лежат преимущественно в
плоскости получения денег от других инвесторов.

Скорость Еще одна ненавидимая учредителями сторона процесса поиска
финансирования — огромное количество времени, которое он занимает.
Поэтому с ростом влияния основателей стартапов раунды будут набираться
за меньшее время.

Поиск инвестиций все еще сильно отвлекает стартап. Если учредители
находятся в середине процесса поиска финансирования, то он приоритетен
в их сознании, что отрицательно сказывается на основном поле
деятельности стартапа. Если процесс поиска занимает два месяца (что
приемлемо по меркам сегодняшнего дня), это означает, что 2 месяца
кампания фактически толчет воду в ступе. А это самое худшее, чем
только может заниматься стартап.

Следовательно, если инвесторы хотят заключать лучшие сделки, то им
придется предоставлять средства на порядок быстрее. Инвесторам не
нужны недели на принятие решения. Мы выносим решение на основе 10
минут чтения заявки и 10 минут интервью с кандидатом и сожалеем только
о 10\% решений. Если мы способны принять решение за 20 минут, то
будущие инвесторы смогут сделать то же самое за несколько дней. [12]

В финансировании стартапов существует множество устоявшихся задержек:
многонедельный брачный танец с инвесторами; различия между начальными
сроками и реальной длительностью сделки; необходимость сложной работы
с документами для каждой серии «А раундов». Их принимают как данность
и инвесторы и учредители — просто потому, что так было всегда. Однако
исходная причина этих задержек в том, что они выгодны инвесторам.
Большее количество времени дает инвесторам больше информации о
развитии проекта, к тому же время обычно делает стратапы более
податливыми в переговорах, поскольку они нуждаются в деньгах.

Изначально эти обычаи не были предназначены для затягивания процесса
финансирования, но только из-за этого эффекта они используются до сих
пор. Медлительность процесса играет на руку инвесторов, стороны более
влиятельной в прошлом. Но как только учредители поймут
необязательность этих задержек, раунды уже не будут занимать месяцы
или даже недели. Это будет справедливо не только для «ангельских
инвестиций», но и для серий «А раундов». Будущее за простыми сделками
со стандартными условиями, которые будут заключаться в кратчайшее
время.

Еще одним небольшим злоупотреблением, которое будет исправлено в
процессе, является пул опционов. При традиционной серии «А раундов»
прежде, чем инвестировать, венчурный фонд заставляет компанию отложить
10-30\% от общего числа акций компании. Этот блок предназначен для
будущих наемных работников компании. Смысл в том, чтобы сохранить
разделение компании за уже вложившимися акционерами. Эта практика не
является нечестной (учредители понимают, что происходит), но излишне
усложняет сделку. В результате оценка рыночной стоимости идет двумя
разными числами. Нет никакой причины продолжать использование этого
способа. [13]

И последней возможностью, которой хотят добиться учредители, является
право продавать часть их собственных акций в последующих «раундах».
Это ничего особо не изменит, поскольку эта практика уже достаточно
распространена. Многие инвесторы не в восторге от этой идеи, но мир от
их недовольства вращаться не перестанет, поэтому подобное будет
происходить все чаще и чаще.

Сюрприз Выше я уже говорил о множестве изменений, которые будут
вынуждены принять инвесторы из-за возросшего влияния основателей. А
теперь к хорошим новостям: в результате инвесторы могут заработать еще
больше денег.

Пару дней назад репортер спросил меня — будет ли рост влияния
учредителей к лучшему или к худшему. Я был удивлен, поскольку никогда
не задавал себе этот вопрос. Сейчас это происходит: к лучшему или к
худшему. Впрочем, после секундного размышления над вопросом, ответ
показался очевидным. Учредители лучше понимают свои компании, чем
инвесторы, и обычно ситуация лучше, когда у людей с большими знаниями
больше власти.

Одна из частых ошибок неопытных пилотов в том, что они слишком
старательно управляют аппаратом. Они чересчур энергично вносят
корректировки. Из-за этого самолет колеблется около нужного курса,
вместо того, чтобы приближаться к нему асимптотически. Похоже, что до
сих пор инвесторы слишком усердно контролировали компании в своем
портфеле. Во многих стартапах больше всего стресса для учредителей
создавали не конкуренты, а инвесторы. Это точно было так для нас в
Viaweb. И это не новый феномен: даже для Джеймса Уатта инвесторы были
самой большой из проблем. Если утеря влияния предотвратит излишний
контроль стартапов со стороны инвесторов, это будет к лучшему не
только для учредителей, но и для самих инвесторов.

Для инвесторов все, возможно, закончится уменьшением их доли в каждом
отдельном стартапе, но для самих стартапов усиление контроля со
стороны учредителей наверняка пойдет проектам на пользу и спровоцирует
появление новых стартапов. Инвесторы конкурируют друг с другом, но они
не являются прямыми конкурентами друг другу. Нашим главным противником
выступают работодатели. И пока они сокрушительно побеждают — лишь
незначительное количество людей, имеющих возможность создать стартап,
реально создают его. В то время как почти все выбирают конкурирующий
продукт — постоянную работу. В чем причина? Давайте взглянем на
продукт, который мы предлагаем. Беспристрастный обзор получится
примерно таким:

Создание стартапа даст вам больше свободы и шанс заработать сильно
больше денег, чем на постоянной работе, но так же это очень тяжелый
труд, часто вызывающий огромный стресс.

Немалая толика стресса идет от взаимодействия с инвесторами. Если
реформирование процесса инвестирования избавит от этого стресса, то мы
сможем сделать наш продукт еще более привлекательным. Тот тип людей,
который создает отличные стартапы, не беспокоится о технических
проблемах. Они даже получают удовольствие от процесса их решения.
Однако они ненавидят проблемы, создаваемые инвесторами.

Инвесторы понятия не имеют, что, обходясь жестоко с одним стартапом,
они тем самым предотвращают появление 10 других. Тем не менее, это
так. Поэтому, когда инвесторы таки прекратят пытаться выжать «еще чуть
больше» из существующих проектов, они обнаружат себя в выигрыше, ведь
появится больше новых сделок.

Одна из наших аксиом в Y Combinator — не считать поток сделок игрой с
нулевой суммой. Наша главная задача в том, чтобы поспособствовать
появлению новых стартапов, а не урвать от них кусок побольше. По
нашему мнению, этот принцип крайне полезен, и если он распространится
вовне, то поможет и инвесторам, вкладывающим деньги на более поздних
стадиях.

Принцип «делайте то, что нужно людям» применим и к нам.

[1] В этом эссе я преимущественно рассуждаю о стартапах, связанных с
разработкой ПО. Эта точка зрения неприменима к стартапам, которые
нуждаются в большом стартовом капитале, как то стартапы в области
энергетики или биотехнологий. Даже дешевые разновидности стартапов
потребуют больших вложений, когда будут нанимать большое количество
сотрудников. Разница в том, как много они успеют сделать до этого
момента.

[2] По экспоненте определяется не только количество хороших стартапов,
но и число потенциально хороших стартапов, иными словами, хороших
сделок для инвесторов. Существует достаточно много потенциальных
победителей, из которых лишь немногие дойдут до конца.

[3] В процессе написания этой статьи я задал вопрос нескольким
предпринимателями, чьи стартапы привлекли «А раунды» от лучших
венчурных фондов. Вопрос звучал так: «Стоило ли оно того?». Они
единогласно согласились, что стоило. Впрочем, репутация инвестора
однозначно важнее, чем тип раунда. Я бы предпочел «ангел-раунд» от
хорошего ангела, «А раунду» от посредственного венчурного фонда.

[4] Учредители стартапов также беспокоятся, что получение
первоначальных инвестиций от венчурного фонда будет означать, что их
проект никуда не годится, если фонд откажется участвовать в
последующем раунде. Тренд первоначальных инвестиций от венчурных
фондов еще настолько нов, что сложно сказать, насколько оправданы эти
опасения.

Другая опасность, подмеченная Митчем Капором (Mitch Kapor), состоит в
том, что венчурные фонды предоставляют небольшие инвестиции лишь в
качестве стартовой точки, чтобы инициировать поток сделок для
дальнейшей серии «А раундов», то цели фондов и основателей стартапов
не совпадают. Учредители хотят, чтобы рыночная стоимость стартапа в
преддверии следующего раунда была высока, в то время как фонд хочет
оставить ее низкой. Повторюсь, пока сложно сказать, насколько
серьезной может стать эта проблема.

[5] Джош Коппельман (Josh Kopelman) подметил другой путь сокращения
трудозатрат фондов в управлении стартапами — сократить время участия в
совете директоров каждого из проектов.

[6] С этой точки зрения, Google, как и многие другие, был шаблоном для
будущего. Для венчурных фондов было бы здорово, если бы сходство
распространялось и на прибыли, но думаю это слишком оптимистичные
ожидания. Впрочем их прибыль, скорее всего, и правда будет выше.
Причину этого я объясню позже.

[7] Следование схеме «rolling close» не означает, что стартап все
время находится в поисках финансирования. Это бы слишком сильно
отвлекало. Смысл здесь в том, чтобы сделать так, чтобы поиск
финансирования занимал как можно меньше времени, а не больше.
Классический «раунд» финансирования означает, что вы не получите
деньги, пока все инвесторы не придут к соглашению. Часто это приводит
к ситуации, когда все инвесторы бездействуют и ждут активности от
других игроков. Следование стратегии «rolling close» обычно
предотвращает такой исход.

[8] Существуют две не исключающие друг друга причины появления
«горячих сделок»: качество компании и эффект домино среди инвесторов.
Первое является очевидно лучшим предвестником успеха.

[9] Частично факт случайности маскируется тем, что инвестирование по
сути является самовыполняющимся пророчеством.

[10] На данный момент сдвиг власти в сторону учредителей несколько
преувеличен, поскольку рынок находится в руках продавцов. При первом
же понижении тренда будет казаться, что я переоценил степень сдвига,
но уже на следующем после него повышении тренда учредители обнаружат
себя более влиятельными, чем когда бы то ни было.

[11] В целом станет менее распространенной ситуация, когда один
инвестор вкладывается в несколько последовательных «раундов».
Исключение — попытка сохранить свою процентную долю в компании. Когда
один и тот же инвестор инвестирует в последовательные «раунды», это
обычно означает, что стартап не получает рыночную цену. Возможно, это
не будет заботить учредителей. Возможно, они предпочтут работать с уже
известным инвестором. Но с повышением эффективности рынка инвестиций
будет все проще получить рыночную цену за стартап при условии, что
учредители этого захотят. В свою очередь это означает все большую
дифференциацию сообщества инвесторов.

[12] Обычно между двумя этими десятиминутками проходят 3 недели. Это
позволяет учредителям рассчитывать на дешевые билеты на самолет.
Впрочем, за исключением этого десятиминутки могут проходить и без
перерыва.

[13] Я не берусь утверждать, что пулы опционов полностью исчезнут. Они
являются удобным административным преимуществом. Что исчезнет, так это
инвесторы, требующие их использования.

\end{document}
