\documentclass[ebook,12pt,oneside,openany]{memoir}
\usepackage[utf8x]{inputenc} \usepackage[russian]{babel}
\usepackage[papersize={90mm,120mm}, margin=2mm]{geometry}
\sloppy
\usepackage{url} \title{Для чего не не создавать стартап?} \author{Пол
  Грэм} \date{}
\begin{document}
\maketitle

Y Combinator существует уже достаточно давно, чтобы можно было делать
какие-то выводы о результатах деятельности компании. В нашем первом
потоке, в 2005 г., было восемь компаний. Судя по всему, из этих восьми
как минимум четыре можно считать успешными. Три были приобретены:
Reddit, который слился с Infogami, а также еще одна компания, о
которой я сейчас не могу говорить. Еще одна компания этого потока -
Loopt, у которой настолько все прекрасно, что они могли бы продать
себя в ближайшие 10 минут, если бы захотели.

Итак, примерно половина основателей компании спустя два года стали
богаты, по крайней мере в их представлении. (Одна из вещей, которую
понимаешь, когда становишься обеспеченным, это то, что существует
много степеней богатства).

Я не уверен, что наш показатель успешности останется на уровне 50\%.
Может быть, первый поток был аномальностью. Однако мы наверняка сможем
держать показатель успешности на уровне превышающим часто
упоминающиется (и, вероятно, выдуманные) 10\%. Я думаю, будет примерно
25\%.

Даже основатели тех компаний, которые считаются неуспешными, не
слишком несчастны, судя по всему. Из тех восьми стартапов три,
вероятно, мертвы. В двух случаях основатели просто занялись другими
вещами в конце лета; я не думаю, что они были слишком расстроены.
Ближе всех к болезненному поражению были Kiko, которые были снесены
Google Calendar спустя год напряженной работы. Впрочем, они остались
довольны: они продали свой софт на eBay за четверть миллиона и,
расплатившись с бизнес-ангелами, оставили себе примерно по годовой
зарплате [1]. Они немедленно начали новый, значительно более
интересный стартап Justin.tv.

Итак, намного более интересная статистика: 0\% основателей пережили
нечто ужасное. Да, были свои взлеты и падения, как и у каждого
стартапа, но я не думаю, что хотя бы один из них променял этот опыт на
работу в офисном отсеке (cubicle). И эта статистика уже, вероятно, не
аномания. Каким бы ни был наш показатель успешности, процент людей,
которые бы жалели бы, что не пошли на обычную работу будет почти
нулевым.

Мне больше всего непонятно почему намного большее количество людей не
основывает компании? Если практически никто не жалеет об этом шаге и
значительный процент этих людей становятся богатыми, почему каждый не
мечтает о своей компании? Многие думают, что мы получаем тысячи заявок
каждый поток. На самом деле - всего несколько сотен. Почему остальные
не подают заявки? В то время как всем кажется, что стартапы появляются
как грибы после дождя на каждом углу, их количество несравнимо с
количеством людей, которые бы могли это делать. Подавляющее
большинство программистов направляются из университета в свой отсек в
офисе и остаются там.

Складывается ощущение, что люди действуют не в своих интересах. Что
происходит? Впрочем, я могу дать ответ. Благодаря тому, что Y
Combinator работает со стартапами на самой первой стадии, мы стали
мировыми экспертами в психологии людей, которые не уверены, стоит ли
основывать компанию.

Нет ничего плохого в том, чтобы быть не уверенным. Если вы хакер,
думающий об основании компании и вы сомневаетесь перед этим большим
шагом, то вы следуете по пути, пройденному многими. Ларри и Сергей не
были уверены, стоит ли основывать Google, а Джерри и Фило сомневались,
создавать ли Yahoo. На самом деле мне кажется, что наиболее успешные
стартапы получаются у сомневающихся хакеров, чем у переполненных
оптимизмом ребят от бизнеса.

У нас есть доказательства. Многие из тех, кого мы финансировали,
признались позднее, что они решили подать заявку в самый последний
момент, иногда за несколько часов до предельного срока.

Чтобы разобраться с неуверенностью, надо разложить ее на компоненты. У
большинства людей, которые не хотят что-то делать, есть восемь причин
для этого и они не знают, какая из этих причин главная. Некоторые из
них обоснованы, а некоторые - нет, но пока вы не знаете относительный
вклад каждой из причин, вы не можете понять, насколько обоснована ваша
неуверенность в целом.

Итак, я собираюсь рассказать об этих восьми причинах, которые
удерживают людей от создания своей компании, и объяснить какие из них
являются настоящими. Тогда те, кто только думает об основании
компании, смогут оценить свои чувства по этому списку.

Я признаюсь, что моей целью является упрочить вашу уверенность в себе.
Однако, есть два разных отличия от традиционных упражнений по
выработке уверенности в себе. Во-первых, у меня есть повод быть очень
честным. Обычно люди, которые развивают в вас уверенность, добиваются
своих целей в тот момент, когда вы покупаете книгу или платите за
семинар, где вам рассказывают какой вы потрясающий. Если же я буду
убеждать создать компанию тех людей, которым не стоит это делать, то
это только добавит мне проблем. Если я смогу уговорить слишком многих
подать документы в Y Combinator, то мне же придется читать все эти
заявки.

Кроме того, мой подход будет не такой, как обычно: я буду говорить не
в положительном ключе, а в отрицательном. Вместо того, чтобы говорить
"давай же, ты можешь", я буду объяснять почему компанию основывать не
стоит и расскажу почему большинство этих причин (но не все) могут быть
проигнорированы. Мы начнем с той проблемы, с которой рождается каждый.

1. Слишком молод

Очень многие думают, что они слишком молоды, чтобы основывать
компанию. Многие из них правы. Медиана возраста по планете - 27 лет,
так что примерно треть населения может небезосновательно заявить, что
они слишком молоды.

Слишком молоды - это сколько? Одной из целей Y Combinator было найти
ту самую нижнюю границу возраста создателей стартапов. Нам всегда
казалось, что инвесторы очень консервативны. Они хотят давать деньги
профессорам, а должны бы финансировать аспирантов, магистров и даже
тех, кто только получает бакалавриат.

Основное, что мы поняли, постоянно снижая планку возраста, - это не
где она находится, а насколько она размыта. Вероятно, нижний предел
где-то на уровне 16 лет. Впрочем, мы не работает с людьми до 18 лет,
потому что с ними нельзя заключить контракт. Впрочем, основатель нашей
самой успешной компании, Сэм Олтман (Sam Altman), был в возрасте 19
лет в момент финансирования.

Сэм Олтман, впрочем, не показатель. Когда ему было 19, казалось, что у
него было опыта на 40 лет. У многих девятнадцатилетных опыта лет на
двенадцать.

Есть причина, по которой людей с какого-то момента начинают называть
специальным слово "взрослый". Есть грань, которую надо перейти.
Формально - это 21 год, но разные люди пересекают ее в очень разные
моменты жизни. Вы достаточно взрослый, чтобы основать компанию, если
вы перешли этот рубеж, сколько бы вам лет ни было.

Как узнать? Есть два теста, которые используют взрослые. Я понял, что
эти тесты существуют после встречи с Сэмом Олтманом, на самом деле. Я
заметил, что у меня осталось ощущение, что я говорил с кем-то, кто
старше меня. После этого я задумался, а что именно дало мне это
ощущение? Что заставило его казаться взрослее?

Первый тест, который используют взрослые - это рефлекс "я же
маленький!". Когда вы ребенок и вас заставляют сделать что-то очень
трудное, вы можете заплакать и закричать: "Я же всего лишь маленький
ребенок!". Быстрее всего, взрослые от вас отстанут. У детей есть
волшебный выход из многих трудных ситуаций: сказать, что они всего
лишь дети. Взрослым не положено так себя вести. Когда они все же так
делают, им безжалостно достается за это.

Другой способ отличить взрослого человека - это посмотреть, как он
принимает вызов. Тот, кто еще не вырос, будет отвечать на вызов от
взрослого, признавая его доминирование. Если взрослый говорит "это
глупая идея", то ребенок или уползает, поджав хвост, или начинает
протестовать. Однако протест - это лишь обратная сторона подчинения.
Взрослый способ отреагировать на позицию "это глупая идея" -
посмотреть в глаза и спросить: "Правда? Интересно, почему?".

Разумеется, есть куча взрослых, которые реагируют на вызовы как дети,
а вот дети, которые принимают вызов как взрослые - это редкость. Если
вы встретили такого ребенка, то знайте, что это взрослый человек,
сколько бы ему лет ни было.

2. Слишком мало опыта

Я как-то написал, что основателям компании должно быть хотя бы по 23
года и у них должен быть опыт нескольких лет работы в чужой компании
перед тем, как создавать свою. Я так больше не думаю; причиной тому
примеры тех компаний, которые мы финансировали.

Я все еще думаю, что 23 - это лучше, чем 21. Однако самый лучший
способ получить опыт в 21 - это создать компанию. Парадоксально, но
факт: если у вас недостаточно опыта, чтобы создать компанию, то надо
создать компанию и получить его. Это намного более эффективно, чем
идти на обычную работу. На самом деле, работа в компании не дает вам
опыта для создания своей: вместо этого вы превращаетесь в ручную
зверюшку, которая убеждена, что работать можно только в офисе, а
задания нужно получать от менеджера.

Что меня убедило в этом, так это основатели Kiko. Они основали
компанию сразу после колледжа. Из-за отсутствия опыта они наделали
кучу ошибок. Впрочем, спустя год, когда мы профинансировали их второй
стартап, они стали исключительно блестящими. Они явно не были ручными
зверюшками. Они бы никогда не стали такими, если бы проработали год в
Microsoft, или даже в Google. Они бы все еще были робкими младшими
программистами.

Поэтому теперь я советую людям основывать компании сразу после
колледжа. Нет лучшего времени, чтобы брать на себя риски, чем
молодость. Конечно, есть риск, что ничего не получится. Но даже
неудача быстрее приблизит вас к конечной цели, чем работа.

Мне немного не по себе, когда я это говорю, потому что по сути я
советую людям набраться опыта, пережив неудачу за наши деньги, но так
все и есть на самом деле.

3. Слишком мало решимости

Вам необходимо быть очень решительным, чтобы основать успешную
компанию. Вероятно, это самый верный признак будущего успеха.

Вероятно, многие недостаточно решительны, чтобы создать компанию. Мне
сложно об этом говорить уверенно, потому что я настолько сильно
стремлюсь к успеху, что мне сложно представить, что происходит в
головах людей, которые не так решительны. Но я знаю, что такие люди
есть.

Думаю, что многие хакеры недооценивают собственную уверенность. Многие
становились значительно более решимыми, и это было видно, когда
основывали свой стартап. Я могу привести в пример многих основателей,
профинансированных нами, которые были когда-то счастливы продать
компанию за два миллиона, а сейчас серьезно думают о мировом
господстве.

Как вы можете определить, достаточно ли вы решительны, если даже сами
Ларри и Сергей не были уверены, создавать ли Google? У меня есть
предположение: критерием может быть уверенность, с которой вы
работаете над собственными проектами. Возможно, Ларри и Сергей не были
уверены, что нужно основывать компанию, они не были покорными
профессорскими ассистентами третьего плана, смиренно выполняющими
приказы сверху. Они работали над собственными проектами.

4. Слишком мало ума

Возможно, вам необходимо быть лишь умеренно умным, чтобы создать
успешную компанию. Впрочем, если вы беспокоитесь по этому поводу, вы,
вероятно, ошибаетесь. Если у вас хватает ума беспокоиться, что вам
может не хватать ума на основание стартапа, то быстрее всего вам на
это ума хватит.

В любом случае, для создания своей компании особого ума не надо.
Впрочем, не всегда. Надо очень хорошо знать математику, чтобы создать
Mathematica. Однако большинство компаний занимаются более
приземленными вещами, где основной критерий успеха - усилие, а не
мозги. Возможно, Кремниевая долина изменит ваше представление об этом,
потому что здесь существует культ ума. Люди у которых не хватает
сообразительности по крайней мере делают вид, что она у них есть.
Однако, если вы считаете, что для того, чтобы разбогатеть нужны мозги,
попробуйте провести пару дней в наиболее причудливых развлекательных
местах Нью-Йорка или Лос-Анжелеса.

Если вы считаете, что вам не хватит ума основать технологически
сложный стартап, попробуйте писать софт для предприятий. Такие
стартапы основаны не на технологиях, а на продажах, а продажи прежде
всего зависят от усилий.

5. Слишком мало знаний о бизнесе

Еще одна причина, которую не стоит принимать в расчет. Вам совершенно
ничего не надо знать о бизнесе для основания компании. Главной целью
должен быть продукт. Все, что вам нужно знать в этой фазе - это как
создать что-то, что нужно людям. Если у вас это получится, придется
думать, как сделать на этом деньги. Однако это просто и вы быстро
разберетесь.

Меня часто критикуют за то, что я советую основателям просто сделать
что-то полезное, а о прибыли думать потом. И все же эмпирические
наблюдения показывают, что почти 100\% компаний, которые создают
что-то популярное, зарабатывают на этом деньги. Люди, приобретающие
компании, говорят мне в частном порядке, что они покупают компании не
за прибыль, а за стратегическую стоимость. Другими словами, за то, что
они сделали что-то полезное. Эти люди знают, что это правило сработает
и для них: если пользователи вас любят, вы как-нибудь из этого сможете
сделать деньги, а если не любят, то самая потрясающая бизнес-модель
вас не спасет.

Так почему же так много людей меня критикуют? Я думаю, они просто не
могут свыкнуться с идеей, что кучка двадцатилетних ребят может
озолотиться, создав что-то классное, но не прибыльное. Они просто не
хотят, чтобы так было. Однако вероятность этого не зависит от их
желания.

Сначала меня расстраивало, что обо мне говорили как о безответственном
разодетом дудочнике, который вел многообещающих основателей в
пропасть. Теперь я понимаю, что подобные пересуды - это хороший знак.

Самые верные истины - это те, в которые не верит большинство людей.
Это как недооцененные акции. Если вы начинаете с них, то вся прибыль
будет вашей. Поэтому когда вы находите хорошую идею, которую критикует
большинство людей, надо не просто проигнорировать их, но и агрессивно
работать над этой идеей. В данном случае вы должны искать идеи,
которые станут популярными, но из которых неясно как будет делать
деньги.

Я ставлю раунд финансирования на то, что вы не сможете создать что-то
популярное, из чего мы не смогли бы извлечь прибыль.

6. Нет сооснователя

Отсутствие сооснователя - это большая проблема. Стартап - это слишком
тяжелое занятие для одного человека. Хотя наше мнение зачастую
расходится с мнением других инвесторов, в этом пункте мы все едины.
Все инвесторы, без исключения, с большим энтузиазмом профинансируют
вас с сооснователем, чем вас одного.

Мы финансировали двух одиночек, но оба раза мы ставили первоочередную
задачу найти сооснователя, и они находили. Однако мы просили их
сделать это до того, как они подавали заявку. Когда проект уже
профинансирован, несложно найти сооснователя, поэтому мы хотели
кого-то, кто был бы достаточно предан этому достаточно сложному делу.

Что делать, если сооснователя нет? Найти. Это важнее, чем что либо
другое. Если там, где вы живете, нет никого, кто бы с вами создал
компанию, переезжайте. Если никто не хочет работать над вашей текущей
идеей, меняйте идею.

Если вы еще учитесь, вы окружены потенциальными сооснователями. Спустя
несколько лет будет намного сложнее их найти. Мало того, что у вас
будет меньше выборка, у многих уже будет работа и, может быть, даже
семьи, которые надо будет содержать. Если у вас есть однокурсники, с
которыми вы обсуждали идеи создания компании, то поддерживайте связь с
ними после окончания универа: это поможет надежде не умереть.

Возможно вы найдете сооснователя на одном из форумов в интернете. Но я
бы слишком на это не рассчитывал. Чтобы понять, способен ли этот
человек основать с вами компанию, с ним надо поработать. [2]

Основной вывод, который вы должны сделать - это не как найти
сооснователя, а то, что основывать компанию надо пока вы молоды и
окружены такими людьми.

7. Нет идеи

В каком-то смысле, отсутствие идеи - это не проблема, потому что
большинство стартапов все равно меняют свою изначальную идею. В
среднем стартапе в Y Combinator 70\% изначальной идеи обновляется к
концу третьего месяца. Иногда идея меняется кардинально.

На самом деле мы настолько убеждены, что личности основателей намного
важнее идеи, что мы хотим попробовать нечто новенькое в следующем
цикле финансирования. Мы разрешим людям подавать заявку без какой либо
идеи вообще. Если хотите, напишите в заявке, что вы совершенно не
представляете, что хотите делать. Если вы действительно что-то из себя
представляете, то мы вас в любом случае профинансируем. Мы уверены,
что мы сядем и придумаем какой-нибудь многообещающий проект.

В действительности, так все и обстоит на самом деле уже сейчас. Мы не
слишком-то обращаем внимание на идею, а спрашиваем о ней больше из
вежливости. Самый важные вопросы в заявке, на которые мы действительно
обращаем внимание - это рассказ о том, какие классные вещи вы уже
сделали. Если вы уже сделали первую версию того, из чего вы хотите
сделать стартап, тем лучше, но нас интересует умение создавать. Роль
ведущего разработчика в популярном проекте с открытыми исходниками -
это почти идеальный необходимый опыт.

Так обстоят дела в Y Combinator. Но так ли все просто в других
случаях? Потому что в каком-то смысле это проблема. Если вы создаете
стартап без идеи, то что же вы будете делать?

Итак, короткий рецепт создания идей для стартапов. Найдите что-то в
своей жизни, чего вам не хватает и удовлетворите эту потребность,
какой бы специфичной она не была. Стив Возняк создал компьютер для
себя. Кто мог предположить, что такому большому количеству людей тоже
понадобится персональный компьютер? Узкая, но существующая
потребность, это лучше, чем потенциально распространенная, но
гипотетическая необходимость. Даже если вам не хватает девушки в
субботу вечером и вы знаете, как решить этот вопрос с помощью
программного обеспечения, вы уже на верном пути, потому что у многих
есть такая проблема.

8. Нет пространства для новых стартапов

Многие смотрят на возрастающее количество новых стартапов и думают:
"Так больше продолжаться не может". Они ошибочно подразумевают, что
есть какой-то лимит на количество стартапов, но его нет. Никто же не
задает лимит на количество людей, которые работают в компаниях с
тысячами служащих. Почему должен быть лимит на количество людей,
которые работают в компаниях, где всего пять сотрудников? [3]

Практически каждый, кто работает, удовлетворяет какую-то потребность.
Деление компании на мелкие части не приводит к исчезновению
потребностей. Существующие потребности лучше бы удовлетворялись сетью
из стартапов, чем несколькими гигантскими, иерархическими
корпорациями, но я не думаю, что это привело бы к уменьшению спроса,
потому что удовлетворение существующих потребностей провоцирует новые
потребности. Во всяком случае, это верно с отдельными людьми. И ничего
плохого в этом нет. Мы принимаем как само собой разумеющееся то, что
средневековые короли сочли бы за безумную роскошь. Нет абсолютного
стандарта на материальное благополучие. Забота о здоровье - это один
из его компонентов и это еще одна черная дыра. В обозримой перспективе
у людей будет желание увеличить свое материальное благополучие, а,
значит, предела объема работы для компаний нет, в частности для
стартапов.

Обычно заблуждение, что стартапов слишком много не высказывается
прямо. Об этом говорят косвенно, например "существует конечное число
стартапов, которые могут приобрести Yahoo, Google и Microsoft".
Возможно, только список поглотителей немного длиннее. И потом, что бы
вы не думали о других компаниях, в Google не дураки сидят. Компании
покупают стартапы потому что они создают что-то ценное. И почему
должен быть какой-то лимит на количество стартапов, которые создают
что-то ценное, если не считать лимита на количество денег, которые
люди хотят заработать? Возможно и есть некоторый предел количества
компаний, которые может поглотить один приобретатель, но если стартап
представляет из себя какую-то потенциальную ценность, которую его
владельцы готовы продать за немедленный платеж, приобретатели
как-нибудь справятся с этим. Рынки в этом отношении довольно умны.

9. Надо содержать семью

А вот это уже серьезно. Я бы не советовал никому, у кого есть семья,
ввязываться в стартап. Я не хочу сказать, что это плохая идея, просто
я не хочу брать на себя ответственность за такие советы. Я готов брать
ответственность за 22-летних парней, которые в случае поражения
научатся многому и не потеряют шанса пойти на работу в Microsoft, если
захотят. Но я не готов ставить в трудное положение матерей.

Если у вас есть семья и желание создать стартап, то вы можете заняться
консалтингом с целью однажды перейти к созданию продукта. Шансы
создать что-то революционное таким образом довольно низки. Google у
вас никогда не получится по такой схеме. С другой стороны, вы никогда
не останетесь без дохода.

Другой способ снизить риски - это присоединиться к существующему
стартапу вместо создания своего. Быть одним из первых сотрудников -
это почти как быть основателем, переживая взлеты и падения. Вы будете
примерно $ 1/n^2 $-ым основателем, где n - это ваш номер как
сотрудника.

Как и в вопросе про сооснователя, основной вывод, который надо сделать
заключается в том, что создавать стартапы надо пока вы молоды.

10. Финансовая независимость

Это мой личный повод не создавать стартап. Стартапы - это стресс.
Зачем он нужен, если есть деньги? На каждого "неисправимого основателя
компаний" двадцать нормальных предпринимателей, которые скажут:
"Создавать новую компанию? С ума сошли?".

Я был пару раз близок к созданию новой компании, но каждый раз
передумывал, потому что я не хочу вкалывать не разгибаясь четыре года
своей жизни. Я слишком хорошо знаю этот бизнес, чтобы понимать, что
без определенного усилия стартап не создать. Именно способность брать
на себя безумно тяжелую работу делает хорошего основателя таким
опасным.

Впрочем, с выходом на пенсию есть свои проблемы. Как и многие другие
люди, я люблю работать. Одна из странных вещей, которые вы узнаете,
когда становитесь богатым, это то, что многие из тех людей, с которым
вы хотели бы работать, не богаты. Они должны работать над чем-то,
чтобы платить по счетам. Поэтому если вы хотите, чтобы они были вашими
коллегами, вам тоже придется работать над чем-то, что позволяет
оплачивать счета, хотя вы можете позволить себе этого не делать. Я
думаю, что этот мотив движет многими из тех, кто создает одну компанию
за другой.

Именно поэтому мне так нравится работать в Y Combinator. Это повод
работать с интересными людьми над тем, что мне нравится.

11. Не готов брать обязательства

Лично для меня это был повод не основывать компанию, когда мне еще не
было 30. Как и многие люди в этом возрасте, я ценил свободу превыше
всего. Я не хотел заниматься ничем, что бы ограничивало меня более,
чем на несколько месяцев. Тем более мне не хотелось заниматься тем,
что обещало поглотить всю мою жизнь без остатка, как это происходит со
стартапами. И это нормально. Если вы хотите провести время
путешествуя, или играя в группе, или занимаясь чем-то еще, это
совершенно нормальный повод не основывать компанию.

Если вы создадите успешный стартап, это займет у вас три или четыре
года. (Если создадите неуспешный, то отделаетесь намного быстрее.)
Поэтому не стоит ввязываться в это, если вы не готовы к обязательствам
на такой срок. Правда, имейте в виду, что если вы найдете обычную
работу, то это может занять те же несколько лет, а свободного времени
у вас будет намного меньше, чем вы ожидали. Так что если вы готовы
повесить на себя бэджик со своей фамилией, чтобы пойти на семинар по
профориентации, вероятно вы готовы основать компанию.

12. Необходим порядок

Мне говорили, что есть люди, у которых в жизни должен быть порядок.
Мне кажется, это красивое описание того, что им нужен кто-то, кто бы
руководил ими. Я верю, что такие люди существуют. Есть много
эмпирических доказательств: армии, религиозные культы и т.д. Наверное,
эти люди даже составляют большинство.

Если вы из этих людей, то не создавайте стартап. Более того, даже не
идите туда на работу. В нормальном стартапе вам никто не будет
говорить, что надо делать. Конечно, кто-то будет занимать позицию
директора, но пока количество сотрудников не перевалит за дюжину,
никто не должен никем руководить. Это слишком неэффективно. Каждый
должен делать то, что он должен делать без указания свыше.

Если вы считаете, что это - прямой путь к хаосу, представьте
футбольную команду. Одинадцать человек работают в достаточно
замысловатых комбинациях и, в то же время, за редким исключением,
никто никем не руководит. Журналист однажды спросил у Дэвида Бэкхема,
есть ли в мадридском Реале какие-то языковые проблемы, весь в команде
были игроки из примерно восьми стран. Он сказал, что никогда никаких
проблемы не было. Все игроки были на таком высоком уровне, что им не
приходилось разговаривать. Они просто делали правильные вещи.

Как определить, что вы достаточно независимо мыслите, чтобы создать
компанию? Если внутри вас возникает желание ощетиниться при одной
мысли, что это не так, то, вероятно, вы готовы к основанию стартапа.

13. Страх неопределенности

Возможно, некоторые люди не создают стартапы, потому что они не любят
неопределенность. Если вы пойдете работать в Microsoft, то вы можете
довольно точно предсказать, как пройдут ближайшие несколько лет. Более
того, даже очень точно предсказать это. Если вы создадите стартап, то
может произойти все, что угодно.

Ну, если вас волнует неопределенность, то я могу решить эту проблему.
Если вы создадите стартап, то вы, быстрее всего, потерпите неудачу.
Серьезно, это не самый плохой способ думать об этом опыте в целом.
Надейтесь на лучшее, но готовьтесь к худшему. В худшем случае, будет
интересно, в лучшем случае - разбогатеете.

Никто не будет вас ругать, если ваш стартап пойдет ко дну, если вы,
конечно, вложили в него достаточно усилий. Может быть раньше и было
время, когда работодатели расценивали подобный опыт как недостаток, но
только не сейчас. Я спрашивал менеджеров крупных компаний и они мне
говорили, что с большим удовольствием взяли бы на работу кого-то, кто
создал неудачный стартап, чем кого-то, кто работал в это время в
крупной корпорации.

Инвесторы тоже не будут считать это вашим недостатком, если стартап,
конечно, не пошел ко дну из-за вашей лени или потрясающей глупости.
Мне говорили, что в других местах - в Европе например - подобная
неудача приравнивается к позору, но только не здесь. В Америке
компании, как и практически все остальное, - расходный материал.

14. Непонимание того, что вы избегаете

Одна из причин, почему основатели, поработавшие пару лет после универа
работают лучше, чем те, кто создает стартап сразу после выпуска - это
осознание того, чего избегают. Если стартап потерпит крах, то им
придется идти на обычную работу, а они знают, какой это кошмар.

Если у вас за спиной летние стажировки, то вы, наверное, думаете, что
знаете каково это - работать в компании. Вероятно, вы заблуждаетесь.
Летние стажировки в технологических компаниях - это не настоящая
работа. Если вы летом работали официантом, то это - работа.
Технологические компании не набирают студентов в качестве источника
дешевой рабочей силы. Они берут вас на стажировку в надежде, что вы
придете к ним на работу после выпуска. Если вы сделаете что-то
полезное во время стажировки, то это будет классно, но они на это не
слишком рассчитывают.

Это все изменится, как только вы закончите универ. Вам нужно будет
зарабатывать свои деньги. Большинство компаний занимаются скучными
вещами, так что вам тоже придется заниматься скучным делом. Несложным,
по сравнению с универом, но скучным. Сначала вам будет казаться
прикольным получать деньги за относительно простые вещи, в то время
как в универе вы вкалывали и еще платили за это деньги. Впрочем, это
пройдет через несколько месяцев. Со временем пройдет всякое желание
работать над тупыми заданиями, сколько бы за это денег не платили и
как бы легко они не доставались.

И это не самое худшее. Самое ужасное в обычной работе - это то, что вы
должны там быть в определенные часы. Даже в Google, судя по всему, это
так. А это значит, как вам скажет любой работающий, что будут моменты,
когда вы будете не в настроении работать, но надо будет сидеть перед
монитором и делать вид, что вы работаете. Для тех, кто любит работать,
как и большинство хакеров, это пытка.

В стартапе всего этого нет. В большинстве из них нет понятия офисных
часов. Работа и жизнь просто перемешиваются. Впрочем, то, что никто не
смотрит косо, если вы живете на работе - это хорошо. В стартапе вы
можете делать все, что вы хотите большую часть времени. Если вы
основатель, то большую часть времени вам будет хотеться работать. Но
вам никогда не надо будет притворяться, что вы работаете.

Если вы заснете в середине рабочего дня в большой компании, это будет
казаться непрофессионализмом. Если вы заснете после обеда в стартапе,
то ваши друзья просто решат, что вы устали.

15. Родители хотят, чтобы вы стали врачом

Значительное количество потенциальных основателей не создают стартапы,
потому что их отговаривают родители. Я не хочу сказать, что их не надо
слушать. В семьях есть свои традиции; кто я такой, чтобы с ними
спорить? Но я назову пару причин, почему безопасная карьера - это не
совсем то, что ваши родители хотят для вас на самом деле.

Во-первых, родители более консервативны в отношении детей, чем они
были бы в отношении себя. Это на самом деле вполне рациональное
поведение в данной ситуации. Родители в итоге больше разделяют
страдания детей, чем успех. Большинство родителей не имеют ничего
против, это их работа; однако это делает их излишне консервативными. А
заблуждение из-за излишнего консерватизма все же является
заблуждением. Практически везде награда пропорциональна риску. Поэтому
удерживая детей от риска, родители, сами того не осознавая, лишают
себя награды. Если бы они это понимали, они бы спокойнее относились к
риску.

Во-вторых, родители часто могут ошибаться потому, что они, как
генералы, всегда бьются как в последний раз. Если они хотят, чтобы вы
стали врачом, то быстрее всего они хотят не чтобы вы помогали больным,
а чтобы у вас была оплачиваемая и престижная работа. [4] Но уже не
такая престижная и высокооплачиваемая, как в те годы, когда было
сформировано их мнение. Когда я был ребенком в семидесятые, быть
доктором - это значило быть кем-то. Доктора как-то неизбежно
ассоциировались с теннисом и Мерседесом 450 SL. Все это давно уже не
актуально.

Родители, которые хотят, чтобы вы стали доктором, не всегда понимают,
как многое изменилось. Они сильно бы расстроились, если бы вы были
Стивом Джоббсом? Я думаю, что к мнению родителей надо относиться не
более как к пожеланиям новой фунциональности в программе. Даже если
они о чем-то явно просят, не спешите исполнять желание, а лучше
подумайте, что им на самом деле надо и как лучше удовлетворить эту
потребность.

16. Искать работу принято

Это последняя и, наверное, самая важная причина, почему люди
устраиваются на работу: так принято. Традиции - это безумно сильный
стимул, потому что они действуют на вас, в то время, как вы об этом не
думаете.

Практически для всех, кроме бандитов, очевидно, что если вам нужны
деньги, надо искать работу. На самом деле, этой традиции всего
какая-то сотня лет. До этого все традиционно занимались фермерством.
Опираться на традицию, которой всего сотня лет - не самая хорошая
идея. В масштабах истории, все меняется очень быстро.

Возможно, мы сейчас на пороге еще одного такого изменения. Я читал
немало работ по истории экономики и хорошо понимаю, как работают
стартапы, и мне кажется, что мы сейчас переживаем примерно такой же
переход, как и век назад, когда был сдвиг от фермерства к
производству.

Знаете что? Если бы вы жили в то время, когда это изменение началось
(примерно 11-ый век в Европе), то любой человек, который ехал в город
зарабатывать деньги казался сумасшедшим. И хотя крепостным было
запрещено покидать феодальное поместье, убежать было не так сложно -
по периметру не было вооруженной охраны. Что не давало многим
крестьянам уйти в город, так это кажущимися безумными риски. Оставить
свой клочок земли? Оставить людей, вместе с которыми ты вырос, чтобы
жить в гигантском городе, где будет три или четыре тысячи незнакомцев?
Как жить? Где брать еду, если ты ее не вырастил?

Когда-то это казалось странным, а сейчас мы зарабатываем на жизнь
своим умом. Думая о том, как страшно основывать компанию, вспомните,
как страшно было когда-то нашим предкам заниматься тем, что мы делаем
сейчас. Как это ни странно, лучше всего об этом знают те, кто пытаются
вас убедить жить по-старому. Как Ларри и Сергей могут приглашать вас
на работу, если они сами никогда не были наемными рабочими?

Сегодня мы оглядываемся на средневековых крестьян и удивляемся: как
они это терпели? Как печально, должно быть, было окучивать одни и те
же грядки всю жизнь без надежды на лучшее, оставаясь во власти
феодалов и священников, которые забирали все излишки и требовали
уважения к себе как к главным. Я не буду удивлен, если однажды люди
будут смотреть на нашу сегодняшнюю "обычную" работу таким же взглядом.
Как печально, должно быть, изо дня в день ехать на работу в свой
офисный отсек бездушного офисного комплекса и получать задания от
того, кого вы должны называть боссом и который, вызвав вас к себе в
офис, попросит присесть и вы присядете! Представьте, что вам нужно
разрешение на публикацию программного обеспечения для пользователей.
Представьте, что у вас портится настроение в воскресенье вечером,
потому что уикэнд закончился и надо будет вставать и идти на работу.
Как они это терпели?

Мысль о том, что мы стоим на грани такого же перехода, как и переход
от фермерства к производству, захватывает. Поэтому меня интересуют
стартапы. Стартапы интересны не потому что они позволяют обогатиться -
мне наплевать на спекуляции с акциями, например. Они интересны
примерно так же, как и паззлы. В мире стартапов многое происходит.
Возможно, они представляют редкий, исторический переход от одного
способа создания богатства к другому.

Именно это заставляет нас работать в Y Combinator. Мы хотим
зарабатывать, даже если бы это было бы единственной целью, мы бы вряд
ли остановились, но это не главная задача. В человеческой истории было
всего несколько великих экономических преобразований. Помочь одному из
них произойти было бы просто потрясающе.

Заметки. 1. Единственные кто проиграли - это мы. У бизнес-ангелов было
конвертируемое долговое обязательство, поэтому они первые получили
доступ к выручке от аукциона. Y Combinator получил лишь 38 центов на
вложенный доллар. 2. Возможно, лучший опыт подобного рода - это работа
над проектами с открытыми исходным кодом, но они не подразумевают
частых встреч. Возможно, начать такой проект - неплохая идея. 3.
Должны существовать большие компании, чтобы покупать стартапы, поэтому
количество больших компаний не может опуститься до ноля. 4. Мысленный
эксперимент. Если бы доктора занимались тем же самым, но были бы
изгоями, зарабатывающими гроши, родители все так же бы хотели бы этой
карьеры для детей?

\end{document}
