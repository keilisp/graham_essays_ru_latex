\documentclass[ebook,12pt,oneside,openany]{memoir}
\usepackage[utf8x]{inputenc} \usepackage[russian]{babel}
\usepackage[papersize={90mm,120mm}, margin=2mm]{geometry}
\sloppy
\usepackage{url} \title{Попсовые задачи} \author{Пол Грэм} \date{}
\begin{document}
\maketitle

Я наблюдаю один и тот же паттерн во многих разных областях: несмотря
на то, что много людей усердно работали в своей области, была
исследована лишь небольшая часть пространства возможностей, потому что
все они работали над одинаковыми вещами.

Даже самые умные, самые изобретательные люди на удивление
консервативны, когда принимают решение, над чем работать. Людей,
которые никогда не мечтали быть попсой, каким-то образом оказываются
втянуты в работу над попсовыми (модными) задачами.

Если вы хотите попробовать работать над непопсовыми задачами, одно из
лучших мест для поиска — это области, которые, по мнению людей, уже
полностью изучены: написание эссе, Lisp, венчурные инвестиции — вы
можете найти здесь общее, паттерн. Если вы сможете найти новый подход
в большом, но уже давно распаханном поле, ценность того, что вы
обнаружите, будет умножена на его огромную площадь поверхности.

Лучшая защита от вовлечения в работу над попсой может быть искренняя
любовь к тому, что вы делаете. Тогда вы продолжите работать над этим,
даже если вы совершите ту же ошибку, что и другие, но не придадите
этому значение.

\end{document}
