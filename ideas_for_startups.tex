\documentclass[ebook,12pt,oneside,openany]{memoir}
\usepackage[utf8x]{inputenc} \usepackage[russian]{babel}
\usepackage[papersize={90mm,120mm}, margin=2mm]{geometry}
\sloppy
\usepackage{url} \title{Идеи для стартапа} \author{Пол Грэм} \date{}
\begin{document}
\maketitle

Откуда вы берете хорошие идеи для стартапов? Из тех вопросов, которые
мне задают, этот, пожалуй, номер один.

Я отвечу так: а почему, собственно, люди думают, что придумать идею
для стартапа настолько сложно?

Может показаться, что это глупый вопрос. Как почему? Если люди не
могут придумать идею, то это довольно сложно, по крайней мере, для
них. Верно?

На самом деле, может быть, и нет. Обычно, люди говорят не то, что они
не могут придумать идею, а что у них нет идей. А это не совсем одно и
то же. То, что у них нет идей, может быть и следствием того, что они
никогда не пытались их придумывать.

Я считаю, что чаще всего причина именно в этом. Люди думают, что
придумать идею для стартапа очень сложно, и поэтому даже не пытаются.
Они считают идеи чем-то из разряда чудес, которые либо появляются в
вашей голове, либо нет.

У меня есть теория, почему люди так думают. Они переоценивают значение
идей. Они думают, что для того, чтобы создать стартап, нужно просто
реализовать какую-нибудь грандиозную начальную идею. И так как
успешный стартап обычно стоит несколько миллионов долларов, то, тем
самым, хорошая идея — это идея на миллион.

Если придумать идею для стартапа означает придумать идею на миллион,
то неудивительно, что это кажется сложным. Настолько сложным, что даже
и пробовать не хочется. Интуиция подсказывает, что нечто столь ценное
не будет так просто валяться на дороге.

На самом деле, идеи для стартапа не стоят миллион долларов. Вот
эксперимент, который можно провести, чтобы доказать это: просто
попытайтесь продать идею. Ничто не развивается так быстро, как рынок.
И тот факт, что рынка для идей не существует, говорит о том, что на
них нет спроса. А это означает, что идеи для стартапа ничего не стоят.

Вопросы

Большинство стартапов умирают на стадии начальной идеи. Основная
ценность начальной идеи состоит в том, что в процессе выяснения того,
что она не работает, вы придумываете настоящую идею.

Начальная идея — это всего лишь точка отправления, это не план, а
задача. Будет проще, если относиться к идее таким образом. Вместо того
чтобы говорить, что вы хотите сделать многопользовательское
интернет-приложение для электронных таблиц, говорите: а можно ли
сделать многопользовательское интернет-приложение для электронных
таблиц? Несколько грамматических трюков, и ужасающая идея превращается
в многообещающий вопрос, который можно исследовать.

И это различие вполне реально, потому что утверждение вызывает
возражения, а вопрос нет. Если вы будете говорить, что собираетесь
создать вышеуказанное приложение, то критики — а наиболее опасный из
них сидит в вашей же голове — начнут говорить, что вы собираетесь
конкурировать с Микрософт, что вы не сможете дать им интерфейс,
который они ожидают увидеть, что пользователи не захотят хранить
данные на вашем сервере и так далее.

Вопрос ставит более простую задачу: давайте попробуем сделать такие
электронные таблицы и посмотрим, насколько далеко мы сможем зайти.
Очевидно, что при таком подходе, у вас будет возможность сделать
что-нибудь полезное. Может быть, то, что получится в итоге, даже не
будет электронными таблицами. Может быть, это будет какой-нибудь новый
инструмент, похожий на электронные таблицы, для которого еще нет
названия. Такой инструмент сложно придумать сразу, к нему можно только
постепенно придти.

Если относиться к идее для стартапа как к вопросу, то цель ваших
поисков меняется. Если бы идея была планом, то было бы необходимо,
чтобы она была верной. Но если это задача или вопрос, то она может
быть и неверной до тех пор, пока ведет к другим идеям.

Одной из ценностей неверных идей является то, что они помогают решить
частную задачу. Когда кто-нибудь работает над большой задачей, я
всегда спрашиваю: есть ли какой-нибудь способ решить частный случай
проблемы и затем постепенно продвигаться к полному решению?

Это обычно срабатывает, если только вы не попались в ловушку
локального максимума, подобно Искусственному Интелекту в стиле 80-х
годов или языку программирования С.

Против ветра

Таким образом, мы заменили проблему придумывания идеи, которая
принесет миллион долларов, на проблему придумывания задачи, которая
может быть и неверной. А это уже гораздо проще, не так ли?

Чтобы придумывать такие задачи, вам нужно две вещи: быть знакомым с
перспективными новыми технологиями и иметь подходящих друзей. Новые
технологии — это ингредиенты, из которых готовятся стартапы, а общение
с друзьями — это кухня, в которой эти ингредиенты варятся.

В университетах есть и то и другое, именно поэтому многие стартапы
начинают в них свой путь. Университеты наполнены новыми технологиями,
потому что они занимаются исследованиями. И в них полно людей, которые
отлично подходят для того, чтобы разделять с ними свои идеи — другие
студенты, которые не только умны, но и достаточно гибки, чтобы легко
пережить ошибку.

Противоположная ситуация у тех, у кого есть хорошо оплачиваемая, но
скучная работа в большой компании. Большие компании с предубеждением
относятся к новым технологиям, а люди, которые в них работают не
подходят для создания стартапа.

В эссе, которое я написал для студентов высших учебных заведений, я
сказал, что хорошее «правило правой руки» — идти против ветра, то есть
заниматься вещами, которые максимально расширят ваши будущие
возможности. Тот же принцип подходит и для взрослых, хотя и нуждается
в небольшой модификации: идите против ветра столько, сколько сможете,
и воспользуйтесь накопленной потенциальной энергией, когда придет
время кормить ваших детей.

Я думаю, что люди не осознают этого, но одна из причин, из-за которой
работы, вроде написания софта на Java для банков, хорошо оплачиваются,
состоит в том, что вы идете по ветру. Рыночная цена подобной работы
высока потому, что она дает меньше возможностей в будущем. Работа,
которая позволяет вам заниматься новыми интересными вещами, обычно
оплачивается меньше, так как в дополнение к жалованию вы получаете
новые навыки.

На самом деле, для того чтобы попасть в подходящую для создания
стартапа среду, не надо идти именно в университет. Достаточно попасть
в ситуацию, в которой вам нужно будет много учиться.

Итак, очевидно, что нужно заниматься новыми технологиями, но зачем
нужны другие люди? Неужели нельзя придумывать идеи самостоятельно?
Опыт показывает, что нет. Даже Эйнштейн нуждался в других людях, чтобы
обмениваться с ними идеями. Идеи развиваются, когда обсуждаешь их с
подходящими людьми. Вам нужно это сопротивление, так же как резчику по
дереву нужно сопротивление древесины.

Это одна из причин, по которой у Y Combinator есть правило против
инвестиций в стартапы с одним основателем. Почти все успешные компании
имеют как минимум двух. А поскольку основатели стартапа работают под
большим психологическим давлением, критично, чтобы они были друзьями

Сейчас, когда я пишу это эссе, я начинаю осознавать, что это может
помочь объяснить, почему так мало женщин — основателей стартапов. Я
читал в интернете (значит это, скорее всего, правда), что только 1.7\%
стартапов, которые получили инвестиции, были основаны женщинами.
Процент женщин-хакеров мал, но не настолько же. Так почему же разница
так велика?

Когда понимаешь, что успешные стартапы обычно основываются несколькими
друзьями, то возможное объяснение само напрашивается на ум. Лучшие
друзья, как правило, одного пола, и если некоторая группа в
меньшинстве, то пары из этой группы будут меньшинством в квадрате[1].

Рисование на полях

То, чем занимаются эти люди в начинающем стартапе, сложнее, чем просто
совместное придумывание идей. Я подозреваю, что наиболее продуктивный
подход — это нечто вроде солянки из совместной и одиночной работы.
Вместе вы обсуждаете некую сложную проблему, вероятно, ничего не
придумываете. Затем, на следующее утро кому-нибудь из вас, когда он
моется в душе, приходит гениальная идея, как решить эту проблему. Он,
сломя голову, несется рассказать ее остальным, и вы вместе работаете
над деталями.

Что происходит в душе? Кажется, что идеи просто впрыгивают в вашу
голову. Но можно ли сказать что-нибудь кроме этого?

Принятие душа — это такая форма медитации. Вы в сознании, но вас
ничего не отвлекает. Именно в такой ситуации, когда ваш разум свободно
блуждает, он натыкается на новые идеи.

Что происходит, когда ваш разум бродит? Это похоже на бессознательное
рисование картинок. У большинства людей есть свои характерные черты
такого рисования. Эта привычка бессознательна, но не случайна. Я
заметил, что стал делать это по-другому, после того, как начал учиться
рисовать. Мои жесты стали такими, как будто я рисую что-нибудь из
жизни. Это элементы разных рисунков, но расположены они случайно[2].

Возможно, позволить разуму блуждать, означает позволить ему рисовать
идеи. Ваш мозг делает определенные жесты, которым он научился в
работе, и если не обращать на это внимания, то он будет делать это, в
каком-то смысле случайно. Грубо говоря, вы вызываете одни и те же
функции, но со случайными параметрами. Вот метафора: функции,
вызываемые с параметрами неверного типа.

Когда я писал это, в моем мозгу всплыло: интересно, возможно ли
использовать метафоры в языках программирования? Я не знаю: у меня нет
времени обдумать это. Но это хороший пример того, как работает мой
мозг. Я провел много времени, придумывая дизайн языков
программирования, поэтому приобрел привычку постоянно спрашивать себя:
“можно ли использовать Х в программировании?”.

Конечно же, ваш способ применять это не должен быть обязательно связан
с тем, над чем вы работаете. На самом деле, зачастую лучше, если он не
связан. Вы ищете не просто идеи, а хорошие и новые идеи, и у вас
гораздо больше шансов придумать такие, если вы комбинируете понятия из
различных областей. Например, мы часто спрашиваем: “что будет, если Х
сделать open-source?”. Что будет, если сделать open-source
операционную систему? Идея хорошая, но не очень новая. Тем не менее,
если спрашивать, можно ли нечто отдать на open-source, можно
наткнуться на что-нибудь интересное.

Являются ли какие-нибудь виды деятельности лучшими источниками таких
вот привычек мозга? Мне кажется, что чем деятельность сложнее, тем
лучшим источником она является, ведь вам нужно мощное оружие, чтобы
атаковать трудные задачи. Я обнаружил, что математика — отличный
источник метафор, достаточно хороший, чтобы изучать ее из-за этого.
Взаимосвязанные области также являются хорошими источниками, особенно,
если их связь неочевидна. Все знают, что компьютеры связаны с
электроникой, но именно потому, что всем это известно, перенос идей из
одной области в другую мало чего дает. Это то же самое, что
импортировать товары из Висконсина в Мичиган. С другой стороны (я это
ответственно заявляю) программирование и рисование тоже связаны, в том
смысле, что и те и другие занимаются творчеством, и этот источник —
практически непаханое поле.

Проблемы

Теоретически, можно сталкивать разные идеи друг с другом в случайном
порядке и смотреть что получится. Можно ли сделать peer-to-peer сайт
знакомств? Можно ли сделать автоматическое бронирование? Можно ли
ввести теоремы в повседневную жизнь? Когда вы соединяете идеи
случайным образом, то они получаются не просто бессмысленными, но даже
семантически выглядят глупо. Что это вообще означает: ввести теоремы в
повседневную жизнь? Вы меня поймали. Я придумал не идею, а только
название.

Наверное, можно придумать что-нибудь полезное и таким способом, но у
меня ни разу не получилось. Это то же самое что знать, что в куске
мрамора спрятана прекрасная скульптура, и все что вам нужно сделать,
это убрать лишний мрамор. Это ободряющая мысль, потому что напоминает
вам, что ответ существует, но на практике ею редко можно
воспользоваться, так как область поиска слишком велика.

Я понял, что для того, чтобы придумать идею, нужно работать над
какой-нибудь задачей. Вы не можете начать с пустого места. Нужно
начать с проблемы и затем позволить вашему сознанию бродить так
далеко, чтобы оно могло формировать новые идеи.

Кстати говоря, увидеть проблему часто сложнее, чем найти ее решение.
Большинство людей предпочитает не замечать проблемы. Причина очевидна:
проблемы раздражают. Ведь они — проблемы! Представьте себе, что люди,
жившие в 1700 году, получили бы возможность увидеть свою жизнь так,
как видим ее мы. Это было бы невыносимо для них. Отрицание проблем
настолько сильно, что даже если предлагать возможные решения, люди
предпочтут думать, что они не сработают.

Я наблюдал этот феномен, когда работал на спам-фильтрами. В 2002 году
большинство людей предпочитали игнорировать спам, а большинство из
тех, кто не игнорировал, думали, что фильтры, которые были тогда
доступны, это лучшее, что может быть.

Я терпеть не мог спама, и я чувствовал, что должен быть способ
распознавать его статистически. И впоследствии выяснилось, что это
все, что было нужно для решения проблемы. Алгоритм, который я
использовал, был до смешного простой. Любой, кто попытался бы по
настоящему решить эту проблему, придумал бы его. Просто никто даже и
не пытался[3].

Позвольте мне повторить рецепт: ищите неразрешимые проблемы, которые,
как вам кажется, можно решить. Это простой рецепт для огромного
количества идей для стартапов.

Богатство

Все, что я говорил до этого, подходит для идей вообще. Что такого
особенного в идеях для стартапов? Идеи для стартапов нужны компаниям,
а компании должны зарабатывать деньги. А для того, чтобы зарабатывать
деньги, нужно делать что-нибудь, в чем нуждаются другие люди.

Таким образом, идея для стартапа — это идея о чем-то, что будет нужно
людям. Может ли любая хорошая идея быть чем-то, что нужно людям? К
сожалению, нет. Я думаю, что придумывать новые теоремы — замечательная
идея, но на них не очень большой спрос. Зато, похоже, большой спрос на
журналы со слухами о жизни знаменитостей. Определение ценностей очень
расплывчато. Хорошие идеи не совсем то же самое, что ценные идеи, и
это различие зависит от индивидуальных вкусов.

И все же хорошие идеи очень близки к ценным идеям. Я думаю, что они
настолько близки, что можно работать так, как будто вашей целью
является придумывание хороших идей, а затем остановиться и спросить
себя: “Будут ли люди на самом деле платить за это?” Очень немного идей
останется после этого вопроса.

Одним из способов сделать что-то нужное является поиск вещей, которыми
пользуются люди, и которые неправильно работают. Первый пример,
который приходит в голову — сайты знакомств. У них миллионы
пользователей, соответственно они точно нужны людям. И все равно, они
до сих пор ужасно работают. Спросите любого, кто когда-нибудь
пользовался ими. Такое впечатление, что разработчики пользовались
подходом “от худшего к лучшему”, но остановились на первом этапе и
выбросили продукт на рынок.

И, конечно же, наиболее очевидный источник проблем в жизни
пользователей — это сама Windows. Но это особый случай: вы не можете
опрокинуть монополию, атакуя ее с фронта. Windows может и должна быть
свергнута, но это нельзя сделать, просто дав людям лучшую операционную
систему. Убить ее можно, переопределив проблему в более общую. Ведь
вопрос не в том, какая операционная система должна стоять на
персональных компьютерах. А в том, как люди должны использовать
программное обеспечение вообще. На этот вопрос существуют ответы,
которые вообще не связаны с персональными компьютерами.

Сейчас все думают, что Google собирается решить эту проблему, но это
очень тонкий вопрос, настолько тонкий, что даже такой гигант как
Google может не справиться с ним. Я думаю, что с вероятностью более
чем 50\% убийца Windows, а если более точно — ее преемник, вырастет из
какого-нибудь небольшого стартапа.

Другой классический способ сделать что-нибудь полезное состоит в том,
чтобы взять нечто, что является роскошью, и ввести в обиход. Если люди
платят за что-то кучу денег, оно им, скорее всего, нужно. И мало таких
вещей, которые вы не смогли бы сделать существенно дешевле, если бы
попытались.

В этом и состоял план Генри Форда. Он сделал автомобили, которые тогда
были роскошью, доступными обычным людям. Но сама идея гораздо старше.
Водяные мельницы сделали доступной механическую энергию, и
использовались еще в Римской империи.

Если вы сделаете продукт дешевле, то сможете продать его больше. Но
если вы сделаете что-нибудь существенно дешевле, то вы можете получить
качественные изменения, потому что люди начнут использовать это
по-другому. Например, когда компьютеры станут настолько дешевыми, что
будут в каждом доме, их можно будет использовать в качестве устройств
для общения.

Часто, для того, чтобы сделать что-нибудь очень дешевым, нужно
сформулировать задачу по-другому. У Модели Т не было всех тех функций,
которые были у других машин. Например, они выпускались только черного
цвета. Но они решали основную задачу, которая заботила людей —
перемещаться из одного места в другое.

Делать вещи дешевле — это, на самом деле, частный случай более общей
техники: делать вещи проще. В течение очень долгого периода эта
техника заключалась именно в упрощении вещей, но в наше время вещи
часто делаются настолько сложными, что она трансформировалась в
другую: делать вещи проще в использовании.

Это область, в которой можно много чего улучшить. Мы все хотим иметь
возможность сказать: эта штука просто работает. Много ли вещей, про
которые мы говорим это?

Простота требует усилий, даже гениальности. Средний программист обычно
делает ужасный пользовательский интерфейс. Пару недель назад, когда я
гостил у родителей, я пытался воспользоваться духовкой. Это была новая
модель, и вместо обычных ручек у нее были кнопки и небольшой
светодиодный дисплей. Я понажимал несколько кнопок, которые
предположительно должны были ее разогреть, и знаете, что она выдала?
“Err.” Даже не “Error.” “Err.” Вы не можете просто говорить “Err.”
пользователю духовки. Нужно разработать дизайн таким образом, чтобы
ошибок не было вообще. И ведь у идиота, который придумал дизайн для
этой духовки, был пример хорошего интерфейса — старый вариант. Нужно
повернуть одну ручку, чтобы выставить температуру, и другую, чтобы
выставить таймер. Что ему не понравилось? Она просто работала.

Похоже, что для среднего инженера, чем больше опций, тем более
запутанный интерфейс он сделает. Таким образом, если вы хотите создать
стартап, то вы можете взять практически любую технологию, созданную
большой компанией и начать упрощать ее использование.

План выхода

Успех для стартапа почти всегда означает его покупку. Вам нужна
какая-нибудь стратегия для выхода из бизнеса, потому что вы не сможете
нанять очень умных сотрудников, если не предложите им достаточно
хороших условий. Это означает, что вам нужно будет либо продать
стартап, либо стать публичной компанией, а число стартапов, которые
становятся публичными, очень невелико.

Если успех с большой вероятностью означает продажу стартапа, то стоит
ли делать это вашей сознательной целью? Некоторое время назад нужно
было отвечать — нет: вы должны были притворяться, что хотите создать
большую публичную компанию, и очень удивляться, когда кто-нибудь
предлагал вам продать стартап. Вы, правда, хотите купить нас?
Вообще-то мы не собирались продаваться, но если вы предложите
подходящую цену…

Я думаю, что сейчас ситуация изменилась. Если в 98\% случаев успех
означает продажу стартапа, то почему бы не делать это открыто? Если в
98\% случаев вы разрабатываете продукт специально для какой-нибудь
компании, то почему бы не сделать это вашей задачей? Одним из
преимуществ такого подхода является то, что это дает еще один источник
идей: посмотрите на большие компании, найдите что-нибудь, что они
должны были бы сделать, и сделайте это за них. Даже если они уже
делают это, у вас, возможно, получится быстрее.

Убедитесь только, что делаете что-то, что понадобится нескольким
потенциальным покупателям. Не надо исправлять Windows, так как
единственный заинтересованный покупатель –- это Микрософт, и поэтому
им некуда торопиться. Они могут просто подождать и затем скопировать
ваш продукт. Если хотите, что ваша цена была высокой, работайте в
области, где есть конкуренция.

Если количество стартапов, созданных специально для разработки
какого-то продукта для большой компании, будет расти, то это и будет
естественным противовесом существующим монополиям.

Способ Стива Возняка

Самый эффективный способ придумать идею для стартапа, как это не
невероятно звучит — случайно. Если посмотреть на то, как начинались
знаменитые стартапы, то оказывается, что большинство из них сначала
вообще не были стартапами. Lotus началась с программы, написанной
Митчем Капором (Mitch Kapor) для друга. Apple появилась потому, что
Стив Возняк хотел делать компьютеры, а его работодатель,
Hewlett-Packard, не дал ему заниматься этим на работе. Yahoo началась
с личной коллекции ссылок Дэвида Фило (David Filo).

Это не единственный способ придумать стартап. Можно сесть и специально
придумать идею компании — мы так и сделали. Но рынком доказано, что
модель “сделай что-нибудь, чем будешь пользоваться сам” может быть
более плодоносна. И уж совершенно точно, это наиболее интересный
способ придумать идею. А так как стартап должен создаваться
несколькими друзьями, то им нужно заниматься тем, чем обычно и
занимаются программисты: писать забавные программки для своих друзей.

Может показаться, что это нарушает какой-нибудь закон сохранения, но
вот мой вывод: лучший способ придумать идею на миллион — просто
заниматься тем, чем хакерам нравится заниматься и так.

Примечания

[1] Этот феномен может объяснить целое множество подобных различий, в
которых сейчас винят различные запрещенные учения. Не нужно считать
такие различия чьим-то злым умыслом, если они могут быть объяснены
математически.

[2] Классический абстрактный экспрессионизм — это как раз рисование
такого типа: художники учатся рисовать, используя такие же жесты, как
обычно, но не стараясь нарисовать что-то определенное. Это объясняет,
почему такие картины (немного) интереснее, чем просто случайные мазки.

[3] Бил Еразунис (Bill Yerazunis) решил эту проблему. Но пришел к
решению другим путем. Он придумал общий алгоритм классификации файлов,
который был настолько хорош, что работал также и для спама.

\end{document}
