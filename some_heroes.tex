\documentclass[ebook,12pt,oneside,openany]{memoir}
\usepackage[utf8x]{inputenc} \usepackage[russian]{babel}
\usepackage[papersize={90mm,120mm}, margin=2mm]{geometry}
\sloppy
\usepackage{url} \title{Мои кумиры} \author{Пол Грэм} \date{}
\begin{document}
\maketitle

У меня в запасе есть несколько тем, о которых можно писать и писать.
Одна из них это «кумиры». \newline

Конечно же, это не список самых почтенных людей в мире. Я думаю, такой
список вряд ли кто и сможет составить, даже имея при этом огромное
желание. \newline

Например, Эйнштейн, его нет в моем списке, но безусловно, он
заслуживает место среди самых уважаемых людей. Однажды я спросил у
своей знакомой, которая изучает физику, был ли Эйнштейн на самом деле
таким гением, и она ответила утвердительно. Так почему же тогда его
нет в списке? Все потому, что здесь находятся те люди, которые
повлияли на меня, а не те — которые могли бы повлиять, если бы я
осознал всю ценность их работ. \newline

Мне нужно было подумать о ком-то и понять, является ли этот человек
моим героем. Мысли были разнообразны. Например, Монтень, создатель
эссе, выбыл из моего списка. Почему? Тогда я спросил себя, что
требуется, чтобы назвать человека героем? Оказывается, нужно всего
лишь представить что бы этот человек сделал на моем месте в той или
иной ситуации. Согласитесь, это не восхищение отнюдь. \newline

После того, как я составил список, я увидел в нем общую нить. У
каждого в списке были две характерные черты: они чрезмерно заботились
о своих произведениях, но тем не менее были предельно честны. Под
честностью я не понимаю исполнения всего чего хочет зритель. Они все
принципиально являлись провокаторами по этой причине, хотя они
скрывают это в разной степени. \newline

\subsection{Джэк Лэмберт}

Я рос в Питсбурге, в 70-ые годы. Если вы не были там в то время, вам
трудно даже представить как город относился к «Стилерс». Все местные
новости были плохими, металлургическая промышленность умирала. Но
«Стилерс» оставалась лучшей командой американского футбола, и в
некотором роде это отражало характер нашего города. Они не совершали
чудес, а просто выполняли свою работу. \newline

Другие игроки были более знамениты: Тэрри Брэдшо, Франко Харрис, Лин
Свон. Но они были в нападении, и вы всегда обращаете больше внимания
на таких игроков. Мне кажется, как 12-летнему эксперту в американском
футболе, что самым лучшим из них был Джэк Лэмберт. Он был совершенно
безжалостным, поэтому он и был так хорош. Он не просто хотел хорошо
играть, он хотел отличной игры. Когда игрок из другой команды владел
мячом на его половине поля, он воспринимал это как личное оскорбление. \newline

Пригород Питсбурга в 1970-х был довольно скучным местом. В школе было
скучно. Все взрослые нудились на своих работах в больших компаниях.
Все, что мы видели в СМИ было одинаковым и производилось где-нибудь в
других местах. Исключением был Джэк Лэмберт. Я никогда не видел
похожих на него. \newline

\subsection{Кеннет Кларк}

Кеннет Кларк — несомненно, один из лучших писателей nonfiction.
Большинство тех, кто пишет об истории искусства абсолютно ничего не
знают об этом, и это доказывают масса мелочей. Но Кларк был
превосходен в своих работах настолько, насколько это можно себе
представить. \newline

Что же делает его таким особенным? Качество идеи. Сперва, стиль
выражения может показаться обыденным, но это обман. Чтение «Наготы»
сравнимо лишь с ездой на Феррари: как только вы устроились, вас
прижимает к сиденью от большой скорости. Пока вы привыкнете, вас будет
кидать по сторонам, когда машина будет поворачивать. Этот человек
настолько быстро производит идеи, что схватить их нет никакой
возможности. Дочитывать главу вы будете с широко открытыми глазами и
улыбкой на лице. \newline

Благодаря циклу документальных работ «Цивилизация» Кеннет был
популярен в свои дни. И если вы хотите ознакомится с историей
искусства — «Цивилизация» это то, что я рекомендую. Это произведение
намного лучше тех, которые студенты вынуждены покупать, изучая историю
искусства. \newline

\subsection{Ларри Михалко}

У каждого в детстве был свой наставник в тех или иных вопросах. Ларри
Михалко был моим наставником. Оглянувшись назад, я увидел некую черту
между третьим и четвертым классами. После того, как я познакомился с
мистером Михалко все стало иначе. \newline

Почему так? Во-первых, он был любопытен. Да, конечно, многие из моих
учителей были достаточно образованными, но не любопытными. Ларри не
вписывался в образ школьного учителя, и я подозреваю, что он
догадывался об этом. Возможно, для него это было сложно, но нам,
студентам, это доставляло удовольствие. Его уроки были путешествием в
иной мир. Именно поэтому мне нравилось ходить в школу каждый день. \newline

Другое, что отличало его остальных — любовь к нам. Дети никогда не
врут. Другие учителя были равнодушны к студентам, а мистер Михалко
стремился стать нашим другом. В один из последних дней 4-го класса он
поставил нам пластинку Джеймса Тейлора, где играла «У вас есть друг».
Просто позови меня и где бы я ни был, я прилечу. Он умер, когда ему
было 59 лет от рака легких. Единственный раз, когда я рыдал — это его
похороны. \newline

\subsection{Леонардо}

Недавно я понял то, чего не понимал в детстве: все лучшее, что нам
удается сделать — мы делаем для себя, а не для других. Вы видите
картины в музеи и полагаете, что написаны они были исключительно для
вас. Большинство из этих работ предназначены для того, чтобы показать
мир, а не удовлетворять людей. Эти открытия порой более приятны, чем
те вещи, созданы для удовлетворения. \newline

Леонардо был многогранен. Одна из его почтеннейших качеств: он делал
так много всего великолепного. Сегодня люди только и знают его как
великого художника и изобретателя летательного аппарата. Из этого
можно полагать, что Леонардо был мечтателем, который отбросил все
концепции ракетоносителей в сторону. На самом деле, он сделал большое
количество технических открытий. Так, можно сказать, что он был не
только великим художником, но и прекрасным инженером. \newline

Для меня все же главную роль играют его картины. В них он пытался
изучить мир, а не показывать прекрасное. И все же, картины Леонардо
стоят на ряду с картинами художником мирового уровня. Никто с тех пор
ни до него ни после него не был настолько хорош в момент, когда никто
на него не смотрел (No one else, before or since, was that good when
no one was looking). \newline

\subsection{Роберт Моррис}

Роберта Морриса всегда характеризовала его правота во всем. Кажется,
что для этого нужно быть всезнающим, но на самом деле это удивительно
легко. Не говори ничего если ты в этом не уверен. Если ты не
всезнающий, просто не говори слишком много. \newline

Точнее, трюк заключается в том, чтобы обращать внимание на то, что ты
хочешь сказать. Используя этот трюк, Роберт, на сколько я знаю, ошибся
только раз, когда был студентом. Когда вышел Мак, он сказал, что
маленькие настольные компьютеры никогда не будут пригодны для
настоящего хакерства. \newline

В этом случае это не называется трюком. Если бы он осознавал, что это
трюк, он бы обязательно оговорился в момент волнения. У Роберта это
качество в крови. Он так же невероятно честный. Он не просто всегда
прав, но он еще знает, что он прав. \newline

Вы наверное подумали, как хорошо было бы никогда не ошибаться, и все
делали это. Это слишком тяжело обращать столько же внимания на ошибки
в идеи, сколько на идею в целом. Но на практике никто этого не делает.
Я знаю насколько это тяжело. После встречи с Робертом я старался
использовать этот принцип в программном обеспечении, он, кажется,
использовал это в аппаратном обеспечении. \newline

\subsection{П. Г. Вудхаус}

Наконец-то люди осознали всю важность персоны писателя Вудхауса. Если
вы хотите, чтобы вас сегодня приняли как писателя — вам нужно быть
образованным. Если ваше творение обрело публичное признание и оно
забавно, то вы тем самым подставляете себя под подозрение. Вот что и
делает произведения Вудхауса такими захватывающими — он писал то, что
хотел и понимал, что за это к нему будут относится с презрением его
современники. \newline

Эвелин Вог признала его лучшим, но в те времена люди назвали это
черезчур рыцарским и в тоже время неправильным жестом. В то время
любой случайный автобиографический роман недавнего выпускника колледжа
мог рассчитывать на более уважительное отношение со стороны
литературного учреждения \newline

Вудхаус, возможно, начал с простых атомов, но то как он объединил их в
молекулы, было почти безупречным. Его ритм в частности. Это заставляет
меня застенчиво писать об этом. Я могу думать только о двух других
писателях, которые приблизились к нему по стилю: Эвелин Во и Нэнси
Митфорд. Эти трое использовали английский язык так словно он им
принадлежал. \newline

Но у Вудхауса ничего не было. Он этого не стеснялся. Эвелин Во и Нэнси
Митфорд волновало то, что о них думали другие люди: он хотел казаться
аристократическим; она боялась, что недостаточно умна. Но Вудхаусу
было наплевать, что о нем думали. Он написал именно то, что хотел. \newline

\subsection{Александр Колдер}

Колдер попал в этот список, потому что делает меня счастливым. Могут
ли его работы соперничать с работами Леонардо? Скорее всего нет. Как и
не может наверное соперничать ничего что относиться к 20 столетию. Но
все хорошее что есть в Модернизме, есть у Колдера, и творит он со
свойственной ему легкостью. \newline

Что хорошо в Модернизме, так это его новизна, свежесть. Искусство 19
века начало задыхаться. Картины популярные в то время были в основном
художественным эквивалентом особняков — большие, вычурные и фальшивые.
Модернизм означал, что придется начать все заново, создавать вещи с
такими же серьезными мотивами как это делают дети. Художники, которые
лучше всех этим воспользовались, были теми, кто сохранил в себе
детскую уверенность подобно Кли и Колдеру. \newline

Кли был импозантным поскольку мог работать во многих разных стилях. Но
из них двоих я люблю Колдера больше, потому что его работы кажутся
более радостными. В конечном счете смысл искусства в том, чтобы
привлечь зрителя. Трудно предсказать что именно ему понравится; часто
то, что кажется сперва интересным, через месяц вам уже наскучит.
Скульптуры Колдера никогда не надоедают. Они просто сидят там тихо,
излучая оптимизм подобно аккумулятору, который никогда не разрядится.
Насколько я могу судить по книгам и фотографиям, счастье в работах
Колдера это отражение его собственного счастья. \newline

\subsection{Джейн Остин}

Все восхищаются Джейн Остин. Добавьте мое имя к этому списку. Мне
кажется, что она лучший писатель всех времен. Мне интересно как
обстоят дела. Когда я читаю большинство романов, я обращаю столько же
внимания на выборы автора сколько и на саму историю.Но в ее романах я
не могу увидеть механизм в работе. Хотя мне интересно как она делает,
то что делает, я не могу этого понять, потому что она настолько хорошо
пишет, что ее истории не кажутся придуманными. Я чувствую себя так
словно я читаю описание того, что на самом деле произошло. Когда я был
моложе, я читал очень много романов. Большинство из них я уже не могу
читать, потому что в них недостаточно информации. Романы кажутся
настолько скудными по сравнению с с историей и биографией. Но читать
Остин все равно что читать научную литературу. Она настолько хорошо
пишет, что вы даже не замечаете ее. \newline

\subsection{Джон Маккарти}

Джон Маккарти изобрел Lisp, область (ну или по крайней мере термин)
искусственного интеллекта, и был одним из первых членов лучших отделов
информатики в Массачусетском технологическом институте и в Стэнфорде.
Никто не станет спорить с тем, что он один из великих, но для меня он
особенный из-за Lisp. \newline

Нам сейчас трудно понять какой концептуальный скачок произошел в то
время. Парадоксально, одна из причин, по которым так тяжело оценить
его достижение, заключается в том, что оно было таким успешным.
Практически каждый язык программирования придуманный за последние 20
лет включает в себя идеи из Lisp, и с каждым годом среднестатический
язык программирования становится все больше похожим на Lisp. \newline

В 1958 эти идеи были совсем не очевидными. В 1958 о программировании
думали в двух ключах. Некоторые люди думали о нем как о математике и
доказывали все, что касалось машины Тьюринга. Другие воспринимали язык
программирования как способ сделать что-либо и разрабатывали языки, на
которые слишком сильно влияла техника того времени. Только Маккарти
преодолел расхождение во взглядах. Он разработал язык, который был
математикой. Но разработал не совсем правильное слово, вернее сказать
обнаружил. \newline

\subsection{Спитфайр}

Когда я писал этот список, я поймал себя на мысли, что думаю о людях
вроде Дугласа Бадера и Реджинальде Джозефе Митчелле и Джеффри Квилле,
и я понял, что хотя все они сделали много вещей в своей жизни, был
один фактор помимо прочих, который связал их: Спитфайр. Это должен
быть список героев. Как в нем может быть машина? Потому что эта машина
была не просто машиной. Она была призмой героев. Необычайная
преданность поступала в нее, и чрезвычайное мужество из нее выходило. \newline

Принято называть Вторую Мировую Войну борьбой между добром и злом, но
между построением боев, так и было. Оригинальное возмездие Спитфайра,
ME 109, жесткий практичный самолет. Это была машина-убийца. Спитфайр
был воплощением оптимизма. И не только в этих красивых строчках: он
был вершиной того, что в принципе можно было изготовить. Но мы были
правы, когда решили, что выше этого. Только в воздухе у красоты есть
край. \newline

\subsection{Стив Джобс}

Люди жившие в то время, когда был убит Кеннеди обычно точно помнят,
где они были, когда услышали об этом. Я помню точно где я был, когда
подруга спросила меня слышал ли я, что у Стива Джобса рак. У меня
словно земля ушла из-под ног. Спустя пару секунд она сказала мне, что
это была редкая операбельная форма рака и что он будет в порядке. Но
те секунды, казалось, длились вечно. \newline

Я не был уверен, стоит ли включать Джобса в список. Большинство людей
в Apple, кажется, боятся его, а это плохой знак. Но он вызывает
восхищение. Нет слова, которое могло бы описать кто такой Стив Джобс.
Он не создавал продукцию Apple сам. Исторически самая близкая аналогия
к тому, что он делал — это меценатство в искусстве в период великого
Ренессанса. Как генерального директора компании, это делает его
уникальным. Большинство руководителей передают свои предпочтения
подчиненным. Парадокс разработки заключается в том, что в большей или
меньшей степени выбор определяется случайным образом. Но у Стива
Джобса был свой вкус — настолько хороший вкус, что он показал всему
миру, что вкус значит гораздо больше, чем они думают. \newline

\subsection{Исаак Ньютон}

У Ньютона странная роль в моем пантеоне героев: он тот, за кого я себя
корю. Он работал над значительными вещами, по крайней мере часть своей
жизни. Так легко отвлечься, когда ты работаешь над мелочами. Вопросы,
на которые ты отвечаешь, всем хорошо знакомы. Вы получаете мгновенный
награды — по сути, вы получаете больше наград в свое время, если вы
работаете над вопросами первостепенной важности. Но мне неприятно
осознавать, что это путь к заслуженной безвестности. Чтобы делать
действительно великие дела, нужно искать вопросы, которые люди даже
вопросами не считали. Вероятно в то время были и другие люди, которые
этим занимались, как и Ньютон, но Ньютон — это моя модель такого
образа мышления. Я просто начинаю понимать, как это, должно быть,
ощущалось для него. У вас только одна жизнь. Почему бы не сделать
что-то грандиозное? Фраза «сдвиг парадигмы» сейчас затаскано, но Кун
понимал что-то. И за этим скрывается большее, отделенная сейчас от нас
стена лени и глупости, которая вскоре покажется нам очень тонкой. Если
мы будем работать как Ньютон. \newline


\end{document}
