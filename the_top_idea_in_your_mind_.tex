\documentclass[ebook,12pt,oneside,openany]{memoir}
\usepackage[utf8x]{inputenc} \usepackage[russian]{babel}
\usepackage[papersize={90mm,120mm}, margin=2mm]{geometry}
\sloppy
\usepackage{url} \title{Главная идея в вашей голове} \author{Пол Грэм}
\date{}
\begin{document}
\maketitle

Недавно я понял, что недооценивал важность того, о чем люди думают в
душе по утрам. Я и раньше знал, что в это время в голову часто
приходят отличные идеи. Теперь я скажу больше: вряд ли вы сможете
сделать что-то действительно выдающееся, если не будете думать об этом
в душе.

Вероятно, каждый, кто работал над сложными проблемами, знаком с этим
явлением: вы прилагаете все усилия, чтобы разобраться, терпите
неудачу, начинаете заниматься чем-то другим – и вдруг видите решение.
Это мысли, которые приходят в голову, когда вы не пытаетесь думать
целенаправленно. Я всё более уверен, что для решения трудных задач
такой способ мышления не просто полезен, но необходим. Проблема в том,
что при этом вы можете только косвенно управлять своим мыслительным
процессом. [1]

Я думаю, у большинства людей в голове в любое время есть одна главная
идея. Это то, о чем человек начинает думать, если позволяет своим
мыслям течь свободно. И эта главная идея, как правило, получает все
выгоды того типа мышления, о котором я писал выше. А значит, если вы
позволили неподходящей идее стать главной, она превратится в стихийное
бедствие.

Я понял это после того, как мою голову дважды на долгое время
захватывала идея, которую я не хотел бы там видеть.

Я замечал, что стартапам удается сделать гораздо меньше, если они
начинают искать деньги, но понять, почему это происходит, мне удалось
только после того, как мы сами их нашли. Проблема не во времени,
которое тратится на встречи с инвесторами. Проблема в том, что как
только вы начинаете привлекать инвестиции, привлечение инвестиций
становится вашей главной идеей. И вы начинаете думать об этом в душе
по утрам. А значит, перестаете думать о других вещах.

Я ненавидел искать инвесторов, когда руководил Viaweb, но забыл,
почему я настолько ненавидел это делать. Когда мы искали деньги для Y
Combinator, я запомнил почему. Денежные вопросы с очень большой
вероятностью становятся вашей главной идеей. Просто потому что они
должны ей стать. Найти инвестора непросто. Это не та вещь, которая
случается сама собой. Инвестиций не будет, пока вы не позволите им
стать вещью, о который вы думаете в душе. И после этого вы почти
перестанете продвигаться во всех остальных делах, над которыми
работаете. [2]

(Я слышал аналогичные жалобы от своих друзей-профессоров. Сегодня
профессора, похоже, превратились в профессиональных фандрайзеров,
которые в дополнение к поиску денег немножко занимаются
исследованиями. Может быть, настало время исправить это.)

Это поразило меня так сильно, что в течение последующих десяти лет я
был в состоянии думать только о том, о чем хотел. Разница между этим
временем и тем, когда я не мог так делать, была велика. Но я не думаю,
что эта проблема актуальна только для меня, потому что почти каждый
стартап, который я видел, тормозит в своем развитии, когда начинает
поиск инвестиций или переговоры насчет поглощения.

Вы не можете непосредственно контролировать свободный ход ваших
мыслей. Если вы их контролируете, они не свободны. Но вы можете
управлять ими косвенно, контролируя, в какие ситуации вы позволяете
себе попадать. Это и стало уроком для меня: смотри внимательнее, чему
ты позволяешь становиться важным для тебя. Загоняйте себя в такие
ситуации, в которых самой актуальной проблемой будет та, над которой
вы хотите подумать.

Конечно, вы не сможете полностью контролировать это. Любая
чрезвычайная ситуация выбьет все остальные мысли из вашей головы. Но,
борясь с чрезвычайными ситуациями, вы имеете хорошую возможность
косвенно влиять на то, какие идеи становятся главными в вашем
сознании.

Я обнаружил, что существует два типа мыслей, которых следует избегать
более всего: это мысли, которые вытесняют интересные идеи, как
нильский окунь – других рыб из водоема. Первый тип я уже упоминал: это
мысли о деньгах. Получение денег по определению приковывает к себе всё
внимание. Другой тип – мысли об аргументации в спорах. Они тоже могут
увлечь, поскольку умело маскируются под действительно интересные идеи.
Но реального содержания в них нет! Поэтому избегайте споров, если
хотите иметь возможность заниматься реальными делами. [3]

Даже Ньютон попадался в эту ловушку. После публикации своей теории
цвета в 1672 он на годы погряз в бесплодной дискуссии, и в конечном
итоге, решил прекратить публиковаться:

Я понял, что сделался рабом Философии, но если я освобожу себя от
необходимости отвечать господину Лайнусу и позволю ему выступать
против меня, то вынужден буду навечно порвать с Философией, за
исключением той ее части, которой занимаюсь для собственного
удовлетворения. Поскольку я считаю, что человек должен либо решиться
не высказывать на публике никаких новых мыслей, либо поневоле
становиться на их защиту. [4]

Лайнус и его студенты в Льеже были среди его наиболее упорных
критиков. По мнению Вестфола, биографа Ньютона, он реагирует на
критику чересчур эмоционально:

к тому времени, как Ньютон написал эти строки, его «рабство»
заключалось в написании пяти писем в Льеж, общим объемом 14 страниц,
на протяжении года.


Но я хорошо понимаю Ньютона. Проблема была не в 14 страницах, а в том,
что этот глупый спор все время не выходил у него из головы, которая
так хотела подумать о других вещах.

Оказывается, у тактики «подставь другую щеку» есть свои преимущества.
Тот, кто наносит вам оскорбление, причиняет двойной вред: во-первых,
собственно оскорбляет, а во-вторых, отнимает ваше время, которое вы
тратите на то, чтобы думать об этом. Если вы научитесь игнорировать
оскорбления, то сможете избежать как минимум второй части. Я понял,
что могу в какой-то мере не думать о неприятных вещах, которые делают
мне люди, говоря себе: это не заслуживает места в моей голове. Я
всегда рад обнаружить, что забыл подробности споров – это означает,
что я не думал о них. Моя жена думает, что я более великодушен, чем
она, но на самом деле мои мотивы чисто эгоистические.

Я подозреваю, многие люди не уверены в том, какая именно главная идея
владеет сейчас их головой. Я сам часто ошибаюсь на этот счет. Зачастую
я принимаю за главную идею ту, которую я бы хотел видеть главной, а не
ту, которая является ей в действительности. На самом деле, главную
идею легко вычислить: просто примите душ. К какой теме всё время
возвращаются ваши мысли? Если это не то, о чем бы вы хотели думать,
возможно, вы захотите что-то изменить.

Примечания

[1] Несомненно, для этого типа мышления уже есть название, но я
предпочитаю называть его «естественное мышление».

[2] Это было особенно заметно в нашем случае, потому что мы довольно
легко получили средства от двух инвесторов, но и с тем и с другим
процесс растянулся на месяцы. Перемещение больших сумм денег никогда
не бывает чем-то, к чему люди относятся небрежно. Необходимость
уделять этому внимание повышается при увеличении суммы, функция это,
может быть, и не линейная, но, безусловно, монотонная.

[3] Вывод: не становитесь администратором, иначе ваша работа будет
состоять из решения денежных вопросов и споров.

[4] Письма к Ольденбургу, цитата по Westfall, Richard, Life of Isaac
Newton, стр. 107.

\end{document}
