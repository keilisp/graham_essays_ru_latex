\documentclass[ebook,12pt,oneside,openany]{memoir}
\usepackage[utf8x]{inputenc} \usepackage[russian]{babel}
\usepackage[papersize={90mm,120mm}, margin=2mm]{geometry}
\sloppy
\usepackage{url} \title{Могут ли венчурные капиталисты стать жертвами
  кризиса?} \author{Пол Грэм} \date{}
\begin{document}
\maketitle

Финансирование венчурных инвесторов, видимо, иссякнет во время
настоящей рецессии, как это и происходит обычно в плохие времена. Но в
этот раз результат может быть иным. В этот раз количество новых
стартапов может и не уменьшиться. И это может быть опасно для
венчурных капиталистов.

Когда финансирование венчурных фондов уменьшилось после
интернет-пузыря, количество стартапов тоже уменьшилось. Не так много
новых стартапов было основано в 2003. Но стартапы не привязаны к
венчурам так, как это было 10 лет назад. Пути венчуров и стартапов
могут расходиться. И если это произойдет, они могут не встретиться
даже тогда, когда экономика поправится.

Причина, по которой стартапы более не зависят от венчурного капитала,
теперь известна всем в бизнесе стартапов: основать стартап сегодня
гораздо дешевле. Существует четыре основные причины: закон Мура сделал
“железо” дешевым; open source сделал софт бесплатным; веб предоставил
бесплатный маркетинг и распространение; а более мощные языки
программирования означают, что команды разработчиков могут быть
меньше. Эти изменения свели стоимость запуска стартапа на уровень
шума. Во многих стартапах - возможно, в большинстве, финансируемых Y
Combinator - самые большие расходы это просто деньги на проживание
основателей. У нас есть стартапы, которые прибыльны при доходах в
\$3,000 в месяц.

\$3,000 - пустяковая цифра для доходов. Почему кто-то должен
переживать по поводу стартапа, зарабатывющего \$3,000 в месяц?
Потому-что, хотя это и незначительная цифра для прибыли, эти деньги
могут полностью изменить ситуацию с финансированием стартапов.

Человек, запускающий стартап, всегда просчитывает на подсознательном
уровне, насколько хватит денег в банке, после чего они могут стать
рентабельными, получить больше денег, или выйти из бизнеса. Как только
вы пересекаете точку безубыточности, какой бы низкой она ни была, ваша
беговая дорожка становится бесконечной. Это качественный скачок,
похожий на то, как ускоряется космический корабль в фильме, чтобы
достичь скорости света - и звезды превращаются в линии, а затем
исчезают. Как только вы становитесь прибыльными, вам не нужны деньги
инвесторов. Так как интернетовские стартапы стали дешевы, порог
безубыточности может быть весьма низким. Это означает, что многие
интернет-стартапы более не нуждаются в инвестициях масштаба венчурных
фондов. Для многих стартапов, венчурное финансирование, на языке самих
фондов, из необходимого (”must-have”) превратилось в желательное
(”nice-to-have”).

Перемена случилась в то время, когда никто ее не ждал, и ее эффект был
долгое время незаметен. Уже после пузыря интернета запуск стартапа
стал ничтожно дешевым, но немногие осознавали это, так как стартапы
вышли из моды. Когда стартапы снова вошли в моду в 2005, инвесторы
снова стали выписывать чеки. И хотя основатели не нуждались в деньгах
инвесторов, они желали их получить, если это было возможно - частично
из-за традициии и частично, потому-что стартапы, как те собаки - не
откажутся ухватить кусок, если есть возможность. Пока инвесторы
выписывали чеки, основатели не пытались найти пределы, насколько мало
денег им действительно нужно. Было несколько стартапов, которые
случайно достигли этих пределов из-за необычных обстоятельств -
наиболее известный из них 37signals, который достиг лимита, потому-что
они достигли территорий стартапов с другого направления: они начинали
как консультационная фирма, поэтому у них были доходы еще до создания
продукта.

Инвесторы и основатели - как две детали, которые должны быть скручены
воедино. В районе 2000 года скрепляющий болт был убран. Так как детали
подчинялись одним и тем же силам, они все так же были соединены
вместе, но на самом деле одна просто прилегала к другой. Легкое
касание - и они разлетятся. Сегодняшний кризис может стать этим
толчком.

Так как позиция Y Combinator находится на самом конце спектра, мы
следим за знаками отделения между основателями и инвесторами, и мы на
самом деле видим их. Например, обвал фоднового рынка сделал инвесторов
более осторожными, но не оказал никакого эффекта на количество людей,
желающих основать стартапы. Мы принимаем новые предложения каждые
полгода. Последний прием закончился 17 октября, как раз после проблем
на рынках, и несмотря на это, мы получили рекордное количество
предложений, на 40\% больше, чем в том же цикле год назад.

Может быть что-то изменится через год, если экономика будет еще хуже,
но пока снижения интереса среди потенциальных основателей мы не
замечаем. Это разительно отличается от ситуации 2001 года. Тогда было
всеобщее ощущение, что время стартапов закончилось, и что каждый
должен вернуться в школу. Этого не происходит сегодня, и часть причины
заключается в том, что даже на спаде экономики не так уж трудно
создать что-то, что приносит \$3,000 в месяц. Если инвесторы
перестанут выдавать чеки - кого это волнует?

Мы также видим признаки расхождения между основателями и инвесторами в
отношении выхода из стартапов, которые мы спонсировали. Я говорил с
одним из них недавно, они завалили раунд финансирования у инвесторов
из-за каких-то пустяков, вроде некорректно заполненной формы -
поверите вы или нет. Этот стартап идет к очевидному успеху - их трафик
и график доходов выглядят как самолет, взмывающий ввысь. Итак, я
спросил их - хотят ли они, чтобы я познакомил их с другими
инвесторами. К моему удивлению, они отказались - они уже потратили
четыре месяца, общаясь с инвесторами, и сейчас они стали намного
счастливее, что больше им не придется этого делать. У них был друг,
которого они хотели нанять на деньги инвесторов, но сейчас они
отложили этот вопрос. Как бы то ни было, они чувствуют, что имеют
достаточно денег на счете, чтобы достичь рентабельности. Чтобы быть
уверенными, они переместились в более дешевое помещение. И в текущей
экономике, я ручаюсь, у них очень хорошие карты.

Я обнаружил настрой “инвесторы больше не проблема” у нескольких
основателей из нашего инкубатора, с которыми разговаривал недавно. Как
минимум один стартап из последнего (летнего) цикла, может и не
привлекать деньги инвесторов. Ticketstumbler стал прибыльным на наши
инвестированные \$15,000 и они надеются, что больше не нуждаются в
деньгах. Это удивляет даже нас. Так как наш фонд основан на идее как
можно более дешевого запуска стартапа, мы никогда не ожидали, что
основатели могут вырастить успешные стартапы без привлечения
дополнительных капиталов.

Если основатели решают, что венчурные капиталы больше не являются
проблемой, это может быть плохо для венчурных фондов. Когда экономика
снова встанет на рельсы через несколько лет и они снова будут готовы
выписывать чеки, они могут обнаружить, что препдриниматели ушли.

Существует сообщество предпринимателей, так же как сообщество
венчурных капиталистов. Они знают друг друга, и техники очень быстро
распространяются среди них. Если один пробует язык программирования
или нового хостинг-провайдера, и получает хорошие результаты, полгода
спустя половина из них уже использует их. То же самое справедливо и
для финансирования. Текущее поколение основателей хочет получать
деньги от инвесторов, особенно от Sequoia, потому-что Ларри и Сергей
получили деньги от венчурных фондов, особенно от Sequoia. Представьте,
что будет с бизнесом этих фондов, если следующая горячая компания
вообще не будет пользоваться финансированием от венчурных
капиталистов.

Капиталисты считают, что они играют в игру с нулевой суммой (один из
игроков выигрывает все, что проиграл другой). На самом деле, это не
так. Если вы не смогли заключить сделку с фондом Benchmark, вы
проиграли эту сделку, но индустрия капитала все равно выиграет. Если
вы не заключили сделку с фондом Ничто, все капиталисты проиграли.

Этот кризис может отличаться от того, что пришел за крахом доткомов. В
этот раз преприниматели могут продолжать запускать стартапы. И если
они продолжат это делать, инвесторам придется продолжать выписывать
чеки, иначе они могут стать неуместными.

\end{document}
