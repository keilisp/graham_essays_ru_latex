\documentclass[ebook,12pt,oneside,openany]{memoir}
\usepackage[utf8x]{inputenc} \usepackage[russian]{babel}
\usepackage[papersize={90mm,120mm}, margin=2mm]{geometry}
\sloppy
\usepackage{url} \title{Будущее интернет-стартапов} \author{Пол Грэм}
\date{}
\begin{document}
\maketitle

Сейчас происходит нечто интересное. Стартапы подвергаются тем же
изменениям, которые происходят с любой дешевеющей технологией.

В технологиях мы наблюдаем этот процесс постоянно. Изначально есть
какое-то очень дорогое устройство, производимое в малых количествах.
Потом кто-то придумывает, как выпускать его дешевле, изделий
производится намного больше -- и в итоге находятся новые способы его
применения.

Похожий пример — это компьютеры. Когда я был ребенком, компьютеры были
огромными, дорогими машинами, и собирались в единичных экземплярах.
Теперь они в массовом производстве. И мы можем применять компьютеры
повсюду.

Это очень старая схема. Многие поворотные точки в истории экономики
были частными случаями такой модели развития. Так было со сталью в
1850-х, и с энергией в 1780-х. Со швейной промышленностью в XIII веке,
порождая изобилие, которое привело к Ренессансу. Даже сельское
хозяйство развивалось по такой схеме.

Эта модель возникла благодаря стартапам, и теперь может применяться к
самим стартапам. Сделать веб-стартап настолько дешево, что их будет
все больше. Если эта модель действительно верна, то это приведет к
значительным переменам.

1. Много стартапов

Итак, мой первый прогноз о будущем веб-стартапов самый прямой: их
будет много. Когда запуск стартапа был дорогим, вам требовалось на это
разрешение инвесторов. Теперь вас ограничивает только смелость.

И даже ее планка становится ниже, так как люди видят, что другие
пробуют свои силы, и у них получается. В последних стартапах, которые
мы финансировали, были основатели, которые говорили, что хотели
заняться этим раньше, но им не хватало уверенности, и вместо этого они
нанялись на работу. Только после того как они услышали об успехе
собственных друзей, они решились попробовать сами.

Начинать стартап тяжело, но и работать с 9 до 17 не легче, а в
некоторых смыслах даже еще тяжелее. При работе над стартапом вы
сталкиваетесь с кучей трудностей, но у вас нет того чувства, что жизнь
проходит мимо, как это бывает при работе на большую компанию. К тому
же вы можете заработать гораздо больше денег.

Пока стартапы будут приносить людям прибыль, их число будет расти, и
может вырасти до количеста, о котором сейчас трудно даже подумать.

Сейчас мы считаем, что работать в компании нормально, хотя исторически
это не имеет прочного основания. Всего два или три поколения назад
основная часть людей в странах, ныне называемых индустриальными, жила
земледелием. Предположение, что большое число людей поменяют способ,
которым они зарабатывают на хлеб, кажется неожиданным, но было бы еще
более неожиданно, если бы они его не меняли.

2. Стандартизация

После того как развитие технологий делает что-либо очень дешевым,
всегда следует процесс стандартизации. Когда вы производите что-то в
огромных количествах, вы стараетесь стандартизировать все что не
нуждается в изменениях.

У нас в Y Combinator до сих пор работает только четыре человека, и
поэтому мы стараемся стандартизировать все что можно. Мы могли бы
нанять работников, но не делаем этого до крайней необходимости, чтобы
выяснить, как лучше развивать инвестирование.

Мы всегда советуем стартапам быстро делать первоначальную минимальную
версию проекта, а дальнейшее развитие определить из нужд
пользователей. По сути — дать самому рынку направлять развитие
продукта. Мы делали так и сами. Мы считаем наши методы для работы с
большим количеством стартапов чем-то вроде программного обеспечения.
Иногда это и есть в прямом смысле ПО, как Hacker News и наша система
приёма заявок.

Одна из самых важных вещей, над стандартизацией которой мы работали, —
это условия инвестирования. До сих пор условия обговаривались
индивидуально. А это проблема для разработчиков, ведь получение денег
происходит дольше, и требует дополнительной работы юристов. И, вместо
того, чтобы проделывать каждый раз однообразную бумажную работу, мы
решили сделать типовой договор финансирования, который могли бы
использовать все стартапы, в которые мы инвестируем, во всех следующих
раундах.

Но некоторые инвесторы все ещё хотят составить собственные условия
договора. Для тех случаев, где сумма составляет \$1 млн или больше,
можно заключать индивидуальные сделки. Но я думаю, что сделки
ангельских раунды будут большей частью проходить по стандартным
условиям. Инвестор, который хочет впихнуть в договор кучу непонятных
условий, все равно вряд ли вам нужен.

3. Новое отношение к поглощению

Другая вещь которая, как мне кажется, начинает стандартизироваться —
поглощения (приобретение стартапа большой компанией). Когда количество
стартапов увеличится, большие компании начнут разрабатывать
стандартные процедуры, которые сделают приобретение стартапа лишь
немного более трудным, чем найм работников.

В этой области лидирует Google, как и во многих областях технологии.
Они покупают множество стартапов — больше, чем думает большинство
людей, потому что объявляют они только о некоторых. И, будучи Google,
они выясняют, как делать это с толком.

Одна сложность, которую они решили, — это как правильно относиться к
приобретению. Для большинства компаний этот процесс все еще несет
клеймо неполноценности. Компании делают это, поскольку им приходится,
но у них вегда есть чувство что так быть не должно — что их
собственные программисты должны были быть в силах сделать все, что им
нужно.

Пример Google должен изменить взгляд остальных на эту проблему. У них
работают буквально самые лучшие программисты которые только могут быть
в публичной компании. Если они не стесняются приобретать стартапы, то
у других должно быть еще меньше затруднений. Конечно, то что делает
Google, компания Microsoft должна делать десятикратно.

Одна из причин, почему в Google не имеют проблем с приобретениями, —
они адекватно оценивают качества и потенциал людей которым предлагают
сотрудничество. Ларри и Сергей основали Google после того как изучили
рынок поисковых движков, попытались продать свою идею и не смогли
найти покупателя. Они сами были ребятами которые ходили в большие
компании, так что они знают, кто может сидеть за столом переговоров
напротив них.

4. Рискованные стратегии допустимы

Риск всегда соответсвует вознаграждению. Чтобы получить хорошую
прибыль, надо делать что-то, что может казаться сумасшедшим, как
создание нового поискового движка в 1998 году, или отказ от миллиарда
долларов за предложение покупки.

И это всегда было проблемой в венчурном обеспечении. У разработчиков и
инвесторов разные взгляды на риск. Зная, что риск в среднем сопаставим
с наградой, инвесторы предпочитают рискованные стратегии. Но у
основателей недостаточно знаний о других стартапах, чтобы понять, что
бывает в среднем на самом деле, и они склонны быть более
консерватвными.

Этот конфликт исчезает, если стартапы делать легко, потому что
разработчики могут начать делать проекты ещё в юном возрасте, когда
разумно идти на больший риск, и могут сделать больше стартапов по ходу
жизни. Когда разработчики могут делать множество стартапов, они могут
относиться к этому так же, как инвесторы. А это значит, что будет
создаваться больше благ и ценностей, потому что стратегии могут быть
более рискованными.

5. Более молодые и умные основатели

Если стартапы становятся дешевым изделием, они могут быть у многих
людей, как компьютеры после того, как микропроцессоры сделали их
дешевыми. В частности, стартапы смогут делать даже более молодые
основатели, чем это было раньше.

Когда создание стартапа много стоило, нужно было убедить инвесторов,
чтобы его начать. И это требует совсем не таких навыков, которые нужны
при самой работе над стартапом. Если бы инвесторы были совершенными
судьями, — эти два труда совпадали бы. Но, к несчастью, большинство
инвесторов ужасные судьи. Я видел, какой огромный обьем работы нужно
проделать, чтобы получить деньги. В любой области так -- чем хуже
соображает покупатель, тем больше нужно приложить усилий чтобы ему
что-то продать.

А если стартапы становятся дешевыми, есть другой способ убедить
инвесторов. Вместо того чтобы идти к венчурным капиталистам с
бизнес-планом и пытаться получить их обеспечение, вы можете начать
дело с первыми десятью тысячами долларов инвестиций, полученных от нас
или вашего дяди, и уже с делом, а не планом, подойти к существующей
компании. Вместо того чтобы стараться выглядеть уверенно и
благополучно, — вы можете просто указать на количество посетителей по
рейтингу Alexa.

Подобный способ убеждения больше подходит для хакеров, которые отчасти
уходят в технологии потому, что чувствуют себя неуютно со всей этой
фальшью, которая нужна в других областях.

6. Центры развития стартапов будут продолжать существовать

Может показаться, что дешевизна стартапов приведет к концу их центров
развития, таких как Силиконовая Долина. Если для начала вам нужны
только заемные деньги, вы можете найти их где угодно.

Это отчасти правда. Вы можете начать где угодно. Но вам нужно не
только начать дело. Вам нужно сделать его успешным. А это более
вероятно в центрах концентрации стартапов.

Я много думал об этом, и мне кажется, что веб-стартапы будут
становиться еще дешевле, если что-то увеличит важность их центров
развития. Их ценность, как и в любом бизнесе, заключается в
индивидуальных встречах, старейшем методе. В обозримом будущем никакая
технология не заменит прогулок по университету и встречу с другом,
который подскажет как исправить баг, что надоедал вам все выходные, не
заменит посещение другого стартапа в конце улицы, и, наконец, общение
с одним из его инвесторов.

Этот вопрос — надо ли быть в стартап-центре — похож на вопрос о том,
нужно ли брать внешние инвестиции. Вопрос не в том, нужны ли они вам,
но в том, даст ли это вам преимущество. Ведь если вы не сделаете то,
что даст преимущество, а ваши соперники сделают, они могут получить
преимущество перед вами. И если кто-то говорит "нам не нужно быть в
Силиконовой Долине", -- именно это "не нужно" показывает, что они не
относятся к вопросу правильно.

И пока стартап-центры столь привлекательны, удешевление создания
стартапов означает, что команды, которые смогут вступать в игру, могут
становиться меньше. Сегодня стартап может быть парочкой 22-х летних
парней. Такая компания намного мобильнее, нежели та, в которой 10
человек, половина из которых имеют детей.

Мы знаем это, потому что мы в Y Combinator заставляем людей переезжать
ближе к нам, и это не является большой проблемой. Преимущество
совместной работы в одном офисе перекрывает неудобства переезда.
Спросите любого, кто уже сделал это.

Мобильность стартапов на ранних стадиях означает что начальное
финансирование -- это бизнес в масштабе страны. Одно из самых частых
писем которые мы получаем, это просьба помочь людям создать у них
местный клон Y Combinator. Но так дело не пойдет. Начальное
инвестирование -- это не местный бизнес, так же как большие
исследовательские институты -- не местный бизнес.

Является ли начальное финансирование не только национальным, но и
интернациональным? Это интересный вопрос. Есть признаки того, что это
может быть и так. Мы имеем нескончаемый поток разработчиков из других
стран, и они хорошо себя показывают, потому что эти люди настолько
настроены на успех, что готовы переехать в другую страну, чтобы его
добиться.

Чем мобильнее будут становиться стартапы, тем сложнее будет создавать
новые силиконовые долины. Если стартапы мобильны, то лучшие местные
таланты отправятся в настоящую Силиконовую Долину, а на местных
"долинах" останутся только те, у которых не хватило энергии переехать.

Эта идея не является националистической, к слову сказать. Соперничают
города, а не страны. Атланта находится в таком же невыгодном
положении, что и Мюнхен.

7. Необходимость лучшей оценки

Если количество стартапов растет молниеносно, то и люди, которые
оценивают их, должны будут научиться делать это лучше. Я подразумеваю
инвесторов и совладельцев. Сейчас мы получаем около 1000 заявок в год.
Что нам делать, если количество заявок достигнет 10,000?

На самом деле это волнующий вопрос. Но мы должны будем его решить. Нам
придется. Возможно это потребует написание некоторого программного
обеспечения, и, к счастью, мы сможем это сделать.

Приобретатели тоже должны будут тщательнее выбирать участников. Обычно
у них это получается лучше, чем у инвесторов, потому что они отбирают
позднее, когда уже есть что оценивать. Но даже самому умелому
приобретателю нужно индивидуально определять компании, которые надо
покупать, а завершение покупки часто подразумевает много лишнего
трения.

Я думаю, что приобретатели могут в итоге сделать специальный отдел
поглощений, который будет сразу и определять лучшие покупки, и
совершать их. В настоящее время эти функции отделены. Новые
многообещающие стартапы часто находятся разработчиками. Если кто-то с
достаточными полномочиями хочет их купить, то переговоры о сделке
ведутся уже другим людьми. Было бы лучше, если бы они были совмещены в
одном отделе, который руководился бы человеком с техническим прошлым и
пониманием того, что именно он покупает. Может быть, в будущем станет
две должности в больших компаниях: директор по разработкам внутрии
компании, и директор по привлечению технологий снаружи.

В настоящий момент никто в больших компаниях не несет ответственности
за то что они тратят \$200 млн на то, что могли купить ранее за \$20
млн. И кто-то должен начать за это отвечать.

8. Колледж изменится

Если лучшие хакеры станут заниматься бизнесом сразу после окончания
колледжа, а не пойдут на работу по специальности, то это изменит то,
что происходит в колледже. Большая часть изменений будет к лучшему. Я
считаю, что в колледжах дела обстоят плохо из-за ожидания, что вас
после него будут оценивать потенциальные работодатели.

Одно из изменений будет касаться выражения "после колледжа". Вместо
того чтобы означать "получить учёную степень" оно будет означать
"бросить". Если вы начинаете собственную компанию, зачем вам нужна
степень? Мы не поощряем делать стартапы еще во время обучения, но
самые лучшие основатели, определенно, способны на это. Некоторые из
наиболее успешных компаний, в которые мы инвестировали, были начаты
"недоучками".

Я рос в то время, когда дипломы считались очень важными, поэтому мне
не очень легко это говорить, но в дипломе нет ничего волшебного. И
ничто не преображается после того, как ты сдаешь последний экзамен.
Единственное, для чего служит диплом — для административных нужд
организаций. Это может отчасти повлиять на вашу жизнь: без оконченного
диплома сложно поступить в аспирантуру, или получать визу работника в
США, но такие проверки станут все менее значительны.

Когда станет менее важно, получают ли студенты диплом, станет и менее
важно то, где они учились. В стартапе вас оценивают пользователи, и их
не заботит, из какого вы колледжа. Так что элитные университеты будут
играть меньшую роль в мире стартапов. Это настоящий позор для США —
то, как легко дети богачей могут поступить в колледжи. Чтобы решить
проблему окончательно, нужно не изменить систему образования, а обойти
ее. В мире технологий мы всегда делали именно такие решения: вы не
пытаетесь уничтожить существующие компании — вы решаете задачу
по-своему и они сами становятся ненужными.

Основная ценность обучения в университетах состоит не в громком имени
заведения, и даже не в занятиях, а в людях, которые там встречаюся.
Если создание стартапов сразу после окончания колледжа станет
нормальным делом, то студенты могут начать этим очень активно
заниматься. Вместо того, чтобы проходить стажировку в компаниях, где
они хотели бы работать, студенты могут сосредоточиться на совместной
работе как будущие совладельцы фирм.

Их деятельность на занятиях тоже изменится. Вместо того, чтобы
пытаться получить наивысший балл и впечатлить этим своих будущих
работодателей, студенты будут учиться не ради оценок, а ради знаний.
Здесь мы говорим о достаточно значительных изменениях.

9. Множество конкурентов

Если сделать стартап становится легко, то это легко и для конкурентов.
Хотя это не снижает преимущество от низкой стоимости. Это не игра с
нулевой суммой. Нет заранее установленного числа стартапов, которые
могут добиться успеха, независимо от того сколько их будет основано.

На самом деле нет никакого ограничения на количество успешных
стартапов. Стартапы становятся успешными, создавая ценности, делая то
что совпадает с желанием людей. И, похоже, желания людей безграничны,
по крайней мере короткосрочные.

Рост числа стартапов значит то, что вы не сможете подолгу сидеть с
одной идеей. Ваша идея есть у других людей, и все более вероятно, что
они будут с ней что-то делать.

10. Быстрое развитие

Есть большая вероятность этого, по крайней мере в отношении
технологии. Технология будет развиваться быстрее, если у людей будет
возможность применять свои идеи, а не держать их.

Некоторые изменения происходят с компаниями сразу, как в модели
эволюции периодически нарушаемого равновесия. Иногда идеи настолько
пугающи, что большим компаниям даже сложно подумать о их воплощении.
Вспомните, как сложно было Microsoft открывать для себя
веб-приложения. Они были как персонаж фильма, когда весь зал понимает
что с героем случится что-то плохое, но он сам этого не видит. Большие
изменения в компаниях явно будут присходить быстрее, если
увеличивается число новых компаний.

На самом деле, прирост скорости будет происходить вдвойне. Люди не
будут долго ждать, чтобы воплотить свои идеи, но и таких идей все
больше будет разрабатываться в стартапах, а не в больших компаниях. И
это значит что технология будет быстро распространяться и по
компаниям.

В больших компаниях не всегда получается делать что-то быстро. Я
недавно говорил с основателем, чей стартап был куплен большой
компанией. Он любил точность, и поэтому оценил продуктивность работы
до и после покупки. Он посчитал количество строк нового кода, что
вообще довольно сомнительный показатель, но в той ситуации имело
смысл, потому что работала та же группа программистов. Он заметил, что
после покупки они стали работать в 13 раз менее продуктивно, чем до
нее.

Компания, которая их купила, была не какой-то особенно глупой. Я думаю
что он на самом деле измерил сложность работы больших корпораций. Я и
сам испытывал подобное, и его числа кажутся мне правильными. В больших
компаниях есть что-то, что вытягивает из тебя всю энергию.

Представьте себе, что бы смогла сделать вся эта энергия, если
использовать ее! У хакеров всего мира есть огромный запас энергии, о
котором многие даже не догадываются. Это важнейшая задача у нас в Y
Combinator — выпустить всю эту энергию, дав хакерам возможность
создавать собственные стартапы.

Система труб

Сейчас процесс создания стартапов похож на водопроводные работы в
старом доме. Его трубы узкие и извилистые, с протечками на каждом
стыке. Со временем этот беспорядок будет заменен на единый, необьятный
проводящий канал. Вода так же будет добираться из одной точки в
другую, но будет делать это быстрее, и без риска развеяться при
случайном попадании в место течи.

Это многое изменит к лучшему. В таком большом и надежном канале усилия
каждого отдельного элемента будут распространяться на всю систему.
Производительность всегда является конечным критерием, но сейчас в
трубопроводе столько дефектов, что многие люди большую часть времени
изолированы от потока. И это дает нам мир, где ученики средней школы
думают, что нужно получить хорошие оценки, чтобы поступить в элитные
ВУЗы; ученики ВУЗов думают что им надо получить хорошие оценки, чтобы
впечатлить работодателей, а их подчиненные больше всего времени тратят
на служебные отношения. Но покупатели все равно должны пользоваться их
продуктами, потому что у них нет другого выбора. Представьте эту
последовательность как один большой, слаженный канал. Тогда
последствия нормального измерения производительности распространяться
везде, вплоть до средней школы, и очистят трубы от всех условностей, с
которыми сейчас оценивают людей. Это и есть будущее веб-стартапов.

\end{document}
