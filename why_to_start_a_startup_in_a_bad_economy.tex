\documentclass[ebook,12pt,oneside,openany]{memoir}
\usepackage[utf8x]{inputenc} \usepackage[russian]{babel}
\usepackage[papersize={90mm,120mm}, margin=2mm]{geometry}
\sloppy
\usepackage{url} \title{Зачем запускать стартап во время кризиса}
\author{Пол Грэм} \date{}
\begin{document}
\maketitle

Экономическая ситуация так очевидно мрачна, что некоторые эксперты
боятся, что мы можем ожидать такого же плохого периода, как в середине
семидесятых.

Когда были основаны Microsoft и Apple.

Как подсказывают эти примеры, спад может не быть сильно плохим
временем для запуска стартапа. Я также не заявляю, что это хорошее
время. Правда гораздо скучнее: состояние экономики не сильно влияет в
любом случае.

Я выучил одну вещь из основания множества стартапов. Они успешны или
проваливаются в зависимости от качеств основателей. Экономика,
конечно, дает некоторый эффект, но в качестве предсказания успеха она
равна ошибке округления по сравнению с основателями.

Что означает, что имеет значение, кто вы, а не когда вы это делаете.
Если вы верный человек, вы выиграете даже в кризис. А если нет,
хорошая экономическая ситуация вас не спасет. Кто-то, кто думает «Мне
лучше не начинать стартап сейчас, потому что экономика так плоха»
делает ту же ошибку, что и люди, думающие во время пузыря [доткомов]
«Всё, что мне нужно, это запустить стартап и я богат».

Так что, если вы хотите увеличить ваши шансы, нужно гораздо больше
думать о том, кого вы можете завербовать как со-основателя, чем о
состоянии экономики. И, если вы беспокоитесь об этих угрозах выживанию
вашей компании, не ищите их в новостях. Смотрите в зеркало.

Но для любой заданной команды основателей, будет ли дело не окупаться
до взятия барьера в ожидании улучшения экономики? Если вы запускаете
ресторан — возможно, но не в случае работы с технологиями.
Технологический прогресс более или менее независим от фондовой биржи.
Так что для любой предлагаемой идеи вознаграждение за быстрые действия
в период кризиса будет выше, чем за ожидание. Первым продуктом
Microsoft был интерпретатор Бейсика для компьютера Altair. Это было
именно то, в чем нуждался мир в 1975 году, но если бы Гейтс и Аллен
решили подождать пару лет, было бы слишком поздно.

Конечно же, идея, которую вы имеете сейчас, не будет вашей последней.
Всегда есть новые идеи. Но если есть особая идея над которой вы хотите
работать, действуйте сейчас. Это не означает, что нужно игнорировать
экономику. И покупатели и инвесторы будут чувствовать себя стеснённо.
Ощущение стеснённости у покупателя не обязательно проблема: вы даже
может получить выгоду от этого, делая вещи, которые экономят деньги.
Стартапы часто делают вещи дешевле, с таким отношением они более
расположены к процветанию во время спада, чем крупные компании.

Инвесторы — большая проблема. Стартапы обычно нуждаются в привлечении
некоторого количества внешнего капитала, а инвесторы склонны к
меньшему желанию инвестировать в плохие времена. Им не стоит делать
это. Все знают, что предполагается покупать в тяжёлые времена и
продавать в хорошие. Но, конечно же, что делает инвестиции столь
неинтуитивными, так это рынок ценных бумаг. Хорошие времена
определяются, когда все думают «время покупать». Вы должны вести себя
наоборот для верности, а только меньшинство инвесторов способны на это
по определению.

Точно так же, как инвесторы в 1999 спотыкались друг о друга в попытках
купить вшивые стартапы, так и инвесторы в 2009, возможно, будут
вынуждены инвестировать даже в хорошие.

Вам нужно адаптироваться к этому. Но здесь нет ничего нового: стартапы
всегда должны подстраиваться под капризы инвесторов. Спросите любого
основателя в любой момент экономики, опишут ли они инвесторов как
непостоянных, и посмотрите на лицо, которое они сделают. В последний
год вам нужно было приготовиться описать, почему и как ваш стартап
вирусный. В следующем году вам понадобится объяснять, почему и как он
кризисо-устойчив.

(Обе эти вещи хороши. Ошибка инвесторов не в используемом критерии, а
в том, что они всегда стремятся сфокусироваться на одном, исключая
остальные.)

К счастью, способ сделать стартап кризисо-устойчивым, это как раз то,
что вам нужно в любом случае: запускать его как можно более дешевым.
Годами я говорил основателям, что надёжнейший маршрут к успеху — быть
тараканами корпоративного мира. Непосредственная причина смерти —
всегда издержки. Чем дешевле деятельность вашей компании, тем сложнее
её убить. К счастью, стало очень дёшево запустить стартап, и
экономический спад может лишь поддержать эту дешевизну.

Если ядерная зима действительно настала, стать тараканом может быть
даже безопаснее, чем удерживать свою работу. Покупатели могут
расходиться по одному, если они не смогут больше позволить вас себе,
но вы не потеряете их всех одновременно. Рынки не «сокращают штаты».

Что, если вы бросите работу для запуска стартапа, который провалится,
и не сможете найти другую? Это может быть проблемой, если вы работаете
в продажах или маркетинге. В таких областях можно провести месяцы в
поиске новой работы во время экономического спада. Хорошие хакеры
[высококвалифицированные компьютерные специалисты] всегда получают
какую-нибудь работу. Может это и не работа вашей мечты, но вы не
умрёте с голоду.

Другое преимущество плохих времён в меньшей конкуренции. Поезда
технологии отправляются с равным интервалом. Если все остальные жмутся
в углу, вы можете забрать всю машину себе.

Вы также и инвестор. Как основатель, вы покупаете капитал работой:
причина такого богатства Ларри и Сергея не столько в том, что они
сделали работы на десятки миллиардов долларов, а в том, что они были
первыми инвесторами в Google. И вы, как любой инвестор, должны
покупать в плохие времена.

Кивали ли вы одобрительно, думая «тупые инвесторы» несколькими
параграфами выше, когда я говорил о том, как инвесторы сопротивляются
вкладыванию денег в стартапы на плохих рынках, даже если в это время
было бы наиболее разумно их покупать? Отлично. Основатели не намного
лучше. Когда времена становятся плохими, хакеры идут в магистратуру. И
нет сомнений, что в этот раз случится то же самое. По существу,
истиной предыдущий параграф делает то, что большинство читателей не
верят в него — по крайней мере настолько, чтобы действовать в
соответствии.

Так что, возможно, экономический спад — хорошее время для запуска
стартапа. Тяжело сказать, перевешивают ли преимущества вроде
отсутствия конкуренции недостатки подобные неохотным инвесторам. Но
это не важно в любом случае. Люди — вот что важно. А для заданного
набора людей, работающих над заданной технологией, время всегда
настоящее.

\end{document}
