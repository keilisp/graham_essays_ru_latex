\documentclass[ebook,12pt,oneside,openany]{memoir}
\usepackage[utf8x]{inputenc} \usepackage[russian]{babel}
\usepackage[papersize={90mm,120mm}, margin=2mm]{geometry}
\sloppy
\usepackage{url} \title{Как поднимать деньги} \author{Пол Грэм}
\date{}
\begin{document}
\maketitle

Большинство стартапов, привлекающих финансирование, делают это не
единожды. Типичный путь для молодой компании выглядит примерно
следующим образом: (1) получить несколько десятков тысяч долларов от
бизнес-инкубатора или бизнес-ангелов, чтобы начать работу, (2)
привлечь несколько сотен тысяч либо несколько миллионов, чтобы
построить компанию, и (3) когда успех компании станет очевидным,
привлечь еще один или несколько раундов для ускорения роста.
Реальность более разнообразна. Некоторые компании привлекают по
несколько раундов на втором этапе. Некоторые компании пропускают
первый этап и переходят сразу ко второму. Y Combinator получает все
больше заявок от компаний, которые уже привлекли сотни тысяч долларов.
Но в любом случае, три описанных этапа – это то, через что, с учетом
возможных отклонений, проходят более или менее все компании. Данный
текст касается в основном второго из этих этапов. Именно на этом этапе
находятся стартапы нашего акселератора, выступающие в «день
презентаций» (Demo Day), и эта статья излагает те советы, которые мы
им даем. Действующие силы. Привлечение средств – это трудная задача во
всех смыслах этого слова: это трудно в том же смысле, в каком трудно
поднять тяжелый вес, и трудно в том же смысле, в каком трудно решить
сложную головоломку. Трудно в первом смысле, потому что в принципе
тяжело убедить людей расстаться с большой суммой денег. Эта задача не
сводится к более простым; она должна быть трудной сама по себе. Но
трудность во втором смысле можно по большей части устранить.
Привлечение денег только кажется запутанным: для большинства
основателей стартапов это незнакомый мир. Я попробую очертить карту
этого мира. Поведение инвесторов часто кажется основателям загадочным
– отчасти из-за того, что мотивация инвесторов непонятна, но отчасти
также из-за того, что инвесторы намеренно их дезориентируют. И такое
поведение инвесторов, в сочетании со склонностью неопытных
предпринимателей принимать желаемое за действительное, может привести
к неутешительным результатам. В YC мы всегда предупреждаем основателей
о такой опасности, и инвесторы, по всей видимости, более осмотрительны
в разговоре со стартапами из YC, чем с другими компаниями – но тем не
менее, мы постоянно наблюдаем последствия таких ситуаций.[1] Если вы –
неопытный основатель бизнеса, для вас единственный способ выжить – это
налагать на себя внешние ограничения. Нельзя доверять своей интуиции.
Я надеюсь дать вам набор правил, которые помогут вам пройти процесс
разговора с инвесторами – если такие правила в принципе существуют. В
какой-то момент у вас будет искушение их проигнорировать. Поэтому –
правило номер ноль: эти правила написаны не просто так. Вам бы не
потребовалось правило, предписывающее вам делать одно, если бы не было
мощных сил, склоняющих вас делать противоположное. Те силы, которые
действуют на вас, вызваны в свою очередь силами, действующими на
инвесторов. Инвесторы зажаты между двумя страхами – страхом
проинвестировать в стартап, который провалится, и страхом не
проинвестировать в компанию, которая взлетит. Причина этого страха –
тот же самый фактор, который делает стартапы такой привлекательной
инвестицией: успешные стартапы растут очень быстро. Но быстрый рост
означает, что инвесторы не могут сидеть и ждать. Если ждать, когда
стартап точно станет успешным, будет уже слишком поздно. Чтобы
получить максимальную отдачу от инвестиций, надо инвестировать в
стартапы тогда, когда их возможный успех еще не стал очевидным. Но в
этом случае инвесторы боятся сделать неудачную инвестицию. И
действительно, такие неудачные инвестиции часто случаются. Инвесторы
хотели бы иметь возможность подождать с вложением. Когда стартапу
всего несколько месяцев, с каждой неделей они получают массу новой
информации о нем. Но если ждать слишком долго, другие инвесторы могут
их опередить. И, конечно, на других инвесторов действуют те же самые
силы. Поэтому, как правило, все инвесторы ждут столько, сколько это
возможно – и когда кто-то из инвесторов начинает действовать,
остальные тоже вынуждены двигаться вперед. Привлекайте деньги только в
том случае, когда это необходимо и возможно. Большая часть успешных
стартапов привлекает внешнее финансирование – поэтому может
показаться, что привлечение денег – определяющая черта стартапа. На
самом деле это не так. Основным отличительным признаком стартапа
является быстрый рост. Большинство компаний, имеющих потенциал
быстрого роста, (а) могут расти еще быстрее при помощи дополнительного
финансирования, и (б) благодаря своему потенциалу роста легко могут
привлечь деньги. Поскольку так много успешных стартапов удовлетворяют
обоим условиям, практически все они привлекают внешнее финансирование.
Но могут быть случаи, когда у компании нет потребности расти еще
быстрее, либо дополнительные инвестиции не помогут ускорить рост. Если
ваша компания подпадает под один из этих случаев, не стоит тратить
время на привлечение инвестиций. Второй случай, когда искать
инвестиции не нужно – это когда это невозможно. Если вы будете
пытаться привлечь инвестиции до того, как у вас появятся убедительные
аргументы для инвесторов, вы потратите свое время и испортите свою
репутацию перед ними. Будьте либо в «режиме поиска инвестиций», либо
нет. Один из самых больших сюрпризов для основателей при привлечении
финансирования – это то, насколько это отвлекает от работы над
проектом. Когда вы начинаете работать над привлечением денег, вся
остальная работа фактически останавливается. Проблема не в том, что
оно отнимает много времени, а в том, что оно становится наивысшим
приоритетом в вашей голове. Стартап не может долго переносить такое
отвлечение внимания. Стартап растет на ранней стадии в первую очередь
благодаря постоянным усилиям основателей – и если те переключаются на
что-то еще, темпы роста резко уменьшаются. Поскольку привлечение денег
требует столько внимания, стартап должен либо находиться в режиме
поиска инвестиций, либо нет. И если вы решили привлечь деньги, вам
нужно полностью сосредоточиться над этой задачей, чтобы решить ее
быстро и вернуться к работе.[2] Вы можете получать деньги от
инвесторов, и не находясь в режиме поиска. Вам просто не следует
распылять на это свое внимание. Внимание в первую очередь тратится на
две вещи – на убеждение инвесторов и на переговоры с ними.
Соответственно, если вы не ведете поиск средств, принимайте деньги от
инвесторов только тогда, когда это не требует убеждения, и когда они
согласны проинвестировать на тех условиях, которые вы готовы принять
без переговоров. Например, если инвестор с хорошей репутацией
предлагает вам конвертируемый заем со стандартными условиями, без
ограничения по цене конвертации либо с достаточно высоким
ограничением, такое предложение можно принять без размышлений.[3] В
этом случае условия инвестиций будут де-факто определяться вашим
следующего раунда акционерного финансирования. Под «не требует
убеждения» имеется в виду, что вам не нужно будет тратить время на
встречи с инвесторами и на подготовку материалов для них. Если
инвестор говорит, что они готовы инвестировать, но нужно встретиться с
кем-то из партнеров, то если вы не ищете деньги специально, говорите
«нет» – участие в такой встрече уже представляет собой работу по
поиску средств.[4] Ответьте им вежливо – скажите, что сейчас вы заняты
бизнесом и вернетесь к ним, когда будете искать средства. Не
позволяйте затянуть вас на скользкий путь. Инвесторы будут стараться
втянуть вас в переговоры и тогда, когда вы не ищете средства. Для них
это выгодно – таким образом, у них будет шанс на сделку с вами прежде
других инвесторов. Они будут вам писать, что хотят встретиться, чтобы
узнать о вас больше. Если вам пишет незнакомый вам менеджер венчурного
фонда, не соглашайтесь на встречу, даже если вы ищете средства. Сделки
так не делаются.[5] Даже если вам пишет партнер, постарайтесь отложить
встречу до момента, когда вы будете активно искать средства. Он может
сказать, что хочет просто встретиться и поболтать, но инвесторы
никогда не хотят встретиться просто поболтать. Что если вы им
понравитесь? Что если они начнут говорить о возможном финансировании?
Способны ли вы будете отклонить такой разговор? Если только у вас нет
достаточного опыта в привлечении средств, чтобы провести с инвестором
неформальную беседу, которая останется неформальной, лучше будет
сказать, что будете рады встретиться позже, когда будете активно
искать средства, но сейчас вы заняты бизнесом.[6] Иногда компании,
которые успешно привлекли средства на втором этапе, сохраняют
отношения с несколькими инвесторами и после того, как завершили поиск
средств. Это совершенно нормально – если вы успешно привлекли
средства, вы сможете общаться с инвесторами без необходимости убеждать
их в чем-либо и вести переговоры. Просите представить вас инвесторам.
Прежде чем говорить с инвесторами, необходимо, чтобы вас им кто-то
представил. Если вы выступите на Demo Day, вы уже будете представлены
сразу большой группе инвесторов. Но даже в этом случае, вам в
дополнение к этому необходимо выходить на других инвесторов
самостоятельно. Необходимо ли, чтобы вас кто-то знакомил с
инвесторами? На этапе 2 – необходимо. Некоторые инвесторы пишут, что
готовы принять бизнес-планы по электронной почте, но уже из того, как
выглядят их сайты, видно, что на самом деле им не нужны стартапы,
обращающиеся к ним напрямую. Не все пути выхода на инвесторов
одинаково эффективны. Самый лучший способ – если вас представляет
известный инвестор, который сам только что вложился в вашу компанию.
Поэтому, когда вы получаете от инвестора предложение, попросите его
представить вас другим инвесторам, которых они могли бы
рекомендовать.[7] Следующий вариант – это рекомендация от основателя
другой компании, в которую они вкладывались. Можно выходить на
инвесторов и через других представителей стартап-сообщества, например,
через юристов и журналистов. Можно выходить на инвесторов через сайты
– такие как AngelList, FundersClub и WeFunder. Мы рекомендуем
стартапам относиться к таким сайтам, как к дополнительному каналу
привлечения. В первую очередь, принимайте финансирование от тех, на
которых вы вышли сами. Как правило, такие инвесторы лучше. Также, вам
будет проще привлечь средства с помощью сайтов, когда у вас уже есть
вложения от известных инвесторов. Воспринимайте любой ответ как «нет»,
пока не услышите твердое «да». Считайте, что инвесторы не согласны в
вас вкладываться, пока они не скажут вам «да» – в форме твердого
безусловного предложения. Как я уже сказал, инвесторам выгодно ждать,
пока это возможно. Особенно опасна для предпринимателей та форма, в
которой они это делают. Фактически, они вас вводят в заблуждение. Они
делают вид, что они вот-вот совершат сделку, до тех пор, пока они не
сказали «нет». В худшем случае они вообще не скажут «нет» – они просто
перестанут отвечать на письма. Инвесторы надеются, что это оставит им
возможность проинвестировать в дальнейшем. Если они решат, что готовы
вложиться – например, потому что им скажут, что ваша компания –
выгодная инвестиция – они смогут сделать вид, что просто на что-то
отвлеклись, и вернуться к разговору как ни в чем не бывало.[8] Это не
худшее, чего можно ждать от инвесторов. Некоторые инвесторы говорят
так, как будто уже договорились о сделке – тогда как на самом деле не
берут на себя никаких обязательств. И основатели, принимающие желаемое
за действительное, готовы им поверить.[9] Следующее правило
представляет собой тактику против такого поведения инвесторов. Но
чтобы его правильно применить, необходимо не дать себя обмануть
отрицательным ответом, выглядящим как положительный. Основатели
настолько часто здесь ошибаются, что мы разработали специальный
протокол, чтобы решить эту проблему. Если вы считаете, что сделка уже
обговорена – попросите инвестора подтвердить это. Если ваши с ним
взгляды расходятся – из-за того, что инвестор выразился неточно, или
из-за того, что вы выдавали желаемое за действительное – необходимость
подтвердить предложение в письменном виде позволит это выявить. И пока
инвестор не даст подтверждения, считаете, что он инвестировать не
будет. Ведите поиск в ширину с учетом ожидаемого выигрыша При общении
с инвесторами следует придерживаться стратегии, которую программисты
называют «поиск в ширину». С инвесторами надо говорить параллельно, а
не последовательно. Последовательное общение с инвесторами занимает
больше времени, к тому же, они не чувствуют конкуренции, которая
заставляла бы их быстрее продвигаться к заключению сделки. При этом не
всем инвесторам следует уделять равное внимание: некоторые инвесторы
являются более перспективными, чем другие. Оптимальным решением будет
вести диалог параллельно со всеми потенциальными инвесторами, но
уделять больше времени наиболее перспективным.[10] Под ожидаемым
выигрышем мы понимаем вероятность, с который инвестор согласится
инвестировать, помноженную на выгоду для вас от такой инвестиции.
Например, фонду с хорошей репутацией, который может инвестировать
крупную сумму, но которого сложно убедить сделать сделку, может
соответствовать такой же ожидаемый выигрыш, как малоизвестному
бизнес-ангелу, который вложит меньше, но убедить которого проще. В то
же время, ожидаемый выигрыш от малоизвестного бизнес-ангела, который
готов вложить лишь небольшую сумму и которому необходимо множество
встреч, прежде чем принять решение, совсем низок. С такими инвесторами
надо встречаться в последнюю очередь, либо не встречаться вообще.[11]
Такая стратегия защитит вас от инвесторов, которые не отказывают вам
явно, но уходят от сделки – фактически вы также будете уходить от
разговоров с ними. Примерно так же распределенный алгоритм может
защитить вас от сбоев в части процессоров. Если кто-то из инвесторов
перестает отвечать на письма или требует множество встреч без
какого-либо прогресса в сделке, вы автоматически будете тратить на
него меньше времени. Но при оценке вероятностей необходим
самоконтроль. То, насколько вы хотите сделку с каким-либо конкретным
инвестором, не должно искажать ваш взгляд на то, насколько он сам
хочет иметь с вами дело. Твердо представляйте себе процесс. Как
оценить ход вашего диалога с инвесторами, учитывая, что инвесторы
всегда кажутся более оптимистичными, чем на самом деле? Необходимо
судить по их действиям, а не словам. У каждого инвестора есть
определенный процесс, который он должен пройти от первого разговора до
перечисления денег, и необходимо всегда понимать, из чего этот процесс
состоит, на каком этапе вы сейчас находитесь и насколько быстро вы
продвигаетесь. Никогда не завершайте встречу с инвестором, не спросив
о дальнейших шагах. Что еще им необходимо, чтобы принять решение?
Нужны ли им дополнительные встречи? Что будет обсуждаться на этих
встречах? Когда они будут? Нужно ли им что-то обсудить между собой,
либо со своими партнерами, либо прояснить какой-то вопрос? Сколько это
займет времени? Не будьте слишком настырным, но выясните, на каком вы
этапе. Если инвесторы отвечают на эти вопросы нечетко или уклончиво,
предполагайте худшее – серьезно настроенные инвесторы, как правило,
будут рады рассказать вам о шагах, необходимых для закрытия сделки,
поскольку они уже прокручивают эти шаги у себя в голове.[12] Если у
вас есть опыт в переговорах, то, скорее всего, вы уже знаете, как
задавать такие вопросы.[13] Если нет – отсутствие опыта не снижает
вашей инвестиционной привлекательности. Технологический стартап может
быть непривлекательным из-за отсутствия у основателей опыта в
технологиях, но не из-за отсутствия опыта в привлечении инвестиций.
Сергей Брин и Ларри Пейдж тоже были новичками в поиске инвестиций.
Просто признайтесь инвестору, что у вас нет опыта в привлечении денег,
и попросите их рассказать, как выглядит процесс и на каком этапе вы
сейчас находитесь.[14] Добейтесь первого предложения. Важнейший
фактор, влияющий на мнение о вас большинства инвесторов – это мнение
других инвесторов. Как только вы получаете первые предложения от
инвесторов, получить последующие предложения становится гораздо проще.
Но с другой стороны, получить первое предложение, как правило, тяжело.
Получение первого значимого предложения часто занимает половину
времени и усилий во всем процессе привлечения средств. То, насколько
предложение является значимым, зависит от суммы и от того, кто это
предложение делает. Средства от семьи и близких, как правило, не
принимаются в расчет независимо от суммы. Но если вы получаете 50 тыс.
долл. от известного венчурного фонда или бизнес-ангела, как правило,
этого достаточно, чтобы получить и другие предложения.[15] Дождитесь
перевода денег. Сделка не завершена, пока деньги не поступили на
банковский счет. Неопытные основатели часто говорят, например, «Мы
получили 800 тысяч долларов», хотя на самом деле сумма, полученная из
этих денег, пока равна нулю. Помните о страхах, которые терзают
инвесторов – о страхе упустить выгодную возможность и о страхе
вложиться неудачно? На венчурном рынке покупатели очень часто жалеют о
только что сделанных инвестициях. И многие вещи могут стать для них
поводом отказаться от сделки. Поведение инвесторов сильнейшим образом
зависит от конъюнктуры фондовых рынков. Если завтра китайская
экономика обвалится, все договоренности будут отменены. Есть много
факторов, из-за которых может сорваться и сделка с отдельным
стартапом. Вдруг у вас завтра появится опасный конкурент, либо суд
вынесет решение против вас, либо уйдет один из основателей?[16] Даже
при промедлении в один день может случиться что-то, из-за чего
инвестор передумает. Поэтому, когда инвестор делает предложение,
дождитесь получения денег. Необходимо понимать, на каком этапе
процесса вы находитесь, и после того, как предложение сделано. Если
инвестор сказал «да», выясните график следующих шагов для получения
денег, и контролируйте процесс, пока средства не будут в вашем
распоряжении. В фондах перечислением денег могут заниматься отдельные
люди, но при работе с ангелами вам может понадобиться отловить их
лично, чтобы получить чек. Как правило, о только что подписанных
сделках жалеют неопытные инвесторы. Те, кто уже давно работает на
венчурном рынке, как правило, относятся к сделанному предложению как к
прыжку с вышки. Также они беспокоятся о своей репутации. Но и у самых
крупных венчурных фондов были случаи, когда они отказывались от
подписанных сделок. Избегайте инвесторов, не выступающих в качестве
«ведущих». Поскольку труднее всего получить первое предложение, при
расчете вашего описанного выше ожидаемого выигрыша следует учитывать
не только вероятность того, что инвестор представит предложение, но и
того, что он даст предложение первым. Не для всех инвесторов эти
вероятности пропорциональны. Некоторые инвесторы известны тем, что
принимают решения быстро, и именно они наиболее полезны на первых
этапах процесса. Соответственно, инвесторы, которые инвестируют только
при участии других инвесторов, на первых этапах процесса бесполезны.
Большинство инвесторов смотрят на то, насколько вы интересны другим,
но у ряда инвесторов принята официальная политика инвестировать только
при участии других инвесторов. Такие люди часто говорят о «ведущих» (
или «якорных») инвесторах. Они, как правило, говорят, что не выступают
в качестве ведущих инвесторов, либо что проинвестируют, как только у
вас будет ведущий инвестор. Иногда они даже говорят, что готовы
выступать в качестве «ведущего» инвестора, имея при этом в виду, что
не будут инвестировать, пока вы не получите определенную сумму от
других инвесторов. (Хорошо, если называя себя «ведущими», они имеют в
виду, что готовы вложиться в одностороннем порядке и в дополнение к
этому помогут вам привлечь других инвесторов. Плохо, когда этот термин
используют те, кто не будет инвестировать, пока вы не привлечете
средства от других.)[17] Откуда происходит термин «ведущий инвестор».
За исключением последних нескольких лет, стартапы, привлекающие деньги
на этапе 2, как правило, продавали акции или доли раундами. В каждом
раунде множество инвесторов предоставляло деньги одновременно и на
однотипных условиях. Можно было согласовать условия с одним «ведущим»
инвестором, а затем подписать одинаковую документацию и провести
расчеты одновременно со всеми участниками. Инвестиции «серии А» и
сейчас согласовываются по такому же принципу, но финансирование на
более ранних этапах сейчас привлекается по-другому. Сейчас стартапы
редко привлекают деньги раундами до «серии А» – как правило, стартапы
получают финансирование от отдельных инвесторов по одному по мере
необходимости. Почему же тогда инвесторы до сих пор ссылаются на
«ведущих» участников? Потому что это аккуратный способ сказать то, что
они действительно думают – то есть то, что их заинтересованность
целиком и полностью определяется заинтересованностью в вас других
участников. Фактически, такое мышление – это отличительный признак
посредственного инвестора. Но при формулировке в терминах «ведущего»
инвестора это выглядит как что-то структурированное, рациональное, и,
соответственно, адекватное. Когда инвестор говорит: «Я бы хотел
проинвестировать, но я не выступаю в качестве ведущего инвестора»,
следует это понимать как «Нет, если только вы не окажетесь популярным
предложением». А поскольку так по умолчанию думает о стартапе любой
инвестор, по факту инвестор не сказал ничего значимого. Когда вы
начинаете искать средства, ожидаемый выигрыш от инвесторов, которые не
выступают в качестве ведущих, равен нулю. Контактируйте с такими
инвесторами в последнюю очередь, либо не контактируйте вообще. Имейте
множество альтернативных планов. Многие инвесторы будут вас
спрашивать, какую сумму вы планируете привлечь. Это может создать
впечатление, что вам следует планировать привлечь конкретную сумму. На
самом деле это не так. В таком непредсказуемом деле, как поиск
финансирования, будет ошибкой иметь жесткие планы. Так почему же
инвесторы спрашивают, сколько вы планируете привлечь? Потому же,
почему продавец в магазине спрашивает, когда вы выбираете подарок
другу, сколько вы планируете потратить. Скорее всего, у вас нет в
голове точной суммы – вы просто хотите найти что-то подходящее, и если
оно недорогое, то тем лучше. Продавец вас спрашивает не потому, что от
вас требуется иметь точный план. Он хочет подобрать для вас подходящие
вещи в нужном вам ценовом диапазоне. Аналогично, когда инвесторы
спрашивают, сколько вы планируете привлечь, они это делают не потому,
что от вас требуется план. Они хотят видеть, насколько ваша компания
подходящий объект для инвестиций в их целевом диапазоне, а также
насколько вы амбициозны, адекватны и на каком этапе вы находитесь в
процессе поиска финансирования. Если вы супер-опытны в разговорах с
инвесторами, вы можете сказать: «Мы планируем привлечь 7 миллионов
долларов в раунде А и принимаем терм-шиты до следующего вторника». Я
знаю нескольких основателей, которые могли бы это сделать так, чтобы
представители фондов не рассмеялись им в лицо. Но если вы –
представитель неопытного, но честного большинства, то рекомендация та
же, что и для питча перед инвесторами: придерживайтесь правильной
стратегии и объясняйте инвесторам, что именно вы делаете. А правильная
стратегия при привлечении средств – иметь несколько планов в
зависимости от того, сколько вы привлечете. В идеале, вы должны
сказать инвесторам следующее – мы сможем стать прибыльными, даже если
вообще не привлечем денег; если мы привлечем несколько сотен тысяч, то
сможем нанять одного или двух толковых сотрудников, а если привлечем
пару миллионов, то сможем нанять целую команду инженеров, и т. д. Для
различных инвесторов подходят различные планы. Если вы говорите с
фондом, который участвует не менее чем в раундах А (хотя таких фондов
сейчас мало), стоит рассказать в деталях только про самый амбициозный
план. Если вы говорите с бизнес-ангелом, который инвестирует не более
20 тысяч за раз, и пока еще не привлекли денег, опишите в первую
очередь самый скромный план развития. Если вам повезло и вам
приходится думать о том, на какой максимальной сумме остановиться,
сделайте простую прикидку – умножьте количество сотрудников, которых
хотите нанять, на 15 тысяч долларов и на 18 месяцев. В большинстве
стартапов практически все расходы пропорциональны количеству людей, и
типичные затраты в месяц на человека составляют 15 тысяч долларов
(включая соцпакет и даже аренду офиса). Пятнадцать тысяч в месяц – это
большая сумма, не тратьте ее полностью. Но при привлечении денег можно
взять именно такую сумму с запасом. Если у вас будут дополнительные
затраты – например, на производство – добавьте их сверху. В
предположении, что у вас таких затрат нет и что вы хотите нанять,
например, 20 человек, максимальная сумма, которую вам стоит
привлекать, составляет 20 x \$15 тыс. x 18 = 5,4 миллиона
долларов.[18] Начните с заниженной суммы. Хотя разным инвесторам можно
озвучивать разные планы, вам стоит в целом скорее называть заниженную,
нежели завышенную, сумму в ответ на вопрос, сколько вы планируете
привлечь. Например, если вы планируете привлечь 500 тысяч долларов,
сначала имеет смысл сказать, что вы планируете привлечь 250 тысяч. В
таком случае, когда у вас будут предложения на 150 тысяч, объем будет
покрыт уже более чем наполовину. Это даст инвесторам понять две вещи:
во-первых, что у вас дела идут хорошо, и во-вторых, что им следует
принимать решение быстро, пока для них еще есть место. И напротив –
если вы сказали, что планируете привлечь 500 тысяч, то 150 тысяч – это
меньше трети планируемого объема. Если процесс поиска средств надолго
застопорится на этой отметке, то про сделку начнут думать как про
неудачную. Если вы вначале скажете, что планируете привлечь 250 тысяч,
вы не будете обязаны на этой сумме остановиться. Если ваша
первоначальная цель будет достигнута и у инвесторов все еще останется
интерес, вы сможете повысить объем. Вообще, большинство компаний, у
которых процесс поиска финансирования идет успешно, привлекает больше
средств, чем изначально планировалось. Я имею в виду не то, что вы
должны лгать, а то, что вам необходимо изначально умерить свои
ожидания. Нет практически никаких рисков в том, чтобы начать с более
низкой суммы. Это не только не ограничит объем привлеченных средств,
но и в конечном счете скорее позволит его увеличить. Хорошая аналогия
в данном случае – угол атаки крыла самолета. Если вы попробуете лететь
со слишком высоким углом атаки, вы свалитесь в штопор. Если вы с
порога скажете, что хотите привлечь раунд А объемом 5 миллионов, вы не
только не получите эту сумму. Вы не получите вообще ничего – если
только у вас нет каких-то исключительных сильных преимуществ. Лучше
начать с низкого угла атаки, набрать скорость, а затем постепенно
наращивать угол, если вам это надо. По возможности будьте
рентабельными. Вы будете выглядеть гораздо увереннее, если в ваших
планах будет вариант действий при отсутствии внешнего финансирования –
то есть, если вы сможете стать рентабельным без привлечения
дополнительных средств. В идеале, вы должны иметь возможность сказать
инвесторам: «Мы добьемся успеха независимо от исхода финансирования,
но дополнительные деньги помогли бы нам вырасти быстрее». Между
привлечением средств и романтическими отношениями есть много общего,
но одна из самых убедительных аналогий следующая: никто не захочет
встречаться с вами, если вы кажетесь отчаявшимся. И лучший способ не
выглядеть отчаявшимся – это не быть им. Именно по этой причине мы
настойчиво просим у компаний YC к моменту Demo Day добиться того, что
мы называем «доширак-рентабельностью». Это может показаться
парадоксальным, но если вы хотите привлечь деньги, лучше всего в них
не нуждаться. Фактически есть два различных сценария при привлечении
средств – в одном случае основатели, которым нужны средства, обходят
инвесторов в поисках денег, потому что иначе придется закрывать
компанию или, по крайней мере, увольнять людей. Во втором случае
основатели, для которых деньги не критичны, привлекают средства, чтобы
расти быстрее, чем просто за счет своей выручки. Неопытные основатели
читают об известных стартапах, привлекавших средства фактически по
второму сценарию, и решают, что тоже должны привлечь деньги – так, по
их мнению, работают стартапы. Но те компании, которые при привлечении
денег не имеют четкой стратегии выхода на рентабельность, вынуждены
разговаривать с инвесторами по первому сценарию – и удивляются тому,
насколько это тяжело и неприятно. Конечно, не все стартапы могут стать
«доширак-рентабельными» за несколько месяцев. И даже среди тех,
которые не могут, есть те, кто становится в диалоге с инвесторами
хозяевами положения за счет других преимуществ – например, за счет
исключительно высоких темпов роста или исключительно сильной команды
основателей. Но при отсутствии рентабельности удерживать в диалоге с
инвесторами сильную позицию становится все сложнее и сложнее с
течением времени.[19] Не стремитесь к максимальной оценке. На какую
оценку вам следует ориентироваться при привлечении денег? Важнейшее,
что надо понять про оценку – это то, что оценка не слишком важна.
Зачастую основатели необоснованно гордятся высокой оценкой, по которой
они привлекли деньги. Многие основатели любят дух соперничества, а
поскольку оценка – это, как правило, единственная публично доступная
характеристика стартапа, они соревнуются, по какой оценке они
привлекут деньги. Это глупо: привлечение денег не является самоцелью
для компании. Настоящей целью является выручка. Привлечение
финансирования – это всего лишь средство. Гордиться тем, как успешно
вы привлекли деньги – все равно, что гордиться своими оценками в
институте. Даже учитывая, что процесс привлечения финансирования не
является самоцелью, оценка – не самое важное, к чему надо стремиться в
этом процессе. Главное на этапе 2 – получить необходимую сумму денег,
чтобы вернуться к бизнесу и сосредоточиться на успехе своей компании.
Второе по важности – найти хороших инвесторов. Оценка – максимум
третий пункт в списке. То, насколько оценка несущественна, можно
увидеть на примерах. Dropbox и Airbnb – самые успешные на данный
момент компании, которые участвовали в нашем акселераторе. После
прохождения Y Combinator они привлекли деньги по оценкам (до
поступления денежных средств) 4 миллиона и 2,6 миллиона долларов,
соответственно. Сейчас цены настолько выросли, что если вы вообще
можете привлечь деньги, вы их, скорее всего, привлечете по более
высокой оценке. Возможно, это удовлетворит ваше тщеславие. Вы уже
обогнали Dropbox и Airbnb! По параметру, который не имеет значения.
Когда вы начнете искать денежные средства, ваша изначальная оценка
(или максимальная цена конвертации) будет определяться ценой сделки с
тем инвестором, который сделал первое предложение. Если к вам будет
значительный интерес, то для других инвесторов вы сможете повысить
цену, но в любом случае, оценка, предложенная первым инвестором – это
то, с чего вы будете начинать. Поэтому, если вы привлекаете деньги от
нескольких инвесторов – как делает большинство компаний на этапе 2 –
вам следует опасаться привлечь первые деньги от слишком оптимистичного
инвестора по оценке, которую вам не удастся удержать. Вы, конечно, при
необходимости сможете снизить оценку (при этом вам нужно будет
предоставить те же условия инвесторам, проинвестировавшим ранее по
более высокой цене), но в процессе вы сможете потерять много
перспективных инвесторов. Если ваши первые инвесторы
сверхоптимистичны, вы можете принять у них деньги на условиях
конвертируемого займа без изначального ограничения цены конвертации с
условием MFN – то есть, с условием, что если другой инвестор даст вам
деньги по более выгодной оценке, то цена конвертации будет
соответствовать этой оценке. Привлечь деньги проще по более низкой
оценке. Вообще говоря, такого быть не должно. Поскольку оценка на
этапе 2 различается максимум в 10 раз, а доходность от по-настоящему
успешных проектов составляет не менее 100 крат, инвесторы должны
выбирать стартапы исключительно на основе вероятности того, что
компания станет сверхуспешной, а не на основе оценки. Но хотя для
инвесторов будет ошибкой беспокоиться о цене, многие беспокоятся.
Стартапу, который нравится инвесторам, но в который не хотят
инвестировать по цене x, будет проще привлечь деньги по цене x/2.[20]
Принципиальное согласие должно быть важнее, чем оценка. Некоторые
инвесторы захотят узнать ваше мнение об оценке, прежде чем вообще
говорить об инвестициях. Если у вас уже есть ориентир в виде оценки,
по которой вкладывались предыдущие инвесторы, можно его назвать. Но
если такого ориентира нет – поскольку у вас еще нет инвесторов – и при
этом вас настойчиво просят назвать цену, не называйте ее. Привлечение
первого инвестора является переломным моментом в привлечении средств.
Вам надо закрыть сделку с ним как можно скорее, и надо
сконцентрировать разговор на закрытии сделки, не отвлекаясь на
обсуждение цены. Есть способ не называть цену во время беседы с
инвесторами. Это не просто переговорная хитрость, это то, как должны
действовать и вы, и инвестор. Скажите им, что оценка для вас не
главное и вы о ней особенно не думали, и что вы ищете инвесторов,
которые были бы для вас в первую очередь надежными партнерами, и в
первую очередь следует обсудить, согласны ли они вообще инвестировать.
Затем, если они решат, что хотят делать сделку, можно уже согласовать
цену. Но разговаривать надо именно в таком порядке. Поскольку главное
– не добиться высокой оценки, а успешно запустить процесс привлечения
средств, мы обычно рекомендуем основателям согласиться при сделке с
первым инвестором на настолько низкую цену, насколько это необходимо.
Этот путь безопасен, если придерживаться следующего совета.[21]
Избегайте инвесторов, озабоченных оценкой. Некоторые инвесторы
называют себя «чувствительными к оценке». На практике это выливается в
то, что они будут навязчиво торговаться с вами и тратить ваше время на
то, чтобы выбить себе как можно более низкую цену. Соответственно, с
такими инвесторами никогда не нужно вступать в переговоры в начале
процесса. Хоть вы и не должны стремиться к высокой оценке, вам не
следует привлекать средства по искусственно заниженной оценке из-за
того, что первый инвестор, который, на чье предложение вы согласились,
был озабочен вопросом оценки. Некоторые из таких инвесторов могут быть
полезны, но разговаривать с ними следует в конце процесса привлечения
средств, когда вы уже можете сказать – «вот цена, по которой
инвестировали все остальные – если вас она не устраивает, не
инвестируйте». Так вы не только добьетесь справедливой цены, но и
потратите меньше времени. В идеале вам следует знать, какие инвесторы
известны своей чувствительностью к вопросам оценки, и отложить
переговоры с ними до более поздних шагов. Но иногда неизвестный вам
ранее подобный инвестор может появиться в начале процесса. Здесь вам
поможет уже известное правило поиска в ширину с учетом ожидаемого
выигрыша: вы сможете приостановить разговор с такими инвесторами и
сосредоточиться на других. Есть небольшая группа инвесторов,
пытающихся проинвестировать по более низкой оценке, даже когда
ориентир уже установлен. Таких инвесторов можно рассматривать как
запасной вариант на случай, если выяснится, что по установленной цене
вы не можете привлечь столько средств, сколько необходимо. Поэтому с
такими инвесторами имеет смысл говорить, только если вы все равно
собираетесь снижать цену. Но поскольку встречи с инвесторами
назначаются как минимум за несколько дней, и за это время не всегда
можно понять, придется ли снижать цену, на практике вряд ли вообще
имеет смысл выходить на таких инвесторов. Если инвестор предлагает вам
неожиданно и неоправданно низкую цену, отложите это предложение как
запасной вариант и не отвечайте на него. Когда кто-то делает
справедливое предложение, вы можете чувствовать себя обязанным
ответить на него оперативно. Но искусственное занижение цены –
нечестный ход, и отвечать на него надо соответствующе. Принимайте
предложения по принципу «жадного алгоритма». Я немного опасаюсь
использовать термин «жадный алгоритм», так как непрограммисты могут
меня понять неправильно. Жадный алгоритм – это алгоритм, который «не
пытается смотреть в будущее». Жадный алгоритм предписывает выбрать
наилучший из тех возможных вариантов, которые уже есть перед глазами.
И именно так должны действовать стартапы при привлечении денег на
втором и последующих этапах развития. Не стоит пытаться смотреть в
будущее, так как (а) будущее непредсказуемо, и вас могут намеренно
дезориентировать, и (б) вашим первым приоритетом в любом случае должно
быть как можно быстрее завершить привлечение средств и вернуться к
работе. Если кто-то делает вам приемлемое предложение, соглашайтесь.
Если у вас несколько взаимоисключающих предложений, выбирайте лучшее.
Не отказывайтесь от предложения из-за того, что ожидаете получить
более выгодное в дальнейшем. Эти простые правила касаются широкого
круга ситуаций. Если вы привлекаете деньги от большого количества
инвесторов, повышайте цену, пока они на это соглашаются. Когда вам
будет казаться, что вы уже привлекли достаточно, вы сможете повысить
порог того, что считается приемлемым. Как правило, предложения
остаются в силе в течение какого-то периода времени. Поэтому, если вы
получаете приемлемое предложение, несовместимое с другими (например,
если оно покрывает больше половины вашей потребности в средствах), вам
нужно сообщить другим потенциальным инвесторам, что у вас уже есть
достаточно хорошее предложение, и дать им несколько дней, чтобы
сделать свои. При этом могут отпасть те из инвесторов, которые сделали
бы предложение, если бы им дали больше времени, но вас по определению
это не беспокоит: для вас приемлемо уже изначально полученное
предложение. Некоторые инвесторы, чтобы помешать вам получить
предложения от их конкурентов, будут давать вам предложения,
«сгорающие» в течение нескольких дней. Качественные инвесторы, как
правило, такие предложения дают реже и срок принятия их предложений
дольше. Например, Фред Уилсон, основатель Union Square Ventures,
никогда не дает «горящих» предложений, так как уверен, что
предприниматели в любом случае выберут его фонд. Но более мелкие и
менее известные инвесторы очень часто дают предложения с коротким
сроком на размышление, так как считают, что никто, у кого есть
альтернативы, их не выберет. Приемлемым сроком принятия решения можно
считать три дня. Если вы говорите с несколькими инвесторами
параллельно, больше вам не понадобится. Но если вам дают меньший срок,
это может означать, что вы говорите со «скользким» инвестором. Как
правило, их можно поймать на блефе, и возможно, что вам так и придется
сделать.[22] Может показаться, что вместо того, чтобы выбирать
предложения «жадным» способом, вам надо добиваться сотрудничества с
самым качественным инвестором. Безусловно, это достойная цель, но на
этапе 2, как правило, эти две цели не противоречат друг другу,
поскольку наиболее качественные инвесторы, как правило, принимают
решения не дольше, чем все остальные. Единственный случай, когда эти
две стратегии входят в конфликт – это когда вам приходится
отказываться от предложения приемлемого для вас инвестора в надежде
получить предложение от более качественного. Если вы говорите с
инвесторами параллельно и не откладываете предложения со слишком
коротким сроком на размышление, такого практически наверняка не
случится. Но даже если это произойдет, стараться заполучить самого
качественного инвестора будет, скорее всего, плохой стратегией. Лучшие
инвесторы, как правило, действуют наиболее избирательно, так как имеют
возможность выбрать из всех стартапов, представленных на рынке. Они
отказывают практически всем компаниям, с которыми они общаются,
поэтому в среднем невыгодно отказываться от твердого предложения
приемлемого инвестора в обмен на шанс получить более предложение от
более качественного. (На первом этапе роста компании ситуация иная. Вы
не можете подать во все инкубаторы одновременно, поскольку некоторые
из них разносят свои расписания специально, чтобы избежать таких
ситуаций. Здесь две стратегии по отбору предложений действительно
противоречат друг другу – поэтому если вы хотите подать заявки в
несколько инкубаторов, то пусть те инкубаторы, в которые вы хотите
попасть больше всего, смогут решить первыми.) Может случиться так, что
если вы ведете переговоры о привлечении средств с несколькими
инвесторами, разговор пойдет о финансировании серии А. В этом случае
данные правила также применимы. Если инвестор заводит с вами разговор
о раунде серии А, продолжайте принимать более мелкие предложения от
других инвесторов, пока вам не предоставят терм-шит. Здесь нет
практических сложностей. Если мелкие предложения представляют собой
конвертируемые займы, они будут просто сконвертированы в рамках раунда
серии А. Возможно, крупный инвестор не захочет видеть более мелких
инвесторов в вашем капитале – но тогда ему тем более не стоит
затягивать с основными условиями. Пока вам их не предоставили, у вас
нет точной уверенности, что их вам дадут – жадный алгоритм подскажет
вам, что делать.[23] Не продавайте на втором этапе более 25\% доли.
Если ваши дела пойдут хорошо, возможно, вы будете привлекать раунд
серии А. Я говорю «возможно», так как ситуация меняется, и возможно, в
дальнейшем стартапы будут пропускать этот этап. Но из тех компаний,
которые мы финансировали, это удалось только одной, так что можно
предполагать, что путь к успеху лежит через раунд А.[24] Таким
образом, на первоначальных раундах надо избегать вещей, которые могут
помешать привлечению серии А. В частности, если вы продали более 40\%
своей компании до раунда А, привлечь финансирование будет сложно, так
как фонды будут беспокоиться, что у основателей осталось слишком
маленькая доля, чтобы сохранить достаточную мотивацию. Наша общая
рекомендация – не продавать на этапе 2 более 25\% доли, в дополнение к
той доле, которую вы продали на этапе 1 – которая, в свою очередь,
должна быть не более 15\%. Если вы привлекаете финансирование на
условиях займа без максимальной цены конвертации, то вам придется
делать предположения о том, какая может быть оценка в будущем раунде
акционерного финансирования. Предположения должны быть реалистичными.
(Поскольку смысл этого правила – избежать помех при дальнейшем раунде
А, его не надо применять в ситуации, когда привлечение финансирования
на втором этапе выливается в раунд А.) Пусть привлечением средств
занимается один человек. Если у вас несколько основателей, выберите
одного, который будет заниматься привлечением денег, чтобы остальные
могли работать над бизнесом. Как мы уже говорили, основная опасность
процесса общения с инвесторами – не то, сколько времени уходит на сами
встречи, а то, что это становится основной заботой основателей.
Поэтому пусть тот основатель, который занимается финансированием,
приложит усилия, чтобы остальные не вовлекались в детали этого
процесса.[25] (Если между основателями есть недоверие, это может быть
проблематично. Но если основатели не доверяют друг другу, перед вами
более серьезные проблемы, чем привлечение финансирования.) Основатель,
который занимается финансированием, должен быть генеральным директором
и, в свою очередь, наиболее убедительно выглядящим из всех
основателей. Даже если генеральный директор – программист, а продажами
занимается другой основатель? Да. Даже в этом случае, для целей
общения с инвесторами у компании должен быть единственный основатель.
Привести на встречу с инвестором всех основателей нормально, если
инвестор готов предоставить значительную сумму и требует такую встречу
в качестве финального шага при принятии решения. Но дождитесь этого
момента. Относитесь к знакомству инвестора со своими партнерами по
проекту так же, как к знакомству вашей девушки или молодого человека с
родителями – это уместно, только если между вами уже достаточно
серьезные отношения. Даже если во время привлечения финансирования
один или несколько основателей продолжает заниматься развитием
бизнеса, рост замедляется. Но попытайтесь сохранить темпы роста,
насколько это возможно: поиск финансирования происходит не мгновенно.
Он требует времени, и то, как вы вырастете за это время, влияет на
исход финансирования. Если между двумя встречами с инвестором ваши
показатели сильно вырастут, инвестор будет сам стремиться закрыть
сделку, и наоборот – если они останутся без изменений или будут
снижаться, инвесторы начнут охладевать к вам. Вам понадобится краткое
резюме проекта и (возможно) презентация. Как правило, при поиске
финансирования на втором этапе компания встречается с инвесторами и
показывает презентацию. На сайте Sequoia описано, из чего такая
презентация должна состоять. Они платят деньги – им можно поверить на
слово. Я говорю «как правило» – я равнодушен к презентациям, и мне
кажется (хоть, возможно, я излишне оптимистичен), что они выходят из
моды. Многие из самых успешных стартапов на этапе 2 обходятся без
слайдов. Они просто говорят с инвесторами и объясняют им, что
собираются делать. Поскольку успешным стартапам, как правило, удается
быстро привлечь интерес инвесторов, они могут сказать, что просто не
успели подготовить слайды. Вам также понадобится краткое резюме вашего
проекта. Оно должно быть не длиннее одной страницы и должно описывать
сухим языком, что вы планируете сделать, почему это хорошая идея, и
чего вам уже удалось добиться в этом направлении. Цель этого документа
– напомнить инвестору (который мог в течение дня говорить с большим
количеством компаний), о чем вы говорили. Если вы даете кому-то копию
вашей презентации или краткого резюме, осознавайте, что они могут
попасть к самым нежелательным лицам. Но это не должно быть поводом не
давать копии инвесторам, с которыми вы встречаетесь. Подобные утечки
информации – неизбежные издержки ведения бизнеса. На самом деле, вреда
от них не так уж много – хоть предприниматели справедливо возмущаются,
когда их планы попадают к конкурентам, я не знаю ни одной компании,
которой бы это навредило. Иногда инвестор будет просить вас прислать
ему ваши слайды и/или краткое описание перед встречей, чтобы решить,
стоит ли с вами встречаться. Я бы не стал этого делать. Это признак
того, что он не слишком заинтересован. Прекратите поиск
финансирования, если он перестал давать результат. Когда вам стоит
прекратить поиск финансирования? В идеале – когда вы привлекли уже
достаточное количество средств. Но что делать, если вы не привлекли
столько, сколько хотели? Когда следует прекращать? Сложно дать общий
совет на все случаи. Многие стартапы продолжали поиск денег, даже
когда это казалось безнадежным, и им невероятным образом везло. Обычно
я даю следующий совет. Когда вы пьете через трубочку, и через трубочку
начинает идти много воздуха – это признак того, что напиток кончается.
Когда ваши возможности по привлечению финансирования кончаются, это
выглядит примерно так же. Прекратите сосать соломинку, если через нее
идет один воздух. Лучше вам не станет. Избегайте привыкания к процессу
поиска финансирования. Для большинства основателей привлечение
финансирования – лишняя забота, но для некоторых оно оказывается более
интересным, чем сама работа над стартапом. Развитие стартапа в ранней
стадии часто представляет собой грязную, тяжелую и непривлекательную
работу. Привлечение средств, особенно если оно идет успешно,
представляет собой противоположность этому. Вместо того, чтобы сидеть
в обшарпанной квартире и выслушивать жалобы пользователей на ваш
продукт, вы обедаете с известными инвесторами в дорогих ресторанах и
получаете от них предложения на миллионы долларов.[26] Искушение от
этого процесса особенно сильно, если он идет гладко. Всегда приятно
работать над тем, что хорошо получается. Если у вас это хорошо
получается – берегитесь! В отличие от выслушивания жалоб
пользователей, привлечение финансирования не делает компанию успешной.
Опасность привыкания к общению с инвесторами в первую очередь не в
том, что вы потратите на это слишком много времени или привлечете
слишком много денег. Если вы начнете думать, что уже добились успеха,
вы можете потерять вкус к непривлекательной работе, необходимой для
того, чтобы действительно добиться успеха. Это может разрушить
компанию. Когда я вижу, что компания с молодыми основателями
добивается огромного успеха в привлечении финансирования, я для себя
понижаю свою оценку вероятности их успеха. В прессе могут
провозглашать их следующим Google, но про себя я думаю: «это плохо
кончится». Не привлекайте слишком большую сумму. Хоть это касается и
очень маленькой доли стартапов, бывают ситуации, когда денег
привлекают слишком много. У этих ситуаций есть скрытые опасные
последствия. Во-первых, это задает слишком высокие ожидания. Если вы
привлекаете слишком большую сумму денег, скорее всего, ваша оценка
очень велика, и есть риск того, что вы не сможете ее достаточно
повысить при следующем раунде. Рынок ожидает, что стоимость компании
растет с каждым следующим раундом финансирования. Обратное является
признаком проблем с компанией и делает ее непривлекательной для
инвесторов. Если вы привлекаете средства на этапе 2 по оценке в 30
миллионов, оценка вашей компании на следующем раунде (до привлечения
денежных средств) должна быть не менее 50 миллионов. Но для того,
чтобы привлечь средства по такой оценке, дела у компании должны идти
очень, очень хорошо. Не стоит допускать, чтобы конкуренция между
инвесторами на текущем раунде задавала вам цели по успехам, которых
нужно добиться в бизнесе, чтобы привлечь следующий раунд. Эти две вещи
очень слабо связаны. Но сами деньги могут быть еще опаснее, чем
оценка. Чем больше вы привлечете, тем выше будут ваши затраты, а
высокие затраты могут привести к катастрофическим последствиям для
стартапа. Чем больше вы будете тратить, тем сложнее вам будет выйти на
прибыльность, и, что еще хуже, тем менее гибкой будет ваша компания.
Основное, на что в компании могут тратиться деньги – это люди, а чем
больше у вас людей, тем сложнее поменять курс движения компании.
Поэтому, если вы привлекли большую сумму, не тратьте ее. (Вы увидите,
что этого совету будет практически невозможно следовать – настолько
деньги будут жечь вам карман. Но, по крайней мере, попытайтесь.)
Будьте тактичны. Основатели стартапов в поисках финансирования часто
отталкивают инвесторов своим высокомерием. Иногда это происходит
потому, что они действительно высокомерны, а иногда – потому, что они
новички, которые неуклюже пытаются подражать «крутым» опытным
предпринимателям. Заносчивое поведение перед инвесторами – это ошибка.
Хотя в ряде случаев некоторым инвесторам нравится определенное
высокомерие, инвесторы между собой сильно различаются, и от выходки,
которая одному инвестору может понравиться, другой инвестор
разозлится. Единственная надежная стратегия здесь – вообще не вести
себя заносчиво. Это требует определенной дипломатичности, поскольку
советы, которые я давал до этого, вообще говоря, предписывают играть
жестко. Когда вы отказываете инвестору во встрече из-за того, что
сейчас не ищете финансирования, или приостанавливаете переговоры с
инвестором, который движется слишком медленно, или рассматриваете
условное предложение как отказ (которым оно фактически и является) и в
соответствии с «жадным» механизмом принятия предложений отвечаете
отказом, инвесторам это будет не нравиться. Вам надо будет смягчить
удар вежливыми выражениями. В YC мы говорим компаниям ссылаться на
нас. А сейчас, когда я это написал, если хотите, можете валить все на
меня. Если вы еще сошлетесь на свою неопытность, это сработает:
«извините, мы думаем, что вы очень подходящий для нас инвестор, но Пол
Грэм пишет, что стартапы не должны делать того-то и того-то. Мы
привлекаем средства впервые и поэтому не хотим рисковать». Опасность
скатиться в высокомерие сильнее всего, если у вас дела идут хорошо.
Когда все хотят с вами работать, сложно устоять соблазну, особенно
если незадолго до этого никто не хотел с вами иметь дела. Сдерживайте
себя. Мир стартапов тесен, и в бизнесе часто встречаются взлеты и
падения. В этой отрасли, как нигде, погибели предшествует
гордость.[27] Будьте вежливыми и тогда, когда инвесторы вам
отказывают. Хорошие инвесторы не упираются в свое изначальное мнение о
вас. Если они отказали вам на этапе 2 и вы хорошо развиваетесь, они
могут предоставить вам финансирование на этапе 3. Вообще, инвесторы,
которые вам отказали – одни из самых перспективных потенциальных
инвесторов на последующих этапах. Инвестор, который потратил на вас
существенное время, скорее всего, на каком-то этапе был достаточно
близок к согласию. Возможно, у вас в фонде есть свой сторонник,
которому нужно лишь еще немного аргументов, чтобы убедить
сомневающихся коллег. Поэтому будет разумно не только вести себя
вежливо с тем, кто вам отказывает, но и (если только они не вели себя
некрасиво) рассматривать это как начало длительных отношений. В
следующий раз планка будет выше. Предполагайте, что деньги, которые вы
привлекли на этапе 2 – последние, которые вам вообще когда-либо дадут.
Вам нужно выйти на рентабельность с помощью этих денег, если это
вообще возможно. До последних нескольких лет, инвестиционное
сообщество, как правило, выбирало небольшое количество «победителей»
на ранних этапах развития и поддерживало их в последующие годы их
развития. Сейчас ситуация постепенно изменилась. Инвесторы финансируют
большое количество стартапов на ранней стадии и безжалостно
отбраковывают их на последующих этапах. По всей видимости, это
оптимальная стратегия для инвесторов. На ранней стадии выбрать
победителей слишком сложно, и лучше доверить это рынку. Но для
стартапов часто оказывается неприятным сюрпризом то, насколько тяжело
привлечь деньги на этапе 3. Когда вашей компании всего пара месяцев,
все, что от нее нужно – представлять собой интересный эксперимент, на
который можно дать деньги и посмотреть, что получится. Но к следующему
этапу эксперимент должен оказаться успешным. Нужно быть на траектории,
которая ведет к успеху. И хотя доказательства успеха могут быть
основаны, например, на скорости ответов на запросы, как правило,
требуется рентабельность. За некоторыми возможными исключениями,
финансирование на этапе 3 должно привлекаться с позиции прибыльной
компании. Есть два сценария, в которых стартап может «испортиться»
между вторым и третьим этапом. Некоторые стартапы просто слишком
медленно развиваются. Они привлекают достаточно денег, чтобы прожить
два года, и не чувствуют особой срочности в том, чтобы добиться
рентабельности. Поэтому в течение года они не предпринимают усилий,
чтобы заработать деньги. Но на втором году, жить, не зарабатывая
деньги, уже входит в привычку. Когда они, наконец, пытаются выйти на
рентабельность, им это не удается. Другие стартапы позволяют затратам
расти слишком быстро. Практически всегда это означает, что они
нанимают слишком много людей. Не стоит нанимать восемь сотрудников
сразу после того, как вы привлекли деньги на втором этапе. Обычно
стоит подождать, пока у вас будет достаточный рост (и, как правило,
выручка), чтобы это имело смысл. Многие венчурные фонды предлагают
основателям быстро расширять штат. Отчасти они это делают потому, что
склонны – как люди из мира финансов – решать проблемы с помощью денег,
а отчасти потому, что хотят получить большую долю в вашей компании на
последующих раундах. Не слушайте их. Не переусложняйте. Я понимаю, что
этот длинный трактат странно подытоживать рекомендацией не
переусложнять процесс, но если вы вернетесь к началу и взглянете на
него еще раз, вы увидите, что он фактически к сводится к простому
совету с большим количеством следствий и пограничных случаев.
Избегайте общения с инвесторами, пока вам не понадобятся деньги, а
когда они вам понадобятся, общайтесь с ними параллельно, с учетом
ожидаемого выигрыша, и принимайте предложения по принципу «жадного»
алгоритма. Вот вся стратегия привлечения средств в одном предложении.
Не вносите ненужных усложнений и не давайте усложнять процесс
инвесторам. Привлечение средств – не самоцель, чтобы стать успешным.
Это лишь средство к достижению цели. Основной задачей должно быть как
можно быстрее с этим закончить и вернуться к тому, что действительно
приведет вас к успеху – к разработке продуктов и общению с
пользователями. И путь, который я описал, для большинства стартапов
самый надежный для достижения этой цели. Ведите себя хорошо, берегите
себя и не сворачивайте с пути! --- Автор выражает благодарность Славе
Ахмечету, Нейту Блехарчику, Джо Геббиа, Трэвису Дейлу, Мэттану
Гриффелу, Джастину Кану, Биллу Клерико, Джону Коллисону, Патрику
Коллисону, Паркеру Конраду, Рону Конуэю, профессору Мориарти, Никхилу
Нирмелю, Сэму Олтману, Дэвиду Петерсену, Джеффу Ралстону, Джошуа
Ривзу, Юрию Сагалову, Раджату Сури, Гарри Тану, Нику Томарелло,
Джейсону Фридмену, Кевину Хейлу, Джейкобу Хеллеру, Айену Хогарту,
Эммету Ширу и Адоре Чун за вычитку ранних версий этой статьи.



[1] Наихудший исход возникает, когда посредственные на первый взгляд
стартапы контактируют с посредственными инвесторами. Хорошие инвесторы
не водят основателей за нос – они слишком дорожат своей репутацией. А
стартапы, которые выглядят перспективными, как правило, могут получить
достаточно денег от хороших инвесторов, чтобы не общаться с
посредственными. Соответственно, тем стартапам, которые не кажутся
перспективными, приходится привлекать деньги от посредственных
инвесторов. И когда инвесторы колеблются, это им особенно вредит, так
как неперспективные (с виду) стартапы больше нуждаются в деньгах. (Не
все стартапы, которые кажутся посредственными, терпят поражение.
Компания может быть “гадким утенком” в том смысле, что она не
соответствует текущей моде среди стартапов.) [2] Один из участников YC
сказал мне: «Мне кажется, в целом мы нормально провели привлечение
средств, но я два раза совершил одну и ту же ошибку – я пытался
одновременно развивать компанию и привлекать деньги». [3] Здесь есть
скрытая опасность, о которой будет сказано ниже: опасность получить
слишком высокую оценку от оптимистичного инвестора, которая помешает
получить деньги от других инвесторов в дальнейшем. [4] Если им
действительно нужна встреча, значит, они еще не готовы инвестировать –
вне зависимости от того, что они говорят. Они еще не приняли решение,
а это значит, вас просят прийти и убедить их. А это уже работа над
привлечением средств. [5] Инвестиционные менеджеры (англ. associate) в
фондах регулярно пишут письма руководителям компаний, которые с ними
не знакомы. Предприниматели наивно думают: «Вау, нами заинтересовался
венчурный фонд!» Но менеджер – это не то же самое, что фонд. Он не
принимает решений. И хотя он может представить компанию партнерам
фонда, как правило, партнеры не относятся серьезно к проектам, которые
приходят к ним таким образом. Я не знаю ни одной сделки с участием
венчурного фонда, которая началась бы с того, что менеджер написал
письмо в незнакомую с ним компанию. Если вы хотите переговорить с
каким-то определенным фондом, постарайтесь выйти на его партнера через
кого-то, кто пользуется его уважением. Можно поговорить с менеджером,
если вас представили фонду, или если они видели вас на Demo Day и для
начала отправили менеджера узнать про вас побольше. Вряд ли этот
инвестор окажется перспективным, и не следует выделять на него много
ресурсов. Но, по крайней мере, это не совсем бесполезно, в отличие от
контакта по электронной почте. Поскольку у термина «инвестиционный
менеджер» сложилась плохая репутация, многие фонды стали давать своим
менеджерам титул «партнера», что может ввести вас в заблуждение. Если
вы участвуете в YC, вы можете спросить у нас, кто есть кто; если нет –
вам придется покопаться в интернете. У настоящих партнеров должность
может называться по-другому. Настоящим партнером, например, может быть
кто-то, кто выступает от имени фонда в прессе или корпоративном блоге,
либо входит в советы директоров компаний. Между титулами «менеджера» и
«партнера» могут быть и другие ступени, например, «директор»,
«principal» или «венчурный партнер». Смысл этих названий слишком
сильно зависит от компании, чтобы делать какие-либо обобщения. [6] По
тем же причинам старайтесь избегать неформальных бесед с теми, кто
потенциально может приобрести вашу компанию. Это может отвлечь от
бизнеса, что еще опаснее, чем при привлечении средств. Вообще не
соглашайтесь на встречу с возможными покупателями, если только не
собираетесь продавать компанию прямо сейчас. [7] Джошуа Ривз
специально рекомендует просить каждого инвестора представить вас двум
другим инвесторам. Не стоит просить инвесторов, которые вам отказали,
представить вас другим инвесторам. Такая рекомендация может сработать
против вас. [8] Такое поведение не всегда бывает намеренным. Вообще,
несостыковки и непонимание очень часто возникают из-за сложившейся в
венчурном бизнесе практики – которая не в последнюю очередь
сформировалась именно в таком виде потому, что она удовлетворяет
интересам инвесторов. [9] Один из основателей, читавших черновик этого
текста, сказал: «Это самый важный параграф. Я думаю, стоит сказать еще
яснее: «Инвесторы будут намеренно изображать больший интерес, чем у
них есть на самом деле, чтобы сохранить для себя варианты. Даже если
кажется, что инвестор очень сильно в вас заинтересован, скорее всего,
он не проинвестирует. Предполагайте худшее – что инвестор всего лишь
притворяется заинтересованным – пока не получите твердое предложение».
[10] Хоть вам и стоит назначать встречи с инвесторами как можно ближе
друг к другу по времени, Джеф Пён (Jeff Buyn) предлагает довод в
пользу обратного: если график встреч слишком плотный, у вас не
найдется времени обкатать свою речь и аргументы. Некоторые основатели
специально сперва назначают встречи с несколькими малоперспективными
инвесторами, чтобы избавиться от недоработок в своем питче. [11] Не
надо думать, что есть четкая связь между необходимыми усилиями и
ожидаемой выгодой. Самые требовательные инвесторы могут быть и самыми
бесполезными. [12] Это вообще базовый принцип продаж. Если хотите
увидеть его в действии, поговорите с любым продавцом подержанных
машин. [13] Я знаю одного очень харизматичного основателя, который
заканчивал встречи с инвесторами словами: «Ну что, я могу рассчитывать
на ваше участие?» таким же тоном, как он сказал бы «Можете передать
мне соль?» Если вы не такой же опытный и харизматичный (если у вас
есть сомнения – вы таким не являетесь), не пытайтесь повторить это.
Нет ничего более отталкивающего для инвестора, чем зажатый основатель,
пытающийся изобразить красноречивого бизнесмена. Инвесторы не имеют
ничего против того, чтобы финансировать «ботаников». И если вы
являетесь один из них, лучше хорошо сработать в образе «ботаника», чем
неудачно пытаться изобразить харизматичного продавца. [14] Айен Хогарт
предлагает хороший способ понять, насколько серьезен потенциальный
инвестор: посмотрите, сколько ресурсов они на вас тратят после первой
встречи. Если инвестор серьезно заинтересован, он будет пытаться
помочь вам еще до того, как даст твердое предложение. [15] В принципе
вы можете задуматься о так риске называемых «рыночных сигналов». Если
престижный фонд дает вам небольшие посевные инвестиции, вдруг они не
захотят инвестировать в вас на следующем раунде. Другие инвесторы
могут предположить, что раз фонд вас хорошо знает, то что-то с вами не
так. Я говорю «в принципе», потому что в жизни это не такая большая
проблема. Такая ситуация возникает редко, а если возникает, то, как
правило, компания действительно развивается плохо и обречена в любом
случае. Если у вас есть шанс выбирать того, кто дает вам посевные
инвестиции, то можно на всякий случай исключить венчурные фонды. Но
это не критично. [16] Конкуренты могут специально угрожать вам
судебными исками, когда вы приступите к поиску средств, зная, что вам
придется раскрыть этот риск потенциальным инвесторам, и надеясь, что
это помешает вам привлечь деньги. Если это произойдет, скорее всего,
вас это напугает больше, чем инвесторов. Опытные инвесторы знают об
этой хитрости и о том, что такие угрозы редко приводят к настоящим
искам. Поэтому если вы подверглись такой угрозе, будьте с инвесторами
прямолинейны. Они сильнее насторожатся, если будут думать, что вы
что-то скрываете, чем если вы им все расскажете. [17] Близкая к этому
фраза, которую можно услышать от инвесторов – это что они
проинвестируют только при участии других инвесторов, потому что иначе
вы будете «недофинансированы». Как правило, это чушь. Они не могут с
такой точностью оценить вашу минимальную потребность в средствах. [18]
Конечно, вам не понадобится нанимать все 20 человек сразу; также, за
18 месяцев вы скорее всего получите какую-то выручку. Но это все
стандартные, или, как минимум, приемлемые, составляющие допустимой
погрешности. [19] Поиск финансирования по второму сценарию настолько
проще, что стоит пересмотреть свою стратегию, чтобы выйти на
прибыльность раньше. Один основатель сказал, что он бы оставил идеи,
которые требуют больших изначальны инвестиций, предпринимателям с уже
сложившейся репутацией. [20] Я не знаю, почему это происходит – потому
ли, что они не умеют считать, или потому, что считают себя абсолютно
неспособными оценить вероятность успеха компании (в этом случае, по
крайней мере, их поведение не кажется иррациональным). Выводы в любом
случае одни и те же. [21] Если вы участвуете в YC и инвестор по
какой-то причине просит вас непременно самим назвать цену, любой
партнер YC сможет определить рыночную цену для вашей компании. [22] И
наоборот – если инвестор ведет себя порядочно, ответьте ему тем же.
Если инвестор дает вам твердое предложение без ограничения срока на
размышление, вам стоит ответить быстро. [23] Если инвесторы заводят с
вами разговор о раунде A, расскажите им про инвестиции меньшего
масштаба, которые вы привлекаете параллельно. Они должны знать о
последних изменениях в вашей таблице капитализации; также это заставит
их действовать быстрее. Им не понравится, что вы параллельно
привлекаете деньги от других инвесторов, и они будут просить вас
прекратить этот процесс, но вы не должны давать им обязательств, пока
они также не дадут вам твердого предложения. Если они хотят, чтобы вы
прекратили переговоры с другими инвесторами, им следует предоставить
вам терм-шит на финансирование серии А с условием об отказе от поиска
других контрагентов. Можно сделать небольшую уступку, если вы
общаетесь с инвестором с хорошей репутацией, который явно стремится
предоставить вам терм-шит как можно быстрее, особенно есть третья
сторона – например, YC – которая следит, чтобы не было непониманий. Но
будьте осторожны. [24] Эта компания – Weebly. Им удалось выйти на
окупаемость за счет посевных инвестиций в 650 тысяч долларов. Они
пытались привлечь раунд серии А осенью 2008 года, но (естественно,
отчасти потому что это была осень 2008 года) условия, которые им
предложили, были настолько невыгодны, что они решили пропустить раунд
А. [25] Еще одно преимущество от участия во встречах с инвесторами
только одного основателя – это возможность избежать переговоров в
реальном времени, которые были бы невыгодны для неопытных
предпринимателей. Один из участников YC говорил мне: «Инвесторы –
профессиональные переговорщики, и они очень хорошо умеют обсуждать
условия на месте. Если на встрече присутствует только один основатель,
он может сказать: «Я должен посоветоваться со своим партнером», прежде
чем давать какие-либо обещания. Я всегда так и делал». [26] Считайте,
что вам повезло, если процесс идет настолько успешно, что вызывает
привыкание. Как правило, у основателей противоположная проблема – не
пасть духом от того, что инвесторы им отказывают. Как сказал один
участник YC (очень успешный) после прочтения черновика этой статьи:
«Справиться с самим фактом получения от инвесторов отказа очень
тяжело, и если вы настроились неправильно, вы потерпите неудачу. Вы
можете нравиться пользователям, но эти, по идее проницательные,
инвесторы, могут вас совсем не понимать. Я до сих пор мучаюсь, получая
отказы, но я пришел к пониманию, что большая часть инвесторов не
слишком вдается в детали, и чтобы выиграть, надо играть по
определенным достаточно печальным правилам (многие из которых вы
перечисляете)». [27] Книга Притчей Соломоновых, 16:18: «Погибели
предшествует гордость, и падению – надменность».
\end{document}
