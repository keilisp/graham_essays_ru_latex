\documentclass[ebook,12pt,oneside,openany]{memoir}
\usepackage[utf8x]{inputenc} \usepackage[russian]{babel}
\usepackage[papersize={90mm,120mm}, margin=2mm]{geometry}
\sloppy
\usepackage{url} \title{Почему Y Combinator?} \author{Пол Грэм}
\date{}
\begin{document}
\maketitle

Вчера один из основателей, которого мы спонсировали, спросил меня,
почему мы организовали Y Combinator. Или, если быть точнее, не для
удовольствия ли мы создали YC.

Верно, но лишь от части. Это очень интересно работать с Робертом
Моррисом и Тревором снова. Я скучал по этому, с тех пор, как мы
продали Viaweb, и все эти годы после у меня были идеи о том, что мы
могли сделать вместе. Это, безусловно, один из аспектов воссоединения
нашей группы в Y Combinator. Каждые несколько дней я туплю и называю
это Viaweb.

Viaweb мы начали чтобы делать деньги. Мне надоело жить, скитаясь от
одного фрилансерского заказа до другого, и я решил работать так
усердно, насколько хватит сил, пока я не смогу решить свои проблемы
раз и навсегда. Viaweb иногда приносил удовольствие, но он не был
предназначен для этого, да и вообще ни для чего не предназначался. Я
бы удивился, если бы какой-нибудь стартап задумывался ради фана. Все
стартапы, в основном, скучные (schleps).

Реальная причина, почему мы начали Y Combinator, — не эгоизм и не
добродетель. Мы не делали это для зарабатывания денег; у нас не было
никаких идей о том, какова будет отдача от проекта, какова она будет
даже через некоторое время. Также мы не планировали помогать молодежи
становиться сооснователями, хотя нам нравилась эта идея, и
периодически мы утешали себя мыслью, что если все наши инвестиции
накопятся, то мы сможем делать что-либо бескорыстно (это довольно
необъективно).

Настоящей причиной, почему мы начали Y Combinator, вероятно, доступна
к пониманию только хакерам. Мы сделали это, потому что это похоже на
огромный хак. Есть тысячи умных людей, кто мог создать свою компанию,
но не создал; достаточно приложить немного силы, направленной в
определенное место, и мы сможем заполнить мир потоком новых стартапов,
которые при других условиях могли и не появиться.

В некотором роде это добродетель, потому что я думаю, что стартапы это
хорошо. Но что действительно нас мотивирует, так это абсолютно
аморальное желание, которое будет стимулировать любого хакера, который
обнаружил какое-то сложное устройство и понял, что с небольшой
корректировкой он может заставить это работать более эффективно. В
этом случае, устройство — это мировая экономика, которая, к счастью,
оказалась open source.

\end{document}
