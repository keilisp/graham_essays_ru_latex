\documentclass[ebook,12pt,oneside,openany]{memoir}
\usepackage[utf8x]{inputenc} \usepackage[russian]{babel}
\usepackage[papersize={90mm,120mm}, margin=2mm]{geometry}
\sloppy
\usepackage{url} \title{Рефрагментация} \author{Пол Грэм} \date{}

\begin{document}
\maketitle

У старости есть одно преимущество, и заключается оно в том, что все
изменения, происходящие в вашей жизни, становятся заметны. Одним из
таких значительных изменений, что мне довелось наблюдать, является
фрагментация. Направления политической деятельности Соединенных Штатов
гораздо противоречивее, чем раньше. На самом деле, общего между ними
меньше, чем когда-либо. Творческие люди толпами устремляются в
определенные города в поисках счастья, и покидают родные места. А
возрастающее экономическое неравенство влечет за собой увеличения
разрыва между богатыми и бедными. И вот вам моя гипотеза: все эти
тенденции являются по сути проявлением одного и того же. Более того,
проблема не в том, что существует сила, разделяющая нас, а в том, что
существуют некие силы, притягивающие нас друг к другу, и такое
притяжение для нас губительно. \newline

Хуже того, для тех, кто переживает по поводу данных изменений, силы,
которые толкали нас друг к другу, считались аномальными, однажды
случившейся чередой обстоятельств, которые, скорее всего, не удастся
повторить. Да и нам самим не хотелось бы их повторять. \newline

Этими двумя силами были война (в основном вторая мировая) и рост
крупных корпораций. \newline

Последствия второй мировой войны носили как экономический, так и
социальный характер. С точки зрения экономики снизилась разница в
доходах. Как и все современные вооруженные силы, американские военные
структуры, в экономическом плане, играли роль социалиста: от каждого
по возможностям, каждому по потребностям. В той или иной степени.
Высокопоставленные военные деятели получают больше (как и всегда для
членов социалистического общества, занимающих высокие посты), но то,
что им причиталось, было четко регламентировано. Эффект выравнивания
не ограничивался участниками военных действий, потому что экономика
США тоже состояла «на службе». В период с 1942 по 1945 год размеры
всех окладов определялись Национальным советом по труду в военной
промышленности. Как и в случае с военными, было принято решение
сделать оклад равным для всех. И эта атмосфера национальной
стандартизации окладов настолько пропитала собой общество, что ее
воздействие все еще можно ощутить спустя годы после окончания войны.
[1] \newline

Никто даже и не предполагал, что у предпринимателей получится
зарабатывать. Франклин Рузвельт отметил: «политика предотвратит
появление «миллионеров, наживающихся на войне». Для воплощения данного
плана, любое превышение прибыли компании над довоенным уровнем
облагалось налогом в 85\%. А когда то, что оставалось после выплаты
налогов на прибыль компании, доходило до отдельных людей, то снова
облагалось налогом по ставке 93\%. [2] \newline

В социальном плане война должна была снизить колебания зарплат. Более
16 млн. людей из разных мест и различных условий собрали вместе, чтобы
они вели в буквальном смысле унифицированный образ жизни. Норма
обслуживания для мужчин, рожденных в начале 20-х годов, приблизилась к
80\%. А работа во благо общей цели, часто в стрессовых ситуациях, еще
больше сплотила их. \newline

Хоть для США вторая мировая, строго говоря, и длилась менее четырех
лет, ее последствия обладали более длительным эффектом. Войны
позволяют центральному правительству усилить свое влияние, а вторая
мировая довела это до крайности. В США, как и во всех других
странах-союзниках, федеральное правительство с трудом отказывалось от
приобретенных им новых рычагов влияния. Конечно, в некотором отношении
война закончилась не в 1945; понятие «враг» только что переключилось
на Советский Союз. По налоговым ставкам, федеральной власти, оборонным
затратам, призывной кампании, а также уровню национализма, даже
десятилетия спустя после окончания войны ситуация больше напоминала
военный, нежели довоенный, период. [3] Также продолжал действовать и
социальный эффект. Ребенок, которого вытащили прямо из стада мулов в
Западной Вирджинии и засунули в армию, уже не мог просто вернуться на
ферму. Его поджидало кое-что еще, что-то, сильно напоминающее армию. \newline

Если крупным политическим событием XX века была война, то крупным
экономическим событием стало появление компаний нового типа, что также
привело к возникновению как социального, так и экономического
единства. [4] \newline

XX век был веком крупных, национальных корпораций: General Electric,
General Foods, General Motors. События в сфере финансов, коммуникаций,
транспорта и производства дали толчок к появлению нового типа
компаний, чьей целью было, прежде всего, увеличение масштабов
производства. Первую версию такого мира можно было сравнить с Lego
Duplo, т.к. он состоял из очень крупных, массивных блоков, где
существует всего несколько крупных фирм, у каждой из которых приличная
доля в рынке сбыта. [5] \newline

На рубеже XIX и XX веков получил свое развитие процесс консолидации,
основанный, в основном, Джоном Морганом (J. P. Morgan). Тысячи
компаний, возглавляемых своими основателями, были объединены в пару
сотен гигантских структур под управлением профессиональных менеджеров.
Тогда всем заправляла экстенсивная экономика. В то время людям
казалось, что это было конечной стадией развития событий. В 1880 Джон
Рокфеллер объявил, что день объединения настал. Индивидуализм ушел, и
не вернется никогда. Он ошибался, но в течение последующей сотни лет
никто не ставил под сомнение его слова. \newline

Консолидация началась в конце XIX века и продолжалась большую часть ХХ
века. К концу второй мировой, как писал Майкл Линд (Michael Lind),
«главные экономические секторы либо выступали как картели с
правительственной поддержкой, либо контролировались несколькими
корпорациями-олигополистами». \newline

Для потребителей такой новый мир везде предвещал одинаковый выбор, но
только из ограниченного числа товаров. В период моего взросления
существовало всего 2 или 3 набора товаров, и, поскольку все они были
нацелены на рынок среднего уровня, между ними было мало различий. \newline

Хорошо иллюстрирует данную ситуацию то, что творилось на телевидении,
где было только 3 варианта: NBC, CBS, и ABC. Ну и еще общественные
каналы для интеллектуалов и заядлых коммунистов. Программы,
предлагаемые этими тремя каналами, было сложно отличить друг от друга.
На самом деле, давление шло с трех сторон. Если бы одна передача
попыталась показать что-то незаурядное, местные филиалы традиционного
рынка вынудили бы ее создателей прекратить это дело. Плюс ко всему, в
связи с тем, что телевидение было удовольствием не из дешевых, семьи
смотрели одни и те же передачи в полном составе, поэтому они
(передачи) должны подходить всем. \newline

Но информация не только подавалась всем в одинаковом виде, но и в одно
время. Сейчас это сложно представить, но каждую ночь десятки миллионов
семей садились перед телевизором, и смотрели то же самое шоу, в то же
самое время, что и их соседи. То, что сейчас происходит во время
финальных матчей Суперкубка (Super Bowl), тогда происходило каждый
вечер. Мы в буквальном смысле были синхронизированы. [6] \newline

В какой-то степени культура телевидения 50-х годов была неплохой.
Представленный ею взгляд на мир был словно взят из детских книг, и,
вероятно, как надеялись родители, эффект был схож с тем, что
наблюдалось при чтении детских книг, побуждая людей вести себя
примерно. Но, как и в случае с детскими книгами, телевидение было
столь же обманчиво. А для взрослых даже опасно обманчиво. В своей
автобиографии Роберт Макнил (Robert MacNeil) рассказывает о жутких
картинах происходящего во Вьетнаме, и пришел к выводу, что такое
показывать семьям во время обеда нельзя. \newline

Я знаю, насколько пропитывающей была общепринятая культура, потому что
я пытался отстраниться от нее, и было практически невозможно найти
альтернативу. Когда мне было 13, я осознал, больше исходя из
собственных рассуждений, чем из каких-либо внешних источников, что все
те идеи, которыми нас кормят по ТВ, попросту чушь собачья, и я
перестал смотреть телевизор. [7] Но дело было не только в нем.
Казалось, что все вокруг меня было сущей ересью. Все политики говорят
одно и то же, бренды создают практически идентичную продукцию,
налепляя разные этикетки, чтобы выделиться, дома с деревянным балочным
каркасом, являющие собой примеры псевдоколониального архитектурного
стиля, автомобили с метрами бесплатных металлических полос на каждой
стороне, которые начинали отваливаться спустя 2 года, «красные спелые»
яблоки, которые хоть и были красными, но только назывались яблоками.
Оглядываясь в прошлое, все это и было чепухой. [8] \newline

Но когда я продолжил поиски альтернатив, чтобы заполнить пустоту, мне
практически ничего не удалось найти. Тогда не было сети Интернет.
Единственным местом поиска была сеть книжных магазинов в местном
торговом центре. [9] Там я нашел выпуск журнала The Atlantic. Жаль,
что я не могу сказать, что это стало окном в новый мир, а напротив,
оказалось делом скучным и малопонятным. Словно ребенок, впервые
попробовавший виски и делающий вид, будто ему понравилось, я трепетно
хранил этот журнал, словно книгу. Уверен, он до сих пор где-то у меня
лежит. Но, хотя и было очевидно, что где-то существует мир без спелых
красных плодов, мне так и не удалось познакомиться с ним до
поступления в колледж. \newline

Крупные фирмы не только сделали из нас идентичных потребителей, они
превратили нас и в одинаковых работников. Внутри компаний находятся
мощные силы, вынуждающие людей действовать в соответствии с единой
моделью поведения. Особенно IBM была этим известна, но она впадала в
крайности всего чуть-чуть по сравнению с другими крупными «игроками»
на рынке. А модели поведения в разных корпорациях не особо отличались,
подразумевая, что от каждого в этом мире ждут, что он будет более или
менее похож на других. И не только от тех, что связаны с корпоративной
сферой, но также и от всех, кто стремится в нее попасть, что, в
середине ХХ века, включало в основном людей, уже покинувших ее
территорию. Большую часть ХХ века люди рабочего класса упорно
старались походить на средний класс. Это видно на старых фотографиях.
Мало кто в 1950 году старался выглядеть рисково. \newline

Возникновение корпораций национальных масштабов не только оказывают на
нас давление в культурном плане, но и в экономическом, причем с обеих
сторон. \newline

Наряду с гигантскими национальными компаниями, нам достались крупные
национальные профсоюзы. А в середине ХХ века корпорации заключали
сделки с теми профсоюзами, в которых они выплачивали заработные платы
выше рыночных показателей. Частично из-за того, что союзы являлись
монополистами. [10] И частично из-за того, что, являясь частью системы
олигополии, корпорации знали, что можно спокойно перенести затраты на
своих покупателей, т.к. их соперникам пришлось бы сделать то же самое.
А также, отчасти из-за того, что в 50-е годы большинство крупных фирм
все еще были сконцентрированы на поиске новых способов выжать максимум
из экстенсивной экономики. Прямо как в ситуации, когда стартапы
исправно доплачивают AWS сверх стоимости поддержки своих серверов, что
позволяет им сфокусироваться на своем развитии. Многие крупные
государственные корпорации предпочли бы побольше доплачивать за труд.
[11] \newline

Наряду с поднятием уровня доходов с нижней границы из-за переплат
профсоюзам, крупные фирмы ХХ века также снижали доходы в вышестоящих
кругах, недоплачивая своим топ-менеджерам. В 1967 году экономист
Гелбрейт (J. K. Galbraith) писал, что «Всего было несколько компаний,
в которых предложили бы поддерживать зарплаты руководителей на
максимальном уровне». [12] \newline

В некотором роде это была иллюзия. Большая часть фактических выплат
руководству никогда не отражалась в налоговых декларациях о доходах,
т.к. она была представлена в виде льгот. Чем выше ставка подоходного
налога, тем больше вынуждены были платить вышестоящим сотрудникам по
ранее указанной статье расходов. (В Великобритании, где налоги были
еще выше, чем в США, фирмы даже оплачивали обучение детей работников в
частных школах.) Самым ценным, что предоставляли своим подчиненным
крупные фирмы середины ХХ века, была гарантия занятости, что также не
отражалось ни в налоговых декларациях, ни в статистике по доходам.
Таким образом, сама суть найма в этих организациях привела к ошибочно
заниженным показателям экономического неравенства. Но, даже учитывая
все вышесказанное, труд своих лучших работников крупные фирмы
оплачивали ниже рыночного уровня. Рынка, как такового, тогда не было и
подразумевалось, что вы будете десятилетиями, если не всю жизнь,
работать на одну и ту же компанию. [13] \newline

Работа была настолько неликвидна, что шансы получить конкурентную
зарплату были малы. И эта же самая неликвидность подавляла желание ее
искать. Если фирма обещала предоставить вам работу, пока вы не
достигнете пенсионного возраста, и осуществлять пенсионные выплаты
после этого срока, у вас пропал бы стимул извлекать из нее такую же
выгоду в этом году, какую могли бы. Вам нужно было бы заботиться о
компании, чтобы она могла позаботиться о вас. Особенно когда вы
десятилетиями работали с одними и теми же людьми. Если бы вы
попытались выжать из фирмы побольше денег, то вы оказались бы в
ситуации, когда вы вымогаете деньги из фирмы, которая заботится о них.
Более того, если бы вы не поставили фирму на первое место, про
повышение можно было бы забыть. А если у вас не получилось бы сменить
карьеру, то текущая работа была бы единственной возможностью
подняться. [14] \newline

Тем, кто несколько лет провел в вооруженных силах в самый расцвет
формирования своей личности, такая ситуация не показалась бы странной,
как сейчас нам. С точки зрения руководителей крупной фирмы, они
офицеры старшего командного состава. Им платят гораздо больше, чем
рядовым. Они обзавелись счетом обеденных расходов в лучших ресторанах
и летают по миру на самолетах Gulfstream за счет компании. Вероятно,
им даже в голову не пришло спросить, а оплачивается ли их труд по
рыночным расценкам. \newline

Основным способом достижения рыночного уровня оплаты труда является
работа на себя, открытие своей собственной фирмы. Это очевидно для
любого человека с амбициями. Но людям в середине ХХ века была чужда
такая идея. И не потому, что открытие своей фирмы казалось чересчур
амбициозным делом, а потому, что это не казалось достаточно
амбициозным. Даже в 70-х годах, когда я вырос, претенциозный план
включал в себя получение образования в престижных университетах, и
последующую работу в другом престижном учреждении с прохождением всех
ступеней иерархии. Ваш престиж заключался в престиже университета, в
котором вы учились. Конечно, люди открывали свой бизнес, но
образованные люди делали это редко, т.к. тогда практически никто не
имел представления о том, как начать то, что мы сегодня называем
стартапом: бизнес, начинающийся с малого, и вырастающий до больших
объемов. В середине ХХ века это было намного сложнее провернуть.
Открытие своего дела означало начать маленький бизнес, который так и
не наберет обороты. Что, в ту эпоху крупных фирм, зачастую означало
суетливые метания в попытке не быть растоптанным слонами. Престижнее
было входить в число руководящего класса, управляющего этими слонами. \newline

К 1970 году, все продолжали удивляться, откуда вообще взялись эти
крупные престижные фирмы. Казалось, что они существовали всегда, как
химические элементы. И действительно, между честолюбивыми детьми ХХ
века и основателями крупных компаний существовала разница. Многие
крупные фирмы были своего рода «самоделками», у которых не было
определенных основателей. А когда таковые появлялись, то они
отличались от нас. Почти ни у кого из них не было образования в том
смысле, что они не учились в колледжах. Таких людей Шекспир называл
грубыми ремесленниками. Колледж учил быть членом общества
специалистов. И никто не думал, что выпускники начнут выполнять
какую-то грязную черную работу как Эндрю Карнеги или Генри Форд. [15] \newline

А в ХХ веке количество выпускников колледжей становилось все больше и
больше. Их число возросло с примерно 2\% от населения 1900 года до
почти 25\% в 2000. В 50-х годах в ходе слияния двух наших мощных
лагерей стал закон GI Bill, в соответствии с которым 2.2 млн человек,
служивших во время второй мировой войны, отправили учиться в колледж.
Мало кто воспринимал это в таком ключе, но в результате попыток
придать колледжу каноничный образ возникает мир, в котором было вполне
приемлемо работать на Генри Форда, но не быть как он. [16] \newline

Я хорошо это помню, т.к. как раз тогда, когда все это стало
завершаться, я достиг совершеннолетия. Во времена моего детства эти
идеи еще преобладали в умах людей. Но уже не так заметно, как раньше.
На примере старых ТВ шоу и ежегодников, по поведению взрослых можно
понять, что люди 50-х и 60-х были даже более консервативными, чем мы.
Модель 50-х годов уже начала устаревать. Но нам в то время все
виделось в ином свете. Максимум, что мы могли отметить, так это то,
что в 1975 году можно было бы быть чуть смелее, чем в 1965. И
действительно, мало что поменялось. \newline

Но изменения вскоре наступили. И когда экономика Duplo начала
распадаться, то распадалась она несколькими способами одновременно.
Вертикальная иерархия компаний в буквальном смысле распалась на
составные части, т.к. это было более эффективно. Действующие фирмы
столкнулись с новыми конкурентами, по мере того, как (a) рынок стал
глобальным и (б) технические инновации начали предоставлять больше
преимуществ, чем давал эффект масштаба, размер становился уже не
преимуществом, а обременением. Менее крупным фирмам удавалось выжить в
большей степени, т.к. ранее узкие потребительские каналы расширились.
Сами рынки стали меняться быстрее, т.к. возникли совершенно новые
категории товаров. И, наконец, федеральное правительство, ранее
воспринимающее представленный Морганом мир как естественное положение
дел, стало осознавать, что это еще не конец. \newline

Если представить Моргана как ось х, то Генри Форд будет осью y. Он все
хотел делать самостоятельно. На вход гигантского завода, который он
построил на River Rouge в период с 1917 по 1928, в буквальном смысле
поступала железная руда, а на выходе получались автомобили. Там
работало 100 000 людей. В те времена именно так выглядело будущее. Но
автомобильные компании сегодня функционируют совсем иначе. Теперь
огромная часть проектных и производственных процессов происходит
внутри длинной логистической цепи, результаты работ которой
автокомпании в конечном счете собирают и продают. Причина, по которой
автоконцерны действуют именно так, кроется в том, что при таком
сценарии все работает лучше. Каждая фирма в логистической цепи
фокусируется на том, что она знает лучше всего. И каждая из этих фирм
должна хорошо выполнять свою работу, иначе ей просто найдут замену. \newline

Почему Генри Форд не осознал того, что сеть взаимодействующих компаний
работает лучше, чем одна большая фирма? Одной из причин является то,
что для развития сети поставщиков требуется значительное количество
времени. В 1917 Форд считал, что создавать все самостоятельно — это
единственный способ достичь нужного масштаба. Вторая причина
заключалась на том, что, если вам нужно решить проблему посредством
сети взаимодействующих фирм, то вы должны быть в состоянии
координировать их работу, а это лучше всего делать с помощью
компьютеров. Компьютеры снижают транзакционные издержки, являющиеся,
по мнению Коуза, разумным основанием существования корпораций. А это
основополагающие перемены. \newline

В начале ХХ века крупные компании ассоциировались с эффективностью. А
в конце ХХ века – с неэффективностью. В какой-то мере причиной
послужило то, что сами компании утратили гибкость. На это также
повлияло и повышение наших стандартов. \newline

Изменения произошли не только внутри существующих отраслей
промышленности. Сами отрасли изменились. Появилась возможность
создавать кучу новых вещей, и иногда существующие фирмы были не
единственными, у кого это получалось лучше всех. \newline

Продолжение следует (Кто хочет помочь с переводом второй части —
пишите в личку.) \newline

\subsection{Примечания}

[1] Лестер Туроу (Lester Thurow), в своих записях 1975 года, говорил,
что разница в оплате труда, преобладающая в конце второй мировой
войны, была настолько «вмурована» в систему, что «воспринималась как
«обоснованная» даже после того, как уравнительное давление второй
мировой войны прекратилось. В принципе, та же самая разница в оплате
труда сохранилась и по сей день, спустя 30 лет». Но Клаудия Голдин и
Роберт Марго полагают, что рыночные отношения в послевоенный период
также способствовали сохранению тенденции к сокращению заработных
выплат, как и во время военный действий, когда специально повысили
спрос на неквалифицированных и увеличили переизбыток образованных
работников. \newline

(Как ни странно, традиционное в Америке перекладывание ответственности
за оплату медицинского страхования на работников появилось вследствие
попыток предпринимателей обойти контроль заработных выплат
Национальным советом по труду в военной промышленности с целью
привлечения работников.) \newline

[2] Как и всегда, налоговые ставки не раскрывают всей картины.
Существовало множество льгот, особенно для физических лиц. А во время
второй мировой войны налоговый кодекс был в новинку, и мало кто
запрашивал у правительства налоговых послаблений. Если богатые платили
большие налоги во время войны, то, в большей степени, это было из-за
того, что они сами изъявляли желание, а не из-за чувства долга \newline

После войны федеральные налоговые поступления в процентном соотношении
от ВВП держались на том же уровне, что и сегодня. В действительности,
на протяжении всего периода с начала войны объем налоговых поступлений
оставался близким к 18\% от ВВП, несмотря на значительные изменения
налоговых ставок. Наименьшая сумма выплат произошла тогда, когда
ставки подоходного налога были самыми высокими: 14.1\% в 1950 году. С
такими данными сложно не прийти к выводу, что налоговые ставки почти
не влияли на фактические выплаты гражданами. \newline

[3] Хотя, на самом деле, за 10 лет до войны была эпоха неограниченной
власти в ответ на Великую депрессию. И это не просто совпадение, т.к.
Великая депрессия являлась одной из причин войны. Во многих смыслах
«Новый курс» был своего рода генеральной репетицией перед мерами,
которые предприняли федеральные власти в военный период. Хотя, версии
(этих мер) военного периода были гораздо радикальнее и оказали большее
влияние. Как писал Энтони Бэджер, «для многих американцев значительные
изменения в их жизни произошли не во время действия «Нового курса», а
во время второй мировой войны». \newline

[4] Я не обладаю достаточной информацией об основополагающих причинах
мировых войн, но невозможно не заметить, что они связаны с появлением
крупных корпораций. Если это так — значит, для сплоченности ХХ века
есть всего одна причина. \newline

[5] Точнее экономика основывалась на двух вершинах. Выражаясь словами
Гэлбрейта (Galbraith), это был «мир динамично развивающихся
технически, максимально нацеленных на выгоду высокоорганизованных
корпораций, с одной стороны, и сотни тысяч мелких фирм и собственников
в традиционном понимании с другой». Деньги, престиж и власть были
сконцентрированы у первых, и говорить о равных возможностях не
приходилось. \newline

[6] Интересно, насколько впоследствии снизилось количество семей,
обедающих вместе, из-за уменьшения количества семей, смотрящих вместе
телевизор. \newline

[7] Я хорошо это помню, т.к. тогда как раз вышел первый сезон сериала
Dallas. Все обсуждали события сериала, а я не имел ни малейшего
понятия, о чем все вокруг говорили. \newline

[8] Я не осознавал этого, пока не начал проводить исследования для
своего эссе, но безвкусица товаров, с которыми я рос, довольно
известный побочный продукт олигополии. Если фирмы не могут
конкурировать по цене, то конкуренция основывается на мелких деталях. \newline

[9] Торговый центр Monroeville Mall по завершению строительства в 1969
был самым крупным в стране. В конце 1970-х годов там снимали фильм
«Рассвет мертвецов» (Dawn of the Dead). Конечно, магазин был не только
съемочной площадкой, но и вдохновил на написание сценария к фильму,
т.к. толпы покупателей, разгуливающих по огромному зданию центра,
напомнали режиссеру Джорджу Ромеро зомби. А я на своей первой работе
стоял на раздаче мороженого в «Баскин Роббинс». \newline

Классический пример — микрокомпьтеры. Пионерами в данной области стали
фирмы на подобие Apple. Когда она стала достаточно крупной, IBM
решила, что стоит обратить внимание на эту сферу. В то время IBM
полностью контролировала компьютерную отрасль. Они предположили, что
теперь, когда рынок созрел, все что нужно сделать, это просто
потянуться и сорвать его. На тот момент большинство согласилось бы с
ними. Но то, что произошло потом, показало, насколько мир усложнился.
IBM выпустила-таки свой микрокомпьютер. Хотя он и был довольно
успешен, он не стал разгромом для Apple. И более того, саму IBM, в
итоге, вытеснил поставщик из другой области — из области ПО, которая
даже не воспринималась как нечто близкое данному бизнесу. Для IBM было
большой ошибкой принять неисключительные права на DOS. Должно быть, в
тот период, такой маневр казался безопасным. Никакому другому
производителю компьютеров не удалось продать больше. Что изменилось
бы, если бы другие производители также могли предлагать DOS? Этот
просчет стал причиной бурного роста недорогих персональных
компьютеров. Теперь PC стандартами заправляет Microsoft, ровно как и
клиентской базой, а сфера микрокомпьютеров превратилась в
противостояние Apple и Microsoft. \newline

Сначала Apple подсидела IBM, а потом Microsoft украла ее
бумажник. В 50-х годах такого бы с крупными фирмами не произошло. Но в
будущем такое будет происходить все чаще. \newline

В основном, в сфере компьютеров перемены происходили сами по себе. В
других областях сначала нужно было избавиться от препятствий в виде
законодательной базы. В 50-х годах федеральное правительство одарило
многих олигополистов соответствующими политическими курсами (а в
военное время и крупными заказами), что не позволяло конкуренции
развиться. В то время государственным должностным лицам это не
показалось столь сомнительным делом, в отличие от нас. Считалось, что
двухпартийная система обеспечивала достаточный уровень конкуренции в
сфере политики, что должно было сработать и для бизнеса. \newline

Постепенно правительство осознало, что курсы, направленные против
конкуренции, приносили больше вреда, чем пользы, и во время нахождения
Картера на посту президента их стали убирать. Слово, выбранное для
описания данного процесса, несло обманчиво узкий смысл:
дерегулирование. Но то, что происходило на самом деле, называлось
деолигополизацией, происходящей в одной сфере за другой. Заметнее
всего это отразилось на двух услугах: на авиа перевозках и на
международных телефонных разговорах. Обе эти услуги существенно
подешевели после дерегулирования. \newline

В 80-х годах дерегулирование также привело к волне враждебных
поглощений. В былые дни единственным пределом неэффективности
компании, помимо банкротства, считалась неэффективность ее
конкурентов. Теперь же фирмам пришлось столкнуться со стандартами
абсолютными, а не относительными. Для любой недостаточно рентабельной
государственной фирмы существовал риск лишиться своего текущего
руководства и обзавестись новым, которое эту рентабельность бы
повысило. Зачастую, новое руководство достигало этого посредством
разделения компании на части, каждая из которых сама по себе обладала
большей ценностью. [17] \newline

Первая версия национальной экономики состояла из нескольких крупных
блоков, чье взаимодействие оговаривались в секретных кабинетах кругом
лиц, занимающих руководящие посты, политиками, регуляторами, и
представителями профсоюза. Версия 2 была более «детализированной»:
число фирм, специализирующихся на разных товарах и услугах, возросло,
их размеры разнились, а их взаимоотношения изменялись еще быстрее. В
этом мире множество решений все еще принимались в ходе переговоров в
тайных кабинетах, но многое было представлено на суд рынка. И это, в
дальнейшем, ускорило фрагментацию. \newline

Может показаться, что упоминание версий в ходе описания непрерывного
процесса немного вводит в заблуждение, но это не совсем так. За
несколько десятилетий произошло много изменений, и то, к чему мы в
итоге пришли, качественно отличалось от предыдущих этапов. В 1958 году
фирмы из списка S\&P 500 находились в нем в среднем 61 год. К 2012
этот период составил 18 лет. [18] \newline

Распад экономики Duplo произошел одновременно с увеличением
вычислительных мощностей. Насколько значительную роль сыграли в этой
истории компьютеры? Чтобы ответить на этот вопрос, пришлось бы
написать целую книгу. Конечно, повышение вычислительной мощности стало
благодатной почвой для стартапов. Подозреваю, что и финансовая сфера
внесла свою лепту. А было ли это причиной глобализации или волны LBO
(Leveraged buyout) сделок? Я не знаю, но не исключал бы такую
возможность. Вероятно, стимулом для рефрагментации послужили
компьютеры, также, как паровые двигатели для промышленной революции.
Были компьютеры предпосылкой или нет, но они точно ускорили дело. \newline

Новое плавное развитие фирм изменило взаимоотношения с сотрудниками.
Зачем подниматься по карьерной лестнице, которую в любой момент могут
из-под тебя выбить? Люди с амбициями начали представлять карьеру не
как подъем по одной лестнице, а как последовательность должностей,
возможно, даже занимаемых в различных компаниях. Все большее
количество переходов (или даже потенциальных переходов) из одной фирмы
в другую повлекли за собой большую конкуренцию в зарплате. Плюс ко
всему, с уменьшением размеров компаний стало проще оценивать влияние
работника на доходы фирмы. Эти изменения подняли зарплаты до уровня
рыночной цены. И, поскольку продуктивность людей существенно
различается, оплата труда по рыночной цене привела к разнице в
окладах. \newline

Неслучайно в начале 80-х был придуман термин «яппи». Это слово не так
широко используется сейчас, потому что явление, которое оно описывает,
для нашего времени вполне привычное, но тогда оно обозначало нечто
новаторское. Яппи называли молодых специалистов, которые много
зарабатывали. Нынешним двадцатилетним людям это не покажется чем-то
выдающимся, заслуживающим отдельного определения. Почему бы молодым
специалистам и не зарабатывать крупные суммы? Но до 80-х годов
недоплачивать на ранних этапах развития карьеры считалось
непосредственной частью на пути к тому, что подразумевалось под словом
«профессионализм». Молодые специалисты оплачивали сборы по ходу
построения своей карьеры. А вознаграждение придет потом. Что в яппи
было новаторским, так это то, что они захотели, чтобы их нынешнюю
работу оценивали по рыночной цене. \newline

Первые яппи в стартапах не работали, т.к. время для стартапов еще не
пришло. Но их также нельзя было встретить и в крупных фирмах. Это были
специалисты из таких областей как юриспруденция, финансы и консалтинг.
Такой пример мгновенно вдохновил их сверстников. Как только они видели
новую BMW 325i, им тут же хотелось такую же. \newline

Фишка с недоплачиваем новичкам работает только в том случае, если так
поступают все. Стоит только одному работодателю нарушить этот порядок,
как все остальные вынуждены будут сделать то же самое, иначе хороших
работников им не заполучить. И как только процесс запустится, он
распространится по всей экономической системе, т.к. в начале своей
карьеры люди не только с легкостью меняют работодателей, но и целые
отрасли. \newline

Но не все молодые специалисты находились в выгодном положении. Чтобы
вам много платили, нужно вкладываться. Неслучайно первые яппи работали
в тех областях, где было легко этот вклад измерить. \newline

В общем, идея была аналогична ситуации с именами (считавшимся
устаревшими только потому, что в течение долгого времени были
редкостью): можно было разбогатеть. А в прошлом существовало множество
способов это сделать. Некоторые разбогатели, формируя свое состояние,
а другие, участвуя в антагонистических играх. Но как только такая
возможность появилась, людям с амбициями приходилось решать, стоит это
делать или нет. В 1990 году физик, предпочитающий науку Уолл-стрит,
жертвовал всем, в отличие от физика 1960 года. \newline

Эта же идея перетекла обратно в крупные фирмы. Генеральные директоры
которых зарабатывают сейчас больше, чем раньше, и, как мне кажется,
все дело в престиже. В 1960, генеральные директоры корпораций обладали
огромным авторитетом. Они занимали самое выгодное положение в
экономической системе. Но если бы они зарабатывали такие же гроши и
сейчас, в реальном долларовом эквиваленте, они были бы мелкими сошками
в сравнении с профессиональными спортсменами и «вундеркиндами»,
зарабатывающими миллионы на стартапах и хеджевых фондах. Такая идея им
не по нраву, поэтому сейчас они пытаются извлечь как можно больше, что
и так превышает их доходы в прошлом. [19] \newline

Межде тем, аналогичная фрагментация происходила и на другом чаше весов
экономики. Поскольку олигополии крупных фирм стали менее защищенными,
возможностей для перекладывания затрат на плечи клиентов стало меньше,
и, следовательно, меньше желания переплачивать за труд. А как только
мир Duplo из нескольких крупных блоков распался на множество фирм с
разным количеством работников, а некоторые из них числились и за
границей, профсоюзам стало сложнее навязывать свою монополию. В итоге,
оклады работников поползли к рыночной стоимости, которая (и это
неизбежно, даже если бы проссоюзы выполняли свою работу) уменьшалась,
и, вероятнее всего, значительно, если бы из-за автоматизации упал
спрос на некоторые виды работ. \newline

И коль скоро модель 50-ых годов вызвала как социальное, так и
экономическое единство, ее разрушение повлекло за собой и социальную,
и экономическую фрагментацию. Люди стали действовать и выглядеть
иначе. Те, кого позже назвали бы «творческим классом», стали
мобильнее. Те, кому до религии не было дела, испытывали меньше
давления из-за того, что они посещали церковь для вида, в то время,
как другие, кому это очень нравилось, предпочитали в большей степени
яркие церемонии и обряды. Некоторые переключились с мясных рулетов на
тофу, а другие – на Hot Pockets (прим. переводчика: под таким
названием выпускался продукт, напоминающий слойки с начинкой из сыра,
мяса или овощей, предназначенный для разогрева в микроволновой печи).
Кто-то перешел с Ford седан на небольшие импортные автомобили, а
остальные – на автомобили типа SUV (прим. переводчика:
спортивно-утилитарный автомобиль, своего рода легкий грузовик для
повседневной эксплуатации). Дети из частных школ, а также те, которые
в таких школах хотели бы учиться, стали одеваться как учащиеся из
элитной частной школы, а дети, которым хотелось казаться бунтарями,
осознанно старались выглядеть дискредитирующе. Люди разделились на
группы по сотням показателей. [20] \newline

Спустя почти 10 лет, фрагментация все еще продолжается. Была ли она в
целом хорошей или плохой? Я не знаю; вопрос риторический. Хотя есть в
ней и плюсы. Мы воспринимаем формы фрагментации, которые нам по нраву,
как само собой разумеещееся, и волнуемся только, когда нас что-то не
устраивает. Но, как человек, которому довелось пережить конформизм
50-х годов, могу сказать, что это не было утопией. [21] \newline

Я не ставил своей целью определить последствия фрагментации как
хорошие или плохие. Мне просто хотелось объяснить, почему она
происходит. Теперь, когда центростремительные силы мировой войны и
олигополии ХХ века почти сошли на нет, что будет дальше? А в именно,
возможно ли повернуть вспять некоторые изменения, возникшие из-за
фрагментации? \newline

Если да, то это должно происходить поэтапно. Невозможно воссоздать
единство 50-х годов таким же, каким оно было изначально. Было бы
безумием развязать войну просто ради стимулирования национального
единства. И как только вы осознаете степень «блочности», в которой
пребывала история экономики ХХ века версии 1, станет ясно: это
воспроизведению не поддается. \newline

По крайней мере, единство ХХ века возникло естественным путем. Война
началась, в основном, из-за внешних сил, а экономика Duplo была фазой
эволюции. Если вам нужно единство сейчас, то пришлось бы ее вызвать
намеренно. И несовсем очевидно, каким именно образом. Полагаю,
единственное, что нам удастся сделать, это среагировать на симптомы
фрагментации. И уже этого может быть достаточно. \newline

Форма фрагментации, которая в последнее время, в основном, волнует
людей, называется экономическим неравенством, и если у вас возникнет
желание его устранить, то придется столкнуться с поистине мощным
препятствием, которое существует еще с каменного века: с техникой.
Техника – это рычаг, увеличивающий эффект от работы. И этот рычаг не
только удлинняется, но и сам темп, с которым это происходит,
ускоряется. \newline

Что, в свою очередь, означает, что разница в объемах состояний,
которые можно накопить, не только увеличивается, но и ускоряет свой
темп. Необычные условия, преобладающие в середине ХХ века, скрыли эту
нижележащую тенденцию. У честолюбивых личностей нет особого выбора,
кроме как вступить в ряды крупных организаций, что заставило бы их
маршировать наряду с другими людьми — как в случае с вооруженными
силами в буквальном смысле, или, если выражаться образно, как в случае
с крупными корпорациями. Даже если бы большие фирмы изъявили желание
оплачивать труд людей пропорционально его ценности, им бы не удалось
определить, как именно это сделать. Но такого давления уже нет. Как
только в 70-е годы эта идея стала ослабевать, потаенные силы снова
дали о себе знать. [22] \newline

Конечно, не все, кому удалось разбогатеть, сделали это посредством
формирования своего состояния. Но это верно для значительного числа
людей, а согласно эффекту Баумоля, все их современники также будут
втянуты в данный процесс. [23] И пока есть возможность разбогатеть
через формирование своего капитала, экономическое неравенство будет
только расти. Даже если вы закроете все прочие пути обогащения. Можно
сгладить ситуацию через субсидии с одной стороны, и налогами с другой,
но пока размеры налогов не препятствуют формированию своего состояния,
вы просто всегда будете отчаянно бороться с возрастающей разницей в
производительности. [24] \newline

Вот такая форма фрагментации, как и прочие, сохранится. А точнее, она
снова нас настигла. Ничто не вечно, но тенденция к фрагментации должна
быть более долгоживущая, чем большинство вещей в этом мире, именно
потому, что для нее нет какой-то особенной причины. Это просто возврат
к среднему значению. Когда Рокфеллер объявил об исчезновении
индивидуализма, он был прав в течении сотни лет. И вот мы снова к
этому пришли, и таковой будет наша реальность на протяжении более
длительного срока. \newline

Боюсь, если мы не признаем этого, то проблем потом не оберемся. Если
вы считаете, что единство ХХ века исчезло из-за нескольких
политических поправок, вы будете ошибочно полагать, что его можно
вернуть (каким-то образом, без недостатков, ему присущих) после
нескольких контрпоправок. А затем, потратим уйму времени на устранение
фрагментации, в то время как было бы лучше, если бы мы подумали, как
смягчить ее последствия. \newline

\subsection{Примечания}

[1] Лестер Туроу (Lester Thurow), в своих записях 1975 года, говорил,
что разница в оплате труда, преобладающая в конце второй мировой
войны, была настолько «вмурована» в систему, что «воспринималась как
«обоснованная» даже после того, как уравнительное давление второй
мировой войны прекратилось. В принципе, та же самая разница в оплате
труда сохранилась и по сей день, спустя 30 лет». Но Клаудия Голдин и
Роберт Марго полагают, что рыночные отношения в послевоенный период
также способствовали сохранению тенденции к сокращению заработных
выплат, как и во время военный действий, когда специально повысили
спрос на неквалифицированных и увеличили переизбыток образованных
работников. \newline

(Как ни странно, традиционное в Америке перекладывание ответственности
за оплату медицинского страхования на работников появилось вследствие
попыток предпринимателей обойти контроль заработных выплат
Национальным советом по труду в военной промышленности с целью
привлечения работников.) \newline

[2] Как и всегда, налоговые ставки не раскрывают всей картины.
Существовало множество льгот, особенно для физических лиц. А во время
второй мировой войны налоговый кодекс был в новинку, и мало кто
запрашивал у правительства налоговых послаблений. Если богатые платили
большие налоги во время войны, то, в большей степени, это было из-за
того, что они сами изъявляли желание, а не из-за чувства долга \newline

После войны федеральные налоговые поступления в процентном соотношении
от ВВП держались на том же уровне, что и сегодня. В действительности,
на протяжении всего периода с начала войны объем налоговых поступлений
оставался близким к 18\% от ВВП, несмотря на значительные изменения
налоговых ставок. Наименьшая сумма выплат произошла тогда, когда
ставки подоходного налога были самыми высокими: 14.1\% в 1950 году. С
такими данными сложно не прийти к выводу, что налоговые ставки почти
не влияли на фактические выплаты гражданами. \newline

[3] Хотя, на самом деле, за 10 лет до войны была эпоха неограниченной
власти в ответ на Великую депрессию. И это не просто совпадение, т.к.
Великая депрессия являлась одной из причин войны. Во многих смыслах
«Новый курс» был своего рода генеральной репетицией перед мерами,
которые предприняли федеральные власти в военный период. Хотя, версии
(этих мер) военного периода были гораздо радикальнее и оказали большее
влияние. Как писал Энтони Бэджер, «для многих американцев значительные
изменения в их жизни произошли не во время действия «Нового курса», а
во время второй мировой войны». \newline

[4] Я не обладаю достаточной информацией об основополагающих причинах
мировых войн, но невозможно не заметить, что они связаны с появлением
крупных корпораций. Если это так — значит, для сплоченности ХХ века
есть всего одна причина. \newline

[5] Точнее экономика основывалась на двух вершинах. Выражаясь словами
Гэлбрейта (Galbraith), это был «мир динамично развивающихся
технически, максимально нацеленных на выгоду высокоорганизованных
корпораций, с одной стороны, и сотни тысяч мелких фирм и собственников
в традиционном понимании с другой». Деньги, престиж и власть были
сконцентрированы у первых, и говорить о равных возможностях не
приходилось. \newline

[6] Интересно, насколько впоследствии снизилось количество семей,
обедающих вместе, из-за уменьшения количества семей, смотрящих вместе
телевизор. \newline

[7] Я хорошо это помню, т.к. тогда как раз вышел первый сезон сериала
Dallas. Все обсуждали события сериала, а я не имел ни малейшего
понятия, о чем все вокруг говорили. \newline

[8] Я не осознавал этого, пока не начал проводить исследования для
своего эссе, но безвкусица товаров, с которыми я рос, довольно
известный побочный продукт олигополии. Если фирмы не могут
конкурировать по цене, то конкуренция основывается на мелких деталях. \newline

[9] Торговый центр Monroeville Mall по завершению строительства в 1969
был самым крупным в стране. В конце 1970-х годов там снимали фильм
«Рассвет мертвецов» (Dawn of the Dead). Конечно, магазин был не только
съемочной площадкой, но и вдохновил на написание сценария к фильму,
т.к. толпы покупателей, разгуливающих по огромному зданию центра,
напомнали режиссеру Джорджу Ромеро зомби. А я на своей первой работе
стоял на раздаче мороженого в «Баскин Роббинс». \newline

[10] В 1914 году антитрестовский закон Клейтона постановил, что
профсоюзы не попадают в рамки антимонопольных законов по причине того,
что человеческий труд не является «предметом потребления или
торговли». Интересно, означает ли это, что компании, занимающиеся
предоставлением услуг, также не входят в эту категорию. \newline

[11] Отношения между профсоюзами и состоящими в них компаниями могут
даже носить симбиотический характер, т.к. профсоюзы способны оказывать
политическое давление для защиты своих участников. По словам Майкла
Линда, когда политики попытались притеснить сеть супермаркетов A\&P за
то, что те не давали местным продуктовым магазинам возможности
развернуться, «A\&P успешно защитила свою позицию, позволив своим
работаникам в 1938 году вступить в профсоюз, таким образом заручившись
его поддержкой». Я и сам наблюдал такое: объединения отелей несут
ответственность за значительную часть политического давления на сервис
Airbnb в большей степени, чем сами отели. \newline

[12] Гэлбрейт был явно озадачен тем, что члены правления так усердно
работали на благо других (акционеров). Большая часть его работы «Новое
индустриальное общество» (The New Industrial State) была посвещена
разбору данного вопроса. \newline

Его теория основывалась на том, что профессионализм заменил деньги как
основной источник мотивации, и что современные руководители корпораций
были, как и (хорошие) ученые, менее мотивированы финансовым аспектом,
и больше вдохновлялись желанием выполнять хорошую работу и, таким
образом, заслужить уважение своих коллег. В этом что-то есть, хотя, я
полагаю, что отсутствие кочевания кадров из одной фирмы в другую, в
совокупности с личными интересами, многое объясняет в таком поведении. \newline

[13] Гэлбрейт (стр. 94) упомянул исследование 1952 года, которое
выявило, что три четверти из 800 наиболее высокооплачиваемых
руководителей 300 крупных фирм проработали в этих фирмах более 20 лет. \newline

[14] Вероятно, в первой трети ХХ века у руководителей были низкие
зарплаты частично из-за того, что компании того времени больше
зависели от банков, которым бы не понравилась ситуация, когда
работники, занимающие руководствующие должности, получали бы слишком
много. Конечно, в начале так все и было. Первые генеральные директоры
крупных фирм были наняты Джоном Морганом. \newline

До 1920 года компании не могли обеспечить себя с помощью
нераспределенной прибыли. До этого им приходилось тратить свой доход
на выплату дивидендов, и, таким образом, они зависели от банков и
продолжали с ними сотрудничать в обмен на предоставление капитала для
расширения. Банкиры продолжали присутствовать в совете корпораций до
принятия закона Гласса-Стиголла в 1933 году. \newline

К середине века доходы крупных фирм субсидировали 3/4 своего роста. Но
первые годы банковской зависимости, подкрепляемой финансовым контролем
второй мировой войны, наверняка оказали значительное влияние на
социальные нормы относительно зарплат руководителей. Поэтому,
возможно, отсутствие текучки кадров было как следствием низких
зарплат, так и их причиной. \newline

Между прочим, переход к финансированию роста средствами из
нераспределенной прибыли в 20-е годы был одной из причин биржевого
краха 1929 года. Теперь банкам приходится искать еще кого-то, кому
можно дать взаймы, и поэтому они набрали еще больше маржинальных
кредитов. \newline

[15] Даже сейчас сложно их заставить. Я считаю, что потенциальным
основателям стартапов сложнее всего понять, насколько важно выполнять
черновую работу определенного рода на ранних этапах развития компании.
Работать над тем, что не масштабируется, в сравнении с началом
карьерного пути Генри Форда все равно, что сравнивать диету с высоким
содержанием клетчатки с принципами питания типичного сельского жителя:
у них не было выбора, кроме как делать то, что правильно, в то время
как нам приходится прикладывать усилия и делать выбор осознанно. \newline

[16] Во времена моего детства основателей не особо чествовали в
прессе. Под словами «наш основатель» представлялась фотография мужчины
сурового вида с усами, как у моржа, и с накрахмаленным воротничком,
распечатанная спустя десять лет после его смерти. В детстве хотелось
быть руководителем. Если вы это время не застали, тогда вам будет
сложно уловить особенности данного термина. Все, что нам хотелось
видеть вокруг, попадало под «управленческую» модель. \newline

[17] Волна враждебных поглощений в 80-х годах образовалась из-за ряда
обстоятельств: судебные решения, подрывающие государственные нормы
регулирования поглощений, начиная с вердикта Верховного суда 1982 года
в деле Джеймса Эдгара против MITE Corp.; относительно доброжелательное
отношение администрации президента Рейгана к поглощениям; закон о
депозитарных учреждениях 1982 года, который разрешал банкам и
сберегательным институтам покупать корпоративные облигации; новое
правило комиссии по ценным бумагам 1982 года (правило 415), которое
позволило быстрее доводить корпоративные облигации до рынка;
организация Майклом Милкеном рынка мусорных облигаций; мода на
конгломераты, прошедшая в предыдущем периоде, и ставшая причиной
объединения многих фирм, чего делать было нельзя; десятилетняя
инфляция, которая привела к тому, что стоимость многих открытых
акционерных компаний была ниже стоимости их активов; и, что не менее
важно, рост самодовольства менеджмента. \newline

[18] Foster, Richard. «Creative Destruction Whips through Corporate
America.» Innosight, февраль 2012. \newline

[19] Возможно, генеральным директорам крупных фирм и переплачивают. У
меня недостаточно соответствующих сведений относительно больших
компаний. Но могу точно сказать, что генеральный директор вполне может
оказывать такое влияние на доходы компании, степень которого в 200 раз
больше по сравнению со среднестатистическим работником. Посмотрите,
что сделал Стив Джобс для Apple, когда вернулся в качестве
генерального директора. Для совета директоров отдать ему 95\% акций
фирмы было бы выгодной сделкой. В день, когда вернулся Стив (в июле
1997 года), рыночная капитализация Apple составила 1.73 миллиардов.
Сейчас (январь 2016) 5\% акций Apple стоили бы около 30 миллиардов. Но
этого бы не случилось, если бы Стив не вернулся; Apple, вероятно,
вообще бы уже не существовала. \newline

Уже сам пример со Стивом может быть достаточным для ответа на вопрос,
переплачивают ли в совокупности генеральным директорам открытых
акционерных компаний. И это не обманчивый маневр, как может
показаться, потому что, вас волнует только расширение вашей доли
собственности, с параллельным увеличением общей суммы. \newline

[20] Конец 60-х известен своим социальным подъемом. Но это больше
напоминало бунт (что может произойти в любой эпохе, если довести людей
до определенного состояния), нежели фрагментацию. Она не заметна, пока
вы не видите, как людей разрывает в разные стороны. \newline

[21] В мировом масштабе тенденция пошла в другом направлении. В то
время, как США становится более фрагментированными, мир, в целом,
менее фрагментированным. И в большинстве случаев в хорошем смысле
этого слова. \newline

[22] Тогда была куча способов разбогатеть, и самый основной заключался
в бурении нефтяных скважин, что для новичков было сферой открытой,
т.к. это не та область, где крупные фирмы могли бы доминировать через
экстенсивную экономику. Как отдельно взятым личностям удалось накопить
большое состояние в эпоху таких высоких налогов? Все дело в лазейках
налогообложения, которые защищались двумя наиболее влиятельными людьми
в Конгрессе США: Сэмом Рейберном и Линдоном Джонсоном. \newline

Стать техаским нефтепромышленником – это не то, к чему в 1950 году мог
кто-то стремиться, чего не скажешь о перспективах создания стартапа
или работы на Уолл Стрит в 2000 году, потому что (a) там
присутствовала мощная местная составляющая, и (б) успех сильно зависел
от удачи. \newline

[23] Эффект Баумоля, вызванный стартапами, очень заметен в Кремниевой
долине. Google готов заплатить миллионы долларов в год, чтобы
предотвратить уход сотрудников в стартап. \newline

[24] Я не утверждаю, что разница в производительности является
единственной причиной экономического неравенства в США. Но ее вклад
существенен, и значимость ее возрастет до такой степени, что если вы
заблокируете другие пути сформировать

\end{document}
