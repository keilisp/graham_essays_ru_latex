\documentclass[ebook,12pt,oneside,openany]{memoir}
\usepackage[utf8x]{inputenc} \usepackage[russian]{babel}
\usepackage[papersize={90mm,120mm}, margin=2mm]{geometry}
\sloppy
\usepackage{url} \title{Что мы ищем в стартаперах и молодых
  предпринимателях?} \author{Пол Грэм} \date{}
\begin{document}
\maketitle

1. Определенность

Это оказалось самым главным качеством. Когда мы основывали Y
Combinator, мы считали, что самое главное качество — ум. Это миф
Силиконовой долины. Конечно, никто не хочет глупых людей в команду. Но
намного более важное качество — определенность. У вашей команды будет
много препятствий. Не думаю, что вы хотите иметь рядом человека,
который будет сдаваться при каждой неудаче.

Билл Клерико и Рич Абермен из «WePay» — хороший тому пример. У них
финансовый стартап, а это означает бесконечные переговоры с крупными,
бюрократическими компаниями. Когда Вы начинаете стартап, который
зависит от соглашений с крупными компаниями, складывается чувство, что
они игнорируют вашу компанию. Но когда Билл Клерико звонит вам с
просьбой, вы не откажете. На него можно положиться, он не подставит
Вас.

2. Гибкость

Не думаю, что Вы хотите иметь в команде человека, для которого
определенность базируется на фразе «не разочаровывайся в своих
мечтах». Мир стартапов — непредсказуем, вы должны уметь управлять
своими мечтами и на старте, и в полете. Самая лучшая метафора, которая
характеризует определение и гибкость — постоянное движение. Люди в
команде должны иметь достаточно решимости идти вперед, предусматривая
повороты и любые отхождения. А иногда и шаги назад, поиск нового пути.

Можно посмотреть это качество на примере Дэниела Гросса из «Greplin».
Он обратился к нам с некоторой плохой идеей электронной коммерции. Мы
сказали ему, что финансировали бы его, если бы он сделал что-то
другое. Он подумал, и в течение секунды, и сказал: хорошо. Он обдумал
еще две идеи, прежде, чем остановиться на «Greplin». Он работал над
ним в течение нескольких дней, после представил их инвесторам. Он
получил большой капитал. Важно быть кошкой, которая в любой ситуации
приземляется на 4 лапы.

3. Воображение

Ум, бесспорно, важен. Но, оказывается, воображение играет еще более
важную роль. Не так важно уметь справляться с проблемами быстро, как
искать новые идеи для реализации. Множество идей в стартап мире в
начале принимались не очень хорошо. Ведь, как считали люди, если бы
они были достаточно хороши, кто-то бы уже придумал их. Поэтому нужно
иметь ум, который рождает идеи с определенной степенью сумасшествия.

«Airbnb» — именно такой случай. Фактически, когда мы финансировали
«Airbnb», мы думали, что это слишком сумасшедшая идея. Мы не верили,
что такое количество людей согласилось бы оставаться у других людей.
Мы инвестировали в их компанию во многом потому, что нам понравились
основатели. И, как оказалось, выиграли.

4. Непослушание

Хотя самые успешные основатели — обычно хорошие люди, они имеют что-то
дерзкое в характере. Их нельзя считать «хорошими парнями». Они скорее
правильно решают вопросы, а не слушают свои приоритеты. Вот почему я
использовал бы слово непослушание, а не зло. Им нравится нарушать
правила, но не фундаментальные. Иногда это качество не обязательно,
часто оно может сочетаться или замещаться воображением.

Сэм Олтмен из «Loopt» — один из наших самых успешных выпускников, мы
спросили у него, что следует задавать людям, чтобы найти как можно
больше таких, как он. Он сказал, что следует спросить о том времени,
когда они были взламывали для себя: т.е. не компьютеры, а систему. Это
стало одним из основополагающих вопросов на интервью.

5. Дружелюбность

В одиночку очень сложно основать стартап. Большинство организаций
имеют 2-3 кофандера. И связь между ними должна быть действительно
крепкой. Они должны нравится друг другу, как хорошие люди и хорошо
работать вместе. Отношения в стартапе можно определить одним
высказыванием: если их можно разрушить, то они будут разрушены.

Эммет Шир и Джастин Кань «Justin.tv» — хороший пример близких друзей,
которые слаженно работают вместе. Они знали друг друга со второго
класса. Они могут практически прочитать мысли друг друга. Я уверен,
что они спорят, как все основатели, но я никогда не ощущал
напряженности между ними.

\end{document}
