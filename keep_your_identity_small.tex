\documentclass[ebook,12pt,oneside,openany]{memoir}
\usepackage[utf8x]{inputenc} \usepackage[russian]{babel}
\usepackage[papersize={90mm,120mm}, margin=2mm]{geometry}
\sloppy
\usepackage{url} \title{Вагонные споры} \author{Пол Грэм} \date{}
\begin{document}
\maketitle

Сегодня я наконец-то понял, почему разговоры о политике и религии
бывают настолько бесполезными.

Как правило, любое упоминание религии на интернет-форуме скатывается в
религиозную перепалку. Почему? Почему это происходит с религией, а не
Javascript или кулинарией или другими темами, которые обсуждаются на
форумах?

Отличие религии в том, что люди считают, что можно иметь мнение, не
имея никаких знаний по теме. Всё, что им нужно — сильно верить, а это
умеет каждый. Никакая форумная ветка про Javascript не будет расти так
быстро, как ветка про религию, потому что люди считают, что чтобы
написать своё мнение, нужно обладать минимумом знаний. А по религии
каждый — эксперт.

Меня поразило вот что: та же проблема с политикой. Политика, как и
религия — тема, где нет порога знаний, чтобы высказывать мнение. Всё,
что нужно — сильные убеждения.

Есть ли у религии и политики что-то общее, объясняющее это сходство?
Возможное объяснение — в том, что в политике и религии вопросы не
имеют определённых ответов, поэтому нет ничего противостоящего мнениям
людей. Поскольку никому нельзя доказать неправоту, каждое мнение
одинаково допустимо, и чувствуя это, каждый кричит своё собственное.

Но эта версия не верна. Есть определённые политические вопросы, у
которых есть определённые ответы, например, во сколько обойдётся новая
политика правительства. Но и более точные политические вопросы
страдают той же бедой, что и нечёткие.

По-моему, у политики и религии общее — то, что люди считают их частью
себя, а когда такие вещи обсуждается, плодотворного спора никогда не
будет — потому что люди уже по определению приверженцы одной из
сторон.

Какие темы затрагивают личности спорящих, зависит от них самих, не от
тем. Например, если люди из нескольких государств будутобсуждать биву,
где участвовали их сограждане, вероятно, дискуссия снизойдёт до
политической перебранки. А разговор сегодня о битве, имевшей место в
бронзовый век, скорее, наверное, нет. Никто не знает, на чьей быть
стороне. Так что не политика источник проблем, а личность. Когда
говорят, что дискуссия выродилась в религиозную войну, под этим имеют
в виду, что тему уже больше движут чувства собственной личности
участников.

Если момент, когда такое случается, зависит скорее от людей, чем от
темы, то не правда что вопрос, порождающий религиозные войны, не имеет
ответа. Например, вопрос относительных заслуг языков программирования
часто переходит в религиозную войну, потому что многие программисты
считают себя программистами либо языка Х, либо Y. Часто это приводит
людей к выводу, что вопрос не имеет ответа — что все языки одинаково
хороши. Очевидно, что это ошибка: всё, что люди создают, может быть
задумано хорошо или плохо. Почему же лишь с языками программирования
это правило не работает? И на самом деле, можно плодотворно обсуждать
достоинства языков программирования, пока вы будете исключать из неё
людей, говорящих от своей личности.

Вообще, продуктивную дискуссию на какую-то тему можно вести только
если дискуссия не задевает личностей участников. Политику и религию
делает опасными, как минные поля, именно то, что они затрагивают
личности большого количества людей. Но в принципе, с некоторыми людьми
на эти темы можно общаться конструктивно. В то же время с другими
людьми лучше не разговаривать на темы, вроде бы, безобидные, как
достоинства пикапов Форд и Шевроле.

Самое увлекательное в этой теории, если она верна, в том, что она не
просто объясняет, каких разговоров стоит избегать, но и как улучшить
свои собственные мысли. Если люди не могут ясно мыслить о том, что
считают частью себя, то при прочих равных лучше всего считать частью
себя как можно меньше вещей.

Большинство читающих это эссэ уже и так достаточно толерантные люди.
Но можно сделать ещё шаг вперёд: не нужно считать, что вы — это А, но
терпимо относитесь к Б, а просто не нужно считать себя А. Чем больше
ярлыков вы на себя вешаете, тем вы глупее.


\end{document}
