\documentclass[ebook,12pt,oneside,openany]{memoir}
\usepackage[utf8x]{inputenc} \usepackage[russian]{babel}
\usepackage[papersize={90mm,120mm}, margin=2mm]{geometry}
\sloppy
\usepackage{url} \title{Как делить доли в стартапе} \author{Пол Грэм}
\date{}

\begin{document}
\maketitle

Инвестор готов дать вам деньги за некий процент вашего стартапа.
Соглашаться? Вы вот-вот наймете своего первого сотрудника. Сколько
акций ему пообещать? \newline

Это одни из тех сложных вопросов, которые встают перед основателями.
Но на это есть ответ:

\[
1/(1 - n)
\]

На что бы вы ни собирались обменять акции вашей компании, будь то
наличные, или сотрудники, или акции другой компании — формула та же.
Вам следует менять n процентов вашей компании в том случае, если в
итоге оставшиеся у вас (100 — n)\% больше, чем стоила компания до
обмена. \newline

Например, если инвестор хочет купить половину вашей компании,
насколько эти инвестиции должны увеличить стоимость всей компании,
чтобы вы остались при своем? Очевидно, стоимость должна увеличиться в
двое: если вы продаете половину за что-то, удваивающее стоимость вашей
компании, вы не останетесь в убытке. У вас останется половина, которая
будет стоить как целое. \newline

В общем, если n является той частью компании, которой вы жертвуете,
сделка будет хорошей в том случае, если стоимость компании будет
больше чем $1/(1 - n)$. \newline

Например, венчурный фонд «Y Combinator» предлагает профондировать вас
в обмен на 6\% вашей компании. В этом случае n равняется 0.06 и $1/(1 -
n)$ будет 1.064. Если мы увеличиваем итоговую стоимость на 10\%, вы в
выигрыше, т.к. оставшиеся у вас 0.94 стоят 0.94 х 1.1= 1.034. [1] \newline

Это справедливое уравнение показывает нам, что, по крайней мере, в
финансовом плане взять деньги у ведущей венчурной компании может быть
отличным ходом. Грег Макаду (Greg Mcadoo) из «Sequoia» недавно сказал
на обеде в «YC», что при самостоятельном инвестировании они обычно
забирают 30\% компании. 1/0.7 = 1.43, т.е. сделка будет стоящей, если
они увеличат итоговую стоимость компании на 43\%. Для стартапа средней
руки это исключительно выгодная сделка. Возможность сказать, что их
проинвестировала «Sequoia», даже если они и не получат никогда этих
денег, увеличит перспективы стартапа больше чем на 43\%. \newline

Вообще сделки с фондом «Sequoia» так хороши именно тем, что он
осознанно забирает малую часть компании. Они даже не пытаются получить
рыночную стоимость за свои деньги; они ограничивают свои аппетиты,
чтобы основатели могли чувствовать, что компания все еще принадлежит
им. \newline

Дело в том, что «Sequoia» получает около 6000 бизнес-планов в год, а
вкладывает средства только в 20. Поэтому шанс оказаться в их числе — 1
к 300. Те, у которых это получилось, являются необычными стартапами. \newline

Конечно, нужно учитывать и другие параметры при привлечении венчурного
финансирования. Это никогда не бывает простым обменом денег на акции.
Но все случалось именно так, то получить деньги от такой популярной
компании было бы удачей. \newline

Вы можете использовать ту же формулу при выделении акций сотрудникам,
но в данном случае она работает по-другому. Если i равняется
приблизительной стоимости вашей компании после найма дополнительного
сотрудника, тогда они стоят такой n, при которой $i = 1/(1 - n)$. Что
значит $n = (i - 1)/i$. \newline

Представим, что вас двое основателей, и вы хотите дополнительно нанять
грамотного программиста, который настолько хорош, что вы чувствуете:
«да, он увеличит общую стоимость вашей компании процентов на 20».
Тогда: $n = (1.2 - 1)/1.2 = 1.167$. Получается, вы ничего не потеряете,
если отдадите ему 16.7\% вашей компании. \newline

Это не значит, что нужно отдавать ему точно 16,7\% компании.
Возможность стать акционером не является единственным преимуществом,
ведь еще есть зарплата, да и другие издержки, которые вы будете нести
в связи с наймом сотрудника. Так что если даже в целом вы не
проиграете, то нет причин сразу так поступать. \newline

Я думаю, можно учесть зарплату и другие издержки в этой формуле,
умножив годовую величину на 1.5. Большинство стартапов либо быстро
растут, либо умирают. В последнем случае вам вообще не придется ему
платить, а в первом вы заплатите ему зарплату исходя из стоимости
компании через год, которая, скорее всего, будет в 3 раза больше. [2] \newline

Сколько дополнительной маржи должно быть у компании в качестве
«энергии активации» сделки? Так как это, в сущности, дает компании
преимущество при найме сотрудников: если рынок оценит вашу
привлекательность, вы сможете требовать большего. \newline

Разберем пример. Компания хочет получить 50\% «прибыли» от нового
вышеупомянутого сотрудника. Тогда отнимаем треть от 16.7\% и получаем
11.1\%, его «розничную» цену. Допустим, со временем он будет стоить 60
тыс. долл. в год, из-за зарплаты и прочих издержек умножаем на 1.5 =
90 тыс. долл., т.е. 4.5\%. 11.1\% — 4.5\% = предложение 6,6\%. \newline

Кстати, обратите внимание, что первым сотрудникам стоит брать
небольшую зарплату. Это позволит получить им больше акций компании. \newline

Конечно же, эти расчеты в определенной степени являются игрой. И я не
призываю раздавать акции компании, строго следуя этой формуле. Часто
вам придется действовать наугад. Но, по крайней мере, вы будете знать
в каких пределах. Теперь выбирая число, следуя собственной интуиции
или из типичной таблицы выплат какой-нибудь венчурной компании, вы
сможете проанализировать его. \newline

Формулу $1/(1 - n)$ можно использовать и в более широком смысле — когда
бы вам не приходилось принимать решение в вопросах акционерного
капитала компании для проверки их на целесообразность. Вы всегда
должны чувствовать себя богаче после размена акций компании. Если же
после продажи стоимость вашей собственной части не подорожает
настолько, чтобы вы остались при своем, вам не следует (не следовало)
этого делать.

\subsection{Примечания}

[1] Вот почему мы не можем поверить в то, что кто-то счел сделку с «Y
Combinator» плохой. Неужели кто-то и вправду считает нас настолько
бесполезными, что мы не сможем увеличить потенциал стартапа на 6,4\%
за три месяца? \newline

[2] Очевидный выбор для оценки вашей компании — пост-инвестиционная
оценка стоимости компании, после получения последнего транша
инвестиций. Вероятно, это выльется в недооцененность вашей компании,
потому что а) несмотря на то что, вам только передали последний транш,
компания предположительно стоит больше и б) оценка на первоначальных
этапах финансирования обычно отражает другое участие инвесторов. \newline

\end{document}
