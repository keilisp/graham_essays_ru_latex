\documentclass[ebook,12pt,oneside,openany]{memoir}
\usepackage[utf8x]{inputenc} \usepackage[russian]{babel}
\usepackage[papersize={90mm,120mm}, margin=2mm]{geometry}
\sloppy
\usepackage{url} \title{Слово о находчивости} \author{Пол Грэм}
\date{}
\begin{document}
\maketitle

Год назад я заметил следующую закономерность в неудачных стартапах,
которые мы финансируем: с их основателями трудно вести диалог. Такое
чувство, будто между нами стена. Никогда бы не сказал этого, если бы
они понимали, что я им говорил.

Это явление привлекло мое внимание, поскольку раньше мы отметили
закономерность среди наиболее успешных стартапов, и поначалу казалось,
что она другого рода. Мы финансировали стартапы и лучше работали те,
об учредителях которых мы могли бы сказать: «они в состоянии
позаботиться о себе сами». Лучшие стартапы подобны самонаводящейся
ракете, в том смысле, что вам всего лишь надо дать им направление, и
они последуют туда, независимо от того, что это за направление.
Например, когда они получают деньги, вы можете начать вникать в дело,
в то же время, отдавая себе отчёт, что вы можете совсем не думать об
этом деле на этом этапе. Вам не нужно нянчиться с ними, чтобы быть
уверенными в результате. Это тип учредителей, которые возвращаются к
вам с деньгами; единственный вопрос: сколько и на каких условиях.

Казалось странным, что успешных и неудачников можно было выявить
несвязанными тестами. Следовало ожидать, что основатели успешных
стартапов с одной стороны обладали выдающимся качеством Х, а
неудачники с другой стороны полным отсутствием качества Х. Была ли
какая-то инверсия между изобретательностью и даром красноречия?

Выходило, что так, и ключом к загадке служит старая пословица:
«Мудрому достаточно одного слова». Дело в том, что эта фраза
используется часто, и часто неправильно (обычно так говорят, перед
тем, как дать совет), большинство людей, слышащих её, не понимают, что
она означает. Она означает, что, если кто-то мудр, то вам достаточно
сказать одно слово, и вас немедленно поймут. Вам не надо описывать
детали, все нюансы будут схвачены.

Аналогично, все, что вам надо сделать, дать правильному основателю
правильную задачу, и он принесёт деньги. Готово. Обсуждение всех
нюансов — даже сомнительных моментов – о которых вам говорят, это
другая задача, это задача красноречия. Как и житейская мудрость,
умение хорошо говорить зачастую означает необходимость делать
неприятные вещи. Выяснение всех подробностей иногда может привести к
неудобным выводам. Хорошо описывает ситуацию отказа действовать слово
«разрыв», но оно имеет слишком узкое значение. Лучше описать ситуацию
следующим образом: неудачники обладают консервативностью, которая
происходит от слабости. Они пересекают пространство идеи осторожно, с
опаской, как старик улицу. [1]

Неудачники отнюдь не глупы. Они так же, как и успешные основатели,
понимают все тонкости проблемы. У них просто нет страстного желания
решить её.

Итак, не сами затруднения в диалоге являются причиной гибели
стартапов. Это знак отсутствия лежащей в основе находчивости. Вот что
их губит. Так же, как теряют тонкости того, что им говорят, неудачники
теряют денежные фонды, пользователей, ресурсы, новые идеи. И самый
очевидный показатель того, что что-то неправильно, это что я не могу
поговорить с ними.

[1] Партнер YC написал: Когда работаешь с плохой группой, то возникает
ощущение, что, приходя в офис, они уже заранее решили, что и как будут
делать. И всё, что им говоришь, они отчаянно пытаются подогнать к уже
принятому решению, или сразу отвергают предложение и потом ищут
рациональное объяснение для отказа. Они могут и не сознавать этого, но
это происходит, когда говоришь с плохими группами, у них есть нечто,
что закрывает другой взгляд на проблему. Не думаю, что это следствие
путаницы или отсутствия понимания, таков стиль работы.

Про работу с хорошими группами можно сказать, что всё, что вы
предлагаете, рассматривается свежим взглядом и даже если отвергается,
то по разумным основаниям. «Мы уже пробовали это», «Наши пользователи
говорят, что это им не нужно», и т. д. Эти группы никогда не
отказываются от другого взгляда на проблему.

\end{document}
