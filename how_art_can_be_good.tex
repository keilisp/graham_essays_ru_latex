\documentclass[ebook,12pt,oneside,openany]{memoir}
\usepackage[utf8x]{inputenc} \usepackage[russian]{babel}
\usepackage[papersize={90mm,120mm}, margin=2mm]{geometry}
\sloppy
\usepackage{url} \title{Искусство и трюки} \author{Пол Грэм} \date{}
\begin{document}
\maketitle

Меня воспитывали в представлении, что вкус — дело личных предпочтений.
Каждому что-то нравится, и ничьи предпочтения не лучше, чем
предпочтения других людей. Хороший вкус? Нет, такого не бывает. \newline

Как и многие другие вещи, которые я усвоил в юности, это неправда — и
я объясню, почему. \newline

Одна проблема с представлением, что хорошего вкуса не бывает —
выходит, что не бывает и хорошего искусства. Если существуют хорошие
произведения искусства, то значит, у людей, которым они нравятся,
более качественный вкус. Именно эти рассуждения заставили меня
разобраться в этом и отказаться от своей детской веры в релятивизм.
Когда ты пытаешься что-то реально сделать, вкус превращается в вопрос
практический. Нужно решить, что делать дальше. Станет ли картина
лучше, если я поменяю вот это? Если «лучше» — понятие относительно, то
что вы делаете, не имеет никакого значения. Неважно даже, пишете ли вы
картины или нет. Можно ведь пойти и купить белый холст. Если не бывает
хорошего и плохого искусства, то этот холст будет не менее великим
достижением, чем потолок Сикстинской капеллы. Ну да, трудозатрат
меньше, но если можно добиться того же уровня мастерства меньшими
усилиями, это ведь тоже впечатляет. \newline

Но что-то здесь не так, правда? \newline

\subsection{Аудитория}

Думаю, ключ к разгадке — то, что у искусства есть аудитория.
У искусства есть цель: заинтересовать аудиторию. Хорошее искусство (и
вообще все хорошее) — это искусство, которое особенно хорошо достигает
своей цели. Понятие «интереса» может быть разным. Одни произведения
искусства создаются, чтобы шокировать, другие — чтобы приносить
удовольствие; одни наскакивают на вас, а другие тихо сливаются с
фоном. Но любое искусство должно как-то подействовать на аудиторию, и
— ключевой момент — у представителей аудитории есть общие черты. \newline

К примеру, почти все люди интересуются человеческими лицами. Это,
похоже, запрограммировано в нас. Младенцы в состоянии распознавать
лица практически с рождения. И похоже, что лица эволюционируют вслед
за нашим интересом к ним; лицо — это как рекламный плакат для тела.
Так что при прочих равных картина с лицами заинтересует людей больше,
чем картина, на которой лиц нет. \newline

Многие считают, что вкус — дело сугубо личных предпочтений, потому что
иначе непонятно, как найти людей с хорошим вкусом. Людей миллиарды, у
каждого свое мнение, так на каком же основании предпочитать одного
другим? \newline

Но если у представителей аудитории есть много общего, то нам не
приходится выбирать из случайного набора личных предубеждений, потому
что этот набор не случаен. Все люди считают лица интересными,
практически по определению. И идея хорошего, достойного искусства —
искусства, которое хорошо выполняет свою задачу, — тоже не
предполагает, что мы выбираем несколько каких-то случайных людей и
признаем только их мнения правильными. Ведь кого бы вы ни выбрали, все
скажут, что лица интересны. \newline

Пришельцы из космоса, возможно, не посчитают человеческие лица
интересными. Но у нас с ними могут найтись другие общие интересы.
Вполне вероятно, это будет математика. Как мне представляется,
пришельцы в большинстве случаев согласятся с нами по поводу того,
какое из двух доказательств лучше. \newline

Как только речь заходит об аудитории, уже не приходится доказывать,
есть стандарты вкуса или нет. Скорее вкус — это серия концентрических
колец, как круги на воде. Некоторые вещи приятны вам и вашим друзьям,
другие — большинству людей вашего возраста, третьи — большинству
людей, а некоторые, наверное — большинству разумных существ (что бы
это ни значило). \newline

На деле, конечно, все несколько сложнее, потому что в середине пруда
разные круги пересекаются. Какие-то вещи особенно интересуют,
например, мужчин или людей из определенной культуры. \newline

Если хорошее искусство — это то искусство, которое заинтересовывает
аудиторию, то когда вы говорите о «качестве» искусства, нужно сразу
сказать — для какой аудитории. Так что же, если кто-то говорит, что
такое-то произведение искусства просто хорошее или просто плохое, то
это не имеет смысла? Нет, потому что есть и такая аудитория — все
люди, каких только можно вообразить. Я думаю, именно об этой аудитории
люди говорят, когда замечают, что вот эта работа хорошая: они имеют в
виду, что она заинтересует любое человеческое существо. \newline

И это осмысленный тест, потому что хотя «человек» — понятие немного
размытое, есть много вещей, которые характерны практически для любого
человека. Кроме интереса к лицам, это, например, распознавание
первичных цветов — такова особенность устройства наших глаз. Также у
большинства людей вызывают интерес изображения трехмерных объектов —
это тоже, видимо, встроено в нашу систему визуального восприятия. А
еще это распознавание границ, благодаря которому изображения с
конкретными формами вызывают больше реакции, чем совсем размытые
картины. \newline

Разумеется, у людей много и других общих характеристик. Я не собираюсь
составлять полный список, просто хочу показать, что здесь есть
реальная основа. Человеческие предпочтения не случайны. Поэтому
художник, работающий над картиной и пытающийся понять, нужно ли в ней
что-то поменять, не думает: «Да зачем что-то решать? Можно просто
подбросить монетку». Нет, он спрашивает себя: «Что сделает эту картину
более интересной для людей?» Сикстинская капелла интереснее для людей,
чем пустой холст, поэтому затмить Микеланджело не так-то просто. \newline

Многим философам трудно было поверить, что в искусстве возможны
объективные стандарты. Казалось очевидным, что, например, красота —
это исключительно представление наблюдателя, а не свойство предмета.
Поэтому она считалась субъективной, а не объективной. Но на самом
деле, если сузить определение красоты и сказать, что это нечто, что
оказывает определенное впечатление на людей, то все-таки это и
свойство объектов тоже. Не нужно выбирать между тем, субъективно
что-то или объективно, если все субъекты реагируют похожим образом.
Хорошее искусство — это такое же свойство объектов, как, например,
токсичность чего-либо для людей: искусство хорошее, если оно
систематически влияет на людей определенным образом. \newline

\subsection{Ошибка}

Так что, лучшие произведения искусства можно найти путем
простого голосования? Если главный критерий — привлекательность для
людей, можно просто их опросить? \newline

Не совсем. Для того, что создано природой, это может подойти. Мне
захочется съесть яблоко, которое все население мира посчитало вкусным,
или отдохнуть на пляже, который они назвали самым красивым, но идти
смотреть на картину, которая победила в голосовании публики — дело
рискованное. \newline

Прежде всего, художники, в отличие от яблонь, часто специально
стараются ввести нас в заблуждение. И порой довольно тонким образом. К
примеру, любое произведение искусства задает определенные ожидания
уровнем вложенного в него труда. Мы не ждем фотографической точности
от того, что выглядит наброском на скорую руку. Поэтому иллюстраторы
часто применяют такой трюк: они добиваются, чтобы картина или рисунок
выглядели так, будто они сделаны быстрее, чем на самом деле. Средний
человек смотрит и думает: какое же это мастерство! Это как сказать во
время беседы что-то, что как будто случайно пришло в голову по ходу
разговора — но на самом деле вы придумали это прошлым вечером. \newline

Еще один аспект, более очевидный — бренд. Если вы пришли в музей
увидеть «Мону Лизу», то вы, наверное, будете разочарованы, поскольку
она укрыта за толстым стеклом, и вокруг сходящая с ума толпа людей,
делающих селфи. В лучшем случае вам удастся разглядеть ее так же, как
друга в другом конце комнаты во время оживленной вечеринки. Лувр может
заменить ее копией, и никто не заметит. И при этом «Мона Лиза» —
маленькая темная картина. Если вы найдете человека, который никогда ее
не видел, и отправите его в музей, где она висит в ряду других картин
с пометкой, что это портрет работы неизвестного художника XV века,
этот человек, скорее всего, пройдет мимо, даже не разглядев ее как
следует. \newline

Для среднего человека бренд доминирует в оценке искусства. Когда они
видят картину, которую узнают по массе репродукций, это настолько
сильное впечатление, что они уже не в состоянии оценить ее как
картину. \newline

И к тому же есть масса трюков, которые люди проделывают сами с собой.
Большинство взрослых, видя некое произведение искусства, беспокоятся,
что если им оно не понравится, их посчитают некультурными. Они не
просто утверждают, что им что-то нравится: они действительно
заставляют себя полюбить то, что они «должны» полюбить. \newline

Вот почему голосование не решает проблему. Привлекательность для людей
— важный критерий, но на практике оценить ее невозможно — как нельзя
найти север, держа в руке одновременно компас и магнит. Бывают
настолько сильные источники искажений, что когда вы что-то измеряете,
то измеряете одни искажения. \newline

Но мы можем подойти к вопросу с другой стороны — самим выступить в
роли морских свинок. Вы человек. Если вы хотите понять, какой будет
базовая человеческая реакция на произведение искусства, вы можете
попробовать сделать это, избавившись от того, что вносит искажения в
ваши собственные суждения. \newline

К примеру, реакция любого человека на знаменитую картину будет прежде
всего определяться ее славой, но есть способы ослабить эти эффекты.
Например — возвращаться к картине снова и снова. Через несколько дней
налет славы поблекнет, и вы начнете видеть в ней картину. Другой
способ — подойти ближе. Картина, знакомая по репродукциям, выглядит
более знакомой с трех метров. Но приблизившись, вы увидите детали,
которые на репродукциях не разглядеть, и которые вы видите впервые. \newline

Есть два главных типа ошибок, которые мешают нам увидеть произведение
искусства: ваши собственные предубеждения и приемы, которые применяет
к вам художник. Последнее не так уж трудно поправить: если вы в курсе
того, что эти приемы есть и они действуют, они обычно перестают
действовать. Когда мне было десять лет, на меня производили огромное
впечатление буквы, нанесенные с помощью аэрографии и блестящие, как
металл. Но когда разберешься, как это сделано, понимаешь, что это
довольно дешевый трюк: это как очень резко дергать за несколько
ниточек нашего визуального восприятия, чтобы шокировать зрителя. Это
как пытаться убедить кого-то с помощью крика. \newline

Как выработать иммунитет к этим приемам? Специально искать их и
отслеживать. Когда вы чувствуете некий аромат нечестности, исходящий
от какого-то вида искусства, остановитесь и разберитесь, что
происходит. Когда кто-то очевидным образом потакает аудитории, которую
легко ввести в заблуждение — будь то блестящие штуки, которые
впечатляют десятилетних, или демонстративно авангардные картины, чтобы
произвести впечатление на желающих сойти за интеллектуалов, —
выясните, как это делается. \newline

Что считать трюком? Грубо говоря, это что-то, сделанное с прицелом на
аудиторию. К примеру, ребята, которые разрабатывали дизайн Ferrari в
1950-х, думаю, придумывали машины, которые их самих восхищали. Но
подозреваю, что в General Motors маркетологи говорят дизайнерам:
«Большинство людей покупают джипы, чтобы показаться мужественнее, а не
чтобы ездить по бездорожью. Поэтому не тратьте время на подвеску;
просто сделайте машину побольше да посуровее видом». \newline

Думаю, приложив некоторые усилия, вы можете приобрести иммунитет к
таким приемам. Избавиться от влияния собственного жизненного опыта
труднее, но можно хотя бы сделать некоторые шаги в этом направлении.
Нужно путешествовать и во времени, и в пространстве. Если вы узнаете,
какие вещи нравятся людям в разных культурах и какие вещи нравились
людям в прошлом, вероятно, это изменит и ваши собственные вкусы.
Сомневаюсь, что возможно стать абсолютно универсальным человеком — по
времени можно путешествовать только в одном направлении. Но если вы
найдете произведение искусства, которое одинаково трогает ваших
друзей, жителей Непала и древних греков, что-то в этом явно есть. \newline

Моя главная мысль здесь не о том, как приобрести хороший вкус, а о
том, что это в принципе возможно. И что хорошее искусство существует.
Это искусство, которое интересует аудиторию, и не случайным образом. А
раз есть такое искусство, то есть и хороший вкус — способность его
различать. \newline

Когда речь о яблоках, я соглашусь, что вкус — дело сугубо личное.
Одним людям нравятся такие яблоки, а другим — такие. Но искусство — не
яблоки. Это вещи, сделанные человеком. Это огромный культурный багаж,
и зачастую это еще применение специальных трюков. Представления
большинства людей об искусстве исходят именно из этих внешних
факторов, это как пробовать блюдо, в котором перемешаны яблоки и
острый перец. Все, что чувствуешь — это перец. Но есть люди, которые
могут различить в этом блюде яблоки. Это люди, которых нелегко
обмануть и которые не просто любят все, с чем они столкнулись в пору
взросления. Если вы найдете людей, которым удалось устранить такое
влияние на свои представления, то вы увидите, что им нравятся разные
вещи. Но и что они во многом соглашаются. И что они предпочтут купол
Сикстинской капеллы чистому холсту. \newline

\subsection{Делайте}

Я написал этот текст, потому что устал слышать, что вкус —
дело субъективное. Каждый, кто создает что-то реальное, чувствует, что
это не так. Но даже в мире искусства люди нервничают, когда им
приходится рассуждать о том, плохое какое-то произведение или хорошее.
Те, чья работа предполагает выдачу таких оценок (например, кураторы),
используют эвфемизмы вроде «значительный» или «важный». \newline

Но я написал этот текст не для тех, кто говорит об искусстве — а для
тех, кто его создает. Дети с амбициями идут в школы искусств — и
натыкаются на стену. Они приходят туда, надеясь, что когда-нибудь
станут столь же достойны, как знаменитые художники, чьи работы они
видели в книгах, но первым делом их учат, что понятие «хорошо» уже
устарело. Каждый должен лишь воплощать в жизнь свое личное видение. \newline

Когда я учился на факультете искусств, мы как-то раз рассматривали на
слайде прекрасную картину XV века, и один из учеников спросил: «Почему
сегодня художники не пишут таких картин?» И в комнате сразу стало
тихо. Этот вопрос редко задают вслух, но он сидит в голове у каждого,
кто изучает искусство. Это как поднять тему рака легких на совещании в
Philip Morris. \newline

«Ну, — ответил профессор, — нас сейчас интересуют другие вопросы». Он
был приятный парень, но в тот момент мне хотелось отправить его во
Флоренцию XV века, чтобы он объяснил Леонардо и компании, как нам
удалось преодолеть их ограниченные, примитивные представления об
искусстве. Просто представьте это себе. \newline

Флорентийские художники XV века создавали гениальные произведения в
том числе и потому, что верили: создавать великие произведения
возможно. Они остро конкурировали между собой и все время пытались
друг друга обойти, как математики или физики сегодня — да впрочем, как
все, кто делает что-то действительно хорошо. \newline

Представление, что можно создавать великие вещи — это не просто
полезная иллюзия. Они были правы. И самое важное следствие осознания,
что искусство бывает хорошее и плохое — это освобождает художников. И
всем людям с амбициями, которые приходят учиться с надеждой создать
когда-нибудь что-то великое, я скажу: не верьте, когда вам говорят,
что это наивные и устаревшие мысли. Хорошее искусство существует, и
если вы пытаетесь создать нечто подобное, вас заметят.

\end{document}
