\documentclass[ebook,12pt,oneside,openany]{memoir}
\usepackage[utf8x]{inputenc} \usepackage[russian]{babel}
\usepackage[papersize={90mm,120mm}, margin=2mm]{geometry}
\sloppy
\usepackage{url} \title{Как дела с философией } \author{Пол Грэм}
\date{}
\begin{document}
\maketitle

В школе я решил, что пойду учиться на философский. У меня был целый
ряд причин для такого решения, одни чуть уважительнее других. Среди
менее уважительных было желание шокировать людей. Там, где я рос, было
принято учиться в колледже, чтобы получить профессию для дальнейшей
работы, а философия в этом смысле исключительно непрактичный предмет.
Для меня это было чем-то вроде нарочитых дыр на джинсах или английских
булавок в ушах, которые тогда только входили в моду — такая же форма
выпендрежа в пику практичности.

Но были и действительно уважительные причины. Мне тогда казалось, что
если изучить философию, то можно стать мудрым. Мои ровесники,
выбиравшие себе другие предметы, получили бы узкоспециальные знания,
тогда как я бы изучил действительную суть вещей.

Я даже пробовал почитать кое-какую философскую литературу. Не новых
авторов — их не водилось в нашей школьной библиотеке. Но я старался
вникнуть в Платона и Аристотеля. Вряд ли я что-то действительно понял,
но мне казалось, что они говорят о чем-то важном. О чем именно — я
собирался узнать в колледже.

Во время своих последних школьных каникул я пошел на подготовительные
курсы. Я узнал много нового на математике, но не слишком многое вынес
из уроков философии. Тем не менее, мое намерение выучиться на философа
оставалось в силе. Я же сам был виноват, что ничего не понял. Нам
задавали кучу книг для чтения, а я их невнимательно читал. Я дал себе
слово более вдумчиво проштудировать "Трактат о принципах человеческого
знания" Беркли, когда поступлю. Такая популярная книга, написанная
таким сложным языком — ну явно же в ней было что-то важное. Понять бы,
что именно.

Двадцать шесть лет спустя я по-прежнему не понимаю Беркли. У меня есть
собрание его сочинений в неплохом издании. Буду ли я его когда-нибудь
читать? Как-то не верится.

Разница между тогда и сейчас заключается в том, что сейчас я понимаю,
почему Беркли, вероятно, не стоит того, чтобы пытаться его понять.
Думаю, сейчас я вижу, что именно пошло не так и как мы могли бы это
исправить.

Слова

Философия все же была моим основным предметом в колледже. Все
получилось не так, как я мечтал. Никаких волшебных истин, в сравнении
с которыми прочие знания являлись бы однобокими, я не постиг. Зато
сейчас я знаю, почему. В отличие от математики или истории, у
философии нет четкого предмета изучения. Нет определенной базы знаний,
которую нужно освоить. Есть разве что некоторое количество отдельных
философов, каждый из которых что-то там высказывал на те или иные темы
в те или иные годы. И немногие из них попали "в яблочко" настолько,
чтобы потомки повторяли открытые ими истины, забыв об авторстве.

У формальной логики есть определенный предмет изучения. Я проходил
логику. Не знаю, вынес я с тех лекций что-нибудь полезное или нет. [1]
В принципе, здорово уметь жонглировать различными идеями: видеть,
когда две идеи не полностью покрывают пространство возможностей, или
когда одна идея аналогична другой, лишь с небольшими различиями.
Однако убедили ли меня лекции по логике, что это важное умение?
Преуспел ли я в таком методе рассуждений? Не знаю.

На философии я действительно кое-чему научился. Самое большое
ученическое потрясение меня настигло почти в начале первого курса, на
лекции Сидни Шумейкера. Я узнал, что меня не существует. Я, равно как
и вы, — не более, чем набор клеток, ведомый туда-сюда различными
силами и называющий себя "я". Однако нет никакой ключевой, неделимой
вещи, которую можно было бы ассоциировать с индивидуальностью. Можно
лишиться половины мозга — и все равно жить, то есть, теоретически мозг
можно разделить пополам и трансплантировать в два различных тела.
Представляете, каково проснуться после такой операции? Раньше состоял
из одного человека, и вдруг состоишь из двух.

Урок на самом деле заключается в том, что в повседневной жизни мы
оперируем расплывчатыми понятиями. Стоит надавить чуть сильнее, и они
расползаются. Даже столь дорогое каждому понятие как "я". Это доходило
до меня довольно долго, но когда все-таки дошло, я был удивлен, как
человек из XIX столетия, который вдруг узнал об эволюции и осознал,
что история сотворения мира, в которую он верил с детства, — неправда.
[2] Вне математики всегда существуют расплывчатые границы, внутри
которых можно рассматривать смысл конкретного слова; фактически,
определив математику как науку о словах, имеющих совершенно конкретное
значение, мы не слишком бы погрешили против истины. Понятия, которыми
мы оперируем в повседневности, не являются безупречно точными. Они
вполне годятся для обихода, поэтому мы этого и не замечаем. Слова
вроде бы работают, как и ньютоновская физика. Но копни поглубже — и
они рассыпаются в прах.

К несчастью для философии, в этом, я бы сказал, и заключается ее суть.
Большинство философских споров не просто заходят в тупик из-за
путаницы со словами, они к этой путанице и сводятся. Есть ли у нас
свобода воли? В зависимости от того, что вы понимаете под свободой.
Существуют ли абстрактные идеи? Зависит от того, что такое, по-вашему,
существовать.

Широко цитируется идея Витгенштейна о том, что большинство философских
парадоксов упираются в языковые нюансы. Не знаю, насколько заслуженно
Витгенштейну приписывают авторство этой идеи. Подозреваю, что
большинство людей прекрасно понимало это и без него, но это не
сподвигло их заняться философией, и поэтому они не превратились в
профессоров и мыслителей.

Как же так получилось? Неужели предмет, который люди изучали тысячи
лет, может в действительности быть пустой тратой времени? Это
интересный вопрос. Может быть, один из наиболее интересных среди всех,
какие можно задать о философии. Самый здравый подход к современной
философской традиции — это не ударяться в беспредметные спекуляции,
как Беркли, и не отрицать спекуляции, как Витгенштейн, а постараться
изучить предмет как наглядный пример того, как разум способен выбрать
ложный путь.

История

Западная философия действительно начинается с Сократа, Платона и
Аристотеля. От их предшественников нам остались только фрагменты работ
и ссылки в более поздних трудах. Их доктрины являли собой
спекулятивную космологию с пространными рассуждениями. Вероятно, ими
двигала та же сила, что заставляет людей в любом обществе однажды
изобрести космологию. [3]

С появлением Сократа, Платона и в особенности Аристотеля эта традиция
изменилась. Все стало подвергаться кропотливому анализу. Подозреваю,
что Платона и Аристотеля вдохновлял прогресс в математике. Математики
к тому времени показали, что знание о вещах можно представить в форме
гораздо более строгой, чем красивая история. [4]

Люди в наше время так много оперируют абстракциями, что мы не можем
вообразить, каким прорывом это было — когда наши предки впервые
научились это делать. Вероятно, это произошло много тысяч лет назад,
где-то между появлением описаний вещей как "холодных" и "теплых" и
чьим-то вопросом: "Что такое тепло?". Развитие, конечно, было
постепенным. Мы не можем знать наверняка, были ли Платон или
Аристотель первыми, кто задался вопросами, которые мы теперь им
приписываем. Но их работы — самые старые источники, в которых эти
вопросы рассматриваются настолько подробно, причем наблюдается
определенная свежесть (если не сказать — наивность) подхода, которая
свидетельствует о том, что некоторые вопросы действительно были новыми
— как минимум, для авторов.

В частности, Аристотель напоминает мне о том явлении, что человек,
открыв что-то новое, настолько этим восхищён, что сразу же осваивает
огромные, неизведанные ранее территории. Это показывает, насколько
новаторским был этот новый образ мышления. [5]

Я хочу сказать, что Платон и Аристотель могут быть одновременно
великими, наивными — и они могут ошибаться. В самом деле, восхищения
заслуживает уже сам факт, что они задавались такими сложными
вопросами. Но это не значит, что они предложили какие-то хорошие
ответы. Мы ведь не считаем кощунственным мнение, что древнегреческие
математики были во многом чересчур наивными или что им, по крайней
мере, не доставало кое-какого понимания, которое облегчило бы их
жизнь. Поэтому я надеюсь, что если я предположу, что философы
древности были примерно настолько же наивны, люди не сочтут, что я
кощунствую. Ведь они, судя по всему, не в полной мере понимали то, что
я недавно назвал сутью философии: словесные понятия рассыпаются, если
копать слишком глубоко.

Род Брукс писал: "К удивлению создателей первых цифровых компьютеров,
программы, которые писались специально для них, обычно не работали".
[6] Нечто похожее произошло, когда люди впервые заговорили об
абстракциях. К их немалому удивлению, они никак не могли сойтись на
каких-то общих ответах. И, в сущности, вовсе не находили ответов как
таковых.

Они находились под впечатлением, споря о выводах, полученных методом
индукции при слишком низкой степени анализа.

Мы можем убедиться, насколько бессмысленны были выводы, к которым они
приходили, по тому, насколько незначительно оказалось их влияние.
Никто не помыслил изменить свою жизнь, прочитав "Метафизику"
Аристотеля. [7]

Я не утверждаю, что у абстрактных идей должно быть практическое
применение, чтобы с ними можно было иметь дело. Совсем не обязательно.
Заявление Харди о том, что теория чисел совершенно бесполезна на
практике, нисколько не обесценило саму теорию. Однако позже оказалось,
что Харди ошибался. Вообще, в математике подозрительно сложно найти
такой раздел науки, который действительно не имел бы какого-нибудь
практического применения. И от аристотелева объяснения главной задачи
философии в первой книге "Метафизики" тоже должна быть какая-то
польза.

Теоретические знания

Целью Аристотеля было выделить наиболее универсальные принципы из всех
существующих. Примеры, которые он приводит, выглядят вполне
убедительно: наемный работник строит дом по инерции, опираясь на
привычку; ремесленник же способен на большее, потому что больше
понимает про материалы и процесс. Тенденция ясна: чем более
универсально знание, тем оно больше в почете. А затем Аристотель
совершает ошибку — возможно, самую влиятельную ошибку в истории
философии. Он замечает, что теоретические познания зачастую получают
ради познания как такового, из любопытства, а не ради практического
применения. Тогда он делает вывод, что есть две разновидности
теоретического знания: полезное и бесполезное на практике. Поскольку
люди, заинтересованные во второй разновидности, стремятся к знанию
бескорыстно, он решает, что их цель благороднее. И главной задачей
своей "Метафизики" он делает исследование бесполезного на практике
знания. Следовательно, ничто не смущает читателя, когда автор
затрагивает масштабные, но едва понятные темы и сам теряется в
мешанине слов.

Он неверно истолковал связь мотива с результатом. Понятно, что люди,
которые стремятся к глубокому пониманию какого-либо предмета, часто
бывают движимы любопытством, а не практической необходимостью. Но это
совершенно не значит, что знание, которое они получают в процессе,
бесполезно. На практике всегда очень ценно обладать глубоким
пониманием того, что ты делаешь, даже если тебе никогда не придется
решать действительно сложные задачи — тебе будет очевиднее, как решать
простые. И твое знание не даст осечку в критической ситуации, как
случилось бы, если бы ты опирался на выведенные кем-то другим формулы,
значения которых не понимаешь. Знание — сила. Именно поэтому иметь
большой багаж теоретических знаний — престижно. И именно потому умные
люди зачастую выражают любопытство по поводу одних вещей и игнорируют
другие. Наш ДНК не настолько беспристрастен, как может показаться.

То есть, чтобы идея показалась нам интересной, у нее не обязательно
должно быть немедленное практическое применение, но у действительно
интересных идей оно, как правило, находится.

Причина, по которой Аристотель зашел в тупик, создавая свою
"Метафизику", — противоречивые цели, которыми он руководствовался. Он
пытался исследовать абстрактные концепции, считая их заведомо
бесполезными. Тем самым он уподобился путешественнику, который искал
некие северные земли, отталкиваясь от представления, что север
находится на юге.

А поскольку его трудом много поколений пользовалось как путеводителем,
получилось, что он всех повел неверной дорогой. [8] Хуже всего,
пожалуй, тот факт, что он же заведомо оградил своих последователей и
от критики, и от личных сомнений в истинности пути, потому что взял за
основу мысль, что самая благородная разновидность знания по
определению бесполезна.

"Метафизика" — неудачный эксперимент. Там есть несколько ценных идей,
но большая часть никуда не годится. Это наименее читаемая из всех
известных книг. И вовсе не потому, что ее трудно понять в том же
смысле, как трудно понять "Принципы" Ньютона, а потому, что трудно
понять искаженное послание.

Возможно, это очень интересный неудачный эксперимент. Но, к сожалению,
последователи Аристотеля пришли к совершенно другому выводу. [9]
Вскоре после западный мир накрыло интеллектуальное затмение. На смену
старой философии не появилось ничего нового. Вместо того, работы
Платона и Аристотеля стали почитать как священные тексты, которые
следовало штудировать и обсуждать. И такое положение дел затянулось на
невероятно долгий срок. Только где-то в начале 17 века (В Европе, где
центр внимания был затем смещён) появились более-менее уверенные
утверждения, что труды Аристотеля — не более чем набор заблуждений. Но
даже тогда эти утверждения редко звучали открыто.

Если вас удивляет, что пауза затянулась на столько столетий,
вспомните, как долго длился застой математики. С эллинов до
Возрождения!

Задолго до переломного момента возникла неприятная тенденция: писать
труды, подобные "Метафизике", стало не только приемлемо, но и
престижно — это было занятие, которому посвящали себя специальные
деятели, которых называли философами. Никому не приходило на ум
поспорить с Аристотелем. Так что вместо того, чтобы исправить беду,
которую Аристотель открыл, попав в нее — слишком пространные
рассуждения о слишком абстрактных понятиях — люди продолжили в нее
попадать.

Сингулярность

Как ни удивительно, порождаемые "философами" творенья пользовались
вниманием у читателей. Традиционная философия в этом смысле странным
образом зачаровывает. Если написать что-то невнятное о чем-то
масштабном, то результат будет как магнит притягивать неопытных, но
жадных до знания учеников. До поры нелегко отличить то, что трудно
понять из-за бардака в голове у автора от того, что трудно понять
просто потому что трудно понять, вроде какого-нибудь математического
доказательства. Для тех, кто не видит разницы, классическая философия
кажется крайне привлекательной дисциплиной: такой же сложной (и, стало
быть, впечатляющей окружающих) как математика, но гораздо более
всеобъемлющей. Именно на это я и соблазнился, будучи еще школьником.

Эта особенность ещё более особенна тем, что в ней есть встроенный
защитный механизм. Когда в чем-то трудно разобраться, люди, которые
подозревают, что разбираться вообще-то не в чем, обычно помалкивают.
Потому что нет способа доказать, что тот или иной труд — бессмыслица.
Самое большее - Вам удастся показать что официальные суждения о
опередённом классе трудов не могут отличить их от подобных имитаций.

И вот, вместо того, чтобы развенчать философию, большинство из тех,
кто догадывался, что это пустая трата времени, просто занялись другими
дисциплинами. Уже сам по себе этот факт компрометирует науку, особенно
если брать во внимание ее амбиции. Ведь философия должна быть наукой
про абсолютные истины. Наверняка все умные люди были бы в ней
заинтересованы, если бы она оправдывала свое намерение.

Поскольку ошибки философии оттолкнули тех, кто мог бы их исправить,
эти ошибки оказались увековечены. Бертранд Рассел писал в 1912:

"...До сих пор философия привлекает в основном тех, кто любит ее
широкие обобщения, которые всегда ошибочны, так что мало людей с умом,
жаждущим точности, занялись этой наукой." [11]

Он хотел, чтобы Витгенштейн отреагировал на этот посыл. И результаты
были ошеломляющими.

Лично я считаю, что имя Витгенштейна стоит увековечить даже не за то,
что он открыл, что практически вся старая философия — это трата
времени. Судя по косвенным доказательствам, к этому выводу пришел
каждый разумный человек, который хотя бы немного вник в философию и
отверг ее, чтобы заняться чем-нибудь другим. Витгенштейн отреагировал
совершенно по-новому. [12] В отличьи от всех остальных, он не
переключился втихомолку на другой предмет. Он поднял шумиху, причем
поднял ее изнутри системы. Он был Горбачевым своего времени.

Витгенштейн навел на философию такого страха, что дисциплину до сих
пор потряхивает. [13] Позже в своей биографии он много говорил о
влиянии формулировок. Поскольку такой подход кажется приемлемым,
многие философы в наши дни занимаются тем же самым. Тем временем,
заметив, что ниша метафизических спекуляций пустует, литературные
критики ***вдруг стали клониться в сторону Канта*** и разводить т.н.
"теорию литературы", ***critical theory***, а если автор особо
амбициозен, то и просто "теорию". Их труды представляют собой
ожидаемую словесную кашу:

Род не похож ни на одну другую грамматическую категорию которые точно
описывют образ понятия безотносительно реальности соответствующей
понятийному образу и, стало быть, не отражает точно чего-то из
окружающей действительности , которой ум может быть сподвигнут к
зачатию мысли об этом даже в случаях, когда мотивом не является что то
подобное. [14]

Особенность, о которую я описал, не исчезает. Существует рынок для
писанины, которая звучит так впечатляюще и не может быть опровергнута.
Всегда будет и то, и другое: спрос и предложение. И если одна из групп
покинет эту территорию, всегда найдутся другие, готовые занять её.

Предложение

Может быть, нам удастся перехватить инициативу. Вот, например,
интересная возможность. Почему бы нам не сделать то, что хотел сделать
Аристотель — до того, что он в результате сделал. В "Метафизике" он
формулирует весьма достойную цель: обнаружить универсальные истины.
Звучит неплохо. Но вместо того, чтобы искать их ради их бесполезности,
давайте их поищем ради их пользы.

Я предлагаю попробовать тот же путь заново, но руководствоваться
теперь критерием, который прежде отвергался как презренный:
практическим применением. Это не даст нам заплутать в абстракциях.
Вместо того, что бы пытаться ответить:

"Какие истины наиболее универсальны?"

давайте попытаемся ответить на вопрос:

"Из всех полезных вещей которые мы можем перечислить, какие наиболее
универсальны?"

Тест на пригодность, который предлагаю я: вызовем ли мы желание у
людей, прочитавших написанное нами, поступать как-то по другому или
нет. Понимая, что нам предстоит, по сути, создать свод конкретных
(даже категорических) советов, мы сделаем все возможное, чтобы не
уходить далеко за пределы анализа понятий, которыми мы будем
оперировать.

Эта цель совпадает с целью Аристотеля, мы просто собираемся подойти к
ней с другого конца.

Хороший пример полезной универсальной идеи — контролируемый
эксперимент. Эту идею широко применяют на практике. Некоторые скажут,
что это неотъемлемая часть науки, но ведь это не является частью
какой-то конкретной науки, это в чистом виде мета-физика (в
современном смысле приставки "мета"). Еще хороший пример — теория
эволюции. У нее очень широкая сфера применения — от алгоритмов
генетики до промышленного дизайна. Франкфуртское различие между ложью
и полной чепухой кажется многообещающим современным примером. [15]

Мне кажется, именно такой должна быть философия: на основе наблюдений
делать универсальные утверждения, которые вдохновят тех, кто их
поймет, на какой-нибудь новый поступок.

Объектом таких наблюдений неминуемо станут вещи, которые нечетко
сформулированы. Как только мы начинаем использовать точные
формулировки, мы занимаемся математикой. В этом смысле
руководствоваться практическим применением — не полное решение
проблемы, которую я описал выше — оно не смоет метафизической
сингулярности. Но все-таки это облегчит задачу. У людей с благими
намерениями появится путеводитель по абстракции. Может быть, это
сподвигнет их делать вещи, на фоне которых творения людей с дурными
намерениями будут выглядеть соответственно дурными.

Один из недостатков этого подхода заключается в том, что он не будет
являться источником работ, которые принесут Вам признание. И дело не
только в том, что это сейчас не в моде. Просто для того, чтобы быть
признанным в любой области, Вы должны прийти к выводам, с которыми
могут согласиться участники этой области. На практике существует два
вида решений этой проблемы. В математике и других точных науках вы
можете доказать то, что сказали, или по крайне мере так построить свои
рассуждения, что бы не произнести ничего заведомо ложного ("у 6 из 8
испытуемых наблюдалось пониженное кровяное давление после
эксперимента"). В гуманитарных науках Вы можете либо избегать
формулировок любых определённых выводов (например выводов о том, что
тема сложна), либо сформулировать выводы такими узкими, что никому не
будет дела до того, чтобы не соглашаться с Вами.

Тот вид философии, который защищаю я, не сможет пойти ни по одному из
этих путей. В лучшем случае Вы сможете достигнуть эссеистического
стандарта доказательств, а не математического или экспериментального.
И тем не менее, Вам не удастся пройти тест пригодности без того, чтобы
сформулировать определенные тезисы, понятные людям. Хуже того, сама
эта проверка на пригодность может вызвать всеобщее разочарование: люди
не любят слушать что-то новое, а то, что они и так знают -- повторять
им довольно бессмысленно.

Однако здесь-то и находится самое интересное. Каждый может заниматься
философией. Может быть, достижение "общего и полезного" начиная лишь с
"полезного", наплевав на "общее", и не годится для юного профессора,
пытающегося завоевать признание, зато отлично подходит всем остальным
и даже другим профессорам. Это хороший плацдарм для последовательной
разработки. Можно начать с небольших, но полезных областей знания,
пусть даже и очень специфических, чтобы потом постепенно расширяться
до общих принципов. Джо умеет делать отличные буррито. Что хорошего в
буррито? Что хорошего в еде? Что вообще хорошо? Тут можно идти как
угодно далеко. Никто не заставляет вас каждый раз идти до конца. Ну а
вы никому и не говорили, что на самом деле занимаетесь философией.

Если идея заняться философией кажется вам привлекательной, хочу вас
еще приободрить. Наука гораздо моложе, чем кажется. Хотя
родоначальники европейской философской традиции жили 2500 лет назад,
нельзя сказать, что философии 2500 лет, потому что большую часть этого
времени ведущие деятели занимались в основном тем, что писали
комментарии к Платону и Аристотелю, опасливо озираясь через плечо на
предмет следующего поколения таких же деятелей. Затем в некоторый
момент философия оказалась безнадежно переплетена с религией, и
освободилась от этих пут буквально пару сотен лет назад, но даже тогда
страдала от структурных проблем, описанных мною выше. Сейчас кто-то
мне возразит, что я недопустимо передергиваю и обобщаю ситуацию;
другие скажут, что это не новость, однако послушайте: судя по
написанным трудам, большинство философов до наших дней тратили время
зря. Из этого следует, что в некотором смысле философия сейчас
находится на ранней стадии развития. [16]

Это утверждение может показаться абсурдным сейчас. А через 10,000 лет
покажется уже не таким абсурдным. Цивилизация всегда кажется чем-то
древним, потому что не с чем сравнивать. Единственный способ
определить, насколько что-то действительно является древним, — это
найти структурные доказательства, а структурная философия как раз
молодое направление, причем она до сих пор не оправилась от
неожиданного облома с несовершенством понятий.

Философия сегодня так же молода, как математика в 16 веке. Большие
открытия еще впереди.

Примечания

[1] На практике от формальной логики не так уж много пользы, потому
что, невзирая на кое-какой прогресс, наблюдаемый последние 150 лет, мы
по-прежнему можем формализовать лишь небольшой процент утверждений. Не
факт, что мы можем сделать это намного лучше -- по той же причине
"представление знаний" в стиле 1980х гг никак не могло заработать;
многие утверждения непредставимы в виде более кратком, чем громоздкая,
аналоговая структура мозга.

[2] Мы не можем даже приблизительно понять, насколько трудно далось
новое знание современникам Дарвинам. Библейская история сотворения
мира — это не просто иудео-христианская концепция. Это то, во что люди
верили во все времена. Самое трудное для понимания в эволюционной было
то, что виды не остаются неизменными во все времена, а происходят от
других, более простых организмов, на протяжении невообразимо длинных
промежутков времени.

Сейчас нам не приходится переживать такой скачок от старых убеждений к
новым. Никто в современном индустриальном обществе не сталкивается с
идеей эволюции настолько неожиданно. Всем рассказывают про эволюцию
еще в детстве — неважно, как про истинное положение дел или как про
ересь, но сама концепция не является незнакомой ни одному взрослому
человеку.

[3] Греческие философы, жившие до Платона, писали свои труды в стихах.
Вероятно, это повлияло на содержание их посланий. Попробуйте описать
природу мира в стихотворной форме — неизбежно получится
заклинание-заговор. Проза позволяет быть более точным и более
свободным в выборе формы подачи.

[4] Философия — это непутевая сестра науки математики. Она родилась,
когда Платон и Аристотель, посмотрев на работы своих предшественников,
воскликнули: "Ну отчего ты не можешь быть как твоя сестра?!" Причем
Расселл фактически говорил то же самое 2300 лет спустя.

Математика — точная составляющая абстрактных идей, а философия —
неточная. Видимо, философия неизбежно всегда будет проигрывать
математике при сравнении, потому что у точности нет степеней. Плохая
математика в худшем случае просто скучна, а вот плохая философия — это
чушь. Тем не менее, в неточной составляющей есть немало хорошего.

[5] Лучшие работы Аристотеля были по логике и зоологии. Можно сказать,
что он открыл ту и другую. Однако главное, что отличало его от
предшественников, — это новый, гораздо более аналитический склад
мышления. Возможно, Аристотель был, по сути, первым ученым.

[6] Brooks, Rodney, Programming in Common Lisp, Wiley, 1985, p. 94.

[7] Некоторые утверждают, что мы обязаны Аристотелю гораздо большим,
нежели нам кажется, потому что его идеи — ключевые ингредиенты нашей
современной культуры. Разумеется, многие понятия, которыми мы сейчас
оперируем, берут начало из работ Аристотеля, однако считать, что без
него у нас не было бы концепции сущности объектов или мы бы не
различали форму и содержание — это все-таки перебор.

Чтобы проверить, действительно ли европейская культура так зависима от
концепций Аристотеля, достаточно ответить на вопрос: какие идеи — из
аристотелевых — были у европейской культуры в 19 веке, которых не было
в китайской?

[8] Время изменило значение слова "философия". В древности оно
охватывало множество дисциплин, которые мы сегодня называем
гуманитарными науками (только без ***methodical implications***). Даже
во времена Ньютона она еще подразумевала некоторый научный подход. Но
суть науки в наше время сохранилась та же, что и при Аристотеле:
философия — это поиск универсальных истин.

Аристотель не называл это метафизикой. Его труды получили такое
название, потому что "Метафизика" появилась после ("мета" значит
"после") "Физики", собрания работ Аристотеля, выпущенного под
редакцией Андроника Родосского три столетия спустя. То, что мы сейчас
называем "метафизикой", Аристотель называл "первой философией".

[9] Кто-то из ранних последователей Аристотеля, вероятно, это понял,
но мы не можем знать наверняка, так как большинство их работ утеряно.

[10] Sokal, Alan, "Transgressing the Boundaries: Toward a
Transformative Hermeneutics of Quantum Gravity," Social Text 46/47,
pp. 217-252.

Абстрактно звучащая чепуха может быть весьма привлекательна, если ее
ее умело подогнать под чьи-то реальные проблемы и страхи. Таким
образом на нее поведутся скорее всего слабейшие. Сильные духом не
нуждаются в лишних подтверждениях.

[11] Letter to Ottoline Morrell, December 1912. Процитировано по:

Monk, Ray, Ludwig Wittgenstein: The Duty of Genius, Penguin, 1991, p.
75.

[12] Мнение, что вся метафизика, имевшая место между Аристотелем и
1783, была пустая трата времени, возникло благодаря И. Канту.

[13] Витгенштейн отстаивал тот тип знания, к которому обитатели
Кембриджа начала XX в. были особенно восприимчивы, быть может, отчасти
потому, что столь многие из них были воспитаны в религии, а потом
перестали верить, и в их головах образовалось свободное место для
кого-то, кто скажет им, что делать (остальные выбирали Маркса или
кардинала Ньюмана), а отчасти по той причине, что такое тихое и
серьезное место как Кембридж в то время не имело природного иммунитета
против мессианских фигур, так же как европейские политики не имели
тогда иммунитета против диктаторов.

[14] Это из "Ordinatio of Duns Scotus" (ок. 1300), только "номер"
изменено на "род"

Wolter, Allan (trans), Duns Scotus: Philosophical Writings, Nelson,
1963, p. 92.

[15] Frankfurt, Harry, On Bullshit, Princeton University Press, 2005.

[16] Некоторые введения в философию сейчас продвигают мысль, что
изучение философии — это процесс, а не поиск определенных истин.
Философы, которые в этих работах упоминаются, наверняка перевернулись
в гробах от такой трактовки. Они-то надеялись, что их труды — нечто
большее, нежели руководство по ведению спора. Для них результат был
значимой частью процесса. Большинство их выводов оказалось ошибочными,
но саму веру в результат нельзя назвать бессмысленной.

Этот аргумент выглядит так же убого, как ученые 16 века, которые
посмотрели на результаты, достигнутые алхимией, и заключили, что она
хороша как процесс и не более того. Они ошибались. Сейчас мы знаем,
что золото действительно можно добывать из свинца (хотя при нынешних
ценах на электричество это не экономный процесс), но путь, который
привел нас к этому знанию, — это возвращение к истокам и проба нового
подхода.

\end{document}
