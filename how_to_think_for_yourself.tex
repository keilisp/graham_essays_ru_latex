\documentclass[ebook,12pt,oneside,openany]{memoir}
\usepackage[utf8x]{inputenc} \usepackage[russian]{babel}
\usepackage[papersize={90mm,120mm}, margin=2mm]{geometry}
\sloppy
\usepackage{url} \title{Кто любопытен, тот независим} \author{Пол
  Грэм} \date{}
\begin{document}
\maketitle

В некоторых видах деятельности для успеха необходимо мыслить иначе,
чем все остальные. Например, чтобы быть успешным ученым, недостаточно
просто выдвигать правильные идеи. Они должны быть одновременно
правильными и новаторскими. Вы не можете публиковать статьи, в которых
говорится о чем-то, что всем уже известно. Вы должны предложить что-то
такое, о чем никто еще не говорил. \newline

То же самое и с инвесторами. Инвестору на фондовом рынке недостаточно
правильно предсказать, как будет работать компания. Если много других
людей сделают такой же прогноз, курс акций скорректируется, и
заработать не получится. Единственные ценные идеи — это те, которые не
разделяют другие инвесторы. \newline

Эта же закономерность наблюдается и среди основателей стартапов.
Стартап с общепризнанно хорошей идеей не нужен, потому что это значит,
что другие компании уже ею занимаются. Вы должны сделать что-то, что
большинству людей покажется плохой идеей, но вы знаете, что это не
так. Например, написать программное обеспечение для крошечного
компьютера, которым пользуются несколько тысяч любителей, или создать
сайт, чтобы люди могли арендовать надувные кровати у незнакомцев. \newline

То же самое с эссеистами. Эссе, рассказывающее людям то, о чем они уже
знают, будет скучным. Нужно рассказать им что-то новое. \newline

Но этот шаблон не универсален. В большинстве занятий он не работает.
Например, в работе администратора все, что вам нужно — только первая
часть, то есть делать правильно. И не обязательно, чтобы все остальные
ошибались. \newline

В большинстве занятий есть место для новизны, но на практике
существует довольно четкое различие между занятиями, в которых важно
иметь независимое мышление, и делами, в которых оно не важно. \newline

Мне жаль, что никто не рассказал мне об этом отличии, когда я был
ребенком, потому что это одна из самых важных вещей при выборе рода
занятий. Хотите ли вы заниматься такой работой, в которой успех
приходит, только если мыслить иначе, чем все остальные? Я подозреваю,
что подсознание большинства людей ответит на этот вопрос прежде, чем
они успеют задуматься. \newline

Независимость мышления, кажется, больше зависит от природы, чем от
воспитания. Это означает, что выбрав неправильный род занятий, вы
будете несчастны. Если вы от природы обладаете независимым разумом,
вам будет некомфортно быть менеджером среднего звена. А если у вас от
природы традиционное мышление, то попытки провести оригинальное
исследование будут выглядеть как поход на парусах против ветра. \newline

Однако одна из трудностей заключается в том, что люди часто ошибаются,
оценивая свое место в этом спектре мышления. Традиционно мыслящим
людям не нравится думать о себе в таком ключе. Им всегда искренне
кажется, что они принимают собственные решения обо всем. И это просто
совпадение, что их убеждения совпадают с убеждениям окружающих. А люди
с независимым мышлением часто не осознают, насколько их идеи
отличаются от традиционных, по крайней мере, пока не заявят о них
публично. \newline

К совершеннолетию большинство людей примерно знают, насколько они умны
(то есть способны решать заранее поставленные задачи), потому что их
постоянно проверяют и оценивают в соответствии с этим. Но школы обычно
игнорируют независимое мышление, кроме тех случаев, когда они пытаются
его подавить. Так что мы не можем получить обратную связь о том,
насколько мы независимы в своих взглядах. \newline

Здесь возможно даже что-то вроде эффекта Даннинга-Крюгера: люди с
заурядным мышлением уверены, что мыслят независимо, в то время как
другие, с подлинно независимым мышлением, опасаются, что думают
недостаточно свободно. \newline

Можно ли развить в себе независимое мышление? Я думаю, да. Это
качество может быть в значительной степени врожденным, но, кажется,
есть способы усилить его или, по крайней мере, не подавить. \newline

Один из самых эффективных методов — это тот, который непреднамеренно
используют большинство гиков: просто меньше задумываться, что вообще
такое общепринятые убеждения. Трудно быть конформистом, если вы не
знаете, чему нужно соответствовать. Хотя, возможно, такие люди уже
обладают независимым мышлением. Традиционно мыслящий человек,
вероятно, чувствует беспокойство, не зная, что думают другие, и
прилагает больше усилий, чтобы выяснить это. \newline

Очень важно ваше окружение. Если вас окружают люди с общепринятым
мышлением, это ограничит то, какие идеи вы можете выражать, и, в свою
очередь, ваши собственные идеи. Но если вы окружите себя людьми с
независимым складом ума, опыт будет противоположным: вы сможете
слышать от других удивительные вещи, и это побудит вас думать о
большем. \newline

Поскольку людям с независимым складом ума некомфортно находиться в
окружении людей с ординарным мышлением, они стараются изолироваться,
как только появляется такая возможность. Проблема в том, что в старшей
школе она им не представляется. К тому же школа, как правило,
представляет собой маленький мир, жители которого испытывают
недостаток уверенности в себе. И то, и другое усиливает силы
конформизма. И поэтому средняя школа — зачастую плохое время для
независимо мыслящего человека. Но есть здесь и преимущество: этот
период учит, чего следует избегать. Если в будущем вы окажетесь в
ситуации, которая напомнит вам школу, знайте, что пора на выход. \newline

Еще одно место, где объединяются независимые и традиционные взгляды, —
это успешные стартапы. Основатели и первые сотрудники почти всегда
мыслят независимо, иначе бизнес не будет успешным. Но людей с
ограниченным мышлением намного больше, чем с независимым, поэтому по
мере роста компании изначальный дух независимости неизбежно
ослабевает. Это вызывает множество проблем (помимо, очевидно, того,
что компания становится отстойной). И в какой-то момент основатели
обнаруживают, что общаются с другими предпринимателями более свободно,
чем с собственными сотрудниками. \newline

К счастью, не обязательно проводить все свое время с независимо
мыслящими людьми. Достаточно одного-двух, с которыми можно регулярно
разговаривать. И когда вы их находите, обычно оказывается, что они так
же хотят общения, как и вы: вы им тоже нужны. Хотя у университетов
больше нет такой монополии на образование, как раньше, хорошие
университеты по-прежнему дают отличную возможность встретить людей с
независимым мышлением. Большинство студентов по-прежнему
придерживаются общепринятых взглядов, но все равно можно найти группы
людей с независимым складом ума. \newline

Это работает и в обратном направлении: стоит не только культивировать
небольшую группу независимо мыслящих друзей, но и знакомиться с самыми
разными людьми. Так вы уменьшите влияние конкретной группы и, будучи
частью нескольких разных миров, сможете импортировать идеи из одного в
другой. \newline

Говоря про разные типы людей, я имею в виду не социодемографические
характеристики. Чтобы метод сработал, это должны быть люди с разными
типами мышления. Конечно, отличная идея — ездить в другие страны, но,
вероятно, даже где-то за углом вы найдете людей, которые мыслят иначе.
Когда я встречаю человека, который разбирается в чем-то необычном (а
это практически каждый, если копнуть достаточно глубоко), я пытаюсь
узнать то, что знает он и не знают другие. Здесь почти всегда бывают
сюрпризы. Это хороший способ завязать разговор с незнакомцем, но я
делаю это не ради беседы. Я действительно хочу что-то узнать. \newline

Вы можете расширить источники влияния как во времени, так и в
пространстве, читая исторические книги. Когда я изучаю историю, я
делаю это не только для того, чтобы узнать, что произошло, но чтобы
попытаться проникнуть в головы людей, которые жили в прошлом. Как они
смотрели на вещи? Это сложно сделать, но приложенные усилия того стоят
по той же причине, по которой стоит путешествовать в далекие страны. \newline

Вы также можете принять более конкретные меры, чтобы не принимать
избитые истины автоматически. Самый общий подход — культивация
скептицизма. Когда вы слышите чье-то высказывание, остановитесь и
спросите себя: «Правда ли это?» Не говорите это вслух. Не стоит
возлагать на каждого собеседника бремя доказывания своих слов. Скорее
советую взять на себя бремя оценки сказанного ими. \newline

Относитесь к этому как к головоломке. Вы знаете, что некоторые
общепринятые идеи позже окажутся ошибочными. Посмотрим, сможете ли вы
угадать, какие именно. Конечная цель не в том, чтобы найти недостатки
в сказанном, а в том, чтобы найти новые идеи, которые скрывались за
ошибочными. Так что это увлекательный квест по поиску новинок, а не
скучный протокол интеллектуальной гигиены. И вы, когда начнете
спрашивать «Правда ли это?», будете удивлены, как часто ответ не будет
немедленным «да». \newline

В более общем плане цель состоит в том, чтобы не допустить ничего
непроверенного в свою голову. И идеи не всегда приходят в голову в
форме утверждений. Порой что-то неочевидное, неявное оказывает на нас
сильное влияние. Как вообще это заметить? Стоя в стороне и наблюдая,
как другие люди придумывают идеи. \newline

Отдалившись на достаточное расстояние, вы увидите, что идеи расходятся
по группам людей, как волны. Самые очевидные из них касаются моды: вы
замечаете, что сначала несколько человек носят определенный тип
рубашки, а затем все больше и больше, и в конце концов половина
окружающих людей носят одинаковые рубашки. Возможно, вам все равно,
что носить, но есть и интеллектуальная мода, и определенно не стоит ей
следовать. Не только потому, что важна власть над своими мыслями, но и
потому, что немодные идеи непропорционально часто приводят к
интересным результатам. Лучшее место для поиска неизведанных идей —
там, где не ищет никто другой. \newline

Чтобы выйти за рамки этого общего совета, нужно взглянуть на
внутреннюю структуру независимого мышления — на отдельные мышцы,
которые нужно тренировать, фигурально выражаясь. Мне кажется, что она
состоит из трех компонентов: требовательность и щепетильность к
правде, сопротивление навязываемым мыслям и любопытство. \newline

Проявлять требовательность к правде — это не просто не верить в ложные
утверждения. Это подразумевает осторожность в степени веры.
Большинство людей склонны впадать в крайности: маловероятное
становится невозможным, а вероятное становится несомненным. Людям с
независимым мышлением это кажется крайне неосмотрительным. У них в
голове может быть что угодно — от весьма спекулятивных гипотез до
(очевидных) тавтологий, — но по предметам, которые их волнуют, все
должно быть разложено по полочкам с тщательно продуманной степенью
веры. \newline

Таким образом, независимо мыслящие люди избегают идеологий, требующих
одновременно принять целый набор убеждений и относиться к ним как к
символам веры. Человеку с независимым складом ума это может показаться
отвратительным, как гурману — кусок сэндвича с разными ингредиентами
неопределенного возраста и происхождения. \newline

Без этой щепетильности к правде невозможно быть по-настоящему
независимыми. Недостаточно сопротивляться навязываемым взглядам.
Немало людей отвергают общепринятые идеи, но заменяют их произвольными
теориями заговора. И поскольку эти теории и создавались для того,
чтобы захватить их разум, эти люди в конечном итоге оказываются еще
менее независимыми, потому что подчиняются гораздо более
требовательному хозяину, чем простая традиционность. \newline

Можно ли стать более требовательными к правде? Думаю, можно. По моему
опыту, если вы просто думаете о чем-то, к чему относитесь щепетильно,
это уже усиливает эту щепетильность. Если так, то это одна из тех
редких добродетелей, которые можно развить просто по желанию. И если
это похоже на другие формы требовательности, то следует поощрять ее в
детях. Я определенно получил сильную дозу от отца. \newline

Второй компонент независимого мышления — сопротивление навязываемому
образу мыслей — самый заметный из трех. Но даже его люди часто
понимают неправильно. Большая ошибка — рассматривать его как простое
отрицание. Язык, который мы используем, усиливает это представление:
вы против условностей; вам не важно, что думают другие. Но это не
просто иммунитет. Люди с сильным независимым образом мыслей не желают,
чтобы им указывали, как думать, и это их положительная сила. Это не
просто скептицизм, а активное восхищение идеями, которые опровергают
мнение большинства: чем противоречивее, тем лучше. \newline

Некоторые новаторские идеи в свое время казались почти розыгрышами.
Подумайте, как часто вашей первой реакцией на новую идею бывает смех.
Я думаю, это не потому, что новые идеи смешны сами по себе, а потому,
что новизну и юмор объединяет определенная неожиданность. Они не
идентичны, но достаточно близки, так что существует определенная
корреляция между чувством юмора и независимым мышлением — точно так
же, как между отсутствием чувства юмора и традиционным мышлением. \newline

Я не думаю, что можно научиться сильнее сопротивляться навязыванию
каких-то взглядов. Кажется, этот компонент больше других обусловлен
врожденными особенностями: люди, обладающие этим качеством во взрослом
возрасте, обычно проявляли видимые признаки и в детстве. Но если мы не
можем сопротивление усилить, то можем по крайней мере укрепить его,
окружив себя людьми с независимым мышлением. \newline

Третий компонент независимого мышления, любопытство, возможно,
интереснее всего. Ведь краткий ответ на вопрос, откуда берутся новые
идеи, — именно любопытство. Это то, что люди обычно чувствуют перед
тем, как их посещает идея. \newline

По моему опыту, независимость взглядов и любопытство идеально
прогнозируют друг друга. Все мои знакомые с независимым складом ума
глубоко любопытны, а вот те, кто придерживается традиционных взглядов,
нет. Кроме, что любопытно, детей. Все маленькие дети любопытны.
Возможно, причина в том, что даже люди с традиционным мышлением должны
вначале проявить любопытство, чтобы понять, что такое общепринятые
нормы. В то время как независимые мыслители никак не могут
удовлетворить свое любопытство и продолжают есть даже после того, как
насытились. \newline

Три компонента независимого мышления работают согласованно:
требовательность к правде и сопротивление навязываемому образу мыслей
освобождают место в мозге, а любопытство находит новые идеи, чтобы его
заполнить. \newline

Интересно, что эти три компонента могут заменять друг друга почти так
же, как и мышцы. Если вы достаточно щепетильны в отношении правды, вам
не нужно слишком сопротивляться навязываемому мышлению, потому что
сама по себе щепетильность создаст достаточные пробелы в ваших
знаниях. И любой из них может компенсировать любопытство, потому что,
если вы освободите в своем мозге достаточно места, дискомфорт от
образовавшегося вакуума усилит ваше любопытство. Или любопытство может
компенсировать остальное: если вы достаточно любопытны, вам не нужно
расчищать пространство в мозге, потому что новые идеи, которые вы
обнаружите, вытеснят принятые по умолчанию. \newline

Поскольку компоненты независимого мышления настолько взаимозаменяемы,
каждый из них может быть развит у вас в разной степени — а результат
будет тот же самый. Поэтому не существует единой модели независимого
мышления. Некоторые люди с независимым складом ума открыто проявляют
бунтарство, а другие тихо любопытны. \newline

Есть ли способ развить любопытство? Для начала — избегать ситуаций,
которые его подавляют. Насколько ваша работа стимулирует любопытство?
Если ответ «немного», возможно, стоит что-то изменить. \newline

Самый важный активный шаг, который вы можете предпринять для развития
любопытства, — это, вероятно, поиск тем, которые его возбуждают.
Немногие взрослые одинаково любопытны во всем, и, похоже, мы не можем
выбирать интересующие нас темы. Так что найти их — ваша задача. Или
придумать, если нужно. \newline

Еще один способ усилить любопытство — потакать ему, исследуя то, что
вас интересует. В этом отношении любопытство отличается от большинства
других потребностей: когда вы ему потворствуете, это его, как правило,
усиливает, а не насыщает. Вопросы приводят к новым вопросам. \newline

Любопытство кажется более индивидуальным, чем щепетильность к правде
или сопротивление навязанным взглядам. В какой бы степени ни были
развиты последние два компонента, они обычно довольно общие, в то
время как разные люди могут интересоваться самыми разными вещами. Так
что, возможно, любопытство тут — главное. Возможно, если ваша цель —
открыть для себя новые идеи, вашим девизом должно быть не столько
«делай то, что любишь», сколько «делай то, что тебе любопытно».

\end{document}
