\documentclass[ebook,12pt,oneside,openany]{memoir}
\usepackage[utf8x]{inputenc} \usepackage[russian]{babel}
\usepackage[papersize={90mm,120mm}, margin=2mm]{geometry}
\sloppy
\usepackage{url} \title{Слепота скуки} \author{Пол Грэм} \date{}
\begin{document}
\maketitle

Восхитительные идеи стартапов валяются неиспользованные прямо у нас
под носом. Одна из причин, по которой мы не видим их, это феномен,
который я называю слепота скуки. Schlep было изначально словом из
еврейского языка, но стало общеупотребительным в США. Оно означает
скучную, противную задачу.

Никто не любит скучных задач, но хакеры особенно не любят их.
Большинство хакеров, которые начали стартапы, мечтали, что они могут
делать это просто за счет написания умных программ, выкладывания их
где-нибудь на сервере, и смотреть как валятся деньги – без
необходимости разговаривать с пользователями, или вести переговоры с
другими компаниями, или иметь дело с чужим нерабочим кодом. Может это
возможно, но я такого не видел.

Одна из многих вещей, которые мы делаем в Y Combinator, это учим
хакеров неизбежности скуки. Нет, вы не можете начинать стартап просто
за счет написания кода. Я помню себя, проходящего через осознание
этого. Этот момент был в 1995 году, когда я продолжал пытаться убедить
себя, что я могу запустить компанию просто за счет написания кода. Но
вскоре я узнал из опыта, что скучные задачи не только неизбежны, но
являются составляющими большую частью бизнеса. Компания определяется
скучными задачами, которые она на себя примет.

И скучные задачи должны рассматриваться так же, как когда вы имеете
дело с холодным бассейном: просто прыгайте в него. Что не значит
сказать, что вам следует искать неприятную работу как таковую, но вы
никогда не должны уклоняться от нее, если она находится на пути к
чему-то возвышенному.

Самая опасная вещь в нашей неприязни к скучным задачам это то, что
большая часть ее подсознательная. Ваше подсознание даже не позволить
вам увидеть идеи, которые включают в себя мучительные скучные задачи.
Это слепота скуки.

Этот феномен не ограничивается стартапами. Большинство людей не
сознательно решают быть не в такой же хорошей физической форме, как
олимпийские атлеты, например. Их подсознание решает за них избегать
связанной с этим работы.

Самый яркий пример слепоты скуки, который я знаю, это Stripe, или
вернее, идея Stripe. Уже более десяти лет, каждый хакер, который
когда-либо совершал процесс оплаты платежей в Интернете, знает, какой
это болезненный опыт. Тысячи людей должны были знать об этой проблеме.
И все же, когда они начинали стартапы, они решили делать сайты
рецептов, или агрегаторы для местных событий. Почему? Зачем работать
над проблемой, которая мало кого заботит и за которую никто не
заплатит, когда вы можете исправить один из самых важных компонентов
мировой инфраструктуры? Потому что слепота скуки предостерегает людей
даже от рассмотрения идеи наладить платежи.

Наверное, никто из обратившихся в Y Combinator, чтобы работать над
сайтом рецептов, начав с вопроса «нам следует исправить платежи, или
сделать сайт рецептов?» выбрал бы сайт рецептов. Хотя идея исправления
платежей была прямо на виду, они никогда не видели ее, потому что их
подсознательная часть разума съежилась от связанных с ней сложностей.
Вам придется заключать сделки с банками. Как вы сделаете это? Кроме
того, вы переводите деньги, так что вы будете иметь дело с
мошенничеством, и люди попытаются взломать ваши сервера. Кроме того,
наверное, есть всякие правила, которые нужно соблюдать. Это гораздо
более пугающе - начинать стартап вроде этого, чем сайт рецептов.

Этот страх делает смелые идеи вдвойне ценными. В дополнение к присущей
им ценности, они как недооцененные акции, в том смысле, что на них
меньший спрос среди основателей. Если вы выбираете амбициозную идею,
вы будете иметь меньшую конкуренцию, потому что остальные будут
отпугнуты связанными с ней сложностями. (Это также верно для запуска
стартапа.)

Как преодолеть слепоту скучных задач? Честно говоря, самым ценным
противоядием к слепоте скучных задач является, наверное,
игнорирование. Большинство успешных основателей, вероятно, сказали бы,
что если бы они знали, когда они начинал свою компанию, о трудностях,
которые они должны были бы преодолеть, они могли никогда не начать.
Может быть это одна из причин, по которой у большинства всех успешных
стартапов так часто молодые основатели.

На практике, основатели растут вместе с проблемами. Но похоже никто не
мог предвидеть их, даже старый, более опытный основатель. Таким
образом, причина, по которой молодые основатели имеют преимущество в
том, что они делают две ошибки, которые уравновешивают друг друга. Они
не знают, насколько они могу вырасти, но они также не знают, насколько
им придется вырасти. Старые основатели могут сделать только первую
ошибку.

Все же невежество не может решить всё. Некоторые идеи настолько
очевидно повлекут за собой настораживающие скучные задачи, что любой
может увидеть их. Как вы видите подобные идеи? Уловка, которую я
рекомендую, это вынести себя из общей картины. Вместо того, чтобы
спрашивать «какую проблему я должен решить?», спросите «какую проблему
я бы хотел, чтобы кто-нибудь другой решил для меня?». Если бы кто-то,
кто проходил обработку платежей до Stripe попытался спросить такое,
Stripe был бы первой из того, что они пожелали бы.

Теперь слишком поздно для того, чтобы стать Stripe, но в мире остается
чрезвычайно много несделанного, если вы знаете, как это увидеть.

\end{document}
