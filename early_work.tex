\documentclass[ebook,12pt,oneside,openany]{memoir}
\usepackage[utf8x]{inputenc} \usepackage[russian]{babel}
\usepackage[papersize={90mm,120mm}, margin=2mm]{geometry}
\sloppy
\usepackage{url} \title{Ранние работы} \author{Пол Грэм} \date{}
\begin{document}
\maketitle

Что мешает нам делать работу на «отлично»? Прежде всего — страх
облажаться. И это оправданный, вполне естественный страх. Даже
создателям многих великих проектов на начальной стадии казалось, что
они делают что-то отстойное. Ощутимый результат всегда требует усилий,
и в первую очередь — усилия переступить через себя, когда кажется, что
всё очень плохо. Но многие не справляются с этой задачей. Большинство
даже не доходит до той стадии, когда уже есть чего стыдиться, — что уж
говорить о дальнейшей работе. Все мы боимся начать.

А что, если б мы могли просто выключить этот страх налажать?
Представьте, сколько крутых вещей можно было бы наворотить.

Встроен ли в нас механизм, который отключает тревогу? Думаю, да. Корни
страха проросли ещё не так глубоко, чтобы нельзя было от них
избавиться.

Создание чего-то нового — далеко не самая привычная вещь для людей.
Человечество всегда что-то изобретало, но вплоть до последних
нескольких веков технологический прогресс шёл так медленно, что был
практически незаметен. История не требовала от нас устойчивых
алгоритмов для оценки свежих идей, и поэтому мы никогда и не
стремились обзавестись ими.

Цивилизация просто не накопила достаточно опыта работы с сырыми, но
потенциально прорывными проектами. Мы не знаем, как на них
реагировать. Поэтому оцениваем их по тем же лекалам, что и проекты,
которые уже близки к завершению или изначально менее перспективны. Мы
просто не можем понять, что прорывные идеи требуют особого к себе
отношения.

Но обратные примеры уже есть. И именно они придают мне уверенность в
том, что мы всё-таки способны должным образом относиться к
инновационным проектам. Например, в этом смысле далеко вперёд ушла
Кремниевая долина. Там никто не станет игнорировать приезжего
незнакомца, который возится с какой-то странной идеей, как это
произошло бы у него на родине. В долине понимают, чем это чревато.

Чтобы правильно оценить новую идею, нужно отнестись к ней как к вызову
вашему воображению. Не просто занизить ожидания на начальном этапе, а
полностью поменять полюсы восприятия: вместо поиска причин, по которым
проект не выгорит, постараться найти способы заставить его работать.
Именно так я поступаю, когда встречаюсь с людьми, предлагающими новые
идеи. И у меня это неплохо получается (да, я много практиковался).
Управлять Y Combinator — значит окунуться в мир нелепых идей от
абсолютно незнакомых людей. Каждые полгода тебя закидывают тысячами
предложений, которые приходится сортировать. При этом принцип
степенного распределения никто не отменял — так что будет крайне
болезненно, если вы в итоге пропустите ту самую иголку в стоге сена.
Оптимизм становится необходимым качеством.

И я надеюсь, что со временем таким оптимизмом заразится настолько
много людей, что он станет общепринятым алгоритмом, а не просто хитрым
трюком, о котором знает лишь пара экспертов. В конце концов, это
довольно прибыльная уловка, а значит, она может быстро стать
популярной.

Конечно, отсутствие опыта — не единственная причина, по которой люди
слишком сурово относятся к сырым (но, возможно, многообещающим)
проектам. Они это делают, чтобы казаться умнее. И в области больших
рисков, как в случае со стартапами, отвергая новую идею, ты с высокой
вероятностью окажешься прав. Правда, не в том случае, когда пытаешься
выстраивать ожидания, исходя из итоговой прибыли.

Есть и куда более удручающая причина, по которой новые идеи встречают
отказом. Когда ты берёшься за что-то амбициозное, многие люди вокруг
(сознательно или бессознательно) желают тебе провала. Их корёжит от
одной мысли, что кто-то может преуспеть и оказаться выше. В некоторых
странах это не просто порок отдельных личностей, а часть национальной
культуры.

Воздержусь от заявлений, что в Кремниевой долине люди подавляют такие
дурные порывы в силу своих высоких моральных качеств. [1] Люди из
долины чаще готовы поверить в ваш успех, надеясь вырасти вместе с
вами. У инвесторов этот стимул проявляется особенно чётко — они хотят,
чтобы вы преуспели, потому что рассчитывают в результате стать богаче.
Но и вне мира сделок можно встретить людей, которые будут
заинтересованы в вашем успехе. Как минимум, они потом смогут всем
рассказывать, что знали вас ещё совсем зелёным.

Да, возможно, та поддержка, которую новые проекты находят в Кремниевой
долине, изначально была основана на личной выгоде каждого. Но со
временем всё это трансформировалось в неподдельную доброжелательность.
Там уже столько лет поддерживают стартапы, что в итоге благосклонность
к ним стала культурной нормой. Какого-то другого отношения к новым
проектам в Долине уже и не ожидаешь.

Нельзя исключать, что люди в Кремниевой долине слишком оптимистичны.
Да, возможно, их легко может одурачить любой самозванец, как обычно
пишут журналисты (а они далеко не оптимисты, мы знаем). Вот только в
прессе обычно фигурирует один и тот же список «самозванцев». И список
этот мало того что короткий, так еще и полон оговорок. [2]. А если
взять в качестве критерия выручку, то, кажется, оптимизм Кремниевой
долины очень даже обоснован. И раз уж такой подход работает, не
сомневайтесь — он получит широкое распространение.

Конечно, новые идеи — это не только про стартапы. Страх облажаться
сдерживает людей из самых разных сфер. Но опыт Долины показывает,
насколько быстро человечество может взять за правило поддержку всего
нового. А это, в свою очередь, доказывает, что привычка отвергать
инновации засела в нас ещё не настолько глубоко, чтобы нельзя было от
неё избавиться.


К несчастью, когда делаешь что-то новое, обычно сталкиваешься с силой,
куда более могущественной, чем скепсис других людей. Имя этой силе —
твой собственный скепсис. К своим проектам на ранней стадии люди и
сами относятся чересчур строго. Как этого избежать?

Здесь-то и кроется главная проблема. Нельзя полностью избавиться от
внутреннего страха сделать что-то отстойное. В конце концов, именно
это чувство побуждает нас добиваться хорошего результата. Страх нужно
всего лишь приглушить на время, как болеутоляющие временно снимают
боль.

Благо, человечество уже разработало пару техник на этот случай. Годфри
Харди, например, предлагает два способа в своей «Апологии математика»:



Хорошая работа делается отнюдь не «скромными» людьми. Одна из первых
обязанностей профессора, преподающего любую науку, состоит в том,
чтобы немного преувеличить важность своего предмета и своего участия в
его развитии.

Когда переоцениваешь значимость собственной работы и при этом слишком
сурово оцениваешь свои первые результаты, в итоге появляется шанс
достичь баланса. Допустим, у вас есть задача, выполнив которую можно
продвинуться на 20\% к цели в виде \$100. Но вам кажется, что этот шаг
даст вам лишь 10\% прогресса, а конечная прибыль при этом составит
\$200. В итоге факторы перекрывают друг друга и мы получаем корректную
оценку затрат и ожиданий даже при изначально неверных составляющих.

Если верить Харди, немного самоуверенности ещё никому не мешало. Я и
сам много раз убеждался, что самые успешные люди частенько грешат
апломбом. На первый взгляд всё это звучит неубедительно, ведь всем
ясно — нет ничего лучше, чем адекватная оценка собственных
способностей. Разве заблуждение может оказаться преимуществом? В
реальности всё обстоит именно так. Небольшой избыток самоуверенности
действует как защита — и от скептицизма других, и от вашего
собственного скепсиса.

Уметь закрывать глаза на незначительные недостатки — тоже важно. Если
вы по природе снисходительный судья, то не страшно ошибиться в оценках
сырого проекта по критериям готового. Сомневаюсь, что возможно
культивировать такой подход, но чисто эмпирически это имеет все шансы
на успех. Особенно когда речь идёт о молодых людях.

Страх первого этапа можно также преодолеть с помощью коллектива —
окружить себя правильными людьми и встать вместе против ветра,
превратив его в вихрь идей. Но мало собрать команду, которая будет
всегда вас поддерживать. Такое отношение быстро обесценивается. Вам
нужны те, кто быстро отличит гадкого утенка от прекрасного юного
лебедя. Как правило, на такое способны лишь люди, работающие над
собственными проектами. Именно поэтому университетские кафедры и
исследовательские лаборатории работают так плодотворно. Разница в том,
что вам не нужны для этого институты. Мыслящие люди собираются вместе
естественным путем. Ускорить этот процесс можно, стараясь наладить
контакт с подобными вам — авторами новых проектов и идей.

Есть и совершенно особенный тип коллег — учителя. Их задача как раз
заключается в том, чтобы разглядеть многообещающую идею в зародыше и
заставить довести её до конца. К сожалению, мало кто из учителей
хорошо справляется с этой работой. Но если вам выпал шанс встретить
настоящего наставника — берите его в свою команду. [3]

Кому-то может помочь обычная дисциплина: нужно просто сказать себе,
что этот чертов начальный этап не может длиться вечно, — достаточно
преодолеть его и не отчаиваться раньше времени. Но, как и в случае с
другими советами из серии «нужно просто», всё намного сложнее, чем
выглядит на словах. И чем мы старше, тем сложнее — ведь наши ожидания
только растут. Но у взрослых людей здесь есть преимущество: они через
всё это уже проходили, и не раз.

Возможно, полезным будет фокусироваться не на сегодняшнем дне, а на
скорости изменений. Когда видишь, что результат трудов становится
лучше, то уже не так переживаешь из-за неудач на первых порах.
Конечно, чем быстрее темпы улучшений — тем лучше. Поэтому, если уж и
начинать работать над чем-то с нуля, хорошо бы посвящать этому
побольше времени. И здесь преимущество уже у молодых: свободного
времени в юности, как правило, много.

Есть ещё довольно распространённый психологический трюк — поначалу
отнестись к новой работе как к чему-то не слишком важному. Например,
писать картину, говоря, что это просто набросок. Или засесть за
разработку ПО, но действовать так, будто быстро придумываешь простой
хакерский трюк. Тогда и ожидания от первых итогов у вас будут
меньшими, и требования к ним — гораздо более низкими. А вот когда дело
пойдёт — вы сможете превратить его в нечто большее. [4]

Неплохо в таком случае обходиться минимальными средствами, с помощью
которых можно работать быстро и не тратить всю энергию в процессе.
Например, гораздо проще убедить себя в том, что делаешь набросок, если
рисуешь в скетчбуке, а не ваяешь скульптуру. Да и на примерный
результат можно полюбоваться куда раньше. [5] [6]

Будет легче решиться на рискованный проект, если поставить перед собой
цель просто научиться чему-то, а не выполнить боевую задачу. Даже если
в конце всё накроется медным тазом, у вас останется ценный опыт. [7]

Лично для меня лучшей мотивацией остается любопытство. Мне нравится
пробовать всё новое ради интереса. Даже Y Combinator начинался с
этого. И когда я создавал язык программирования Bel — именно
любопытство заставляло меня работать. Я так долго копался в различных
диалектах языка Lisp, что мне просто стало интересно, в чём же его
истинная природа. Куда меня приведет чистый аксиоматический метод?

Конечно, все эти психологические ловушки и попытки избежать
разочарования от первых результатов проекта выглядят несколько
странно. Мы всё время пытаемся обвести себя вокруг пальца и уловками
заставить поверить во что-то эфемерное. Но это эфемерное в итоге
оказывается реальностью. Никуда не годная на первый взгляд версия
амбициозного проекта — это на самом деле едва ли не самое ценное, что
есть в работе. Поэтому лучшим решением, возможно, будет просто
научиться видеть это.

Можно изучать биографии людей, сделавших что-то стоящее для
человечества. О чём они думали в самом начале пути? С чего начинали?
Порой сложно найти ответы на эти вопросы, так как даже великие часто
стыдятся своих ранних работ и не пытаются показывать их публике (и в
этом они тоже ошибаются). Но когда стараешься восстановить точный
карьерный путь известных творцов, часто замечаешь — их первые шаги на
этом пути были лишь робкой поступью. [8]

Не исключено, что изучив достаточно подобных биографий, вы станете
более справедливы к самим себе. Это даёт определенный иммунитет и от
скепсиса других, и от собственных страхов. Начинаешь ценить первые
труды по достоинству.

Забавно, но чтобы отучить себя от категоричных оценок, нужно понять,
что наше отношение к сырым проектам — само по себе тоже сырое.
Пытаться измерить всё одним мерилом — это и есть недоработанная
«демоверсия» критического мышления. Но уже сейчас человечество
эволюционирует в этом отношении, и очень скоро нам воздастся.

\end{document}
