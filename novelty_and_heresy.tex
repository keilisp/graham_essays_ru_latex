\documentclass[ebook,12pt,oneside,openany]{memoir}
\usepackage[utf8x]{inputenc} \usepackage[russian]{babel}
\usepackage[papersize={90mm,120mm}, margin=2mm]{geometry}
\sloppy
\usepackage{url}

\title{Да здравствует ересь!} \author{Пол Грэм} \date{}

\begin{document}
\maketitle

Если вы открываете что-то новое, существует значительный шанс, что вас
обвинят в той или иной форме ереси. \newline 

Чтобы совершать открытия, нужно работать над идеями, которые хороши,
но не очевидны: если идея явно хорошая, другие люди, вероятно, уже
взялись за нее. Один из распространенных способов найти хорошую
неочевидную идею — поискать в тени ошибочного представления, к
которому люди очень привязаны. Но все, что вы обнаружите, работая над
такой идеей, будет противоречить всеобщему убеждению, которое скрывало
ее ценность. И люди, приверженные этой ошибочной аксиоме, выплеснут на
вас свое недовольство. Яркие примеры — Галилей и Дарвин, но в
принципе, видимо, такое сопротивление новым идеям вообще неизбежно. \newline

Поэтому для организации или общества особенно опасно иметь культуру, в
которой травят за глупости. Подавляя еретические идеи, вы не просто
мешаете людям бороться с ошибочными аксиомами, которые пытаетесь
защитить. Вы также подавляете любые идеи, что косвенно подразумевает
их ложность. \newline

Вокруг каждой заветной ошибочной аксиомы есть мертвая зона неизученных
идей. И чем нелепее аксиома, тем больше ее мертвая зона. \newline

У этого явления есть и положительная сторона. Один из способов найти
новые идеи — искать глупости. Если посмотреть на этот вопрос таким
образом, удручающе обширные мертвые зоны вокруг ошибочных
представлений превращаются в невероятные прииски новых идей. 

\end{document}
