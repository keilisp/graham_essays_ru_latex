\documentclass[ebook,12pt,oneside,openany]{memoir}
\usepackage[utf8x]{inputenc} \usepackage[russian]{babel}
\usepackage[papersize={90mm,120mm}, margin=2mm]{geometry}
\sloppy
\usepackage{url} \title{Ошибки, которые убивают стартапы} \author{Пол
  Грэм} \date{}
\begin{document}
\maketitle

Во время дискуссии после одного из моих выступлений кто-то спросил по
каким причинам проваливаются стартапы. Я был в замешательстве, и,
немного поразмыслив, понял, насколько это каверзный вопрос. Все равно,
что спросить как сделать стартап успешным – если избегать того, что
неминуемо ведет к провалу, очевидно, вас ждет успех. Это слишком
сложный вопрос, чтобы ответить на него без подготовки.

Позже я понял, что полезно рассмотреть проблему с этой точки зрения.
Если у вас есть список того, что не следует делать, вы получите рецепт
успешной работы, просто поступая наоборот. Гораздо проще поймать себя
на том, что делаешь что-то неправильно, чем постоянно держать себя в
руках и делать только то, что нужно. Впрочем, это не полный список
причин неудач, а только те параметры, которые вы можете
контролировать. Некоторые другие факторы вы контролировать не можете,
например, отсутствие необходимых навыков и невезение.

В некотором смысле, стартапы убивает одна-единственная ошибка: вы
делаете не то, чего хочет целевая аудитория. Если вы сможете
предложить то, что будет востребовано целевой аудиторией, вероятно, вы
добьетесь успеха независимо от того, что вы делаете или чего не
делаете помимо этого. А если вы предлагаете не то, чего от вас ждут,
то вы обречены на неудачу независимо от всего остального. Далее я
привожу 18 причин, по которым стартапы оказываются совсем не тем, чего
хочет целевая аудитория.

Единственный основатель

Вы никогда не замечали, как мало успешных стартапов было основано
одним человеком? Даже те компании, у которых, как вы думаете, всего
один основатель, подобно Oracle, создавались несколькими учредителями.
Вряд ли это простое совпадение. Почему же единственный основатель –
это недостаток? Во-первых, это говорит об отсутствии доверия.
Вероятно, единственный основатель не смог убедить ни одного из своих
друзей создать компанию вместе с ним. Это наводит на размышления,
поскольку никто не знает человека лучше его друзей. Но даже если все
друзья основателя ошибались, и у компании неплохие перспективы, все
равно отсутствие нескольких основателей – это недостаток. Стартап –
слишком тяжелое дело для одного человека. Даже если вы можете сделать
всю работу самостоятельно, нужны коллеги, вместе с которыми вы искали
бы новые идеи, и которые могли бы отговорить вас от принятия глупых
решений и подбодрить вас, когда дела идут не лучшим образом. Последняя
причина, вероятно, важнее всего. Запуск может оказаться настолько
тяжелым, что не каждый справится в одиночку. Когда у компании
несколько основателей, их связывает дух единства, который может помочь
справиться с любыми трудностями. Каждый думает: ”я не могу подвести
своих друзей”. Это одна из самых ярких черт человеческой природы,
которой не хватает компаниям с единственным основателем.

Неудачное расположение

Стартапы удаются в одних местах и не удаются в других. Силиконовая
долина остается лидером по числу удачных стартапов, за ней идут
Бостон, Сиэтл, Остин, Денвер и Нью-Йорк. И это почти весь список,
причем в Нью-Йорке количество стартапов составляет всего-навсего
двадцатую часть от их количества в Силиконовой долине. В Хьюстоне,
Чикаго или Детройте стартапы – настолько редкое явление, что нет
смысла даже заниматься подсчетами. Почему возникает такой разрыв? По
тем же причинам, что и в любой другой индустрии. Назовите шестой по
значимости город в мире моды в США? Шестой по значимости город в
нефтяном бизнесе или в мире финансов? А в издательском деле? Неважно,
что это за города, но они настолько отдалены от центра, нельзя даже
называть их значимыми. Было бы интересно выяснить, почему некоторые
города становятся центрами успешных стартапов, одна причину можно
назвать точно: в таких городах сконцентрированы лидеры. Действуют
более высокие стандарты, люди вокруг небезучастны к тому, что вы
делаете. Потенциальные работники живут недалеко, равно как и
вспомогательные отрасли. Вы регулярно взаимодействуете с сотрудниками
других компаний. Никто не скажет, как именно сочетание всех этих
факторов влияет на успехи стартапов в Силиконовой долине и их неудачи
в Детройте, но их очевидное влияние подтверждается количеством
успешных стартапов в каждом из этих регионов.

Узкоспециальная ниша

Большинство предпринимателей, которые обращаются в Y Combinator,
сталкиваются с одной и той же проблемой: они выбирают узкоспециальную
нишу в надежде избежать конкуренции. Если понаблюдать за тем, как
маленькие дети играют в спортивные игры, станет заметно, что дети
младше определенного возраста боятся мяча. Когда мяч оказывается рядом
с ними, они инстинктивно пытаются избежать контакта. Когда мне было
восемь лет, я не мог похвастаться большими успехами в бейсболе, потому
что каждый раз, когда отбитый мяч летел в мою сторону, я закрывал
глаза и поднимал перчатку для защиты, а не для того, чтобы поймать
мяч. Выбор малопривлекательных проектов можно сравнить с моей игрой в
восьмилетнем возрасте. Если вы занимаетесь чем-то стоящим, у вас будут
конкуренты, и вы должны быть к этому готовы. Избежать конкуренции вы
сможете только в одном случае – если у вас не будет хороших идей. Мне
кажется, что подобное стремление избежать серьезных проблем в
значительной степени продиктовано подсознанием. Не думаю, что люди
думают о чем-то грандиозном, но занимаются чем-то менее значительным,
только потому, что это не так рискованно. Ваше подсознание не позволит
вам даже думать о чем-то большом. Поэтому выход – это ставить большие
цели, абстрагируясь от самого себя. Какие серьезные задачи могут быть
в чьем-то стартапе?

Вторичная идея

В большинстве заявлений, которые мы получаем, копируются схемы
развития уже существующих успешных компаний. Это не самый лучший
источник идей. Рассматривая успешные проекты, становится очевидно, что
лишь малая доля занималась копированием запущенных ранее проектов.
Откуда берутся идеи? Чаще всего основатели выявляют некую проблему,
требующую решения. Мы создавали компанию для разработки программного
обеспечения для интернет-магазинов. Когда мы начинали, ничего
подобного просто-напросто не было. Несколько сайтов, на которых можно
было делать заказы, создавались вручную за большие деньги. Мы знали,
что если интернет-магазины начнут развиваться, их создания потребуется
специальное ПО и решили написать его сами. Ничего особенного мы не
придумали. Есть мнение, что лучше всего решаются проблемы,
затрагивающие вас лично. У Apple все получилось, потому что Стиву
Возняку был нужен компьютер, Ларри и Сергей создали Google, потому что
не могли что-то найти в сети, а сервис Hotmail возник в результате
того, что Сабир Бхатия и Джек Смит не могли обмениваться электронными
письмами на работе. Поэтому вместо того, чтобы копировать Facebook с
какими-то изменениями, от которых создатели Facebook имели все
основания отказаться, начните поиск идей в других направлениях. Вместо
того чтобы ориентироваться на другие компании и возвращаться к
проблемам, которые они успешно решили, ищите другие проблемы и
старайтесь придумать компанию, которая сможет их решить. Чем
недовольны люди? Чего не хватает лично вам?

Упрямство

В некоторых областях путь к успеху лежит через четкое понимание того,
чего вы хотите достичь, независимо от трудностей на пути к
поставленной цели. Стартап – совсем другое дело. Четко следовать
намеченному курсу необходимо, когда вы ставите целью выиграть
олимпийскую медаль, и задачи ясно определены. У стартапов гораздо
больше общего с научными исследованиями, когда вы идете по следу, не
зная, куда он ведет. Поэтому держитесь за ваш исходный план, ведь он
может быть ошибочным. В большинстве успешных стартапов в результате
получалось не совсем то, что планировалось изначально, а иногда
результат настолько отличается от замысла, что даже не похоже, что
речь идет об одной и той же компании. Вы должны быть готовы
рассмотреть новые и лучшие идеи, если они появляются. Самое сложное
при этом – найти в себе силы отказаться от первоначального замысла. Но
открытость новым идеям тоже следует контролировать. Смена направлений
каждую неделю также неминуемо приведет к неудаче. Существует ли
какая-то внешняя проверка, на которую вы сможете положиться? Например,
проанализировать, связаны ли ваши идеи друг с другом. Если в каждом
новом проекте вы можете использовать что-то из предыдущих, скорее
всего, вы на правильном пути. Если же каждый раз вам приходится
начинать с нуля, ни о чем хорошем это не говорит. К счастью, всегда
есть, к кому обратиться за советом: не забывайте о целевой аудитории.
Если ваши пользователи заинтересованы в новом направлении развития
стартапа – стоит попробовать.

Плохие программисты

Я забывал включить этот пункт в предыдущие варианты списка, потому что
почти все известные мне основатели компаний – программисты. Конечно,
они могут случайно нанять некомпетентного сотрудника, но это не
разрушит их компанию. В крайнем случае, они могут сделать все
самостоятельно. Но, думая о том, что же погубило большинство стартапов
в области электронной торговли в 90-х, я склоняюсь к выводу, что это
были именно плохие программисты. Многие из таких компаний были
основаны предпринимателями, думавшими, что для удачного стартапа
достаточно иметь достойную идею и пригласить программистов, чтобы ее
реализовать. На деле же это гораздо сложнее, а, зачастую,
просто-напросто невыполнимо: предприниматели не знают, что такое
хороший программист. Они с ними никогда не сталкивались, ведь ни один
действительно хороший программист не станет искать работу, которая бы
состояла в реализации идеи какого-то неизвестного ему предпринимателя.
На практике же предприниматели выбирают тех, кто им кажется хорошими
программистами (в резюме же указано, что он является сертифицированным
разработчиком Microsoft), на деле таковыми не являясь. Вскоре
основатели с удивлением узнают, что их стартап движется со скоростью
бомбардировщика времен Второй мировой войны, в то время как конкуренты
несутся вперед как современные истребители. Такой стартап обладает
недостатками большой компании, но при этом не имеет ее преимуществ.
Как же выбрать хорошего программиста, если вы сам не программист?
Сомневаюсь, что ответ существует. Я чуть было не написал, что нужно
найти хорошего программиста, чтобы он помогал набирать сотрудников. Но
как, если вы не можете отличить хорошего программиста от плохого?

Неправильно выбранная платформа

С предыдущим пунктом связана другая проблема (поскольку подобные
решения принимают плохие программисты) – это неправильный выбор
платформы. Например, я считаю, многие стартапы периода краха доткомов
оказались неудачными из-за решения заниматься разработкой серверных
приложений на Windows. Сервис Hotmail продолжал работать на базе
FreeBSD в течение еще нескольких лет после его приобретения компанией
Microsoft: сервера на Windows не могли справиться с такой нагрузкой.
Если бы основатели Hotmail использовали Windows, почтовик бы обречен.
PayPal также был близок к тому, чтобы прекратить существование. После
слияния сX.com новый исполнительный директор хотел перейти на Windows
даже несмотря на то, что один из основателей PayPal Макс Левчин
утверждал, что на Windows сервис будет масштабироваться лишь на 1\% от
возможностей Unix. К счастью для PayPal, вместо этого они поменяли
исполнительного директора. Платформа – понятие расплывчатое. Оно может
означать операционную систему, язык программирования или фреймворк.
Платформа – как фундамент дома, поддерживает строение, но и
накладывает ограничения. С платформами связана следующая опасность: та
или иная платформа может показаться подходящей человеку со стороны, но
выбор этой платформы погубит весь проект, как было в случае с Windows
в 90-х. Java-апплеты – один из ярчайших примеров. Они должны были
стать новым способом доставки приложений. Предположительно, они
погубило 100\% стартапов, основатели которых разделяли веру в
технологию. Как подобрать правильную платформу? Пригласить классных
программистов и предоставить им возможность выбирать. Но даже если вы
не программист, есть один хороший способ сделать правильный выбор:
посетите факультет информатики известного университета и посмотрите,
что они используют в исследовательских проектах.

Промедление с запуском

Компании любого уровня переживают нелегкие времена, работая над
программным обеспечением. Это отличительная особенность индустрии:
программа всегда готова только на 85\%. Бывает достаточно тяжело
переступить через себя и объявить о релизе. (Стив Джобс пытался
мотивировать своих сотрудников словами “настоящие художники выполняют
заказ вовремя”. Это отличные слова, но, к сожалению, это неправда.
Многие известные произведения искусства так и остались незаконченными.
Это утверждение верно для направлений, в которых работа ограничена
определенными сроками, например, для архитектуры и кино, но даже там
есть тенденция тянуть с представлением готового продукта до
последнего). Основатели ищут любые оправдания, чтобы отложить запуск.
Большинство из них совпадает с причинами прокрастинации в повседневной
жизни. Сначала должно что-то произойти. Наверное. Но если бы программа
была готова на 100\% по нажатию кнопки, разве кто-нибудь стал бы
ждать? Одна из причин для скорейшего запуска заключается в том, что в
этом случае вы будете вынуждены действительно закончить какую-то часть
работы. Ничто не может считаться финальным пока вы не запустились. Об
этом свидетельствует огромное количество работы, сопровождающей любой
релиз, независимо от того, насколько готовым к релизу казался ваш
продукт. Другая причина не медлить с запуском заключается в том, что
только по реакции ваших пользователей на идеи, у вас образуется полная
картина. Некоторые стандартные проблемы выливаются как раз в
промедление с запуском: слишком медленная работа, непонимание проблемы
на самом деле, страх перед реакцией пользователей, боязнь оценки,
работа над большим количеством разных задач, чрезмерный перфекционизм.
К счастью, есть простое средство справиться с такими проблемами:
просто заставьте себя запустить что-нибудь достаточно быстро.

Поспешный запуск

Промедление при запуске погубило в сто раз больше стартапов, чем
спешка, но слишком быстрый запуск также может нести в себе угрозу.
Опасность в этом случае связана с тем, что вы рискуете погубить свою
репутацию. Вы запускаете что-то, пользователи это пробуют, и если им
не нравится, они могут больше никогда к вам не вернуться. Что же
входит в тот минимум, необходимый для запуска? Мы считаем, что для
стартапа следует разработать план ваших действий и определить
некоторые ключевые элементы, которые (а) были бы полезны сами по себе
и (б) могли бы, постепенно расширяясь, использоваться в завершенном
проекте, и реализовать их как можно быстрее. Именно этот подход я (и
многие другие программисты) использую для написания программ. Сначала
определяется конечная цель, затем пишется небольшой кусок, который
может оказаться полезным. Если это просто вспомогательный кусок
программы, его все равно придется написать, так что даже в худшем
случае вы не потратите время впустую. Но при этом есть вероятность,
что, получив работающий кусок программы, вы почувствуете себя боле
уверенно и сможете разобраться, что именно следует делать дальше.
Самые первые пользователи, на которых вам нужно произвести
впечатление, достаточно терпимы. Они не думают, что только что
запущенный продукт будет способен на все; но он должен делать хоть
что-то.

Отсутствие представления о пользователе

Нельзя создать продукт, который понравится пользователю, не зная его.
Ранее я уже говорил, что большинство удачных стартапов начинались, как
попытки разрешить какие-то проблемы, с которыми сталкивались
основатели. Можно вывести правило: качество вашего продукта
пропорционально пониманию проблемы, которую вы хотите решить, а
собственные проблемы вы всегда понимаете лучше, чем проблемы кого-то
еще. (Может иметь значение еще один фактор: основатели пытаются
пользоваться новейшими технологиями, так что проблемы, которые они
пытаются решить, наиболее актуальны). Но просто теория. Обратное
утверждение, впрочем, верно всегда: если вы пытаетесь решить проблему,
не понимая ее сути, у вас нет шансов на успех.

На удивление много основателей, похоже, предполагают, что кому-то,
причем точно неизвестно, кому, понадобится их продукт. Нужно ли он
основателям? Нет, они не целевая аудитория. А кто тогда? Молодежь.
Люди, интересующиеся местными событиями (вот это перманентная
ловушка). Или бизнесмены. Но какой отрасли? Автозаправки? Киностудии?
Оборонные предприятия? Конечно, можно заниматься проектами, не
актуальными для себя, а только для пользователей. Мы так делали. Но
нужно понимать, насколько это рискованное предприятие. Фактически, вы
летите по приборам, поэтому необходимо (а) осознавать, когда
перестраиваться, не полагаясь на интуицию, и (б) следить за приборами.
В данном случае приборы – это ваши пользователи. Работая для других
людей, вы не имеете права на ошибку. Вы не можете просто гадать, что
сработает, а что нет; вам нужно найти пользователей и наблюдать за их
реакцией. Поэтому, если вы собираетесь делать продукт для молодежи,
или для бизнесменов, или для любой другой группы, в которую вы сами не
входите, вам придется научиться убеждать таких людей в том, что им
нужен ваш продукт. Если вы не сможете этого сделать, вы на пути в
никуда.

Недостаточные инвестиции

Большинство стартапов привлекают внешнее финансирование на том или
ином этапе. Как и присутствие нескольких основателей, это преимущество
хотя бы и по статистике. Но сколько денег следует инвестировать в
проект? Финансирование стартапов связано со временем. У каждого не
приносящего прибыли стартапа (то есть, изначально, почти у всех) есть
определенное количество времени, пока не закончатся деньги и не
придется остановить проект. Иногда, по аналогии с авиацией, его
называют длиной разбега; часто спрашивают: сколько длины у вас
осталось для разбега? Хорошее сравнение, потому что оно напоминает о
том, что как только деньги кончатся, вы либо взлетите, либо погибните.
Поэтому, если вы берете деньги у инвесторов, необходимо взять сумму,
достаточную для перехода на следующий уровень, каким бы он ни был.
(Нужно брать в полтора-два раза больше той суммы, которая, по вашим
прогнозам, вам потребуется, потому что на написание программ и на
заключение сделок уходит больше времени, чем вы думаете). К счастью,
вы можете контролировать свои расходы и думать о следующих шагах. Наш
совет – первое время тратьте по минимуму, и просто сконцентрируйтесь
на создании надежного прототипа. Это даст вам пространство для
маневров.

Слишком большие расходы

Сложно провести грань между слишком большими расходами и слишком
маленькими инвестициями. Если у вас кончаются деньги, причиной может
быть как первое, так и второе. Единственный способ понять, в чем дело,
это провести сравнение с другими стартапами. Если вы получили \$5
миллионов инвестиций, и у вас кончились деньги, вероятно, вы тратили
слишком много. Необоснованно большие расходы теперь встречаются крайне
редко. Основатели, похоже, выучили урок. Кроме того, основывать
стартапы постепенно становится дешевле. На момент написания данной
статьи слишком большие расходы имели место всего в нескольких
стартапах. Ни в одном из финансируемых нами проектов мы не столкнулись
с данной проблемой. (И не только потому, что мы не инвестируем большие
суммы; многие стартапы в дальнейшем получили дополнительные
инвестиции). Наиболее распространенный способ потратить больше, чем
следует, это набрать большой штат сотрудников. Так вы не только
увеличиваете расходы, но и замедляете свою работу – вы не только
больше тратите, но и сроки приходится закладывать длиннее. Большинство
хакеров понимает, почему так происходит; Фред Брукс объяснил это в
книге ”Мифический человеко-месяц“. У нас есть три предложения по
набору персонала: (а) не набирайте людей, если без них можно обойтись,
(б) предлагайте сотрудникам долю в компании, а не фиксированный оклад.
Не только потому, что это способ сэкономить, но и потому, что вам
нужны преданные сотрудники, готовые согласиться с такой формой оплаты;
и (в) нанимайте только тех, кто будет или писать код или привлекать
пользователей, потому что это единственное, что вам потребуется на
первом этапе.

Слишком большие инвестиции

Очевидно, что недостаточное финансирование означает неудачу, но
существует ли такое понятие, как избыточное финансирование? И да, и
нет. Проблема заключается не столько в самих деньгах, сколько в том,
что с ними связано. Как сказал один из инвесторов, сотрудничающих с Y
Combinator: “как только вы получаете от меня несколько миллионов
долларов, часики начинают тикать”. Если инвесторы дают вам деньги, они
не позволят вам просто положить эти деньги в банк и продолжать
работать вдвоем и питаться «дошираком». Инвесторы хотят, чтобы их
деньги работали. Как минимум вы переедете в нормальный офис и наймете
новых сотрудников. Это изменит общую атмосферу, и не во всем к
лучшему. Теперь большинство людей вокруг вас будут вашими
сотрудниками, а не основателями. Они не будут так же лояльны; им нужно
будет говорить, что делать, и они станут участвовать в жизни
коллектива. Когда вы получаете большие инвестиции, ваша компания
переезжает в пригород, и у нее появляются дети.

Возможно, более опасно то, что когда вы берете много денег, становится
гораздо труднее изменить направление деятельности. Предположим, вы
нацеливаетесь на корпоративный рынок. Взяв деньги инвесторов, вы
нанимаете менеджеров по продажам. Что происходит, если вы вдруг
понимаете, что ваш продукт больше подходить для потребительского
рынка? Это совершенно другой тип продаж. На практике же, понимание к
вам так и не приходит. Чем больше людей у вас работает, тем увереннее
вы придерживаетесь изначального плана. Другой недостаток крупных
инвестиций – это время, которое тратится на их привлечение. Чем больше
денег вам нужно, тем дольше займет их получение. Когда речь идет о
миллионах, инвесторы становятся предельно осторожными. Венчурные
капиталисты никогда не говорят да или нет, вместо этого они втягивают
вас в бесконечную дискуссию. На получение венчурных инвестиций уходит
много труда, порой больше, чем на запуск самого стартапа. Не стоит
тратить все свое время на разговоры с инвесторами, в то время как
конкуренты работают над своими проектами. Наш совет ищущим инвестиции
основателям – соглашайтесь на первое разумное предложение, которое вы
получаете. Если вам поступает предложение от солидной организации без
каких-либо необоснованно обременительных условий, просто примите его и
продолжайте создавать компанию. (Некоторые инвесторы предложат вам
заведомо заниженную сумму, чтобы проверить, хватит ли у вас смелости
попросить еще. Это не очень красивые игры, но некоторые инвесторы в
них играют. Если вам попался такой, следует попросить немного
увеличить размер инвестиций). Какая разница, что кто-то другой мог
предложить вам на 30\% больше? С точки зрения экономики запуск
стартапа – это все или ничего. Поиск лучшего предложения среди
инвесторов – пустая трата времени.

Плохие взаимоотношения с инвесторами

Как основатель вы должны выбрать правильную манеру поведения с
инвесторами. Их не следует игнорировать, потому что у них могут
возникать заслуживающие внимания идеи. Но вы также не должны позволять
им управлять компанией. Предполагается, что это ваша работа. Если
инвестор обладает видением, чтобы самим управлять компанией, которую
они финансируют, то почему они не основали компанию самостоятельно?
Игнорирование инвесторов, вероятно, менее опасно, чем постоянные
уступки. В нашем стартапе мы уделяли инвесторам слишком много
внимания. Мы потратили огромное количество энергии на споры с ними
вместо того, чтобы работать над продуктом. Но гораздо хуже было бы
просто уступить – в этом случае компания, скорее всего, развалилась
бы. Если основатели знают, что делают, будет лучше, если на продукт
направлена половина их внимания, чем все внимание инвесторов, не
имеющих никакого представления о сути стартапа. Насколько тяжелым
окажется ваше сотрудничество с инвесторами, зависит от размера
полученных вами инвестиций. Получив крупные инвестиции, жесткого
контроля инвесторов избежать не удастся. Если инвесторам принадлежит
большинство голосов в совете директоров, они фактически становятся
вашим руководством. Чаще всего основатели и инвесторы представлены в
равной степени, и решающие голоса принадлежат независимым директорам;
инвестору достаточно убедить таких директоров в своей правоте, для
контроля над компанией. Если ваши дела идут хорошо, все это не имеет
никакого значения. Если развитие идет быстрыми темпами, инвесторы не
будут вас беспокоить. Но стартапы далеко не всегда развиваются гладко.
Инвесторы доставляли неприятности даже самым успешным компаниям. Один
из самых известных примеров – это Apple, чей совет директоров принял
близкое к роковому решение об увольнении Стива Джобса. И даже Google
на ранних стадиях пострадал от своих инвесторов.

Жертвовать пользователями ради (предполагаемой) прибыли

Когда в самом начале я сказал, что если вы сделаете что-то, что будет
востребовано пользователями, то все будет отлично, вы, вероятно,
заметили, что я ничего не сказал о выборе правильной бизнес-модели. Не
потому, что получение прибыли неважно. Я не предлагаю основателям
создавать компании без шансов на получение прибыли с надеждой
избавиться раньше, чем они обанкротятся. Причина, по которой мы
убеждаем основателей не волноваться насчет бизнес-модели с самого
начала, заключается в том, что создать что-то такое, что нужно людям,
существенно труднее. Я не знаю, почему так тяжело создать что-то,
понравится людям. Это кажется простым. Но небольшое количество удачных
стартапов говорит о том, насколько это сложная задача. Создать что-то
нужное людям гораздо сложнее, чем заработать на этом деньги, и поэтому
выбор бизнес-модели следует отложить, как вы отложили бы внедрение
какой-то очевидной, но трудозатратной функции до второй версии
программы. В первой же версии необходимо решить основную задачу. А
основанная задача стартапов заключается в создании достойного
предложения (= заинтересованные пользователи умножить на количество
таких людей), а не в переводе этого предложения в денежные средства.
Успешные компании в первую очередь думают о пользователях. Например,
Google. Сначала они создали поисковую систему, и только потом стали
думать о том, как на этом заработать. Но некоторые основатели
по-прежнему считают, что было бы безответственно не фокусироваться на
бизнес-модели с самого начала. И их часто поддерживают инвесторы, чей
опыт связан с работой в более традиционных отраслях. Не думать о
бизнес-модели – это действительно безответственно. Но гораздо более
безответственно не думать о продукте.

Нежелание пачкать руки

Почти все программисты были бы счастливы, занимаясь только написанием
кода и предоставив возможность зарабатывать на этом кому-нибудь
другому. И речь здесь идет не только о лентяях. Наверняка Ларри и
Сергей сначала думали так же. Первое, что они сделали сразу после
разработки своего алгоритма поиска, это попытались найти компанию,
которая бы его купила. Создать компанию? Какой ужас! Большинство
компьютерщиков предпочли бы не идти дальше идей. Но Ларри и Сергей
выяснили, что никто не хочет покупать идеи. Никто не доверяет идее,
пока она не представлена в виде продукта, которым пользуется все
больше и больше пользователей. Вот тогда они готовы платить, и много.
Возможно, в будущем такая ситуация изменится, но я сомневаюсь, что
существенно. Ничто так не убеждает покупателей, как размеры целевой
аудитории. Не то, чтобы риск уменьшался. Просто покупатели тоже люди,
и им нелегко решиться заплатить паре молодых парней несколько
миллионов долларов только за то, что они такие умные. Когда идея
превращается в компанию с множеством клиентов, они могут сказать себе,
что покупают именно эту аудиторию, а не чьи-то способности, и с этим
им проще смириться. (Представьте себе, что создатели YouTube приходят
в Google в 2005 году и говорят: “Дизайн Google Video никуда не
годится. Дайте нам 10 миллионов долларов, и мы укажем вам все ваши
ошибки”. Никто бы не стал с ними разговаривать. Через полтора года
компания Google заплатила за аналогичный урок 1.6 миллиарда долларов
отчасти потому, что они могли сказать себе, что покупают явление,
сообщество или что-то похожее). Если вы хотите привлечь пользователей,
вероятно, вам придется оторваться от компьютера и поискать их. Это
неприятная работа, но если вы заставите себя ее сделать, ваши шансы на
успех значительно повысятся. Основатели самых первых стартапов,
которые мы финансировали летом 2005 года, тратили почти все свое время
на разработку. Но один из них половину времени тратил на переговоры с
руководителями сотовых операторов, пытаясь заключить сделки. Разве
может быть что-то более тягостным для хакера? Но усилия окупились:
этот стартап оказался на порядок успешнее, чем остальные из той же
группы. Если вы собираетесь запустить стартап, вам придется смириться
с тем, что нельзя лишь писать код. Хотя бы один из хакеров должен
будет заниматься ведением бизнеса.

Разногласия между основателями

Как ни странно, разногласия между основателями – достаточно
распространенное явление. Примерно в 20\% финансируемых нами стартапов
мы сталкивались с такой ситуацией. Это происходит настолько часто, что
нам пришлось пересмотреть наше отношение к опционам. Мы по-прежнему не
выдвигаем жестких требований, но теперь мы советуем основателям
распределять доли таким образом, чтобы у основателей оставалась
возможность покинуть проект без вреда для кого бы то ни было. Уход
основателя не обязательно убивает стартап. Многие успешные стартапы
прошли через это. (Гораздо больше, чем принято считать, потому что
компании это не афишируют. Известно ли вам, что изначально у Apple
было три основателя?). К счастью, уходят наименее целеустремленные
основатели. Если из трех основателей уходит один, работавший без
особого энтузиазма, в этом нет ничего страшного. Если же уходит один
из двух основателей, или если проект покидает человек, на котором
держалась вся техническая часть, это уже серьезная проблема. Но и с
ней можно справиться. У Blogger остался единственный основатель, и
компания все равно смогла успешно развиваться. Большинства разногласий
между основателями, которые я видел, можно было избежать, если бы
основатели более тщательно выбирали партнеров. Большинство споров
возникают не в связи с какими-то ситуациями, а из-за людей. Поэтому
они неизбежны. И у большинства основателей, пострадавших от таких
споров, были опасения, которые они подавляли, создавая компанию. Не
старайтесь подавлять такие чувства! Гораздо проще все выяснить до
создания компании, чем после. Не создавайте компанию вместе с кем-то,
кто вам по-человечески не нравится, но обладает какими-то нужными вам
навыками, а вы боитесь, что не найдете никого другого. Люди – самая
важная составляющая стартапа, поэтому ни в коем случае не идите на
компромисс.

Недостаточные усилия

Большинство неудачных стартапов, о которых вы слышали, провалились с
шумом и треском. Это высшая категория неудачных проектов. Но чаще
всего к неудаче приводят не выдающиеся ошибки, а недостаточные усилия
основателей – о таких проектах мы с вами ничего не знаем, потому что
их начинали пара человек в свободное от работы время, который ни к
чему не привел, и его постепенно забросили. С точки зрения статистики,
если вы хотите избежать неудач, самое важное, что вам нужно сделать,
это уйти с работы. Большинство основателей неудачных стартапов не
уходят с работы, в то время как почти все удачные стартапы создаются
людьми, которым не требуется ходить на работу каждый день. Если бы
неудачный стартап был заболеванием, Центр по контролю за заболеваниями
выпускал бы предупреждения о необходимости избегать полной занятости
на работе. Означает ли это, что вам нужно уволиться с работы? Не
обязательно. Мне кажется, что большинству потенциальных основателей не
хватает решимости для создания компании, и они сами это понимают
где-то в подсознании. Они не посвящают своему стартапу больше времени,
потому что осознают – оно того не стоит. Можно предположить, что
каким-то основателям не удалось создать успешный стартап, потому что
они не уделяли ему все свое время. Я не знаю, много ли таких
основателей, но если посмотреть на статистику удачных и неудачных
стартапов, можно прийти к выводу, что количество людей, у которых все
могло бы получиться, если бы они ушли с работы, вероятно, на порядок
больше количества людей, у которых все действительно получается. Как
же узнать, относитесь ли вы к категории людей, которым следует уйти с
работы для создания компании, или к заведомо большей категории людей,
которым этого делать не следует? Я хотел сказать, что такое решение
трудно принять самому, и лучше прислушаться к советам со стороны, но
потом понял, что мы, в том числе, даем и такие советы. Мы считаем себя
инвесторами, но, с другой стороны, Y Combinator – это сервис,
помогающий людям понять, следует или не следует им уходить с работы.
Мы можем ошибаться, и нам, действительно, не удается избежать ошибок,
но при этом мы, по крайней мере, не боимся рисковать своими деньгами,
принимая решения. Если это так, то большинство стартапов, которые
могли бы оказаться успешными, проваливаются, потому что их основатели
не посвящают им все свое время. Это, безусловно, совпадает с тем, что
я вижу вокруг себя. Большинство стартапов не удаются, потому что их
основатели не предлагают людям то, что было бы ими востребовано, а
происходит это, главным образом, из-за недостаточных усилий
основателей. Другими словами, для стартапов действуют самые общие
правила. Главная ошибка, которую вы можете допустить – это не
прилагать достаточно усилий. Не игнорировать это правило – вот рецепт
к успеху, в той степени, в которой он вообще возможен.

\end{document}
