\documentclass[ebook,12pt,oneside,openany]{memoir}
\usepackage[utf8x]{inputenc} \usepackage[russian]{babel}
\usepackage[papersize={90mm,120mm}, margin=2mm]{geometry}
\sloppy
\usepackage{url} \title{Как стать богатым} \author{Пол Грэм} \date{}
\begin{document}
\maketitle

С чего бы вы начали, если бы вдруг решили разбогатеть? Я думаю, что
самым лучшим решением было бы запустить новую компанию — стартап, или
присоединиться к уже существующему. Этот способ показывает себя
эффективным в течение сотен лет. Само слово «стартап» возникло в
шестидесятые годы прошлого века, но то, что происходило в то время,
было очень похоже на авантюрные торговые путешествия, которые
предпринимались в средние века.

Обычно понятие стартап связано с высокими технологиями. Настолько
сильно, что выражение «технологический стартап» — почти тавтология.
Как правило, это небольшая компания, которая взялась за решение
сложной технической проблемы.

Многие люди становятся богатыми, не зная ничего из того, о чем идет
речь в этом эссе. Вам вовсе не нужно знать физику, чтобы быть хорошим
торговцем. Однако, понимание некоторых основных принципов, может
помочь. Почему стартапы должны быть маленькими компаниями? Обязательно
ли они перестают быть стартапами, если вырастают? Почему они почти
всегда занимаются разработкой новых технологий? Почему так много
стартапов продают новые лекарства или программное обеспечение, и так
мало подсолнечное масло или моющие средства?

Тезис

С экономической точки зрения к стартапу можно относиться, как к
способу сжать всю вашу рабочую жизнь в несколько лет. Вместо того
чтобы сорок лет работать не очень интенсивно, вы работаете четыре года
настолько сильно, насколько это возможно. Это особенно важно в высоких
технологиях, где успех зависит от скорости вашей работы.

Давайте порассуждаем. Предположим, что вы хороший хакер (в
первоначальном, положительном смысле этого слова) 20-25 лет. Вы можете
получить работу с зарплатой, например, около 80 000\$ в год. Такой
хакер должен выполнять работу, стоимость которой будет не меньше 80
000\$ в год только для того, чтобы не быть своей компании в убыток.
Положим теперь, что вы сможете работать в два раза больше, чем обычный
служащий, и если вы будете работать достаточно сосредоточенно, то вы
сможете сделать за час работы в три раза больше, чем обычно.[1] Если
избавиться от некомпетентного менеджера, стоящего над вами, то можно
увеличить эффективность еще в два или три раза. Теперь подумайте,
насколько вы на самом деле умнее и работоспособнее, чем это
предполагается вашей должностью? Возможно, что в два или даже в три
раза. Теперь сложите все эти множители, и я утверждаю, что вы можете
быть в 36 раз продуктивнее, чем это ожидается на любой корпоративной
работе.[2] И если неплохой хакер в большой корпорации нарабатывает на
80 000\$ в год, то отличный хакер, который работает очень интенсивно
без всей этой корпоративной чуши, способен сделать работы на три
миллиона.

Как и в других приблизительных расчетах, в моем тоже есть много места
для маневра. Я не буду спорить, действительно ли цифры будут точно
такими, как я сказал. Но я уверен, что схема вычислений верна. Я не
утверждаю, что множитель равен именно 36, но он точно больше 10 и
иногда даже приближается к 100.

Если три миллиона кажутся вам слишком большой суммой, то вспомните,
что мы говорим о предельном случае: когда у вас не только нет
свободного времени, но вы еще и работаете настолько интенсивно, что
это может повредить вашему здоровью.

Стартапы — это не волшебство. Они не изменяют законов мироздания.
Закон сохранения остается в силе: если вы хотите заработать миллион
долларов, вам придется выдержать напряжение эквивалентное этой сумме.
Например, одним из способов сделать миллион, является пожизненная
работа в почтовом отделении, при которой вы будете вынуждены экономить
каждый цент. Вообразите весь стресс пятидесятилетней работы на почте и
сожмите его в четыре года. Разумеется, вы получите некоторую скидку,
если купите весь этот стресс оптом. Но обмануть фундаментальный закон
сохранения вы не сможете. Если бы начать стартап было легко, каждый бы
делал это.

Внутреннее устройство многих компаний напоминает коммунистический
режим. Если вы верите в свободный рынок, почему бы не сделать свою
компанию похожей на него?

Гипотеза: компания будет максимально прибыльной, если будет оплачивать
работу своих сотрудников пропорционально прибыли, которую они
приносят.

Миллионы — не миллиарды

Если для одних людей три миллиона могут показаться огромной суммой,
для других это не так уж и много. Три миллиона? Скажите лучше, как мне
стать миллиардером, таким как Билл Гейтс?

Ладно, оставим пока Билла Гейтса в покое. Не очень хорошо использовать
знаменитостей в качестве примеров. В прессе обычно пишут о самых
богатых, а они выбиваются из общей картины. Билл Гейтс умный,
решительный и трудолюбивый человек, но быть таким недостаточно для
того, чтобы заработать столько же денег, сколько есть у него. Кроме
всего этого, вы еще должны быть очень удачливы.

В успехе любой компании присутствует большой элемент случайности.
Парни, о которых читаешь в газетах, это умные и решительные люди, но
кроме этого им удалось выиграть в лотерею. Билл Гейтс, несомненно,
очень умный и очень целеустремленный товарищ, но Микрософт также
посчастливилось извлечь пользу из одной из самых выдающихся ошибок в
бизнесе: соглашения о лицензии DOS. Конечно, Билл сделал все, что было
в его силах, чтобы IBM сделала эту ошибку, и он проделал отличную
работу, однако, если бы на стороне IBM был хоть один человек с
мозгами, будущее Микрософт могло бы быть совершенно другим. На этой
стадии Микрософт имела лишь небольшое влияние на IBM. Фактически, они
были поставщиками компонент. Если бы IBM потребовала эксклюзивную
лицензию, как они и должны были бы сделать, Микрософт все равно бы
заключила соглашение. Это бы все равно означало для них кучу денег, а
IBM могла бы легко получить операционную систему где-нибудь в другом
месте.

Вместо этого IBM использовала все свое влияние на рынке, чтобы
позволить Микрософт контролировать стандарт PC. С этого момента
Микрософт действовала как по намеченному плану. Им никогда не
приходилось принимать рискованных решений. Они просто заняли жесткую
позицию в отношении лицензий и производили инновационные продукты как
можно быстрее.

Если бы IBM не сделала этой ошибки, Микрософт все равно была бы
успешной компанией, но она бы не смогла вырасти так сильно и так
быстро. Билл Гейтс был бы богатым, но он был бы где-нибудь внизу
списка Forbes 400 вместе с другими богачами его возраста.

Есть много способов стать богатым, и это эссе только об одном из них.
Это эссе о том, как делать деньги, создавая ценности. Есть куча других
способов добыть денег, включая лотерею, спекуляцию, законный брак,
наследство, воровство, вымогательство, мошенничество, взяточничество,
подделку денежных знаков и так далее. Большинство величайших
состояний, скорее всего, были сделаны с привлечением некоторых из этих
методов.

Создание ценностей, как способ делать деньги, обладает преимуществом
не только потому, что это более законно (множество других методов
сейчас уже незаконны). Но и потому, что этот способ более прямой.
Нужно просто делать то, что необходимо другим людям.

Деньги — не ценности

Если вы хотите создавать ценности, то нужно понимать что это такое.
Ценности — это не то же самое, что деньги.[3] Ценности существуют на
протяжении всей человеческой истории. Деньги же были изобретены
сравнительно недавно.

Ценности являются фундаментальным понятием. Это то, что нужно людям:
еда, одежда, дома, машины, гаджеты, путешествия в интересные места и
так далее. Вы можете быть богатым, даже если у вас нет денег. Если бы
у вас было устройство, которое по вашему желанию могло сделать машину,
или приготовить вам ужин, или постирать белье, или сделать что-нибудь
еще, чтобы вы захотели, вам не нужны были бы деньги. Или, например,
если бы вы были в Антарктиде, где нечего покупать, количество ваших
денег тоже не имело бы значения.

Ценности — это то, что может удовлетворить ваши желания, это не
деньги. Но почему, если ценности настолько важное понятие, все говорят
о том, как делать деньги? Деньги — это способ обмена ценностями и на
практике они взаимозаменяемы. Но это не одно и то же, и если вы не
планируете стать фальшивомонетчиком, разговоры о том, как делать
деньги могут только усложнить понимание того, как это сделать в
действительности.

Деньги — это побочный эффект специализации. В специализированном
обществе вы не сможете самостоятельно сделать большинство вещей,
которые вам нужны. Если вам нужны картошка, карандаш, или жилье, вы
получаете все это от других людей. А как можно заставить человека,
который выращивает картошку, дать ее вам? Только дав ему то, в чем он
нуждается в свою очередь. Однако, если мы будем напрямую обмениваться
вещами с теми людьми, которые в них нуждаются, мы далеко не уйдем.
Если вы делаете скрипки, и они не нужны ни одному из местных фермеров,
что вы будете есть?

Решение, найденное обществом по мере того, как оно становилось все
более специализированным, заключалось в том, чтобы разбить процесс
торговли на два шага. Вместо того, чтобы напрямую обменивать скрипки
на картофель, вы меняете их, скажем, на серебро, которое затем
обмениваете на что-нибудь другое, в чем вы нуждаетесь. Промежуточное
звено — посредник обмена — может быть чем угодно, что достаточно редко
и что можно легко носить с собой. Исторически так сложилось, что для
этой цели чаще всего использовались металлы, но сравнительно недавно
они были заменены другим посредником, который называется доллар и в
действительности физически не существует. Однако он может выступать в
качестве посредника, потому что его ценность гарантированна
правительством США.

Преимущество использования такого посредника в состоит в том, что он
делает торговлю возможной. Недостаток же в том, что он часто заслоняет
действительное значение торговли. Люди начинают думать, что бизнес —
это делание денег. Но деньги — это всего лишь промежуточное звено,
всего лишь эквивалент того, что на самом деле нужно людям. То, чем
сейчас занимается большинство компаний – это производство ценностей.
Они производят нечто, что нужно другим людям.[4]

Когда начинаешь свой бизнес, очень легко начать думать, что люди хотят
то же, что и вы. Во время бума доткомов, мне довелось разговаривать с
женщиной, которая очень любила гулять на свежем воздухе. И собиралась
сделать портал, посвященный этой теме. Знаете, какой бизнес нужно
открывать, если вы любите свежий воздух? Восстановление данных с
поврежденных винчестеров.

Какая связь? Да никакой. Просто это моя точка зрения. Если хотите
создавать ценности (или другими словами – не голодать), будьте очень
критичны к планам создать бизнес, связанный только с тем, что вам
нравится делать.

Миф Пирога

Удивительно большое количество людей еще с детства усваивают идею
того, что в мире существует фиксированное количество богатства. Да, в
любой нормальной семье в каждый момент есть какое-то определенное
количество денег. Но это не одно и то же.

Когда о богатстве говорят в этом контексте, оно часто описывается в
виде пирога. «Вы не можете сделать пирог больше, чем он есть», говорят
политики. Когда речь идет о деньгах в какой-то конкретной семье или о
деньгах, собранных правительством с помощью налогов за год, это
действительно так. Если один человек получает больше, кто-то другой
должен получить меньше.

Я помню, когда я был маленьким, то верил, что если несколько богачей
владеют кучей денег, то их меньше остается другим людям. Такое
впечатление, что много людей продолжают верить во что-нибудь вроде
этого, когда вырастают. Обычно этот пирог подразумевается всегда,
когда вы слышите чьи-нибудь рассуждения о том, что X процентов людей
владеет Y процентами богатств. Если вы планируете начать стартап, то
хотите вы этого или нет, вы собираетесь разоблачить Миф Пирога.

Что заставляет людей заблуждаться, так это то, что деньги абстрактны.
Деньги сами по себе не богатство. Они только то, что помогает
обмениваться ценностями. И хотя в конкретный момент в определенном
месте (например, в вашей семье за этот месяц) может быть фиксированное
количество денег, доступное для торговли с другими людьми, количество
богатства в мире непостоянно. Вы можете увеличить количество богатства
в мире. Ценности создавались и уничтожались (а затем снова
создавались) на протяжении всей человеческой истории.

Предположим, у вас есть сломанная машина. Вместо того, чтобы сидеть на
диване следующим летом, вы можете потратить некоторое количество
времени и отремонтировать ее. Делая это, вы создаете богатство. Мир, а
в особенности вы, станет богаче на одну работоспособную машину. И не
только в метафорическом смысле. Если вы продадите машину, то получите
за нее больше, чем за сломанную.

Отремонтировав сломанную машину, вы сделали себя богаче. Никто не стал
беднее. Следовательно, никакого Всемирного Пирога не существует. Когда
смотришь на проблему с этой точки зрения, удивляешься, как вообще
кто-то мог думать иначе?[5]

Ремесленник

Люди, которые лучше всего понимают, что богатство в мире непостоянно,
это те, у кого хорошо получается делать вещи своими руками,
ремесленники. Их самодельные работы часто можно продавать в магазине.
Но с ростом индустриализации количество таких людей уменьшается. Одна
из наибольших оставшихся категорий ремесленников — это программисты.

Программист может сидеть перед компьютером и создавать ценности.
Хорошая программа сама по себе является ценностью. Если кто-нибудь
сядет и напишет супербраузер, который не будет глючить (отличная идея,
кстати), мир станет намного богаче.[6]

Люди в одной компании работают вместе для того, чтобы создавать
богатство. Многие из них (например, менеджеры по персоналу или
водители), на самом деле, напрямую не производят никаких вещей.
Программисты же в буквальном смысле извлекают продукт прямо из своей
головы, строчку за строчкой. Поэтому для них очевиднее, что богатство,
это нечто, что производится, а не распределяется неким воображаемым
дядей.

Очевидно также, что качество создаваемого богатства может очень сильно
различаться. У нас в Viaweb был один программист, который был каким-то
суперпроизводительным монстром. Я смотрел на то, что он делал за один
рабочий день, и понимал, что он каждый день увеличивает рыночную
стоимость компании на несколько сотен тысяч долларов. Подобный
суперпрограммист может создать богатства на миллион долларов всего за
несколько недель. Среднестатистический программист за тот же срок
произведет, вероятно, нулевое, если не отрицательное (добавив
несколько багов) количество богатства.

Вот почему большинство хороших программистов являются либералами. В
этом мире ты либо плывешь, либо тонешь, смягчающие обстоятельства
никого не волнуют. Когда люди, далекие от прямого производства
ценностей — студенты, репортеры или политики — слышат, что 5\%
богатейших людей планеты владеют половиной всех денег в мире, они
начинают думать, что это несправедливо! Опытный программист, скорее
всего, подумает: «и это все?». Лучшие 5\% программистов мира,
вероятно, пишут 99\% хорошего программного обеспечения.

Ценности можно создавать не только для того, чтобы продать. Ученые, по
крайней мере, до недавнего времени, фактически дарили свои знания
миру. Мы все получаем выгоду от изобретения пенициллина, люди стали
реже умирать от инфекций. Ценности — это то, что нужно людям, а
сохранение жизни совершенно точно необходимая вещь. Программисты часто
вносят свой вклад в общий котел, разрабатывая открытое ПО. Я стал
гораздо богаче, получив операционную систему FreeBSD, которая работает
на моем компьютере. Или Yahoo, которая использует ее на всех своих
серверах.

Что такое работа

В индустриальных странах люди обычно принадлежат какому-нибудь
институту, по крайней мере, до тех пор, пока им не исполнится 20 лет.
После долгих лет привыкаешь относить себя какой-нибудь группе людей,
которые просыпаются утром примерно в одно время, ходят в одни и те же
здания, что и вы, и, как правило, занимаются вещами, которые им не
очень нравятся. Постепенно это становится частью вашей личности: имя,
возраст, роль, институт. Если вы рассказываете о себе, или кто-нибудь
другой представляет вас, то обычно это выглядит примерно так: Джон
Смит, 20 лет, студент такого-то колледжа.

Предполагается, что когда Джон Смит закончит учебу, он пойдет
работать. Для него это все равно, что присоединиться к другой группе
людей. Если не вдаваться в подробности, то работа — это почти то же
самое, что и колледж. Вы выбираете компании, в которых вы бы хотели
работать и посылаете им резюме. Если они решают взять вас, то вы
становитесь членом новой группы. Вы просыпаетесь утром и ходите уже в
другие здания, и делаете вещи, которые обычно делать не любите. Есть
несколько отличий: жить стало не так весело, и вам платят зарплату. Но
отличий гораздо меньше, чем совпадений. Джон Смит превратился в Джона
Смита, 22 года, разработчик в такой-то компании.

На самом деле, жизнь Джона Смита поменялась сильнее, чем он думает. С
точки зрения общества, работа в компании очень похожа на учебу в
колледже, но если присмотреться, то отличий гораздо больше.

Для того чтобы существовать, компания должна зарабатывать деньги. И
большинство из них делает это, создавая ценности. Кстати говоря,
создавать ценности могут не только те компании, которые занимаются
производством. Вспомните ту волшебную машинку, которая производила бы
для вас еду, одежду и прочее. От нее было бы мало пользы, если бы она
отправляла создаваемые вещи куда-нибудь в Среднюю Азию. Если ценности
— это то, в чем нуждаются люди, то компании, которые занимаются
транспортировкой, тоже создают их. Тоже самое с остальными компаниями,
которые не производят ничего материального. Почти все компании в мире
существуют для того, что делать то, в чем нуждаются другие люди.

И это как раз то, чем, по идее, вы должны заниматься в компании. Есть,
правда, один момент, который обычно сразу незаметен. Ваш личный вклад
в работу компании усредняется другими сотрудниками. Иногда, вы можете
вообще не думать насколько то, что вы делаете, нужно другим людям. Ваш
вклад может быть косвенным. Но компания в целом должна производить
необходимые другим людям вещи, иначе она не будет зарабатывать деньги.
Если ваша зарплата равна X долларов в год, то, по крайней мере, ваша
работа должна приносить эти деньги компании обратно, иначе она будет
тратить больше, чем зарабатывает и, в конце концов, выйдет из бизнеса.

Выпускнику колледжа говорят, и он, в конце концов, сам начинает так
думать, что ему нужно найти работу. Как будто это очень важно —
присоединиться к какой-нибудь группе. Более прямо было бы говорить,
что он должен начать делать то, что нужно другим. А для этого
необязательно присоединяться к какой-нибудь компании. Компания — это
всего лишь группа людей, которые работают вместе, чтобы в итоге
сделать то, в чем нуждаются остальные.[7]

Для большинства людей пойти работать в какую-нибудь существующую
компанию будет хорошим решением. Нужно только понимать, что работа,
это процесс производства ценностей, результаты которого усредняются
остальными сотрудниками компании.

Работать больше

Усреднение несет с собой ряд проблем. Я думаю, что самая большая
проблема состоит в том, что сложно оценить работу каждого сотрудника в
отдельности. В большой компании вам платят за вашу работу какую-то
фиксированную сумму. От вас ждут, что вы не будете некомпетентны или
ленивы, но и не требуют посвящать работе всю вашу жизнь.

На самом деле, здесь тоже работает принцип экономии при увеличении
масштаба производства. При правильно построенном бизнесе, тот, кто
посвятит существенную часть своей жизни работе, будет в десять или
больше раз производительнее обычного сотрудника. Например,
программист, вместо того, чтобы сидеть и исправлять существующий софт,
может написать новый и тем самым создать еще один источник прибыли.

Как правило, устройство компаний не позволяет вознаграждать подобных
людей. Вы не можете подойти к начальнику и сказать, что хотите
работать в десять раз больше, не могли бы они в десять раз увеличить
вашу зарплату? Хотя бы потому, что считается, что вы уже и так
работаете настолько хорошо, насколько можете. Но гораздо более
серьезная причина состоит в том, что у компании нет инструмента,
позволяющего измерить вашу производительность.

Продавцы — исключение из этого правила. Их вклад в прибыль можно легко
измерить, и обычно они получают процент с продаж. Если продавец хочет
зарабатывать больше, он просто начинает больше работать, и это
происходит автоматически.

Есть еще одно исключение из правила — менеджеры высокого звена. И тоже
по тем же причинам: их производительность можно измерить. Их работа
состоит в том, чтобы увеличивать производительность компании в целом.
А в этом случае мы имеем дело уже с конкретными цифрами, и если дела
компании плохи, это касается топ-менеджеров напрямую.

Компания, которая смогла бы выплачивать всем своим сотрудникам
вознаграждение пропорциональное их работе, была бы невероятно
успешной. Большинство наемных работников работали бы лучше, если бы
могли больше получать за это. Еще существеннее то, что такая компания
привлекала бы тех людей, которые хотели бы работать особенно усердно.
Такая компания легко разгромила бы своих конкурентов.

К сожалению, компании не могут платить всем так же, как продавцам или
топ-менеджерам. Продавцы работают в одиночку, а обыкновенные служащие,
как правило, связаны между собой. Представьте себе компанию, которая
производит некоторые устройства. Инженеры разрабатывают функционал,
дизайнеры внешний вид и упаковку, и затем маркетологи убеждают всех,
что этот продукт как раз то, что им необходимо. Как можно вычислить,
сколько процентов общей прибыли принадлежит той или иной группе? Или
как определить, что именно повысило репутацию компании после выпуска
устройства? Даже если бы вы могли читать мысли ваших потребителей, вы
бы все равно обнаружили, что эти все факторы влияют друг на друга.

Если вы хотите быстро двигаться по карьерной лестнице, то столкнетесь
с проблемой того, что ваша работа связана с работой других
сотрудников. Невозможно измерить производительность отдельного
человека в большой группе. И остальные участники будут вас тормозить.

Измерение и рычаг

Для того чтобы стать богатым, вы должны сделать так, чтобы вашу
производительность можно было измерять, иначе вы не сможете получать
больше прибыли оттого, что качество вашей работы улучшилось. Еще одним
условием является то, что в вашей работе должно быть место для
приложения рычага, в том смысле, что решения, которые вы принимаете,
должны иметь заметный эффект.

Одной только возможности измерять недостаточно. В качестве примера
можно привести сдельную работу в кондитерском магазине. Ваша
производительность может быть измерена и вашу работу оплачивают в
соответствии с ней. Однако рычаг отсутствует. Единственное, что вы
можете контролировать, это скорость вашей работы, а с помощь нее можно
увеличить доходы максимум в два или три раза.

Примером работы, в которой присутствует и то и другое, может быть
работа киноактера. Его производительность может быть измерена сборами
от показа фильма. А качество его игры имеет значительное влияние на их
величину, что и является рычагом в данном случае.

Я думаю, что все люди, у которых есть возможность влиять на уровень
своих доходов, находятся в такой ситуации, в которой их
производительность может быть измерена и им есть куда приложить рычаг.
По-крайней мере, все те, кто приходят мне в голову: топ-менеджеры,
киноактеры, менеджеры инвестиционных компаний, профессиональные
спортсмены. Верным признаком того, что в работе есть рычаг, является
возможность провала. Там, где есть возможность много выиграть, должна
быть и возможность много проиграть. Актеры, торговцы на бирже, атлеты,
все они живут под Дамокловым мечом. Как только они начинают работать
слабо, они выходят из бизнеса. Если вы работаете на безопасной
должности, вы никогда не разбогатеете, потому что там, где нет
опасности, почти наверняка нет места для приложения рычага.

Для того чтобы попасть в такую ситуацию, необязательно становиться
актером или топ-менеджером. Все что нужно, это присоединиться к
маленькой группе людей, которая занимается решением сложной проблемы.

Маленький = измеримый

И хотя невозможно точно измерить вклад каждого сотрудника в
отдельности, это можно сделать приблизительно, если измерять работу
небольшой группы.

Один из уровней, на котором можно измерять вклад каждого сотрудника,
это уровень компании в целом. Чем компания меньше, тем ближе вы к
точным результатам. Я считаю, что максимальный размер жизнеспособного
стартапа — десять человек.

Работа в стартапе очень близка к такой, в которой вы можете подойти к
начальнику и сказать, чтобы он платил вам в десять раз больше за то,
что вы будете в десять раз лучше работать. Но с двумя отличиями: вы
говорите это не боссу, а напрямую потребителям (в конце концов,
начальник, это только посредник между вами и ими), и говорите это не
самостоятельно, а вместе с другими вашими коллегами.

Как правило, стартап — это группа. За редким исключением, компания не
может состоять из одного человека. Поэтому в ваших интересах, чтобы
остальные участники были достаточно хороши, так как именно их работа
будет усреднять вашу.

Большая компания похожа на огромную галеру с тысячей гребцов. Две вещи
сохраняют скорость галеры постоянной. Первая — это то, что отдельный
гребец не видит никаких изменений оттого, что гребет сильнее. Вторая —
то, что средний гребец предпочитает оставаться средним.

Если вы наугад возьмете десять гребцов из этой галеры и поместите их в
отдельную лодку, они, скорее всего, смогут плыть быстрее. Энергичного
гребца будет подбадривать то, что он имеет существенное влияние на
скорость лодки. А если кто-нибудь будет лениться, остальные это быстро
заметят и смогут его урезонить.

Но еще большее преимущество десятиместной лодки проявляется тогда,
когда вы сажаете туда десять лучших гребцов. Их будет дополнительно
мотивировать то, что они работают в маленькой группе. Гораздо
выгоднее, когда ваша работа усредняется трудолюбивыми людьми.

В этом и состоит действительное значение стартапа. В идеале, вы
начинаете работать с людьми, которые тоже хотят получать за то, что
работают лучше, гораздо больше, чем они могли бы получать в любой
большой компании. И так как стартапы обычно создаются людьми, которые
хорошо знают друг друга, то и точность измерения вклада каждого
участника гораздо выше. Стартап — это не просто десять человек, это
десять человек с похожими качествами.

Стив Джобс однажды сказал, что успех стартапа зависит от первых десяти
сотрудников. Я с ним согласен. Но даже не столько величина стартапа
делает его способным надрать задницу большим компаниям, сколько
возможность выбрать первых сотрудников. Вы должны быть не просто
маленькой компанией, вы должны быть маленькой компанией лучших.

Чем больше группа, тем больше ее средний участник предпочитает
оставаться средним. При прочих равных условиях способный человек в
большой компании будет находиться в худших условиях, чем в стартапе,
так как его производительность будет снижаться меньшей
производительностью остальных сотрудников. Конечно, для него деньги
могут быть не так важны, как стабильность, но если они важны, то ему
лучше уйти и работать среди равных.

Технология = рычаг

Стартапы дают возможность каждому человеку оказаться в ситуации, в
которой их работа может быть измерена и к ней может быть приложен
рычаг. Это возможно потому, что, во-первых, они маленькие, и,
во-вторых, они занимаются разработкой новых технологий.

Что такое технология? Технология — это метод, способ, с помощью
которого мы делаем разные вещи. И если вы изобретете новый способ
делать что-нибудь, его стоимость будет умножена всеми людьми, которые
будут им пользоваться. В этом и состоит разница между стартапом и,
например, рестораном или парикмахерской. Вы жарите яичницу или
стрижете волосы только одному человеку одновременно. Если же вы
решаете техническую проблему, которая касается многих людей, вы
помогаете им всем. Это и есть рычаг.

Если вы посмотрите на историю, то окажется, что большинство людей,
которые стали богатыми, создавая ценности, сделали это, разработав
новую технологию. Никто не разбогател с помощью того, что стал быстрее
жарить яйца или стричь волосы. Флорентинцы стали богатой нацией после
того, как в тринадцатом веке разработали способ делать
высокотехнологичный продукт того времени — красивую тканую одежду.
Голландцы — после того, как в шестнадцатом веке изобрели новый способ
делать корабли и навигационные приборы, что позволило им доминировать
на Дальнем Востоке.

К счастью, существует естественная связь между величиной стартапа и
способностью решать сложные проблемы. Для того чтобы быть на пике
технологий, нужно двигаться очень быстро. То, что сегодня считается
высокими технологиями, через несколько лет не будет стоить и цента.
Маленьким компаниям гораздо легче удержаться на этом острие, так как
на них не давит груз бюрократии. Кроме того, новые технологии обычно
появляются из нетрадиционных подходов, и маленькие компании меньше
ограничены правилами.

Большие корпорации тоже способны изобретать новые технологии. Они
просто не могут делать это быстро. Размер делает их медленными и не
позволяет достойно вознаграждать сотрудников за те выдающиеся усилия,
которые необходимы в таких разработках. Таким образом, на практике
оказывается, что большие компании обычно берутся за разработку в тех
областях, в которых нужен большой капитал, таких как производство
микропроцессоров, электростанций или аэробусов. И даже здесь они
сильно зависят от стартапов, занимающихся производством компонентов и
идей.

Очевидно, что стартапы, занимающиеся разработкой биотехнологий или
программного обеспечения, решают сложные технические задачи. Однако, я
думаю, что иногда это может быть верно и для тех бизнесов, которые с
первого взгляда не занимаются технологиями вообще. Например,
McDonald’s, который разбогател, разработав систему и торговую марку,
которые затем были скопированы по всему миру. Рестораны McDonald’s
контролируются настолько строгими правилами, что к ним можно
относиться почти как к компьютерной программе.

Выбирайте сложные проблемы не только в качестве основных целей
компании, но и в ключевых моментах вашей работы. У нас в Viaweb одним
из правил правой руки было бежать вверх по лестнице. Предположим, что
вы маленький, приземистый парень и за вами гонится огромный, жирный
толстяк. Вы открываете дверь и обнаруживаете за ней лестницу. Куда вы
побежите, вверх или вниз? Я вам советую бежать вверх. Вниз толстяк,
скорее всего, будет бежать с той же скоростью, что и вы. Если он будет
бежать вверх, то его размер сразу станет для него недостатком. Вам
бежать наверх трудно, но ему вдвойне труднее.

На практике это означало, что мы специально искали себе сложные
задачи. Если нам приходилось выбирать между реализацией двух
возможностей в нашей программе, и их стоимость была одинакова в
отношении качества к сложности, мы выбирали более сложную. Не только
потому, что она была лучше, но гораздо больше потому, что она была
сложной. Тем самым, мы заставляли наших больших и неповоротливых
конкурентов догонять нас по зыбкой почве. Я помню, что иногда мы были
просто истощены борьбой с какой-нибудь сложной технической проблемой,
но я получал от этого удовольствие, потому что, если это было сложно
для нас, это будет невозможно для конкурентов.

На самом деле, это единственный способ работы стартапа. Инвесторы
знают это и у них даже есть специальное выражение: барьеры для входа.
Если вы придете к венчурному капиталисту и попросите его инвестировать
в вашу идею, то первое о чем он вас спросит, это насколько сложно
будет кому-нибудь другому разработать то же самое?[8] И вам лучше
подготовить что-нибудь поубедительнее, почему ваша технология не
сможет быть скопирована конкурентами. Иначе, как только какая-нибудь
большая компания поймет, что ваш продукт начинает ей угрожать, она
сделает свой такой же, и их бренд, капитал и дистрибьюторская сеть
помогут им выбросить вас из бизнеса.

Один из способов расставить барьеры — это патенты. Но, к сожалению,
они не очень помогают. Как правило, конкуренты находят способы обойти
ограничения, накладываемые патентами. А если не могут обойти, то они
могут просто забить на них и заставить вас судиться с ними. Большая
компания не боится судебных процессов, для них это обыкновенное дело.
Они позаботятся о том, чтобы процесс был долгим и дорогостоящим.
Когда-нибудь слышали о Фило Фарнсворте? Он изобрел телевидение. А не
слышали вы о нем потому, что его компания не смогла сделать на этом
деньги.[9] Зато смогла другая — RCA, а все, что получил Фарнсворт —
это десять лет судебной тяжбы.

Лучшая защита — это нападение. Если вы сможете разработать технологию,
которую будет сложно повторить вашим конкурентам, вам не нужны будут
другие барьеры. Начните решать сложную задачу, и затем, в каждый
момент, когда нужно будет принимать решение, выбирайте более
сложное.[10]

Ловушки

Если бы проблема была только в том, как работать лучше обыкновенного
служащего и получать за это больше, было бы очевидно, что нужно
начинать стартап. До какого-то момента это было бы даже забавно. Я не
думаю, что многим людям нравится медленная поступь больших компаний, а
также все эти нескончаемые митинги и тупые менеджеры.

К сожалению, существует несколько неприятных моментов. Один их них
состоит в том, что вы не можете, например, решить, что будете работать
в два или три раза больше, получать за это больше денег и жить
припеваючи. Когда вы запустите стартап, ваши конкуренты оценят,
насколько сильно вы работаете, и решат работать так же как вы или еще
сильнее.

Еще одна сложность в том, что ваша прибыль пропорциональна вашем
усилиям только в среднем. Как я уже говорил, существует много
случайных факторов, влияющих на ее величину. И на практике получается,
что коэффициент умножения будет где-нибудь между нулем и тысячей.
Большинство стартапов оказываются провальными, и не только пресловутые
порталы собачьего питания, о которых мы все слышали, когда лопнул
Интернет-пузырь. Это обыкновенная история, когда стартап разрабатывает
новую супертехнологию, немного задерживается с ее выпуском, расходует
все свои деньги и выходит из бизнеса.

Стартапы — как комары. Медведь может их раздавить, а краб защищен от
них панцирем, но преимущество комаров как вида, в том, что их много.
Хотя, конечно, для отдельного комара это маленькое утешение.

Стартапам, как и комарам приходится действовать по принципу все или
ничего. Viaweb тоже приходилось иногда действовать напролом. Наша
траектория напоминала синусоиду, но к счастью, нас купили на вершине
цикла. Пока мы договаривались с Yahoo в Калифорнии о продаже им нашей
компании, нам пришлось одолжить комнату для переговоров, чтобы убедить
инвесторов не прекращать финансирование.

Конечно, никому не нравится этот аспект в стартапах. Разработчики в
Viaweb были людьми, которые избегали риска во что бы то ни стало. Если
бы существовал способ просто работать очень много и гарантированно
получать за это деньги, было бы очень здорово. Мы предпочли бы 100\%
вероятность заработать миллион долларов, чем 20\% вероятность
заработать десять миллионов, несмотря на то, что теоретически второй
вариант вдвое прибыльнее. К сожалению, в современном мире нет такого
бизнеса, в котором вы могли бы получить первый вариант.

Можно получить почти гарантированную прибыль, продав стартап на ранней
стадии, тем самым, отказавшись от большой (но рискованной) прибыли в
пользу маленькой (но верной). У нас был такой шанс, но мы совершенно
тупо, как мы тогда думали, упустили его. Сразу после этого мы до
смешного страстно захотели продаться. В течение года или двух, если бы
кто-нибудь проявил хоть небольшой интерес к Viaweb, мы бы попытались
продать им компанию. Но так как никто не заинтересовался, нам пришлось
работать дальше.

Мы тогда были бы очень удачной покупкой. Но компании, достаточно
большие для того, чтобы кого-нибудь купить, также достаточно велики,
чтобы быть довольно консервативными. А люди, которые принимают решения
о таких приобретениях, еще более консервативны, потому что, как
правило, они приходят в компанию из бизнес-школ уже на поздних
стадиях. Они предпочитают переплачивать за более безопасные варианты.
Таким образом, проще продать стартап, когда его будущее уже
определилось, даже за огромную сумму.

Набирайте пользователей

Я думаю, что продажа компании — хорошая идея. Запустить бизнес это не
то же самое, что управлять им. Кроме того, продажа компании позволит
вам вложить деньги во что-нибудь другое. Что бы вы сказали о
финансовом менеджере, который вкладывал бы все деньги в акции только
одной молодой компании?

Что нужно делать для того, чтобы вас купили? На самом деле, почти то
же самое, что бы вы делали, если бы не собирались продавать компанию.
Например, получать прибыль. Однако, кроме этого есть еще много
хитростей, и мы провели массу времени, пытаясь в них разобраться.

Потенциальные покупатели вашего бизнеса постоянно оттягивают покупку
настолько, насколько это возможно. Самое сложное в деле продажи
компании, это заставить их действовать. Для большинства людей наиболее
сильным мотивом является не возможность заработать, а возможность
потерять. Для потенциальных покупателей страх, что вас купят их
конкуренты, является наиболее сильной мотивацией. Насколько мы поняли,
этот страх лишает топ-менеджеров рассудка. Следующим по силе является
страх, что вы начнете расти очень быстро и будете стоить дороже или
даже станете их конкурентами.

Вы, наверное, думаете, что компания, которая собирается вас купить,
будет глубоко изучать то, что вы делаете и на основании этого
принимать решение о покупке? Абсолютно не так. Все, что они захотят
знать, это то, сколько у вас пользователей.

Потенциальные покупатели думают, что потребители лучше знают, чьи
технологии заслуживают внимания. И это умнее, чем выглядит на первый
взгляд. Пользователи — это единственное доказательство того, что вы
сделали что-то ценное. Ценности — это, что нужно людям, и если они не
используют ваше программное обеспечение, то может быть это не потому,
что вы плохие маркетологи, а потому, что вы не сделали того, что им
было бы нужно?

У венчурных капиталистов есть список плохих признаков, на которые
нужно обращать внимание. Один из главных следующий: компания
управляется техниками глубоко поглощенными решением технических задач
и не занимается своими пользователями. Работая в стартапе, вы не
просто решаете сложные проблемы, вы решаете проблемы, которые
беспокоят пользователей.

Я советую вам поступать также как и потенциальные покупатели вашего
бизнеса. Для оптимизации вашей работы измеряйте производительность
стартапа количеством пользователей. В оптимизации компьютерной
программы ключевым моментом является измерение. Если вы пытаетесь
просто угадать, где же она тормозит, и что может это исправить, то вы,
скорее всего, ошибетесь.

Количество пользователей может быть не очень точным тестом, но
достаточно близким к этому. Это то, что заботит потенциальных
покупателей бизнеса в первую очередь. Это то, от чего зависит прибыль.
Это то, что заставляет страдать конкурентов. Это то, что впечатляет
журналистов и новых пользователей.

Помимо всего остального, такой тест позволит вам избежать еще одной
проблемы, которая заботит венчурных капиталистов — слишком долгая
разработка продукта. Избегайте постоянного улучшения программы.
Выпустите версию 0.1 как можно скорее. До тех пор, пока у вас
недостаточно пользователей, вы оптимизируете, полагаясь на догадки.

Всегда помните о том, что ценности – это то, что необходимо людям.
Если вы собираетесь разбогатеть, создавая ценности, вы должны знать,
что хотят люди. Так мало бизнесов действительно уделяют внимание тому,
чтобы делать своих пользователей счастливыми. Как часто вы идете в
магазин, или звоните в какую-нибудь компанию по телефону, испытывая
при этом страх? Неужели, когда вы слышите: «Ваш звонок очень важен для
нас, пожалуйста, оставайтесь на линии», вы думаете: «О боже,
наконец-то все будет в порядке?»

Ресторан может позволить себе иногда подать подгоревший обед. Но в
технологиях любая разница между вашим продуктом и тем, что хотят
пользователи, умножается во много раз. Можно сказать, что вы
удовлетворяете или разочаровываете потребителей оптом. Чем ближе вы к
тому, что им нужно, тем больше ценностей вы производите.

Богатство и власть

Производство ценностей — это не единственный способ разбогатеть. На
протяжении почти всей человеческой истории этот способ был не очень
распространенным. Всего несколько столетий назад основными источниками
богатства были рудники, рабы, земли и крупный рогатый скот, и
единственными способами добыть все это быстро были наследство, брак,
завоевание и конфискация. У богатства была плохая репутация.

С тех пор изменилось две вещи. Первой была власть закона. Во все
времена было так, что если кому-нибудь везло и у него появлялось
богатство, правитель или его слуга находили способ отобрать его. Но в
средневековой Европе произошло нечто новое. В городах появился новый
класс торговцев и мануфактурщиков. Вместе они были способны
противостоять властителю или местному феодальному лорду. Впервые в
истории, хозяева перестали грабить своих подчиненных. Это было
гигантским шагом, возможно именно это послужило причиной второго
изменения — индустриализации.

Огромное количество литературы посвящено причинам Индустриальной
Революции. Но совершенно точно, что необходимым ее условием было то,
чтобы люди, которым посчастливилось сделать состояние, могли
насладиться им в мире. Доказательством может служить то, что случилось
со странами, которые пытались вернуться к старой модели, Советским
Союзом и в меньшей степени Англией в 1960х и начале 1970х. Заберите у
людей возможность разбогатеть и технический прогресс остановится.

Вспомните, что с экономической точки зрения стартап – это способ
сказать, я хочу работать быстрее, платите мне больше. Вместо того
чтобы медленно копить деньги и получать регулярную зарплату пятьдесят
лет, я хочу получить все как можно быстрее. Правительство, которое не
позволяет сделать это, заставляет людей работать медленно. Они не
против того, чтобы вы заработали три миллиона долларов за пятьдесят
лет, но они не хотят, чтобы сделали это за два года. Они как ваш босс,
к которому вы не можете подойти и сказать, что вы хотите работать в
десять раз лучше и получать за это в десять раз больше. Отличие в том,
что вы не можете уйти и основать свою страну.

Медленная работа не означает, что технические инновации появляются
медленно. Обычно она означает, что инноваций вообще нет. Вы начинаете
отчаянно искать сложные задачи только тогда, когда хотите с помощью их
решения ускорить получение прибыли. Разработка новых технологий — это
гвоздь в вашей заднице. Это, как говорил Эдисон, один процент
вдохновения и девяносто девять процентов потения. Без возможности
разбогатеть никто не будет этим заниматься. В принципе, инженеры стали
бы работать над такими вдохновляющими проектами, как истребитель или
лунный модуль, но более приземленные технологии, такие, как
электрическая лампочка или полупроводники изобретаются
предпринимателями.

Стартапы — это не только то, что происходило в Кремниевой долине
последние несколько десятилетий. С тех пор, как стало возможным
разбогатеть, создавая ценности, каждый, кто сделал это, использовал
одинаковый рецепт: измерение и рычаг. Этот рецепт был тем же самым во
Флоренции в тринадцатом веке, он остается таким же и сейчас.

Понимание этого поможет ответить на вопрос: почему Европа столь
сильна. Неужели это из-за ее географического положения? Или может
европейская раса чем-то лучше? А может это из-за религии? Ответ в том,
что европейцы поставили на новую великую идею: дайте тем, кто сумел
разбогатеть, возможность сохранить свое богатство.

Как только вы сделаете это, люди, которые хотят разбогатеть, начнут
создавать ценности, вместо того, чтобы воровать их. При этом
технический прогресс будет трансформироваться не только в богатство,
но и в военную мощь. Теория, с помощью которой был построен
самолет-невидимка, была разработана советским математиком. Но так как
у Советского Союза не было компьютерной индустрии, она так и осталась
для них теорией.

В этом смысле Холодная война учит нас тому же уроку, что и Вторая
Мировая, да и остальные войны человечества. Не позволяйте правящему
классу давить предпринимателей. То, что делает личность богатой,
делает страну могущественной. Позвольте людям сохранить свое
богатство, и вы будете править миром.

Примечания

[1] Одна из самых замечательных вещей, которые присутствуют, по всей
видимости, только в новых компаниях, это то, что вы можете работать ни
на что не отвлекаясь. Если вы будете каждые пятнадцать минут
отвелекать человека, который ищет ошибки в тексте, его
производительность каждый раз будет немного уменьшаться. Но с
программированием все еще хуже, часто нужно не меньше часа, чтобы
только вникнуть в суть проблемы. Поэтому стоимость вызова вас в
бухгалтерию для того, чтобы вы заполнили какую-нибудь форму, может
быть очень высокой.

[2] Сталкиваясь лицом к лицу с идеей, что люди, работающие в молодых
компаниях, могут быть в 20 или 30 раз производительнее тех, кто
работает в больших корпорациях, менеджеры обычно интересуются, как
можно сделать так, чтобы их сотрудники работали так же эффективно?
Ответ простой: платите им.

[3] До недавнего времени мне казалось, что даже правительство не
понимает разницы между деньгами и ценностями. Адам Смит иногда
говорил, что они пытаются сохранить богатство страны, запрещая экспорт
серебра и золота. Но большое количество «посредника обмена» не сделает
страну богаче. Если денег будет больше, чем материальных ценности,
которые они представляют, то результатом будут только более высокие
цены.

[4] У понятия «ценности» существует довольно много значений. Я не
собираюсь сейчас выяснять какое из них более правильное. Здесь я
обсуждаю несколько техническое значение этого слова. Ценности – это
то, за что люди дают деньги. Это довольно интересный вид ценностей для
изучения, потому что именно он позволяет вам не умереть от голода. И
то, за что люди дают вам деньги, зависит от них, а не от вас.

[5] Строго говоря, вы все-таки могли сделать мир немного беднее,
микроскопически загрязнив окружающую среду. Но это все равно не равно
количеству созданного богатства. Кроме того, вы могли просто
привинтить что-нибудь и ничего не загрязнять.

[6] Эссе было написано до того, как вышел Firefox.

[7] Многие люди испытывают депрессию в этом возрасте. Еще бы, в
колледже жизнь казалась такой веселой. А теперь, вы из гостя
превратились в обслуживающий персонал. На самом деле, в этом новом
мире жизнь тоже может быть веселой. В конце концов, теперь вы можете
свободно проходить в двери с надписью «Только для персонала». Но в
целом, подобное изменение — это шок.

[8] Когда они спрашивали об этом нас, мы обычно отвечали, что это
вообще невозможно. Однако я думаю, что мы выглядели наивными или
лгунами.

[9] Немного технологий изобретено одним человеком. Обычно если вы
знаете изобретателя чего-либо (телефона, самолета, электрической
лампочки), это означает, что его компания смогла сделать на этом
деньги. Если не знаете, это означает, что деньги сделала другая
компания.

[10] Это, кстати, вообще хороший способ принимать решения в жизни.
Если вы стоите перед выбором, выбирайте более сложное решение. Если вы
выбираете между пойти позаниматься спортом или посмотреть телевизор,
идите заниматься. Думаю, что причина, по которой этот трюк очень
хорошо работает, заключается в том, что когда у вас есть два пути,
простой вы выбираете только из-за лени. Причем подсознательно, вы об
этом знаете, выбор более сложного пути просто заставляет вас осознать
это.

\end{document}
