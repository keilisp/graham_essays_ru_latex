\documentclass[ebook,12pt,oneside,openany]{memoir}
\usepackage[utf8x]{inputenc} \usepackage[russian]{babel}
\usepackage[papersize={90mm,120mm}, margin=2mm]{geometry}
\sloppy
\usepackage{url} \title{Локальная революция} \author{Пол Грэм} \date{}
\begin{document}
\maketitle

Недавно я осознал, что в моей голове крутятся две идеи, которые
взорвутся, если их соединить. Первая, что стартапы — это новая
экономическая эра, по типу промышленной революции. Я не уверен в этом,
но, с большой долей вероятности, можно утверждать, что это так.
Основатели и персонал стартапов работают намного продуктивнее
(представьте, насколько меньше добились бы Ларри и Сергей[имеются в
виду основатели Google], если бы они работали на большую компанию), и
эта возросшая продуктивность может изменить общественные устои.

Вторая идея, что стартапы это такой вид бизнеса, что процветает в
определенных местах, которые специализируются. Кремниевая долина
специализируется на стартапах так же, как Лос-Анжелес на фильмах или
Нью-Йорк на финансах. [1]

Что, если обе идеи правильны? Что, если стартапы одновременно и новый
этап промышленной революции, и такой вид бизнеса, что процветает в
определенных местах?

Если так, то эта революция будет совершенно необычна. Все предыдущие
революции распространялись. Земледелие, города, индустриализация
широко распространились по миру. Если стартапы закончат как
кинобизнес, сосредоточенный в нескольких местах и одном доминирующем
центре, то это будут новые последствия.

Уже сейчас есть признаки, указывающие, что стартапы могут не
распространиться достаточно широко. Стартапы распространяются
медленнее, чем промышленная революция, несмотря на то, что средства
связи стали быстрее. В течение нескольких десятилетий после основания
компании «Boulton\&Watt» паровые машины распространились по Европе и
Северной Америке. Индустриализация какое-то время не выходила за
пределы этих регионов. Потому что это были земли, где сложился крепкий
средний класс, и человек мог основать предприятие, не боясь
конфискации в случае успеха. А иначе не имело смысла вкладывать деньги
в бизнес. А в странах с крепким средним классом промышленная техника
внедрялась достаточно легко. Частный собственник шахты или фабрики
решал поставить паровую машину и в течение нескольких лет он, скорее
всего, находил поблизости кого-то, кто мог сделать её для него.
Поэтому паровые машины быстро распространились. И они распространились
широко, потому что расположение шахт и фабрик определялось такими
факторами, как реки, порты и залежи полезных ископаемых. [2]

Стартапы не распространяются с такой скоростью с одной стороны потому,
что это скорее социальное явление, а не техническое, а с другой
потому, что они не привязаны к каким-либо географическим объектам.
Частный предприниматель в Европе мог приобретать промышленную технику
и она прекрасно работала на новом месте. Со стартапами так не
получается: вам требутся сообщество специалистов, также как в
киноиндустрии. [3] К тому же отсутствуют факторы, которые бы
способствовали распространению стартапов. Когда появились железные
дороги и электрические сети, они потребовались в каждом регионе.
Территория, где их не было, становилась потенциальным рынком. Но со с
стартапами всё совсем иначе. Нет нужды создавать Microsoft во Франции
или Google в Германии.

Правительства могут принять решение поощрять стартапы. Однако
государственная политика, не приводит к их появлению, как могла бы
привести естественная потребность. Так что же получается? Если бы меня
спросили сейчас, я бы сказал, что стартапы будут распространяться, но
очень медленно, потому что этому не поможет государственная политика
(которая не работает) или рыночная потребность (которой нет), в той
мере, в какой это вообще происходит, по тем же случайным причинам, что
приводили к распространению стартап-культуры до сих пор. И уже
существующие хабы сосредоточения стартапов будут всё сильнее влиять на
эти случайные причины. Кремниевая долина находится там, потому что
Вильям Шокли (William Shockley) хотел вернуться в Пало Альто, где он
вырос, и специалистам, которых он заманил на Запад для работы с ним,
так понравилось, что они остались. Сиэтл во многом обязан своему
положению технического хаба той же причине: Гейтс и Аллен хотели
вернуться домой. Иначе Альбукерке мог бы занять место Сиэтла. Бостон —
технический хаб потому, что это интеллектуальная столица США и,
вероятно, мира. И, если бы венчурная компания Battery Ventures, не
отказалась от переговоров с Фейсбук, Бостон сейчас занимал бы более
значительное место на карте стартапов. И, разумеется, не случайно
Фейсбук нашел финансирование в Долине, а не в Бостоне. Даже студенты
знают, что здесь инвесторов больше, и они более смелые, чем в Бостоне.

Пример Бостона иллюстрирует те проблемы, с которыми вы столкнетесь при
создании нового стартап-хаба на нынешнем этапе. Если вы будете
создавать стартап-хаб, следуя тем путем, каким прошли существующие
хабы, вам следовало бы учредить первоклассный исследовательский
университет в таком прекрасном месте, где захотели бы жить богатые
люди. В этом случае место окажется гостеприимно для обеих групп,
которые вам нужны: для основателей, и для инвесторов. Именно такая
комбинация привела к появлению Кремниевой долины. Но у Кремниевой
долины не было конкурента в лице другой Кремниевой долины. Если вы
сейчас будете создавать стартап-хаб путем строительства большого
университета в прекрасном месте, то вы убедитесь, что время сейчас не
благоприятствует этому, потому что многие из лучших стартапов, которые
возникнут в вашем хабе, легко переместятся в другие.

Недавно я придумал короткий способ создания стартап-хаба: платите
стартапам за переезд. Когда вы наберёте достаточное количество хороших
стартапов в одном месте, начнется цепная реакция. Основатели будут
перебираться туда уже без оплаты с вашей стороны, потому что там будут
люди, близкие им по духу, и инвесторы тоже появятся, потому что там
можно будет заключить выгодные сделки.

Я сомневаюсь, что у какого-либо государства хватит яиц, чтобы решиться
создать стартап-хаб или мозгов, чтобы сделать это правильно. Моё эссе
— не руководство к действию, а исследование о том, что потребуется для
сознательного формирования стартап-хабов. Самый вероятный сценарий
дальнейших событий таков: (1) никакое государство не сумеет создать
стартап-хаб, и (2) распространение культуры стартапов будет
определятся случайными факторами, как это происходило до сих пор, но
(3) на эти факторы всё сильнее будут влиять уже существующие
стартап-хабы. Итог: если это и революция, она будет локализованной.

Благодарю Патрика Коллисона (Patrick Collison), Джессику Ливингстон (
Jessica Livingston) и Фреда Уилсона (Fred Wilson) за прочтение
заготовок этого эссе.

Примечания

[1]Есть два кардинально различающихся вида стартапов: первые — те,
которые развиваются естественным путем, и вторые — те, которые
вырастают при коммерциализации научных идей. Большинство компьютерных
и программистских стартапов принадлежат к первому виду, а большинство
фармацевтических — ко второму. В своем эссе я говорю о стартапах
первого вида. Никаких сложностей с распространением стартапов второго
типа нет; коммерциализировать новые идеи, которые выдвигают
исследователи так же просто, как построить новый аэропорт. Стартапы
второго типа не требуют и не создают стартап-культуру. Это значит, что
стартап второго типа никогда не перерастет в стартап первого типа. К
примеру в Филадельфии много стартапов второго типа и практически нет
первого типа.

Между прочим, Google могла бы появиться и как стартап второго вида, но
этого не произошло. Google не коммерциализировала рейтинг
интернет-страниц. Можно было бы использовать другие алгоритмы, и все
равно результат был бы тот же. Google стала Google потому, что она
сосредоточилась на том, чтобы сделать хороший поиск в критический
момент развития интернета.

[2] Паровая машина существовала и до Ватта. Его вклад заключается в
том, что он сделал ее гораздо эффективнее: он изобрел изолированную
камеру для конденсации. Кажется, это умаляет его роль. Но он подошел к
проблеме с неожиданной стороны и взялся за неё с такой энергией, что
буквально всё изменил. Вероятно, было бы точнее говорить, что Ватт
изобрел паровую машину заново.

[3] Крупнейший контрпример — это Скайп. Если то, чем вы занимаетесь,
будет запрещено в США, то оно получит развитие где-либо в другом
месте. Поэтому Kazaa занял место Napster. А опыт и связи, которые
основатели приобрели в работе над Kazaa, помогли обеспечить успех
Скайп.

\end{document}
