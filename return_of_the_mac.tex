\documentclass[ebook,12pt,oneside,openany]{memoir}
\usepackage[utf8x]{inputenc} \usepackage[russian]{babel}
\usepackage[papersize={90mm,120mm}, margin=2mm]{geometry}
\sloppy
\usepackage{url} \title{Возвращение Макинтоша} \author{Пол Грэм}
\date{}
\begin{document}
\maketitle

Все лучшие хакеры, которых я знаю, постепенно переходят на Маки. Мой
друг Роберт рассказал, что недавно вся его исследовательская группа в
МТИ (Массачусетский Технологический Институт) купила себе Powerbook’и.
И эти парни не графические дизайнеры или старушки, которые покупали
Маки во времена кризиса Apple в середине 90-х. Они одни из самых
высококвалифицированных системных программистов.

Разумеется, причиной является OS X. Powerbook’и отлично спроектированы
и работают под управлением FreeBSD. А что ещё надо?

Я купил свой Powerbook в конце прошлого лета и полностью пересел на
него когда на моём IBM Thinkpad умер жесткий диск. И когда мой друг
Тревор недавно заглянул ко мне на огонек, оказалось, что ноутбук у
него точно как у меня.

Для большинства из нас это не переход на Apple. Это возвращение. В
середине 90-х было трудно в это поверить, но в своё время Мак был
каноническим хакерским компьютером.

Осенью 1983 года, один из моих преподавателей по информатике как-то
заявил, словно пророчествуя, что скоро выпустят компьютер, работающий
со скоростью выполнения команд в пол-миллиона операций в секунду и
настолько маленький, что его можно будет засунуть под кресло в
самолете. В придачу, он будет ещё и настолько дешевым, что на него
можно будет накопить за одну летнюю практику. Аудитория была в
изумлении. А когда Мак вышел, то он оказался даже лучше, чем мы могли
надеяться. Как и было обещано, он был маленьким, мощным и дешевым. Но,
в приложении ко всему, он был таким, каким, мы думали, не может быть
ни один компьютер — отлично спроектированным.

Я просто был обязан заполучить такой. И так думал не один я. С
середины и до конца 80-х все хакеры, которых я знал, либо
программировали под Мак, либо хотели делать это. Похоже, на каждом
диванчике в Кембридже лежала раскрытая толстая белая книга. На её
обложке было написано: «Inside Macintosh» (‘Макинтош изнутри’ — прим.
пер.).

Потом появились Linux и FreeBSD и хакеры, следуя за самой мощной
операционной системой, куда бы она не вела, начали переходить на
машины от Intel. Вы могли купить Thinkpad, если вас волновал дизайн,
потому что он был не настолько отвратителен, если снять все эти
наклейки от Microsoft и Intel.

Хакеры вернулись вместе с OS X. Зайдя в Apple Store в Кембридже, я
почувствовал, что попал домой. Конечно же, многое изменилось, однако в
воздухе по-прежнему витал дух Apple и чувство, что здесь работают те,
кому действительно это важно, а не какие-то случайные продавцы.

Ну и что, может сказать рынок. Кого волнует то, что хакеры снова любят
Apple? Да и вообще, насколько большой этот рынок хакеров?

И хотя он достаточно маленький, он очень важен. Всё, что делают
хакеры, касаемо компьютеров, будут делать простые люди примерно лет
через 10. Практически все технологии, от UNIX и до растровых мониторов
для веба сначала получили популярность среди факультетов информатики и
исследовательских лабораторий, а уже потом постепенно распространились
по всему миру.

Я помню, как говорил своему отцу в 1986, что вышли новые компьютеры
Sun, которые были серьезными UNIX машинами, но настолько маленькими и
дешевыми, что он мог бы купить себе такой, вместо того, чтобы сидеть
за терминалом VT100, подключенным к центральному Vax. Я предположил,
что, возможно, ему следовало бы купить несколько акций этой компании.
И теперь он жалеет, что тогда не послушал меня.

В 1994 мой друг Колинг, пытаясь сэкономить на звонках своей девушке в
Тайвань, написал программу, которая конвертировала аналоговый звук в
пакеты данных, которые можно было бы отправить по Интернету. Мы не
были тогда уверены, правильно ли мы используем Интернет, который ещё
был полу-правительственной сетью. То, что он тогда делал теперь
называют VoIP и это громадный и быстро растущий бизнес.

Если вы хотите узнать, чем будут заниматься обычные люди через 10 лет,
прогуляйтесь по факультету информатики в хорошем университете. Чтобы
они не делали, это будете делать вы.

Что касается «платформ», то тут эта тенденция ещё более выражена,
поскольку инновационные программы создают великие хакеры. И делают они
это под теми компьютерами, которыми пользуются сами. Так программное
обеспечение продает аппаратное. Большинство первых покупателей Apple
II были людьми, которым нужен был VisiCalc. А почему Бриклин и
Фрэнкстон написали VisiCalc для Apple II? Да просто потому что он им
нравился. Они могли выбрать любую машину и превратить её в звезду.

Если вы хотите привлечь хакеров для написания программ, которые будут
продавать ваше оборудование, вы должны сделать из него то, чем они
будут пользоваться. И совсем недостаточно просто сделать это открытым.
Оно должно быть открытым и хорошим.

Маки снова открытые и хорошие. Прошедшие годы привели нас ситуации,
когда Apple популярна среди нижнего и верхнего сегментов рынка, но не
среди среднего. Моя семидесятилетняя мама пользуется Маком. Мои
друзья, доктора технических наук, тоже пользуются Маками. Но рыночная
доля Apple всё равно мала.

Я знаю, что эта ситуация временная.

Так что, папа, есть такая компания — Apple. Они делают новые
компьютеры, которые также хорошо спроектированы, как и стерео-системы
Bang \& Olufsen и работают на лучшей Unix машине, которую ты можешь
купить. Конечно, цена пока высока, но, думаю, много людей захотят их
купить.

\end{document}
