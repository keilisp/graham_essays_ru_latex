\documentclass[ebook,12pt,oneside,openany]{memoir}
\usepackage[utf8x]{inputenc} \usepackage[russian]{babel}
\usepackage[papersize={90mm,120mm}, margin=2mm]{geometry}
\sloppy
\usepackage{url} \title{Что случилось с Yahoo} \author{Пол Грэм}
\date{}
\begin{document}
\maketitle

Когда я начал работать в Yahoo после того, как они купили наш стартап
в 1998 году, эта компания казалась центром мира. Казалось, что это
будет следующей большой вещью. Казалось, что она будет тем, чем стала
Google.

Что же пошло не так? Проблемы, которые тянули компанию назад,
появились давно, фактически с самого основания компании. Эти проблемы
были уже хорошо видны, когда я пришел туда в 1998. У Yahoo были две
проблемы, которых не было у Google: легкие деньги и нежелание быть
технологической компанией.

Деньги


Когда я впервые встретился с Джерри Янгом, у нас было разное мнение
относительно того, зачем мы встретились. Он думал, что мы встретились,
чтобы он мог лично нас проверить перед тем, как купить. Я думал, что
мы встретились для того, чтобы мы могли показать ему нашу новую
технологию, Цикл доходов. Это был способ сортировки результатов поиска
товаров. Продавцы предлагали процент от продаж, который они готовы
заплатить за траффик, но результат был отсортирован не по этому
проценту, а по проценту, помноженному на среднее количество товара,
которое купит пользователь. Это было похоже на алгоритм, который
сейчас использует Google для сортировки рекламных объявлений, но это
происходило весной 1998 года, когда Google еще не была основана.

Цикл прибыли был оптимальным способом сортировки результатов поиска
товаров, в том смысле что он сортировал в том порядке, в котором Yahoo
получит максимальную прибыль с каждой ссылки. Но это было не
единственным преимуществом. Ранжирование результатов поиска по
поведению пользователей делает поиск лучше. Пользователи улучшают
поисковый движок: вы можете начать сортировать результаты, основываясь
на простом совпадении текста, но со временем, когда пользователи
покупают больше, поисковый движок с каждым разом улучшается.

Но Джерри это не волновало. Я был немного смущен. Я показывал ему
технологию, которая позволяла получить максимум прибыли из поискового
траффика, а его это не волновало. Я не мог понять — то ли я плохо
объяснял ему это, то ли он просто не хотел показывать своего
удивления.

Я не знал ответа до тех пор, пока не начал работать в Yahoo. Ни одно
из моих предположений не было верным. Причина, по которой Yahoo не
волновала технология, которая позволяла максимизировать прибыль от
траффика, была в том, что рекламодатели уже переплачивали за это. Если
бы они просто получали действительную стоимость, они бы получали
меньше.

Сейчас в это тяжело поверить, но в то время большие деньги делались на
баннерной рекламе. Рекламодатели хотели платить нелепые суммы за
баннерную рекламу. Отдел продаж Yahoo решил использовать этот источник
дохода. Возглавляемые большим и ужасно внушительным человеком,
которого звали Анил Сингх, маркетологи из Yahoo летели в офис Procter
\& Gamble и возвращались назад с миллионными заказами на показ
баннеров.

Цены казались низкими по сравнению с печатной рекламой, с которой
рекламодатели сравнивали, не имея другого сравнения. Но они были
высокими по сравнению с тем, сколько они реально стоили. Эти большие,
тупые компании предоставляли опасный источник доходов. Но был другой,
более опасный источник: другие интернет-стартапы.

В 1998 году Yahoo была на вершине пирамиды. Инвесторов очень
интересовал интернет. Одна из причин, по которой это случилось, был
рост прибыли Yahoo. И они инвестировали в интернет-стартапы. Затем
стартапы использовали эти деньги, чтобы купить рекламу на Yahoo,
которая позволяла получить траффик. Это приводило к еще большему
доходу Yahoo, что опять убеждало инвесторов, что в интернет стоит
инвестировать. Когда я однажды понял это, я прыгнул, как Архимед в
своей ванной, только вместо «Эврика!» вскрикнул «Продажи!».

Интернет-стартапы и Procter \& Gamble рекламировали свой бренд. Они не
задумывались о целевой аудитории. Они просто хотели, чтобы много людей
увидели их рекламу. Траффик стал вещью, которую можно получить в
Yahoo. Неважно, какой траффик. [1]

Это делала не только Yahoo. Все поисковики делали это. Вот почему они
хотели, чтобы люди стали называть их порталами, а не поисковиками.
Несмотря на действительное значение слова портал, они имели в виду,
что это сайт, где пользователи могли найти то, что хотят, на самом
сайте, вместо того, чтобы переходить на другие сайты, как в
поисковиках.

Я помню, как говорил Дэвиду Фило в 1998 или начале 1999, что Yahoo
должна купить Google, потому что я и большинство других программистов
в компании использовали его, а не Yahoo, для поиска. Он сказал мне,
что об этом не стоит беспокоиться. Поиск составлял всего
6% от нашего траффика, и мы росли на 10% каждый месяц. Вряд ли стоило улучшать это.

Я не сказал, что поисковый траффик стоит больше, чем весь другой. Я
сказал: «А, хорошо». Потому что я сам не понимал, сколько стоит
поисковый траффик. Я даже не уверен, что Ларри и Сергей понимали это.
Если бы они понимали, Google вряд ли потратила бы усилия на поиск для
предприятий.

Если бы обстоятельства были иными, руководители Yahoo, возможно,
поняли бы, как важен поиск. Но от правды их отделяло самое тяжелое
препятствие — деньги. Пока клиенты выписывали солидные чеки за
баннерную рекламу, было трудно принимать поиск всерьез. Google это не
сбило с толку.

Хакеры


Но у Yahoo была еще и другая проблема, которая мешала ей сменить
направление. Они были выбиты из равновесия с самого начала из-за их
нежелания быть технологической компанией.

Когда я начал работать в Yahoo, самым странным из того, что я узнал,
было то, что они настаивали на том, что должны называться медийной
компанией. Если вы были в их офисах, они выглядели, как компания,
которая производит программное обеспечение. Рабочие места были
заполнены программистами, которые писали код, менеджерами, которые
думали о списке фич и датами завершения, операторами техподдержки (да,
там были операторы техподдержки), которые говорили пользователям
перезагрузить браузер, и так далее, как и в любой IT-компании. Почему
же они называли себя медийной компанией?

Одна из причин — это способ, которым они зарабатывали: продажа
рекламы. В 1995 году было сложно представить себе технологическую
компанию, которая делала деньги таким образом. Технологические
компании зарабатывали деньги с продажи программ. Медийные компании
продавали рекламу. Значит они, должно быть, медийная компания.

Другим большим фактором был страх перед Microsoft. Если бы кто-нибудь
в Yahoo допустил, что они — технологическая компания, следующей мыслью
было бы то, что Microsoft их разрушит.

Для тех, кто намного моложе меня, тяжело представить себе, какой страх
вызывала Microsoft в 1995 году. Представьте себе компанию, в несколько
раз мощнее Google, но намного более агрессивную. Страх перед ними был
абсолютно оправдан. Yahoo видела, как они разрушили первую
интернет-компанию — Netscape. Было бы логично предположить, что если
бы они попытались стать такими, как Netscape, они бы разделили их
судьбу. Как они могли догадаться, что Netscape была последней жертвой
Miscrosoft?

Было разумнее претендовать на роль медийной компании, чтобы сбить
Microsoft с толку. Но, к сожалению, Yahoo попыталась быть медийной
компанией. Например, менеджеры проектов в Yahoo назывались
продюсерами, а различные части компании — владениями. Что было
действительно нужно Yahoo — это быть технологической компанией, но они
пытались быть чем-то иным, и все закончилось тем, что они не стали ни
тем, ни другим. Вот почему Yahoo никогда не была компанией с четко
определенной индивидуальностью.

Самым плохим последствием этого было то, что они не воспринимали
программирование всерьез. У Microsoft (в те дни), Google и Facebook
была культура, ориентированная на программистов. Но в Yahoo к
программированию относились как к товару. В Yahoo программное
обеспечение, рассчитанное на конечных потребителей, контролировалось
менеджерами проектов и дизайнерами. Работа программистов заключалась в
том, чтобы взять работу дизайнеров и менеджеров проектов и довести ее
до финальной стадии, превращая ее в код.

Одним из очевидных результатов этой практики было то, что когда Yahoo
создавала что-то, эти вещи часто были не очень хороши. Но это было не
худшей проблемой. Худшей проблемой было то, что они нанимали плохих
программистов.

Microsoft (в те дни), Google и Facebook активно нанимали самых лучших
программистов. Yahoо не делала этого. Они предпочитали хороших
программистов плохим, но они не были сконцентрированы на том, чтобы
нанимать умнейших людей так, как это делали победители. И если вы
примете во внимание то, какой была конкуренция за программистов в то
время, во время Пузыря, неудивительно, что качество их программ было
неровным.

В технологии, если у вас плохие программисты, вы погибли. Я не могу
придумать пример, в котором компания утонула в технической
посредственности, а затем ожила. Хорошие программисты хотят работать с
другими хорошими программистами. Как только качество программистов в
вашей компании падает, вы входите в спираль, из которой невозможно
выбраться. [2]

Yahoo вошла в эту спираль рано. Если в Yahoo и было время, когда она
привлекала таланты с такой же силой, как и Google, это время
закончилось к 1998 году, когда я начал там работать.

Казалось, что компания преждевременно постарела. В большинстве
технологических компаний власть переходит к людям в костюмах и
менеджерам среднего звена. В Yahoo казалось, что они нарочно ускоряют
этот процесс. Они не хотели быть группой хакеров. Они хотели быть
людьми в костюмах. Медийная компания должна управляться людьми в
костюмах.

Когда я впервые посетил Google, там работало около 500 человек,
столько же, сколько было в Yahoo, когда я устроился туда на работу. Но
разница в работе была заметной. Это все еще была культура,
ориентированная на программистов. Я помню, как я разговаривал в
кафетерии с программистами о проблеме манипуляции результатами поиска
(сейчас это известно как SEO), и они спросили: «Что же нам делать?»
Программисты в Yahoo не задали бы этот вопрос. Им и незачем было это
делать — их работой было делать то, что скажут менеджеры. Я помню, как
я возвращался из Google, думая: «Вау, это все еще стартап.»

Мы не можем научиться многому из первой фатальной ошибки Yahoo. Скорее
всего, нужно просто надеяться, что компания избежит эту проблему и не
будет полагаться на ненадежный источник доходов. Но стартапы могут
извлечь хороший урок из второй ошибки. В бизнесе программного
обеспечения вы не можете позволить себе не иметь культуру,
ориентированную на программистов.

Возможно, самый впечатляющий пример культуры, ориентированной на
программистов, который я слышал, привел Марк Цукерберг в Startup
School в 2007 году. Он сказал, что раньше Facebook нанимал
программистов даже на те позиции, которые не предполагают
программирования, такие, как HR и маркетинг.

Так каким компаниям нужна культура, ориентированная на программистов?
Какие компании заняты в бизнесе программного обеспечения? Как
показывает опыт Yahoo, сфера применения этого правила значительно
шире, чем предполагает большинство. Ответ такой: любая компания,
которая нуждается в хорошем программном обеспечении.

Как великие программисты могут захотеть работать в компании, в которой
нет культуры, ориентированной на программистов, в то время как в
других компаниях она есть? Я могу придумать две причины: им хорошо
платят, или сфера деятельности сама по себе интересна, и в этой сфере
нет компаний с такой культурой. Никак иначе вы не можете привлечь
хороших программистов работать в культуре, ориентированной на людей в
костюмах. А без хороших программистов не удастся производить хорошие
программы, и неважно, сколько людей при этом будут работать над
проблемой и сколько процессов будут обеспечивать «качество».

Культура, ориентированная на программистов, часто кажется
безответственной. Вот почему люди, которые предлагают уничтожить ее,
часто используют такие фразы, как «зрелая система контроля». Это
фраза, которую использовали в Yahoo. Но есть вещи, значительно худшие,
чем кажущаяся безответственность. Поражение, например.

[1] Ближе всего мы подошли к таргетингу, когда я там работал, когда мы
создали pets.yahoo.com, чтобы спровоцировать войну между тремя
стартапами, которые занимались кормами для животных, за место
генерального спонсора.

[2] Теоретически вы можете выйти из спирали, покупая хороших
программистов, вместо того, чтобы нанимать их. Вы можете заполучить
программистов, которые никогда бы не пришли устраиваться к вам на
работу, покупая их стартапы. Но компании, у которых хватает ума
сделать это, не нуждаются в этом.

\end{document}
