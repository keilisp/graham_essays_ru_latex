\documentclass[ebook,12pt,oneside,openany]{memoir}
\usepackage[utf8x]{inputenc} \usepackage[russian]{babel}
\usepackage[papersize={90mm,120mm}, margin=2mm]{geometry}
\sloppy
\usepackage{url} \title{На смену корпоративной лестнице} \author{Пол
  Грэм} \date{}

\begin{document}
\maketitle

Тридцать лет назад нужно было прокладывать себе путь по служебной
лестнице. Сейчас это уже не считается правилом. Наше поколение хочет
получать деньги на передовых позициях. Вместо того, чтобы разработать
продукт для какой-нибудь большой компании в ожидании получения
гарантии рабочего места, мы сами разрабатываем продукт, в стартапе, и
продаем его большой компании. По меньшей мере мы хотим иметь выбор. \newline

Помимо прочего этот сдвиг породил появление стремительного роста
экономического неравенства. Но на самом деле эти два примера не
настолько разные, как это представлено в экономической статистике. \newline

Экономическая статистика вводит в заблуждение, потому что она
игнорирует значение безопасных рабочих мест. Легкая работа, с которой
нельзя уволить, стоит денег; обмен на вторую работу это одна из
наиболее распространенных форм коррупции. Синекура (Хорошо
оплачиваемая должность, не требующая большого труда) — это эффект,
аннуитет (или финансовая рента — общий термин, описывающий график
погашения финансового инструмента, выплаты вознаграждения или уплаты
части основного долга и процентов по нему). Только вот синекуры не
появляются в экономической статистике. Если бы они появлялись, было бы
ясно, что на практике социалистические страны имеют нетривиальное
неравенство достатка, потому что они, как правило, имеют мощный класс
бюрократов, которым платят в основном за выслугу лет, и которые
никогда не могут быть уволены. \newline

Между тем не синекура, а положение на служебной лестнице, было
действительно ценным, потому что большие компании старались не
увольнять людей и повышать их в должности, базируясь в основном на
стаже работы. Положение на служебной лестнице имело значение
аналогичное «гудвил» (goodwill), которое в самом деле довольно часто
используется как реальный элемент в оценке компаний. Это означало, что
можно ожидать в будущем высокооплачиваемую работу. \newline

Одной из главных причин распада карьерной лестницы является тенденция
поглощений, которые начались в 1980-х годах. Зачем тратить свое время
на лестнице, которая может исчезнуть, прежде чем вы достигнете
вершины? \newline

И не случайно карьерная лестница стала одной из причин, по которой
ранние корпоративные рейды были так успешны. Не только экономическая
статистика игнорирует значения безопасных рабочих мест. Корпоративные
балансы тоже так делают. Причиной того, что делить компании и
продавать их по частям было выгодно в 1980-ых заключается в том, что
они формально едва признавали их подразумеваемый долг перед
сотрудниками, которые проделали хорошую работу и ждали за нее награду
в виде высокооплачиваемых должностей, когда придет их время. В фильме
Уолл-Стрит, Гордон Гекко высмеивает компании, перегруженные
вице-президентами. Но компания может быть не такой коррумпированной,
как кажется; эти удобные рабочие места вице-президентов были,
вероятно, оплатой за проделанную ранее работу. \newline

Мне больше нравится новая модель. С одной стороны, относиться к
должностям как к наградам кажется плохой идеей. Много хороших
инженеров стали плохими менеджерами таким образом. И старая система
подразумевала, что люди должны были столкнуться с гораздо большим
количеством корпоративной политики, для того чтобы защитить те усилия,
которые они вложили в свое положение на карьерной лестнице. \newline

Большой минус новой системы заключается в том, что она включает в себя
больше риска. Если вы развиваете идеи в стартапе, а не в большой
компании, любое количество случайных факторов может потопить вас,
прежде чем вы можете закончить. Но, возможно, старшее поколение будет
смеяться надо мной, когда я скажу, что способ, с помощью которого мы
все выполняем более рискованный. Ведь проекты внутри крупных компаний
постоянно аннулировались в результате волевых решений сверху. Вся
промышленность моего отца (размножающие реакторы) исчезла таким
образом. \newline

Так или иначе сама идея служебной лестницы вероятно навсегда канула в
лету. Новая модель кажется более ликвидной и более эффективной. Но это
не такое уж и изменение, финансово, как можно подумать. Наши отцы не
были настолько глупы. 

\end{document}
