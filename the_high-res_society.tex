\documentclass[ebook,12pt,oneside,openany]{memoir}
\usepackage[utf8x]{inputenc} \usepackage[russian]{babel}
\usepackage[papersize={90mm,120mm}, margin=2mm]{geometry}
\sloppy
\usepackage{url} \title{Высоко-технологичное общество} \author{Пол
  Грэм} \date{}
\begin{document}
\maketitle

На протяжении почти всей истории человечества успех общества
определялся его способностью создавать большие организации,
подчинённые дисциплине. Выигрывали те, кто делал ставку на сокращение
расходов за счет роста производства, а это значит, что крупнейшие
организации были самыми успешными.

Сегодня всё сильно изменилось, и нам уже трудно поверить, что всего
лишь несколько десятилетий назад крупнейшие организации стремились
быть на волне прогресса. В 1960 году амбициозный выпускник колледжа
хотел работать в огромных, сверкающих офисах «Форда», или «Дженерал
электрик», или НАСА. Малый бизнес означало мелкий. Малый в 1960 не
значило крутой стартап. Это значило обувной магазин дяди Сида.

Когда я рос в 1970-х, идея «служебной лестницы» была всё ещё жива.
Стандартный жизненный план состоял в том, чтобы поступить в хороший
колледж, закончив который, можно было получить приглашение в
какую-нибудь организацию и затем подниматься по служебной лестнице.
Более амбициозные отличались только тем, что надеялись подниматься по
этой лестнице быстрее. [1]

Но в конце XX века кое-что изменилось. Оказалось, что сокращение
расходов за счёт роста производства стало не единственной силой в
промышленности. Особенно в технологии, преимущество скорости, которую
можно было получить от маленьких организаций, стало превосходить
преимущества размера.

Будущее оказалось не таким, как мы ожидали в 1970-х. Мы думали о
городах, покрытых куполом, и летающих автомобилях, но этого не
случилось. К счастью у нас есть спецодежда, навыки, умения и
специальность. Всё выглядит так, что вместо того, чтобы быть в руках
нескольких гигантских, разветвлённых организаций, экономика будущего
будет представлять гибкую сеть небольших независимых единиц.

Конечно, большие организации никуда не делись. Скорее всего, такие
знаменитые, успешные организации как Римская армия, Британская
Ост-Индская компания страдали от условностей и интриг, как и
современные компании такого же размера. Но они конкурировали с
оппонентами, которые не могли менять правила на лету, используя новые
технологии. Сегодня правило «побеждают большие и дисциплинированные
организации» должно быть дополнено: «в играх, где правила меняются
медленно». Никто не догадывался об этом, пока изменения не набрали
достаточную скорость.

Большие организации оказываются в проигрыше, потому что теперь они не
получают лучшее. Амбициозные выпускники колледжей сегодня не хотят
работать на большую компанию. Они хотят работать на быстро растущий
стартап. А если они по-настоящему амбициозны, то начинают свой.[2]

Это не значит, что большие компании исчезнут. Когда говорят, что
стартапы будут преуспевать, подразумевается, что большие компании
будут существовать, потому что успешный стартап либо сам становится
крупной компанией, либо приобретается большой фирмой. [3] Но большие
организации, видимо, никогда больше не будут играть ту ведущую роль,
какую они имели до последней четверти XX века.

Неожиданно, что тенденция, которая наблюдалась так долго,
прекратилась. Как часто случается, что правило, работавшее тысячи лет,
меняется на противоположное?

Тысячелетний забег под лозунгом «больше – лучше», оставил нас с кучей
традиций, которые на сегодня устарели, но очень глубоко укоренились.
Это значит, что стремящийся к успеху человек может пересмотреть их.
Очень важно точно понять, какие традиции принять, а какие можно
отвергнуть.

Распространение малых организаций началось в мире стартапов.

Всегда бывали отдельные случаи, в частности в США, когда честолюбивый
человек начинал своё дело и служебная лестница вырастала под ним;
вместо того, чтобы начать снизу и годами карабкаться вверх. Однако, до
недавнего времени, это был нетипичный путь, которому следовали только
дилетанты. Совсем неслучайно, что великие промышленники XIX века были
мало образованы. Пока их компании разрастались до огромных размеров,
они сами оставались механиками или лавочниками. Это была социальная
ступень, которую выпускник колледжа не занял бы никогда, если бы мог
этого избежать. До появления стартапов и, в частности,
интернет-стартапов, для образованных людей было очень необычно начать
свой бизнес.

Те восемь человек, которые оставили «Шокли семикондактор» (Shockley
Semiconductor) и основали «Феирчайлд семикондактор» (Fairchild
Semiconductor), свежий стартап в Кремниевой долине, поначалу даже не
пытались создать компанию. Они искали такую компанию, которая примет
их на работу как группу. Затем один из родителей ребят представил их
владельцу небольшого инвестиционного банка, который предложил
профинансировать их, если они начнут свой бизнес. И они начали. Однако
идея начать своё дело была чужда им, они были вынуждены сделать это.
[4]

С уверенностью предположу, что практически каждый выпускник Стенфорда
или Беркли, который знает как программировать, как минимум обдумывал
идею начать стартап. Так же дело обстоит в университетах Восточного
побережья и Великобритании. Эта картина показывает направление, в
котором движется мир.

Конечно интернет-стартапы лишь малая толика мировой экономики. Может
ли тенденция, наблюдаемая на их примере, быть столь могущественна?

Именно так я и думаю. Нет никаких оснований полагать, что есть
какие-то ограничения роста в этой области. Как и наука, благосостояние
растёт рекурсивно (1). Паровая энергетика была крошечной частью
Британской экономики, когда Ватт занялся ею. Но его изобретение
распространялось до тех пор, пока не захватило всю экономику.

То же самое может произойти и с интернетом. Если интернет-стартапы
предложат амбициозным людям лучшие возможности, тогда много
амбициозных людей займётся ими, и эта малая часть экономики
разрастётся рекурсивно.

Даже если проекты, связанные с интернетом, составят десятую часть
экономики, они зададут тон. Самая динамичная часть экономики всегда
задаёт тон во всём: от зарплат, до дресс-кода. Не только потому, что
это престижно, а ещё и потому, что принципы, лежащие в основании самой
динамичной части экономики, работают.

Тенденция ближайшего будущего, на которую стоит делать ставку, сети
небольших автономных групп, чья эффективность индивидуальна. И в
выигрыше окажется общество, которое создаёт им меньше препятствий.

Так же, как во время индустриальной революции, некоторые общества
окажутся более эффективными. Зародившись в Англии, при жизни одного
поколения промышленная революция распространилась по Европе и Северной
Америке. Но она не распространилась дальше. Этот новый образ жизни мог
укорениться только там, где была соответствующая почва. Необходим был
энергичный средний класс.

Аналогичный социальный компонент был нужен для трансформации,
начавшейся в Кремниевой долине в 1960-х. Здесь зарождались две новые
технологии: производство интегральных схем и создание компаний нового
типа, быстро растущих благодаря внедрению новых технологий.

Производство интегральных схем быстро распространилось и в другие
страны. Но технология стартапов – нет. Пятьдесят лет спустя стартапы
широко распространены в Кремниевой долине, встречаются кое-где в США,
и совершенно чужды остальному миру.

Одна из причин – возможно главная причина – того, что стартапы не
распространились как промышленная революция, в их социальной
деструктивности. Внеся много изменений, промышленная революция не
затронула принцип «больше – лучше». Напротив, она продолжила его.
Новые промышленные компании приспособились к традициям существования
больших организаций военных или гражданских, и это прекрасно работало.
«Капитаны индустрии» отдавали приказы «армиям рабочих», и все знали,
чего от них ждут.

Стартапы идут против основ общества. Им трудно расцвести в обществах,
где в цене иерархия и стабильность, так же как индустриализации было
трудно укорениться в обществах, где торговый класс не имел власти. Но
к тому времени, как началась промышленная революция, был целый ряд
стран, прошедших эту стадию. Но, похоже, в сегодняшней ситуации их
нет.

(еще свеженькое — Пол Грэм: «Почему Y Combinator?»)

[1] Из этой модели вытекало то обстоятельство, что для того, чтобы
зарабатывать больше, надо было становиться руководителем. Это правило
сохраняется и в стартапах.

[2] Есть много причин, почему американские автомобильные компании хуже
японских, но по крайней мере в одном они превосходят производителей
авто Страны восходящего солнца: у американских специалистов больше
выбор вариантов.

[3] Возможно, когда-нибудь компании смогут вырасти в большие по
доходам, а не по количеству сотрудников, но пока в этом направлении мы
ушли недалеко.

[4] Лекуер, Кристоф «Рождение Кремниевой долины», МИТ Пресс, 2006.
(Lecuyer, Christophe, Making Silicon Valley, MIT Press, 2006.)

\end{document}
