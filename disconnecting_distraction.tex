\documentclass[ebook,12pt,oneside,openany]{memoir}
\usepackage[utf8x]{inputenc} \usepackage[russian]{babel}
\usepackage[papersize={90mm,120mm}, margin=2mm]{geometry}
\sloppy
\usepackage{url} \title{Отключившись от отвлекающих факторов}
\author{Пол Грэм} \date{}
\begin{document}
\maketitle

Фактор, отвлекающий вас от работы, коренится в разного рода
развлечениях, мешающих вам сконцентрироваться. Большинство людей
испытывают дискомфорт, если сидят без дела; вы уклоняетесь от работы,
занимаясь чем-то другим.

Итак, чтобы перестать уклоняться от работы, необходимо устранить то,
что мешает вам сконцентрироваться. Но это не так просто, как кажется,
потому что есть люди, которые усердно работают, чтобы отвлекать вас от
работы, развлекая. Развлечение — это не есть некое статическое
препятствие, которое вы можете обойти, как, к примеру, камень, лежащий
на дороге. Развлечение ищет вас.

Честерфилд (Chesterfield) описывает грязь как вещество, оказавшееся в
неподобающем месте. Точно так же, мы желаем развлечений в неподобающее
время. А технологии по созданию все большего количества желаемых вещей
постоянно совершенствуются. Это означает, что как только мы научились
игнорировать один вид развлечений, на смену им постоянно приходят
новые, подобно бактериям, устойчивым к лекарственным препаратам.

Например, телевидение после 50 лет совершенствования достигло той
точки, в которой оно стало наркотиком для глаз. Когда мне было 13 лет,
я понял, что телевидение вызывает пристрастие, и я перестал смотреть
телевизор. А недавно я прочел, что средний американец смотрит
телевизор 4 часа в день. Четверть своей жизни.

Сейчас телевидение находится в упадке, но только потому, что люди
нашли новые способы впустую тратить время, вызывающие еще большее
пристрастие. И что особенно опасно, многие из этих способов связаны с
вашим компьютером. И это не случайно. Огромный процент офисных
служащих сидят перед компьютерами, подсоединенными к интернету, и к их
услугам самые разные вещи, отвлекающие от работы.

Я помню, было время, когда компьютеры, по крайней мере, для меня,
служили исключительно для работы. Я мог иногда соединиться с сервером,
чтобы получить электронную почту или скачать файлы, но большую часть
времени я был отключен. Все, что я делал, так это писал и
программировал. Сейчас у меня такое чувство, словно кто-то украдкой
поставил телевизор на мой рабочий стол. Вещи, вызывающие ужасно
сильное привыкание, находятся от меня на расстоянии лишь щелчка мышью.
Столкнулись с трудностью в том, над чем работаете? Гм, интересно, а
что новенького в онлайн? А не проверить ли мне?

После многих лет, в течение которых я тщательно пытался избежать
классических видов напрасной траты времени таких, как телевидение,
игры и Usenet, я все же ухитрился пасть жертвой отвлекающих факторов,
поскольку я не осознавал, что они эволюционируют. То, что обычно было
безопасным — использование интернета — постепенно становилось все
более и более опасным. В иные дни я просыпаюсь, выпиваю чашку чая и
проверяю новости, затем проверяю почту, затем снова проверяю новости,
потом отвечаю на несколько сообщений, и вдруг обнаруживаю, что уже
обеденное время, а я еще ничего не сделал по работе. И это стало
происходить все чаще и чаще.

Мне потребовалось на удивление много времени, чтобы понять, насколько
отвлекающим стал интернет, потому что проблема возникала лишь время от
времени. Я игнорировал ее, как мы игнорируем какого-нибудь жука,
который надоедает нам лишь время от времени. Когда я работал над
проектом, отвлекающие факторы не были проблемой, на самом деле. Вот
когда я закончил один проект, и раздумывал над тем, что делать дальше,
они меня и обуздали.

Еще одна причина, почему было трудно заметить опасность этого нового
отвлекающего фактора, так это то, что он еще не стал одной из привычек
на уровне социума. Если бы я провел все утро, сидя на диване и
уставившись в телевизор, я бы очень быстро это заметил. Это известный
признак опасности, как выпивать в одиночку. Но пользование интернетом
все еще выглядит как работа, и дает ощущение, что ты занят работой.

В конце концов, все же стало ясно, что интернет становится все более и
более отвлекающим фактором, поэтому я был вынужден изменить свое
отношение к нему. По сути, я должен был добавить новое приложение в
мой список известных отвлекающих от дела факторов, поглощающих время:
Firefox.

Это одна из трудноразрешимых проблем, потому что интернет так или
иначе нужен большинству людей. Если вы слишком много пьете, вы можете
решить эту проблему, полностью прекратив выпивать. Но вы не можете
решить проблему переедания, прекратив есть. Я не мог полностью
игнорировать интернет, как я сделал это в отношении других пустых
времяпрепровождений.Сначала я попытался ввести правила. Например, я
говорил себе, что буду заглядывать в интернет только два раза в день.
Но я недолго придерживался этих правил. Иногда все равно что-то такое
возникало, что заставляло меня пользоваться интернетом чаще, чем я
устанавливал для себя. И тогда я постепенно скатывался на прежние
позиции.

К вещам, вызывающим пристрастие, необходимо относится как к умному
противнику; словно в вашей голове сидит маленький человечек, который
постоянно держит наготове весьма и весьма убедительные аргументы в
пользу того, чтобы сделать то, что вы стараетесь прекратить делать. И
если вы оставите хотя бы маленькую лазейку, он обязательно ее найдет.

Ключом, кажется, является доступность. Нежелание признавать очевидное
мешает нам избавляться от большей части плохих привычек. Поэтому
необходимо признать очевидное, чтобы больше не делать то, что вы
стараетесь прекратить делать. Нужно включать сигнал тревоги.

Может быть, в будущем правильным ответом на вопрос, как управляться с
поглощающими время отвлекающими факторами, предлагаемыми интернетом,
будет создание программы, которая будет отслеживать и контролировать
их. Ну, а пока я нашел более радикальное решение, которое определенно
работает: я установил еще один компьютер специально для пользования
интернетом.

Теперь на моем основном компьютере wifi включается лишь тогда, когда
мне надо переслать файл или отредактировать веб-документ; в другом
углу моей комнаты находится еще один лэптоп, который я использую для
проверки сообщений или чтобы «поползать» по интернету. (Парадокс в
том, что это именно тот компьютер, на котором Стив Хаффман (Steve
Huffman) написал Реддит. Когда Стив и Алексис продавали свои лэптопы
на благотворительном аукционе, я купил их для музея Y Combinator).

Мое правило заключается в следующем: я могу проводить столько времени
в онлайн, сколько захочу, при условии, что я делаю это на этом
компьютере. И этого оказалось достаточно. Когда мне приходится сидеть
в этом углу комнаты, проверяя почту или просматривая файлы в сети, я
четко сознаю это. По крайней мере, в моем случае, осознание того, что
трудно провести более часа в онлайн, является действительно четким.

А мой основной компьютер теперь свободен для работы. Если вы
попробуете это трюк, вы, вероятно, будете ошеломлены совершенно новым
ощущением, возникающим из-за того, что ваш компьютер отключен от
интернета. Это новое чувство, охватившее меня, когда я сел за
компьютер, который можно было использовать только для работы,
встревожило меня, поскольку я понял, как много времени я, должно быть,
потратил впустую.

Круто! Все, что я могу делать на этом компьютере, так это — работать.
Ну ладно, в таком случае, я, пожалуй, поработаю.В чем же положительный
эффект? Ваши старые привычки теперь помогут вам работать. Вы привыкли
сидеть за вашим компьютером часами. Но теперь вы не можете бродить по
интернету или проверять почту. Так что вы намерены делать? Вы не
можете просто сидеть. Следовательно, вы начинаете работать.

\end{document}
