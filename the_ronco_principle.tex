\documentclass[ebook,12pt,oneside,openany]{memoir}
\usepackage[utf8x]{inputenc} \usepackage[russian]{babel}
\usepackage[papersize={90mm,120mm}, margin=2mm]{geometry}
\sloppy
\usepackage{url} \title{Принципы Ронко} \author{Пол Грэм} \date{}
\begin{document}
\maketitle

Никто из венчурных капиталистов или бизнес-ангелов не вложил в топовые
стартапы больше, чем Рон Конвей. Он знает, о каждой произошедшей в
Долине сделке, тем более половину из них он осуществил сам.

Он отличный парень. На самом деле, «отличный» это даже не то слово.
Рон — лучший. Я не знаю ни одного случая, когда он повел бы себя
скверно. Такое трудно себе даже представить.

Когда я впервые приехал в Кремниевую долину я подумал: «Какое счастье,
что есть некто настолько влиятельный, насколько и великодушный.» Но
постепенно я понял, что это не простое стечение обстоятельств. Только
будучи великодушным, Рон смог стать настолько влиятельным. Все
проинвестированные им проекты пришли к нему по рекомендации от других
людей. В длинном списке таких компаний есть Google, Facebook и
Twitter. Причина того, что так много людей предпочитают иметь с ним
общие дела в том, что он проявил себя как достойный парень.

Но не стоит думать, что он слабак. Я бы не хотел столкнуться с Роном в
гневе. Он может сердиться на вас, только если вы что-то не так
сделали. Рон Конвей — старой закалки, такой же как Ветхий Завет. Без
всякой злобы он поразит вас в своем праведном гневе.

Почти в каждой сфере есть свои выгоды казаться достойным. Такая
практика заставляет людей довериться вам, но на самом деле быть
достойным — слишком дорогостоящий способ, чтобы просто казаться
таковым. Для безнравственного человека может показаться слишком
дорогостоящим.

В некоторых отраслях — может быть, но, несомненно, не в мире
стартапов. Среди инвесторов множество подонков, но есть четкая
тенденция: самые успешные инвесторы всегда достойные люди. Сортировка
инвесторов по доброжелательности не эквивалентна сортировке по
прибыли, но если по оси Х отложить положительность инвестора, а по оси
У прибыли, то можно увидеть явную тенденцию к росту.

Но так было не всегда. Не уверен, что двадцать лет назад я мог бы
говорить об инвесторах тоже самое.

Что изменилось? Мир стартапов стал куда более прозрачным и
непредсказуемым. Казаться достойным человеком гораздо трудней, не
будучи таковым.

Ясно почему прозрачность оказывает такой эффект. Сегодня когда
инвестор отвратительно обращается с предпринимателями, это сразу
становится достоянием общественности. Может быть не все попадает в
прессу, но другие предприниматели все равно узнают об этом, и это
приводит к тому, что инвестор начинает терять сделки. Y Combinator, в
частности, объединяет данные многих стартапов и имеет довольно полное
представление о поведении инвесторов.

Эффект непредсказуемости является более изысканным, усиливающим
несовместимость подходов. Лицемерие становится крайне сложной штукой,
вы всегда должны знать когда быть любезным, а когда неприятным типом.
В мире стартапов, все меняется очень быстро. Случайный студент с
которым вы разговариваете сегодня, может через пару лет оказаться
генеральным директором самого «горячего» стартапа в Долине. Если вы не
знаете когда нужно быть любезным, вы должны быть таким все время.
Наверное, только искренне хорошие люди способны на это.

В полностью связном и непредсказуемом мире, вы не сможете просто
казаться достойным человеком.

Как это часто бывает, Рон узнал, как быть хорошим инвестором по
случайному стечению обстоятельств. Он не предвидел будущее стартап
инвестирования. Просто вел себя таким образом, чтобы не было стыдно.
Он чувствовал себя неестественно, если ему приходилось вести себя
по-другому. Он уже живет в будущем.

К счастью, будущее не ограничивается миром стартапов. Он просто более
прозрачен и непредсказуем, чем большинство, но практически везде
тенденция одинакова.
\end{document}
