\documentclass[ebook,12pt,oneside,openany]{memoir}
\usepackage[utf8x]{inputenc} \usepackage[russian]{babel}
\usepackage[papersize={90mm,120mm}, margin=2mm]{geometry}
\sloppy
\usepackage{url} \title{Измени свое имя} \author{Пол Грэм} \date{}
\begin{document}
\maketitle

Если у вас есть американский стартап с названием Х, и нет прав на
домен x.com, вам, вероятно, следует сменить название.

Причина не столько в том, что люди не смогут найти вас. Для компаний с
мобильными приложениями, особенно, имеющих право на соответствующее
доменное имя не так важно, как то, как оно будет использоваться для
получения пользователей. Проблема с отсутствием одноименного домена
заключается в том, что его отсутствие показывает вашу слабость. Если
вы не настолько велики, что ваша репутация опережает вас,
второстепенный домен предполагает, что вы второстепенная компания. В
то же время, наличие домена x.com сигнализирует о вашей силе, причем
даже если домен не имеет никакого отношения к тому, что вы делаете.

Даже хорошие основатели могут отрицать этот факт, но это отрицание
стоит на двух мощных фундаментальных причинах: идентичность и
отсутствие воображения.

Первое говорит о том, что Х, напрямую характеризует идею основателей,
а второе о том, что не существует никакого другого подходящего имени.
И первое, и второе утверждение неверны.

Для начала можно исправить первую причину этой проблемы. Представьте,
что вы бы назвали свою компанию как-нибудь иначе. Если так, конечно,
вы бы так же просто добавили бы к этому имени нужный домен. Идея
переключения текущего имени довольно отталкивающая.

По существу ничего великого в текущем названии нет. Почти вся ваша
привязанность к нему проистекает из вашего к нему отношения. Название
дорого для вас поскольку стало частью вашей идентичности и иногда этот
факт просто сбрасывается со счетов под некими благовидными
представлениями, вне зависимости от внутренних качеств.

Чтобы нейтрализовать вторую причину отрицания, вашу неспособность
придумывать другие потенциальные наименования, нужно признать, что с
неймингом у вас плохо. Навык присваивания имён является совершенно
отдельной способностью. Вы можете быть крутым стартапером, но
совершенно безнадежным в деле изобретения имени вашей компании.

После того, как вы признаете это, вы перестаете верить в невозможность
смены наименования. Существует множество потенциально лучших названий,
просто вы не можете придумать их.

Как их найти? Простейший ответ решения проблем, в которых вы не
сильны, заключается в том, чтобы привлечь к процессу кого-то еще. Но с
названиями компаний возможен еще один подход. Оказывается почти любое
слово или словосочетание не являющееся очевидно плохим, достаточно
хорошо для этого, и количество таких доменов является числом настолько
большим, что можно найти множество достаточно дешевых или вообще
неиспользованных доменов. Так что составьте список и попытайтесь
прикупить несколько.

Причиной моего знания о том, что создание стартапов и их названия
ортогональные навыки является то, что в названиях я сам не силен.
Раньше, когда я создавал Y Combinator и проводил много времени со
стартапами, я часто помогал им находить новые названия.

Теперь, когда я нахожусь в офисе, я сосредотачиваюсь на более важных
вопросах, например на основной деятельности компании. Я советую им,
когда им стоит изменить свое имя, но я знаю, что иногда моих советов
не достаточно. Бывает, что стартаперы осознают проблему отсутствия
одноименного домена, но, ошибаясь, они считают, что позже смогут его
приобрести, без всяких на то оснований. Не верьте, что домен продается
до того момента, пока владелец не сообщит вам желаемую цену.

Конечно, имеются примеры стартапов, добившихся успеха, не имея
одноименного домена. Есть стартапы, добившиеся успеха, вообще
наперекор различным ошибкам, но ошибка с названием менее простительна,
чем большинство других. Она может быть исправлена в течение нескольких
дней, если вы достаточно дисциплинированы для признания проблемы.

100\% компаний лучшей двадцатки Y Combinator обладают одноименными
доменами. 94\% из пятидесяти тоже. И лишь 66\% из общего количества
компаний пакета Y Combinator владеют одноименным доменом в зоне .com.
Так или иначе стоит извлечь урок из опыта большинства лучших.

\end{document}
