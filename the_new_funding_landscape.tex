\documentclass[ebook,12pt,oneside,openany]{memoir}
\usepackage[utf8x]{inputenc} \usepackage[russian]{babel}
\usepackage[papersize={90mm,120mm}, margin=2mm]{geometry}
\sloppy
\usepackage{url} \title{Супер-ангелы} \author{Пол Грэм} \date{}
\begin{document}
\maketitle

После едва заметных изменений в течение последних десяти лет, бизнес
по финансированию стартапов сейчас находится в таком состоянии, что
можно было бы, по крайней мере в сравнении, назвать беспорядком. Мы
увидели значительные изменения в среде финансирования стартапов в Y
Combinator (YC). К счастью, одним из них можно назвать гораздо более
высокие оценки стоимости.

Тенденции, которые мы увидели, возможно, конкретно для YC не
специфичны. Жаль, что я не могу сказать об этом в прошедшем времени,
но основная причина, скорее всего, заключается в том, что мы сначала
видим тенденции частично потому, что стартапы, финансируемые нами,
очень сильно вплетены в структуру Кремниевой долины и быстро
используют преимущество разных нововведений, а также частично из-за
того, что мы так много вкладываем, что у нас достаточно данных, чтобы
четко увидеть всю схему.

То, что мы наблюдаем сейчас, вероятно, все увидят в следующие два
года. Поэтому я объясню то, что мы видим, и что это будет означать для
вас, если вы попытаетесь заполучить субсидирование.

Супер-ангелы

Позвольте начать с описания того, на что был некогда похож мир
финансирования стартапов. Существовали два резко отличающихся типа
инвесторов: бизнес-ангелы и венчурные капиталисты. Бизнес-ангелы – это
отдельные богатые люди, которые вкладывают небольшие суммы из
собственных денег, в то время как венчурные капиталисты – это
сотрудники фондов, которые инвестируют огромные средства,
принадлежащие другим людям.

В течение десяти лет были только эти два типа инвесторов, но сейчас
появился третий промежуточный тип: так называемые супер-ангелы. [1] Их
появляение спровоцировало венчурных капиталистов на реализацию
большого числа инвестиций как у бизнес-ангелов. Поэтому ранее четкая
грань между бизнес-ангелами и венчурными инвесторами стала безнадежно
размытой.

Ранее между бизнес-ангелами и венчурными капиталистами была
нейтральная зона. Бизнес-ангелы инвестируют от 20 до 50 тысяч долларов
за раз, а венчурные капиталисты обычно от миллиона и более. Таким
образом, раунд финансирования бизнес-ангела подразумевал накопление
ангельских инвестиций, которые в сумме составляли, может быть, 200
тысяч долларов, а раунд финансирования венчурного капиталиста
представлял собой раунд серии А, в котором один венчурный фонд (или
изредка два) инвестирует 1-5 миллионов долларов.

Нейтральная зона между бизнес-ангелами и венчурными капиталистами была
достаточно неудобным обстоятельством для стартапов, т.к. она совпадала
с тем количеством денег, которые многие хотели заполучить. Большинство
стартапов, представленных на Demo Day (закрытое мероприятие, на
котором группа компаний, финансируемых YC, представляют самих себя
специально приглашенному кругу лиц), хотели собрать сумму около 400
тысяч долларов. Но состыковать это требование было крайней сложно,
поскольку сумма была слишком крупной для ангельских инвестиций, а
большинству венчурных капиталистов делать такие мелкие вложения было
неинтересно. Вот главная причина возникновения супер-ангелов. Они
реагируют на потребности рынка.

Появление нового типа инвесторов стало важным известием для стартапов,
потому что было всего два типа, и они редко когда могли конкурировать
друг с другом. Супер-ангелы конкурируют и с теми, и с другими. Это
вскоре изменит правила поиска инвестиций. Я еще не знаю, какими будут
эти правила, но похоже, что большинство этих изменений пойдут только
на пользу.

Супер-ангел обладает некоторыми качествами бизнес-ангела, и некоторыми
качествами венчурного капиталиста. Обычно это отдельный человек, как и
в случае с бизнес-ангелами. На самом деле многие из нынешних
супер-ангелов были изначально бизнес-ангелами классического типа. Но,
как и венчурные инвесторы, они вкладывают средства, принадлежащие
другим людям. Это позволяет им инвестировать суммы крупнее, чем у
бизнес-ангелов: типичное вложение супер-ангела на данный момент
составляет около 100 тысяч долларов. Они быстро принимают
инвестиционные решения, как и бизнес-ангелы. А еще они заключают
гораздо больше инвестиционных сделок в расчете на одного партнера, в
отличие от венчурных капиталистов, — до 10 раз больше.

Тот факт, что супер-ангелы инвестируют деньги других людей, делает их
вдвойне опасными для венчурных капиталистов. Они не просто соревнуются
за стартапы, они также соревнуются за инвесторов. Чем на самом деле
являются супер-ангелы, так это новой формой динамичного, легковестного
венчурного фонда. И некоторые из нас из мира технологий, знают, что
обычно происходит, когда появляется то, что можно описать с данной
точки зрения. Обычно это называют заменой.

А будет ли она? На данный момент, некоторые из стартапов, принимающие
деньги от супер-ангелов, исключают возможность взять деньги у
венчурного капиталиста. Они просто отсрочивают это событие, что все
еще является проблемой для венчурных инвесторов. У некоторых
стартапов, которые откладывают использование таких средств, могут так
хорошо пойти дела на деньги бизнес-ангелов, что им и в голову не
придет просить еще. А те, кто воспользовался финансированием
венчурного капиталиста, сможет получить более высокие оценки
стоимости, когда им снова эти деньги понадобятся. Если самый лучший
стартап получает оценки по стоимости в 10 раз выше, когда они
используют деньги из раундов финансирования серии А, то это сократило
бы для венчурных капиталистов выручку от победителей по меньшей мере в
10 раз. [2]

Таким образом, я считаю, что венчурным фондам всерьез стоит
побеспокоиться о супер-ангелах. Но есть одна вещь, которая может в
некоторой степени их спасти. Это неравное распределение конечных
результатов: фактически вся выручка сконцентрирована в нескольких
стартапах, имеющих большой успех. Математическое ожидание стартапа
есть выраженная в процентах вероятность того, что этот стартап Google.
Поэтому, если воспринимать победу как вопрос о полном возврате
средств, то супер-ангелы могли бы выиграть практически все битвы за
частные стартапы и все же проиграть войну, если бы им просто не
удалось бы заполучить тех нескольких крупных победителей. И существует
возможность, что это могло бы произойти, потому что у топовых
венчурных фондов лучше репутация, и они также могут сделать больше для
своих поддерживаемых компаний. [3]

Поскольку супер-ангелы вкладывают большее количество инвестиций в
расчете на одного партнера, то у них этих партнеров меньше на каждую
инвестицию. Они не могут уделять вам столько же внимания, как мог бы
венчурный капиталист, находясь в составе совета директоров вашего
проекта. Сколько стоит это дополнительное внимание? Оно будет сильно
разниться от одного партнера к другому. В общем случае в данном
вопросе соглашение еще не достигнуто. Поэтому на текущем этапе каждый
стартап решает это индивидуально.

До сегодняшнего момента, жалобы венчурных капиталистов на то, какую
стоимость они добавили, были похожи на правительственные. Возможно,
они дали вам почувствовать себя лучше, но в таком деле у вас не было
бы выбора, если бы вам нужна была такая сумма, какой обеспечить могли
бы только венчурные инвесторы. Теперь, когда у венчурных капиталистов
появились конкуренты, это повлечет за собой установку рыночной цены на
помощь, которую они предлагают. Но что интересно, никто еще не знает,
какой эта цена будет.

Разве стартапам, которые действительно хотят перерости в крупные
проекты, нужны некоторого рода совет и связи, которыми могут
обеспечить только топовые венчурные капиталисты? Или просто подойдут
деньги супер-ангела? Венчурные капиталисты будут говорить, что вы
нуждаетесь в них, а супер-ангелы скажут, что нет. Но правда в том, что
никто пока не знает, даже сами венчурные инвесторы и супер-ангелы.
Все, что известно супер-ангелам, это то, что их новая модель кажется
достаточно перспективной, и ее стоит испробовать, а все, что знают
венчурные капиталисты, это то, что она кажется достаточно
перспективной, чтобы об этом беспокоиться.

Раунды финансирования

Каким бы ни был исход, конфликт между венчурными капиталистами и
супер-ангелами является хорошей новостью для учредителей. И не только
по очевидной причине того, что более высокая конкурентная борьба за
сделки ознаменовывает путь к лучшим их условиям. Изменяется сама
структура этих сделок.

Одним из существенных отличий между бизнес-ангелами и венчурными
инвесторами является размер доли вашей компании, которую они захотят.
Венчурные капиталисты хотят очень большую долю. В раундах
финансирования серии А они захотят треть вашей фирмы, если они смогут
ее получить. Их не особо волнует сколько они за это заплатят, но они
хотят большую долю потому, что количество инвестиций серии А, которые
они могут осуществить, довольно мало. В традиционных инвестициях серии
А, по крайней мере один партнер из венчурного фонда занимает место в
руководстве вашей компании. [4] Поскольку находиться на руководящей
позиции возможно только около пяти лет, и каждый партнер не может
занимать больше десяти за раз, это означает, что венчурный фонд может
совершить только около двух сделок серии А с одним партнером в год. А
это означает, что им нужно заполучить как можно большую долю компании,
какую только можно, в каждом из этих раундов финансирования. Вам
действительно придется быть очень многообещающим стартапом, чтобы
венчурный капиталист потратил бы одну из своих десяти возможностей
быть в числе руководителей проекта всего лишь на малый процент доли
вашей фирмы.

Поскольку обычно бизнес-ангелы не занимают мест в руководстве проекта,
для них такого сдерживающего фактора нет. Они буду счастливы
приобрести даже малый процент доли вашей фирмы. И хотя супер-ангелы во
многих отношениях являются мини венчурными фондами, они сохраняют это
критичное свойство бизнес-ангелов. Они не занимают руководящие места,
поэтому им не нужен большой процент вашей компании.

Хотя это и означает, что вы, соответственно, получите меньше внимания
с их стороны, в остальном это хорошие новости. Учредителям никогда
особо и не нравилось отдавать такую большую долю капитала, которую
запрашивают венчурные капиталисты. Это было слишком большой частью
фирмы, чтобы расстаться с ней в один момент. Большинство учредителей,
заключая сделки серии А, предпочли бы забрать половину этой суммы за
половину пакета акций, а потом посмотреть, какую оценку они бы могли
получить за вторую половину пакета акций после использования первой
части средств для повышения его ценности. Но венчурные капиталисты
такой возможности никогда не предлагали.

Сейчас у стартапов есть другая альтернатива. Теперь легко войти в
ангельские раунды в половину суммы раундов серии А. Многие из
стартапов, что мы финансируем, идут таким путем, и я полагаю, это
будет справедливо для стартапов в целом.

Типичный крупный раунд ангельского финансирования мог быть на 600
тысяч долларов на основе конвертируемых займов с оценкой верхнего
предела в 4 миллиона долларов до выплаты инвестиций. Подразумевая,
что, когда облигация конвертируется в пакет акций (в поздних раундах,
или при приобретении), в этом раунде инвесторы получат 0,6 / 4,6, или
13\% от доли компании. Что намного меньше, чем 30-40\% от доли фирмы,
которые вы обычно отдаете в раундах серии А, если начнете эту
процедуру раньше. [5]

Но преимущество этих раундов на суммы средних размеров не только в
том, что обычно это способствует меньшему разводнению капитала. За
вами также сохраняется больше возможностей для контроля. Почти всегда
после ангельского раунда учредители еще могут влиять на дела фирмы, в
то время как после раунда серии А этим они обычно похвастаться не
могут. Типичная структура совета директоров после раунда серии А
состоит из двух учредителей, двух венчурных капиталистов и
(предположительно) беспристрастного пятого лица. К тому же, условия
серии А обычно дают инвесторам право на запрет различного рода важных
решений, включая продажу компании. У учредителей обычно де-факто
достаточно широкие возможности для контролирования ситуации после
раунда серии А, пока все идет хорошо. Но это не то же самое, что и
просто иметь возможность делать что хочешь, как раньше.

Третье и довольно существенное преимущество ангельских раундов состоит
в том, что сделки на них заключаются с меньшей нервотрепкой. Заключить
сделку на обычный раунд серии А ранее занимало недели, если не месяцы.
Когда венчурная компания может заключать только две сделки с одним
партнером в год, она становится осмотрительнее в своих действиях.
Чтобы заполучить обычный раунд серии А, вам придется пройти через
серию совещаний, что может завершиться полноценной партнерской
встречей, где фирма в целом и даст ответ да или нет. Это действительно
пугающая часть для учредителей: не только потому, что раунды серии А
отнимают так много времени, но и потому, что в конце этого длительного
процесса венчурные инвесторы все еще могут отказать. Шанс быть
отвергнутым после такой встречи в среднем составляет около 25\%. В
некоторых фирмах этот показатель выше 50\%.

К счастью для учредителей, венчурные капиталисты стали намного
проворнее. Сейчас инвесторам Кремниевой долины, вероятнее всего,
потребуется две недели, а не два месяца. Но они все еще не так быстры,
как бизнес-ангелы и супер-ангелы, наиболее решительные из которых
иногда определяются с выбором за несколько часов.

Заключение сделок на ангельский раунд не только проходит быстрее, но
вы также будете получать обратную связь в ходе процесса. Ангельский
раунд не относится к области «все или ничего», как в раундах серии А.
Он состоит из множества инвесторов с различной степенью серьезности
намерений, от честных, которые четко вводят в дело свои ресурсы, до
подонков, которые выдают вам фразы типа «возвращайтесь ко мне
заполнить раунд». Вы обычно начинаете собирать деньги с наиболее
обязательных инвесторов, и старательно избегаете сомнительных, чей
интерес повышается по мере завершения раунда.

Но в каждой точке такого процесса вы знаете, как у вас обстоят дела.
Если инвесторы охладеют к вашему проекту, то, может быть, вы соберете
меньше денег, но когда инвесторы теряют интерес, находясь в ангельском
раунде финансирования, то процесс, по крайней мере, учтиво сходит на
нет, а не рушится у вас на глазах, оставляя ни с чем, как это
происходит, когда вас отвергает венчурный фонд после полноценного
партнерского собрания. Хотя, если кажется, что инвесторы зажглись
вашей идеей, то вы не можете вот так взять и побыстрее завершить раунд
финансирования. Но теперь конвертируемые займы становятся нормой, и
фактически повышают цену, чтобы отразить спрос.

Оценка стоимости

Однако, у венчурных капиталистов есть оружие, которое они могут
использовать против супер-ангелов, и они уже начали это делать.
Венчурные капиталисты тоже начали осуществлять инвестиции, соизмеримые
с бизнес-ангелами. Понятие «раунд ангельского финансирования» не
означает, что все инвесторы, которые в него входят, являются
бизнес-ангелами; оно просто описывает структуру раунда. Все чаще и
чаще в состав участников входят венчурные инвесторы, осуществляющие
вложения на сотню тысяч или две. А когда венчурные капиталисты
вкладываются в ангелькие раунды финансирования, они могут делать то,
чего так не любят супер-ангелы. Венчурные инвесторы довольно
независимы от оценочных процедур в ангельских раундах — частично
потому, что таковыми они являются в целом, и частично потому, что им
не особо важна выручка с ангельских раундов, которые они в основном
все еще расматривают как способ вербовки стартапов на раунды серии А в
дальнейшем. Поэтому венчурные капиталисты, инвестирующие в ангельских
раундах, могут увеличить оценку стоимости для бизнес-ангелов и
супер-ангелов, которые также вкладываются в этих раундах. [6]

Кажется, что некоторых супер-ангелов заботят процессы оценки
стоимости. Некоторые из них отказали стартапам, спонсируемым YC, после
Demo Day, потому что оценки их стоимости были слишком высокими. Это не
было проблемой стартапов; по определению высокая оценка стоимости
означает достаточное количество инвесторов, желающих ее принять. Но
для меня было загадкой, почему супер-ангелы препираются по поводу этих
оценок. Разве они не понимали, что крупная выручка идет всего из
нескольких проектов, обладающих большим успехом, и, следовательно,
гораздо больше значит, какие стартапы вы выбрали, а не сколько вы за
них заплатили?

Поразмыслив над этим некоторое время и обратив внимание на другие
верные признаки, у меня возникла теория, объясняющая, почему
супер-ангелы могуть буть умнее, чем кажутся. Для супер-ангелов имело
бы смысл желать низких оценок, если они надеются вкладываться в
стартапы, которые раньше купят. Если вы надеетесь замутить еще один
Google, вас не должно волновать, достигнет ли оценка его стоимости 20
миллионов. Но если вы ищете фирмы, которых купят за 30 миллионов, вам
будет до этого дело. Если вы вкладываете 20, а компанию покупают за
30, то ваша выручка будет только в 1,5 раза выше. Вполне можно и Apple
купить.

Таким образом, если некоторые из супер-ангелов искали бы фирмы,
которые быстро можно было бы приобрести, это бы объяснило, почему их
волновали бы оценки стоимости. Но зачем им такие искать? Затем, что в
зависимости от значения «быстро», это могло бы быть на самом деле
очень выгодно. Компания, которую приобретают за 30 миллионов, для
венчурного капиталиста считается провалом, но для бизнес-ангела это
вылилось бы в 10-кратную выручку, и даже более того, в быструю
10-кратную выручку. Рентабельность – вот что важно в инвестировании, и
не кратность полученной выручки, а кратность выручки за год. Если
супер-ангел получит 10-кратно увеличенный доход за 1 год, то такая
рентабельность выше, чем та, которую венчурный капиталист надеялся бы
получить от компании, которой понадобилось 6 лет на то, чтобы перейти
в разряд открытых. Чтобы получить аналогичные показатели, венчурному
инвестору пришлось бы учитывать кратность, равную 106, т.е. 1 миллион.
Даже Google не удалось к такому показателю и близко подойти.

Поэтому я думаю, что по крайней мере некоторые из супер-ангелов ищут
компании, которых купят. Это единственное рациональное объяснение
того, почему они фокусируются на правильных оценках стоимости, а не на
правильных фирмах. И если это так, то с ними вести дела нужно не так
как с венчурными капиталистами. Их требования к оценкам стоимости
будут жестче, но более уступчивы, если вы захотите продать проект
по-раньше.

Прогноз

Так кто же победит: супер-ангелы или венчурные капиталисты? Думаю, что
ответом на это будет, некоторые представители из тех, и других. Они
оба станут более похожими друг на друга. Супер-ангелы начнут
инвестировать более крупные суммы, а венчурные капиталисты постепенно
разработают способы для более быстрой орагизации большего количество
инвестиций на меньшие суммы. Спустя 10 лет этих игроков рынка уже
сложно будет различить, и, скорее всего, будут выжившие из каждой
группы.

Что это означает для учредителей? Во-первых, то, что высокие оценки
стоимости, которые сейчас получают стартапы, не будут такими вечно.
Вплоть до того, что такие оценки управляются нечувствительными к ценам
венчурными капиталистами, и они снова упадут, если венчурные инвесторы
станут больше походить на супер-ангелов и начнут меньше проявлять
инициативу в отношении этих оценок. К счастью, если такое и
произойдет, то на этот процесс уйдут годы.

Прогноз на краткосрочный период таков: между инвесторами конкуренция
только возрастет, что для вас является хорошей новостью. Супер-ангелы
попытаются дестабилизировать венчурных капиталистов посредством более
быстрых действий, а венчурные капиталисты попытаются вывести из
равновесия супер-ангелов регулированием оценок стоимости. Что для
учредителей выльется в идеальную комбинацию: быстро завершающиеся
раунды финансирования с высокими оценками стоимости.

Но помните, что чтобы получить такую комбинацию, ваш стартап должен
будет приглянуться как супер-ангелам, так и венчурным капиталистам.
Если вам не кажется, что у вас есть потенциал перейти в разряд
открытых организаций, то у вас не получится использовать венчурных
инвесторов для повышения оценки стоимости ангельского раунда
финансирования.

Существует опасность того, что венчурные капиталисты появятся в
ангельском раунде финансирования: так называемый сигнальный риск. Если
венчурные инвесторы делают это только в надежде на то, что в
дальнейшем будут больше инвестировать, то что тогда произойдет, если
это окажется не так? Это является сигналом для всех о том, что они
считают ваш проект неудачным.

Стоит ли об этом сильно беспокоиться? Значимость сигнального риска
зависит от того, в каком состоянии ваш проект находится. Если к тому
моменту, когда вам понадобятся деньги, у вас на руках будут диаграммы,
отражающие повышение доходов или улучшение торговых дел помесячно вам
не нужно беспокоиться о каких-либо сигналах, посылаемых вашими
нынешними инвесторами. Ваши результаты будут говорить сами за себя.
[7]

Хотя, если в следующий раз вам понадобится деньги, а точных
результатов у вас еще не будет, вам, может быть, нужно больше думать о
сообщении, которое могут посылать ваши инвесторы, если они больше
вкладываться не планируют. Я еще не уверен, насколько сильно вам
придется поволноваться, т.к. вся эта ситуация с венчурными
капиталистами, оперирующими в рамках ангельских инвестиций, возникла
совсем недавно. Но инстинкты подсказывают мне, что вам не придется
сильно беспокоиться. Сигнальный риск на вид как то, о чем так
переживают учредители, а на самом деле не является серьезной
проблемой. Как правило, единственное, что может убить хороший стартап
– это сам стартап. Стартапы причиняют себе вред гораздо чаще, чем,
например, их соперники. Подозреваю, что сигнальный риск тоже из этой
категории.

Единственное, что сейчас делают финансируемые YC стартапы для
смягчения риска использования денег венчурных капиталистов в
ангельских раундах, это не брать слишком много ни у одного из них.
Вероятно, это поможет, если вы можете позволить себе роскошь отклонить
денежное предложение.

К счастью, все у большего количества стартапов появится такая
возможнось. После десятилетий соперничества, которое лучше всего можно
описать как внутреннее, в бизнесе финансирования стартапов,
наконец-то, завяжется настоящая конкурентная борьба. Такое должно
происходить по меньшей мере в течение нескольких лет, а, возможно, и
намного дольше. Если не будет какого-нибудь крупного биржевого краха,
то следующая пара лет будет прекрасным временем для поиска денежных
средств для стартапов. И такая ситуация довольно волнительна, т.к. это
означает, что появится еще больше стартапов.

Примечания

[1] Я также слышал, что их называют «мини-инвесторы» и
«микро-инвесторы». Не знаю, какое именно из наименований закрепится.

Было несколько предшественников. Рон Конвей управлял ангельским фондом
начиная с 90-х годов, и, в некотором смысле, его First Round Capital
ближе к супер-ангелам, чем к венчурному фонду.

[2] Это не сократило бы их общую выручку в десятикратном размере,
потому что инвестирование в дальнейшем, скорее всего, (а) приведет их
к меньшим потерям от неудачных вложений, и (б) не позволит им иметь
такой большой процент владения стартапом, как сейчас. Поэтому сложно
точно предсказать, что случится с их выручкой.

[3] Репутация инвестора (его бренд) происходит в основном из успеха
поддерживаемых им компаний. Таким образом, успешные венчурные
капиталисты обладают преимуществом крупного бренда. И так продолжалось
бы бесконечно, если бы они воспользовались этим для привлечения на
свою сторону всех самых лучших новых стартапов. Но не думаю, что у них
получится. Чтобы заполучить все самые лучшие стартапы, вам придется
сделать гораздо больше, чем просто заставить их вас захотеть. Вам
также придется захотеть их; вам придется распознать их, когда вы их
увидите, а это намного сложнее. Супер-ангелы снимут самые сливки,
которые венчурные капиталисты упустят из виду. А это повлечет за собой
постепенное нивелирование разрыва в восприятии топовых венчурных
капиталистов и супер-ангелов.

[4] Несмотря на то, что в ходе типичного раунда серии А венчурные
капиталисты помещают двух своих партнеров в состав совета директоров
вашей фирмы, существуют признаки того, что венчурные инвесторы могут
начать беречь свою квоту на занятие руководящей позиции, переключаясь
на то, что ранее подразумевалось под руководством ангельских раундов,
состоящим из двух учредителей и одного инвестора. Что также на руку
учредителям, если это означает, что они все еще управляют компанией.

[5] В раунде финансирования серии А вам обычно придется отдать большую
часть фактической доли пакета акций, которую покупают венчурные
капиталисты, т.к. они настаивают на понижении вами ценности этих акций
(путем увеличения их количества), не говоря уже о «запасе вариантов».
Предполагаю, что, все же, такая практика постепенно исчезнет.

[6] Самое лучшее для учредителей, если у них получится заполучить, это
конвертируемый заем без вообще какой-либо оценки верхнего предела. В
этом случае деньги, инвестируемые в ангельском раунде, просто
конвертируются в пакет акций в момент оценки следующего раунда,
неважно, насколько крупного. Бизнес-ангелы и супер-ангелы
недолюбливают неограниченные займы. Они и понятия не имеют насколько
крупную часть компании покупают. Если у фирмы дела идут хорошо и
оценка следующего раунда высока, они в итоге останутся лишь с одной
маленькой долей компании. Поэтому, соглашаясь на неограниченные займы,
венчурные инвесторы, которым наплевать на оценки стоимости в
ангельских раундах, могут выдвигать предложения, с которыми
супер-ангелы просто терпеть не могут состязаться.

[7] Понятно, что сигнальный риск тоже не будет проблемой, если вам
никогда не понадобится больше денег. Но стартапы часто ошибаются на
этот счет.

\end{document}
