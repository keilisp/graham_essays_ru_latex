\documentclass[ebook,12pt,oneside,openany]{memoir}
\usepackage[utf8x]{inputenc} \usepackage[russian]{babel}
\usepackage[papersize={90mm,120mm}, margin=2mm]{geometry}
\usepackage{csquotes}

\sloppy
\usepackage{url} \title{Чему я научился у Hacker News} \author{Пол
  Грэм} \date{}

\begin{document}
\maketitle

Hacker News исполнилось два года на прошлой неделе. Изначально
предполагалось, что это будет параллельный проект — приложение для
оттачивания Arc и место обмена новостями между нынешними и будущими
основателями Y Combinator. Он становился больше и требовал больше
времени, чем я ожидал, но я не сожалею об этом, потому что я многому
научился работая над этим проектом. \newline

\subsection{Рост}

Когда мы запустили проект в феврале 2007 года, в будние дни трафик
составлял около 1600 ежедневных уникальных посетителей. С тех пор он
увеличился до 22000. \newline

Этот темп роста немного выше, чем хотелось бы. Я бы хотел, чтобы сайт
развивался, потому что если сайт не растет хотя бы медленно, то он,
вероятно, уже мертв. Но я не хотел бы, чтобы он достиг роста Digg или
Reddit — в основном потому что это ослабит характер сайта, но также
потому что я не хочу тратить все свое время работая над
масштабированием. \newline

У меня уже достаточно проблем с этим. Помню, изначальной мотивацией
для HN было испытание нового языка программирования и более того
испытание того языка, который ориентирован на эксперименты с дизайном
языка, а не его производительностью. Каждый раз, когда сайт становился
медленным, я поддерживал себя, вспоминая знаменитую цитату Макилроя и
Бентли \newline

\begin{displayquote}
Ключ к эффективности в элегантности решений, а не в переборе всех
возможных вариантов. \newline
\end{displayquote}

и искал проблемные места, которые я мог устранить минимумом кода. Я до
сих пор в состоянии поддерживать сайт, в смысле сохранения прежней
производительности, несмотря на 14-ти кратный рост. Не знаю, как буду
справляться дальше, но вероятно, что-нибудь придумаю. \newline

Это мое отношение к сайту в целом. Hacker News это эксперимент,
эксперимент в новой области. Сайтам такого типа обычно всего несколько
лет. Обсуждению в интернете как таковому всего несколько
десятилетий.Поэтому мы, вероятно, обнаружили лишь малую часть того,
что обнаружим в итоге. \newline

Вот почему я так оптимистично настроен по отношению к HN. Когда
технология настолько новая, существующие решения, как правило, ужасны,
а значит можно сделать что-то гораздо лучше, что в свою очередь
значит, что многие проблемы, которые кажутся неразрешимыми на самом
деле таковыми не являются. В том числе, надеюсь, проблема, которая
преследует многие сообщества: разрушение из-за роста. \newline

\subsection{Спад}

Пользователи беспокоились об этом с тех пор как сайту было всего
несколько месяцев. До сих пор эти опасения были напрасными, но так
будет не всегда. Спад это сложная проблема. Но, вероятно, решаемая;
это не значит, что открытые разговоры «всегда» были уничтожены ростом,
когда «всегда» означает лишь 20 случаев. \newline

Но важно помнить, что мы пытаемся решить новую проблему, потому что
это означает, что мы должны пробовать что-то новое и большинство из
этого, вероятно, не будет работать. Пару недель назад я пытался
отобразить имена пользователей, имеющих наивысший средний счет
комментариев, оранжевым цветом.[1] Это была ошибка. Внезапно культура,
которая была более или менее единой, разделилась на имущих и неимущих.
Я не осознавал насколько объединенной была культура пока не увидел её
разделенной. Было больно на это смотреть.[2] \newline

Поэтому оранжевые имена пользователей не вернутся. (Простите за это).
Но будут и другие идеи, которые так же вероятно сломаются в будущем и
те из них, которые будут работать, вероятно, будут казаться такими же
сломанными как и те, что таковыми не являются. \newline

Пожалуй, самое важное, что я узнал о спаде это то, что он измеряется
скорее в поведении, чем в самих пользователях. Вы хотите скорее
устранить плохое поведение нежели плохих людей.Поведение пользователей
оказывается на удивление податливым. Если ты ждешь от людей, что они
будут вести себя хорошо, они обычно так и делают; и наоборот. \newline

Хотя, конечно, запрет плохого поведения зачастую устраняет плохих
людей, потому что они чувствуют себя неприятно ограниченными в месте,
где они должны вести себя хорошо. Такой способ избавления от них мягче
и, вероятно, эффективнее, чем другие. \newline

Довольно ясно теперь, что теория разбитых окон подходит и по отношению
к общественным сайтам. Теория заключается в том, что малые проявления
плохого поведения поощряют проявления еще более плохого поведения:
жилой район с большим количеством граффити и сломанных окон становится
тем районом, где зачастую происходят грабежи. Я жил в Нью-Йорке, когда
Джулиани представил реформы, которые прославили эту теорию, и
изменения были удивительными. И я был пользователем Reddit, когда
произошло прямо противоположное, а изменения были такими же
впечатляющими. \newline

Я не критикую Стива и Алексис. То что случилось с Reddit не было
последствием пренебрежения. С самого начала у них была политика
цензурирования исключительно спама. К тому же у Reddit по сравнению с
Hacker News были другие цели. Reddit были стартапом, а не сторонним
проектом; их цель заключалась в том, чтобы расти как можно быстрее.
Совместите быстрый рост и нулевую спонсорскую помощь и получите
вседозволенность. Но я не думаю что они бы стали что либо делать
по-другому, если бы им представилась возможность. Судя по трафику,
Reddit гораздо более успешны, чем Hacker News. \newline

Но то что произошло с Reddit, не обязательно случится с HN. Есть
несколько локальных высших пределов. Могут быть места с полной
вседозволенностью а есть места более осмысленные, так же как и в
реальном мире; и люди будут вести себя по-разному в зависимости от
места где они находятся, так же как и в реальном мире. \newline

Я наблюдал это на практике. Я видел людей осуществлявших перекрестную
рассылку в Reddit и Hacker News, которые не поленились записать две
версии, обидное сообщение для Reddit и более сдержанную версию для HN. \newline

\subsection{Материалы}

Существует два основных типа проблем, которые такой сайт как Hacker
News должен избегать: плохие истории и плохие комментарии.И кажется,
ущерб от плохих историй меньше. На данный момент истории, размещенные
на главной странице, еще примерно такие же, как и те, что были
размещены, когда HN еще только начинал свою деятельность. \newline

Я когда-то думал, что мне придется обдумать решения, не позволяющие
всякой бредятине появляться на главной странице, но до сих пор мне не
приходилось этого делать. Я не предполагал, что главная страница будет
оставаться такой замечательной и я до сих пор не совсем понимаю почему
так происходит. Возможно только более осмысленные пользователи
достаточно внимательны, чтобы предложить и лайкнуть ссылки, поэтому
предельные затраты на одного рандомного пользователя стремятся к нулю.
Или возможно главная страница защищает себя, размещая объявления о том
какие предложения она ожидает. \newline

Самое опасное для главной страницы это материал, который слишком
просто пролайкать. Если кто-то доказывает новую теорему, читателю
необходимо провести некоторую работу, чтобы решить стоит ли это
лайкать.Забавный мультфильм занимает на это меньше времени. Громкие
слова с не менее кричащими заголовками получают нули, потому что люди
лайкают их, даже не читая. \newline

Это то, что я называю Ложным Принципом: пользователь выбирает новый
сайт, ссылки которого легче всего поддаются суждению, если вы не
примете конкретные меры для предотвращения этого. \newline

У Hacker News два вида защиты от чепухи. Самые распространенные типы
сведений не имеющих никакой ценности забанены как оффтоп. Фотографии
котят, обличительные речи политиков и прочее находятся под особым
запретом. Это отсеивает большую часть ненужной ерунды, но не всю.
Некоторые ссылки являются и чепухой, в том плане, что они очень
короткие, и в то же время актуальным материалом. \newline

Нет единого решения для этого. Если ссылка является просто напросто
пустой демагогией, редакторы иногда уничтожают её, даже несмотря на
то, что она актуальна в теме хакерства, потому что она не является
актуальной согласно реальному стандарту, который подразумевает, что
статья должна возбуждать интеллектуальное любопытство. Если посты на
сайте именно этого типа, то я иногда баню их, а значит весь новый
материал по этому URL будет автоматически уничтожен. Если у поста
заголовок содержит ссылку-наживку, то редакторы иногда перефразируют
его, чтобы он стал более фактическим. Это особенно необходимо для
ссылок с кричащими заголовками, потому что в противном случае они
становятся скрытыми “проголосуй, если веришь в это и это” постами, а
это наиболее ярко выраженная форма никому ненужной ерунды. \newline

Техника борьбы с такими ссылками должна развиваться, так как сами
ссылки развиваются. Существование агрегаторов уже повлияло на то, что
они объединили. Сейчас писатели сознательно пишут то, что поднимет
трафик за счет агрегаторов — иногда довольно специфичные вещи.(Нет,
ирония этого высказывания мной не потеряна). Есть более зловещие
мутации как линкджэкинг — публикация пересказа чьей-то статьи и выдача
его вместо оригинала. Подобное может получить много лайков, так как в
нем остается много хорошего, что есть в первоначальной статье; на
самом деле чем больше пересказ похож на плагиат, тем больше хорошей
информации в статье сохраняется. [3] \newline

Я думаю важно, чтобы сайт, который отвергает предложения обеспечивал
пользователей способом увидеть, что было отвергнуто, если они этого
хотят. Это заставляет редакторов быть честными и, что не менее важно,
позволяет пользователям чувствовать себя более уверенно, так как они
узнают, если редакторы будут лукавить. Пользователи HN могут сделать
это щелкнув на поле showdead в своем профиле (“покажите мертвых”, если
дословно). [4] \newline

\subsection{Комментарии}

Плохие комментарии кажутся более серьезной проблемой нежели плохие
предложения. В то время как качество ссылок на главной странице не
сильно изменилось, качество среднестатистического комментария в
каком-то роде ухудшилось. \newline

Есть два основных вида испорченности комментариев: грубость и
глупость.Между этими двумя характеристиками много пересечений — грубые
комментарии вероятно так же глупые — но стратегии борьбы с ними
разные. Грубость легче контролировать. Ты можешь установить правила,
говорящие, что пользователь не должен быть грубым и если вы заставите
их вести себя хорошо, то держать грубость под контролем вполне
возможно. \newline

Держать под контролем глупость сложнее, возможно потому что глупость
не так легко отличить. Грубые люди зачастую знают, что они грубые, в
то время как многие глупые люди не осознают, что они глупые. \newline

Самая опасная форма глупого комментария это не длинное, но ошибочное
утверждение, а тупая шутка. Длинные, но ошибочные утверждения
встречаются крайне редко. Существует сильная корреляция между
качеством комментария и его длиной; если вы хотите сравнить качество
комментариев на общественных сайтах, средняя длина комментария будет
хорошим показателем. Вероятно, причиной является человеческая натура,
а не что-то конкретное для обсуждения темы. Наверное, глупость просто
чаще принимает форму наличия нескольких идей, нежели неправильных
идей. \newline

Вне зависимости от причины глупые комментарии обычно короткие. А так
как трудно написать короткий комментарий, который отличается от
количества информации, которую он передает, люди стараются выделиться,
пытаясь быть смешными. Наиболее соблазнительный формат глупых
комментариев это якобы остроумные оскорбления, вероятно потому что
оскорбления это самая легкая форма юмора. [5] Поэтому одним из
преимуществ запрета грубости является то, что он тоже ликвидирует
подобные комментарии. \newline

Плохие комментарии подобны кудзу: они стремительно берут вверх.
Комментарии имеют гораздо больший эффект на другие комментарии нежели
предложения на новые материалы. Если кто-то предлагает отстойную
статью, другие статьи не становятся от этого неудачными. Но если
кто-то публикует тупой комментарий в обсуждении, это влечет за собой
тонну таких же комментариев в этой области. Люди отвечают на тупые
шутки тупыми шутками. \newline

Возможно, решение заключается в добавлении задержки перед тем как люди
могут ответить на комментарий, и продолжительность задержки должна
быть обратно пропорциональна предположительному качеству комментария.
Тогда глупых обсуждения станет меньше. [6] \newline

\subsection{Люди}

Я заметил, что большинство методов, которые я описал, консервативны:
они нацелены на сохранение характера сайта, а не на его
усовершенствовании. Я не думаю, что я предвзято отношусь к проблеме.
Это связано с формой проблемы. Hacker News посчастливилось удачно
начать, так что в данном случае это буквально вопрос сохранения.Но я
думаю, этот принцип применим к сайтам различного происхождения. \newline

Хорошие вещи в общественных сайтах приходят скорее от людей нежели от
технологий; техника обычно вступает в игру, когда нужно предотвратить
появление плохих вещей. Технологии, безусловно, могут расширить
обсуждение. Вложенные комментарии, например. Но я бы предпочел
пользоваться сайтом с примитивными функциями и умными, приятными
пользователями, чем навороченным сайтом, которым пользуются только
идиоты и тролли. \newline

Самое главное, что должен делать общественный сайт это привлекать
людей, которых он хочет видеть в качестве своих пользователей. Сайт,
который старается быть настолько большим насколько это возможно
пытается привлечь всех. Но сайт направленный на определенный вид
пользователей должен привлекать только их — и, что не менее важно,
отталкивать всех остальных. Я сознательно пытался это сделать с HN.
Графический дизайн сайта настолько простой насколько это возможно и
правила сайта препятствуют появлению драматических заголовков. Цель
состоит в том, чтобы человек, появившийся на HN впервые, был
заинтересован идеями, высказывающимися здесь. \newline

Недостатком создания сайта нацеленного только на определенный вид
пользователей является то, что для этих пользователей он может быть
слишком привлекательным. Я прекрасно знаю насколько Hacker News может
быть вызывающим зависимость. Для меня, как и для многих пользователей,
это некая виртуальная городская площадь. Когда я хочу отдохнуть от
работы, я иду на площадь, так же как я мог бы, например, пройтись по
скверу Гарварда или Университетскому проспекту в физическом мире. [7]
Но площадь в сети более опасна чем реальная. Если я провел полдня
слоняясь по Университетскому проспекту, я это замечу. Я должен пройти
милю, чтобы добраться туда, и, посещение кафе отличается от работы. Но
посещение онлайн форума требует от вас всего один клик и внешне очень
похоже на работу. Может вы и тратите свое время, но вы не
прохлаждаетесь. Кто-то в интернете неправ и вы устраняете проблему. \newline

Hacker News определенно полезный сайт. Я многому научился из того, что
я прочитал на HN. Я написал несколько эссе, которые начинались как
комментарии здесь. Я бы не хотел, чтобы сайт исчез. Но я хочу быть
уверенным, что это не сетевая зависимость от продуктивности. Какой бы
ужасной катастрофой было бы завлечь тысячи умных людей на сайт, чтобы
они просто впустую потратили свое время. Хотелось бы мне быть на 100\%
уверенным, что это не описание HN. \newline

Мне кажется зависимость от игр и социальных приложений по-прежнему в
основном является нерешенной проблемой. Ситуация такая же как с крэком
в 1980-ых: мы изобрели ужасные новые вещи, вызывающие привыкание и мы
еще не усовершенствовали способы защиты от них. Мы усовершенствуем в
итоге и это одна из тех проблем, на которых я хочу сосредоточиться в
ближайшем будущем. \newline

\subsection{Примечания}

[1] Я пытался ранжировать пользователей и по среднестатистическому и
по среднему количеству комментариев и среднестатистический (высокий
балл отбрасываем) кажется более точным показателем высокого качества.
Хотя среднее количество комментариев может оказаться более точным
показателем плохих комментариев. \newline

[2] Еще одна вещь, которую я узнал из этого эксперимента, это то, что
если ты собираешься различать людей, то убедись, что ты делаешь это
правильно. Это тот вид проблемы, где быстрое создание прототипов не
работает. В самом деле, разумным честным аргументом является тот факт,
что различать разные типы людей возможно не лучшая идея. Причина не в
том, что все люди одинаковые, а в том, что плохо ошибиться и трудно не
допустить ошибку. \newline

[3] Когда я замечаю грубые линкджэйкинговые посты я замещаю URL-адрес
тем, что было скопировано. Сайты, которые часто используют
линкджэйкинг, банят. \newline

[4] Digg печально известен своим отсутствием четкой идентификации
личности. Корень проблемы не в том, что ребята, владеющие Digg, особо
скрытные, а в том, что они используют неверный алгоритм для генерации
своей главной страницы. Вместо того, чтобы раздуваться с вершины в
процессе получения большего количества голосов как Reddit, истории
начинаются вверху страницы и сталкиваются вниз новыми поступлениями. \newline

Причина этого различия заключается в том, что Digg заимствован от
Slashdot, в то время как Reddit заимствован от Delicious/popular. Digg
это Slashdot с голосованием вместо редакторов и Reddit это
Delicious/popular с голосованием вместо закладок. (Вы все еще можете
увидеть остатки их происхождения в графическом оформлении.) \newline

Алгоритм Digg очень чувствителен к играм, потому что любая история,
которая попадает на главную страницу это новая история. Что в свою
очередь заставляет Digg применять крайние контрмеры. Многие стартапы
имеют какой-то секрет касательно того к каким уловкам им пришлось
прибегнуть в ранние дни, и я подозреваю, что у Digg секрет заключается
в том, что лучшие истории по факту выбираются редакторами. \newline

[5] Диалог между Beavis и Butthead был основан в основном на этом и
когда я читаю комментарии на реально плохих сайтах я могу слышать их
голоса. \newline

[6] Я подозреваю, что большинство методов борьбы с глупыми
комментариями до сих пор не обнаружено. Xkcd реализовал самый умный
метод на своем IRC канале: не позволяй делать одно и то же дважды. Как
только кто-то сказал “провал”, не позволяй ему произнести это снова.
Это позволит наказывать короткие комментарии в особенности, потому что
у них меньше возможностей избежать повторений. \newline

Еще одна многообещающая идея это глупый фильтр, который является
вероятностным спам-фильтром, но обученном на базе конструкций глупых и
нормальных комментариев. \newline

Возможно необязательно уничтожать плохие комментарии, чтобы избавиться
от проблемы.Комментарии в нижней части длинного обсуждения можно редко
увидеть, поэтому вполне достаточно включить прогнозирование качества в
алгоритм сортировки комментариев. \newline

[7] Что делает большинство пригородов такими деморализующими там это
отсутствие центра, по которому можно погулять. \newline

\end{document}
