\documentclass[ebook,12pt,oneside,openany]{memoir}
\usepackage[utf8x]{inputenc} \usepackage[russian]{babel}
\usepackage[papersize={90mm,120mm}, margin=2mm]{geometry}
\sloppy
\usepackage{url} \title{Жив или мёртв ваш стартап?} \author{Пол Грэм}
\date{}
\begin{document}
\maketitle

Основатель и партнёр Y Combinator Тревор Блэквелл создал калькулятор
роста стартапов, который может определить, жив стартап или находится
на пути к смерти.

Пол Грэм, сооснователь, говорит, что калькулятор предлагает пугающие,
но крайне важные ответы на вопрос о том, верным ли путём идёт
компания.


Фаундеры должны постоянно оценивать, в какой точке пути находятся их
компании и куда ведёт этот путь — стартап “по умолчанию жив” или “по
умолчанию мёртв”?

Заканчиваются ли у стартапа деньги, судя по темпу роста выручки? Или
он сможет выйти на прибыль?

«Причина, по которой я в первую очередь хочу знать, жив или мёртв
стартап заключается в том, что весь остальной разговор зависит от
ответа на этот вопрос. Если стартап жив, то мы можем говорить с ним о
новых амбициозных вещах, которые он может сделать.

Если он мёртв — нам, вероятно, нужно говорить о том, как его спасти.
Мы знаем, что текущая траектория этого стартапа заканчивается плохо.
Как он может с неё сойти?» — говорит Пол Грэм.

Сигнал тревоги и последняя капля Когда компания по умолчанию мертва,
не может привлечь капитал и не имеет достаточно времени, чтобы выйти
на рентабельность, наступает так называемая “Последняя капля” («the
fatal pinch»).

Это краткий предсмертный этап стартапа, когда ресурсы на поддержание
компании уже закончились, и наступило время экстренных мер для выхода
из пике — таких, как массовые увольнения и спешные попытки повысить
выручку.

Как правило, “последняя капля” длится несколько месяцев, в которые всё
ещё есть шансы спасти ситуацию.

Пол Грэм рассказывает:

“В последние несколько месяцев мы наблюдали, как некоторые стартапы
рушатся в результате слишком поспешного найма и неуправления скоростью
сгорания инвестиций (burn rate). В августе стартап Zirtual уволил 400
сотрудников по электронной почте за 1 ночь.

Другие стартапы (и крупные компании) тоже предпринимают увольнения,
чтобы остаться на пути к прибыльности. Flipagram сократили 20\% своих
сотрудников в октябре, Zomato уволил 300 сотрудников. Так же недавно
сокращения прошли в более крупных компаниях, таких как Snapchat,
Evernote и Twitter.

В стартапах ранней стадии проблема чаще всего в том, что продукт лишь
умеренно привлекателен, и не является тем хитом, которым он должен
быть. Некоторые стартапы начинают нанимать людей для создания
продукта, но в реальности они создают только расходы и ставят себя на
путь к “смерти по умолчанию” и “последней капле”.

«Вопрос о том, является ли ваш стартап живым или мёртвым, может спасти
вас от этой участи. Возможно, сигналы тревоги, которые он подаст,
нейтрализуют силы, которые толкают вас к избыточному найму», — пишет
Грэм.

Тревор Блэквелл:

“Калькулятор считает, сколько инвестиций нужно вашему стартапу.

Калькулятор показывает, когда ваш стартап выйдет на прибыль и сколько
денег вы потратите до того, как это произойдёт. Предполагается, что
ваши расходы постоянны, а доход растёт.

Как только ваш стартап вышел на прибыль — вы контролируете свою судьбу
и, если захотите, можете поднять больше денег, чтобы расти быстрее.

Потяните за красные или зелёные ручки на калькуляторе, чтобы выставить
расходы, доходы и рост. Между прямыми расходов и доходов появится
синяя область — это и есть капитал, который понадобится вашему
стартапу.

Если вы привлечёте инвестиции в размере, точно совпадающем с
результатами калькулятора, и всё пойдёт по плану, то в месяц, когда вы
достигнете рентабельности, на вашем банковском счёте будет \$0 —
довольно стрессовая ситуация. Так что привлекайте капитал с комфортным
запасом сверх этой нормы.

По умолчанию калькулятор показывает недели, но есть кнопка для показа
месяцев. Если вам интересно, как работает калькулятор — вот код на
github.”

Как пользоваться калькулятором: Задайте свои еженедельные, ежемесячные
или ежегодные расходы, доходы и рост (предполагается, что они
постоянны) Оцените, через какое время стартап выйдет на прибыль и
требуемый для этого капитал по появившейся между осями синей зоне

\end{document}
