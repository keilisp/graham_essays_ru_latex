\documentclass[ebook,12pt,oneside,openany]{memoir}
\usepackage[utf8x]{inputenc} \usepackage[russian]{babel}
\usepackage[papersize={90mm,120mm}, margin=2mm]{geometry}
\sloppy
\usepackage{url} \title{То, что вы хотели бы знать заранее}
\author{Пол Грэм} \date{}

\begin{document}
\maketitle

Когда я сказал, что собираюсь читать речь в школе, мои друзья
удивились. Что ты скажешь старшеклассникам? Тогда я спросил их, а что
бы вы хотели, чтобы вам кто-нибудь сказал, когда вы были в старших
классах? Их ответы были заметно схожи. Так что я расскажу вам то, что
мы все хотели бы, чтобы нам кто-то сказал заранее. \newline

Я начну с того, что расскажу вам нечто, что вы не обязаны знать в
старших классах: чем вы собираетесь заниматься в своей жизни. Вас
постоянно об этом спрашивают, и вы полагаете, что вы должны знать
ответ. Только дело в том, что взрослые спрашивают это просто чтобы
начать разговор. Они хотят узнать, что вы за человек, и этот вопрос -
это просто, чтобы разговорить вас. Они спрашивают это примерно так же,
как вы могли бы тыкать краба-отшельника в приливной яме, чтобы
посмотреть, что он будет делать. \newline

Если бы я сейчас был старшеклассником, и кто-то спросил бы меня о моих
планах, я бы сказал, что первым делом я собираюсь узнать, из чего я
могу выбирать. Вы совершенно не должны торопиться выбрать дело всей
вашей жизни. Что вы должны сделать, это понять, что вам нравится. Вы
должны заниматься тем, что вам нравится, если вы хотите быть хорошим
специалистом в своём деле. \newline

Может показаться, что нет ничего проще, чем решить, что вам нравится,
но, как оказывается, это очень даже непросто. Отчасти из-за того, что
сложно получить точное представление о большинстве профессий. Быть
врачом на самом деле это совсем не то, как это изображается по
телевизору. К счастью вы всегда можете понаблюдать работу настоящих
врачей, если пойдёте поработать санитарами-добровольцами в больницу.
[1] \newline

Но ещё есть такие виды деятельности, о которых вы не можете ничего
узнать, потому что пока ещё нет никого, кто бы ими занимался. Большая
часть дел, которыми я занимался за последние десять лет, ещё не
существовали когда я был в старших классах. Мир быстро изменяется, и
скорость, с которой он изменяется, всё увеличивается. В таком мире,
иметь жёсткие планы - не слишком хорошая идея. \newline

И, несмотря на это, каждый год, в мае, ораторы по всей стране читают
``стандартное обращение к выпускникам``, тема которого: не отступайте от
своей мечты. Я знаю, что они хотят сказать, но это плохой способ
выразить это, потому что он подразумевает, что вы должны быть
привязаны к какому-то заранее сделанному плану. В компьютерном мире
этому есть название: преждевременная оптимизация. И это тождественно
бедствию. Эти ораторы сделали бы лучше, если бы просто сказали: не
сдавайтесь. \newline

На самом деле они имеют в виду: не теряйте духа. Не думайте, что вы не
можете сделать то, что могут другие люди. И я согласен с ними, вы не
должны недооценивать свой потенциал. Люди, которые создали великие
вещи кажутся нам немного другой расой. И большинство биографий только
увеличивает эту иллюзию, отчасти благодаря преклоняющейся позиции, в
которую неизбежно впадают биографы, отчасти потому, что зная, чем
история заканчивается, они не могут не обтачивать сюжет до тех пор,
пока жизнь описываемого субъекта не будет казаться предопределённой
судьбой, просто раскрытием некоего врождённого гения. Если честно, я
подозреваю, что если бы с вами в школе учились шестнадцатилетние
Шекспир и Эйнштейн, они бы казались впечатляющими, но не совершенно
отличными от остальных ваших друзей. \newline

Это неприятная мысль. Если они были такими же, как мы, значит они
должны были очень упорно работать, чтобы сделать то, что они сделали.
Это одна из причин, по которым нам нравится верить в гениальность. Она
даёт нам оправдание нашей лени. Если эти парни смогли сделать то, что
сделали, только благодаря некоей волшебной Шекспировскости или
Эйнштейности, то это не наша вина, что мы не можем сделать нечто,
настолько же хорошее. \newline

Я не говорю, что нет такой вещи как гениальность, но если вы пытаетесь
выбрать между двумя теориями, и одна из них даёт оправдание вашей
лени, то возможно, что другая верна. \newline

Пока что мы отшлифовали стандартное обращение к выпускникам с ``Не
отступайте от своей мечты`` до ``Что может сделать другой, можете
сделать и вы``. Но её всё ещё надо шлифовать дальше. Существуют
некоторые отличия в естественных способностях. Большинство людей
переоценивают их роль, но они существуют. Если бы я разговаривал с
парнем ростом метр-двадцать, который хочет играть в NBA, я бы
чувствовал себя довольно глупо, если бы говорил ``ты можешь достичь
чего угодно, если только будешь как следует стараться``. [2] \newline

Нам нужно отшлифовать стандартное обращение к выпускникам до ``Что
может сделать другой, с вашими способностями, можете сделать и вы. И
не недооценивайте свои способности.`` Но, как часто случается, чем
ближе подбираешься к истине, тем более путаным становится предложение.
Мы взяли красивый, аккуратный слоган и перемесили его, как грязную
лужу. Он уже больше не может служить основой для хорошего обращения.
Но, что ещё хуже, он уже больше не говорит вам, что нужно делать.
Кто-то с вашими способностями? Какие у вас способности? \newline

\subsection{Против ветра}

Я думаю, решение в том, чтобы работать в другом направлении. Вместо
того, чтобы работать к цели, работайте от перспективных ситуаций. В
любом случае, это как раз то, что делает большинство успешных людей. \newline

В том подходе, который предлагает обращение к выпускникам, вы решаете,
где вы хотите оказаться через 20 лет, и спрашиваете: что я должен
сделать сейчас, чтобы попасть туда? Я предлагаю, чтобы вы вместо того,
чтобы привязываться к чему-то в будущем, просто посмотрели на
возможности, доступные вам сейчас, и выбрали те, которые дадут вам
наиболее широкий диапазон возможностей потом. \newline

Не так важно, над чем именно вы работаете, если только вы не теряете
время. Работайте над вещами, которые интересуют вас и вы увеличите
свои возможности выбора, а потом уже будете беспокоиться, которую из
них вам выбрать. \newline

Представьте, что вы новичок в колледже, который решает,
специализироваться ему на математике, или экономике. Математика даст
вам больше возможностей выбора: вы можете пойти почти в любую область
из математики. Если вы специализируетесь на математике, вам будет
легко попасть в аспирантуру по экономике, но если вы специализируетесь
на экономике, будет сложно попасть в аспирантуру по математике. \newline

Полёты на планере хорошая метафора в данном случае. Из-за того, что у
планера нет двигателя, невозможно лететь по ветру не теряя при этом
значительно высоту. Если вы позволите себе пролететь далеко по ветру в
поиске хорошего места для приземления, ваш выбор сузится до
некомфортного. Как правило, вам лучше держаться против ветра. Вот это
я и предлагаю в качестве заменителя для ``не отступайте от своей
мечты``. Держитесь против ветра. \newline

Как же это делать? Даже если математика находится против ветра
относительно экономики, откуда вы знаете это пока вы ещё только
старшеклассники? \newline

Честно говоря, ниоткуда. Вот именно это вы и должны выяснить. Ищите
умных людей и сложные задачи. Умные люди склонны собираться в группы,
и если вы найдёте такую группу, вероятно имеет смысл войти в неё. Но
это не так просто, найти такие группы, потому что кругом полно
подделок. \newline

Свежеприбывшему новичку все факультеты университета кажутся примерно
одинаковыми. Профессора все кажутся отталкивающе заумными и
публикующими статьи, непостижимые для неспециалистов. Однако, в то
время как некоторые статьи непостижимы из-за того, что полны сложных
идей, другие намеренно написаны запутанным языком, чтобы казалось, что
в них написано что-то важное. Это может казаться скандальным
предположением, но оно было экспериментально подтверждено знаменитой
афёрой ``Общественный текст``. Подозревая, что статьи, публикуемые
литературными теоретиками, часто являются научно звучащей
бессмыслицей, один физик намеренно написал статью, полную научно
звучащей бессмыслицы, и отправил её в журнал о литературной теории,
который её и опубликовал. \newline

Лучшая защита от этого - постоянно работать над сложными задачами.
Писать рассказы сложно. Читать рассказы - нет. Сложные проблемы - это
те, которые беспокоят вас: если вы не беспокоитесь, что то, что вы
делаете, может получиться плохо, или что вы не сможете понять что-то,
что вы учите, значит задача недостаточно сложная. Всегда должна
присутствовать некая напряжённость. \newline

Вы можете подумать, это кажется суровым взглядом на мир. Я говорю вам,
что вы должны беспокоиться? Да, но это не так плохо, как кажется.
Преодолевать беспокойство - это очень бодряще. Редко увидишь лица,
намного более счастливые, чем у людей, которые выиграли золотые
медали. И знаете, почему они так счастливы? Облегчение. \newline

Я не хочу сказать, что это единственный способ стать счастливым.
Просто некоторые виды беспокойства не так плохи, как они кажутся. \newline

\subsection{Честолюбие}

На практике, ``Держитесь против ветра`` уменьшается до ``работайте над
сложными задачами``. И вы можете начать уже сегодня. Хотел бы я знать
это в старших классах. \newline

Большинству людей нравится быть хорошими специалистами в том, что они
делают. В так называемом настоящем мире это желание является мощной
силой. Но старшеклассники редко получают от этого выгоду, потому что
их заставляют заниматься чем-то ненастоящим. Когда я был в старших
классах, я уверил себя, что моя работа заключается в том, чтобы быть
старшеклассником. И, таким образом, я удовлетворил своё желание делать
хорошо свою работу просто тем, что учился в школе хорошо. \newline

Если бы вы меня спросили, когда я был старшеклассником, в чём разница
между старшеклассниками и взрослыми, я бы сказал, что взрослые должны
зарабатывать на жизнь. Неправильно. Разница в том, что взрослые сами
несут ответственность за себя. Зарабатывание на жизнь только маленькая
часть этого. Намного более важно самому нести умственную
ответственность за себя. \newline

Если бы я должен был снова отучиться в старших классах, я бы относился
к школе, как к подработке. Я не имею в виду, что я бы расслаблялся в
школе. Делать что-то в качестве подработки, не значит делать это
плохо. Это значит не определять себя по этой работе. Я имею в виду,
что я бы не думал о себе как о старшекласснике, как музыкант, который
подрабатывает официантом, не думает о себе, как об официанте. [3] А в
то время, когда я не занят подработкой, я бы начал пытаться делать
настоящую работу. \newline

Когда я спрашиваю людей, о чём они больше всего сожалеют, связанном со
старшими классами, все отвечают примерно одно и то же: они сожалеют,
что потеряли так много времени. Если вы задаётесь вопросом, что вы
такое делаете сейчас, о чём вы будете больше всего потом сожалеть,
вероятно это и есть ответ. [4] \newline

Некоторые могут сказать, что это неизбежно, что старшеклассники ещё не
способны заниматься чем-то дельным, но я не думаю, что это правда. И
доказательство тому то, что вам скучно. Я полагаю, что вам не было
скучно, когда вам было восемь лет. Когда вам восемь лет, это
называется ``играть``, а не ``тусоваться``, хотя это одно и то же. Когда
мне было восемь лет, мне редко было скучно. Дайте мне задний двор и
ещё несколько детей, и я мог бы играть целый день. \newline

Теперь я понимаю, что в средних и старших классах это потеряло свою
привлекательность потому, что я уже был готов к чему-то другому.
Детство уходило в прошлое. \newline

Я не хочу сказать, что вы не должны тусоваться с друзьями, что вы
должны стать маленькими роботами, лишёнными чувства юмора, которые
только и работают. Тусование с друзьями - это как шоколадный торт. Он
нравится больше, если вы едите его время от времени, чем если вы едите
исключительно шоколадный торт на завтрак, обед, и ужин. Не важно, как
сильно вам нравится шоколадный торт, после третьего приёма пищи только
из этого блюда, вас будет тошнить от него. Вот именно это недомогание
чувствуют в старших классах: умственную тошноту.[5] \newline

Вы можете подумать, ``мы должны делать больше, чем просто получать
хорошие оценки, мы должны работать сверх учебной программы``. Только вы
и сами отлично знаете, насколько всё это фальшиво. Сбор пожертвований
на благотворительные цели это достойное уважения занятие, но это не
сложно. Вы не достигаете этим какой-то цели. Под достижением какой-то
цели я понимаю, например, когда человек учиться хорошо писать
сочинения, или программировать компьютеры, или изучает, какой, на
самом деле, была жизнь в преиндустриальных обществах, или учится
рисовать человеческое лицо с натуры. Такие вещи редко упоминают в
заявлениях на приём в колледж. \newline

\subsection{Разложение}

Нацеливать всю свою жизнь на поступление в колледж - опасно, потому
что люди, которых вы должны впечатлить, чтобы попасть в колледж - не
слишком проницательная публика. В большинстве колледжей решение о
поступлении принимают не профессора, а члены приёмной комиссии, а они
и близко не так умны. Они ``неуполномоченные чиновники``
интеллектуального мира. Они не могут сказать, насколько вы умны. Само
по себе существование подготовительных школ [пер.2] доказывает это. \newline

Мало родителей стали бы платить такие большие деньги, чтобы их дети
ходили в школу, которая не увеличивает их шансы поступления.
Подготовительные школы открыто заявляют, что это одна из их целей. Но
это означает, если перестать думать о поступлении, что они могут
провести приёмную комиссию: они могут взять того же самого ребёнка и
сделать так, чтобы он выглядел более привлекательным кандидатом, чем
если бы он ходил в обычную школу рядом с домом. [6] \newline

Сейчас большинство из вас думает, что ваша задача в жизни - это быть
многообещающим абитуриентом колледжа. Но это значит, что вы планируете
свою жизнь для того, чтобы удовлетворить процесс, который настолько
бездумен, что существует целая индустрия, предназначенная для
обведения его вокруг пальца. Неудивительно что вы становитесь
циничными. Недовольство, которое вы чувствуете, то же самое, что
чувствует продюсер телевизионного реалити-шоу, или руководитель в
табачной промышленности. А вам, в отличие от них, совсем не платят за
это. \newline

Так что же делать? Что вы не должны делать - это бунтовать. Я, в своё
время, взбунтовал, и это было ошибкой. Я не понял что именно
происходило с нами, но чувствовал неладное. И я просто сдался.
Очевидно, что мир был отстой, чего было дёргаться? \newline

Когда я обнаружил, что одна из наших учительниц сама пользовалась
Cliff's Notes[пер.3], это казалось очевидной оценкой всего предмета.
Естественно, что получить хорошую оценку по такому предмету ничего не
значило. \newline

Оглядываясь назад, я считаю, что это было глупо. Это всё равно, если
бы кого-то ударили в футболе и он бы сказал ``Эй! Ты меня ударил, это
не по правилам`` и ушёл бы с поля в негодовании. Нарушения случаются.
Если вас задели, вы не должны терять хладнокровия. Просто продолжайте
играть. \newline

Поставив вас в такую ситуацию, общество нечестно сыграло с вами. Да,
как вы и подозреваете, многое из того, что вы учите, просто чепуха. А
также, как вы и подозреваете, процесс приёма в колледж во многом
просто лотерея. Но как и многие нарушения, это было нечанным. [7] Так
что просто продолжайте играть. \newline

Бунтарство почти также глупо, как покорство. В любом их этих случаев
вы позволяете определять себя тем, что вам говорят делать. Лучший
план, по-моему, встать на перпендикулярный вектор. Не надо просто
делать, что вам говорят, и не надо просто отказываться от этого.
Вместо этого относитесь к школе, как к подработке. Как это бывает с
подработкой, она довольно приятна. В три часа вы уже свободны, и вы
даже можете заниматься своими собственными делами пока вы в школе. \newline

\subsection{Любопытство}

А в чём же тогда ваша настоящая работа? Если только вы не Моцарт, то
это как раз и есть ваша первейшая задача: выяснить, в чём. Где можно
делать великие вещи? В каких областях работают впечатляющие люди? И,
самое важное, чем вы интересуетесь? Слово ``склонность`` вводит в
заблуждени, потому что оно подразумевает нечто врождённое. Самый
сильный вид склонности это всепоглощающий интерес к какому-нибудь
вопросу, и такой интерес часто является приобретённым чувством. \newline

Искаженная версия этой идеи просочилась в популярную культуру под
словом ``страсть``. Я недавно видел рекламу для официантов, в которой
говорилось, что им нужны люди со ``страстью к обслуживанию``.
Обслуживание столиков не назовёшь настоящим делом. И страсть плохое
слово для него. Любопытство было бы лучшим словом. \newline

Дети любопытны, но я имею в виду любопытство другого рода. Детское
любопытство широкое и мелкое; они спрашивают ``почему`` совершенно
беспорядочно обо всём подряд. В большинстве взрослых это любопытство
иссякает полностью. И так и должно быть: невозможно сделать что-либо,
если ты всё время спрашиваешь ``почему`` обо всём. Но у целеустремлённых
взрослых, вместо того, чтобы иссякать, любопытство становится
узконаправленным и глубоким. Грязная поверхность преобразуется в
родник. \newline

Любопытство превращает работу в игру. Для Эйнштейна теория
относительности не была книгой, полной сложных вещей, которые он
должен был выучить к экзамену. Она была загадкой, которую он пытался
разгадать. Так что вероятно, что изобрести её для него было легче, чем
для кого-то другого выучить её в школе. \newline

Одна из самых опасных иллюзий, которую вы приобретаете в школе,
заключается в идее, что создание великих вещей требует огромной
самодисциплины. Большинство предметов преподаётся таким скучным
образом, что только посредством самодисциплины вам удаётся продраться
через них. Так что я был весьма удивлён, когда в самом начале обучения
в колледже прочитал цитату Виттгенштейна о том, что у него полностью
отсутствовала самодисциплина, и он никогда не мог отказать себе ни в
чём, даже в чашке кофе. \newline

Сейчас я знаю довольно много людей, которые делают великолепную
работу, и они все схожи в этом. У них слабая дисциплина. Они все
ужасно долго откладывают, и практически не в состоянии заставить себя
сделать что-то, что им не интересно. Один из них до сих пор не
отправил благодарственные открытки со своей свадьбы, которая была
четыре года назад. У другой в папке ``Входящие`` 26000 писем. \newline

Я не говорю, что вы можете обойтись совсем без самодисциплины. Я
думаю, что вам нужно примерно столько дисциплины, сколько необходимо,
чтобы пойти бегать. Мне часто неохота идти бегать, но после того, как
я начал, я получаю от бега удовольствие. И если я не бегаю несколько
дней, я чувствую себя неважно. С людьми, которые делают великие вещи,
то же самое. Они знают, что они будут чувствовать себя плохо, если не
будут работать, поэтому они достаточно дисциплинированы, чтобы усадить
себя за свои столы и начать работать. Но как только они начали,
интерес начинает преобладать и дисциплина больше не нужна. \newline

Думаете Шекспир скрипел зубами и усердно пытался написать Великую
Литературу? Конечно нет. Он получал удовольствие. Именно поэтому он
так хорош. \newline

Если вы хотите сделать нечто хорошее, вам нужно огромное любопытство
по какому-нибудь перспективному вопросу. Критическим моментом для
Эйнштейна было когда он посмотрел на уравнения Максвелла и спросил:
что здесь, чёрт возьми, происходит? \newline

Поиск продуктивного вопроса может занять годы, потому что на то, чтобы
понять, о чём, собственно, речь, могут уйти многие годы. В качестве
крайнего случая давайте рассмотрим математику. Большинство людей
думает, что они ненавидят математику, но те скучные вещи, которые вы
делаете в школе под вывеской ``Математика``, совсем не похожи на то, чем
занимаются настоящие математики. \newline

Великий математик Г.Г. Харди говорил, что ему тоже не нравилась
математика в школе. Он выбрал её только потому, что у него получалось
лучше, чем у других учеников. Только потом он обнаружил, что
математика была интересна - только потом он начал задавать вопросы
вместо того, чтобы просто отвечать на них правильно. \newline

Когда один мой друг начинал ворчать, что он должен написать какою-то
работу для школы, его мать говорила ему: придумай способ, как сделать
это интересным. Вот что вам надо сделать: найти вопрос, который делает
мир интересным. Люди, которые делают великие вещи, смотрят на тот же
мир, что и все остальные, но замечают какую-то деталь, которая хранит
в себе непреодолимую таинственность. \newline

И это относится не только к мыслительным материям. Великий вопрос
Генри Форда был: почему машины должны быть роскошью? Что было бы, если
бы к ним относились, как к товару широкого потребления? Вопрос Франца
Бекенбауера был, по сути: почему все должны стоять в его позиции?
Почему защитники не могут тоже забивать голы? \newline

\subsection{Сейчас}

Если сформулировать великий вопрос занимает годы, что вам делать
сейчас, когда вам шестнадцать лет? Работайте над поиском такого
вопроса. Великие вопросы не появляются неожиданно. Они постепенно
застывают у вас в голове. И заставляет их сворачиваться опыт. То есть,
чтобы найти великий вопрос не надо искать его - не надо бродить в
раздумьи, какое великое открытие мне совершить? Вы не можете ответить
на этот вопрос; если бы вы могли, вы бы уже сделали это открытие. \newline

Чтобы в вашей голове появилась большая идея, надо не охотиться за
большими идеями, а вложить много времени в работу, которая интересует
вас, и в процессе держать ум достаточно открытым, чтобы большая идея
могла войти в него. Эйнштейн, Форд, Бекенбауер - все использовали этот
рецепт. Они все знали свою работу, как пианист знает клавиши. Поэтому,
когда им что-то казалось ошибочным, у них было достаточно уверенности
в себе, чтобы сказать это. \newline

Как вложить время и во что? Просто выберите проект, который кажется
интересным: овладеть каким-то материалом, или сделать что-то, или
ответить на какой-то вопрос. Выберите проект, который займёт меньше
месяца и пусть это будет что-то, для чего вам хватит средств на
завершение. Сделайте что-нибудь достаточно сложное, чтобы напрячь вас,
но только слегка, особенно поначалу. Если вы выбираете между двумя
проектами, выбирайте тот, который кажется более увлекающим. Если один
проект провалится, начинайте другой. Повторяйте пока, как в двигателе
внутреннего сгорания, процесс не станет самоподдерживающимся и каждый
проект не начнёт производить следующий. (Это может занять годы.) \newline

Не делать какой-то проект ``для школы`` вполне может быть обоснованным,
если он ограничивает вас, или делает школу похожей на работу.
Привлеките своих друзей, если хотите, но не слишком много, и только
если они не яркие индивидуалы. Друзья дают моральную поддержку (мало
фирм открывается одним человеком), но секретность тоже имеет свои
преимущества. Есть что-то приятное в секретном проекте. И, к тому же,
вы можете больше рисковать, потому что никто не узнает если вы
провалитесь. \newline

Не переживайте, если проект кажется отклонившимся от пути, на котором
вы, вроде как, должны быть. Пути могут растягиваться намного больше,
чем вы думаете. Так что дайте пути вырасти из проекта. Самое
интересное в этом то, что вы учитесь непосредственно делая что-то. \newline

Не пренебрегайте недостойными мотивами. Один из наиболее мощных
мотивов - это желание быть лучше, чем другие люди, в чём-то. Харди
сказал, что именно это дало ему начальный толчок, и я думаю, что
единственная необычная черта в нём это то, что он признал это. Другой
мощный мотив это желание делать, или знать, что-то что, как
предполагается, вы не должны знать. Близкородственным мотивом к нему
является желание делать что-то отважное. Предполагается, что
шестнадцатилетние не пишут романов. Так что если вы попробуете, то что
бы вы не достигли, окажется на положительной стороне шкалы; если вы
совершенно провалитесь, это всё равно будет не хуже ожиданий. [8] \newline

Опасайтесь плохих моделей. Особенно если они оправдывают лень. Когда я
был старшеклассником, я писал короткие ``экзистенциалистские`` рассказы,
вроде тех, какие я видел у известных писателей. В моих историях не
было много интриги, но они были очень глубокими. И их было легче
написать, чем развлекательные рассказы. Я должен был бы знать, что это
был признак опасности. И в действительности мои рассказы были довольно
скучными; меня возбуждала идея написания чего-то серьёзного,
интеллектуального, как известные писатели. \newline

Сейчас у меня достаточно опыта, чтобы понять, что те писатели были, на
самом деле, отстойными. Множество известных людей делают отстой; в
краткосрочной перспективе, качество работы человека составляет только
маленькую часть известности. Я должен был меньше думать о том, чтобы
сделать что-то, что выглядело бы круто, и больше делать то, что мне
нравилось. На самом деле это и есть дорога к крутости. \newline

Ключевой составляющей многих проектов, даже почти отдельным проектом,
является поиск хороших книг. Большинство книг плохие. Почти все
учебники плохие. [9] Поэтому не надо полагать, что предмет надо учить
по любой книге, которая попадётся под руки. Вы должны активно искать
крошечное количество хороших книг. \newline

Важно сдвинуться с места и начать делать. Вместо того, чтобы ждать
пока вас научат, пойти и выучить. \newline

Вашей жизнью не должны руководить члены приёмной комиссии. Ей вполне
может руководить ваше собственное любопытство. Оно руководит жизнями
всех целеустремлённых взрослых. И вы не обязаны ждать, чтобы начать.
По правде говоря, вы не должны ждать, чтобы стать взрослыми. В вас нет
никакого переключателя, который волшебным образом поворачивается,
когда вы достигаете определённого возраста, или получаете диплом
какого-то заведения. Вы становитесь взрослыми когда вы решаете принять
на себя ответственность за свою жизнь. Вы можете сделать это в любом
возрасте. [10] \newline

Это может звучать как полная туфта. Вы можете подумать, я всего лишь
несовершеннолетний, у меня нет денег, я должен жить дома, я должен
делать то, что взрослые говорят мне целый день. Ну, большинство
взрослых работают в таких же сложных условиях, и они справляются. Если
вы думаете, что быть ребёнком обременительно, представьте себе каково
иметь детей. \newline

Единственное настоящее различие между взрослыми и старшеклассниками в
том, что взрослые осознают, что они должны делать то-то и то-то, а
старшеклассники - нет. Это осознание ударяет в голову большинства
людей около 23-х лет. Но я посвящаю вас в этот секрет заранее, так что
начинайте работать. Возможно вы сможете стать первым поколением,
которое не будет сожалеть, как много времени оно потеряло в старших
классах. \newline

\subsection{Примечания}

[1] Мой друг-доктор предупреждает, что даже это может дать
неправильное представление. ``Кто бы знал, как много времени это
займёт, как мало свободы действий я получу за бесчисленные годы учёбы,
и как невыносимо раздражает, когда ты должен носить с собой пейджер!`` \newline

[2] Его лучшей ставкой было бы, наверное, стать диктатором, и угрожать
NBA, чтобы ему позволили играть. Пока что ближе всех смог приблизиться
к этому только секретарь лейбористской партии. \newline

[3] Подработка - это работа, на которую вы идёте, чтобы оплачивать
счета, чтобы вы могли заниматься тем, что действительно хотите.
Например, играть в музыкальной группе, или изобретать теорию
относительности. \newline

Отношение к школе как к подработке, могло бы, на самом деле, помочь
некоторым получать хорошие оценки. Если вы относитесь к урокам как к
игре, вы не будете деморализованы, если они кажутся бессмысленными. \newline

Как бы ни были плохи ваши уроки, вам надо получать на них хорошие
оценки, чтобы попасть в приличный колледж. И это стоит делать, потому
что университеты в наши дни - это те места, где собирается множество
умных людей. \newline

[4] Второе наибольшее сожаление было о том, что они так много
беспокоились о совершенно не важных вещах. Особенно о том, что другие
люди думали о них. \newline

Я думаю, они на самом деле имели в виду то, что посторонние люди
думали о них. Взрослых в той же мере волнует, что другие люди думают,
но они становятся более избирательными в выборе ``других`` людей. \newline

У меня примерно тридцать друзей, чьи мнения волнуют меня, а мнение
остального мира едва ли влияет на меня. Проблема в старших классах в
том, что ваше окружение составлено случайностями возраста и географии,
а не выбранными вами людьми, основываясь на уважении к их суждениям. \newline

[5] Ключ к бессмысленной трате времени - развлечения. Без развлечений
вашему мозгу становится слишком очевидно, что вы совершенно не
используете его, и вы начинаете чувствовать себя неуютно. Если хотите
измерить, насколько вы стали зависимы от развлечений, попробуйте
провести следующий эксперимент: выделите некоторый промежуток времени
в выходные, сядьте наедине, и думайте. Можете взять блокнот, чтобы
записывать свои мысли, но ничего больше: никаких друзей, телевизора,
музыки, телефона, аськи и иже с ней, электронной почты, интернета,
игр, книг, газет, журналов. В течение часа большинство людей
почувствует сильную тоску по развлечениям. \newline

[6] Я не подразумеваю, что единственной целью подготовительных школ
является провести вступительную комиссию. Они также в целом дают
лучшее образование. Но попробуйте провести такой мысленный
эксперимент: представьте, что подготовительные школы дают такой же
превосходящий уровень образования, но имеют крошечное (.001)
отрицательное влияние на поступление в колледж. Сколько родителей
отправили бы детей в такие школы несмотря на это? \newline

Можно также утверждать, что дети, которые учились в подготовительных
школах, являются лучшими кандидатами в колледж, из-за того, что они
учили больше. Но это кажется эмпирически неверным. То, что учат даже в
самой лучшей школе, это всего лишь ошибка округления по сравнению с
тем, что учат в колледже. Дети из обычных школ немного отстают
вначале, но они начинают выбиваться в лидеры на втором курсе. \newline

(Я не имею в виду, что дети из обычных школ умнее, чем из
подготовительных, а только то, что они есть в каждом колледже. Это
неизбежно следует, если вы согласитесь с тем, что подготовительные
школы увеличивают шансы на поступление.) \newline

[7] Почему общество нечестно играет с вами? Равнодушие, в основном.
Просто нет таких внешних сил, которые заставляли бы школу быть
хорошей. Система регуляции воздушного движения работает, потому что
иначе самолёты начали бы сталкиваться. В бизнесе, производители должны
делать поставки, потому что иначе конкуренты переманят клиентов. Но
если ваша школа отстой, самолёты не упадут. И конкурентов у неё нет.
Школа не злонамеренная, она просто никакая, а никакая - это довольно
плохая. \newline

[8] Ну и конечно, ещё есть деньги. Они не являются большим фактором в
старших классах, потому что вы ещё не особо умеете делать то, что
кому-нибудь надо, но в целом, довольно много великих вещей было
сделано в основном ради зарабатывания денег. Самуэль Джонсон (Samuel
Johnson) сказал: ``только болван мог когда-либо написать что-то не за
деньги.`` (Многие надеются, что он преувеличивал.) \newline

[9] Даже в колледже учебники плохи. Когда вы поступите в колледж, вы
обнаружите, что учебники (за небольшим исключением) написаны не
ведущими учёными в описываемой области. Написание колледжских
учебников это неприятная работа, которой занимаются в основном люди,
которым нужны деньги. Она неприятна потому, что издатели очень строго
контролируют её, и мало можно придумать вещей, более непрятных, чем
когда за вами надзирает кто-то, кто не разбирается в том, что вы
делаете. Это явление несомненно ещё хуже в производстве учебников для
старших классов. \newline

[10] Ваши учителя постоянно говорят вам, чтобы вы вели себя как
взрослые. Интересно, как бы им понравилось, если бы вы действительно
начали? Вы можете быть громкими и неорганизованными, но вы очень даже
послушные в сравнении со взрослыми. Если бы вы действительно начали
вести себя как взрослые, это было бы как если бы кучу взрослых
переместили бы в ваши тела. Представьте себе агента ФБР, или водителя
такси, или журналиста, если бы им сказали, что они должны спросить
разрешения, чтобы пойти в туалет, и только один человек может быть в
туалете одновременно. Не говоря уже о тех вещах, которым вас учат.
Если бы куча настоящих взрослых неожиданно обнаружила себя заключённой
в старших классах, первым делом они бы сформировали профсоюз и
передоговорились бы обо всех правилах с администрацией. \newline

\subsection{Примечания переводчика}

[пер.1] Имеется в виду американская ``high school``. Я не буду вдаваться в описание американской системы образования, только скажу, что эта high school - это школа, в которую идут после получения основного общего образования, для того, чтобы впоследствии поступить в колледж, т.е. это примерно соответствует 10-11 классам российской школы. \newline

[пер.2] В Соединённых Штатах существуют специальные платные подготовительные школы, выпускники которых по статистике чаще поступают в колледжи. \newline

[пер.3] Распространённая серия коротких аннотаций к книгам для тех, кому лень читать и составлять собственное мнение. http://www.cliffsnotes.com/ \newline

\end{document}
