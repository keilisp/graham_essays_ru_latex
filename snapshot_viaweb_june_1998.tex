\documentclass[ebook,12pt,oneside,openany]{memoir}
\usepackage[utf8x]{inputenc} \usepackage[russian]{babel}
\usepackage[papersize={90mm,120mm}, margin=2mm]{geometry}
\sloppy
\usepackage{url} \title{разбор полетов: Viaweb июня 1998 года}
\author{Пол Грэм} \date{}
\begin{document}
\maketitle

За несколько часов до того, как продаться «Yahoo» в июне 1998 года, я
сделал скриншот сайта Viaweb. Мне показалось, что интересно будет
однажды на него взглянуть. \newline

Первая вещь, на которую вы сразу обратите внимание, это то, как
компактны страницы. В 1998-ом экраны были заметно меньше нынешних.
Если я правильно припоминаю, то наша главная страница помещалась как
раз в стандартное окно, открываемое большинством пользователей в те
времена. \newline

Браузеры тогда (IE 6 появился только спустя 3 года) имели всего
несколько шрифтов, и у тех не было сглаживания. Если вам хотелось,
чтобы страница выглядела хорошо, вам нужно было обрабатывать
отображаемый текст в изображения. \newline

Возможно, вы заметили некоторую схожесть логотипов Viaweb’a и Y
Combinator’a. Когда мы начинали Y Combinator, это было нашей
внутрикомпанейской шуткой. Учитывая то, как прост красный круг, я был
удивлен тем, как мало компаний используют его в качестве своих
логотипов, однако позже я понял почему: \newline

На странице, посвященной нашей компании, вы можете обнаружить
загадочного индивида, которого зовут Джон МкАртъем. Роберт Моррис
(также известный как “Rtm”) был так отстранен от общественности после
своего «Червя», что не захотел, чтобы его имя присутствовало на сайте.
Я смог убедить его пойти на компромисс: мы использовали его биографию,
а имя заменили. После этого он немного успокоился на этот счет. \newline

Тревор выпустился из университета примерно в то же время, когда
произошла продажа “Yahoo”. Таким образом, ему удалось за 4 дня
превратиться из неплатежеспособного выпускника университета в
миллионера-кандидата наук. Именно статья, в которой отмечалось это
мероприятие, и стала кульминационной в моей карьере журналиста. В ней
я также приложил рисунок Тревора, сделанный мной во время той встречи. \newline

(Тревор также появлялся как “Trevino Bagwell” в категории
веб-дизайнеров нашего сайта. Там состояли люди, которых
предприниматели могли нанять, чтобы те разработали для них
онлайн-магазины. Мы внедрили его на случай, если кто-то из конкурентов
захочет запугать наших веб-дизайнеров. Кстати, наше предположение о
том, что его лого может отпугивать наших клиентов, оказалось
ошибочным.) \newline

В 90-ых, чтобы привлечь виртуальных посетителей, нужно было светиться
в газетах и журналах – не было тех путей быть найденными в сети,
которые есть сейчас. Поэтому мы отдавали \newline

\$16,000 ежемесячно одной PR-фирме, чтобы быть упомянутыми в прессе. К
счастью, газетчики нас полюбили. \newline

В нашей статье о получении трафика с поисковых систем (не думаю, что
термин “SEO” в те времена имел место быть) мы назвали только 7
значимых для этой функции поисковиков: “Yahoo”, “AltaVista”, “Excite”,
“WebCrawler”, “InfoSeek”, “Lycos”, и “HotBot”. Не кажется, что чего-то
не хватает? “Google” появился в сентябре того же года. \newline

Наш сайт поддерживал возможность интернет-транзакций с помощью сервиса
“Cybercash”, ибо, если бы этой возможности у нас не было, у нас были
бы серьезные проблемы со способностью к конкуренции на рынке услуг. Но
сервис был настолько ужасен, а заказы, поступавшие из магазинов, были
столь малы, что было бы проще, если бы предприниматели перешли на
систему заказа по телефону. У нас на сайте даже была страничка, на
которой был призыв к продавцам использовать именно этот метод с
клиентами, которые покупают физический товар, а не ПО. \newline

Весь сайт был сделан как мост, который сразу отправлял людей на “Test
Drive”. Это возможность была новой для нас – опробовать наше ПО
онлайн. Для того, чтобы не показывать конкурентам, как работает наш
код, мы поместили CGI-bin’ы в наши динамические адреса. \newline

У нас было несколько завсегдатаев. Стоит отметить, что «Frederick's of
Hollywood» получал больше всего трафика. Мы установили налог в
\$300/месяц на самые крупные магазины нашего хостинга, ибо было
несколько тревожно с финансовой точки зрения иметь пользователей с
большими объемами трафика. Однажды я посчитал, сколько стоит для нас
обеспечивать трафик для «Frederick's of Hollywood», и получилось
что-то около \$300/месяц. \newline

Учитывая то, что мы держали все магазины на своих серверах (в общей
сложности они набирали около 10 миллионов посещений за месяц), мы
потребляли, как на тот момент оказалось, очень много трафика. У нас
было проведено 2 линии типа T1s (пропускная способность ~3Мб/секунду),
ибо в те времена не существовало AWS. Даже находящиеся рядом сервера
казались нам слишком рисковой идеей, если учесть, что с ними вечно
что-то не ладное творится. В общем, наши сервера находились в наших
офисах. Если точнее, то в офисе Тревора. Он не захотел делить свой
офис с людьми, поэтому ему пришлось делить свой офис с шестью гудящими
серверами башенного типа. Мы даже назвали его офис «Банькой» из-за
количества тепла, которое выделяли эти громадины. Хотя, по большей
части его стопка оконных кондиционеров справлялась. \newline

Для страниц с описанием мы использовали шаблонный язык, именуемый
RTML. Оно должно было как-то расшифровываться, но на самом деле я его
так назвал в честь Rtm’a. RTML был Common Lisp’ом, который дополнили
макросами и библиотеками, а также конструктором структуры, который
создал ощущение, будто в нем есть строй, порядок. \newline

Мы постоянно обновляли ПО, поэтому оно толком не имело версий, однако
пресса тех времен привыкла, что они есть, поэтому мы их выдумали. Если
мы хотели стать действительно популярными, мы выкатили версию под
номером an integer (целочисленная). Надпись «Версия 4.0» была создана
нашим генератором случайных чисел. Кстати, весь сайт Viaweb был создан
нашим онлайн ПО, ведь мы хотели своими глазами увидеть, как и что
будет использовать клиент. \newline

В конце 1997 года мы выпустили многоцелевой шоппинговый поисковик
названный “Shopfind”. На тот момент он был достаточно сложным и
продвинутым технически: в нем был «паучек», который мог «посетить»
практически любой онлайн-магазин и найти нужный товар. \newline

\end{document}
