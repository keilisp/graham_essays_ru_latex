\documentclass[ebook,12pt,oneside,openany]{memoir}
\usepackage[utf8x]{inputenc} \usepackage[russian]{babel}
\usepackage[papersize={90mm,120mm}, margin=2mm]{geometry}
\sloppy
\usepackage{url} \title{Неравенство и риск} \author{Пол Грэм} \date{}
\begin{document}
\maketitle

«Вы просите государственных служащих сделать одну вещь, на которую они
менее всего способны: пойти на риск. Любой когда-либо работавший на
госслужбе знает, что самое важное заключается в принятии не правильных
решений, а тех, которые потом можно оправдать, если они окажутся
неверными.»

«Как все преступные действия, связь между богатством и властью
процветает в условиях тайны. Выставьте напоказ все транзакции, и вы
сильно преуспеете в устранении подобных вещей. Регистрируйте все».

image

Предположим, вы хотите избавиться от экономического неравенства. Есть
два способа сделать это: дать денег бедным или забрать их у богатых.
Правда, разница тут небольшая: если хотите дать денег бедным, их надо
где-то взять. Ведь не у бедных же их забирать — это их может доконать.
Придется забрать их у богатых.

Есть еще вариант сделать бедных богаче без банального отъема средств у
богатых. Вы можете помочь бедным стать более продуктивными, например,
улучшив доступность образования для них. Вместо того чтобы забирать
деньги у инженеров и отдавать их кассирам, вы могли бы помочь кассирам
стать инженерами.

Это прекрасная стратегия — сделать бедных богаче. Но как
свидетельствуют последние 200 лет, это не сокращает экономическое
неравенство, так как богатые тоже станут богаче. Если будет больше
инженеров, появится больше возможностей нанимать их и продавать им
вещи. Генри Форд не смог бы сколотить состояние на производстве машин
в обществе фермеров, ведущих натуральное хозяйство, у него не было бы
ни рабочих, ни покупателей.


Если вашей целью является устранение экономического неравенства, а не
просто повышение уровня жизни, не достаточно просто поднять черту
бедности. Что если один из новоиспеченных инженеров окажется
амбициозным и захочет стать еще одним Билом Гейтсом? Экономическое
неравенство только усилится. Если вы на самом деле хотите сократить
разрыв между бедными и богатыми, нужно давить и сверху вниз и снизу
вверх.

Как давить сверху? Вы можете уменьшить продуктивность людей, которые
делают деньги больше всех: заставить хирургов оперировать левой рукой,
известных актеров разжиреть и т.д. Но этот подход затруднительно
реализовать. Единственное практическое решение — дать людям делать то,
что у них лучше всего получается, а потом (или налогами или
ограничениями) конфисковать все то, что вы посчитаете излишком.

Поэтому давайте проясним, что значит сокращение экономического
неравенства: это не что иное, как отъем денег у богатых.

Преобразуя математическое уравнение в другую форму, можно зачастую
заметить новое. То же самое и в этом случае. Вы увидите, что отнимать
деньги у богатых с целью сократить неравенство может привести к
непредвиденным последствиям.

Все дело в том, чтобы привести к соответствию риск и награждение.
Ставка с 10\%-й вероятностью выигрыша должна приносить больше, чем
ставка с вероятностью 50\%, иначе никто не поставит. Снизив награду,
сократите и желание людей рисковать.

Транспонирую нашу начальную фразу: сокращение экономического
неравенства значит сокращение рисков, на которые люди готовы пойти.

Существует множество рисков, которые люди откажутся принимать при
снижении максимальной отдачи. Одна из причин, по которой высокие
налоги разрушительны, — из-за них запуск новых компаний становится
рискованным предприятием.

Инвесторы

Стартапы по своей сути рискованны. Стартап можно сравнить с маленькой
лодкой в открытом море. Одна большая волна — и вы тонете. Конкурентный
продукт, кризис в экономике, задержка в финансировании или в получении
разрешения, патентный иск, изменение в технических стандартах, уход
ключевого сотрудника, большие расходы — любое из перечисленного может
все разрушить в одночасье. Кажется, всего лишь один из десяти
стартапов успешен [1].

Наш стартап заплатил за первый этап финансирования своим инвесторам
сумму, которая в 36 раз больше вложенных средств. Получается, что, с
учетом текущего налогообложения в США, было бы выгодно инвестировать в
наш стартап с вероятностью успеха 1 к 24. Вот это уже похоже на
правду. Скорее всего, именно так мы и выглядели — пара «ботаников» без
реального делового опыта, работающих в квартире. Если подобный риск не
окупается, венчурное инвестирование не осуществляется.

Это бы не пугало, если бы существовали другие источники финансирования
для новых компаний. Почему бы не предоставить правительству или таким
огромным почти правительственным организациям вроде «Fannie Mae»
заниматься венчурным инвестированием вместо частных фондов?

Я скажу вам, почему это не сработает. Да потому что в этом случае вы
просите государственных или подобных им служащих сделать одну вещь, на
которую они менее всего способны: пойти на риск.

Любой когда-либо работавший на госслужбе знает, что самое важное
заключается в принятии не правильных решений, а тех, которые потом
можно оправдать, если они окажутся неверными. Если есть безопасное
решение, именно его выберет бюрократ. Что совершенно не подходит для
венчурных инвестиций. Природа бизнеса предполагает принятие даже
жутких рисков, если результат выглядит достаточно привлекательно.

Получение вознаграждения венчурными компаниями зависит от их внимания
к потенциалу: они получают процент от доходов фонда. И это помогает
преодолеть их объяснимый страх вкладывать средства в компанию, которая
управляется «ботаниками», которые выглядят как студенты (что вполне
возможно).

Если венчурным компаниям запретили бы богатеть, они бы вели себя как
бюрократы. Без расчета на выгоду они бы страшились потерь. И принимали
бы неверные решения. Они бы променяли «ботаников» на складно говорящих
дипломников MBA в костюмах, потому что инвестиции легче можно было бы
оправдать при неудачном исходе.

Основатели

Но получись у вас каким-либо образом заставить венчурные фонды
работать без цели разбогатеть, останется еще один тип инвестора,
которого вы просто не сможете заменить: основатели стартапа и
первоначальные сотрудники.

Их инвестиционный капитал — это их время и идеи, что эквивалентно
деньгам. О чем инвесторы часто забывают, относясь к ним как к сменным
частям, которые могут работать и без оплаты.

Тот факт, что вы инвестируете время, не меняет отношение риска к
прибыли. Вы будете вкладывать свое время во что-то сомнительное,
только если возможная выгода соответствующе велика [2]. Если большие
дивиденды будут запрещены, вы тоже начнете перестраховываться.

Как у многих основателей стартапов у меня получилось разбогатеть. Но
не потому что я хотел покупать дорогие вещи. Я хотел лишь
безопасности. Я хотел сделать достаточно денег, чтобы не беспокоится о
деньгах. Если бы мне запретили добиться этого через стартап, я бы
стремился к этому другими способами, например, устроился бы работать в
большую стабильную организацию, из которой сложно быть уволенным.
Вместо того чтобы отдавать все силы стартапу, я бы попытался найти
хорошую, спокойную работу в большой исследовательской лаборатории или
занял бы постоянную должность в университете.

Вот что происходит в обществе, где риск не вознаграждается. Если вы
сами не можете обеспечить себе безопасность, лучшее, что вы можете
сделать, это свить себе гнездышко в какой-нибудь большой организации,
где ваш статус будет зависеть от стажа [3].

Если бы мы каким-либо образом могли заменить инвесторов, то я не
представляю, кем можно заменить основателей. Инвесторы в основном
вкладывают деньги, которые из любого источника одинаково хороши. Но
вклад основателей — идеи. Их не заменишь.

Давайте еще раз вспомним цепочку рассуждений. Я нацелился на вывод, от
которого многих читателей надо будет отгонять пинками и окриками,
поэтому я постарался сделать каждое звено неразрывным.

Уменьшение экономического неравенства означает отъем денег у богатых.
Если риск и награда эквивалентны, снижение награды автоматически
умерит желание рисковать. Стартапы по сути своей рискованны. Не ожидая
соответствующей награды за риск, основатели не будут вкладывать свое
время в стартап. Основатели незаменимы. Поэтому, устранив
экономическое неравенство, вы устраните стартапы.

Экономическое неравенство не является следствием стартапов. Оно — их
движущая сила, словно падающая вода для водяной мельницы. Люди
начинают стартапы в надежде стать намного богаче, чем были раньше.

Рост

Тут наблюдается пропорциональность. Дело не только в том, что устранив
экономическое неравенство, вы устраните стартапы. Снижение количества
стартапов будет коррелировать со снижением разницы в благосостоянии
членов общества [4]. Увеличьте налоги, и желание рисковать
соответственно упадет.

А от этого будет хуже всем. Новые технологии и новые рабочие места
создаются новыми компаниями. Если вдруг не станет стартапов, то вскоре
не станет новых компаний, так же как без детей не будет взрослых.

Звучит благородно, когда мы говорим о сокращении экономического
неравенства. Кто будет спорить с этим? Неравенство должно быть плохим,
так? Но намного хуже звучит, что мы должны сократить количество новых
компаний. Хотя первое предполагает второе.

Умеряя желание инвесторов рисковать, мы не убьем, конечно, все
стартапы, но большинство жертв будут именно из разряда многообещающих.
Мне кажется, более рискованные стартапы показывают лучшие результаты.
И это пугающая мысль.

Конечно, не все состоятельные люди разбогатели посредством стартапов.
Что если мы введем налоги на все остальное, кроме стартапов? Не
удастся ли хотя бы таким образом сократить социальное неравенство?

В меньшей степени, чем вы можете подумать. Все те, кто хочет
разбогатеть, ринуться в область стартапов. И это могло бы быть
великолепно. Но я не думаю, что это сильно повлияло бы на
распределение богатства. Просто будет больше стартапов, которые только
на бумаге будут выглядеть стартапами — не так просто написать
настолько аккуратно законы.

Но представим, что мы настолько настойчивы в устранении неравенства,
что готовы отказаться от стартапов. Что тогда?

Как минимум, мы должны будем смириться с низкими темпами
технологического развития. Если вы считаете, что крупные компании
могут каким-то образом также быстро развивать новые технологии как и
стартапы, пожалуйста, с интересом выслушаю как именно (если вы можете
придумать такую правдоподобную историю, то точно сделать состояние на
книгах по бизнес-консалтингу для больших компаний) [5].

Ок, у нас замедлились темпы роста. Неужели это так плохо? Что ж,
остальные страны не станут нас ждать. Со временем окажется, что мы уже
ничего не изобретаем — все уже изобретено где-то еще. И в обмен мы
сможем предложить только сырье и дешевую рабочую силу. И когда вы так
низко падете, другие страны смогут делать с вами все, что им
заблагорассудится: устанавливать марионеточные правительства,
перекачивать себе ваших лучших работников, использовать ваших женщин
как проституток, хоронить свои токсичные отходы на вашей территории —
все то, что мы делаем сейчас с бедными странами. Единственным выходом
будет изоляция, как поступили коммунистические страны в 20-м веке. Но
проблема в том, что для этого надо будет превратиться в полицейское
государство.

Богатство и Власть

Я прекрасно понимаю, что не стартапы являются целью сторонников
устранения неравенства. Они выступают против богатства, которое в
союзе с властью становится самоподдерживающимся явлением. Например,
строительные фирмы, спонсирующие политиков, получают господряды, или
дети богатых родителей, поступающих в хорошие колледжи, потому что они
учились в профильных дорогостоящих школах. Но если вы попробуете
нападать на такой тип богатства с помощью экономических мер, вы
попутно нанесете вред и всем стартапам.

Проблема тут не в богатстве, а в коррупции. Так почему бы не взяться
за коррупцию?

Нам не нужно будет бороться с богатством, если мы сможем предотвратить
его слияние с властью. И в этом направлении уже добились успехов. Так,
прежде чем помереть от выпивки в 1925 г., никчемный внук Коммодора
Вандербилта Регги пять раз наезжал на пешеходов, двое из них умерли. К
1969 г., когда Тед Кеннеди вылетел с моста на острове Чаппакуиддик,
казалось, что рамки установлены на отметке «1». Сейчас, возможно, на
«0».

Но изменился не разброс в уровне богатства. Изменилась возможность
перевести богатство во власть. Как разбить связь между богатством и
властью? Требуйте прозрачности. Внимательно смотрите, как используется
власть, и требуйте отчет о том, как принимаются решения. Почему не все
полицейские допросы снимаются на видео? Почему 36\% абитуриентов
Принстонского университета в 2007 г. были из частных профильных школ,
в то время как только 1,7\% американских детей учатся в них? Почему на
самом деле США вторглись в Ирак? Почему официальные лица не раскрывают
больше информации о своих доходах?

Один из моих друзей, который хорошо разбирается в компьютерной
безопасности, говорит, что на самом деле нужно просто все фиксировать
и записывать. Когда он был ребенком, пытающимся взломать компьютеры,
он больше всего волновался за то, чтобы не оставить следов. Это его
беспокоило больше, чем сами системы защиты.

Как все преступные действия, связь между богатством и властью
процветает в условиях тайны. Выставьте напоказ все транзакции, и вы
сильно преуспеете в устранении подобных вещей. Регистрируйте все. Эта
стратегия уже показывает хорошие результаты, без побочных эффектов
вроде повсеместной бедности.

Я не уверен, что все осознают наличие связи между экономическим
неравенством и риском. Я сам осознал это не так давно. Я, конечно,
всегда знал, что если ничего не выйдет со стартапом, то можно
попробовать найти уютную штатную должность в исследовательской
лаборатории. Но я не понимал всю совокупность факторов, управляющих
моим поведением. Казалось очевидным, что страна, которая не дает людям
разбогатеть, обречена, и что это одинаково верно и для Рима времен
Диоклетиана, и для Великобритании времен Гарольда Вильсона. Но я не
понимал, какую важную роль во всем этом играет риск.

Если вы пойдете войной на богатство, вы устраните желание рисковать, а
вместе с ним и развитие. Поэтому, если мы стремимся к более
справедливому миру, нам следует тогда бороться с богатством, когда оно
сливается с властью.

Примечание

Мои благодарности Chris Anderson, Trevor Blackwell, Dan Giffin,
Jessica Livingston и Evan Williams за чтение черновиков данного эссе,
а также Langley Steinert, Sangam Pant и Mike Moritz за информацию о
венчурных инвестициях.

[1] Успех здесь определяется первоначальной точкой зрения инвестора:
первичное размещение акций или удачная продажа. Привычная статистика
«1 из 10» выглядит подозрительно точной, но общение с венчурными
фондами говорит за ее правдоподобность. Однако ведущие венчурные фонды
ожидают лучшего результата.

[2] Я не призываю основателей садиться и высчитывать ожидаемую прибыль
после налогообложения. Пример уже добившихся успеха людей служит им
мотивацией. И эти примеры на самом деле дают представление о размере
прибыли после налогообложения.

[3] Предположение: изменение в благосостоянии в стране или организации
(не коррумпированной) будет обратно пропорционально значимости
ранговой системы. Поэтому, если вы снизите разброс величин в области
благосостояния, система рангов станет соответственно более важной.
Пока что я не знаю противоречащих этому примеров, хотя в
коррумпированных странах оба явления могут сосуществовать (благодарю
Daniel Sobral за это верное замечание).

[4] В стране с по-настоящему феодальной экономикой вы можете успешно
перераспределять богатство, так как там нет стартапов.

[5] Стартапы так хорошо окупаются именно благодаря присущей им
скорости внедрения новых технологий. Как я объяснял в «Как делать
богатство» («How to Make Wealth»), в стартапе вы сжимаете работу
длинною в жизнь в несколько лет. Это очевидно.


\end{document}
