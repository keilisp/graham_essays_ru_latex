\documentclass[ebook,12pt,oneside,openany]{memoir}
\usepackage[utf8x]{inputenc} \usepackage[russian]{babel}
\usepackage[papersize={90mm,120mm}, margin=2mm]{geometry}
\sloppy
\usepackage{url} \title{Как найти идею для стартапа} \author{Пол Грэм}
\date{}
\begin{document}
\maketitle

Лучший способ найти идею для стартапа — не думать о ней. Найдите
проблему, и лучше — если она есть у вас самого.


У лучших стартап идей есть 3 общих черты: там есть что-то, что нужно
самим фаундерам; они сами могут это создать; и только нескольким
другим людям это что-то кажется достойным реализации. Microsoft,
Apple, Yahoo, Google, Facebook — все прошли по этому пути.

Проблемы

Почему же так важно работать над проблемой, которая есть у вас самих?
Одна из причин — убедиться, что проблема действительно существует.
Звучит очевидно, что работать надо только над проблемами, которые
реально существуют. И тем не менее это самая частая ошибка,
совершаемая стартапами, — решать проблему, которой ни у кого нет.

Я и сам ее совершил. В 1995 запустил проект, призванный вывести все
художественные галереи в онлайн. Но так не работает бизнес искусства.
Так почему я потратил 6 месяцев, работая над этой идиотской идеей?
Потому что игнорировал пользователей. Я изобрел модель мира, которая
не соответствовала реальности, и работал, исходя из нее. И я не
заметил, что модель неверна, пока не попытался убедить пользователей
платить за то, что мы разработали. Даже после этого потребовалось
много времени, пока я не осознал все. Я был так привязан к своей
модели мира и так много потратил времени на разработку софта, что
пользователи были обязаны хотеть его!

Так почему так много фаундеров строят проекты, которые никому не
нужны? Потому что они начинают с попыток придумать идею для стартапа.
Такой способ мышления вдвойне опасен: он не просто не принесет никаких
хороших идей, он породит плохие идеи, которые покажутся достаточно
правдоподобными, чтобы обмануться и начать работать над ними.

У нас в YC (YCombinator) мы называем их «выдуманными» или «ситкомами».
Представьте кто-нибудь из телесериала решит начать стартап. Сценаристу
пришлось бы придумать что-нибудь. Но придумать хорошую стартап-идею
тяжело. Это не то, что можно сделать по просьбе. Поэтому сценарист
предложит идею, которая будет звучать правдоподобно, но на самом деле
будет плохой.

Например, социальная сеть для владельцев домашних питомцев. Она не
кажется безусловно ошибочной. У миллионов есть питомец. Зачастую они
очень заботяться о них и тратят на них много денег. Конечно многие из
владельцев хотели бы иметь сайт, чтобы общаться с другими владельцами.
Вероятно не все из них, но если бы 2-3 процента из них постоянно
посещали сайт, у вас могли бы быть миллионы пользователей. Вы могли бы
показывать им целевую рекламу и возможно предложить какие-нибудь
премиум фичи.

Опасность подобной идеи в том, что если вы опросите ваших друзей,
имеющих питомцев, они никогда не скажут, что никогда не воспользуются
данным сервисом. Они скажут: «Да, я могу себе представить, что
пользуюсь чем-то подобным». Даже когда стартап будет запускаться, оно
будет звучать правдоподобно для многих людей. Сами они не хотят это
использовать, по крайней мере прямо сейчас, но они могут представить,
что другие люди пользуются этим. Сагрегируйте эту реакцию среди всего
населения и в результате получите ноль пользователей.

Колодец

Когда стартап запускается, должно быть по крайней мере несколько
пользователей, которые реально нуждаются в том, что вы делаете. А не
просто люди, которые возможно видят себя когда-нибудь пользующимися
вашим сервисом, но которым он не нужен прямо сейчас. Обычно эта
первоначальная группа пользователь мала. Причина проста: если бы эта
группа, которой срочно что-то требуется, была большой и это что-то
можно было построить теми силами, которые стартапу обычно требуются
для разработки первой версии, оно вероятно уже существовало бы. Таким
образом, вам необходимо чем-то пожертвовать: вы можете построить
что-нибудь для большой аудитории, которой это не очень нужно; либо для
маленькой аудитории, которой это действительно нужно. Выбирайте
второе! Не все идеи этого типа хорошие стартап-идеи, но почти все
хорошие стартап-идеи относиться к этому виду.

Представьте график, где ось Х отражает количество людей, которые хотят
ваш продукт, а ось Y — как сильно они хотят. Если зеркально отобразить
ось Y относительно оси Х, можно представить себе компании в виде дыр.
Google будет огромным кратером: сотни миллионов людей используют его и
он им очень нужен. Новорожденный стартап не должен ожидать, что выроет
котлован такого объема. Таким образом, у вас две опции относительно
формы дыры, с которой вы начинаете: либо вы роете широкую, но плоскую
дыру; либо узкую, но глубокую, как колодец.

Надуманные стартап-идеи обычно относятся к первому типу. Много людей
средне заинтересованы в социальной сети для владельцев домашних
питомцев.

Почти все хорошие стартап-идеи — второго типа. Microsoft был колодцем,
когда делал Altair Basic. Было всего несколько тысяч пользователей, но
без него им пришлось бы программировать на машинном языке. 30 лет
спустя у Facebook была такая же фигура. Их первый сайт был только для
студентов Гарварда, всего для несколько тысяч, но эти пользователи
хотели этот сайт сильно.

Если у вас есть идея для стартапа, спросите себя: кому это нужно прямо
сейчас? Кому нужно это так сильно, что он будет пользоваться даже
уродливой первой версией продукта, сделанной стартапом из двух
человек, о котором он никогда не слышал? Если у вас нет ответа,
возможно у вас плохая идея.

Но вам не нужна узость колодца сама по себе. Вам нужна его глубина.
Узость получается как побочный продукт оптимизации глубины (и
скорости). И вы ее — узость — почти всегда получите. На практике связь
между глубиной и шириной так сильна, что это всегда хороший сигнал,
когда вы знаете, что идея сильно притягивает определенную группу или
тип пользователей.

И хотя спрос в форме колодца является почти необходимым условием
хорошей стартап-идеи, оно не достаточное. Если бы Марк Цукерберг
построил что-то, что могло бы понравиться только студентам Гарварда,
это не было бы хорошей стартап-идеей. Facebook был хорошей идеей,
потому что запускался на небольшом рынке в то время как был быстрый
путь «наружу». Колледжи достаточно похожи в том смысле, что если ты
построил facebook, который подходит Гарварду, оно подойдет и любому
другому колледжу. Так, что вы можете распространить идею быстро среди
всех колледжей. После того, как у вас все студенты колледжей, вы
получите всех остальных просто пуская их в систему.

Тоже самое с Microsoft: Basic для Altair; Basic для других
вычислительных машин; другие языки, помимо Basic; операционная
система; софт; IPO.

Сознание

Но как определить, есть ли путь «наружу» у идеи? Как понять, является
ли это зачатком большой компании или всего лишь продуктом, обреченным
на узкую нишу? Часто это не возможно сделать. Фаундеры AirBnb не
осознавали в начале, на рынок какого масштаба они покусились.
Изначально у них была более узкая идея. Они собирались позволить
владельцам сдавать свое пространство во время выставок, конференций.
Они не предвидели экспансию этой идеи; она сама себя навязывала
постепенно. Все что они знали в начале, что на крючке какая-то рыба.
Возможно, это столько же, сколько знали изначально Билл Гейтс и Марк
Цукерберг.

Изредка очевидно с самого начала, что у идеи есть путь «наружу» из
первичной ниши. А иногда я вижу путь, который сразу не очевиден — это
одна из наших специализаций здесь в YC. Но все же есть пределы того,
как точно это можно сделать. И не важно, сколько у вас опыта. Самое
главное в понимании пути «наружу» — признать факт, что его сложно
увидеть.

Но если ты не можешь предсказать, есть ли у идеи этот большой путь, —
как выбирать между идеями? Правда огорчительна, но интересна: если вы
человек правильной «породы», то у вас правильный тип интуиции. Если вы
на острие достижений в области, которая быстро меняется, и у вас
предчувствие, что чем-то стоит заняться, вероятность того, что вы
правы, выше.

В книге «Дзэн и искусство ухода за мотоциклом» Роберт Пирсинг пишет:
«Вы хотите знать как нарисовать совершенный рисунок? Это просто.
Сделайте самого себя совершенным, а потом просто рисуйте совершенно
естественно».

Я удивлялся этому отрывку с тех пор как прочитал его в универе. Я не
уверен, на сколько этот совет полезен конкретно для рисования, но
идеально описывает нашу ситуацию. Эмпирически: лучший путь к хорошей
стартап-идеи — стать тем типом человека, который их имеет.

Быть на острие достижений в какой-то области не означает, что вам
нужно быть тем, кто двигаеть это все вперед. Можно быть и просто
юзером на острие. Facebook показался Марку хорошей идеей, не столько
потому, что он был программистом, сколько потому, что он пользовался
компьютером много. Если бы вы спросили большую часть 40-летних людей в
2004 году, не хотели бы они сделать свою жизнь полу-публичной в
интернете, они бы ужаснулись этой идеей. Но Марк уже жил в онлайне;
для него это было естественно.

Пол Бухайт говорит, что люди на острие быстроизменяющейся области
«живут в будущем». Если эту фразу соединить с фразой Пирсинга,
получим:

Живи в будущем и создавай то, чего не хватает.

Это отлично описывает путь, по которому многие (если не все) большие
стартапы начинали свое путешествие. Ни Apple, ни Yahoo, ни Google, ни
Facebook не планировали в начале стать компаниями. Они выросли из
вещей, которых, как казалось их фаундерам, не хватает в этом мире.

Если вы посмотрите на способ, которым успешные фаундеры находили свои
идеи, то обнаружите, что обычно это результат какого-либо внешнего
раздражителя, поражающего «подготовленное» сознание. Билл Гейтс и Пол
Аллен услышали о Altair и подумали «Бьюсь об заклад, мы могли бы
написать интерпретатор бэйсика для него». Дрю Хастон понял, что забыл
свою USB-флэшку и подумал «Мне нужно сделать так, чтобы мои файлы жили
в онлайне». Многие люди слышали о Altair, многие забывали флешки.
Причина, по которой эти раздражители заставляли их фаундеров начинать
проекты, заключалась в том, что их опыт научил замечать те
возможности, которые им представляются.

Глагол, который следует использовать с уважением к стартап-идеям, — не
«придумывать», а «замечать». У нас YC мы называем идеи, выросшие
естественным образом из опыта их фаундеров, — органическими
стартап-идеями. Большинство успешных стартапов начинают именно так.

Возможно, это не то, что вы хотели услышать. Вероятно, вы ожидали
услышать рецепт, как придумать идею, а вместо этого я вам рассказываю,
что ключевое — иметь правильно подготовленное сознание. Но это тоже
тип рецепта: просто такой, следование которому может занять в
неблагоприятном случае год вместо одних выходных.

Если вы не на острие чего-то быстроизменяющегося, вы можете туда
попасть. Например, любой достаточно умный человек может попасть на
острие программирования (как то разработка мобильных приложений) в
течение года. Поскольку успешный стартап съест 3-5 лет вашей жизни,
год подготовки был бы резонным инвестированием. Особенно, если вы еще
ищете ко-фаундера.

Но вам не нужно учиться программировать, чтобы быть на острие сферы,
которая быстро меняется. Другие сферы тоже меняются быстро. И хотя
учиться программировать не обязательно, это резонно для обозримого
будущего. Как говорит Макр Андрэссен, софт съедает мир, и этот тренд
будет продолжаться ближайшие десятилетия.

Умение программировать также означает, что когда у вас будут идеи, вы
будете в состоянии их реализовать. Это не вопрос жизни и смерти (Джэф
Безос -основатель Amazon.com — не умел), но преимущество. Хорошее
преимущество, если вы рассматриваете идею о запуске Facebook. Вместо
простых размышлений типа «Это интересная идея», вы можете думать «Это
интересная идея. Постараюсь построить первую версию сегодня вечером».
Оно даже лучше, когда вы программист и одновременно целевой
пользователь, так как тогда цикл разработки новой версии и
тестирования ее на пользователях может происходить в одной голове.

Замечать

Как только вы начинаете жить в будущем (в каком-то отношении), способ
заметить стартап-идею — это посмотреть на вещи, которых не хватает.
Если вы действительно на острие быстроразвивающейся сферы, то всегда
найдете кучу вещей, которых не хватает. Что не будет очевидным, так
это то, что они — идеи для стартапа. Поэтому, если вы хотитие найти
стартап-идею, необходимо не просто включить фильтр «Чего не хватает?»
Также нужно выключить любой другой фильтр, в частности «Может ли это
стать большой компанией?» Будет еще достаточно времени сделать эту
проверку позже. Но если вы будете думать об этом изначально, это
приведет к тому, что вы не только отфильтруете много хороших идей, но
и к тому, что вы сфокусируетесь на плохой идее.

Большинство вещей, которых не хватает, потребуют некоторого времени,
чтобы их заметить. Нужно практически обмануть самого себя, чтобы
увидеть эти идеи вокруг нас.

Но мы знаем — идеи окружают нас. И нет такой проблемы, у которой нет
решения. Вероятность ничтожно мала, что сейчас именно тот момент,
когда технологический процесс остановится. Можете быть уверенными,
люди построят в ближайшие пару лет вещи, которые вас заставят
удивиться: «Как же я жил без этого до сих пор?»

И когда эти проблемы будут решены, вероятно, ретроспективно они будут
выглядеть невероятно очевидными. Поэтому вам необходимо выключить
фильтр, который мешает видеть их. Самая мощное препятствие — это
воспринимать текущее состояние мира как данность. Даже самые открытые
и восприимчивые из нас часто страдают этим. Вы не смогли бы добраться
от вашей кровати до входной двери, если бы не перестали подвергать все
сомнению и задаваться вопросами.

Но если вы ищите стартап-идею, вы можете пожертвовать некоторой
эффективностью принятия все как данность, и начать задаваться
вопросами. Почему наша почта переполнена? Потому что мы получаем
слишком много писем или потому что тяжело удалить письмо из входящей
папки? Почему мы получаем слишком много писем? Какие проблемы люди
пытаются решить с помощью писем? Есть ли способы лучше решить ее?
Почему мы не удаляем письма после прочтения? Является ли входящая
папка идеальным решением для этого?

Обратите особое внимание на вещи, которые вас раздражают. Преимущество
от принятия текущего положения вещей в качестве данности не только в
том, что это делает жизнь (локально) более эффективной, но и в том,
что мы становимся более терпимыми. Если бы вы знали про все те вещи,
что появятся в следующие 50 лет, но не имели бы сейчас, вам бы
показался сегодняшний день весьма ограниченным, также как если бы вас
отправили назад на 50 лет в машине времени. Если вас что-то
раздражает, возможно, это потому, что вы живете в будущем.

Если вы найдете правильный тип проблемы, вероятно вы сможете описать
ее как очевидную. По крайней мере для вас самих. Когда мы запускали
Viaweb, все онлайн магазины создавались руками: веб-дизйнеры создавали
каждую страницу отдельно в HTML. Для нас, программистов, было
очевидно, что эти сайты должны генерироваться автоматически софтом.

Это означает достаточно странную вещь, что нахождение стартап-идеи —
вопрос видения очевидного. Это говорит о том, на сколько весь процесс
странный: вы пытаетесь увидеть вещи, которые очевидны, но тем не
менее, которые еще не видели.

Поскольку, все что вам нужно сделать, это отпустить сознание,
возможно, лучше не пытаться в лоб атаковать проблему, т.е. сесть и
начать думать о ней. Лучшее решение — просто запустить процесс в
фоновом режиме и приглядываться к вещам, которых еще не хватает.
Работайте над сложными проблемами, следуя вашему любопытству, но
найдите секунду заглянуть самому себе через плечо, отмечая недостатки
и «аномалии».

Дайте себе немного времени. У вас достаточно контроля над скоростью, с
которой вы превращаете свое сознание в «подготовленное», но у вас
гораздо меньше контроля над внешними раздражителями, зажигающими идеи,
когда они — раздражители — поражают сознание. Дрю Хастон работал над
менее перспективной идеей до DropBox: стартап по подготовке к SAT
(экзамен). Но DropBox был намного лучшей идеей: как в абсолютном
смысле, так и в плане соответствия его опыту и навыкам.

Один из способов обмануть себя, чтобы приметить идею, — работать над
проектом, который кажется должен быть крутым. Когда вы чем-то подобным
занимаетесь, вы неизбежно, совершенно естественно склонны строить
вещи, которых нет. Строить что-то, что уже существует, не показалось
бы вам таким интересным.

В то время как попытка придумать стартап-идею обычно заканчивается
плохо, работа над штукой, которая могла бы быть отверженна как
«игрушка», часто приводит к хорошим идеям. Когда что-то описывается
как игрушка, это означает, что у нее есть все, что присуще идеи, кроме
значимости. Это круто; юзерам нравится, просто оно не имеет
значимости. Но если вы живете в будущем, и вы строите, что-то крутое,
что нравиться юзерам, то это может иметь большее значение, чем оно
кажется посторонним. Микро-компьютеры казались игрушками, когда Apple
и Microsoft начали работать над ними. Я достаточно стар, чтобы помнить
об этой эпохе: людей, владевших своим собственным компьютером,
называли «любителем». BackRub казался непоследовательным научным
проектом. Facebook был всего лишь способом для студентов подглядывать
друг за другом.

У нас в YC мы всегда заинтригованы встретить стартап, работающий над
чем-то, что в нашем представлении какой-нибудь всезнайка с форума
назовет игрушкой. Для нас это добрый знак, что идея хорошая.

Если вы можете себе позволить принять долгосрочную перспективу (а вы
не можете не позволить), фраза: «Живи в будущем и создавай то, чего не
хватает» следует перефразировать:

Живи в будущем и создавай то, что кажется интересным.

Университет

Вместо того, чтобы пытаться научиться «предпринимательству», я бы
посоветовал студентам колледжей следующее: предпринимательство — это
навык, который лучше всего приобретается «в бою». Примеры самых
успешных фаундеров тому доказательство. В колледже стоит тратить время
на перемещение себя в будущее. И колледж в этом смысле — бесподобная
возможность. Какое упущение, пожертвовать возможностью решить сложную
часть задачи запуска стартапа — стать тем типом людей, у которых есть
органические идеи, — потратив время на изучение простой части.
Особенно, если принять во внимание, что вы даже сильно ничему не
научитесь. Не больше того, что можно узнать о сексе, сидя в классе.
Все, что вы учите, — слова.

Стык областей — особенно плодородный источник идей. Если вы умеете
программировать и начинаете учить что-то из другой области, вы
вероятно встретите проблемы, которые софт мог бы решить. На самом
деле, вероятность найти хорошую проблему в других областях даже выше:
во-первых, обитатели этих областей вряд ли имели опыт решения проблем
при помощи софта, такой как у программистов; во-вторых, поскольку вы
проникаете в новую область абсолютно несведущим, вы ничего не знаете о
ее состоянии, которое воспринималось бы как данность.

Так, если вы студент по теории вычислительных машин, и вы хотите
запустить стартап, то лучше вместо курса по предпринимательству взять
курс, например, по генетике. Или еще лучше: пойти работать в компанию,
связанную с биотехнологиями. Студентам-программистам обычно легко
устроиться на летнюю практику в компании, занимавшиеся железом или
софтом. Но если вы хотите найти стартап-идею, то лучше найти
стажировку в какой-нибудь не связанной области.

Или просто не берите дополнительных курсов и просто создавайте вещи.
Это не случайность, что и Microsoft, и Facebook оба стартовали в
январе. В Гарварде это «время чтения», когда у студентов нет занятий,
потому что они готовятся к выпускным экзаменам.

Но у вас не должно быть чувства, что вы должны создать вещь, которая
станет стартапом. Это всего лишь предварительная оптимизация. Просто
создавайте. Желательно, совместно с другими студентами. Не только
занятия делают университет хорошим местом по перемещению себя в
будущее. Вы также окружены другими людьми, которые пытаются делать те
же вещи. Если вы будете работать с ними над проектами, то это
закончится тем, что вы не просто будете создавать органические идеи,
но и органические идеи в органической команде — и это, по моему опыту,
лучшая комбинация.

Остерегайтесь исследований. Если студент создает что-то, что все его
друзья начинают использовать, то это скорее всего представляет из себя
хорошую идею. В то время как кандидатская диссертация — вряд ли. По
какой-то причине, чем больше проект должен считаться исследованием,
тем меньше вероятность, что там есть что-то, что можно было бы
превратить в стартап. Я думаю, что причина в том, что подмножество
идей, которые могут считаться предметом исследования, такое узкое, что
весьма невероятно, что проект, который удовлетворяет этим ограничениям
также удовлетворяет ортогональным ограничениям, заключающимся в
решении какой-либо пользовательской проблемы. В то время как если
студент (или профессор) работают над побочным проектом, они
автоматически тяготеют в сторону решения чей-нибудь проблемы —
возможно даже с дополнительной энергией, источник которой —
освобождение от множества ограничений исследования.

Конкуренция

Поскольку хорошая идея должна выглядеть очевидной, то когда у вас
будет именно такая, вы будете склонны думать, что уже опоздали. Пусть
это вас не пугает. Беспокойство о том, что вы опоздали — один из
признаков хорошей идеи. 10 минут поиска в интернете обычно решают этот
вопрос. Даже если вы найдете еще кого-то, работающего над этой же
идеей, то возможно, вы еще не опоздали. Редкий случай, когда стартап
гибнет от рук другого стартапа. Так редко, что этой вероятностью можно
пренебречь. Если только вы не обнаружили конкурента, который замыкает
на себе пользователей (lock-in — типа соцсетей), препятствуя им
выбрать вас, — не отбрасывайте идею!

Если вы не уверены, спросите пользователей. Вопрос о том, опоздали ли
вы, является частью другого вопроса: нужно ли срочно кому-нибудь то,
что вы планируете делать. Если у вас есть что-то, чего нет у
конкурентов и некоторое подмножество пользователей срочно в этом
нуждается, считайте, у вас есть хороший плацдарм.

Вопрос только в том, достаточно ли большой этот плацдарм. А еще важнее
— кто там окажется: если плацдарм будет состоять из людей, которые
делают что-то, чем будут пользоваться гораздо больше людей в будущем,
то этот плацдарм достаточно большой, каким бы маленьким он сейчас ни
был. Например, вы создаете что-то отличное от конкурентов, но оно
работает только на новых смартфонах, — это, вероятно, все равно
достаточно большой плацдарм.

Ошибайтесь на той же стороне, что и ваш конкурент. Неопытные фаундеры
преувеличивают опасность конкурента, которую он реально из себя
представляет. Добьетесь ли вы успеха зависит гораздо больше от вас,
чем от конкурента. Так что лучше хорошая идея и конкуренты, чем плохая
— но без них.

Вам не нужно беспокоиться о выходе на переполненный рынок пока у вас
есть четкое понимание того, что все остальные упускают из вида. На
самом деле это очень перспективная отправная точка. Google был именно
такого рода идеей. Ваше понимание должно быть более точным, чем «мы
собираемся сделать Х, которое будет не отстойным». Даже если. Вы
должны быть в состоянии сформулировать это с точки зрения того, что
упускают из вида конкуренты. Лучше всего, если вы можете сказать, что
у них не хватило мужества следовать своим убеждениям. А вы планируете
сделать то, что сделали бы они, следуя, не смотря ни на что, своему
пониманию проблемы. И Google опять тому пример. Предшествующие
поисковые системы уклонялись от наиболее коренных внедрений, над
которыми они работали. В частности, потому что, чем лучше бы они
делали свою работу, тем быстрее уходил пользователь.

Переполненный рынок на самом деле хороший знак, так как это означает
две вещи: во-первых, что есть спрос; во-вторых, что ни одно из
существующих решений достаточно хорошее. Стартап не должен надеяться
зайти на рынок, который и большой, и на котором нет конкурентов. Таким
образом, любой стартап, добивающийся успеха, либо входит на рынок с
существующими конкурентами (но вооружен секретным оружием, которое
позволит переманить пользователей; например — Google), либо входит на
рынок, который выглядит маленьким, но который станет большим
(например, Microsoft).

Фильтры

Есть еще два фильтра, от которых вам нужно избавиться, чтобы начать
замечать стартап-идеи: фильтр «непривлекательности» и фильтр
«геморроя».

Большинство программистов мечтают начать стартап следующим образом:
написать гениальный код, выложить на сервер и приобрести
пользователей, которые платят им кучу денег. Они предпочитают не
связываться со скучными проблемами или грязной работенкой из реального
мира. Это предпочтение вполне резонно, поскольку такие вещи наводят
тоску. Но это предпочтение на столько распространено, что пространство
с удобными стартап-идеями уже обобрали подчистую. Если вы позволите
своему сознанию спуститься по улице на несколько кварталов вниз к
грязным, скучным идеям, вы найдете пару ценных, которые только сидят и
ждут реализации.

Фильтр «геморроя» так опасен, что я написал отдельное эссе о том
состоянии, в которое он вводит, и которое я называю «геморройная»
слепота. Я привожу в пример Stripe: стартап, который извлек пользу от
устранения этого фильтра. И пример достаточно яркий. Тысячи
программистов могли увидеть эту идею; тысячи программистов знают каким
неудобным был процесс оплаты до Stripe. Но когда они ищут идею для
стартапа, они не видят эту идею, потому что подсознательно они
избегают имение дела с платежными системами. Для Stripe все эти
системы тоже геморрой, но вполне выносимый. На самом деле они,
возможно, имели его вовсе не так много. Поскольку страх от геморроя с
платежными системами удерживал большинство людей на расстоянии от
идеи, это позволило Stripe достаточно гладко решить проблемы в других
вопросах, которые обычно болезненные. Например, наращивание
пользовательской базы. Им не нужно было сильно стараться, чтобы те
обратили на них внимание, потому что пользователи в отчаянии ждали то,
что Stripe создавал.

Фильтр «непривлекательности» похож на фильтр «геморроя», за тем
исключением, что он держит вас в стороне от проблем, которые вы
презираете, а не которых боитесь. Мы преодолели этот фильтр, когда
работали над Viaweb. Были интересные вещи, связанные с архитектурой
софта, но мы не были заинтересованы в e-commerce как таковом. Мы
просто видели, что это та проблема, которая требует решения.

Устранение фильтра «геморроя» важнее, чем устранение фильтра
«непривлекательности», поскольку геморрой скорее всего лишь иллюзия. И
даже если это в какой-то степени не так, он — худшая форма потакания
своим слабостям. Успешный стартап — дело достаточно тяжелое и
утомительное. И не важно, о чем ваш стартап. И даже, если сам продукт
не подразумевает большого количества геморроя, все равно прийдется
иметь дело с инвесторами, наймом и увольнением сотрудников и т.д.
Таким образом, если у вас есть какая-то идея, о которой вы думаете,
что она крутая, но вы сторонитесь ее из-за страха огрести проблем, —
не беспокойтесь: любая достаточно хорошая идея обеспечит вас геморроем
в избытке.

Фильтр «непривлекательности», хоть и источник ошибок, но не на столько
бесполезен как фильтр «геморроя». Если вы работаете на острие
быстроменяющейся сферы, ваше понимание о том, что привлекательно,
будут в какой-то степени коррелировать с тем, что действительно
полезно на деле. Особенно с годами и приобретением опыта. Плюс, если
вы найдете привлекательную идею, вы будете работать над ней с большим
энтузиазмом.

Рецепты

И хотя лучший способ обнаружить стартап-идею — стать тем типом людей,
у которых они есть, и после этого создавать с интересом все, что бы то
ни было, — тем не менее часто это роскошь, которую мы не можем себе
позволить. Иногда вам нужна идея прямо сейчас. Например, вы работаете
над стартапом, и ваша первоначальная идея оказывается неудачной.

Остаток эссе я посвящу уловкам, которые помогают найти стартап-идею по
требованию. И хотя, по моему опыту, лучше использовать органический
путь, этот способ тоже может привести вас к успеху. Просто вам
потребуется больше самоконтроля. Когда вы используете органический
подход, вы даже не замечаете идею, если только она не заявляет во
всеуслышание, что чего-то по-настоящему не хватает. Но когда вы
предпринимаете сознательную попытку подумать об идее для стартапа, вам
необходимо заменить свои естественные ограничения на самоконтроль.
Тогда вы увидите гораздо больше идей; большинство из них — плохих.
Поэтому вы должны быть в состоянии отфильтровать их.

Органические идеи таят в себе опасность, они ощущаются как
вдохновение. Есть куча историй про успешные стартапы, которые
начинались, когда у фаундеров было то, что казалось сумасшедшей идеей,
но они «просто знали», что их идея перспективна. Когда у вас такое
чувство по отношению к идее, которую вы пытались выдумать, вероятно,
вы ошибаетесь.

Когда ищите идею, сфокусируйтесь на сферах, в которых вы разбираетесь.
Если вы эксперт по базам данных, не создавайте приложение-чат для
подростков (если только вы сами не подросток). Возможно, это и хорошая
идея, но вы не можете доверять своим суждениям относительного этого,
поэтому — игнорируйте ее. Должны быть другие идеи с базами данных, о
качестве которых вы можете судить. Вам кажется, тяжело найти хорошую
идею с использованием БД? Все потому, что ваши опыт и знания завышают
стандарты. Но ваши суждения о приложении-чате такие же плохие, вы
просто позволили себе пройти в эту область, воспользовавшись эффектом
Даннинга — Крюгера.

Но правильное место для начала поиска идей — вещи, которые нужны вам.
Всегда есть вещи, которые вам необходимы.

Неплохой трюк — спросить себя, не случалось ли на вашей предыдущей
работе, что бы вы говорили: «Ну почему никто не сделает Х? Если бы
кто-нибудь это сделал, мы бы купили это в следующий же момент». Если
вы можете припомнить о таком Х, о котором люди так говорили, возможно,
у вас уже есть идея. Вы знаете, что есть спрос, а люди не стали бы так
убиваться по поводу вещей, которые невозможно создать.

В целом, постарайтесь спросить себя, нет ли чего-то необычного у вас,
что делает ваши потребности отличными от потребностей других людей.
Возможно, вы не одиноки. Особенно хорошо, если вы отличаетесь тем, к
чему люди неизбежно, но постепенно придут.

Если вы поменяли идею, то одна необычная вещь у вас уже есть — идея,
над который вы работали до этого. Не обнаружили ли вы каких-либо
потребностей, пока работали над ней? Несколько хорошо известных
стартапов начинали именно так. Hotmail начинали как нечто, что его
фаундеры написали, чтобы общаться по поводу их предыдущего стартапа во
время нахождения на основной работе.

Особенно перспективный способ быть необычным — это быть молодым.
Некоторые из наиболее ценных идей уходят корнями в подростковый
(13-20) и 20-летный возраст. И хотя у молодых фаундеров есть некоторые
недостатки, они единственные, кто реально понимает своих сверстников.
Было бы очень сложно кому-нибудь, кто не учился в колледже, придумать
Facebook. Если вы молодой фаундер (до 23), есть ли вещи, которые вы и
ваши друзья хотели бы сделать, но которые не позволяют сделать текущие
технологии?

Но лучше вашей личной неудовлетворенной потребности —
неудовлетворенная потребность кого-то другого. Постарайтесь поговорить
со всеми, с кем можно, про пробелы, которые они обнаружили в этом
мире. Чего не хватает? Чего бы они хотели сделать, но не могут? Чего
утомительного и раздражающего, и особенно в их работе? И пусть
разговор будет достаточно общим; не слишком усердствуйте в поисках
стартап-идеи. Вы всего лишь ищите что-нибудь, что сможет зажечь в вас
идею. Возможно, что обнаружите проблему, которую они не осознавали,
потому что, в отличие от них, вы знаете, что ее можно было бы решить и
как.

Когда вы найдете неудовлетворенную потребность, чужую, она может
показаться сперва очень размытой. Человек, которому что-то нужно,
может не знать, что именно. В этих случаях я рекомендую фаундерам
действовать в роли консультантов: действовать так, как если бы их
пригласили решить проблему данного пользователя. Проблемы людей
достаточно похожи, поэтому код, написанный вами таким способом, может
быть использован вновь для другого человека. А то, что не пригодится,
будет небольшой ценой за уверенность, что вы достигли дна колодца.

Один из способов убедиться, что вы действительно решаете чужие
проблемы, — сделать их своими. Когда Раджат Сури из E la Carte решил
написать софт для ресторанов, он нашел работу в качестве официанта,
чтобы узнать, как устроены рестораны. Это может показаться крайностью,
но стратапы сами по себе крайность. Мы любим, когда фаундеры делают
подобные вещи.

На самом деле, одна из стратегий, которые я рекомендую людям, ищущим
новую идею, — не просто избавиться от фильтров «непривлекательности» и
«геморроя», а именно искать идеи, которые непривлекательны и
геморройны. Не пытайтесь создать Twitter. Эти идеи на столько редкие,
что их невозможно найти, ища. Сделайте что-нибудь «непривлекательное»,
за что люди будут вам платить.

Хороший прием обойти фильтр «геморроя» и в какой-то степени
«непривлекательности» — спросить, чтобы вы хотели, чтобы кто-то другой
создал, а вы бы могли использовать. А сколько вы готовы были бы за это
заплатить?

Поскольку стартапы часто утилизируют обанкротившиеся компании и
отрасли, хороший прием — присмотреться к тем из них, кто умирает или
заслуживает этого, и попытаться представить, какой тип компании мог бы
извлечь выгоду из их смерти. Например, журнализм сейчас находится в
свободном падении. Но возможно можно получить прибыль от чего-то,
похожего на журнализм. Какой тип компании может заставить людей в
будущем сказать «это пришло на смену журнализму» в каком-то аспекте?

Но представьте, что спрашиваете это в будущем, а не сейчас. Когда одна
компания или отрасль сменяет другую, это происходит обычно с
неожиданной стороны. Поэтому не ищите заменитель чему-то; ищите
что-то, о чем люди потом скажут, что оно оказалось заменителем X. И
включите воображение относительно тех измерений, в которых будут
происходить эти преобразования. Традиционная журналистика, например,
это способ для читателей получить информацию и убить время, для
журналистов — заработать деньги и привлечь внимание, и кроме всего
механизм для размещения нескольких типов рекламы. И она могла бы
видоизменяться в любом из этих измерений (на самом деле, процесс уже
начался и во всех направлениях).

Когда стартапы съедают гигантов, они обычно начинают с обслуживания
небольшого, но важного рынка, который большие игроки игнорируют.
Особенно хорошо, если присутствует некоторое пренебрежение в позиции
большого игрока, поскольку это часто сбивает их с толку. Например,
после того, как Стив Возняк создал компьютер, который стал Apple I, он
чувствовал себя обязанным передать своему тогдашнему работодателю
Hewlett-Packard возможность производить его. К счастью для него HP
отказался. И одна из причин — компьютер использовал ТВ в качестве
монитора, что казалось недопустимым снижением класса для компании,
производящей высококачественное железо, как HP в те времена.

Но есть ли группы нечесаных, но утонченных пользователей, какими были
ранние любители микрокомпьютеров, которые сейчас игнорируются большими
игроками? Стартап с прицелом на большие вещи зачастую может легко
захватить маленький рынок, потратив усилия, которые оправдаются за
счет одного этого рынка.

Поскольку наиболее успешные стартапы обычно ловят волну, которая
больше их самих, может быть хорошим трюком поиск таких волн и вопрос
как извлечь из них пользу. Ценность секвенирования генома и 3D печати
медленно снижается по аналогии с законом Мура. Какие вещи мы сможем
делать в новом мире, в котором мы будем жить через 5 лет? Что мы
исключаем подсознательно как невозможное, но что будет возможным
скоро?

Органика

Но, если говорить о прямолинейном поиске этих волн, хотелось бы
уточнить, что такой рецепт — это план «Б» в поиске стартап-идеи. Поиск
волн — это по сути имитация органического подхода. Если вы на острие
быстроменяющейся сферы, то вам не нужно искать волны; вы и есть волна.

Поиск стартап-идей — тонкое дело, и поэтому так много людей, из тех
кто пытается, терпят неудачу. Не получится просто придумать хорошую
стартап-идею. Если получилось, то у вас, скорее всего, плохая идея,
которая выглядит правдоподобной, но от того только опаснее. Лучший
способ — менее прямой: если у вас правильный бэкграунд, хорошие
стартап идеи будут казаться вам очевидными. Но даже тогда — не сразу.
Займет некоторое время, чтобы столкнуться с ситуацией, где вы
заметите, что чего-то не хватает. И часто эти пробелы не будут
казаться идеями, способными стать компаниями, они будут просто вещами,
которые будет интересно построить. По этой причине хорошо иметь время
и склонность строить вещи просто потому, что они интересны.

Живите в будущем и стройте то, что кажется интересным. Как бы оно
странно ни звучало: это и есть реальный рецепт.

\end{document}
