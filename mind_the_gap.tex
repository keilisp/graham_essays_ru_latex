\documentclass[ebook,12pt,oneside,openany]{memoir}
\usepackage[utf8x]{inputenc} \usepackage[russian]{babel}
\usepackage[papersize={90mm,120mm}, margin=2mm]{geometry}
\sloppy
\usepackage{url} \title{«Осторожно, разрыв»} \author{Пол Грэм} \date{}
\begin{document}
\maketitle

Когда человек хочет что-то сделать на совесть, то тот, кто делает это
лучше всех, добивается куда более серьезных результатов, чем все
остальные. Разрыв между Леонардо да Винчи и второсортными
современниками, вроде Боргоньоне, огромен. Тот же разрыв будет между
Рэймондом Чандлером и средним автором детективов. Шахматист высокого
уровня может провести десять тысяч игр против обычных любителей, и ни
разу не проиграть.

Точно так же как шахматы или изобразительное искусство, делать деньги
это довольно специализированное занятие. Но мы почему-то относимся к
нему по-другому. Никто не ворчит, если небольшое число людей лучше
всех играют в шахматы или пишут романы - но если какое-то небольшое
число людей зарабатывают больше денег, чем все остальные, то в газетах
сразу трагично пишут, что это очень плохо.

Но почему? Статистическое распределение ничем не отличается от любого
другого умения. Почему же у людей такая сильная реакция в том случае,
если речь идет об умении делать деньги?

Я думаю, есть три причины, из-за которых мы по особому смотрим на
умение делать деньги: ошибочная модель богатства, которую мы получаем
в детстве; несправедливость, которая до недавнего времени лежала в
основе большинства крупных состояний; и обеспокоенность тем, будто
ситуация, при которой у небольшого числа одних людей больше денег, чем
у большого числа других, является чем-то вредным для общества.
Насколько я могу судить, первое просто ошибочно, второе устарело, а
третье эмпирически ложно. А что, если, в современных демократиях,
вариация в уровне доходов это на самом деле признак здоровья?



1. Папина Модель Богатства

Когда мне было пять лет, я думал, что электричество создается
розетками. Я понятия не имел, что где-то стоят электростанции, которые
эту энергию вырабатывают. Точно так же большинство детей понятия не
имеют, что богатство это нечто, что можно создавать. Они думаю, что
это такая штука, которая происходит от родителей.

Из-за тех обстоятельств, при которых дети знакомятся с богатством, они
обычно неправильно его понимают; они путают богатство и деньги. Они
думают, что в мире ограниченное количество богатства - и что это
нечто, распределяемое властями (а значит, должно распределяться
поровну), а не что-то, что можно создавать - а значит, может
создаваться неравномерно.

На самом же деле богатство это не деньги. Деньги это лишь удобный
способ обмениваться богатствами. Богатство это то, чем обмениваются -
товары и услуги, которые мы покупаем. Когда ты приезжаешь в другую
страну, богатую или бедную, тебе не надо заглядывать людям в
банковские счета, чтобы понять, в какую именно ты попал. Ты видишь
само богатство - в зданиях и дорогах, в одежде и здоровье людей.

Откуда берется богатство? Его создают люди. Это было куда проще
понять, когда большинство людей жили на фермах, и большинство из
вещей, которые им были нужны, они делали своими руками. Поэтому ты мог
наблюдать - в доме, в стадах, в хранилищах - богатство, созданное
каждой семьей. И было очевидно, что богатства мира это не какое-то
конечное количество, которое надо как-то поделить, вроде пирога. Если
ты хотел больше богатства, ты просто его создавал.

Все остается на своих местах и сегодня, хотя лишь немногие из нас
создают окружающее нас богатство напрямую (кроме небольшого числа
домашних дел). Как правило, мы создаем богатство для других людей, и
обмениваем его на деньги, а их уже обмениваем на те формы богатства,
которые хотим.[1]

Дети не могут самостоятельно создавать богатство, поэтому все, что у
них есть, они получили от кого-то другого. А когда богатство это
что-то, что дается, то разумеется, оно должно даваться поровну.[2] В
большинстве семей так и делается. И дети за этим очень ревниво следят.
"Нечестно!" - кричат они, если один получает больше, чем другой.

Во взрослом мире уже нельзя жить за счет родителей. Если ты чего-то
хочешь, то либо сделай сам, либо сделай что-либо эквивалентной
ценности для кого-то другого, чтобы тебе в обмен дали денег,
достаточных для покупки желаемой вещи. Во взрослом мире, богатство -
за исключением некоторых узких специалистов, вроде взломщиков или
спекулянтов - это что-то, что ты создаешь сам, а не что-то,
раздаваемое Папой. А так как желания и способности создавать сильно
варьируются от одного человека к другому, то и богатство тоже не
бывает равным.

Тебе платят за то, что ты создаешь нечто, нужное другим людям; и те,
кто больше получают, обычно просто просто преуспели в том, что бы
делать это нечто. Звезды кино обычно зарабатывают намного больше, чем
второстепенные актеры. Да, второстепенные могут быть почти такими же
харизматичными, но когда люди идут в кино, и смотрят на список картин,
они хотят того самого, что есть у звезд.

Производить что-то, нужное людям - это, конечно, не единственный
способ делать деньги. Еще можно грабить банки, или вымогать взятки,
или создать монополию. Эти стратегии, действительно, объясняют
какую-то часть вариации богатств, и даже некоторые из самых больших
состояний, но они не являются корнем разницы в доходе. Корень этой
разницы, согласно бритве Оккама, тот же самый, что и разница в любых
других человеческих умениях.

В США, CEO большой публично торгуемой компании получает примерно в 100
раз больше, чем средний человек.[3] Баскетболисты - примерно в 128 раз
больше. Бейсболисты - в 72. Газеты пишут об этой статистике с ужасом.
Но у меня не вызывает никакого удивления, что один человек может в
чем-то быть в 100 раз продуктивнее другого. В Древнем Риме цена двух
рабов могла отличаться в 50 раз, в зависимости от их навыков.[4] И это
даже не учитывая мотивации, или дополнительных бонусов к
продуктивности, которые позволяют получить современные технологии.

Газетчики, пишущие про атлетов или CEO, напоминают мне
ранне-христианских авторов, споривших, исходя из первых принципов, о
том, является ли земля круглой - когда можно было бы просто выйти на
улицу и проверить.[5] Кто сколько "стоит" - это не теоретический
вопрос, и не вопрос политики. Это вопрос, на который рынок уже дал
готовый ответ.

"Неужели он один стоит сотни таких, как мы?" - заламывает руки
газетчик. Ну, это зависит от того, что понимать под словом "стоит".
Если в том смысле, готов ли рынок платить за их умения такую цену, то
ответ, очевидно, "да".

Доходы некоторых CEO действительно являются следствием каких-то
нечестных манипуляций. Но разве нет большинства других, чьи доходы
действительно отображают богатство, которое было ими создано? Стив
Джобс спас компанию, находившуюся на грани гибели. И не просто так,
как это делают кризисные менеджеры - скажем, урезая затраты; он решил,
каким будет следующий продукт Apple. Мало кто мог это сделать. А если
оставить CEO в стороне, то вряд ли кто будет спорить с тем, что
зарплаты профессиональных баскетболистов действительно отражают
картину спроса и предложения.

Может показаться неубедительным, будто один человек действительно
может создать настолько больше богатства, чем другой. Чтобы найти ключ
к этой загадке, надо вернуться к вопросу, "неужели он один стоит сотни
таких, как мы?!". Ну, скажем, поменяла бы баскетбольная команда одного
из своих игроков на 100 случайных людей? Или - как бы выглядел
следующий продукт Apple, если бы Стива Джобса заменили на коммитет из
100 случайных людей?[6] Такие вещи не масштабируются линейно. Может,
CEO или профессиональный атлет обладает всего лишь в десять раз
большим (что бы это ни значило..) умением и напором, чем средний
человек; но этого более чем достаточно, если оно сконцентрировано в
одном человеке.

Когда мы говорим, что за один вид работы "переплачивают", а за другой
"недоплачивают", что мы на самом деле говорим? В свободном рынке, цена
определяется тем, чего желает покупатель. Люди любят бейсбол больше,
чем поэзию, и поэтому бейсболисты зарабатывают больше поэтов. Поэтому,
говорить, будто какая-то работа недостаточно оплачивается - это, в
сущности, то же самое, что говорить, что люди хотят неправильно. Ну
разумеется, они хотят неправильно! Было бы странно этому удивляться.
Но еще более странно - говорить, что какая-то работа недостаточно
оплачивается.[7] Потому что в таком случае ты говоришь, что
несправедливо, когда люди неправильно хотят. Это, пожалуй, печально,
что люди предпочитают реалити-шоу и сосиски Шекспиру и вареным овощам,
но где же тут несправедливость? Это как говорить, что синее - тяжелое,
а вверх - круглое.

Появление слова "несправедливо" в данном случае - безошибочный маркер
Папиной Модели. Откуда бы еше оно могло взяться? Но если говорящий все
еще находится внутри детской модели богатства, текущего из некоего
общего источника, и нуждающегося в честной дележке - в отличие от
модели, где богатство создается, производится путем создания того, что
нужно другим людям - то он будет воспринимать мир именно в таком
свете, когда увидит, что одни люди имеют намного больше, чем другие.

Когда мы говорим о "неравном распределении доходов", мы всегда должны
задать вопрос - откуда приходят доходы?[8] Кто создал богатство,
которое они представляют? Потому что если доходы варьируются в
зависимости от того, сколько богатства люди создают, распределение
может быть неравным - но вряд ли нечестным.


2. Присвоить!

Вторая причина, по которой мы настороженно относимся к разнице в
уровне богатства, заключается в том, что на протяжении большей части
человеческой истории обычным способом накопления богатств был грабеж.
В кочевых культурах - угон скота; в аграрных - захват земли во время
войны, и налогообложение в мирное время.

Во время конфликтов, победители получали владения побежденных. В
Англии в 1060-х, когда Вильгельм Завоеватель раздавал своим соратникам
земли побежденных англо-саксов, конфликт был военным; в 1530-х, когда
Генрих Восьмой раздавал монастырские земли своим сторонникам, конфликт
был политическим.[9] Но принцип и там, и там был один и тот же.
Сегодня то же самое можно наблюдать в Зимбабве.

В более организованных обществах, таких как Китай, правитель и его
чиновники чаще использовали налоги, а не конфискацию. Но и здесь тот
же самый принцип - чтобы стать богатым, не надо было создавать
богатство, а надо было служить сильному правителю, и получать
богатство от него.

Этот принцип начал слабеть в Европе с усилением среднего класса. Это
сейчас мы воспринимаем средний класс как людей, которые ни богаты, ни
бедны, но изначально это была отдельная группа людей. В феодальном
обществе было только два класса - воины-аристократы, и крестьяне,
которые работали на их землях. Средний класс был новой, третьей
группой людей, которая жила в городах и кормила себя мануфактурой и
торговлей.

Начиная с 10 и 11 века, мелкие аристократы и бывшие крепостные
собирались в городах, которые постепенно делались достаточно сильными,
чтобы позволить себе игнорировать местных феодальных лордов.[10] Как и
крепостные, средний класс в основном жил тем, что создавал богатство
(а в портовых городах, вроде Генуи и Пизы, занимались еще и
пиратством) - но, в отличие от крепостных, у них был стимул создавать
больше богатства. Любое богатство, созданное крепостным, принадлежало
его хозяину. Не было смысла делать больше, чем ты мог съесть или
спрятать. А вот независимость горожан позволяла им сохранять созданное
ими богатство.

А когда появилась возможность стать богатым, создавая богатство,
общество в целом стало очень быстро богатеть. Почти все, что мы имеем,
создано средним классом. Остальные классы практически исчезли в
индустриальных обществах, а их названия перешли на крайние слои
среднего класса (в классическом смысле этого слова, Билл Гейтс -
представитель среднего класса).

И только тогда, когда закончилась Индустриальная Революция, создание
богатства заменило коррупцию в качестве лучшего способа разбогатеть.
Во всяком случае, в Англии коррупция только тогда вышла из моды
(собственно, только тогда получила название "коррупции"), когда
появились другие, более быстрые способы разбогатеть.

Англия 17 века была тем, что сегодня мы называем страной третьего
мира, в том смысле, что правительственная должность была стандартным и
понятным способом разбогатеть. Большие состояния той эпохи все еще
получались из того, что сегодня мы зовем коррупцией, а не из
коммерции.[11] К 19 веку это поменялось. Все еще существовали взятки -
они, наверное, никогда не исчезнут - но политика стала скорее уделом
людей, которых больше интересовали соображения тщеславия, чем
жадности. Развитие технологии сделало возможность создавать богатство
быстрее, чем его можно было разворовывать. Придворный перестал быть
архетипом богача - на его место пришел промышленник.

С возвышением среднего класса, богатство перестало быть игрой с
нулевой суммой. Джобсу и Возняку уже не надо было делать нас бедными,
чтобы самим сделаться богатыми. Ровно наоборот - они создали вещи,
которые сделали наши жизни относительно богаче. Они были вынуждены это
сделать, ведь иначе бы мы их не покупали.

Но так как на протяжении большей части мировой истории магистралью на
пути к богатству был разбой, мы по инерции продолжаем относиться к
богатым с подозрением. Идеалистические студенты, читая знаменитых
авторов прошлого, находят "подтверждение" своим инфантильным
экономическим моделям. Ошибочное встречается с устаревшим.

"За каждым большим состоянием стоит большое преступление", писал
Бальзак. С той лишь разницей, что он этого не писал. Что он
действительно писал, так это то, что за большим состоянием, у которого
нет очевидной причины, скорее всего стоит преступление, которое так
мастерски исполнено, что о нем уже забыли. Если мы говорим о Европе
1000-х, или о странах третьего мира сегодня, то это неправильное
цитирование будет справедливым. Но Бальзак жил во Франции 19 века, где
Индустриальная Революция уже шла полным ходом. Он знал, что можно
сколотить состояние и без грабежа - он и сам в этом преуспел, будучи
популярным писателем.[12]

Только немногие страны - и, конечно, это совсем не совпадение, что
страны самые богатые - дошли до этой стадии. В большинстве же стран
коррупция все еще была столбовой дорогой к богатству. Поэтому когда мы
наблюдаем за увеличивающимся неравенством в доходах в богатой стране,
наша тенденция - волноваться, что мы движемся назад, в сторону новой
Венесуэлы. А по-моему, происходит ровно противоположное. Мы видим
страну, которая на шаг впереди Венесуэлы.


3. Рычаг Технологии

Увеличит ли технология разрыв между богатыми и бедными? Совершенно
точно, что она увеличит разрыв между продуктивными и непродуктивными.
В этом, собственно, весь смысл технологий. С трактором энергичный
фермер может вспахать за день в 6 раз больше земли, чем с целой
командой лошадей. Но - только в том случае, если он научится новому
способу пахать.

Только за время своей жизни я успел уже увидеть, как растет рычаг
технологии. В средней школе я зарабатывал деньги тем, что стриг газоны
и черпал мороженное в Баскин-Робинс. Это были немногие виды работы,
доступные в то время. Сейчас же школьники могут писать софт, или
разрабатывать веб-сайты. Но заниматься этим будут только некоторые -
остальные продолжат черпать мороженное.

Я отчетливо помню, как в 1985 г. развитие технологии позволило мне
купить собственный компьютер. Через несколько месяцев я уже
пользовался им, чтобы зарабатывать в качестве программиста-фрилансера.
За пару лет до того я не мог бы этого сделать. За пару лет до этого
просто не было такого явления - "программист-фрилансер". Но Apple
создали новое богатство, на этот раз в форме мощных и недорогих
компьютеров - и программисты немедленно занялись созданием новых форм
богатства.

Как видно из этого примера, с развитием технологии наша продуктивность
растет полиномиально, а не линейно. Поэтому в будущем стоит ожидать
еще больших вариаций в продуктивности. Увеличит ли это разрыв между
богатыми и бедными? А вот это зависит от того, что понимать под
"разрывом".

Технология должна увеличить отрыв в доходах - но прочие виды отрывов
она, скорее, уменьшает. Сто лет назад, богатые вели совсем другую
жизнь, нежели их современники из числа прочих людей. Они жили в домах,
полных слуг, носили сложные и неудобные костюмы, и путешествовали в
каретах, движимых упряжками лошадей; их лошади сами нуждались в
отдельных домах и слугах. Сегодня же, благодаря технологии, богатые
живут почти так же, как и средний человек.

Автомобили - хороший пример того, почему. Да, можно купить дорогой
автомобиль ручной сборки, за несколько сотен тысяч долларов. Но
особого смысла в этом не будет. И компании делают деньги на создании
большого числа обычных машин, а не маленького числа дорогих. Из-за
этого компания, производящая массовые автомобили, может позволить себе
потратить куда больше на их разработку и дизайн. Если же ты купишь
автомобиль ручной работы, он будет постоянно ломаться. Так что
единственная причина его купить - продемонстрировать, что тебе это по
карману.

Или возьмем часы. Пятьдесят лет назад, потратив много денег на часы,
ты получал часы, которые лучше работали. Во времена часов с
механическим ходом, дорогие часы показывали более точное время. Но это
больше не так. С изобретением кварцевого резонатора, обычный Таймекс
будет более точным, чем Patek Philippe стоимостью в сто тысяч
долларов.[13] Как и с дорогими машинами - если хочешь потратить много
денег на часы, придется привыкнуть к неудобствам - не только менее
точное время, их еще и требуется постоянно заводить.

Единственное, что технология не может заменить - это брэнд. Что,
кстати, и объясняет причину того, почему мы так много о них слышим.
Брэнд - это то, что осталось после исчезновения дистанции между
богатыми и бедными. Но лэйбл на твоей вещи это значительно менее
важное соображение по сравнению с тем, есть она у тебя вообще, или
нет. В 1900, если у тебя была карета, никто не спрашивал, какой она
"марки". Если у тебя была карета, ты был богатый. А если ты не был
богатым, то ты садился в омнибус, или шел пешком. А теперь даже самые
бедные американцы водят машины - и только глаз, натренированный
рекламой, может отличить особенно дорогие.[14]

И это происходит в каждой индустрии. Если какой-то товар пользуется
достаточным спросом, технология сделает его достаточно дешевым, чтобы
можно было продавать его в больших объемах - и произведенный массово,
он будет если не лучше, то уж точно более удобным.[15] А богатые ничто
не ценят больше, чем удобство. Все мои знакомые богатые люди ездят на
тех же машинах, одеваются в ту же одежду, пользуются той же мебелью, и
едят ту же пищу, что и остальные мои друзья. Их дома могут быть в
других районах, или в тех же районах, но большего размера, но жизнь
внутри этих домов та же. Дома эти сделаны по одним и тем же
технологиям; внутри находятся более-менее одинаковые предметы. Иметь
что-то дорогое и штучной работы просто неудобно.

Богатые и время свое проводят примерно так же, как и все остальные.
Берти Вустер канул в Лету. Сегодня почти все люди, которые могли бы
позволить себе вообще не работать, все равно работают. Не из-за
социального давления; просто безделие одиноко и депрессивно.

Да и социальной разницы, какая была сто лет назад, тоже нет. Романы и
руководства по этикету того периода сегодня читаются как заметки о
традициях диковинных племен. "Что касается продолжения старых
дружеских отношений," - намекает миссис Битон в Книге Ведения
Домашнего Хозяйства (1880), "для дамы иногда бывает необходимо, по
принятии положения хозяйки дома, прекратить некоторые из тех, что
начались в предыдущий период ее жизни." От женщины, вышедшей за
богатого мужчину, ожидалось прекращение дружбы с теми подругами, кто
не вышел. Сегодня такое поведение показалось бы нам варварством; да и
жизнь такая была бы весьма скучной. Люди все еще организуются в
группы, но куда чаще по уровню образования, чем по уровню
достатка.[16]

Таким образом, технология не увеличивает материальный и социальный
разрывы - напротив, она их уменьшает. Если бы Ленин пришел сегодня в
офис компании Yahoо, Intel или Cisco, он бы решил, что коммунизм
победил. Все одинаково одеваются, сидят в одинаковых офисах или
кубиклах, пользуются одинаковой мебелью, и зовут друг друга по имени,
без почтительных обращений. Все, как он и предсказывал! - пока он не
заглянет в их банковские счета. Упс...

Так есть ли какая-то проблема в том, что технология увеличивает разрыв
в доходах? Пока что не похоже. Разрыв в доходах увеличивается - но
большинство других разрывов уменьшаются.


4. Альтернатива аксиоме

Мы часто слышим, как политические решения критикуются на том
основании, что они приведут к увеличению разрыва в доходах между
богатыми и бедными - как будто бы мы точно знаем, что это что-то
плохое. Конечно, возможно, что это может быть плохо - но это
совершенно неочевидно, и нуждается в доказательствах.

А может быть, это и вовсе ложное утверждение - скажем, в
индустриальных демократиях. В обществах, где люди делились на феодалов
и крепостных, разница в доходах действительно была симптомом болезни.
Но насилие - не единственная возможная причина вариации в уровне
доходов. Пилот 747 Боинга получает где-то в 40 раз больше, чем
кассирша в супермаркете, не потому, что он феодал, который держит ее в
рабстве. Просто его навыки значительно выше ценятся.

Поэтому я предлагаю такую альтернативную идею - в современном
обществе, рост разницы в уровне доходов это признак здоровья.
Технология нелинейно увеличивает разницу в продуктивности. Если мы не
видим такой же разницы в росте доходов, то тут могут быть только 3
возможных объяснения: 1) - технические инновации остановились 2) -
люди, которые могли бы создать большую часть богатства, этого не
делают 3) - ... или им за это не платят

Я думаю, можно сразу сделать вывод, что и 1), и 2) было бы довольно
плохо. Если Вы не согласны, то попробуйте прожить годик только на тех
ресурсах, которые были доступны франкскому аристократу в 800 году, и
потом расскажите нам, как оно (это я добрый, и не посылаю Вас в
каменный век).

Поэтому в ситуации с обществом, в котором растет общий уровень
благосостояния, а разрыв в уровне доходов не увеличивается, остается
только третий вариант объяснения - что люди создают большие богатства,
но ничего за это не получают. То есть, что Джобс и Возняк, к примеру,
радостно будут работать по 20 часов в сутки, создавая персональный
компьютер Apple для общества, которое, после налогов, милостиво
позволит им оставить себе столько же, сколько бы у них осталось после
расслабленной работы с 9 до 5 в большой компании ("от каждого по
способностям, каждому по потребностям" - прим. пер.)

Будут ли люди создавать богатство, если им за это ничего не платить?
Да - но только до тех пор, пока это весело и интересно. Люди буду
писать операционные системы забесплатно. Но они не будут их
инсталлировать, или заниматься поддержкой, или учить людей ими
пользоваться. А ведь как минимум 90\% работы даже в самых крутых
технологических компаниях - именно из этого, не самого прикольного,
разряда.

И любые неприкольные формы создания богатства драматически замедляются
в обществе, в котором частные накопления отбираются. Это можно
проверить эмпирически. Скажем, ты слышишь странный шум, который, вроде
бы, исходит от работающего рядом вентилятора. Ты выключаешь
вентилятор, и шум прекращается. Ты включаешь его обратно - и шум
возобновляется. Выкл. - тишина. Вкл. - шум. Не имеюя дополнительной
информации, будет резонно предположить, что шум действительно исходит
от вентилятора.

В разных местах, в разные исторические периоды, "вентилятор"
возможности владеть результатами своего труда то выключался, то
включался. Северная Италия, 800-е - выкл. (феодалы все забирали).
Северная Италия, 1100 - вкл. Центральная Франция, в 1100 - все еще
феодализм - выкл. Англия в 1800-х - вкл. Там же, в 1974 - выкл. (98\%
налога на доход с инвестиций). США в 1974 - вкл. У нас даже было
двойное исследование: Западная Германия - вкл; Восточная - выкл. Во
всех случаях, создание богатства прекращалось и возобновлялось, как
шум вентилятора, как только выключалась или включалась возможность им
владеть.

Здесь присутствует и некоторая инерция. Нужно целое поколение для
того, чтобы превратить людей в жителей ГДР (Англии повезло). И если бы
мы просто изучали наш вентилятор, а не такую политизированную тему,
как богатство, никто вообще бы не сомневался, что вентилятор является
источником звука.

Если вы задавливаете вариацию в доходах - грабежом ли, как феодальные
бароны, или налогами, как современные правительства, результат всегда
один и тот же. Общество в целом беднеет.

Если бы у меня бы выбор, жить в обществе, где я был бы лучше
обеспечен, чем сегодня, но при этом был бы среди беднейших его членов
- или быть одним из самых богатых, но при уровне жизни хуже, чем
сейчас, то я бы не раздумывая выбрал первый вариант. Если бы у меня
были дети, то делать иной выбор было бы еще и аморально. Надо бороться
с бедностью по абсолютной шкале, а не относительной. И если факты
свидетельствуют о том, что выбирать приходится между одной или другой,
значит, надо выбирать относительную.

Наше общество нуждается в богатых людях не потому, что они создают
рабочие места, когда тратят свои деньги. Мы нуждаемя в том, что они
создают, чтобы стать богатыми. Это не "эффект просачивания". Я не
говорю, что пусть Генри Форд будет богатый, и тогда он наймет тебя
официантом на следующую свою вечеринку. Я говорю - пусть он придумает
и сделает тебе трактор, чтобы заменить твою лошадь.


Примечания

[1] Отчасти причина того, что эта тема такая спорная - в том, что
самые громкие мнения относительно богатства обычно у студентов,
наследников, профессоров, политиков и журналистов - то есть, у людей с
наименьшим опытом в создании богатств (этот феноменом будет знаком
любому, кто слышал беседы на тему спорта в пивных).

Студенты, в большинстве своем, все еще сидят на шее у родителей, и
поэтому еще не задумывались всерьез о том, откуда берутся деньги.
Наследники, как правило, сидят на шее у родителей всю свою жизнь.
Профессора и политики живут внутри социалистических завихрений в
экономике; во-первых, они на шаг удалены от создания богатства, а
во-вторых, им платят зарплату, которая не привязана к тому, сколько
сил они тратят на работу. Ну а журналисты, в силу специфики своей
профессии, отделены от той части бизнеса, которая приносит доход (т.е.
от рекламы). Многие из этих людей никогда не сталкивались лицом к лицу
с тем фактом, что деньги, ими получаемые, представляют собой богатство
- богатство, которое (за исключением случая журналистов) кто-то уже
создал. Они живут в мире, в котором доходы распределяются некоей
центральной властью, в соответствии с некоторым справедливым
механизмом (или вообще случайно - как в случае с наследниками), а
вовсе не приходят от людей, которые дают их в обмен на желаемый товар.
Поэтому им, натурально, кажется "нечестным", что рыночная экономика
работает по-другому.

(Некоторые профессора создают очень значительное богатство для
общества - но деньги, которые они получают, от этого не зависят; это
скорее вроде долгосрочных инвестиций).

[2] Если почитать о происхождении Фабианского Общества, то будет такое
ощущение, что оно выросло в сознании мечтательного мальчика из рода
героев Эдит Несбит в "Wouldbegoods" (собственно, одна из основательниц
Фабианского Общества - прим. пер.)

[3] Согласно исследованию Corporate Librarу, средний доход CEO из
числа компаний S\&P 500, с учетом зарплаты, бонусов и акций,
составляла \$3.65 миллионов в 2002 г. Согласно Sports Illustrated;
средняя плата игрока НБА за сезон 2002-2003 г. составляла \$4.54
миллиона, а средняя плата бейсболиста из главной лиги в начале сезона
2003 г. составила 2.56 миллиона. Согласно данным Бюро Трудовой
Статистики, средняя годовая зарплата по США в 2002 году составила
\$35.560.

[4] В ранней Империи, цена обычного взрослого раба составляла примерно
2000 сестерциев (Horace, Sat. ii.7.43). Служанка стоила 600 (Martial
vi.66); Колумелла (iii.3.8) пишет, что опытный винодел стоил 8000.
Врач П. Децимус Эрос Мерула, заплатил 50,000 сестерциев за свою
свободу (Dessau, Inscriptiones 7812). Сенека (Ep. xxvii.7) сообщает,
что некий Кальвисиус Сабинус платил по 100,000 за каждого раба,
сведущего в греческой классической литературе. Плиний (Hist. Nat.
vii.39) рассказывает, что самая большая цена, когда-либо заплаченная
за раба, составила 700,000 сестерциев - за лингвиста (и, видимо,
учителя) Дафниса - но потом и этот рекорд был побит актерами,
выкупавшими собственную свободу.

В классических Афина цена варьировалась похожим образом. Обычный
рабочий шел за 125-150 драхм. Ксенофонт (Mem. ii.5) упоминает о ценах
в диапазоне от 50 до 6,000 драхм (за управляющего серебряной шахтой).
Для дальнейших подробностей относительно экономики Древнего Рима,
рекомендую Jones, A. H. M., "Slavery in the Ancient World," Economic
History Review, 2:9 (1956), 185-199, reprinted in Finley, M. I. (ed.),
Slavery in Classical Antiquity, Heffer, 1964.

[5] Эратосфен (276-195 до н.э.) использовал длину теней в разных
городах, чтобы подсчитать длину окружности планеты. Он ошибся всего на
2\%.

[6] Ответы: "нет" и "Windows".

[7] Главное расхождение между Папиной Моделью и реальностью -
стоимость тяжелой и упорной работы. В Папиной Модели, честная и
упорная работа сама по себе достойна вознаграждения. В реальности,
богатство измеряется тем, что ты произвел, а не тем, как это было
тяжело. Если я покрасил дом, хозяин не обязан доплачивать мне за то,
что я пользовался исключительно зубной щеткой.

Многим из тех, у кого до сих пор Папина Модель в голове, покажется,
что тут все равно нечестно, если кто-то много и тяжело трудится, а
получает мало. Давайте упростим - у нас рабочий на необитаемом
острове, охотится и собирает фрукты. Если он плохо делает эту работу,
хоть и выкладывается по полной, то еды он все равно много не добудет.
Это нечестно? И кто же нечестен по отношению к нему?


[8] Отчасти, трудности с тем, чтобы избавиться от Папиной Модели могут
проистекать из значения слова "распределение". Когда экономисты
говорят о "распределении доходов", они-то имеют в виду статистическое
распределение. Но если часто использовать эту фразу, трудно не начать
ассоциировать ее с другим смыслом слова "распределение" (скажем,
"распределение милостыни"), и подсознательно начать думать о богатстве
как о чем-то, что течет из крана. Слово "регрессивный", когда его
применяют к подоходному налогу, так же работает - как что-то
регрессивное может быть хорошим?!


[9] "From the beginning of the reign Thomas Lord Roos was an assiduous
courtier of the young Henry VIII and was soon to reap the rewards. In
1525 he was made a Knight of the Garter and given the Earldom of
Rutland. In the thirties his support of the breach with Rome, his zeal
in crushing the Pilgrimage of Grace, and his readiness to vote the
death-penalty in the succession of spectacular treason trials that
punctuated Henry's erratic matrimonial progress made him an obvious
candidate for grants of monastic property."

Stone, Lawrence, Family and Fortune: Studies in Aristocratic Finance
in the Sixteenth and Seventeenth Centuries, Oxford University Press,
1973, p. 166.

[10] There is archaeological evidence for large settlements earlier,
but it's hard to say what was happening in them.

Hodges, Richard and David Whitehouse, Mohammed, Charlemagne and the
Origins of Europe, Cornell University Press, 1983.

[11] Уильям Сесил и его сын Роберт каждый по очереди были самыми
могущественными министрами Короны, и оба использовали свою позицию для
собирания одного из самых больших состояний своего времени. Особенно
Роберт, который довел свое взяточничество до государственной измены.
"Как Государственный Секретарь, и главный советник короля Джеймса по
внешней политике, он был особым получателем подарков, получая большие
взятки от датчан, чтобы не заключать мира с Испанией, и в то же время
большие взятки от Испании, чтобы мир заключить. (Stone, op. cit., p.
17.)

[12] Хотя Бальзак и получал большие деньги за свое писательство, он
был ужасно непредусмотрителен, и всю жизнь был в долгах.


[13] Таймекс ошибается примерно на пол-секунды в день. Самые точные
механические часы, Patek Philippe 10 Day Tourbillon, ошибаются в от
-1.5 до +2 секунд. Стоят они около \$220,000.

[14] Если спросить среднего жителя Англии начала 20 века, какая машина
дороже - хорошо сохранившийся лимузин 1989 г. Lincoln Town Car
(\$5,000), или 2004 г. Mercedes S600 седан (\$122,000), он, скорее
всего, ошибется.

[15] Чтобы сказать что-то осмысленное про тренды доходов, надо
говорить о реальном доходе, т.е. о доходе, который измеряется по тому,
что за него можно купить. Но обычный способ подсчета реального дохода
игнорирует многое из того, что касается роста богатства со временем,
потому что опирается на индекс потребительских цен, создаваемый
стыковкой друг с другом ряда цифр, каждая из которых точна только
локально, и не включает цену новых изобретений, пока они не займут
прочное место на рынке, и их цена не стабилизируется.

Поэтому, хотя мы можем думать, что намного лучше жить в мире с
антибиотиками, пассажирскими авиа-перелетами и электрическими сетями,
чем в мире без всего этого, статистический подсчет реального дохода
классическим способом покажет нам, что из-за всех этих вещей мы стали
лишь чуточку богаче...

Или другой подход - скажем, если бы ты отправился назад во времени в
год Х, сколько бы ты потратил на торговлю, чтобы сколотить состояние?
Например, если бы ты отправился назад в 1970, это было бы меньше чем
500, потому что процессор, который можно купить за 500, в 1970 стоил
бы как минимум 150 миллионов. Функция приближается к ассимптоте
довольно быстро, потому что за промежуток времени больше чем 100 лет,
ты мог бы получить любые богатства в обмен на совершенную ерунду. В
1800, пустая пластиковая бутылка с закручивающейся крышечкой была бы
чудом техники.

[16] Некоторые скажут, что это это то же самое, потому что у богатых
большие возможности для образования. В этом есть резон. В наше время
можно, до какой-то степени, купить своим детям дорогу в лучшие
колледжи, посылая их в частные школы, у которых есть свои пути
организовать поступление в колледжи. Согласно докладу Национального
Центра по Статистике Образования, примерно 1.7\% американских детей
ходят в частные нерелигиозные школы. В Принстоне, 36\% выпускников в
2007 были из таких школ (интересно, что в Гарварде процент значительно
ниже, около 28\%). Очевидно, что это довольно большая лазейка. Но хотя
бы она уменьшается, а не увеличивается. Может быть, разработчикам
процесса поступления в ВУЗы стоило бы поучиться у специалистов
компьютерной безопасности, и вместо того, чтобы считать, будто их
системы не ломаются, измерять то, насколько они ломаются.


\end{document}
