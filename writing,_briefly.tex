\documentclass[ebook,12pt,oneside,openany]{memoir}
\usepackage[utf8x]{inputenc} \usepackage[russian]{babel}
\usepackage[papersize={90mm,120mm}, margin=2mm]{geometry}
\sloppy
\usepackage{url} \title{Пишите кратко} \author{Пол Грэм} \date{}
\begin{document}
\maketitle

Многие просят моего совета о том, как писать. Насколько важно уметь
гладко излагать мысли и как некто может стать лучшим литератором? В
процессе ответа на этот вопрос я совершенно случайно написал крохотное
эссе на эту тему.

Обычно, я трачу на написание эссе недели. А вот это заняло 67 минут --
23 на написание и 44 на переписывание. И в онлайн оно попало только в
качестве эксперимента, поскольку оно весьма емкое, как минимум.

Я считаю, что хорошо писать гораздо важнее, чем предполагают многие
люди. Любая литературная форма не просто передает идеи, само ее
создание порождает их. И если вы плохо выражаете свои мысли в письме и
вам не нравится этот процесс, то вы пропустите многие идеи, которые
пришли бы к вам во время литераторства.

А вот краткий рецепт "как быть хорошим писателем:" быстро напишите
самый плохой вариант, переписывайте его снова и снова, отбрасывайте
все ненужное, пишите в разговорной манере, надо развить нюх на плохой
стиль для обнаружения его в своих работах, подражайте нравящимся вам
авторам, если трудно начать, то расскажите кому-либо о будущей теме и
запишите свои слова, ожидайте, что 80\% идей придут к вам во время
написания и что половина исходных окажется неверна, уверенно режьте
текст, дайте вашим друзьям ваше творчество и пусть они скажут вам, что
их смущает или тормозит, не стоит (всегда) делать подробные наброски,
оставить идеи в тепле на несколько дней перед началом творчества,
носите с собой записную книжку или листочки бумаги, начинайте писать в
момент прихода мысли о первой фразе, а если поджимает время, то
начните с самой важной мысли, пишите о том, что вам нравится, не стоит
пытаться звучать выспренно, не стоит стесняться менять на лету темы,
используйте сноски для отклонений от темы, используйте анафору для
соединения предложений, читайте свои эссе вслух для выявления плохо
читаемых фраз и скучных частей (те параграфы, которые страшно читать),
расскажите читателю что-либо новое и полезное, пишите долго,
перечитайте уже имеющееся перед началом переписывания, завершайте
трудный этап работы для того, чтобы на следующее утро начать с чего-то
легкого, учитывайте темы, которые вы собираетесь осветить в конце, не
считайте своим долгом разобрать хоть какую-либо из них, пишите для
читателя, который не будет читать столь же внимательно, как вы
(наподобие попсы, которая нормально звучит на дрянной автомагнитоле),
исправляйте свои ошибки сразу, пусть друзья укажут вам на ту фразу, о
которой вы пожалеете больше всего, вернитесь и смягчите жесткие
замечания, выкладывайте творения в Интернет, поскольку аудитория
заставит вас писать больше и создавать больше идей, распечатайте
наброски, а не читайте их с экрана, используйте простые,
общеупотребительные слова, научитесь отличать неожиданность и
отклонения от темы, научитесь распознавать близость финала, и
испоьзуйте ее при первой возможности.

\end{document}
