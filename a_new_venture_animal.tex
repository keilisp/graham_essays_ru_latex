\documentclass[ebook,12pt,oneside,openany]{memoir}
\usepackage[utf8x]{inputenc} \usepackage[russian]{babel}
\usepackage[papersize={90mm,120mm}, margin=2mm]{geometry}
\sloppy
\usepackage{url} \title{Новый зверь среди венчурных инвесторов}
\author{Пол Грэм} \date{}
\begin{document}
\maketitle

(Это эссе выросло из того, что я писал для себя, пытаясь понять что мы
делаем. Несмотря на то что модели Y Combinator сейчас 3 года, мы
продожаем пытаться понять её значение.)

Я был раздражен, прочитав описание Y Combinator, которое говорило: "Y
Combinator обеспечивает начальное финансирование стартапов". Что
особенно раздражало в этом описании, то, что написал его я. Это не
передавало того, что мы реально делаем. И причина этой неточности, как
ни парадоксально, финансирование стартапов на самых ранних стадиях -
не главное в финансировании.

Говорить, что YC дает стартовый капитал для стартапов - значит давать
описание в терминах старых моделей. Это как называть автомобиль
самодвижущимся экипажем.

Когда масштаб животного увеличивается, не все растет пропорционально.
Например, объем расширяется согласно функции возведения в куб линейных
измерений (ВхШхД), однако поверхность изменяется как квадратичная
функция. Таким образом, когда животное становится больше, у него
возникают проблемы с самоохлаждением. Вот почему у мышей и кроликов
есть мех, а у слонов и гиппопотамов -нет. Вы не можете получить мышь
из слона, просто уменьшив его масштаб

YC представляют собой новый вид маленького животного. Настолько
маленького, что все обычные правила -другие

До нас, большинство компаний, занятых финансированием стартапов,
представляли собой венчурные фонды. Венчурные инвесторы, как правило,
финансируют стартапы на более поздних стадиях, чем мы. И они
предоставляют столько финансирования, что рассматривать их только как
источник денег не так уж и неправильно, даже несмотря на многие другие
ценные вещи, которые они могут сделать для вас. Хорошие венчурные
инвесторы - это "умные деньги", но все же таки деньги.

Все хорошие инвесторы дают комбинацию денег и помощи. Но эти вещи
масштабируются по-разному, так же, как объем и площадь. Инвесторы на
поздней стадии привносят огромные количества денег и оказывают
сравнительно небольшую помощь: когда компания, готовящаяся к выходу на
биржу, получает 50 млн. долларов, сделка идет почти полностью вокруг
денег. Если перемещаться по венчурному процессу в более ранний период,
коэффициент помощь / деньги увеличивается, потому что компании на
ранних стадиях развития имеют другие потребности. Компаниям на ранних
стадиях развития нужно меньше денег, потому что они меньше, и не
требуют много денег для работы, но им необходимо больше помощи, потому
что их существование шатко. Так что когда венчурные компании делают
серию А за, скажем, 2 млн. долларов, они обычно собираются
предоставить значительное количество помощи дополнительно к этим
деньгам.

YC занимает самые ранние этапы жизни проекта. Между ним и венчурным
финансированием остается как минимум один-два шага. (Хотя некоторые
проекты переходят от YC сразу к венчурным фондам, более
распространенный путь лежит через финансирование ангельского /первого
- прим. переводчика/ раунда). И то, что происходит со стартапами в YC,
настолько же отличается от получения финансирования первого раунда,
насколько финансирование этого раунда отличается от финансирования с
конвертацией долговых обязательств в акции, называемого мезанинным
финансированием.

На нашем конце шкалы деньги являются таким фактором, которым можно
почти пренебречь. Команда стартапа состоит в основном из его
основателей. Их личные расходы являются основными затратами компании,
а так как большинство из учредителей моложе 30, их личные расходы
невысоки. Но на ранних стадиях компании очень нуждаются в помощи.
Практически все вопросы остаются пока без ответов. Некоторые компании,
которые мы финансировали, разрабатывали свои программные продукты уже
год или более, однако другие еще не определились над чем они хотят
работать, и даже кто будет входить в состав основателей.

Когда пиарщики и журналисты рассказывают истории нулевых проектов
после того, как те уже стали известными, они всегда недооценивают ту
неопределенность, которая существовала в начале. (или насколько
неоднозначными были вещи, ситуация в начале ). И делают они это не со
злым умыслом. Когда мы смотрим на такую компанию как Гугл, сложно себе
представить, что когда-то они были маленькой и беспомощной компанией.
Конечно, с одной стороны они были просто парочкой "ребят в гараже" ,
но даже при этом мы воспринимаем их как великих, и все что им надо
было делать - это ехать по прямым рельсам, проложенным их судьбой.

Далеко не так. Множество стартапов, которые начинали так же
многообещающе, закончились неудачей. Сейчас Гугл превратились в такого
монстра, что сложно теперь кому-либо их остановить. Но это произошло
на начальном этапе, было бы достаточно, если бы два ее сотрудника в
течение шести месяцев сфокусировались бы не на том, на чем надо - и
компания бы погибла.

Мы убедились на собственном опыте, насколько уязвимы стартапы на самых
ранних стадиях своего развития. Как это ни смешно, но именно поэтому
основатели намереваются разбогатеть на них. Вознаграждение всегда
пропорционально риску, а стартапы на очень ранних стадиях просто
безумно рискованы.

Что мы действительно делаем в YC так это правильно запускаем нулевые
проекты. Одна из многих метафор, что применима к YC - катапульта
(пусковое устройство) на борту авианосца. Мы поднимаем нулевые проекты
в воздух. Они лишь слегка парят над землей, но этого достаточно, чтоб
они смогли хорошо ускориться.

Когда вы запускаете самолеты, они должны быть правильно
сконструированы, иначе вы просто запускаете снаряды. Они должны быть
четко расположены на палубе, крылья направлены правильно, двигатель
(турбины) работать на полной мощности, пилот должен быть готов. Вот с
какими задачами мы обычно сталкиваемся. После того, как мы финансируем
нулевые проекты, мы работаем с ними очень тщательно в течение 3
месяцев - на самом деле настолько тщательно, что мы просим их
переехать к нам (?). Чем мы занимаемся в эти три месяца - это проверка
того, чтобы все было готово к запуску. Если существует конфликт между
соучредителями, мы помогаем с ним разобраться. Мы прорабатываем все
формальные документальные вопросы, чтобы не было неприятных
последствий далее. Если учредители не уверены на чем сконцентрировать
свое внимание сначала, мы стараемся с этим разобраться. Если
существует препятствие прямо перед ними, мы либо стараемся его убрать,
либо провести проект в обход. Задача состоит в том, что убрать любой
рассредотачивающий фактор с пути, чтобы учредители могли использовать
время для создания (или для завершения) чего-то внушительного. И,
наконец, в конце этих трех месяцев, мы нажимем на кнопку старта
пускового механизма в виде презентации (вообще-то это называется роуд
шоу), на которой текущая команда нулевого проекта представляется почти
всем инвесторам Силиконовой Долины.

Запуск компании не есть то же самое, что и запуск продукта. Хотя мы
действительно тратим много времени на то, чтобы запустить стратегии
для продуктов, некоторые вещи отбирают слишком много времени в случае
нулевых проектов, чтобы запустить их до того как они достигнут
следующующей фазы финансирования. Некоторые из самых многообещающих
нулевых проектов, которые мы финансировали, не запустили еще своих
продуктов, однако однозначно функционировали как компании.

На ранних стадиях нулевые проекты не только ставят больше вопросов, но
эти вопросы к тому же еще и сильно отличаются от обычных. На поздних
стадиях вопросы касаются сделок, найма работников или организационной
структуры. В начале они скорее касаются технологий и дизайна. Что вы
производите? Это первая задача, которую необходимо решить. Вот почему
наш девиз :"Сделай что-то, чего хотят люди". Это всегда полезно для
компаний, однако это важнее на ранних стадиях, потому как
устанавливает рамки для любого другого вопроса. Кого вы нанимаете,
сколько вы поднимете денег, как вы отпозиционируете себя на рынке -
все это зависит от того, что вы производите.

Так как самые первые задачи касаются в основном технологии и дизайна,
вы, вполне вероятно, должны быть знатоком в том, что делаете. Хотя ВФ
имеют определенный технический бекграунд (информационное поле) я не
знаю ни один, который будет писать код. Их эксперты в основном
специализируются на вопросах бизнеса (управления) - так и должно быть,
потому как именно в такой экспертной оценке вы нуждаетесь между САС и,
если вам повезет, публичным размещением на бирже, так называемым IPO.

Мы настолько далеки от ВФ, что представляем собой совершенно
обособленный "вид животных". Можем ли мы утверждать, что учредители
выигрывают от появления такого нового типа венчурных компаний (не
уверена в точности?)? Я полностью согласен, что ответ будет - да,
потому как YC является усовершенствованной версией того, что случилось
с нашим нулевым проектом, а наш случай был типичным (двойное
отрицание)))). Нашим стартовым капиталом для проекта Viaweb были 10
тыс USD, одолженных нашим другом Джулианом. Он был юристом
(законником) и занялся всеми формальностями - таким образом мы просто
писали код. Мы потратили 3 месяца на создание версии 1, которую мы
затем представили инвесторам, чтоб поднять больше денег. Очень похоже,
не так ли? Но YC намного продвинулась дальше. Джулиан много знал о
законах и бизнесе, но его советы на этом и заканчивались; он не был
парнем нулевого проекта. Так что мы совершили несколько типичных
ошибок на начальном этапе. И когда мы себя представляли инвесторам, их
было лишь двое, поскольку больше мы не знали. Если бы мы сумели себя
подбодрить и посоветовать себе что-то, а потом в определенный день
продемонстрировать наши достижения - возможно все было намного лучше.
Мы собрали бы сумму раза в 3-5 больше чем тогда.

Если мы получаем 6\% от компании, в которую мы инвестируем, то
основатели в конечном итоге должны работать только на 6.4\% лучше в их
следующем раунде финансирования. Мы определенно справляемся с этим.

Если то, что мы делали было общественно полезным, почему никто другой
не сделал этого до нас? Есть два ответа на этот вопрос. первый
заключается в том, что люди делали это уже ранее, но менее
организованно и в меньшем масштабе. До нас начальное финансирование
поступало в основном от инвесторов-ангелов. Ларри и Сергей, например,
получили свой стартовый капитал от Энди Бехтольшайма, одного из
учредителей Sun. И так как он был парнем нулевого проекта он,
возможно, дал им полезный совет. Но , если ты получаешь деньги от
ангела-инвестора, ты либо пан, либо пропал. Для большинства из них ты
побочный бизнес, так что они помогают в течение года и не тратят
слишком много времени на нулевые проекты, с которыми работают. И они
труднодоступны, потому как не хотят, что разные случайные нулевые
проекты донимали их своими бизнес-планами. Парням из Гугл повезло, так
как они знали человека, который был знаком с Бехтольшеймом. Обычно
необходимо быть лично представленным инвестору-ангелу

Другой ответ на вопрос состоит в том, что никто до нас не занимался до
сих пор точно тем же, чем мы. Было слишком накладно запускать нулевые
проекты. Заметьте, мы никогда до текущего момента не занимались
запуском никаких биотехнологических нулевых проектов. Это до сих пор
дорого обходится. Но развивающиеся технологии сделали нулевые
веб-проекты настолько дешевыми, что вы можете спокойно запустить
компанию при помощи 15 тыс USD. Если вы хотя бы знаете как работать с
пусковым устройством.

На самом деле произошло вот что - появилась новая экологическая ниша,
и YC новый вид животного, который туда заселился. Мы не являемся
заменой ВФ. Мы занимаем новую соседнюю нишу. И условия в ней
достаточно отличаются. И не только потому что проблемы с которыми мы
сталкиваемся отличаются - вся структура бизнеса другая. ВФ
ориентируются на конечную балансовую прибыль. Они все соревнуются за
фиксированную сумму отдачи от бизнес-потока и это хорошо объясняет их
поведение.

Что до нас, то наша цель - создать новый бизнес-поток, ободряя
высококвалифицированных наемных программистов запустить собственный
стартап вместо того, чтобы работать на кого-то. Мы больше пересекаемся
с кадровыми агентствами, нежели чем с ВФ

И что не удивительно - так иногда и случается. Большинство разделов, в
то время как они развиваются, становятся более специализированными -
более проработанными. А область нулевых проектов определенно бурно
развивалась в последние два десятилетия. Венчурному бизнесу в той
форме, в которой он существует сейчас, не более 40 лет от роду. Он в
боевой готовности.. (далее не совсем понятно) будет развиваться

И это нормально, что новая ниша сначала будет описана даже своими
обитателями в терминах старой ниши. И на самом деле YC скорее не
инвестирует в бизнес, а является маленьким, но эффективным пусковым
механизмом.

\end{document}
