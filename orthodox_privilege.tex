\documentclass[ebook,12pt,oneside,openany]{memoir}
\usepackage[utf8x]{inputenc} \usepackage[russian]{babel}
\usepackage[papersize={90mm,120mm}, margin=2mm]{geometry}
\sloppy
\usepackage{url} \title{Привилегия ортодоксальности} \author{Пол Грэм}
\date{}
\begin{document}
\maketitle

В последнее время много говорят о привилегиях. Хоть эта концепция и
перегружена, в ней что-то есть. В частности идея о том, что привилегии
ослепляют – имея привилегию, вы не можете увидеть то, что заметно тем,
чья жизнь значительно отличается от вашей.

Существует один из наиболее распространенных примеров подобной
слепоты, я о нем ранее не упоминал. Я назову его привилегией
ортодоксальности: Чем более традиционные и общепринятые у человека
взгляды, тем больше ему кажется, что все могут безопасно выражать свое
мнение.

Такие люди действительно могут безопасно выражать свое мнение, потому
что их мнение основываются на идеях, которые приемлемы для всех. Им
кажется, что эта безопасность распространяется на всех. Они просто не
могут представить себе правдивое утверждение, которое могло бы
доставить им неприятности.

И все же, на каждом этапе истории существовали истины, высказывание
которых могло привести к ужасным неприятностям. Может быть к нашей
истине это не относится? Это было бы удивительное совпадение.

Конечно, необычность наших времен и существование идей, опасных для
высказывания, укладываются в абсолютно стандартное предположение. Кто
бы мог подумать. Но большинство людей проигнорируют все исторические
факты и станут спорить с этим утверждением со слюной у рта.

Скрытый слоган привилегии ортодоксальности – «Почему бы тебе просто не
сказать это?” Если ты думаешь, что существует некая истина, которую
люди боятся озвучить, почему бы тебе не набраться храбрости и стать
первым? Более радикальные собеседники также представят себе, что у вас
в голове поселилась жуткая ересь, и обвинят вас в этом. Более того,
если в вашу эпоху существуют разные варианты ереси, то обвинения будут
носить неопределенный характер: вас назовут либо одним „-истом“, либо
другим.

Иметь дело с такими людьми обидно, но важно понимать, что они
искренни. Они действительно считают, что идея не может быть и
неортодоксальной, и правдивой. Так для них выглядит мир.

Как реагировать на привилегию ортодоксальности? Присвоение ей
устойчивого термина может принести пользу лишь отчасти – если вы
столкнетесь с подобной проблемой, то вспомните почему беседа с
некоторым собеседникам кажется такой странной и неразумной. Мы говорим
об исключительно живучей форме привилегий. Люди могут преодолеть
слепоту, вызванную большинством форм привилегий, получив знания о
реальном положении дел. Но они не могут преодолеть привилегию
ортодоксальности, просто узнав больше. Для этого люди должны стать
более независимыми. Если это вообще возможно, то одного разговора для
этого будет мало.

Возможно, удастся убедить некоторых людей в том, что привилегия
ортодоксальности существует, пусть они ее и не чувствуют. С темной
материей все точно так же. Возможно, существуют люди, которые считают,
что вряд ли мы впервые находимся в исторической точке, когда не
существует истин, которые трудно высказать – даже если эти люди не
смогут привести примеров.

Впрочем, даже если отойти от таких людей, вряд ли в случае с нашей
проблемой будет достаточно сказать „подумай о своих привилегиях“. Дело
в том, что обладатели этой привилегии о ней даже не догадываются. Люди
с общепринятым мнением не думают, что их мнения общеприняты. Просто им
кажется, что они правы. И они в этом уверены.

Возможно, стоит попробовать воззвать к вежливости. Если кто-то
говорит, что слышит высокочастотный шум, а вы его не слышите, то
вежливо поверить ему на слово. Не надо требовать доказательств,
которые невозможно предъявить, или просто все отрицать. Представьте
насколько это будет грубо. Подобным образом, если кто-то утверждает,
что существуют правдивые идеи, которые не могут быть высказаны, то
вежливо будет просто поверить на слово – даже если и подумать о
подобном не могли.

Как только вы поймете, что привилегия ортодоксальности существует,
многое прояснится. Например, вы поймете как так может быть, что
большое количество здравомыслящих, умных людей беспокоятся о том, что
они называют „культурой отмены“, в то время как другие (тоже
здравомыслящие и умные) люди отрицают, что эта проблема существует?
Как только вы поймете концепцию привилегии ортодоксальности, вы
увидите источник этого противоречия. Если вы верите, что не существует
истин, которые не могут быть высказаны, то любой, кто попадет в беду
за свои слова, должно быть этого заслуживал.

\end{document}
