\documentclass[ebook,12pt,oneside,openany]{memoir}
\usepackage[utf8x]{inputenc} \usepackage[russian]{babel}
\usepackage[papersize={90mm,120mm}, margin=2mm]{geometry}
\sloppy
\usepackage{url} \title{Железный ренессанс} \author{Пол Грэм} \date{}
\begin{document}
\maketitle

Одно из преимуществ Y Combinator состоит в том, что мы видим тенденции
гораздо раньше большинства людей. В последней волне профинансированных
нами стартапов видна одна такая тенденция. Семь из восьмидесяти
четырёх компаний занимаются железом. Это много. И в целом их успехи
выше среднего.

Конечно, стартапы, производящие железо, наталкиваются на сопротивление
инвесторов. У инвесторов есть глубоко сидящее предубеждение против
них. Но мнение инвесторов — лишь вторичный индикатор. Самые
талантливые основатели видят будущее гораздо лучше самых
проницательных инвесторов, ведь они сами его создают.

Эта тенденция не имеет единой движущей силы. Хардварные проекты хорошо
идут на краудфандинговых сайтах. Распространение планшетов и
смартфонов открывает возможности для создания устройств,
контролируемых ими, или даже включающих их в себя. Улучшается
технология производства электродвигателей. Беспроводные сети есть
почти везде. Проще стало организовать производство. Arduino,
3D-печать, лазерные резаки и доступные фрезеровальные станки с ЧПУ
облегчают создание прототипов. Распространение становится всё меньшей
проблемой, так как люди всё больше покупают в интернете.

Я могу ответить на вопрос, почему вдруг производить железо стало
круто. На самом деле это было круто всегда. Вещи, которые можно
пощупать — это прекрасно. Просто до недавнего времени это был далеко
не такой прекрасный способ построить быстрорастущий бизнес, как софт.
Но положение может измениться. Оно ведь сложилось не так давно,
примерно в 1990 году. Возможно, окажется, что преимущество софтверных
компаний было временным. Хакеры обожают делать физические устройства,
а потребители — покупать их. Поэтому если лёгкость производства и
распространения железа приближается к легкости производства и
распространения софта, «железные» стартапы растут как грибы после
дождя.

Не первый раз то, что когда-то было плохой идеей, вдруг оказывается
хорошей. И не в первый раз основатели преподают урок инвесторам.

Так что, если вы хотите создавать физические устройства, пусть вас не
останавливают опасения, что инвесторы посмотрят на вас косо. Конкретно
мы, Y Combinator, будем только рады, если вы придёте к нам с
«железным» стартапом.

Мы твёрдо уверены, что в истории найдётся место для второго Стива
Джобса. И почти наверняка знаем, что и для первого <подставьте своё
имя> тоже.

\end{document}
