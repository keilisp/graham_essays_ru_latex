\documentclass[ebook,12pt,oneside,openany]{memoir}
\usepackage[utf8x]{inputenc} \usepackage[russian]{babel}
\usepackage[papersize={90mm,120mm}, margin=2mm]{geometry}
\sloppy
\usepackage{url} \title{Банальное и прорывное} \author{Пол Грэм}
\date{}
\begin{document}
\maketitle

Наиболее ценные идеи являются и банальными и прорывными. Например, F =
ma. Но этого довольно трудно добиться. Эта территория обычно
разбирается подчистую, именно потому, что эти идеи так ценны.

Обычно, лучшее на что способны люди, это сделать одно без другого:
либо прорывное, но при этом небанальные идеи (например, сплетни), либо
всем известные, но при этом совсем не поразительные идеи (например,
банальности).

Интересное начинает происходить, когда появляются умеренные инсайты.
Этого можно добиться, добавляя по чуть чуть того, что не хватало.
Самый распространенный случай — это добавить немного банальности:
часть сплетни, которая становится больше, чем просто сплетней в том
случае, когда она рассказывает что-либо интересное о мире. Менее
распространенный подход в том, чтобы сосредоточиться на банальных
идеях и посмотреть можно ли рассказать о них что-нибудь новое. Из-за
того, что информация изначально является довольно обобщенной,
достаточно добавить лишь немного новизны, чтобы получить полезную
идею.

Небольшая доля новизны это все, что вы будете получать большую часть
времени. Это значит, что если вы пойдете по этому пути, то ваши идеи
будут казаться во многом похожими на уже существующие. Иногда вы
будете осознавать, что просто заново открыли идею, которая уже
существовала. Но не расстраивайтесь. Помните об огромном множителе,
который увеличивается каждый раз, когда вы думаете о чем-то хоть
немного новом.

Мораль: чем более обобщенными являются идеи, о которых вы говорите,
тем меньше вам стоит беспокоиться о том, что вы будете повторяться.
Если вы пишите достаточно, то повторов не избежать. Ваш мозг остается
одним и тем же из года в год, как и озарения, которые его посещают. Я
чувствую себя немного плохо, когда обнаруживаю, что сказал что-то
похожее на то, что я уже говорил раньше, словно я сплагиатил сам себя.
Но если мыслить рационально, то этого делать не стоит. Вы не скажите
что-то в точности так же, как вы сделали это в прошлый раз, и это
изменение увеличивает шанс на то, что вы получите этот крошечный, но
критически необходимый процент новизны.

И, разумеется, идеи порождают идеи. (Это звучит знакомо.) Идея с
небольшим процентом новизны может привести к идее, в которой этот
процент будет значительно выше. Но только в том случае, если вы
продолжаете этим заниматься. Так что вдвойне важно не падать духом,
когда люди будут утверждать, что не так много нового в том, что вы
обнаружили. “Не так много нового” это хорошее достижение, когда вы
говорите об в большей степени обобщенных идеях. Возможно, если вы
продолжите этим заниматься, то обнаружите больше классных идей.

Говорят, что “ничто не ново под Луной”. Это не так. Есть некоторые
домены, где нет почти ничего нового. Но есть большая разница между
“ничего нового” и “ПОЧТИ ничего нового” в масштабах этой деятельности.

\end{document}
