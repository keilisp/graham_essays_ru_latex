\documentclass[ebook,12pt,oneside,openany]{memoir}
\usepackage[utf8x]{inputenc} \usepackage[russian]{babel}
\usepackage[papersize={90mm,120mm}, margin=2mm]{geometry}
\sloppy
\usepackage{url} \title{95\% превосходных мировых программистов
  остаются не у дел, впустите их} \author{Пол Грэм} \date{}
\begin{document}
\maketitle

Американские технологические компании хотят, чтобы правительство
упростило процесс иммиграции, потому что они не могут найти достаточно
программистов в США. Противники иммиграции, говорят, что вместо того,
чтобы позволять иностранцам занимать рабочие места, следует обучать
программированию больше американцев. Кто прав?

Технологические компании правы. Противники иммиграции не понимают, что
есть большая разница в способностях между компетентными программистами
и исключительными, и вы можете обучать людей, чтобы они становились
компетентны, но не можете делать их исключительными. Исключительные
программисты имеют склонности и интерес к программированию, это не
просто результат обучения. [1]

В США живёт меньше 5\% населения мира. Выходит, если те качества,
которые делают из человека великого программиста, распределяются
равномерно, то 95\% этих великих программистов рождены за пределами
США.

Противникам иммиграции нужно придумать какое-то объяснение, почему
технологические компании прилагают такие усилия, чтобы сделать
иммиграцию проще. Поэтому они утверждают, что те делают это с целью
снизить зарплаты. Но если говорить о стартапах, вы обнаружите, что
практически каждый из них крупнее определённого размера проходил через
юридические проволочки, чтобы привезти программистов в США, где они
платили им столько же, сколько платили бы американцам. Зачем они идут
на лишние хлопоты, получая программистов за ту же цену? Единственное
объяснение состоит в том, что они говорят правду: вокруг просто
недостаточно хороших программистов. [2]

Темп глобализации современного мира в дальнейшем будет только
нарастать.


Способность адаптироваться к этим реалиям и определяет
конкурентоспособность. Аутсортинг и оффшоринг при создании ПО — как
раз следование данному тренду.

Доступ к российским талантам и научно-исследовательскому потенциалу —
это центр разработки программного обеспечения EDISON.



Я спросил главу одного стартапа, в штате которого около 70
программистов, сколько ещё он бы нанял, если бы мог заполучить всех
хороших программистов, которых хотел. Он сказал: «Мы бы взяли ещё 30
завтра утром». И это один из горячих стартапов, который всегда
выигрывает рекрутинговые гонки. То же самое по всей Силиконовой
Долине. Настолько стартапы обделены талантами.

Было бы замечательно, если бы больше американцев обучалось
программированию, но никакой объём обучения не может справиться с
таким непреодолимым разрывом как 95 к 5. Особенно с учётом того, что в
других странах программистов тоже обучают. Если исключить какой-нибудь
катаклизм, всегда будет верно то, что большинство выдающихся
программистов рождается вне США. Всегда будет верно то, что
большинство выдающихся в чём бы то ни было людей рождается вне США.
[3]

Исключительная производительность предполагает иммиграцию. Страна,
составляющая всего несколько процентов от населения мира, будет
исключительной в какой-либо области, только если в ней работает много
иммигрантов.

Но во всей этой дискуссии считают само собой разумеющимся, что если мы
будем пускать больше хороших программистов в США, они будут хотеть
переехать. Сейчас это так, и мы не осознаём, как нам с этим повезло.
Если мы хотим сохранить такой расклад, лучше всего будет использовать
это преимущество: чем больше выдающихся программистов со всего мира
находится здесь, тем больше остальные будут хотеть приехать сюда.

А если мы не будем, то США могут серьёзно облажаться. Понимаю, это
сильно сказано, но люди, которые рассуждают об этом, похоже, не
осознают, какие мощные силы здесь действуют. Технологии дают лучшим
программистам огромные рычаги влияния. По всей видимости мировой рынок
программистов становится значительно более неустойчивым. А поскольку
хорошим людям нравятся хорошие коллеги, это значит, что лучшие
программисты могут сосредоточиться всего в нескольких центрах.
Возможно, преимущественно в одном центре.

Что если большинство выдающихся программистов соберутся в одном
центре, и он будет не здесь? Такой сценарий может казаться
маловероятным сейчас, но не в том случае, если за следующие 50 лет
ситуация изменится так же, как за предыдущие 50.

Мы можем гарантировать, что США сохранит технологическую суперсилу,
всего лишь пуская несколько тысяч хороших программистов в год. Каким
колоссальным провалом было бы дать такой возможности ускользнуть. Это
легко может стать той самой ошибкой, за которую позже нынешнее
поколение американских политиков будет знаменито. И в отличие от
других потенциальных ошибок подобного масштаба, эту ничего не стоит
исправить.

Так что, пожалуйста, давайте покончим с этим.

Примечания

[1] Насколько выдающийся программист лучше обычного? Настолько, что вы
даже не сможете непосредственно измерить разницу. Отличный программист
не просто делает ту же работу быстрее. Отличный программист изобретает
такие вещи, о которых обычный даже и не подумал бы. Это не значит, что
выдающийся программист бесконечно более ценен, потому что у всего есть
конечная рыночная стоимость. Но легко представить ситуацию, когда
выдающийся программист изобретает продукт стоимостью в 100 или даже
1000 раз больше его средней зарплаты.

[2] Есть некоторые консалтинговые фирмы, предоставляющие в наём
большие группы иностранных программистов, которых они приглашают по
визам H1-B. Всеми путями с этим нужно бороться. Должно быть просто
составить законодательство, которое бы различало их, поскольку они
весьма отличаются от технологических компаний. Но со стороны
противников иммиграции несправедливо утверждать, что компаниями вроде
Google и Facebook движут те же мотивы. Приток недорогих, но
посредственных программистов — это последнее, чего они хотят; это
просто уничтожит их.

[3] Хотя в этом эссе речь идёт о программистах, группа людей, которых
нам следует привозить, шире — от дизайнеров до инженеров-электриков.
Наилучшим термином для них мог бы быть «цифровые таланты». Однако
лучше было сделать параметр слишком узким, чем запутать всех
неологизмом.

\end{document}
