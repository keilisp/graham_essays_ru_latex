\documentclass[ebook,12pt,oneside,openany]{memoir}
\usepackage[utf8x]{inputenc} \usepackage[russian]{babel}
\usepackage[papersize={90mm,120mm}, margin=2mm]{geometry}
\sloppy
\usepackage{url} \title{Что изменилось в мире стартапов} \author{Пол
  Грэм} \date{}
\begin{document}
\maketitle

Работа Y Combinator

Венчурный фонд Y Combinator на сегодняшний день профинансировал 564
стартапа, включая текущую партию из 53 стартапов. Совокупная оценочная
стоимость 287 стартапов, которые были оценены (вследствие привлечения
раунда финансирования, приобретения кем либо или закрытия стартапа
вследствие чего либо) составляет приблизительно \$11,7 миллиарда.
Предшествующие текущей партии 511 стартапов вместе составили
приблизительно \$1,7 миллиарда. [1]

Как правило, эти цифры формируются благодаря лидерам списка. На топ-10
стартапов приходится 8,6 из этих 11,7 миллиарда. Однако за ними
следует группа более молодых стартапов. Есть еще порядка 40, которые
намерены стать действительно крупными.

Ситуация немного вышла из-под контроля прошлым летом, когда в партии
было 84 компании повысили планки оценок, чтобы сократить размер этого
списка. [2] Несколько журналистов попытались интерпретировать это как
доказательство некого придуманного ими макропроцесса, однако причина
не имела ничего общего с какой-либо внешней тенденцией. Мы просто
поняли, что использовали алгоритм n², и нам необходимо было выиграть
время, чтобы это исправить. К счастью, мы придумали несколько методов
распределения Y Combinator, и проблема теперь, кажется, решена.
Благодаря новой более масштабируемой модели и небольшому количеству
компаний, работать с текущей партией теперь кажется проще простого. Я
думаю, мы сможем увеличить количество стартапов в 2 или 3 раза до
того, как нам снова придется что-то изменять в нашем алгоритме. [3]

Тенденции мира стартапов

Одним из следствий финансирования такого большого количества стартапов
является видение тенденций на раннем этапе. Поскольку привлечение
средств является одним из главных аспектов, с которыми мы помогаем
стартапам, мы находимся в выгодном положении, которое позволяет нам
видеть тенденции в инвестировании.

Изменение финансирования

Я собираюсь рассказать о том, куда ведут эти тенденции. Давайте начнем
с самого основного вопроса: будет будущее лучше или хуже настоящего?
Заработают инвесторы в целом больше денег или меньше?

Я думаю, что больше. Действуют различные силы, некоторые из них
сократят прибыль, а некоторые увеличат. Я не могу с уверенностью
сказать, какие силы будут преобладать, поэтому просто опишу их, а вы
сможете решить для себя.

Изменения в финансировании стартапов по двум причинам (силам). Первое
— удешевление стоимости запуска стартапа, второе — стартапы все
плотнее входят в нашу жизнь, становятся обычным делом.

Когда я закончил колледж в 1986 году, у меня было лишь два варианта
выбора: устроиться на работу или пойти в аспирантуру. Теперь есть
третий: основать собственную компанию. Это большая перемена. В
принципе, свою компанию можно было основать и в 1986 году, но тогда
это казалось нереальным. Возможным казалось основать консалтинговую
компанию или компанию, предлагающую нишевый продукт, но не компанию,
которая могла бы стать крупной. [4]

Такая перемена, когда вместо 2 вариантов появляется 3, представляет
собой крупный социальный сдвиг, который случается один раз на
несколько поколений. Я думаю, что этот сдвиг только начался. Сложно
представить, насколько крупным он будет. Таким же крупным, как
промышленная революция? Скорее всего, нет. Но достаточно крупным,
чтобы удивить всех, что всегда делают крупные социальные сдвиги.

Изменение количества стартапов

Одно можно сказать с уверенностью: количество стартапов существенно
возрастет. Монолитным компаниям с весомой иерархией середины XX века
приходят на смену сети более мелких компаний. Это процесс сейчас
происходит не только в Кремниевой долине. Он начался несколько
десятилетий назад и имеет место даже в автомобильной отрасли. Ему
предстоит долгий путь. [5]

Увеличение контроля хозяев над компаниями

Еще одной важной причиной перемен является удешевление стоимости
запуска стартапа. В действительности две силы связаны: уменьшение
стоимости запуска является одной из причин становления стартапов
обыденными.

Тот факт, что стартапам требуется меньше денег, означает, что
основатели будут становиться все большими хозяевами положения в
отличие от инвесторов. Вам все еще нужно столько же их энергии и
воображения, но им уже не нужно столько сторонних денег. Поэтому доля
основателей в своих компаниях и контроль над ними будут увеличиваться.

Означает ли это, что инвесторы будут зарабатывать меньше денег? Не
обязательно, поскольку количество хороших стартапов будет расти. Общая
сумма востребованных и доступных инвесторам акций старатапов скорее
всего увеличится, поскольку количество востребованных стартапов
вероятно будет расти быстрее, чем уменьшаться доля, продаваемая
инвесторам.

Изменение процента успешных стартапов

В бизнесе венчурного капитала считается, что в год действительно
успешными становится примерно 15 компаний. Несмотря на то, что
множество инвесторов несознательно рассматривают это число как
космологическую постоянную, я уверен, что оно таковым не является.
Вероятно, существуют ограничения скорости, с которой могут развиваться
технологии, но сейчас не это является ограничивающим фактором. Если бы
это было так, каждый успешный стартап основывали бы в тот же месяц,
когда это становилось возможным. В данный момент ограничивающим
фактором больших успехов является количество достаточно хороших
основателей, запускающих компании, и это количество может и будет
увеличиваться. Все еще есть множество людей, из которых бы вышли
отличные основатели, но которые так и не взялись за это. Это можно
наблюдать по тому, как случайно возникли некоторые из самых успешных
стартапов. Очень много крупнейших стартапов не были полностью
реализованы, поэтому должно быть много таких же хороших еще фактически
не реализованных стартапов.

Возможно, есть в 10 или даже в 50 раз больше хороших основателей. По
мере того, как больше из них будут двигаться вперед и основывать
стартапы, эти 15 самых успешных в год могут легко превратиться в 50
или даже 100. [6]

А что же прибыли? Движемся ли мы к миру, в котором прибыли будут
повышаться и повышаться? Я думаю, что топовые фирмы на самом деле
будут зарабатывать больше денег, чем в прошлом. Высокие прибыли не
являются результатом инвестиций в компании с низкой оценочной
стоимостью. Они являются результатом инвестиций в компании, которые
действительно хорошо работают. Поэтому, если с каждым годом таких
компаний будет становиться все больше, тех, кто научатся делать лучший
выбор, ждет повышенный успех.

Это означает, что бизнес венчурного капитала станет более
разнообразным. Компании, которые смогут рассмотреть и привлечь лучшие
стартапы, будут еще более успешными, потому что увеличится количество
стартапов, которые можно рассмотреть и привлечь. В то же время плохие
компаний будут получать отходы, как и сейчас, но будут платить за них
большую цену.

Я также не думаю, что сохранение основателями контроля над своими
компаниями будет оставаться проблемой. Эмпирическое доказательство
этого уже четко обозначилось: инвесторы зарабатывают больше денег,
обслуживая основателей, чем командуя ими. Несколько унизительно, но на
самом деле хорошая новость для инвесторов, поскольку на обслуживание
основателей уходит меньше времени, чем на контроль каждого их шага.

Бизнес-ангелы

А что же бизнес-ангелы? Я думаю, здесь много возможностей. Как
инвестор-ангел я терпел неудачи. Невозможно получить доступ к лучшим
сделкам, если вы не такой успешный, как Энди Бехтолшайм, и когда вы
уже инвестируете в стартап, венчурные капиталисты, придя позже, могут
попытаться отнять у вас вашу долю. Теперь ангел может сделать что-то
вроде Demo Day или AngelList и получить доступ к тем же сделкам, что и
венчурные капиталисты. Времена, когда венчурные капиталисты могли
вытеснить ангелов из таблицы капитализации давно прошли.

Я считаю, что одна из самых больших неиспользованных возможностей в
инвестировании в стартапы на сегодняшний день – это быстрые инвестиции
ангельских размеров. Мало инвесторов понимают расходы, которые несут
стартапы в связи с привлечением инвестиций. Если компания состоит
только из основателей, на время привлечения средств вся работа
останавливается, что может легко занять 6 недель. Текущие высокие
расходы на привлечение средств означают наличие места для
малозатратных инвесторов, которые могут обойти остальных. В данном
контексте малозатратный означает быстро принимающий решение. Если бы
нашелся авторитетный инвестор, инвестирующий \$100000 на хороших
условиях и обещающий решить «да» или «нет» в течение 24 часов, он бы
получил доступ почти ко всем лучшим сделкам, поскольку каждый хороший
стартап сначала бы обратился к нему. Такой инвестор сам смог бы
выбирать, поскольку каждый плохой стартап тоже бы сначала обратился к
нему, но по крайней мере он бы видел полную картину. Если же инвестор
известен тем, что ему требуется много времени на принятие решения или
переговоры об оценке, основатели оставят его на потом. И в случае
самых перспективных стартапов, которые легко привлекают средства, до
оставленного на потом инвестора очередь может не дойти.

Будет ли количество крайне успешных стартапов расти прямо
пропорционально количеству новых стартапов? Скорее всего, нет по двум
причинам. Первая это то, что боязнь основывать стартап в былые времена
была довольно эффективным фильтром. Теперь, когда цена неудачи
снижается, следует ожидать, что будут запускаться все больше
стартапов. Это не плохо. В сфере технологий это обычная практика,
когда инновация, снижая стоимость неудачи, увеличивает количество
неудач и при этом оставляет вас впереди.

Конфликты идей

Еще одна причина, по которой количество крайне успешных стартапов не
будет расти пропорционально количеству новых стартапов, это ожидаемое
увеличение количества конфликтов идей. Несмотря на то, что
ограниченность количества хороших идей не является причиной того, что
мы имеем только 15 крайне успешных стартапов в год, количество хороших
идей все же ограничено, и чем больше будет стартапов, тем больше мы
будем наблюдать ситуацию, когда несколько компаний делают одно и то же
в одно и то же время. Будет интересно, в плохом смысле, если конфликты
идей станут обычным явлением. [7]

Пирамида стартапов

В основном из-за увеличения количества ранних неудач, бизнес стартапов
будущего изменит свою форму, он увеличит свои масштабы. То, что было
обелиском, станет пирамидой. Она будет немного шире вверху, но намного
шире внизу.

Что это означает для инвесторов? Во-первых, инвесторы получат больше
возможностей на самой ранней стадии, потому что именно здесь объем
нашего воображаемого массива растет быстрее всего. Представьте обелиск
инвесторов, который соответствует обелиску стартапов. По мере его
расширения в пирамиду для соответствия пирамиде стартапов все
содержимое будет стремиться вверх, оставляя вакуум внизу.

Эта возможность для инвесторов в основном означает увеличение
количества возможностей инвесторов, поскольку степень риска, которую
существующий инвестор или компания комфортно принимает, является для
них одной из самых сложных для изменения вещей. Различные типы
инвесторов адаптированы к различным степеням риска, при этом каждый из
них принимает конкретную прочно установившуюся степень риска, и это
касается не только процедур, которым они следуют, но также и людей,
которые работают в такой компании.

Инвестиции серии А

Я считаю, что самая большая опасность для венчурных капиталистов, а
также самая большая возможность, кроется в стадии инвестиций серии А.
Или скорее в том, что было стадией инвестиций серии А, пока серия А
фактически не превратилась в раунды серии В.

В настоящее время венчурные капиталисты часто сознательно инвестируют
слишком большие средства на стадии серии А. Делают они это потому, что
чувствуют, что должны урвать большой кусок каждой компании в серии А,
чтобы компенсировать стоимость упущенной возможности, связанной с
получением места в правлении. Это означает, что там, где присутствует
большая конкуренция за сделку, изменяющаяся цифра – это оценочная
стоимость (и поэтому инвестируемая сумма), а не продаваемая доля
компании. Это означает, особенно в случае более перспективных
стартапов, что инвесторы серии А часто заставляют компании брать
больше денег, чем те хотят получить.

Некоторые венчурные капиталисты обманывают и заявляют, что компании
действительно нужно так много. Другие более искренние и признают, что
их финансовые модели требуют владения определенным процентом каждой
компании. Но мы все знаем, что суммы, которые привлекаются в раундах
серии А, определяются, не спрашивая, что будет лучше для компаний. Они
определяются венчурными капиталистами исходя из доли компании, которой
они хотят владеть, и рынка, который обуславливает оценочную стоимость,
а значит и инвестируемую сумму.

Как многое плохое, такая ситуация сложилась непреднамеренно. Бизнес
венчурных капиталистов столкнулся с ней после постепенного устаревания
своих изначальных предположений. Традиции и финансовые модели бизнеса
венчурных капиталистов зародились, когда основатели нуждались в
инвесторах больше. В те времена было естественным для основателей
продавать большую часть своей компании в раунде серии А. Теперь
основатели предпочитают продавать меньшую долю, а венчурные
капиталисты упираются, так как не уверены, что смогут заработать,
купив менее 20\% каждой компании в серии А.

Я описываю это как опасность, потому что инвесторы серии А все больше
конфликтуют со стартапами, которые они вроде бы обслуживают, и эта
тенденция в конце концов обернется против них. Я описываю это как
возможность, поскольку сейчас накапливается много потенциальной
энергии, так как рынок ушел от традиционной бизнес-модели венчурных
капиталистов. Это означает, что первый венчурный капиталист, который
«нарушит строй» и начнет делать инвестиции серии A за ту долю, которую
основатели желают продать (и без «пула опционов» только из акций
основателей) сорвет огромную прибыль.

Что будет с бизнесом венчурных капиталистов, когда это произойдет? Не
знаю. Но бьюсь об заклад, что какая-то конкретная фирма окажется
впереди. Если одна из лидирующих компаний венчурного капитала начнет
делать инвестиции серии А исходя из суммы, которую компания хочет
привлечь, и оставит приобретаемую долю на усмотрение рынка, а не
наоборот, она мгновенно получит почти все лучшие стартапы. Именно
здесь кроются деньги.

Нельзя всегда противостоять силам рынка. Последние десять лет мы
наблюдаем, что доля компаний, продаваемых в раундах серии А неудержимо
снижается. Раньше обычными были 40\%. Сейчас венчурные капиталисты
борются за то, чтобы удержать планку на уровне 20\%. Но я каждый день
жду, что эта планка рухнет. Это произойдет. Можете смело ожидать
этого.

Кто знает, может венчурные капиталисты будут зарабатывать больше,
поступая правильно. Это был бы не первый такой случай. Венчурный
капитал – это бизнес, в котором случайный большой успех дает
стократные прибыли. Насколько уверенным можно быть в таких финансовых
моделях? Большие успехи должны стать лишь немного менее случайными,
чтобы компенсировать снижение в два раза долей, продаваемых в раундах
серии А.

Итог

Если вы хотите найти новые возможности для инвестирования, обратите
внимание на то, на что жалуются основатели стартапов. Они – ваши
клиенты, а их жалобы – это неудовлетворённый спрос. Я привел два
примера того, на что больше всего жалуются основатели: инвесторы,
которые слишком долго принимают решения, и чрезмерное увеличение
капитала в раундах серии А. Именно на это сейчас стоит обратить
внимание. И более универсальный рецепт: делайте то, чего хотят
основатели.

Примечания

[1] Я понимаю, что истинным мерилом успеха стартапа является прибыль,
а не привлеченные средства. Мы представили статистику по привлеченным
средствам, потому что это те цифры, которые у нас есть. Мы не смогли
бы содержательно говорить о прибылях, не включив цифры самых успешных
стартапов, а и у нас их нет. Мы часто обсуждаем рост прибыли со
стартапами на более ранней стадии, потому что именно так мы измеряем
их прогресс, но когда компании достигают определенного размера, то
такие действия со стороны посевного инвестора становятся наглостью.

В любом случае, рыночная капитализация компаний в конечном счете
начинает функционально зависеть от прибылей, и постинвестиционные
оценки представляют собой по меньшей мере догадки профессионалов
относительно того, к чему приведет такая капитализация.

Причиной, по которой оценку прошли только 287 стартапов, является то,
что остальные преимущественно привлекли средства на условиях
конвертируемых займов, и хотя конвертируемые займы часто имеют предел
оценки, он представляет собой лишь верхний предел оценочной стоимости.
Назад

[2] Мы не старались взять определенное количество. Мы бы не смогли это
сделать, даже если бы хотели. Мы просто старались быть более
разборчивыми. Назад

[3] С узкими местами никогда ничего не ясно, я думаю, что следующим
будет координация усилий между партнерами. Назад

[4] Я понимаю, что основать компанию не обязательно означает основать
стартап. Многие люди основывают обычные компании. Но это не актуально
для аудитории инвесторов.

Джеофф Ральстон говорит, что в Кремниевой долине казалось возможным
основать стартап в середине 1980х. Это бы началось там. Но я знаю, что
студенты Восточного побережья так не думали. Назад

[5] Эта тенденция – одна из основных причин увеличения экономического
неравенства в США с середины двадцатого столетия. Человек, который в
1950 году был бы генеральным менеджером подразделения x
Мегакорпорации, сейчас является основателем компании x и владеет
существенной ее долей. Назад

[6] Если Конгресс примет стартап-визу в неизмененной форме, то только
это в принципе может дать нам увеличение до 20 раз, поскольку 95\%
мирового населения живет за пределами США. Назад

[7] Если конфликты идей зайдут достаточно далеко, это может изменить
само понятие о стартапе. На данный момент мы, как правило, советуем
стартапам игнорировать конкурентов. Мы сравниваем стартапы с
соревнованиями по бегу, а не футболом: отбирать мяч у другой команды
не требуется. Но если конфликты идей станут достаточно обычным
явлением, возможно, придется начать это делать. Это было бы досадно.

\end{document}
