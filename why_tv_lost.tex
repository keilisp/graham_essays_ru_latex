\documentclass[ebook,12pt,oneside,openany]{memoir}
\usepackage[utf8x]{inputenc} \usepackage[russian]{babel}
\usepackage[papersize={90mm,120mm}, margin=2mm]{geometry}
\sloppy
\usepackage{url} \title{Почему телевидение погибло} \author{Пол Грэм}
\date{}
\begin{document}
\maketitle

Около двадцати лет назад люди заметили, компьютеры и ТВ вышли на курс,
грозящий катастрофой столкновения, и начали думать, что им
производить, когда они сойдутся в одной точке. Теперь мы знаем ответ:
компьютеры. Теперь ясно, что даже используя слово «сходиться» мы
давали ТВ слишком много чести. Это будет не столько «схождение»,
сколько замена. Люди могут продолжать смотреть то, что они называют
«ТВ шоу», но они будут смотреть их в основном на компьютерах.

Что решило противостояние в пользу компьютеров? Четыре силы, три из
которых можно было предсказать, и одна, которую предсказать было
труднее.

Одна из предсказуемых причин победы в том, что Интернет это открытая
платформа. Любой может построить на ней все, что захочет, и рынок
выберет победителей. Таким образом, инновация зависит от скорости
хакеров, вместо скорости крупных компаний. Второе это Закон Мура,
который сотворил волшебство с пропускной способностью Интернета.
Третья причина выигрыша компьютеров это пиратство. Пользователи
предпочитают его не только потому, что это бесплатно, а потому, что
это еще и удобнее. Bittorrent и YouTube уже подготовили новое
поколение зрителей, для которых местом просмотра шоу является экран
компьютера.

Несколько более удивительной была сила одного конкретного типа
нововведений: социальные приложения. Средний подросток имеет
безграничные возможности для общения с друзьями. Но они не могут
физически быть с ним все время. Когда я учился в школе, решением был
телефон. Теперь – социальные сети, многопользовательские игры, и
различные приложения для обмена сообщениями. Путь ко всем ним лежит
через компьютер. Что означает, что каждый подросток (а) хочет
компьютер с подключением к Интернету, (б) имеет стимул научиться им
пользоваться, и (в) проводит бесчисленные часы перед ним.

Это была самая мощная сила. Поэтому каждый захотел иметь компьютер.
Умники получили компьютеры, потому что любили их. Затем геймеры
получили их, чтобы играть в игры. Но это стало связью со всеми
остальными: поэтому даже бабушки и 14-летние девочки хотят компьютеры.

После десятилетий работы внутривенной капельницы, подключенной прямо к
их аудитории, люди в развлекательном бизнесе, разумеется, думали, что
те довольно пассивны. Они думали, что были в состоянии диктовать
условия того, как шоу достигают своей аудитории. Но они недооценили
силу их желания связываться друг с другом. Facebook убил ТВ. Это дикое
упрощение, конечно, но, вероятно, настолько близко к правде, насколько
можно выразить в трех словах.

---

ТВ сети уже, кажется, скрепя сердце, смотрят, где идут дела, и
отвечают, неохотно выкладывая свой материал в онлайн. Но они
продолжают тащиться как черепахи. Они до сих пор, кажется, хотели бы,
чтобы люди, вместо этого, смотрели шоу по ТВ, как газетам, которые
выкладывают материал в онлайн, хотелось бы, чтобы люди ждали до
следующего утра и читали его, напечатанным на бумаге.

Они были бы в лучшем положении, если сделали это раньше. Когда
возникает новая среда, достаточно мощная, чтобы заставить сотрудников
понервничать, то она, возможно достаточно мощная, чтобы выиграть, и
лучшая вещь, которую они могут сделать, это прыгнуть в нее немедленно.


Нравится им это или нет, грядут большие перемены, потому что Интернет
разрешает два краеугольных камня СМИ: синхронность и локальность. В
Интернете, вы не должны посылать каждому один и тот же сигнал, и вам
не придется посылать его из местного источника. Люди будут смотреть
то, что хотят, где хотят, и группируются в соответствии с общими
интересами, чувствуют себя сильнее. Может быть, их сильнейший общий
интерес будет их физическим расположением, но, я полагаю, нет. Что
означает, что местное ТВ вероятно мертво. Это был артефакт,
накладываемый старыми технологиями. Если бы кто-то создавал
Интернет-ТВ компанию с нуля, они могли бы иметь какой-то план для шоу,
направленных на регионы, но это не было бы главным приоритетом.

Синхронность и локальность связаны между собой. Отделения ТВ сети
беспокоятся о том, что происходит в 10, потому что это покажут
зрителям в 11 часовых новостях. Однако, эта связь добавляет больше
хрупкости, чем силы: люди не смотрят новости в 10, потому что они
хотят смотреть новости позже.


ТВ сети будут бороться с этими тенденциями, потому что недостаточно
гибкие, чтобы к ним адаптироваться. Они зажаты в местных отделениях
так же, как автокомпании зажаты дилерами и союзами. Безусловно, люди,
управляющие сетями, пойдут легким путем и попытаются сохранить старую
модель работающей в течение нескольких лет, как сделали
звукозаписывающие компании.

В недавней статье в Wall Street Journal описывалось, как ТВ сети
пытались добавлять больше прямых трансляций, частично как способ
заставить зрителей смотреть ТВ синхронно, вместо просмотра записанных
шоу тогда, когда им удобно. Вместо того, чтобы давать зрителям то, что
они хотят, они пытаются заставить их изменить свои привычки в
соответствии со старой бизнес моделью ТВ сетей. Это никогда не
сработает, если только у вас не будет монополии или картеля,
заставляющих так делать, и даже тогда это сработает только временно.

Другая причина, по которой ТВ сети любят прямые трансляции в том, что
они дешевле в производстве. Идея верная, но они не довели ее до конца.
Онлайн-контент может быть намного дешевле, чем представляют себе ТВ
сети, и способ получить выгоду от резкого снижения стоимости это
увеличение объема. ТВ сети не имеют возможности понять всю цепочку
рассуждений, потому что они все еще думают о себе как о
широковещательном бизнесе, посылающим один сигнал каждому.

---

Сейчас хорошее время, чтобы начать любую компанию, которая
конкурировала бы с ТВ сетями. Это то, чем являются многие Интернет
стартапы, хотя они и не думали ставить это своей целью. У людей много
часов свободного времени в день и ТВ основывается на таких длинных
сеансах (в отличие от Google, который гордится тем, что пользователи
быстро стартуют), что всё, что занимает их время, конкурирует с ними.
Но кроме таких косвенных конкурентов, я думаю ТВ компании будут все
чаще сталкиваться с прямыми конкурентами.

Даже в системах кабельного ТВ, длинный хвост был отрублен прежде, чем
переступить порог, чтобы запустить новый канал. В Интернете это будет
длиннее и мобильнее. В этом новом мире существующие игроки будут иметь
такие же преимущества, как и любая крупная компания на рынке.

Это изменит баланс сил между сетями и людьми, которые производят шоу.
Сети привыкли быть стражниками. Они распространяли вашу работу и
продавали на нее рекламу. Теперь люди, которые производят шоу могут
распространять его сами. Главная ценность поставок через сети это
продажи рекламы. Которая будет стремиться поставить их в положение
поставщиков услуг, а не издателей.

Шоу изменится еще больше. В Интернете нет никакой причины, чтобы
сохранять их нынешний формат, или даже то, что они должны быть одного
формата. На самом деле, интереснее сближение, которое происходит между
шоу и играми. Но на вопрос какой вид развлечений станет
распространенным через 20 лет в Интернете, я бы не рискнул делать
какие-то прогнозы, за исключением того, что все сильно изменится. Мы
получим все, что только могут сделать самые творческие люди. Именно
поэтому Интернет победил.

Замечания

[ 1 ] Спасибо Тревор Блэквелл за эту точку зрения. Он добавляет: «Я
помню как глаза телефонных компаний сверкали в начале 90-х, когда они
говорили о сближении. Они думали, что на программирование будет
большой спрос,, и они реализуют его и сделают много денег. Это не
сработало. Они полагали, что их местные сетевые инфраструктуры будут
иметь решающее значение в производстве видео по требованию, потому что
вы не могли направить его из нескольких центров обработки данных через
Интернет. В то время(1992) вся пропускная способность Интернет была
недостаточной для одного видеопотока. Но пропускная способность
увеличилась больше, чем они ожидали и их побили iTunes и Hulu.

[ 2 ] Владельцы авторских прав, как правило, сосредоточены на аспекте
того, как они видят пиратство, которое является причиной потерянного
дохода. Таким образом, они думают, что пользователями движет желание
получать что-то бесплатно. Но iTunes показывает, что люди будут
платить за продукт в онлайне, если вы сделаете это простым. Важным
компонентом пиратства является просто то, что оно предлагает лучший
способ использования.

[ 3 ] Или телефон, который, на самом деле, компьютер. Я не делаю
каких-либо прогнозов о размере устройства, которое заменит ТВ, оно
просто будет иметь браузер и получать данные через Интернет.

[ 4 ] Эмметт Шир пишет: «Я бы поспорила, что длинный хвост для спорта
может быть даже больше, чем длинный хвост для любого другого вида
контента. Любой может транслировать футбольную игру в школе, и это
будет интересно для десяти тысяч человек или около того, даже если
качество продукции не такое уж хорошее.»
\url{http://en.wikipedia.org/wiki/Long_Tail}

\end{document}
