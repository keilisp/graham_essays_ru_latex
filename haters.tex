\documentclass[ebook,12pt,oneside,openany]{memoir}
\usepackage[utf8x]{inputenc} \usepackage[russian]{babel}
\usepackage[papersize={90mm,120mm}, margin=2mm]{geometry}
\sloppy
\usepackage{url} \title{Хейтеры} \author{Пол Грэм} \date{}
\begin{document}
\maketitle

Если вы становитесь достаточно известным, вы получите несколько
поклонников, которым нравитесь слишком сильно. Этих людей иногда
называют “фанаты”, и хотя мне не нравится это определение, я буду
использовать его здесь. Нам нужно какое-то слово для них, потому что
они явно отличаются от тех, кому просто нравится ваша работа.


Фанаты одержимы и не принимают критики. Любовь к вам становится их
частью, и в их голове ваш образ сильно лучше настоящего. Всё что вы
делаете — хорошо, потому что вы это делаете. Если вы делаете что-то
плохое, они найдут способ представить это как хорошее. При этом их
любовь к вам обычно не тихая и негласная. Они хотят чтобы все знали,
насколько вы прекрасны.


Ну, вы можете подумать “Я мог бы обойтись и без навязчивых
поклонников, но я знаю, что в мире существуют разные люди, и если это
худшее последствие известности, то это не так уж и плохо”.


К сожалению, это не худшее следствие известности. Вместе с фанатами, у
вас будут и хейтеры.


Хейтеры одержимы и не принимают критики. Ненависть к вам становится их
частью, и в их голове ваш образ сильно хуже настоящего. Всё что вы
делаете — плохо, потому что вы это делаете. Если вы делаете что-то
хорошее, они найдут способ представить это как плохое. При этом их
ненависть к вам обычно не тихая и негласная. Они хотят чтобы все
знали, насколько вы ужасны.


Если вы думаете поискать отличия, я сохраню вам время. Второй и пятые
абзацы одинаковы кроме слов “хороший” замененных на “плохой” и так
далее.


Я потратил годы ломая голову над поведением хейтеров. Кто они и откуда
появляются? И однажды меня осенило. Хейтеры — просто фанаты со знаком
“минус”.


Имейте в виду, что под хейтерами я не имею в виду просто троллей. Я не
говорю о людях, которые скажут пару плохих вещей о вас и пойдут
дальше. Нет, я говорю о намного более узкой группе людей, для которых
это становится своего рода одержимостью, и которые повторяют это снова
и снова в течение долгого времени.


Как и фанаты, видимо хейтеры — это автоматическое следствие
известности. Они будут у кого угодно достаточно знаменитого. И, как и
фанаты, хейтеры питаются известностью того, кого ненавидят. К примеру,
они слышат песню какого-то поп-певца. Она им особо не нравится. Если
певец был бы неизвестным, они бы просто об этом забыли. Но вместо
этого они продолжают слышать его имя, и видимо это сводит некоторых
людей с ума. Все продолжают говорить о певце, но он абсолютно не
хорош! Он мошенник!


Тут важно слово “мошенник”. Это именно то, что думают хейтеры
относительно объекта их ненависти. Хейтеры не могут отрицать
известность их объектов. На самом деле, известность только
преувеличивается. Они замечают каждое упоминание имени певца потому
что каждое упоминание делает их злее. В их головах, они преувеличивают
как известность певца, так и отсутствие у него таланта. Для них
единственный способ примириться с этими двумя вещами — это заключить,
что певец всех обманул.


Какие люди становится хейтерами? Кто угодно может стать одним из них?
Я в этом не уверен, но заметил некоторые схожести. Хейтеры обычно
неудачники в очень определенном смысле: хотя они порой талантливы, они
никогда не достигали многого. И в самом деле, любой достаточно
успешный человек, который достиг значительной известности, в
большинстве случаев вряд ли будет отзываться о другом известном
человеке как о мошеннике, потому что любой известный человек знает как
известность случайна.


Но хейтеры не всегда полные неудачники. Они не всегда тот самый
шаблонный парень, живущий в мамином подвале. Многие да, но у других
есть талант. На самом деле, я подозреваю, что чувство нереализованного
таланта это и есть то, что движет некоторыми людьми на пути
становления хейтерами. Они не просто говорят “Это нечестно, что кто-то
известен”, они говорят “Это нечестно, что кто-то известен, но не я”.


Может хейтер быть вылечен если достигнет чего-нибудь впечатляющего? Я
бы сказал, что это спорный вопрос, потому что он никогда этого не
сделает. У меня была возможность наблюдать достаточно долго, так что я
уверен, что шаблон работает в обе стороны: не только люди, которые
делают работу отлично никогда не станут хейтерами, но и хейтеры
никогда не сделают работу отлично. Хотя мне не нравится слово “фанат”,
оно говорит кое-что важное о хейтерах и фанатах. Оно подразумевает,
что фанат настолько раболепно предсказуем в его восхищении, что в
результате он унижается, становится чем-то меньшим чем человек.


Хейтеры кажутся ещё более униженными. Я могу представить себя будучи
фанатом. Я могу представить людей, работой которых я восхищаюсь
настолько сильно, что я мог бы принизить себя перед ними из чистой
признательности. Если бы Пелам Вудхаус (P. G. Wodehouse) был всё ещё
жив, я мог бы увидеть себя его фанатом. Но я не могу вообразить себя
хейтером.


Знание того, что хейтеры — просто фанаты со знаком “минус” делает
обращение с ними намного легче. Нам не нужно отдельной теории для
хейтеров. Мы можем просто использовать существующие способы обращения
с навязчивыми поклонниками, самый важный из которых это просто не
думать много о них. Если у вас появятся хейтеры, вашей, как и
большинства достаточно знаменитых людей, первоначальной реакцией будет
непонимание. Что у этого парня против меня? Откуда исходит его
навязчивость и что делает его так ужасающе неприятным? Что я мог
такого сделать, чтобы вывести его? Это то что я могу исправить?


Ошибка здесь в том, чтобы думать о хейтере как о ком-то, с кем у вас
разногласия. Когда вы в разногласиях с кем-то, понять почему они
расстроены и попытаться исправить всё если это возможно, это обычно
хорошая идея. Разногласия отвлекают. Но думать о хейтере как о ком-то,
с кем у вас спор, это ложная аналогия. Это непонятно, если вы никогда
не сталкивались с хейтерами до этого. Но когда вы понимаете, что
имеете дело с хейтером и кто такой хейтер, становится очевидно, что
даже думать о них это трата времени. Если у вас есть навязчивые
поклонники, вы проводите время любопытствуя о том, что заставляет их
так чрезмерно вас любить? Нет, вы просто думаете “некоторые люди
отчасти сумасшедшие” и всё.


Так как хейтеры равнозначны фанатам, точно так же следует обращаться и
с ними. Может быть и было что-то, что вывело их из себя, но это не то,
что вывело бы из себя обычного человека, так что нет причин тратить
время думая об этом. Проблема не в вас, проблема в них.


Примечания [Пола Грэма]

Конечно есть и настоящие мошенники. Если X называет Y обманщиком, как
можно отличить случай, когда X — хейтер, от случая, когда Y
по-настоящему мошенник? Посмотрите на чужое мнение. Настоящие
мошенники обычно бросаются в глаза, вдумчивые люди редко увлечены ими.
Так что если есть несколько мыслящих людей которым нравится Y, обычно
можно предположить, что он не мошенник. Исключение составляют
подростки, которые иногда ведут себя так, что они в прямом смысле сами
не свои. Я могу представить подростка-хейтера, который потом вырос из
этого. Но не кого-то старше двадцати пяти. У меня намного хуже память
на проступки, чем у моей жены Джессики, но я бы не хотел это менять.
Большинство споров это трата времени даже если вы правы. Легче
закопать топор войны с кем-то, если не можете вспомнить почему злились
на него. Знающий хейтер не будет просто нападать на вас в одиночку, он
попытается натравить толпу. В некоторых случаях вы можете захотеть
опровергнуть фальшивые заявления, которые хейтеры сделали, чтобы
вызвать это, но ошибетесь, потому что в конечном счёте это скорее
всего не будет иметь значения.

\end{document}
