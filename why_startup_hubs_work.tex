\documentclass[ebook,12pt,oneside,openany]{memoir}
\usepackage[utf8x]{inputenc} \usepackage[russian]{babel}
\usepackage[papersize={90mm,120mm}, margin=2mm]{geometry}
\sloppy
\usepackage{url} \title{Каботают инкубаторы стартапов} \author{Пол
  Грэм} \date{}
\begin{document}
\maketitle

Если вы взглянёте на упорядоченный по числу жителей список
американских городов, вы заметите, что количество успешных стартапов
на душу населения зависит от их размера. Почему-то это выглядит так,
как если бы большинство небольших городков было обрызгано
стартапосептиком.

Этот вопрос интересовал меня годами. Мне представлялось, что средний
городок – это мотель с тараканами для стартап-амбиций: умные,
амбициозные люди входили в него, но стартапы не выходили никогда. Но я
не понимал, что же происходит внутри мотеля — что же это за яд,
который убивает все потенциальные стартапы1.

Пару недель назад меня,наконец, осенило. Я неправильно обозначил
вопрос. Проблема не в том, что большинство городков убивает стартапы.
Стартапы по умолчанию мертвы, и большинство городков их не спасает.
Вместо того чтобы думать, что большинство городков обрызгано
стартапосептиком, более корректным было бы думать, что все стартапы
отравлены, и только некоторые места обрызганы противоядием.

В других местах стартапы делают то, что положено делать стартапам:
терпят неудачу. По-настоящему важно другое: что спасает стартапы в
таких местах, как Кремниевая долина2?

Окружающая среда

Я думаю, у противоядия две составляющих: атмосфера, в которой
заниматься стартапами — это круто, и случайные встречи с людьми,
которые могут вам помочь. И обе составляющих напрямую зависят от
количества «стартаперов» вокруг вас.

Первая составляющая особенно полезна на начальном этапе существования
стартапа, когда вы переходите от простого желания создать компанию к
реальным шагам. Создать стартап – это настоящий рывок. Это не
заурядное решение. Но в Кремниевой Долине это кажется нормальным3.

В большинстве других мест, когда вы открываете стартап, люди
воспринимают вас как безработного. Обитатели Долины не будут
автоматически впечатлены вами исключительно из-за того, что вы
открываете компанию, но они обратят на вас внимание. Любой, кто провёл
здесь какое-то время, знает, что не следует предаваться скептицизму,
безотносительно того, насколько неопытным вы кажетесь и насколько
малообещающим представляется ваш проект, потому что они видели
неопытных основателей с малообещающими идеями, которые через несколько
лет становились миллиардерами.

Окружающие, которых волнует то, чем ты занимаешься — это необычайно
могущественная сила. Даже самые своенравные люди к ней восприимчивы.
Примерно через год после того, как мы открыли Y Combinator, я как-то
бросил партнёру из известной венчурной фирмы что-то, что произвело на
него (ошибочное) впечатление, будто я обдумываю открытие ещё одного
стартапа. Он ответил так увлечённо, что полсекунды я по-настоящему это
обдумывал.

В большинстве городов перспектива открыть стартап просто не выглядит
реальной. В Долине это не только возможно, но даже модно. Что, без
сомнения, приводит к тому, что стартапы открывает большое количество
людей, которым не следовало бы это делать. Но, думаю, это нормально.
Очень немногие люди годятся для запуска стартапа, и этих людей очень
сложно вычислить заранее (что я знаю слишком хорошо, так как мой
бизнес — прозревать), так что большое количество людей, открывающих
стартапы, хотя им и не следовало бы — это наилучшее положение вещей.
Пока возраст позволяет вам мириться с риском провала, наилучший способ
выяснить, можете ли вы заниматься стартапом — попробовать.

Удача

Вторая составляющая противоядия — это шанс встретиться с людьми,
которые могут тебе помочь. Эта сила действует на обеих этапах: и при
переходе от желания основать стартап к его реальному открытию, и на
пути от открытия стартапа к успеху. Влияние случайных знакомств более
непостоянно, чем влияние окружающих, которых интересуют стартапы,
которое, как фоновая радиация, затрагивает всех одинаково, но когда
этих встреч много, они действуют гораздо более сильно.

Случайные знакомства творят чудеса, что компенсирует несчастья,
которые, как им это и свойственно, приключаются со стартапами. В
Долине, как и везде, ужасные вещи случаются со стартапами постоянно.
Причина, по которой стартапы здесь чаще достигают успеха в том, что с
ними случаются и прекрасные вещи. В Долине молния оставляет заметный
след.

К примеру, вы создаёте сайт для студентов колледжа, и вы решаете
отправиться на лето в Долину, чтобы поработать над ним. И вот на
случайной узкой улочке в Поло Альто вы встречаете Шона Паркера,
который действительно хорошо разбирается в этой области, потому что он
сам основал такой же стартап. Вдобавок он знаком с инвесторами. И,
кроме того, у него продвинутые взгляды (для 2004-го) относительно
учредителей, сохраняющих контроль над своими компаниями.

Вы не можете точно сказать ни каким именно будет чудо, ни случится ли
оно вообще. Уверенно можно сказать следующее: если вы в центре
притяжения стартапов, с вами, вероятно, случатся приятные
неожиданности, особенно если вы их заслуживаете.

Держу пари, это верно даже для стартапов, в которые мы вкладываем
деньги. Даже когда мы работаем над тем, чтобы события происходили с
ними целенаправленно, а не случайно, вероятность неожиданной случайной
встречи в Долине настолько велика, что она многое добавляет к тому,
что мы можем обеспечить.

Случайные встречи играют такую же роль, какую отдых играет для новых
идей. Большинству людей знакома ситуация, когда вы много работаете над
проблемой, не можете её решить, бросаете и идёте спать, а потом
обдумываете решение во время утреннего душа. Ответ заставляет
появиться то, что вы позволили своим мыслям немного поплыть по
течению, и они просочились с неверного пути, которым вы следовали
прошлой ночью, на нужный, находившийся по соседству.

Случайные встречи позволяют вашим знакомствам так же плыть по течению,
как вы позволяете своим мыслям, когда принимаете душ. Решающий момент
в обоих случаях, это то, что они идут самотёком ровно столько сколько
нужно. Хороший пример – встреча Ларри Пейджа и Сергея Брина. Они
позволили себе плыть по течению, знакомясь с людьми, но только совсем
немного, они оба общались с теми, с кем у них было много общего.

Для Ларри Пейджа важнейшим компонентом противоядия был Сергей Брин, и
наоборот. Противоядие – это люди. Кремниевую долину приводит в
действие не физическая инфраструктура, или погода, или что-нибудь
другое в этом роде. Это помогло всему начаться, но сейчас это цепная
реакция, основной компонент которой – люди.

Многие наблюдатели заметили, что одна из наиболее характерных черт
центров притяжения стартапов – это, как много люди помогают друг
другу, не ожидая ничего взамен. Возможно, это потому, что в стартапах
значительно меньше от игры с нулевой суммой, чем в большинстве видов
бизнеса, их редко убивают конкуренты. Или, возможно, это потому, что
так много основателей стартапов вышли из мира науки, где
сотрудничество поощряется.

Значительная часть функций YC заключается в том, чтобы ускорить этот
процесс. Мы — что-то вроде Долины внутри Долины, где плотность людей,
работающих над стартапами и их готовность помогать друг другу
искусственно увеличены.

Многочисленность

Обе составляющие противоядия — окружение, которое поощряет стартапы и
случайные встречи с людьми, которые тебе помогают — вызваны одной
лежащей в основе причиной: количеством занятых стартапами людей вокруг
тебя. Для того, чтобы сделать центр притяжения стартапов, вам нужно
много людей, заинтересованных в стартапах.

Этому есть три причины. Первая, естественно, в том, что если у вас нет
достаточной плотности, случайные встречи не происходят4. Вторая
заключается в том, что разные стартапы требуют настолько разных вещей,
что вам понадобится множество людей, чтобы снабдить каждый стартап
всем наиболее необходимым. Шон Паркер был именно тем, в чём в 2004
году нуждался Facebook. Другой стартап мог нуждаться в парне,
разбирающемся в базах данных, или в ком-то со связями в кино.

Между прочим, это одна из причин, по которым мы вкладываемся в такое
большое количество компаний. Чем больше сообщество, тем выше шансы,
что оно содержит человека, у которого есть то, в чём вы больше всего
нуждаетесь.

Третья причина того, что вам нужно много людей для того, чтобы создать
центр притяжения стартапов, заключается в том, что когда у вас есть
достаточно людей, интересующихся одной и той же проблемой, они
начинают устанавливать социальные нормы. И это особенно ценно тогда,
когда окружающая вас атмосфера побуждает вас сделать что-то, что при
других обстоятельствах казалось бы слишком амбициозным. В большинстве
мест атмосфера просто поставит вас на место.

Несколько дней назад я прилетел в Bay Area. Каждый раз, когда я лечу
над Долиной, я замечаю: каким-то образом вы чувствуете, что что-то
происходит. Очевидно, вы можете оценить процветание по тому, насколько
хорошо выглядит город. Но есть разные виды процветания. Кремниевая
Долина не выглядит как Бостон, или Нью-Йорк, или Лос-Анджелес, или
Вашингтон. Я пытался спросить себя, каким словом можно опасть
ощущение, которым облучена Кремниевая Долина, и слово, которое пришло
мне на ум, было оптимизм.

Примечания

[1] Я не хочу сказать, что нельзя преуспеть в городе с небольшим
количеством других стартапов, просто это труднее. Если вы сами
прекрасно справляетесь с тем, чтобы поддерживать боевой дух, вы можете
справиться без посторонней поддержки. Wufoo были основаны в Тампе, и
они пришли к успеху. Но Wufoo исключительно дисциплинированы.

[2] Между прочим, это явление не относится только к стартапам. Большая
часть необычных начинаний проваливается, если их создатели не находят
походящее общество.

[3] Основание компании – обычное явление, но основание стартапа
случается редко. Я уже говорили о разнице между этими понятиями, но
вкратце, стартап – это новый бизнес с большими амбициями.

[4] Когда я это писал, я наткнулся на пример, демонстрирующий
плотность связанных со стартапами людей в Долине. Джессика и я
отправились на велосипедах на Юнивёрсити Авеню в Пало Альто, для того,
чтобы пообедать в восхитительном «Oren's Hummus». Когда мы вошли, мы
увидели Чарли Чивера, сидящего около двери. Selina Tobaccowala
остановилась, чтобы поздороваться на выходе. Затем Josh Wilson
заскочил, чтобы забрать заказ с собой. После обеда мы пошли поесть
замороженного йогурта. По дороге мы встретили Rajat Suri. Когда мы
добрались до йогурта, мы встретили Dave Shen, и по дороге обратно мы
столкнулись с Yuri Sagalov. Мы прошли с ним около квартала и напали на
Muzzammil Zaveri, и ещё через квартал Aydin Senkut. Такова обычная
жизнь в Пало Альто. Я не старался встречаться с людьми, я просто
обедал. И я уверен, что на каждого основателя стартапа или инвестора,
которых я знал, там было пятеро, которых я не знал. Если бы с нами был
Ron Conway, мы бы встретили 30 его знакомых.

\end{document}
