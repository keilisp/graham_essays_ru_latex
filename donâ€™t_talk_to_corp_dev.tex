\documentclass[ebook,12pt,oneside,openany]{memoir}
\usepackage[utf8x]{inputenc} \usepackage[russian]{babel}
\usepackage[papersize={90mm,120mm}, margin=2mm]{geometry}
\sloppy
\usepackage{url} \title{Не разговаривайте со специалистами по
  корпоративному развитию} \author{Пол Грэм} \date{}
\begin{document}
\maketitle

Отделы корпоративного развития внутри компаний занимаются покупкой
других компаний. И если вы говорите с кем-то оттуда — именно об этом и
идет речь, осознаете вы это или нет.

Как правило, вы совершаете ошибку, вступая в беседу со специалистами
по корпоративному развитию, кроме случаев, когда вы: а) намерены
продать вашу компанию прямо сейчас и б) уверены в том, что получите
предложение приемлемой цены. На практике это означает, что стартапам
стоит общаться с отделами корпоративного развития только тогда, когда
дела у них идут действительно хорошо или совсем плохо. Если ваша
компания умирает, спокойно общайтесь, ведь терять вам уже нечего. Если
же у вас все отлично, вы также можете спокойно разговаривать с ними:
все понимают, что цена будет высокой, а если вам покажется, что эти
ребята зря тратят ваше время, вам хватит уверенности в себе, чтобы
просто выгнать их.

Специалисты по корпоративному развитию представляют опасность для
компаний, находящихся посредине. Особенно для молодых компаний —
быстро растущих, но все еще недостаточно зрелых. Общение с ними будет
ошибкой для многообещающей компании, которая не просуществовала и
года.


Эту ошибку основатели компаний делают снова и снова. Когда с ними
пытается встретиться кто-нибудь из отдела корпоративного развития, они
полагают, что стоит хотя бы выяснить, чего он хочет. Кроме того, отказ
от встречи с Большой Компанией может быть расценен как проявление
неуважения.

“Смесь отрицания и привычки выдавать желаемое за действительное лежит
в основе большинства ошибок основателей”

Я расскажу вам, чего хотят специалисты по корпоративному развитию. Они
хотят обсудить покупку вашей компании. Это и означает «корпоративное
развитие». Так что прежде, чем соглашаться на встречу с кем-нибудь из
отдела корпоративного развития, спросите себя: «Хотим ли мы продавать
компанию прямо сейчас?» И если ваш ответ «нет», скажите специалистам
по корпоративному развитию: «Извините, но сейчас мы сосредоточены на
развитии компании». Их это не обидит. И, конечно, владельцев Большой
Компании это тоже не обидит. Вы можете только вырасти в их глазах,
напоминая им их самих. Они тоже не продались, и поэтому сейчас
покупают другие компании [1].

Большинство основателей компаний в курсе, зачем с ними связывается
отдел корпоративного развития. Но даже при том, что они знают, чем
этот отдел занимается, как знают и то, что не хотят продавать
компанию, они соглашаются на встречу. Зачем? К этому приводит та же
смесь отрицания и привычки выдавать желаемое за действительное,
которая лежит в основе большинства ошибок основателей. Поговорить с
кем-то, кто хочет вас купить, всегда лестно. И кто знает — а вдруг их
предложение окажется на удивление высоким. Стоит как минимум
ознакомиться с ним, так ведь?

Нет. Если бы они выслали вам предложение электронным письмом, вы бы,
конечно, открыли его. Но общение с отделом корпоративного развития
работает не так. Если вы вообще получите предложение, оно поступит в
самом конце долгого и невероятно выматывающего процесса. И если оно
вас чем-то и удивит, то только тем, что будет неожиданно низким.

“Это похоже на то, как если бы фрагмент генетического материала
устаревшего бандитского бизнеса вживили в мир стартапов”

Отвлекаться — непозволительная роскошь в стартапе. И разговоры с
представителями по корпоративному развитию — худшее, что может вас
отвлечь, ведь они не только поглощают ваше внимание, но и подрывают
боевой дух. Если вы хотите пережить этот изнурительный процесс, не
стоит останавливаться, чтобы понять, как же вы устали. Наоборот: нужно
влиться в поток. [2] Представьте, что произойдет с вами, если на 32
километре марафона кто-то подбежит к вам и скажет: «Вы, наверное,
сильно устали. Не хотите ли остановиться и отдохнуть?» Общение с
отделом корпоративного развития — как раз такое, только хуже,
поскольку их предложение остановиться совмещается в вашей голове с
высокой ценой, предложение которой вы рассчитываете от них получить.



И вот вы в настоящей беде. Если им удается, специалисты по
корпоративному развитию обращают против вас ваше же оружие. Они любят
довести вас до того, чтобы вы прилагали усилия, чтобы убедить их
купить вас, а не им приходилось бы уговаривать вас продать. И удается
им это удивительно часто.

Это очень скользкая дорожка, политая мощнейшими факторами влияния на
позиции основателей. И подводят вас к ней опытные профессионалы, чья
работа состоит непосредственно в том, чтобы столкнуть вас на эту
дорожку.

Методы, которыми они сталкивают вас, обычно довольно жестоки. Вся
работа людей из отделов корпоративного развития заключается в том,
чтобы покупать компании, им даже не нужно выбирать, какие.
Единственное мерило их эффективности — насколько дешево они могут вас
купить, и самые амбициозные из них не остановятся ни перед чем, чтобы
добиться минимальной цены. Например, они почти всегда начинают с
заниженного предложения — просто чтобы проверить, согласитесь ли вы на
него. Даже если вы не пойдете на это, низкая стартовая цена
деморализует вас, облегчая им дальнейшее манипулирование.

И это — самая невинная из их тактик. Иногда они ждут, пока вы
согласитесь на цену и сочтете сделку состоявшейся, после чего
возвращаются и сообщают, что их босс отменил сделку и не согласен дать
больше, чем половину согласованной ранее цены. Так происходит все
время. Если вы считаете, что инвесторы могут плохо себя вести, — это
ничто в сравнении с тем, на что способны специалисты по корпоративному
развитию. Даже из отделов корпоративного развития компаний,
благожелательно настроенных во всем остальном. Помню, как я однажды
пожаловался другу из Google на кое-какой грязный прием, который
применил их отдел корпоративного развития к стартапу YC. «А как же
девиз “Не делай зла”?» — спросил я. «Я не думаю, что отдел
корпоративного развития знает о нем», — ответил он. [3]

Тактика, используемая при обсуждении слияний и поглощений, может быть
абсолютно противна всему остальному в сравнительно честном мире
Силиконовой долины. Это похоже на то, как если бы фрагмент
генетического материала устаревшего бандитского бизнеса вживили в мир
стартапов.

“Если вы запомнили только заголовок этой статьи, вам уже известна
большая часть того, что нужно знать о слияниях и поглощениях на
протяжении первого года”

Самый простой способ защитить себя — использовать прием Джона Д.
Рокфеллера, который тот практиковал, чтобы не стать алкоголиком
(каковым был его дед). Однажды он задал вопрос классу в воскресной
школе: «Парни, а знаете, почему я не стал алкоголиком? Потому что я
так и не выпил первую рюмку». Хотите продать компанию прямо сейчас? Не
когда-то, а сейчас же? Если нет, просто не идите на первую встречу.
Они не обидятся. А вашей наградой станет гарантированное избавление от
опыта, хуже которого для стартапа нет.

Если вы действительно собираетесь продавать компанию сейчас, для этого
существует другой набор техник. Но самая большая ошибка, которую
совершают основатели компаний при общении с отделами корпоративного
развития, — это вовсе не плохо проведенная беседа со специалистами по
корпоративному развитию, когда они готовы вас купить, а
несвоевременный разговор — когда они еще не готовы. В общем, если вы
запомнили только заголовок этой статьи, вам уже известна большая часть
того, что нужно знать о слияниях и поглощениях на протяжении вашего
первого года.

Примечания

[1] Я не говорю, что вы никогда не должны продавать компанию. Я имею в
виду, что вы должны четко понимать, хотите вы продавать ее или нет, и
если нет — вы должны пресекать любые попытки заставить вас продать
раньше, чем вы планировали, путем манипуляций или из-за подмены
действительного желаемым.

[2] В стартапе, как и в самых соревновательных видах спорта, эта
задача практически выполняет сама себя: вы слишком заняты, чтобы
уставать. Но как только вы теряете эту защиту, т. е. с финальным
свистком, истощение ударит по вам волной. Общение с отделами
корпоративного развития равносильно такой волне посреди игры.

[3] Если быть до конца честным, очевидная неприемлемость поступков
людей из отделов корпоративного развития усиливается тем фактом, что
они представляют крупные организации, которые часто сами не знают,
чего хотят. Покупатели могут быть удивительно нерешительными в
отношении своих приобретений, и к моменту, когда дело доходит до вас,
их необязательность становится неотличима от обмана.

\end{document}
