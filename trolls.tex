\documentclass[ebook,12pt,oneside,openany]{memoir}
\usepackage[utf8x]{inputenc} \usepackage[russian]{babel}
\usepackage[papersize={90mm,120mm}, margin=2mm]{geometry}
\sloppy
\usepackage{url} \title{Тролли} \author{Пол Грэм} \date{}
\begin{document}
\maketitle

Один пользователь недавно оставил на Hacker News комментарий, который
заставил меня задуматься:

"В хакерской культуре меня всегда поражала одна вещь -
язвительность... Я просто не понимаю, зачем эти люди занимаются
троллингом."

В последние два года я много думал о проблеме троллинга. Это старая
проблема, такая же старая, как и форумы, но мы до сих пор пытаемся
понять, в чем её причины и как с ними бороться.

У слова "тролль" есть два значения. Изначально это понятие применялось
к человеку, обычно пришедшему со стороны, который намеренно вызывал
перепалки на форуме, высказывая противоречивые суждения. [1] Например,
человек, не работающий с определенным языком программирования, заходил
на форум, где собирались люди, работающие с этим языком, и
пренебрежительно высказывался о нем, после чего наблюдал, как люди
ловятся на его удочку. Такой вид троллинга представлял собой нечто
вроде шутки, вроде той, когда в комнату, полную людей, запускают
крысу.

Впоследствии определение стало распространяться и на людей, которые
вели себя на форумах как полные уроды, независимо от того, намеренно
или нет. Сейчас, когда люди говорят о "троллях", они, как правило,
имеют в виду именно это расширенное определение. Хотя, в какой-то
степени, такое использование слова исторически неверно, с другой
стороны, это более точно, потому что, когда кто-то ведет себя как
урод, никогда нельзя быть уверенным, делает ли он это нарочно или нет.
Это, несомненно, одно из определяющих свойств уродов.

Мне кажется, у троллинга в широком смысле есть четыре причины.
Наиболее важная причина -- это расстояние. Люди анонимно говорят в
форумах такие вещи, которые они никогда бы не осмелились произнести в
лицо, так же, как и за рулем автомобиля они делают то, что никогда не
сделали бы, будь они пешеходами: едут вплотную к другим машинам,
сигналят им, подрезают их.

Особенно троллинг распространён в форумах на компьютерные темы, и я
думаю, это связано с тем типом людей, которых там можно встретить.
Большинство из них (в том числе и я) больше привыкли иметь дело с
абстрактными понятиями, чем с живыми людьми. Хакеры могут быть
грубиянами и в жизни. А когда они приходят на анонимный форум,
проблема только усугубляется.

Третья причина троллинга связана с некомпетентностью. Если вы
несогласны с кем-то, гораздо легче сказать "идиот", чем попытаться
сформулировать и объяснить, с чем именно вы несогласны. К тому же,
такого рода высказывания невозможно опровергнуть. В этом отношении
троллинг очень похож на граффити. Граффити -- это смесь амбиций и
некомпетентности: люди хотят послать своё сообщение миру, но не
находят других способов, кроме как буквально написать это сообщение на
стене.

И последний фактор - это культура форума. Тролли похожи на детей (а
многие и есть дети) в том, что они могут вести себя очень по-разному,
в зависимости от того, что, по их мнению, сойдет им с рук, а что -
нет. В местах, где грубость не допускается, большинство посетителей
ведут себя вежливо. И наоборот.

К троллингу может быть примением закон Грэшема (Gresham's Law):
троллям нравится участвовать на форумах с множеством интеллигентных
людей, но интеллигентные люди не желают быть на форуме со множеством
троллей. Это означает, что как только троллинг начинает иметь место,
как правило, он начинает преобладать на форуме. Это уже произошло со
Slashdot и Digg к тому времени, как я начал просматривать там
комментарии, и я видел как это происходило с Reddit.

News.YC отличается от других тем, что здесь мы пытаемся предотвратить
эту напасть. Правила сайта явно предписывают пользователям не писать
вещи, которые бы они не могли бы сказать в лицо. Если кто-то начинает
грубить, другие пользователи вступают и говорят ему прекратить. Когда
люди начинают намеренно троллить, мы их безжалостно баним.

Также могут помочь технические решения. На Reddit оценки ваших
комментариев не влияют на вашу карму, на News.YC они влияют. И похоже,
что людей, которые дорожат своей репутацией на сайте, это удерживает
от дурацких комментариев. Довольно часто люди, написав комментарий,
немного подумав, позже его удаляют.

Могут возникнуть опасения, что такая система будет вынуждать людей не
писать спорные, неоднозначные вещи, но опыт показывает что это не так.
Когда пользователи пишут что-то, что кажется им важным, и это что-то
оценивается негативно, они всё равно упорно поднимают эту тему. Люди с
лёгкостью удаляют лишь такие комментарии, как саркастические
замечания, потому что для них они не имеют большого значения.

Похоже, что пока эта система работает. Уровень обсуждений на News.YC
не ниже, чем на любом форуме, который я когда-либо видел. Хотя,
конечно, у нас пока всего 8000 уникальных посещений в день. На Reddit
тоже были хорошие дискуссии пока его сообщество было небольшим. Нам
предстоит попытаться сохранить наш уровень и в будущем.

Я думаю, что у нас получится. Мы не полагаемся только на техническую
сторону. Основная масса пользователей News.YC это люди, покинувшие
другие сайты, которые были захвачены троллями. Их отношение к троллям
примерно такое же, как отношение беженцев с Кубы или Восточной Европы
к диктатурам. Таким образом, большое количество людей у нас
прикладывает много усилий, чтобы этого не произошло опять.

Примечания

[1] Под форумом я подразумеваю любое место для обмена мнениями.
Первоначально, интернет-форумы были группами Usenet, а не веб-сайтами.

[2] Я имею в виду обычные надписи на стенах. Некоторые графитти
довольно впечатляющие (всё что угодно может стать искусством, если вы
вкладываете в это свою душу), но обычно графитти это просто визуальный
мусор.



\end{document}
