\documentclass[ebook,12pt,oneside,openany]{memoir}
\usepackage[utf8x]{inputenc} \usepackage[russian]{babel}
\usepackage[papersize={90mm,120mm}, margin=2mm]{geometry}
\sloppy
\usepackage{url} \title{Microsoft мертва} \author{Пол Грэм} \date{}
\begin{document}
\maketitle

Пару дней назад я внезапно осознал, что Microsoft мертва. Я обсуждал с
молодым основателем стартапа, чем Google отличается от Yahoo. Я
говорил, что в Yahoo с самого начала слишком боялись Microsoft. Именно
поэтому они позиционировали себя как медиакомпанию (а не как компанию,
которая занимается технологиями). И тут я посмотрел на собеседника и
понял, что он потерял нить. С тем же успехом я мог ему рассказать, как
в середине 80-х девочки пищали от Барри Манилова. Какого-какого Барри?

Microsoft? Он ничего не сказал, но уж поверьте мне: в его голове
просто не укладывается, чтобы кто-нибудь мог бояться Microsoft.

Начиная с поздних 80-х и в течение почти двадцати лет софтверная
индустрия жила в тени Microsoft. Я помню времена, когда на месте
Microsoft была IBM. Лично я по большей части просто игнорировал их
влияние. Я никогда не использовал программы от Microsoft, так что
сталкивался с ними не напрямую - к примеру, в спаме, который мне
присылают боты. И поскольку я не уделял внимание Microsoft, я не
заметил, как тень исчезла.

Но ее нет. Я чувствую это. Никто больше не боится Microsoft. Они все
еще зарабатывают кучу денег - как и IBM, кстати. Но они уже не опасны.

Когда Microsoft умерла и от чего? Я знаю, что еще в 2001 году она
казалась опасной, потому что как раз тогда я написал эссе ("Другой
дорогой", 2001) о том, что не так страшна Microsoft, как ее малюют.
Думаю, все было кончено к 2005 году. Когда мы запускали Y Combinator1,
мы уже не воспринимали Microsoft как возможного конкурента нашим
стартапам. На самом деле, мы никогда не приглашали их на демо-дни, где
стартапы делают презентации для потенциальных инвесторов. Мы звали
Google и Yahoo и другие интернет-компании, но позвать Microsoft нам и
в голову не приходило. Да и они не прислали нам ни одного е-мэйла. Они
живут в другой реальности.

Что их убило? Четыре вещи - и все они одновременно случились в
середине 2000-х.

Самая очевидная - Google. Боливар не вынесет двоих, и на нем,
очевидно, удержался Google. Сегодня именно Google - самая опасная
компания (в плохом и хорошем смысле). Microsoft, в лучшем случае,
прихрамывая, плетется следом.

Когда Google вырвался вперед? Несмотря на соблазн взять в качестве
точки отсчета их IPO в августе 2004 года, я думаю, что диктовать
правила игры они начали не тогда, а в 2005 году. Не последнюю роль
сыграл в этом и Gmail, который показал, что Google может заниматься не
только поиском.

Также Gmail наглядно продемонстрировал, сколь многого можно достичь с
помощью веб-софта, если у вас есть преимущество, которое позднее
получит название Ajax. И это - вторая причина смерти Microsoft: всем
стало очевидно, что дни настольных систем сочтены. Приложения
переезжают в веб - не только почтовые клиенты, а все, вплоть до
Photoshop. И даже в Microsoft уже понимают это.

Забавно, что Ajax нечаянно помогла создать сама Microsoft. Буква х в
Ajax отсылает нас к объекту XMLHttpRequest, который позволял браузеру
обращаться к серверу в фоновом режиме, во время показа страницы
(изначально единственный способ получить новую информацию от сервера -
это полная перезагрузка страницы). XMLHttpRequest придумали в
Microsoft в конце 90-х, он им понадобился для Outlook. Однако в
Microsoft не увидели, что этот объект может быть полезен и множеству
других людей - точнее, всем, кто хочет создавать веб-приложения,
подобные настольным пакетам.

Следующим важнейшим компонентом Ajax является JavaScript, язык
программирования, работающий в браузере. Microsoft понимала опасность
JavaScript и держала его в нерабочем состоянии так долго, как только
могла [1]2. Но в конечном счете сообщество open source победило,
создав JavaScript-библиотеки, которые опутали болячки Internet
Explorer, как растение опутывает колючую проволоку.

Четвертая причина смерти Microsoft - широкополосный Интернет. Если
человеку нужно, он легко может получить быстрый доступ к Сети. А чем
шире канал, тем меньше нужен десктоп.

Последний гвоздь в крышку гроба вбила Apple. Благодаря OS X, Apple
восстала из мертвых (что с технологическими компаниями случается
крайне редко3). Их победа столь полна, что я даже удивляюсь,
наткнувшись где-нибудь на компьютер, работающий под Windows. Почти
все, кого мы финансировали в Y Combinator, используют ноутбуки от
Apple. Такая же ситуация в школе стартапов. Все компьютерщики сегодня
работают на Маках или под Linux. Винды - для бабулек (как Маки в
90-х). Так что не только десктопы потеряли свое значение. Люди,
которых интересуют компьютеры, перестали использовать ПО от Microsoft.

И, конечно, Apple обошла Microsoft в музыкальном бизнесе (с ТВ и
телефонами скоро произойдет то же самое).

Я рад, что Microsoft мертва. Она все равно что Нерон или Коммод. Она
зла, как злы те, кому великая власть досталась в наследство. Потому
что монополия Microsoft родилась раньше Microsoft. Она досталась ей в
наследство от IBM. Софтверный бизнес был задавлен монополией с
середины 1950-х до примерно 2005 года. То есть практически с момента
своего рождения. И одна из причин эйфории вокруг Web 2.0 - это
понимание, может, сознательное, а может, и нет), что эра монополии
закончилась.

Конечно, как хакер, я не могу перестать думать о том, как починить то,
что сломалось. Может ли Microsoft вернуться к жизни? Теоретически да.
Чтобы понять как, представьте себе две вещи:

количество кэша у Microsoft на руках; как Ларри и Сергей обходили
десять лет назад поисковики, чтобы продать идею Google за миллион
долларов, и как все им отказали. Интересно, что замечательных хакеров
- опасно замечательных хакеров - можно купить очень дешево, по
стандартам такой богатой компании, как Microsoft. Умных людей им уже
не нанять (см. "Работа по найму выходит из моды" Пола Грэма. - Прим.
ред.), зато купить они могут столько, сколько нужно, лишь выложив на
порядок больше денег4. Так что если они захотят вернуться на ринг,
сделать это они могут следующим образом:

Скупив все хорошие `вебдваноль`-стартапы. На то, чтобы скупить всё,
потребуется меньше денег, чем пришлось бы выложить за Facebook5.
Отправить все эти стартапы в Кремниевую Долину, где поместить их в
здание, защищенное от любых контактов с Редмондом свинцовым экраном. Я
так спокойно говорю об этом, потому что Microsoft никогда не сделает
ни первого, ни второго. Главная слабость Microsoft в том, что компания
все еще не понимает, в каком положении оказалась. Они все еще думают,
что могут сами писать программы. Возможно, и могут - для настольных
систем. Но эта эпоха закончилась несколько лет назад.

Я уже знаю, какую реакцию вызовет это эссе. Половина читателей скажет,
что Microsoft - все еще невероятно прибыльная компания, а мне нужно
быть поосторожней, строя рассуждения на мнениях тех, кто варится в
нашем маленьком, но шумном котле Веб 2.0. Но вторая половина - те, что
помоложе, - заявит, что ничего нового я не сообщил.

P.S.Мои слова о том, что Microsoft мертва не нужно понимать буквально.
Да это и невозможно. Компании неживые, так что и умереть они не могут.

На самом деле, выражение "Майкрософт мертва" можно считать метафорой.
Я имел в виду нечто иное. Но по вопросу, что же я имел в виду,
наблюдается некоторое несогласие. Те, кто был возмущен моим эссе,
убеждали себя, что я имел в виду что-то совсем уж глупое: например,
что Microsoft вскоре вылетит из бизнеса. И усердно меня опровергали.

Так что, наверное, мне лучше все-таки объясниться. Я не говорил, что
Microsoft перестанет зарабатывать деньги. Я говорил, что люди,
находящиеся на переднем крае софтверного бизнеса, могут больше не
брать её в расчет.

В мире хватает прибыльных компаний такого плана. Например, SAP. Они
зарабатывают кучу денег. Но разве разработчики новой технологии должны
задумываться о SAP? Я сомневаюсь. Когда я говорил, что Microsoft
мертва, я хотел сказать, что они - как IBM до них - ушли от нас в этот
другой мир.

Уход не означает, что компания вылетит из бизнеса в следующем году
(точно так же как потеря популярности не означает, что поп-звезде пора
на паперть). Но, возможно, это предвестник будущих неприятностей. И
если актеры и музыканты порой возвращаются на сцену, технологическим
компаниям это почти никогда не удается. Они летят вперед. И именно
поэтому можно назвать мертвую компанию мертвой задолго до того, как у
нее дебет не сойдется с кредитом. Компания вполне может приносить
доход еще пять или даже десять лет.

Мне приписывают различные постыдные мотивы, побудившие меня сказать,
что Microsoft мертва: что я-де хотел получить побольше ссылок на свой
сайт, или, что публично высмеивая Microsoft, я надеялся обратить их
внимание на стартапы Y Combinator (нет, я не настолько плох в
продажах). На самом деле, мой постыдный мотив в другом. Я хотел быть
первым, кто скажет об этом. Но мой поступок сопряжен с некоторым
риском. Тем, кто хочет быть первым, лучше бы не ошибаться. Если
окажется, что монстр живее всех живых - если они изменят себя так, что
стартапам придется вновь беспокоиться о Microsoft - я буду выглядеть
дураком. Но я готов рискнуть.

1. Y Combinator - это венчурный микроинвестор, предлагающий стартапам
несколько тысяч долларов в обмен на миноритарную долю в бизнесе. -
Прим. ред. [вернуться]

2. Чтобы делать несовместимое ПО, необязательно прилагать сознательное
усилие. Все, что от вас требуется, - не очень упорно работать над
исправлением ошибок (которых хватает в продуктах любой большой
компании). Нечто подобное происходит с теоретиками литературы. Они
ведь не стараются специально выражаться туманно. Они всего лишь не
пытаются выражать свои мысли ясно. Но это не окупается. [вернуться]

3. В частности, потому, что Джон Скалли выгнал Стива Джобса (такое,
опять же, случается в технологических компаниях нечасто). Если бы
совет директоров Apple не допустил этого, то и расхлебывать все
остальное им бы тоже не пришлось. [вернуться]

4. Имеется в виду, что вместо покупки людей Microsoft может покупать
компании, в которых работают интересующие ее люди. - Прим. ред.
[вернуться]

5. Facebook так пока никто и не купил, но в прошлом году ходили
упорные слухи о продаже сервиса. Правда, основным покупателем
считается вовсе не Microsoft, а Yahoo, которая готова была выложить за
Facebook до полутора миллиардов долларов. Нынешних владельцев Facebook
предложенная сумма не устроила.

\end{document}
