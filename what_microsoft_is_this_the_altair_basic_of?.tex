\documentclass[ebook,12pt,oneside,openany]{memoir}
\usepackage[utf8x]{inputenc} \usepackage[russian]{babel}
\usepackage[papersize={90mm,120mm}, margin=2mm]{geometry}
\sloppy
\usepackage{url} \title{Что для Microsoft значил Altair BASIC?}
\author{Пол Грэм} \date{}
\begin{document}
\maketitle

Внимательное изучение опыта самых успешных компаний и раскрытие причин
их начального успеха является, пожалуй, самым ценным упражнением,
которое вы могли бы попробовать для понятия (философии) стартапов.
Практически каждая из таких компаний, на первый взгляд, казалась
ущербной. Не просто маленькой, а ущербной. Самое начало их
деятельности больше походит, отнюдь, не на первый шажок к вершине, а
на робкое прощупывание первой болотистой кочки непроходимой топи.

Интерпретатор языка программирования BASIC для Альтаир? Как подобное
можно представить себе началом многомиллиардной компании? Сдача в
аренду трех койко-мест в чужой съемной квартире? Студенческий сайт для
слежки друг за другом? Одноплатный компьютер для любителей,
использующих телевизор в качестве монитора? Еще один новый поисковик
при существовавшем десятке подобных и общей недооцененности самой идеи
поиска? Эти идеи не просто казались незначительными, они не вызывали
никакого доверия. Ими можно было не просто пренебречь, они казались
смехотворными.

Часто основатели сами не знают, почему их идеи становятся столь
перспективными. Такие идеи приходят непроизвольно, потому что их
создатели живут будущим и чувствуют, когда в этом мире чего-то не
хватает. Правда сами первооткрыватели не могут даже выразить словами,
как подобные гадкие утята превращаются в прекрасных лебедей.

Услышав подобный бред, большинство людей испытывают непреодолимое
желание рассмеяться. Каждый считает себя «сами с усами».

Когда я сталкиваюсь со стартапом в основе которого лежит какая-то
бредовая, на первый взгляд, идея, я спрашиваю себя: «А непохоже ли
это, случайно, на тот самый майкрософтовский интерпретатор для
Альтаира?» Бремя решения этой головоломки лежит на мне. Поразительно,
насколько часто удается найти решение, хотя иногда я не вижу его,
особенно, когда идея уникальна. Часто бывает, что один из учредителей
еще не понимает своей удачи.

Интересно, что иногда решение многозначно. Несколько дней назад я
разговаривал с основателями стартапа, который может перерасти в три
разных компании типа Microsoft. Они, вероятно, различаются по размеру
на несколько порядков. Никогда нельзя точно предсказать, насколько
велика будет новая Microsoft, в таких случаях я призываю учредителей
следовать своим собственным, интересным только для них, путем. Ведь их
инстиктивное видение мира может завести их очень далеко. Зачем
останавливаться сейчас?

\end{document}
