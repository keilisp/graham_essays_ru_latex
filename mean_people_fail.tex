\documentclass[ebook,12pt,oneside,openany]{memoir}
\usepackage[utf8x]{inputenc} \usepackage[russian]{babel}
\usepackage[papersize={90mm,120mm}, margin=2mm]{geometry}
\sloppy
\usepackage{url} \title{Почему подлецы проигрывают?} \author{Пол Грэм}
\date{}

\begin{document}
\maketitle

В последнее время меня поражает, как мало подлых людей среди моих
знакомых, добившихся успеха. Исключения, конечно, бывают, но их очень
мало. \newline

Подлость — не редкость. На самом деле, одна из вещей, которую открыл
нам интернет — это насколько подлыми могут быть люди. Несколько
десятков лет назад, только известные люди и профессиональные писатели
публиковали свои мнения. Теперь это может делать каждый, и мы видим
весь «длинный хвост» подлости, которые раньше был скрыт. \newline

И, хотя вокруг явно полно подлецов, их практически нет среди самых
успешных людей, которых я знаю. В чём же тут дело? Существует ли
обратная корреляция между подлостью и успехом? \newline

Частично, дело здесь в предвзятости моей выборки. Я знаю только тех
людей, которые работают в определённых областях: основателей
стартапов, программистов, преподавателей. Я могу поверить, что в
других областях есть успешные подлые люди. Может быть, успешные
менеджеры хедж-фондов — подлецы (я знаю недостаточно, чтобы судить).
Кажется вполне вероятным, что большая часть успешных наркодилеров —
полдецы. Но, по крайней мере, в мире есть большие территории, над
которыми подлецы не властны, и это пространство, по-видимому, растёт. \newline

Моя жена и сооснователь Y Combinator, Джессика — из тех редких людей,
обладающих рентгеновским взглядом на человеческий характер. Быть её
мужем — это как стоять напротив сканнера багажа в аэропорте. Она
пришла в мир стартапов из инвестбанкинга, и она всегда поражалась,
насколько последовательно основатели успешных стартапов оказывались
хорошими людьми, и насколько последовательно плохие люди проваливались
в качестве основателей. \newline

Почему так происходит? Я думаю, здесь есть несколько причин.
Во-первых, подлость отупляет. Вот почему я ненавижу сражения. Вы
никогда не сделаете свою работу наилучшим образом, если боретесь,
потому что борьба не даёт достаточно пространства. Победа — всегда
функция ситуации и вовлечённых в неё людей. Сражения выигрываются не
благодаря мыслям о больших идеях, но за счёт мыслей о хитростях,
которые могут вам помочь в данной конкретной ситуации. И всё же, бои
занимают не меньше времени, чем мысли о реальных проблемах. Что
особенно болезненно для тех, кого заботит, как именно используется их
мозг: а в драке он расходует себя быстро, но безрезультатно, как
машина, колёса которой буксуют на месте. \newline

Стартапы не выигрывают за счёт нападения. Они выигрывают, превосходя.
Конечно, есть исключения, но обычно путь к победе заключается в том,
чтобы мчаться вперед, не останавливаясь для драк. \newline

Ещё одна причина проигрышей фаундеров-подлецов заключается в том, что
они не могут получить лучших людей в свою команду. Они могут лишь
нанять людей, которые будут мириться с ними из-за того, что им нужна
работа. Но у лучших людей есть другие варианты. Подлый человек не
может убедить лучших людей работать на него, если только он не
супер-убедителен. И, хотя команда из лучших людей — сильное подспорье
для любой компании, для стартапов её наличие особенно критично. \newline

В работе существует дополнительная сила: если вы хотите делать великие
вещи, это поспособствует следованию пути хороших дел. Основатели
стартапов, которые, в конечном итоге, становятся самыми богатыми — не
те люди, кто изначально был движим деньгами. Мотивированные жаждой
денег основатели принимают предложение сделки по продаже их компании,
которое на своём пути встречает практически каждый успешный стартап.
Теми, кто продолжает растить свой стартап, вместо того, чтобы продать
его, движет что-то другое. Они могут не говорить этого напрямую, но,
как правило, их цель — улучшить мир. Всё это означает, что у людей,
желающих менять мир к лучшему, есть естественное преимущество. \newline

Самое интересное заключается в том, что стартап — не просто один
случайный вид работы, с обратной корреляцией подлости и успеха. За
таким типом работы будущее. \newline

Большую часть нашей истории слово «успех» было тождественно контролю
над ресурсами, находящимися в недостатке. Кто-то получал его в борьбе,
будь то буквально, как в случае пастырей-кочевников, приводящих
охотников-собирателей в малоплодородные земли, или в переносном
смысле, как в случае Позолоченного Века, когда финансисты
конкурировали друг с другом за сбор железнодорожных монополий. На
протяжении большей части истории, слово «успех» означало успех в играх
с нулевой суммой. И в большинстве из них подлость была не помехой, а,
скорее, преимуществом. \newline

Но теперь такое положение дел меняется. Возрастает значимость игр с
ненулевой суммой. Вы всё чаще выигрываете, не сражаясь за контроль над
скудными ресурсами, а имея новые идеи и создавая что-то новое. \newline

Игры, в которых выигрывают за счёт существования новых идей,
существуют давно. В третьем веке до нашей эры Архимед выиграл именно
так. По крайней мере, выигрывал, до вторжения Римской армии, которая
его убила. Что иллюстрирует причины, по которым происходит эта
перемена: чтобы новые идеи имели значение, необходима определённая
степень общественного порядка. И не только в состоянии войны. Также
нужно предотвратить все виды экономического насилия, которые магнаты
19-го века практиковали в отношении друг друга, а коммунистические
страны — в отношении своих граждан. Люди должны чувствовать, что то,
что они создают, не может быть украдено. \newline

Это всегда было актуально для мыслителей, для думающих людей, поэтому
с них всё и началось. Когда вы думаете об исторических примерах
успешных людей, которые не были безжалостными, вспоминаются
математики, писатели и художники. Самое интересное заключается в том,
что их Modus operandi (образ действия) получает всё большее
распространение. Игры, в которые играют интеллектуалы, всё больше
проникают в реальный мир, и это инвертирует историческую полярность
отношения подлости и успеха. \newline

Так что теперь я действительно рад, что разобрался и перестал думать
об этом вопросе. Джессика и я всегда упорно работали над тем, чтобы
научить наших детей не быть подлыми. Мы терпим шум, беспорядок,
нездоровую еду, но не подлость. И теперь у меня есть и дополнительный
повод для расправы над ней, и дополнительный аргумент, когда я так
поступаю: подлость заставляет людей проигрывать. \newline

\subsection{Примечания}

[1] Я не говорю, что все основатели, которые принимают
предложение о покупке их стартапа, движимы только лишь деньгами.
Скорее что те, кто этого не делают, точно мотивированы не ими. Плюс
можно иметь добродетельные мотивы, движимые деньгами — например,
заботиться о своей семье, или быть открытым для работы над проектами,
которые улучшают мир. \newline

[2] Вряд ли каждый успешный стартап улучшает мир. Но их основатели,
как родители, искренне верят в то, что будет именно так. Успешные
основатели влюблены в свои компании. И, хотя этот вид любви так же
слеп, как любовь людей друг к другу, она подлинная. \newline

[3] Питер Тиль бы указал, что успешные основатели так же обогащаются
от контроля монополий, просто это монополии, которые они создали сами,
а не захватили. И, хотя это во многом верно, это означает большую
перемену в типе личности, которая выигрывает. \newline

[4] Справедливости ради: римляне не хотели убивать Архимеда. Римский
полководец специально приказал, чтобы его пощадили. Но в хаосе его всё
равно убили. \newline

В особенно смутные времена, даже мышление требует контроля над
ограниченными ресурсами, просто из-за жизни в условиях их
ограниченности.

\end{document}
