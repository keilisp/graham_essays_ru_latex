\documentclass[ebook,12pt,oneside,openany]{memoir}
\usepackage[utf8x]{inputenc} \usepackage[russian]{babel}
\usepackage[papersize={90mm,120mm}, margin=2mm]{geometry}
\sloppy
\usepackage{url} \title{Два вида суждений} \author{Пол Грэм} \date{}
\begin{document}
\maketitle

У людей есть два разных способа судить о вас. Иногда корректное
суждение является конечной целью. Но есть второй, более
распространённый тип суждения, в котором такой цели нет. Мы обычно
относим все суждения о нас к первому типу. Возможно, мы жили бы
счастливее, осознав, какое суждение таковым является, а какое нет.

Первый тип суждения, в котором судить о вас – конечная цель, включает
в себя дела в суде, оценки на уроках и большинство соревнований. Такие
суждения, конечно, могут быть ошибочными, но поскольку конечная цель –
судить о вас правильно, обычно есть возможность аппелировать. Если вам
кажется, что вас засудили, вы можете подать протест о том, что к вам
отнеслись нечестно.

Почти все суждения, которые делают о детях – именно этого типа,
поэтому с раннего возраста мы привыкаем думать, что и вообще все
суждения такие же.

Но на самом деле есть второй, более широкий класс суждений, где
суждение о вас – всего лишь средство для чего-то ещё. В эти суждения
входят те, что имеют место при приёме в вузы, найме и решениях об
инвестициях, и, конечно, на свиданиях. Суждения этого вида, на самом
деле, не о вас.

Поставьте себя на место того, кто выбирает игроков в сборную. Ради
простоты предположим, что в игре нет амплуа, и вам нужно выбрать 20
игроков. Будет несколько звёзд, которые явно должны будут быть в
команде, и множество игроков, которые точно нет. Ваше суждение имеет
смысл только в граничных случаях. Представим, что вы «напортачили» и
недооценили 20-го из лучших, из-за чего он не вошёл в команду, а его
место занял 21-й. Всё равно вы собрали хорошую команду. Если у игроков
распределение навыков обычное, то 21-й игрок будет только чуть-чуть
хуже 20-го. Возможно, разница между ними будет меньше погрешности
измерения.

20-й игрок, возможно будет чувствовать себя засуженным. Но вашей целью
тут не было сделать услугу по оценке способностей людей. Целью было
набрать команду, и если разница между 20-м и 21-ми лучшими игроками
меньше погрешности измерения, вы всё равно сделали всё оптимально.

Даже применить слово «нечестно» к такому ошибочному суждению –
неверно. Суждение было нацелено не на то, чтобы дать корректную
оценку, а чтобы выбрать достаточно оптимальный набор.

Что сбивает нас с толку – это кажущаяся позиция силы выбирающего.
Из-за этого он кажется похожим на судью. Если же вы отнесётесь к
судящему о вас как к потребителю, а не судье, то не будете ждать от
него честности. Автор хорошего романа не жаловался бы, что читатели
были с ним нечестными, предпочтя его работе халтуру в яркой обложке.
Возможно, глупыми, но не нечестными.

Воспитание в раннем возрасте вместе с эгоцентризмом заставляют нас
верить, что каждое суждение о нас – именно про нас. На самом деле,
большинство – вовсе нет. Это редкий случай, когда быть менее
эгоистичным значит быть более уверенным в себе. Поймите, как мало
важно большинству людей точно оценить вас. Например, из-за нормального
распределения большинства конкурсантов, совсем не важно точно судить,
хотя как раз в этом случае суждение наиболее сильно. Осознав это, вы
не будете воспринимать отказ так лично.

Весьма любопытно, что менее личное восприятие отказов поможет вам реже
получать отказы. Если, по-вашему, тот, кто судит о вас, будет
стараться делать это точно, вы можете позволить себе быть пассивным.
Но большинство суждений очень подвержены влиянию случайных, внешних
факторов, большинство людей, судящих о вас – скорее похожи на
переменчивых покупателей романов, чем на мудрых и проницательных
судей. Чем больше вы это осознаёте, тем больше вы можете сделать,
чтобы повлиять на результат.

Хорошее место для применения этого принципа – поступление в вуз.
Большинство школьников, подающих документы в вузы, делают это с
обычной детской смесью чувства собственной ничтожности и эгоцентризма:
ничтожности – в том, что они считают, что приёмные комиссии –
всевидящие; эгоцентризма – в том, что предполагают, что комиссии
интересуются ими настолько, чтобы копаться в приёмных документах и
выяснять, хорош ли абитуриент. Вместе всё это заставляет абитуриентов
быть пассивными при подаче документов и чувствовать себя уязвлёнными в
случае отказа. Если бы они понимали, насколько быстр и безличен
процесс выбора, то приложили бы больше усилий, чтобы подать себя, и
приняли бы отказ не так лично.

\end{document}
