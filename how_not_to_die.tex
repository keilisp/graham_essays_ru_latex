\documentclass[ebook,12pt,oneside,openany]{memoir}
\usepackage[utf8x]{inputenc} \usepackage[russian]{babel}
\usepackage[papersize={90mm,120mm}, margin=2mm]{geometry}
\sloppy
\usepackage{url} \title{Как не умереть} \author{Пол Грэм} \date{}
\begin{document}
\maketitle

Несколько дней назад в беседе с журналистом я сказал что мы ожидаем
успеха у трети компаний, в которые вкладываемся. На самом деле я был
консервативен. Я надеюсь что это как минимум половина. Было бы круто
если бы нам удалось достичь 50% доли успеха, правда?

Говоря другими словами, половина из вас умрет. Конечно, это звучит не
очень хорошо. По сути, это похоже на предсказание, когда думаешь об
этом, потому что успех определяется тем, что основатели разбогатеют.
Если половина наших стартапов будут успешными, то половина из вас
разбогатеет, а другая половина не получит ничего.



Если вы просто избежите смерти, то разбогатеете. Это звучит как шутка,
но это хорошо описывает то что происходит в типичном стартапе. И это в
точности описывает то что произошло с нами в Viaweb - мы избегали
смерти до тех пор, пока не разбогатели.

Но это было напряжно. Даже очень. Когда мы посещали Yahoo с
разговорами о нашем поглощении, нам приходилось бросать все и занимать
одну из их переговорных, чтобы уговаривать инвестора, уже
отказывающегося от очередного раунда финансирования, необходимого нам
чтобы выжить. Так что даже в середине пути к богатству мы сражались со
смертью.

Вероятно вы слышали ту цитату про то что удача - это когда возможность
встречается с готовностью. Вот сейчас у вас есть готовность. Работа,
которую вы проделали, привела вас к ситуации, когда вы можете поймать
удачу: вы можете разбогатеть просто не дав своей компании умереть. Это
больше, чем есть у большинства людей. Так давайте поговорим, как же,
собственно, не умереть.

Мы сделали это уже пять раз (видимо имеется ввиду сезон финансирования
в YCombinator - прим. перев.) и видели как несколько стартапов умерли.
Около десяти или около того. Мы точно не знаем что происходит когда
они умирают, потому что они не делают это как-то громко и героически.
В основном они ползают где-то и потом умирают.

Для нас основным сигналом скорой кончины это отсутствие вестей от вас.
Когда мы не слышим от или о стартапе несколько месяцев - это плохой
знак. Если мы отсылаем емейл спрашивая как дела и он остается без
ответа - это уже очень плохо. Пока это стопроцентный предвестник
гибели.

Тогда как стартап регулярно что-то делает и выпускает, а также шлет
нам письма и присутствует на наших встречах, то он вероятно выживет.

Я понимаю, что это звучит наивно, но возможно, эта связь работает в
двух направлениях. Может быть, если вы сделаете так, что мы будем о
вас слышать что-нибудь, то вы не умрете.

Это на самом деле может быть не так наивно, как это звучит. Вы
возможно заметили, что обедая каждый вторник с нами и другими
основателями, сподвигает вас сделать больше, чем вы бы сделали в
противном случае. Так как каждый обед - это мини презентация. Каждый
обед это дедлайн. Таким образом, обыкновенное обязательство регулярно
контактировать с нами, будет заставлять вас делать что-то, иначе вам
будет стыдно сказать что вы не продвинулись с момента предыдущей
встречи.

Если это работает, то это удивительный трюк. Было бы очень круто, если
простой контакт с нами сделает вас богатыми. Это звучит бредово, но
есть неплохой шанс что это работает.

Как вариант можно оставаться на связи с другими нашими стартапами. В
Сан-Франциско сейчас их целое поселение. Если вы туда переедете, то
давление сверстников, заставляющее вас работать усерднее все лето,
продолжит действовать.

Когда стартап погибает, официальная причина этого всегда истекший
запас денег или категорический отказ основателя. Часто это происходит
одновременно. Но я думаю что первопричина обычно в деморализации. Вы
врядли услышите о стартапе, который круглосуточно работает делая дела
и выпуская новые фичи и умирает из-за того, что провайдер отключил их
сервер за неуплату. Стартапы редко умирают в середине кодирования. Так
продолжайте кодировать!

Если так много стартапов деморализуются и терпят крах просто в
процессе ожидания богатства, вы должны понимать что запуск стартапа
может деморализовывать. Это абсолютная правда. Я был там, и поэтому
никогда не пробовал еще раз. Нижний предел в стартапе просто
невероятно низок. Я думаю даже у гугла были моменты когда все казалось
безнадежным.

Знание этого может помочь. Если вы знаете, что порой может быть просто
ужасно, то когда такое случается, вы не подумаете "о Боже, это кошмар,
я сдаюсь". Так происходит со всеми, и если вы подождете, то, вероятно,
станет лучше. Метафора описывающая это состояние, это как минимум
американские горки, но не утопание. Вы не опускаетесь все ниже и ниже,
но есть подьемы после падений.

Другое чувство, кажущееся тревожным, но являющееся нормальным для
стартапа, это ощущение что ничего не работает. Причина по которой вы
это можете ощущать состоит в том, что ваше детище возможно
действительно не заработает. Стартапы почти всегда в первый раз делают
все неправильно. Вероятнее всего вы запустите что-нибудь, что никто не
заметит. Не думайте, что когда такое произойдет, то вы провалились.
Это нормально для стартапов. Но и не сидите сложа руки. Продолжайте
запуски.

Мне нравится совет Пола Buchheit'а стараться сделать то, что по
крайней мере кому-то очень нравится. Пока у вас есть то, от чего
восторгаются всего несколько пользователей - вы на правильном пути.
Это очень хорошо для вашего морального состояния иметь горстку
пользователей, которые по настоящему любят вас, а стартапы держатся на
моральном состоянии. Также это подскажет вам на чем
сконцентрироваться. Что такого в вас что нравится им? Можете вы
сделать еще больше этого? Где вы можете найти еще больше людей которым
нравятся такие штуки? Пока у вас есть ядро любящих вас пользователей,
все что вам нужно делать это расширять его. Это может занять некоторое
время, но если вы продолжаете корпеть, то выиграете в конце концов.
Blogger и Delicious сделали так. Оба потратили годы чтобы стать
успешными. И оба начинали с горстки фанатично настроенных
пользователей. И все что Эвану и Джошуа нужно было делать - это
увеличивать это ядро постепенно. Wufoo сейчас на этой же траектории.

Таким образом, когда вы выпускаете что-либо, что никому не интересно,
посмотрите более внимательно. Там ноль пользователей, которые вас
любят, или все-таки есть какая-то группа таких? Вполне возможно что
будет ноль. В таком случае поправьте продукт и пробуйте еще раз.
Каждый из вас работает в области, в которой есть как минимум одно
выигрышное изменение. Если вы просто продолжите ваши старания, то вы
найдете его.

Позвольте мне упомянуть о некоторых вещах, которые не стоит делать.
Вещь номер один которую не стоит делать - это делать другие вещи. Если
вы обнаружите себя говорящим фразу заканчивающуюся на "… но мы
продолжим работать над стартапом", то у вас крупные неприятности. Боб
собирается закончить институт, но мы продолжим работу над стартапом.
Мы возвращаемся в Миннесоту, но мы продолжим работу над стартапом. Мы
возьмем несколько проектов для консалтинга, но мы продолжим работу над
стартапом. С таким же успехом вы можете просто перевести эту фразу как
"мы закрываем стартап, но еще не в силах себе в этом признаться",
потому что как правило, имеется ввиду именно это. Стартап настолько
тяжел, что работа над ним не может начинаться с каких-то "но".

В частности, не идите в магистратуру, и не начинайте других проектов.
Отвлечение фатально для стартапов. Идя (или возвращаясь) в институт -
надежнейший предвестник гибели, потому что вдобавок к отвлечению, он
позволяет вам сказать что вы чем-то заняты. Если вы делаете только
стартап, то когда он терпит крах, вы терпите крах. Если вы в
магистратуре и ваш стартап гибнет, вы можете позже сказать "Да, у нас
был этот стартап во времена магистратуры, но ничего не получилось."

Вы не можете использовать эфимизмы типа "ничего не получилось" для
чего то, что является вашим единственным занятием. Люди вам не
позволят.

Одна из самых любопытных вещей обнаруженных нами в нашем инкубаторе,
это то что основатели больше мотивируются страхом выглядеть плохо, чем
надеждой заполучить миллионы долларов. Таким образом если вы хотите
получить миллионы, поставьте себя в положение, где провал будет
публичным и оскорбительным.

Когда мы первый раз встретились с основателями Octopart, они казались
очень умными, но с не слишком высокими шансами на успех, так как не
казались сильно заинтересованными. Один из них был еще в магистратуре.
Это была обычная история: он бы бросил если бы увидел что стартап
полетел. С тех пор он не только бросил институт, но и появился в
полный рост в Ньюсвике со словом "миллиардер" на всю грудь. Теперь он
просто не может провалиться. Все кого он знает видели эту картинку.
Девушки, которые его отверги, тоже видели ее. Его мама возможно
повесила ее на холодильник. Это невыносимо унизительно провалиться
теперь. После этого момента он был готов бороться на смерть.

Я желаю каждому нашему стартапу повиться в статье Ньюсвик, где он
описывается как следующее поколение миллиардеров, потому что тогда
никто из них не сможет просто так сдаться. Вероятность успеха
подскочила бы до 90%. Я не шучу.

Когда мы познакомились с ребятами из Octopart, они были беззаботными
веселыми парнями. Теперь когда мы говорим с ними, они выглядят сурово
решительными. Распространители электродеталей пытаются раздавить их,
чтобы сохранить свою монополию на ценообразование. (Если это кажется
странным для вас, что люди заказывают детали из бумажных каталогов в
2007, то на это есть причина. Распространители хотят избежать
прозрачности, которая наступит, если цены будут онлайн). Мне не очень
нравится то, что мы превратили веселых ребят в сурово решительных. Но
такое происходит в нашей области. Если стартап выигрывает, вы
получаете миллионы долларов. Но вы не получаете таких денег просто
попросив о них. Нужно понимать что это может быть весьма болезненно.

И хотя жесткие вещи происходят с Octopart, я думаю они выиграют.
Возможно им прийдется трансформироваться во что-то совсем другое, но
просто так они не отползут умирать. Они умные, они работают на
многообещающем поприще, и они не могут просто так сдаться.

Все из вас тоже имеют две этих составляющих: вы все умные и работаете
над многообещающими идеями. И закончите ли вы среди живых, или гибель
будет вашим третьим ингридиентом, не сдавайтесь.

И теперь я говорю вам: грядет полное дерьмо. Оно всегда в стартапе.
Случайность, что сразу после запуска вы становитесь ликвидными без
каких-либо бедствий бывает раз на тысячу. Поэтому не теряйте духа. И
когда случится беда, просто скажите себе: окей, это то о чем Пол
предупреждал. И что он сказал делать? А, да. Не сдаваться.

\end{document}
