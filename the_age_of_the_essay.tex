\documentclass[ebook,12pt,oneside,openany]{memoir}
\usepackage[utf8x]{inputenc} \usepackage[russian]{babel}
\usepackage[papersize={90mm,120mm}, margin=2mm]{geometry} \sloppy
\usepackage{url} \title{Век Сочинения} \author{Пол Грэм} \date{}
\begin{document}
\maketitle

Помните, какие сочинения [прим. перев. 2] вам приходилось писать в
школе? Тема, вводный параграф, тезисы, заключение. Заключение
утверждало, к примеру, что герой Ахаб в романе ``Моби Дик``
уподоблялся Христу. \newline

Н-да. Так вот, я намереваюсь описать другую сторону медали: что такое
на самом деле сочинение, и как его следует писать. Или, по меньшей
мере, как лично я пишу мои эссе. \newline

\subsection{Различия}

Самое очевидное отличие настоящих эссе от школьных сочинений в том,
что тематика истинных эссе не ограничивается литературой. Конечно,
школа должна обучать письму. Однако вследствие ряда исторических
случайностей обучение письменной речи стало неразрывно связано с
изучением художественной литературы. И теперь школьники во всех концах
страны пишут не о том, как бейсбольная команда на маленьком бюджете
могла бы соревноваться с лидирующей командой ``Янки``, или о роли
цвета в моде, или о составляющих вкусного десерта, а о символизме в
романах Дикенса. \newline

В результате сама идея написания сочинений выглядит занудной и
бессмысленной. Кого волнует вопрос символизма у Дикенса? Самому
Дикенсу было бы интереснее читать сочинение о цвете или о бейсболе.
\newline

Как же дело к этому пришло? Для ответа нам придется вернуться почти на
тысячу лет назад. Около 1100 года Европа наконец-то начала переводить
дух после столетий хаоса, и как только люди смогли себе позволить
роскошь любопытства, они открыли для себя то, что мы называем античным
классицизмом. Эффект был подобен приземлению инопланетян. Те ранние
цивилизации были настолько более высоко развиты, что на протяжении
нескольких последующих веков труды европейских ученых складывались в
основном из переваривания классических познаний. \newline

В этот период изучение античных источников приобрело высокий престиж;
это занятие определяло сущность научного труда. По мере развития
европейской научной мысли важность этого подхода постепенно снизилась.
К 1350 году желающие изучать науки могли найти лучших
учителей-современников, чем Аристотель был в свою эпоху. [1] Однако
учебные заведения меняются медленнее, чем научные круги. В 19 веке
изучение античных источников все еще было основной составляющей
учебной программы. \newline

Назревал естественный вопрос: если изучение античных текстов является
уважаемым направлением научных исследований, то почему бы не изучать и
современную литературу? Ответ имеется: корни исследования классицизма
шли от этакой интеллектуальной археологии, в которой не было
необходимости в случае современных авторов. По понятным причинам никто
не желал дать такой ответ, ведущий к неприятным выводам: раз
археологические труды были почти завершены, то само по себе изучение
классицизма являлось если не напрасной тратой времени, то уж точно
работой над проблемами низкой значимости. \newline

И так началось изучение современной литературы. На первых порах ему
оказывали немалое сопротивление. Первые университетские курсы по
английской литературе были введены в вузах новой волны, в особенности
в США. Дартмут, университет Вермонта, Амхерст и университетский
колледж в Лондоне преподавали английскую литературу в 1820-х годах.
\newline

Но в Гарварде не было профессора английской литературы до 1876 года, а
в Оксфорде - до 1885. (В Оксфорде кафедра китайского языка появилась
раньше, чем английского!) [2] \newline

Перемена общественного мнения, по меньшей мере в США, произошла под
влиянием идеи о том, что университетские профессора должны вести
научные исследования вдобавок к преподаванию. Эта идея (вместе с
концепцией докторской степени, кафедры и вообще устройства
современного университета) была заимствована из Германии в конце 19
века. [прим. перев. 3] Начиная с университета Джона Хопкинса в 1876
году, новая модель вуза быстро приобрела распространение. \newline

Одной из жертв новой системы стало сочинительство. Вузы с давних пор
обучали навыкам письменной речи. Но как проводить исследования по
письму? Профессора, преподававшие математику, могли разрабатывать
новые темы в математической науке, профессора истории могли заняться
написанием научных работ по истории, но что было делать профессорам
реторики и письма? Что им следует исследовать? Самой подходящей
областью казалась английская литература. [3] \newline

Таким образом в конце 19 века профессора английского языка и
литературы унаследовали обязанности обучения студентов письменной
речи. Недостатков в этом было два: (а) эксперту в области литературы
совсем не обязательно быть хорошим писателем, равно как и специалисту
по истории искусств не требуется быть хорошим художником, и (б)
тематика сочинений теперь тяготела к художественной литературе,
поскольку она была основным предметом интереса профессоров. \newline

Школы подражают университетам. Семена наших страданий над сочинениями
в старших классах были посеяны в 1892 году, когда Ассоциация по
Национальному Образованию [США] ``дала формальную рекомендацию
объединения литературы и письма в одном школьном предмете для старших
классов``. [4] Обучение письменной речи влилось в предмет
``английского языка`` со странным побочным эффектом: старшеклассникам
теперь приходилось писать сочинения об английской литературе,
имитируя, сами того не подозревая, журнальные публикации профессоров
английского языка давностью в несколько десятилетий. \newline

Неудивительно, что сочинения кажутся школьникам бесполезным занятием,
ведь теперь мы трижды удалены от реальной работы: студенты имитируют
профессоров английского языка, которые имитируют ученых-классицистов,
которые в свою очередь являлись наследниками традиции, основанной на
трудах, какие были востребованы и захватывающе интересны... 700 лет
назад.

\subsection{Не аргумент}

Другое крупное различие между настоящим эссе и тем, что нам
приходилось писать в школе, состоит в том, что эссе в подлинном смысле
слова не выбирает позицию и не защищает ее. Принцип выбора сторон и
аргументации своего выбора, как и идея написания сочинений на
литературные темы, оказывается очередным пережитком с давно забытыми
корнями. \newline

Люди часто ошибочно полагают, что средневековые университеты были по
большей части духовными семинариями. На самом деле они больше походили
на школы юриспруденции. И традиционно юристы обучаются как адвокаты,
готовые принять любую сторону в споре и выстроить для нее серьезную
защиту. Было ли это причиной или следствием, теперь неважно, но этот
подход стал основополагающим в ранних университетах. Изучение
реторики, искусства спора и аргументации, составляло треть программы
для студентов. [5] После лекций обсуждение проходило чаще всего в
форме диспута, что сохранено по сей день, хотя бы номинально, в виде
защиты диссертации. Большинство англоязычных людей используют слова
``тезис`` и ``диссертация`` как взаимозаменяющие, но первоначально
тезис представлял собой выбранную позицию, а диссертация включала в
себя аргументы в пользу этого тезиса. \newline

Защита позиции может быть необходима в легальном диспуте, но это
далеко не лучший способ выяснить правду, в чем юристы, я уверен, со
мною согласятся. Истинная проблема состоит в том, что при этом подходе
невозможно изменить сущность вопроса. \newline

Тем не менее вокруг этого принципа построена вся идея школьных
сочинений. Вводное предложение - это ваш тезис, заранее выбранная
позиция по отношению к теме; каждый аргументирующий параграф - удар по
противнику; а заключение - хмм, что же такое заключение? У меня на
этот счете не было уверенности еще в школе. Вроде бы от нас ожидали
повторения того, что мы уже сказали в первом параграфе, но другими
словами, чтобы никто не догадался. Ради чего, казалось бы? Но если
понять происхождение такого рода ``сочинений``, то становится ясно,
зачем нужно заключение. Это ваше заключительное обращение к жюри
присяжных. \newline

Хорошая проза должна быть убедительной, бесспорно, но убедительной
оттого, что вы нашли правильные ответы на вопросы, а не благодаря
хитрой аргументации. Когда я прошу друзей взглянуть на черновик эссе,
мне интересны два момента: какие части им скучно читать, и какие
выглядят неубедительно. Скучные отрывки обычно можно поправить путем
удаления. Но я не пытаюсь выправить неубедительные куски, добавляя
хитроумные аргументы; я их переосмысливаю в разговоре. \newline

Должно быть, по меньшей мере, я что-то плохо объяснил. Если так, то
разговор меня вынудит найти более ясное объяснение, которое я потом
могу включить в эссе; чаще всего мне приходится менять не только
формулировку, но и смысл. Быть убедительным для меня не самоцель.
Умного читателя убедит только правда, так что если мои слова убеждают
интеллектуальную аудиторию, значит, я близок к правде. \newline

Проза, задающаяся целью убедить читателя, вполне может быть достойной
(или хотя бы неизбежной) литературной формой, но приравнивать такую
прозу к эссе было бы исторически неверно. Эссе - это нечто иное.

\subsection{Попытка}

Чтобы понять, что такое истинное эссе, нам вновь придется обратиться к
истории, хотя на сей раз не так далеко в прошлое - к Мишелю Монтень,
опубликовавшему в 1580 году собрание работ, названных им ``эссе``
[прим. перев. 4] . Его труды резко отличались от того, чем занимаются
юристы, и это отличие отражено в термине. Французский глагол
``essayer`` означает ``пытаться``, а ``essai`` - это попытка. Эссе
пишется ради того, чтобы попытаться что-то понять. \newline

Что понять? Этого вы пока не знаете, а потому не можете начать с
тезиса, ибо его у вас нет и может никогда не быть. Эссе начинается не
с законченной мысли, а с вопроса. В истинном эссе вы не станете
выбирать позицию, а потом ее защищать. Вы видите приотворенную дверь,
распахиваете ее и входите, дабы увидеть, что же там за нею. \newline

Если ваша задача - в чем-то разобраться, зачем вообще нужно что-то
писать? Почему бы просто не сесть и не подумать? Ага, в том-то и
заключается великое открытие Монтаньи. Выражая идеи, вы помогаете им
формироваться. Не то слово - помогаете; подавляющая часть моих эссе
состоит из мыслей, пришедших мне в голову после того, как я начал
писать эссе. Именно поэтому я их и пишу. \newline

В школьных сочинениях, в теории, вы всего лишь объясняете свои идеи
читателю. В подлинном эссе вы пишете для себя, вы думаете вслух.
\newline

Но это не совсем так. Подобно тому, как ожидание гостей мотивирует вас
прибрать в доме, написание чего-то, что другие люди будут читать,
заставляет вас хорошо подумать. Получается, аудитория важна. Когда я
пишу исключительно для себя, получается не очень-то хорошо. Такие
сочинения чахнут. Если я сталкиваюсь с затруднениями, то приписываю
пару туманных вопросов, бросаю эту затею и иду пить чай. \newline

Многие опубликованные эссе точно так же заканчиваются ничем; особенно
те, что пишутся штатными сотрудниками публицистических журналов.
Нештатные писатели обычно поставляют статьи с ярко выраженным мнением,
где защищают избранную позицию и быстро приходят к возбуждающему
читательский интерес (и предсказуемому) заключению. Штатных же
сотрудников положение обязывает производить нечто
``сбалансированное``. Раз они пишут для популярного журнала, то
начинают свои статьи с наиболее противоречивых, взрывчатоопасных
вопросов, от которых (поскольку они пишут для популярного журнала!)
тут же в ужасе отпрянут. Аборты: за или против? Эта группа говорит
одно. Та утверждает другое. Ясно одно: это очень непростой вопрос. (Но
не злитесь на нас - мы же никаких выводов не сделали.) \newline

\subsection{Река}

Одной постановки вопроса недостаточно - в эссе следует предложить
ответы. Конечно, это не всегда происходит; иногда начинаешь с
многообещающего вопроса, но ни к чему не приходишь. Только подобные
эссе не надо публиковать, они как эксперименты, не принесшие значимых
результатов. Опубликованное эссе должно сообщить читателю что-то
такое, чего он сам не знал. \newline

А вот что конкретно вы сообщите читателю - это неважно, только бы было
интересным. Мои эссе иногда обвиняют в бессвязности. В прозе с целью
защиты избранной позиции это было бы изъяном. Там вы не занимаетесь
правдоискательством, вы уже знаете, куда направляетесь, и хотите
прибыть прямо к месту назначения, сминая препятствия на своем пути и
перебираясь через болота на авось. Но в эссе у вас иная задача, эссе
как раз должно заключать в себе поиски правды. Полное отсутствие
бессвязности было бы подозрительным. \newline

[Прим. перев.: в английском языке идея бессвязности, перескакивания с
одной темы на другую, а также извилистости, выражается глаголом
``meander``.] ``Meander`` (другое название Мендерес) - это река в
Турции. Как можно того ожидать, она изгибается как только можно. Но
это не потехи ради; русло этой реки представляет собой наиболее
экономичный путь к морю. [6] \newline

Алгоритм течения реки весьма прост: шаг за шагом, течь вниз. Для
писателя-эссеиста это означает ``течь интересно``. Из всех возможных
направлений мысли следует выбирать самое интересное. Да, человек - не
река, ему довольно трудно совсем не заглядывать вперед; я всегда знаю
в общем и целом, о чем хочу написать. Однако конкретные выводы мне
заранее не известны, я пускаю идеи на самотек и позволяю им принять
нужную форму от одного параграфа к другому. \newline

Такой подход не всегда срабатывает. Иногда можно наткнуться на стену -
и с реками такое бывает. Тогда я поступаю точно так же, как река:
поворачиваю вспять. При написании этого эссе в какой-то момент я
обнаружил, что в текущем направлении у меня больше не осталось идей.
Мне пришлось вернуться на семь параграфов назад и начать снова в ином
русле. \newline

В своей основе эссе - это поток мысли, но очищенный от мусора, так же
как и литературный диалог - подчищенная разговорная речь. Настоящая
мысль, как настоящая беседа, полна фальшстартов, и читать ее в
подлиннике было бы утомительно. Автор должен вырезать излишки и
заполнить пробелы в выражении мысли, как иллюстратор, раскрашивающий
карандашный набросок. Но не меняйте слишком многое, чтобы не потерять
спонтанность оригинала. \newline

Если баланса не достичь, то пусть лучше река перевесит. Эссе - это не
энциклопедия, которую читатель листает в поисках специфического ответа
и чувствует себя обманутым, если ответ не найден. Я с гораздо большим
удовольствием прочту эссе, которое поток мысли унес в неожиданном, но
интересном направлении, чем сочинение, покорно прошествовавшее по
заранее заданному маршруту. \newline

\subsection{Сюрприз}

Что людям интересно? Мне интересно то, что неожиданно.
Пользовательские интерфейсы, по словам Джеффри Джеймса, должны
придерживаться принципа наименьшей неожиданности. Если кнопка выглядит
так, будто ее нажатие остановит машину, то пусть это будет кнопка для
остановки, а не ускорения. В эссе следует добиваться противоположного
эффекта - чем больше сюрпризов, тем лучше. \newline

Я долгое время боялся летать самолетом и путешествовал только
косвенно, через чужой опыт. Когда друзья возвращались из поездок в
дальние края, я их расспрашивал об увиденном не только из вежливости,
я действительно хотел знать. И я обнаружил, что лучший способ получить
от них сведения - спросить о том, что их удивило. Что в чужих местах
не соответствовало их ожиданиям? Это исключительно полезный вопрос.
Его можно задать самым ненаблюдательным людям, и в ответ они выдадут
информацию, о владении которой они сами не подозревали. \newline

Сюрпризы - это не просто вещи, о которых вы не знали, они противоречат
вашим текущим (возможно, неверным) представлениям. Таким образом,
сюрпризы - самые что ни на есть полезные из всех познаваемых фактов.
Они как еда, которая не только является здоровой пищей, но и выступает
противоядием ко всем уже съеденным нездоровым блюдам. \newline

Как можно обнаружить сюрпризы? Ааа, в этом кроется половина труда по
написанию эссе. (Вторая половина - толковое выражение своих мыслей.)
Весь секрет в том, чтобы поставить себя на место читателя. Пишите
только о предметах, которые вы хорошо обдумали. И если какая-то идея
удивит вас, то есть человека, хорошо знакомого с темой, то и для
большинства читателей эта идея станет сюрпризом. \newline

К примеру, в недавнем эссе я отметил, что качество компьютерных
программистов можно оценить только в процессе работы с ними, и
следовательно, никто не знает, кто же является наилучшими из всех
программистов. Я сам этого не осознавал, когда начал писать то эссе, и
даже теперь нахожу идею немного странной. Это именно то, что вам
нужно. Для написания эссе требуются два ингредиента: несколько тем,
над которыми вы как следует подумали, и определенная способность
вынюхивать неожиданности. \newline

О каких вопросах можно думать? Я бы сказал, это не имеет значения,
ведь что угодно может быть интересным, если внимательно посмотреть.
Возможное исключение - предметы, из которых нарочно изъяли все
отклонения; например, работа в ресторанах быстрого питания.
Оглядываясь назад, было ли что-нибудь интересное в моей работе в
мороженице Баскин-Роббинс? Ну, важность цвета для покупателей была
любопытным наблюдением. Дети определенного возраста зачастую покажут
пальцем на контейнер с мороженым и объявят, что они хотят желтое.
Какое именно - французская ваниль или лимон? Они на вас просто тупо
посмотрят. Они хотели желтое. А еще помню загадку непреходящей
привлекательности сорта ``пралине со сливками``. (Теперь я думаю, что
весь секрет был в соли.) И разница между поведением отцов и матерей,
покупающих мороженое детям: отцы вели себя как благосклонные короли,
проявляющие щедрость к подданным, а матери измотанно поддавались
давлению со стороны отпрысков. Видите, даже в ресторанах быстрого
питания есть пища для ума. \newline

Надо сказать, что я не заметил всех этих особенностей в то время,
когда события происходили. В возрасте шестнадцати лет я был примерно
так же наблюдателен, как валун. Теперь я больше могу увидеть в
обрывках воспоминаний о тех временах, чем в сами моменты, когда
события случались вживую у меня на глазах. \newline

\subsection{Наблюдение}

Значит, способность вынюхивать неожиданное не является врожденной,
этому можно обучиться. Но как? \newline

В какой-то степени это ничем не отличается от изучения истории. Когда
впервые читаешь учебник истории, это лишь вихрь имен и дат. Ничего не
запоминается. Но чем больше узнаешь, тем больше возникает крючков,
которыми факты друг за друга цепляются, и в итоге объем познаний
возрастает, как говорится, экспоненциально. Если помнишь, что
норманские племена покорили Англию в 1066 году, то заинтересуешься,
услышав, что другие норманны покорили южную Италию приблизительно в
тот же период. А тогда задумаешься о Нормандии и сделаешь заметку,
узнав из другой книги, что норманны на территории современной Франции
были не из тех племен, что пришли туда после распада Римской империи,
а из викингов (``норманн`` = ``северный человек``), которые туда
прибыли четырьмя столетиями позже в 911 году. Если знаешь это, то
легче запомнить, что город Дублин тоже был основан викингами в 840-х
годах. И так далее, и тому подобное, возведенное в квадрат. \newline

Накопление сюрпризов происходит подобным же образом. Чем больше
аномалий повидал, тем проще заметить новые. Что означает, как ни
странно, что с возрастом жизнь должна становиться все более
удивительной. Когда я был маленьким, то думал, что взрослым известны
ответы на любые вопросы. Получается наоборот - это детям известны все
ответы! Только они все неправильные. \newline

В отношении сюрпризов богатые продолжают богатеть. Но (как и в
отношении богатства) существуют умственные привычки, помогающие
осуществлению процесса. Полезно иметь привычку задавать вопросы,
особенно те, что начинаются с ``почему``. Но только не случайным
образом, как трехлетние малыши спрашивают ``почему``, ведь вопросам
нет числа. Как отобрать плодотворные? \newline

Лично мне бывает полезно спрашивать о вещах, которые кажутся странными
или неправильными. Например, почему есть связь между юмором и
несчастьем? Почему нам бывает смешно, когда герой (даже тот, что нам
нравится) поскальзывается на банановой кожуре? За этим вопросом
наверняка кроется целое эссе, полное сюрпризов. \newline

Если хочешь замечать странные вещи, в этом помогает здоровая доля
скептицизма. Я принимаю за аксиому то, что мы достигаем всего 1\%
наших возможностей. Такое убеждение выступает противовесом правилу,
которое нам вбивают в голову в детстве: вещи обстоят так, как обстоят,
потому что иначе и быть не может. Например, с кем бы я ни говорил в
процессе написания этого эссе, все в душе полагали, что уроки
английского языка и литературы были бессмысленным занятием. Однако ни
у кого из нас в свое время не хватило духу высказать мысль о том, что
подход к обучению был на самом деле ошибочным. Мы все думали, что
чего-то не допонимали. \newline

Интуиция мне подсказывает, что следует присматриваться к вещам,
которые кажутся не просто странными и неправильными, а странными в
смешной манере. Мне всегда приятно, когда кто-то смеется, читая
черновик эссе. Почему? Моя цель - хорошие идеи. Почему хорошие идеи
должны быть забавными? Возможно, они связаны моментом неожиданности.
Сюрпризы заставляют нас рассмеяться, потому их стоит выискивать.
\newline

Я делаю записи в блокноте о вещах, которые меня удивляют. Получается
так, что я их никогда не перечитываю и напрямую не использую в эссе,
но позже я воспроизвожу те же идеи. Так что ценность блокнотов может
быть в том, что записывание на бумаге оставляет отпечаток мысли и в
мозгу. \newline

Людям, старающимся выглядеть ``круто``, будет сложнее замечать и
накапливать сюрпризы. Удивление подразумевает ошибку. А сущность
крутизны, как вам разъяснит любой четырнадцатилетний подросток, - это
``nil admirari`` (ничему не удивляться). Если допустил ошибку, много
над этим не раздумывай, веди себя как ни в чем не бывало, авось никто
не заметит. \newline

Один из методов достижения крутизны - избежание ситуаций, где велик
шанс глупо выглядеть из-за неопытности. Если хочешь обнаружить
сюрпризы, следует себя вести противоположным образом. Изучайте
множество разных вещей, поскольку самые любопытные неожиданности могут
быть найдены на границе различных областей знаний. К примеру, джем,
бекон, соленья и сыр (весьма вкусные вещи) изначально предназначались
для консервации; то же относится к книгам и картинам. \newline

Что бы вы ни изучали, включите в ваш круг интересов историю - но
социальную и экономическую, а не политическую историю. История
настолько важна, в моем понимании, что было бы неверно с нею
обращаться как всего лишь с отдельной наукой. Лучше ее описывать так:
``все, что мы знаем на данный момент``. \newline

Среди прочего, изучение истории придает уверенность в том, что хорошие
идеи ждут своего открытия прямо перед нашим носом. Мечи получили
развитие в бронзовом веке от ножей, у которых (как и у их
предшественника кремня) рукоятка была отделена от лезвия. Поскольку
мечи длиннее, их рукоятки все время отламывались. Но лишь через
пятьсот лет кого-то осенила идея лить лезвие и рукоятку как единое
целое. \newline

\subsection{Непослушание}

Прежде всего, возьмите себе за правило обращать особое внимание на
вещи, которые вас учили не замечать потому, что либо ``так не
принято``, либо это не считается важным, либо не имеет отношения к
вашей текущей работе. Если вам что-то любопытно, доверьтесь инстинкту.
Разматывайте нити клубка, привлекшие ваше внимание. Если вас что-то
действительно интересует, все нити в итоге приведут вас к интересному
предмету, как разговор людей, которые чем-то особенно гордятся, всегда
вращается вокруг предмета гордости. \newline

К примеру, меня всегда интриговали прически с сильным начесом, в
особенности их экстремальная разновидность, когда человек выглядит
так, будто на нем надет берет из собственных волос. Без сомнений, это
весьма приземленный интерес из той категории, что по праву принадлежит
девочкам-подросткам. И тем не менее за ним что-то стоит. Я понял, что
любопытный вопрос тут такой: неужели владелец начеса не осознает, как
нелепо он выглядит? А ответ в том, что он пришел к такой прическе
постепенно. Началось с того, что он стал чуть-чуть начесывать волосы
для прикрытия небольшой проплешины, и постепенно за 20 лет привычка
начесывания переросла в ужасающую прическу. Постепенность перемен -
очень сильное оружие. И эта сила может быть использована и для
конструктивных целей: точно так же, как можно постепенно, сам того не
замечая, превратиться в потеху публики, можно постепенно, сам того не
замечая, сотворить нечто настолько грандиозное, о чем и не мечтал - на
планирование чего-то подобного духу не хватало. На самом деле так и
создается большая часть хорошего программного обеспечения. Начинаешь с
написания простенького ядра системы (чего в этом сложного?), а оно
постепенно вырастает в полноценную операционную систему. Следующий шаг
в наших размышлениях: возможно ли проделать то же самое в
изобразительном искусстве или написании романа? \newline

Видите, что можно извлечь из, казалось бы, пустякового вопроса? Если
бы меня попросили дать единственный совет авторам эссе, вот он: не
следуйте чужим указаниям. Не верьте тому, во что вас обучили верить.
Не пишите в ваших эссе то, чего читатель ожидает; от ожидаемого ничему
нельзя научиться. И еще одно: не пишите в такой манере, как вас в
школе учили. \newline

Наиважнейшая разновидность непослушания - это само по себе написание
эссе. К счастью, этот подвид неповиновения буйно разрастается, судя по
всему. Раньше лишь небольшое количество официально признанных
писателей были вправе сочинять эссе. Некоторые работы публиковались в
журналах и оценивались не столько по содержанию, сколько по
известности автора; журнал мог напечатать рассказ малоизвестного
писателя, если литературное качество было на высоте, но для публикации
эссе на тему Х требовалось, чтобы его автору было как минимум 40 лет,
и чтобы наименование его профессиональной должности включало в себя Х.
Что проблематично, поскольку специалист в данной области не все может
высказать вслух именно потому, что он ``инсайдер``. \newline

Интернет изменяет условия игры. Кто угодно может опубликовать свое
эссе на вебсайтах, и расцениваются такие работы, как должно любым
литературным произведениям, по тому, что в них написано, а не по тому,
кто их сочинил. Почему вы вправе писать на тему Х? Да потому, что вас
определяет то, о чем вы пишете; раз написали, значит, вправе! \newline

Популярные журналы превратили период между распространением
грамотности и изобретением телевидения в золотой век рассказа как
литературного жанра. Интернет вполне в состоянии сделать наше время
золотым веком эссе. Кстати, этого я точно не осознавал, когда начал
работать над данным эссе. \newline

\subsection{Примечания автора}

[1] Я имею в виду Оразма (жил между 1323-82 гг.). Но выбрать точную
дату нелегко, т.к. объем научных исследований сильно сократился как
раз в то время, когда европейцы закончили переосмысление классической
науки. Причиной могла послужить эпидемия чумы 1347 года; тренды
научного прогресса следуют кривой роста населения. \newline

[2] Вильям Р. Паркер. ``Откуда взялись институтские кафедры
английского языка?`` Опубликовано на английском языке в источнике:
Parker, William R. ``Where Do College English Departments Come From?``
College English 28 (1966-67), pp. 339-351. Reprinted in Gray, Donald
J. (ed). The Department of English at Indiana University Bloomington
1868-1970. Indiana University Publications. \newline

Роберт В. Даниэлс. Университет Вермонта: Первые двести лет.
Опубликовано на английском языке в источнике: Daniels, Robert V. The
University of Vermont: The First Two Hundred Years. University of
Vermont, 1991. \newline

Фридрих М. Мюллер. Письмо в газету Полл Молл. Опубликовано на
английском языке в источнике: Mueller, Friedrich M. Letter to the Pall
Mall Gazette. 1886/87. Reprinted in Bacon, Alan (ed). The
Nineteenth-Century History of English Studies. Ashgate, 1998. \newline

[3] Я слегка ужимаю ход истории. Сперва место литературы было занято
филологией, которая (а) более солидно выглядела как наука, и (б) была
популярна в Германии, где обучались многие ведущие ученые того
поколения. \newline

В некоторых случаях учителя письма были in situ (прямо на месте)
произведены в профессоров английского языка. Фрэнсис Джеймс Чайлд,
служивший профессором реторики в Гарварде с 1851 года, стал в 1876
году первым профессором английского языка в истории университета.
\newline

[4] Паркер, стр. 25 в вышеописанном источнике [2]. \newline

[5] Общеобразовательная программа обучения студентов, или ``trivium``
(``собрание трех`` - отсюда слово ``тривиальный``), включала в себя
грамматику латыни, реторику и логику. Аспиранты (кандидаты на степень
магистра) изучали сверх того ``quadrivium`` (``собрание четырех``)
арифметики, геометрии, музыки и астрономии. Вместе взятые, эти
предметы составляли семь гуманитарны наук. \newline

Изучение реторики было напрямую унаследовано из Рима, где считалось
самым важным предметом. Думаю, мы не ошибемся, преположив, что общее
образование в классических цивилизациях заключалось в обучении сыновей
землевладельцев искусству ораторства, чтобы они могли защитить свои
интересы в политических и юридических диспутах. \newline

[6] Тревор Блэквелл отмечает, что это не совсем так, поскольку внешние
края изгибов [русла реки] больше подвержены эрозии. \newline

\subsection{Примечания переводчика}

[прим. перев. 1] Пол Грэм - известный "хакер" (т.е. отличный
программист, любящий свою работу), а также художник и писатель. Его
публицистические статьи (эссе) опубликованы на его личном вебсайте, а
также были недавно изданы отдельной книгой "Хакеры и художники" (Paul
Graham. Hackers \& Painters: Big Ideas from the Computer Age), которая
на данный момент - сентябрь 2004 года - не переведена на русский язык.
Краткая биография автора (источник здесь): Пол Грэм сейчас работает
над новым языком программирования под названием Арк (Arc). В 1995 году
он совместно с Робертом Моррисом разработал первое веб-приложение
(т.е. программное обеспечение, функционирующее полностью на сервере и
доступное пользователям через их веб-браузер без необходимости
загружать какое-либо ПО на их компьютеры). Их фирма и само ПО
назывались Виавеб (Viaweb) и были куплены фирмой Яху (Yahoo) в 1998 г.
В 2002 году Пол Грэм выпустил описание простого, но эффективного
фильтра электронной почты для удаления спама (т.е. невостребованной
рекламной почты), основанного на байесовой теории вероятности.
Современные фильтры спама основаны на этой идее. \newline

[прим. перев. 2] Слова "сочинение" и "эссе" в моем переводе являются
практически синонимами, хотя я чаще употребляю термин "сочинение" для
принудительных школьных работ, а "эссе" - для добровольных творческих
изысков. Автор использует английское слово "эссе" во всех случаях,
включая обсуждение школьных сочинений. \newline

[прим. перев. 3] Материал эссе основан на истории высшего образования
в США, но прекрасно подходит и к российским вузам, поскольку система
высшего образования в России была также построена на немецкой модели.
\newline

[прим. перев. 4] В русском переводе этот труд известен как "Опыты".
Спасибо Лиане за уточнение.

\end{document}
