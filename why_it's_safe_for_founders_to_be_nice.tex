\documentclass[ebook,12pt,oneside,openany]{memoir}
\usepackage[utf8x]{inputenc} \usepackage[russian]{babel}
\usepackage[papersize={90mm,120mm}, margin=2mm]{geometry}
\sloppy
\usepackage{url} \title{Почему стартаперу выгодно быть щедрым}
\author{Пол Грэм} \date{}
\begin{document}
\maketitle

Недавно я получил письмо от основателя, которое помогло мне понять
кое-что важное: стартаперам выгодно быть хорошими людьми.

Я вырос с мультяшным образом очень успешного бизнесмена: алчный,
курящий сигары, шумный, властный и не слишком привередливый в
средствах. Как я уже писал раньше, одна из вещей, которая меня больше
всего удивляет в стартапах, заключается в том, что некоторые из самых
успешных создателей приятные люди. Может быть успешные люди в других
отраслях похожи на этот мультяшный образ, я не знаю, но только не
стартаперы.

Многие думают, что успешные основатели стартапов одержимы деньгами. На
самом же деле секретное оружие самых успешных создателей в том, что
они не такие. Наиболее успешными учредителями движет прежде всего их
компания как проект.

К этому я пришел эмпирически, но никогда не видел теоретических
доказательств этому до того, как получил письмо этого стартапера. В
нем он говорил об обеспокоенности своим принципиальным мягкосердечием,
из-за которого слишком многое делает бесплатно, и считает, что ему
нужно быть немного жестче.

Я отвечал ему, что не стоит беспокоиться об этом, поскольку его
поведение помогает быстрому росту его стартапа. Если бы он был
целеустремлен на максимальное извлечение дохода из своего бизнеса,
рост будет несколько меньшим, но график роста будет точно такой же
формы.

В качестве примера рассмотрим ситуацию, когда ваша компания в
настоящее время приносит \$1000 в месяц, и вы делаете, что-то
настолько крутое, что растете на 5\% в неделю. Два года спустя, вы
будете зарабатывать около \$160 тыс. в месяц.

Теперь предположим, что вы не такой алчный и берете с ваших клиентов
половину того, что могли бы брать. Это означает, что два года спустя
вы заработаете \$80 тыс. в месяц вместо \$160 тыс. Насколько вы
отстаете? Сколько нужно времени, чтобы перейти к такому же уровню
дохода? Всего 15 недель. После двух лет щедрый стартапер всего за 3,5
месяца догонит по уровню дохода самого алчного дельца (при прочих
равных).

Если вы собираетесь оптимизировать числа, единственный путь
заключается в наращивании темпов роста. Пусть, как и раньше за свои
услуги вы берете вдвое меньше от возможного, но подобная щедрость
увеличивает темп вашего роста до 6\% в неделю вместо 5\%. Сравним
уровни дохода щедрой и алчной стратегии после двух лет ведения
бизнеса? Первая дает уже \$214 тыс. в месяц по сравнению со \$160 тыс.
при ускоренных темпах. За иной год вы сможете заработать \$4,4 млн в
месяц против жадного основателя с \$2 млн.

Очевидно, что жадность имеет место быть только в случае, когда ее
проявление обеспечивает быстрый рост. В отличие от многих других
отраслей, это правило не применимо к стартапам. Стартапы обычно
выигрывают, делая что-то настолько классное, что люди рекомендуют их
своим друзьям, как следствие жадность не только не поможет вам в этом,
но и, вероятно, принесет дополнительные проблемы.

Другой причиной выкачивания денег из клиентов является то, что, как
правило, стартапы в начале своей деятельности теряют деньги, и
повышение рентабельности за счет клиента позволяет легче достичь
прибыльности, раньше чем первоначальное финансирование иссякнет. В то
же время высокая смертность стартапов заключается в недостатке
средств, вызванном, как правило, медленным ростом или чрезмерными
расходами, при этом недостаточные усилия по извлечению клиентских
средств никакого отношения к этому не имеют.

Так что, если вы стартапер, вот сделка, которую вы можете заключить с
самим собой. Она сделает вас счастливым, а вашу компанию успешной.
Скажите себе, что вы можете быть хорошим, так долго, как хотите,
трудолюбиво работая над темпами вашего роста для компенсации.
Большинство успешных стартапов приходят к этому компромиссу
бессознательно. Может быть, если вы сделаете его сознательно вас ждет
еще больший успех.

\end{document}
