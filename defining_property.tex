\documentclass[ebook,12pt,oneside,openany]{memoir}
\usepackage[utf8x]{inputenc} \usepackage[russian]{babel}
\usepackage[papersize={90mm,120mm}, margin=2mm]{geometry}
\sloppy
\usepackage{url} \title{Дать определение «собственности»} \author{Пол
  Грэм} \date{}
\begin{document}
\maketitle

Будучи ребенком, я прочел книгу рассказов о знаменитом судье XVIII
века Тадасукэ Ооока. Одно из дел, которые он рассматривал, было
возбуждено владельцем продуктового магазина. Бедный студент, который
мог позволить себе только рис, ел этот рис, наслаждаясь ароматом
свежеприготовленной пищи, исходящим из магазина. Владелец хотел, чтобы
студент платил за приятный запах. Студент крадёт его аромат!

Я часто вспоминаю эту историю, когда слышу, что Американская
ассоциация звукозаписывающих компаний и Американская киноассоциация
обвиняют людей в краже музыки и фильмов.

Для нас считать запах за имущество — нелепость. Но я могу представить
сценарии, при которых можно было бы взимать плату за запах.
Вообразите, что мы живем на лунной базе, где нам приходится покупать
литрами воздух. Можно предположить, что поставщики воздуха смогут
добавлять запахи за дополнительную плату.

Относить запахи к имуществу нам кажется нелепым, потому что это не
сработает. Хотя, это будет работать на лунной базе.

То, что считается собственностью, зависит от того, что мы готовы
принять за собственность. И это восприятие не только может измениться
— оно уже изменилась. Люди всегда (в случаях определенного значения
слов «люди» и «всегда») относились к маленьким вещицам, которые нес
человек как к собственности. Однако охотники никогда не относили землю
к собственности, как мы это делаем сейчас.[1]


Причина, почему многие люди думают, что собственность можно описать
единым неизменным определением, в том, что это определение меняется
очень медленно. [2] Но мы живем в самой гуще такого изменения.

Звукозаписывающие и киностудии распространяли свой продукт так же как
поставляли бы через трубки воздух на лунную станцию. Но с приход сети
Интернет был равнозначен переезду на планету с пригодной атмосферой.
Сведения сегодня распространяются как запах. А из-за союза желания
выдавать желаемое за действительное и краткосрочной алчности, студии
оказались в качестве того самого владельца продуктового магазина,
который обвиняет нас в краже запахов.

(Я употребил «краткосрочная алчность», так как основная проблема
звукозаписывающих и киностудий в том, что руководящие ими людьми
больше заинтересованы в получении сиюминутных бонусов больше, чем
капитала. Если бы они руководствовались последним, они бы искали пути
получать выгоду с технологического изменения, а не бороться с ним.
Однако строительство нового занимает слишком много времени. Их бонусы
зависят от доходов текущего года, а самый лучший способ увеличить их —
извлечь больше денег из материала, который они уже сделали.)

Тогда что все это значит? Люди не должны взимать плату за контент? На
этот вопрос нет однозначного ответа. Люди должны брать деньги за
контент тогда, когда эту плату получается взимать.

Но под «получается» я имею в виду больше, чем просто «когда они могут
выиграть состязание». Я вкладываю туда ситуацию, когда люди могут
взимать плату за контент без принуждения к этому общества. В конце
концов, компании, продающие запахи на лунной базе, могут продолжить
свою деятельность и на земле, но при условии, что они пролоббируют
закон, который обяжет нас и здесь дышать через трубки, несмотря на то,
что мы в этом уже не нуждаемся.

Сумасшедшие правовые меры, которые были приняты звукозаписывающими и
киностудиями, имеют именно такой привкус. Газеты и журналы тоже
закручивают гайки, но их количество хотя бы грациозно уменьшается.
Американская ассоциация звукозаписывающих компаний и Американская
киноассоциация могут заставить нас дышать через трубки, если они
захотят.

В конечном счете, все сводится к здравому смыслу. Когда вы оскорбляете
законодательство тем, что пытаетесь использовать массовые судебные
процессы против рандомно выбранных людей ради показательного
наказания, или же тем, что лоббируете законы, которые могут взорвать
интернет, что является доказательством неработающего определения
собственности.

Вот здесь полезно иметь действительную демократию и несколько
суверенных государств. Если бы в мире было только одно деспотичное
правительство, компании бы смогли купить закон, который установил бы
нужное им определение собственности. К счастью, все еще есть страны,
которые не являются «колониями» США, да и в Америке политики все еще
боятся многочисленности активных избирателей. [3]

Людям, управляющим США, может не понравиться ситуация, когда
избиратели или другие страны откажутся подчиниться их воле, однако, в
наших общих интересах не допустить существования одного способа для
атаки, который бы работал на тех, кто пытается изменить закон исходя
из собственных побуждений. Частная собственность — невероятно полезная
идея, возможно, одна из наших лучших изобретений. Поэтому каждое ее
новое определение приносило нам увеличения богатства. Логично
предположить, что то же произойдет и с новейшим определением. Будет
катастрофа, если мы все должны будем продолжать работать с устаревшей
версией только потому, что несколько сильных людей были слишком
ленивы, чтобы обновить.

Примечания

[1] Если вы хотите больше узнать об охотниках, я настойчиво рекомендую
книги «The Harmless People» и «The Old Way: A Story of the First
People» Елизабет Маршал Томас.

[2] Изменение определения собственности в большинстве случаев связано
с технологическим прогрессом, а так как прогресс ускоряется, то это
отражается на скорости изменения понятия «собственность». Что значит,
для общества становится важнее умение правильно реагировать на эти
изменения, потому что они будут появляться с возрастающей скоростью.

[3] Насколько я знаю, термин «copyright colony» (Колонии авторского
права) был введен Майлсом Петерсоном.

[4] Состояние технологий — не просто функция понятия «собственность».
Каждый из них ограничивает другой. Но тогда вы не можете возиться с
понятием «собственность», не затрагивая (и, возможно, не причиняя
вред) состоянии технологий. История СССР — яркий тому пример.

\end{document}
