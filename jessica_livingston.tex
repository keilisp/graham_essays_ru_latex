\documentclass[ebook,12pt,oneside,openany]{memoir}
\usepackage[utf8x]{inputenc} \usepackage[russian]{babel}
\usepackage[papersize={90mm,120mm}, margin=2mm]{geometry}
\sloppy
\usepackage{url} \title{Поговорим о Джессике Ливингстон} \author{Пол
  Грэм} \date{}
\begin{document}
\maketitle

Несколько месяцев назад в одном издании появилась статья о проекте Y
Combinator, в которой говорилось, что на ранних стадиях своего
развития это был «театр одного актера». Как ни печально, но с
подобными вещами встречаешься довольно часто. Но проблема с этим
описанием не только в том, что оно несправедливо. Оно также вводит
читателей в заблуждение. То, что представляет собой YC сейчас – это в
большей мере заслуга Джессики Ливингстон (Jessica Livingston). Если вы
не понимаете ее, то вы не поймете YC. Так что позвольте мне рассказать
вам о Джессике.

У компании YC было четыре учредителя. Одним прекрасным вечером мы
вдвоем с Джессикой решили основать его, и уже на следующий день мы
привлекли моих друзей Роберта Морриса (Robert Morris) и Тревора
Блэкуэлла (Trevor Blackwell). Джессика и я управляли YC изо дня в
день, а Роберт и Тревор читали резюме и проводили с интервью вместе с
нами.

Мы с Джессикой начали встречаться еще до рождения YC. Сначала мы
попытались быть «профессионалами» и поступать соответствующе, то есть
забыть о чувствах. Но позднее это показалось нам смешным, и мы решили
больше не притворяться. Мы с Джессикой вместе оказали огромное влияние
на YC – мало кто оспорит данный факт. Эта площадка стала для нас
чем-то вроде семьи. Основатели стартапов на первых порах были в
основном молодыми людьми. Мы все собирались вместе за обедом раз в
неделю, первые пару лет угощение для таких встреч готовил я. Наш
первый офис находился в здании частного дома, атмосфера которого
поразительно отличалась от офисов венчурных фирм на Сэнд-Хилл-роуд. И
в каком-то смысле это было преимуществом. Каждый, кто заходил к нам,
чувствовал искренность, подлинность наших побуждений. И это значит,
что люди не просто верили нам – это было тем замечательным качеством,
которое нужно прививать стартапам. Подлинность – это одна из самых
важных особенностей, которую YC ищет в основателях, не только потому,
что мошенники и авантюристы раздражают, но и потому, что подлинность
является одной из главных особенностей, которые выделяют успешные
стартапы среди всех остальных.

В свои ранние годы YC был семьей, и Джессика была его матерью. И она
создала культуру, которая затем стала одним из самых важных
нововведений YC. Культура имеет важное значение для любой организации,
но в YC культура определяла не только то, как мы вели себя во время
работы над продуктом. В YC наша культура и была продуктом.

Джессика также была матерью и в другом смысле – за ней было последнее
слово. Все, что мы сделали как организация, сначала проходило через
нее – кого финансировать, что говорить общественности, что делать с
другими компаниями, кого нанимать.

До того, как мы обзавелись детьми, YC в какой-то мере стал нашей
жизнью. У нас не было четкой границы между работой и отдыхом. Мы
говорили о YC постоянно. И нам нравилось это вопреки тому, что
некоторые сферы предпринимательства способны испортить частную жизнь.
Мы основали его потому, что это было нам интересно. И некоторые
проблемы, которые мы пытались решить, были бесконечно сложны. Как
распознать хорошего основателя? Это можно обсуждать годами. И мы
обсуждали — и продолжаем до сих пор.

С чем-то я справляюсь лучше, чем Джессика, что-то она делает лучше,
чем я. Но в одном ее не превзойти – в умении разбираться в людях. Она
одна из немногих, кто с одного лишь взгляда способен уловить суть
человека. Она практически моментально обличает мошенников. В YC она
получила прозвище «Социальный Радар», и это ее особое качество было
критично для того, чтобы YC стал тем, чем является сегодня. Чем раньше
вы выбираете стартап, тем больше этот выбор основан на личности его
учредителя. На более поздних стадиях инвесторы могут оценить продукты
и статистику роста. На том этапе, на котором YC инвестирует в
стартапы, часто нет ни продукта, ни каких-либо цифр.

Другие думали, что в YC существует особое понимание будущего
технологий. Но в основном наше понимание можно было выразить словами,
некогда сказанными Сократом: мы хотя бы знали, что ничего не знаем.
Что сделало YC успешным, так это способность выбрать хороших
основателей. Мы думали, что Airbnb был плохой идеей. Мы дали компании
финансирование потому, что нам понравились ее основатели.

Во время интервью Роберт, Тревор и я забрасывали кандидатов
техническими вопросами. Джессика главным образом наблюдала за
происходящим. Множество кандидатов, наверное, считали ее кем-то вроде
секретарши, особенно поначалу. Потому что именно она выходила, чтобы
вызвать каждую группу, и не спрашивала много вопросов. Она была совсем
не против. Наоборот, для нее было легче наблюдать за людьми, если они
ее не замечали. Но после интервью мы втроем разворачивались к Джессике
и спрашивали: «Что говорит Социальный Радар (см. примечание 1 в конце
статьи)?»

И хотя поначалу мы делали это просто потакая себе, но оказалось, что
такая практика невероятно ценна для YC. Мы не понимали этого в начале,
но люди, которых мы выбирали, образуют сеть воспитанников YC. И как
только мы выбрали их, они навсегда становились его частью (за тем
редким исключением, если им не угораздило вытворить чего-то из ряда
вон выходящего). Сейчас некоторые считает, что сеть воспитанников YC –
это его самая ценная особенность. Лично я думаю, что консультирование
YC тоже весьма неплохо, хотя сеть выпускников, конечно, является одной
из его ключевых особенностей. Уровень доверия и помощи поразителен для
группы такого масштаба. И в основном это все благодаря Джессике.

(Как мы узнали позднее, наши потери были невелики, когда мы отказывали
людям, в личных качествах которых мы сомневались, потому что
существует зависимость между тем, насколько хороши основатели и тем,
как они справляются со своими задачами. Если плохим основателям вдруг
посчастливилось добиться успеха, то они склонны к ранней продаже
бизнеса. В большинстве своем успешные основатели отличаются
положительными личными качествами.)

Если Джессика была так важна для YC, почему об этом знает так мало
людей? Отчасти потому, что я писатель, а писатели всегда привлекают
непропорционально большое внимание. Бренд YC изначально был моим
брендом, и наши кандидаты были людьми, которые читали мои эссе. Но
есть и другая причина: Джессика ненавидит внимание. Общение с
репортерами раздражает ее. Мысль о том, чтобы выступить с докладом,
парализует ее. Даже на нашей свадьбе она чувствовала себя неуютно,
потому что невеста всегда находится в центре внимания (см. примечание
2 в конце статьи).

Она ненавидит повышенное внимание к себе не только потому что она
застенчива, но потому, что это мешает ее роли Социального Радара. Она
не может быть собой. И вы никогда не сможете наблюдать за людьми, если
все наблюдают за вами.

Кроме этого есть и другая причина, по которой внимание людей для нее
тягостно – она не любит хвастовства. Когда она делает что-либо
открытое для обозрения публичности, самым большим ее страхом (кроме
очевидной боязни сделать свое дело плохо) является то, что ее
начинания будут казаться показными. Она утверждает, что излишняя
скромность является распространенной проблемой для большинства женщин.
Но в ее случае это нечто большее. Глубоко в ней сидит страх перед
показной роскошью, практически переросший в фобию.

Также она ненавидит выяснение отношений. Она не может парировать – ее
просто парализует. И к сожалению, будучи общественным лицом
организации, нередко приходится держать удар.

Таким образом, несмотря на то, что в большей степени YC обязан своей
уникальностью именно Джессике, те ее качества, которые помогли ей
сделать это, в ответе и за то, что Джессика не хотела бы фигурировать
в истории YC. Каждый верит в рассказы о том, как Пол Грэм основал YC,
а его жена как-то ему в этом помогала. Даже ненавистники YC принимают
это. Когда пару лет назад на нас посыпались обвинения в том, что мы не
инвестируем средства в большее число женщин-предпринимателей, люди
рассматривали YC как нечто тождественное Полу Грэму. Признание
ключевой роли Джессики в YC испортило бы рассказы наших обвинителей.

Джессика кипела от возмущения при мысли о том, что люди обвиняют ЕЕ
компанию в сексизме. Я еще никогда не видел ее в таком бешенстве. Но
она не противоречила им. Не публично. Между нами она здорово их
бранила. Она написала три эссе относительно вопроса
женщин-предпринимателей, но так и не решилась опубликовать их. Она
видела, сколько желчи в этих дебатах, и не решилась вступать в них.

И это не только потому, что она не любит ссориться. Она настолько
ранимый человек, что у нее вызывает неприязнь даже выяснения отношений
с бесчестными личностями. Идея столкновения с журналистами желтых
изданий или троллями в Twitter показалась бы ей не просто пугающей, но
и отвратительной.

Но Джессика понимала, что ее пример успешной женщины-учредителя
побудит больше женщин у тому, чтобы основать собственные компании.
Поэтому в прошлом году она сделала то, чего YC еще не видывал за всю
свою историю – она наняла PR-агенство для того, чтобы сделать ряд
интервью. В одном из первых интервью с Джессикой репортер отбросил в
сторону все ее знания о стартапах и состряпал сенсационный рассказ о
том, как какой-то парень пытался флиртовать с ней возле барной стойки,
где была назначена их встреча. Джессика была поражена частично потому,
что парень ничего плохого не сделал, но в истории ее выставили
жертвой, которая только тем и важна, что она женщина, нежели одним из
наиболее осведомленных инвесторов во всей Долине.

После этого она решила прекратить сотрудничество с этим PR-агентством.

Вы не услышите в прессе о том, чего достигла Джессика. Так что
позвольте мне рассказать о ее достижениях. Y Combinator в основе своей
является сообществом людей — как, например, университет. Он не создает
продукт. Люди являются его сутью. Джессика больше чем кто-либо
приложила свою руку к отбору этих людей и способствовала их
дальнейшему развитию. В этом плане она создала YC.

Джессика знает больше о персональных качествах основателей стартапов,
чем кто-либо. Ее огромный опыт и умение разбираться в людях являются
идеальным сочетанием для нашего бизнеса. Персональные качества
основателей лучше всего расскажут о том, каких успехов сможет добиться
стартап. И стартапы в свою очередь являются самым важным источником
роста в развитых странах.

Джессика Ливингстон является тем человеком, который знает все о самом
важном факторе экономического роста в развитых странах. Не кажется ли
вам, что такой человек должен быть более известным?

Примечания

[1] Хардж Тэггэр (Harj Taggar) напомнил мне, что несмотря на тот факт,
что Джессика не задавала много вопросов, в основном они были важными:

«Джессика всегда могла унюхать, если с командой или намерениями
основателей что-то было не так, и задавала обезоруживающие вопросы,
которые обычно раскрывали куда больше, чем подозревали сами
учредители.

[2] Или более точно: в то время как ей нравится привлекать внимание в
смысле получения заслуженного одобрения за свой труд, она не любит
другого вида внимания – постоянного пристального наблюдения. К
сожалению, не только для нее, но и для большого количества людей,
первое во многом зависит от второго.

Если вам доведется увидеть Джессику на общественном мероприятии, то вы
никогда не предположите, что она ненавидит внимание к себе потому что:
(a) она очень вежлива, (b) когда она нервничает, то больше улыбается.

[3] О существование людей как Джессика должны узнать не только медиа,
но и феминистки. Есть успешные женщины, которые не любят брани. А это
значит, что они не будут участвовать в публичных схватках по поводу
женщин и их прав, если разговоры пропитаны негативом.

Я выработал свой собственный Закон Грэма для разговоров. Когда страсти
накаляются до определенного уровня, выходя за границы вежливости, то
более разумные люди предпочитают уйти. Никто не понимает
женщин-предпринимателей лучше, чем Джессика. Но маловероятно, что
когда-нибудь она заявит публично о своих мыслях относительно данной
темы. Она уже раз попробовала войти в эту реку и решила, что больше
никогда не повторит подобного.

\end{document}
