\documentclass[ebook,12pt,oneside,openany]{memoir}
\usepackage[utf8x]{inputenc} \usepackage[russian]{babel}
\usepackage[papersize={90mm,120mm}, margin=2mm]{geometry}
\sloppy
\usepackage{url} \title{Как хорошо писать и хорошо выступать}
\author{Пол Грэм} \date{}
\begin{document}
\maketitle

Я не очень хороший оратор. Я часто говорю «э-э». Иногда мне приходится
делать паузу, когда я теряю мысль. Хотелось бы мне уметь лучше
выступать. Но не настолько, насколько мне хочется научиться лучше
писать. Чего мне на самом деле хочется — это побольше хороших идей. А
это гораздо больше помогает хорошо писать, чем хорошо говорить.

Хорошие идеи — главная составляющая хорошего текста. Если вы знаете, о
чем говорите, то вы можете выразить это в самых простых словах, и
читатели воспримут это как хороший стиль. С разговорным жанром все
наоборот: хорошие идеи — тревожно малая составляющая успешных
выступлений.

Впервые я это заметил на одной конференции несколько лет назад. На ней
выступал еще один человек — выступал гораздо лучше меня. На его
выступлении мы все катались от смеха. По сравнению с ним я казался
неловким, запинающимся. После этого я, как обычно, вывесил свою речь в
интернет. По ходу дела я попытался представить себе, как выглядела бы
расшифровка выступления того другого человека. И только тогда я понял,
что сказал он не так уж много.

Может быть, это стало бы очевидно для того, кто знает больше об
ораторстве. Но для меня то, насколько мало идеи значат в искусстве
выступления, стало откровением.

Несколько лет спустя я услышал выступление человека, не просто
говорящего лучше меня — он был знаменитым оратором. О, как он был
хорош. И я решил внимательно прислушаться к тому, что именно он
говорит, чтобы понять, как он это делает. Примерно десять предложений
спустя мне в голову пришла мысль: «Я не хочу быть хорошим оратором».

Умелое ораторство не просто никак не пересекается с наличием хороших
идей, но во многих отношениях толкает вас в противоположном
направлении. К примеру, когда я собираюсь выступать, я обычно пишу
текст заранее. Я знаю, что это ошибка. Я знаю, что рассказывать
заранее написанный текст — это усложняет работу с аудиторией. Привлечь
внимание аудитории можно, если отдать ей все свое внимание. А когда вы
выступаете с заранее написанным текстом, ваше внимание неизбежно
делится между аудиторией и речью — даже если вы заучили ее наизусть.
Если вы хотите общаться с аудиторией, лучше начать лишь с наброска
того, что вы хотите сказать, и сочинять конкретные предложения на
ходу, импровизировать. Но если вы поступаете так, вы не можете тратить
на обдумывание каждой фразы больше времени, чем занимает ее
произнесение. Иногда сам факт разговора с новой аудиторией побуждает
вас придумать нечто новое. Но в целом этот процесс не настолько
стимулирует к выдвижению новых идей, как письмо, когда вы можете
потратить на каждое предложение столько, сколько хочется.

Если вы достаточно прорепетируете записанный заранее текст, вы можете
приблизиться к тому взаимодействию с аудиторией, какого вы добиваетесь
при импровизации. Актеры так и делают. Но здесь все равно придется
идти на компромисс между гладкостью речи и наличием идей. Все то
время, что вы репетируете речь, вы могли бы потратить на то, чтобы
улучшить ее. Актеры не сталкиваются с таким искушением, за теми
редкими исключениями, когда они сами пишут сценарий. Но любой оратор
сталкивается. Прежде чем я выхожу на сцену, я обычно сижу где-то в
углу с распечатанным текстом, пытаясь мысленно его отрепетировать. Но
я всегда в итоге трачу большую часть времени на переписывание текста.
Всякий раз я выступаю по тексту, в котором много всего вычеркнуто или
переписано. В силу этого, конечно, я еще чаще говорю «э-э», потому что
у меня просто нет времени, чтобы отрепетировать новые фрагменты
текста.

Все зависит от аудитории: вы можете столкнуться с еще более неприятным
компромиссом. Аудитории нравится, когда ей льстят. Ей нравятся шутки.
Ей нравится, когда ее сбивает с ног стремительный поток слов. По мере
снижения интеллекта аудитории хороший оратор все больше должен уметь
вешать лапшу на уши. Это касается и написания текстов, конечно, но с
выступлениями кривая идет вниз более круто. Любой отдельно взятый
человек будет глупее в роли слушателя, чем в роли читателя. Подобно
импровизирующему оратору, который может тратить на обдумывание
предложения не больше, чем понадобится на его произнесение, слушатель
может потратить на обдумывание предложения не дольше, чем слышит его.
К тому же люди в аудитории всегда обращают внимание на реакции
окружающих, и реакции, которые распространяются от одного слушателя к
другому, заметно более грубого свойства. Подобно тому, как низкие ноты
лучше проходят через стены, чем высокие. Любая аудитория — это
потенциально неистовствующая толпа, и хороший оратор пользуется этим.
Я так много смеялся на выступлении хорошего оратора на той самой
конференции отчасти потому, что смеялись все остальные.

Так что, выступления бесполезны? В качестве источника идей они
определенно хуже, чем письменное слово. Но выступления нужны не только
для этого. Когда я иду на выступление, обычно меня интересует оратор.
Большинство из нас не могут вступить в настоящую беседу с кем-нибудь
вроде президента страны, и выслушать выступление — самое близкое к
такому разговору, что нам достается.

Выступления также успешно мотивируют меня что-то сделать. Наверное,
это не совпадение, что многие известные ораторы удачно выступают с
мотивирующими речами. Может быть, в этом и есть смысл публичных
выступлений. Возможно, для этого они и были придуманы. Эмоциональные
реакции, которые вы можете вызвать своей речью — мощная сила. Хотелось
бы сказать, что эта сила чаще используется в благих целях, но в этом я
не уверен.

\end{document}
