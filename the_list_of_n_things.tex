\documentclass[ebook,12pt,oneside,openany]{memoir}
\usepackage[utf8x]{inputenc} \usepackage[russian]{babel}
\usepackage[papersize={90mm,120mm}, margin=2mm]{geometry}
\sloppy
\usepackage{url} \title{Лист из N вещей} \author{Пол Грэм} \date{}
\begin{document}
\maketitle

Готов с вами поспорить, что в свежем выпуске «Cosmopolitan» есть
статья, название которой начинается с числа. «7 вещей, которые он
никогда не расскажет вам о сексе», или что-то вроде этого. Некоторые
популярные журналы размещают на обложке каждого номера статьи такого
типа. Это не случайно. Редакторы знают – это привлекает читателей.
Почему же читатели настолько любят список из N вещей. Преимущественно
потому, что его, по отношению к обычным статьям, легче читать [1].
Структурно, список из N вещей – это упрощенный вариант эссе. Эссе
может быть обо всём, что только захочет автор. В списке из N вещей
автор ограничивает себя списком пунктов, которые примерно равны по
значимости, и он явно показывает их читателям.

При прочтении статьи нужно приложить определенное усилие для понимания
её структуры – образно выражаясь то, что в университете мы называли
остовом. Конечно, не обязательно, но у того, кто действительно понял
статью, наверняка останется что-то в голове, что, впоследствии, будет
соответствовать этому остову. В списке N вещей, эта работа сделана за
вас. Его структура уже внешне очерчена. Наравне с понятностью,
структура гарантирует наипростейший из возможных типов: несколько
главных тезисов, несколько второстепенных, и никакой конкретной связи
между ними. Список из N вещей доступен в произвольном порядке, потому
что главные тезисы несвязанны. Нет никакой причинно-следственной
связи, которой вы должны следовать. Вы можете читать список в
произвольном порядке. И из-за того, что пункты списка независимы один
от другого, они работают как водонепроницаемые отсеки в непотопляемом
корабле. Если вам скучно, не можете понять, или не согласны с одним из
пунктов, вам не обязательно бросать всю статью. Вы можете оставить
этот пункт и перейти к следующему. Список N вещей - параллельный и
поэтому терпим к различным недостаткам. Временами, все, что нужно
автору – это именно этот формат. Иногда, то, что вы обязаны написать,
фактически,– список из N вещей. Однажды я написал эссе про ошибки,
которые убивают стартап, и некоторые люди смеялись надо мной из-за
названия, которое начинается с числа. Но в этом случае, я
действительно пытался сформировать полный каталог независимых вещей.
Фактически, один из вопросов, на который пытался ответить: «насколько
их много». Существуют еще, менее серьёзные, причины использования
этого формата. Например, я использую его, когда конечный срок близок.
Если у меня запланировано выступление, и я даже не начинал готовиться
к нему, хотя бы за несколько дней, предпочитаю не рисковать и
подготавливаю лист из N вещей для выступления. Лист из N вещей проще
для автора, также как и для читателей. Когда вы пишете настоящее эссе,
у вас всегда есть шанс зайти в тупик. Настоящее эссе – это свободный
полёт мыслей, некоторые мысли заводят не туда. Это тревожная
вероятность, когда у вас выступление через несколько дней. Что же
делать, если вы исчерпали все идеи? Блочная структура списка N вещей
защищает автора от его же глупостей так же, как она защищает
читателей. Если у вас иссякли все идеи относительно одного из пунктов,
ничего страшного, это не испортит эссе. Вы можете выбросить целый
пункт, если нужно, эссе всё равно останется в живых. Процесс написания
листа из N вещей так расслабляет. Вы придумываете n/2 пункта за первые
5 минут. И бамс, вот вам и структура, всего лишь остается заполнить
её. Как только вы придумываете новые пункты, вы просто добавляете их в
конец. Возможно, вы уберете, перегруппируете или объедините несколько
пунктов, но на каждом этапе у вас будет годный (хотя вначале и мало
наполненный) список из N вещей. Это как стиль программирования, у вас
очень быстро готова первая версия, которая потом постепенно
модифицируется, но на каждом этапе у вас имеется работающий код, или
стиль рисования, где вы начинаете с законченного, но довольно таки
размытого эскиза, сделанного за час, потом тратите неделю на улучшение
его разрешения. Не всегда это плохой знак, когда читатели предпочитают
лист из N вещей, ведь он же проще и для авторов. Это не обязательное
свидетельство, что читатели ленивы, это может также означать, что они
не очень доверяют автору. В этом аспекте лист N вещей – это чизбургер
формы эссе. Когда вы кушаете в ресторане, который считаете плохим, ваш
лучший выбор – заказать чизбургер. Даже плохой повар может приготовить
приличный чизбургер. Существует довольно таки строгие ограничения на
внешний вид чизбургера. Можно предположить, что повар не будет
пытаться делать, что-то ужасное и художественное. Лист N вещей,
аналогично, ограничивает весь вред, который может причинить плохой
автор. Вы знаете, что оно будет содержать хотя бы что-нибудь, что
упомянуто в названии, и этот формат препятствует полету фантазии
автора. Это хороший формат для начинающих авторов, потому что лист из
N вещей самая простая форма эссе. И фактически, это то, чему учится
большинство начинающих авторов. Классическое эссе из пяти абзацев в
действительности – лист из n вещей, где n = 3. Но студенты пишущие их
не осознают того, что они используют такую же структуру, как и в
статьях которые они читают в «Cosmopolitan». Им нельзя использовать
числа, и для связи абзацев они используют ненужные вводные слова и
словосочетания («Кроме того, …») и оборачивают их введением и
заключением, внешне это выглядит как настоящее эссе [2]. Кажется, что
для студента список из N вещей это замечательный план для старта. Это
самая легкая форма. Но если мы собираемся так делать, почему бы это не
делать открыто. Давайте разрешим им писать список N вещей как
профессионалам, с числами и никаких связующих слов или «в заключение».
Существует одно обстоятельство, где лист из N вещей – нечестный
формат: когда вы, используя его для привлечения внимания, ложно
объявляете этот список полностью исчерпывающим, например, если вы
пишете статью о 7-ми секретах успеха. Такого рода названия сродни
противоположной задачи детектива. Вы должны, по крайней мере,
просмотреть статью, чтобы проверить, что это те же 7 пунктов, которые
есть у вас. Не пропустили ли вы один из секретов успеха? Лучше
проверить. Хорошо поставить «все» перед числом, если вы действительно
верите, что сделали полный список. Но очевидно большинство вещей с
таким названием – это линкбэйтинг (linkbait). Самый большой недостаток
списка из N вещей – мало пространства для новых идей. Квинтэссенция
написания эссе, когда вы правильно его делаете, это новые мысли,
появляющиеся у вас, во время его написания. Настоящее эссе, как
подразумевает название, живое: когда начинаете его, вы не знаете, о
чем будете писать. Оно будет о том, что вы изучите, пока будете
писать. В списке из N вещей это может происходить только в
ограниченном объеме. Вначале придумываете название, и это именно то, о
чем будете рассказывать. Вы не можете добавлять новые идеи, кроме тех,
которые наполнят те соответствующие независимые компоненты,
определенные вначале. И, ваш мозг понимает это, поэтому когда у вас
нет места новым идеям, они не нужны. Другое преимущество, для
начинающих писателей, признания того, что 5 абзацное эссе, по сути,
является списком N вещей, – мы можем предупредить их об этом. Это даст
вам опыт для определения характеристики эссе, написание его в
маленьком объёме: мысли на одно-два предложения. И особенно опасно,
что пяти абзацное эссе скрывает список N вещей под видом чего-то
похожего на более сложное по стилю эссе. Если вы не осознаёте, что вы
используете эту форму, то вы не знаете, что вам нужно избежать это.


Сноски

[1] Статьи этого типа также поразительно популярны на Delicious, но я
думаю, это потому что delicious/popular увлечены закладками, не потому
что пользователи Delicious глупые. Пользователи Delicious
коллекционеры, и лист N вещей в высокой степени коллекционируемый,
потому что она сама по себе коллекция.

[2] Большинство «проблемных слов» в школьной математике подобным
образом вводят в заблуждение. Внешне они выглядят как применение
математики для решения реальных задач, но они не подходят. Если
что-либо, они усилят впечатления, что математика просто сложный, но
бессмысленный набор материалов для запоминания.

\end{document}
