\documentclass[ebook,12pt,oneside,openany]{memoir}
\usepackage[utf8x]{inputenc} \usepackage[russian]{babel}
\usepackage[papersize={90mm,120mm}, margin=2mm]{geometry}
\sloppy
\usepackage{url} \title{«Четыре квадранта конформизма»} \author{Пол
  Грэм} \date{}
\begin{document}
\maketitle

Один из наиболее показательных способов классификации людей –
определение степени и агрессивности их конформизма. Представьте себе
декартову систему координат, горизонтальная ось которой проходит слева
направо от традиционности к независимости мышления, а вертикальная ось
— от пассивности снизу к агрессии сверху. Полученные четыре квадранта
определяют четыре типа людей. Начиная с верхнего левого и двигаясь
против часовой стрелки: агрессивный конвенционализм, пассивный
конвенционализм, пассивная независимость и агрессивная независимость.

Я думаю, что вы найдете людей всех этих типов почти в любом обществе.
Квадрант, к которому будут относиться эти люди, в основном
определяется их личностью, а не убеждениями, преобладающими в их
обществе. [1]

Лучше всего оба пункта можно доказать на примере маленьких детей.
Любой, кто побывал в начальной школе, видел людей всех четырех типов.
Факт огромной условности школьных правил представляет веское
доказательство того, что класс и квадрант, к которому относится
человек, больше зависят от него самого, нежели от правил.

Дети в верхнем левом квадранте, агрессивные конвенционалисты, – это
болтуны. Они верят не только в то, что правила должны соблюдаться, но
и в то, что те, кто не подчиняется им, должны быть наказаны.

Дети в нижнем левом квадранте, пассивные конвенционалисты, похожи на
невинных овечек. Они осторожны, подчиняются правилам, но когда правила
нарушают другие дети, наши овечки переживают о том, что
дети-нарушители будут наказаны, хотя не добиваются этого наказания.

Дети в нижнем правом квадранте, чье мышление пассивно-независимо – это
мечтатели. Их не очень волнуют правила, и они, вероятно, не в полной
мере с ними знакомы.

Дети в правом верхнем квадранте, чье мышление агрессивно-независимо,
непослушны. Если они сталкиваются с каким-либо правилом, у них
возникает желание усомниться в нем. Если им говорят что и как делать,
им хочется делать все наоборот.

Конечно, говоря о конформизме нужно уточнять к чему именно он
относится, а также учитывать, что мышление детей меняется с возрастом.
В случае с маленькими детьми, конформизм относится к правилам,
установленным взрослыми. С возрастом источником правил становятся их
ровесники. Таким образом, группа подростков, одинаково пренебрегающих
школьными правилами, не является независимой, скорее наоборот.

Во взрослой жизни мы можем распознать эти четыре вида по их
отличительным признакам – так же, как мы отличаем между собой разные
виды птиц. Слоган агрессивно настроенных конвенциалистов — «Сломить
<название другой группы>»! (довольно тревожно видеть восклицательный
знак после переменной, обозначающей название группу людей, но в этом
вся проблема агрессивных конвенционалистов). Слоган пассивных
конвенционалистов – «Что подумают соседи?». Слоган людей с
пассивно-независимым мышлением – «Каждому своё». Наконец, слоган
агрессивно-независимых – «И всё-таки она вертится!».

Все эти четыре типа не являются одинаково распространенными. Пассивных
людей больше, чем агрессивных, а людей с традиционным мышлением
гораздо больше, чем независимых. Таким образом пассивные
конвенционалисты – самая большая группа, а агрессивно-независимых
людей меньше всего.

Поскольку квадрант по большей части определяется личностью человека, а
не природой правил, большинство людей оказались бы в одном и том же
квадранте независимо от общества, в котором они выросли.

Принстонский профессор Роберт Джордж недавно написал:

Иногда я спрашиваю студентов – какова была бы их позиция по отношению
к рабству, если бы они были белыми и жили на Юге до освобождения
рабов. И знаете что? Они все были бы борцами за отмену рабства! Они
все смело выступили бы против рабства и неустанно добивались его
отмены.

Джордж слишком вежлив, чтобы говорить об этом, но конечно все было бы
не так. И нам следует не просто предполагать, что его студенты вели бы
себя так же, как и другие люди. Идея в том, что агрессивно настроенные
конвенционалисты вели бы себя так же и в ту эпоху. Иными словами они
бы не просто не боролись с рабством, они были бы его ярыми
защитниками.

Признаю, я предвзят, но мне кажется, что агрессивные конвенционалисты
несут ответственность за огромное количество неприятностей в мире. Я
считаю, что многие обычаи, сформировавшиеся еще со времен Просвещения,
были созданы именно для того, чтобы защитить общество от агрессивных
конвенционалистов. В частности, можно вспомнить отказ от концепции
ереси и ее замену принципом свободного обсуждения всевозможных идей.
Этот принцип касался даже тех идей, которые в настоящее время
считаются неприемлемыми, а также исключал наказание тех, кто проверял
жизнеспособность этих идей на практике. [2]

Зачем защищать людей с независимым мышлением? Затем, что все новые
идеи живут именно в их головах. Чтобы быть успешным ученым, например,
недостаточно просто быть правым. Нужно быть правым, когда ошибаются
все остальные. Люди с традиционным мышлением так не могут. Именно
поэтому все руководители успешных стартапов не просто обладают
независимым мышлением, их разум может проявлять определенную агрессию.
Тот факт, что общества процветают только в той степени, в которой у
них развит обычай держать на расстоянии людей с традиционным
мышлением, не случаен. [3]

Многим из нас стало заметно, что за последние несколько лет ослабли
обычаи, защищающие свободное мышление. Некоторые говорят, что мы
слишком остро реагируем – якобы они не ослабли, либо ослабли, но это
поможет в достижении великой цели. Сразу опровергну последнее
утверждение. Когда обыватели берут верх, они всегда говорят, что все
делают во имя великого блага. Просто каждый раз оказывается, что речь
идет о совсем другом, неуместном великом благе.

Что касается остальных тезисов, касающихся беспокойства о
чувствительности людей с независимыми взглядами и ограничений свободы
мысли – об этом нельзя судить, если вы сами не обладаете независимым
умом. Вы не можете оценить количество ущемленных идей, если у вас их
нет. Только свободно мыслящие люди могут выдавать передовые идеи.
Именно поэтому они, как правило, очень чувствительны к переменам в
плане свободы мысли. Эти люди – канарейки в шахте.

Люди с традиционным мышлением говорят, что они не хотят сворачивать
обсуждение вообще всех идей, они хотят исключить лишь плохие идеи.

Вы можете подумать, что по одной этой фразе ясно, в какую опасную игру
они играют. Я все равно объясню. Существует две причины, объясняющие,
почему мы должны иметь возможность обсуждать даже «плохие» идеи.

Во-первых, любой способ решения вопроса о том, какие идеи запретить,
неизбежно приведет к ошибкам. Тем более, что никто из умных не хочет
заниматься подобной работой, а значит, она в итоге ее будут выполнять
люди глупые. Если какой-либо процесс может приводить к множеству
ошибок, нужно оставлять пространство для маневра. В данном случае это
значит, что нужно запрещать меньше идей, чем хочется. Агрессивно
настроенные конвенционалисты так не могут. Отчасти потому, что им
нравится видеть, как людей наказывают – это у них с детства. Отчасти
потому, что они соревнуются друг с другом. Сторонники ортодоксальности
не могут допустить, чтобы существовали какие-либо пограничные идеи,
потому что это дает другим людям возможность потеснить их в плане
нравственной чистоты, возможно эти стороны даже будут стравлены. В
итоге вместо пространства для маневра мы летим на дно, и на этом дне
запрещают все, что можно в принципе запретить. [4]

Вторая причина опасности запрета обсуждения идей заключается в том,
что идеи связаны более тесно, чем кажется. Это значит, что ограничение
обсуждения некоторых тем, может повлиять не только на эти темы.
Ограничения распространятся на все темы, связанные с запрещенной. И
это даже не краевая ситуация. Именно так и бывает с лучшими идеями:
они влияют на сферы, очень далекие от изначальных. Вынашивать идеи в
мире, где запрещены какие-либо другие идеи, это как играть в футбол на
поле, где перед одними из ворот есть минное поле. Вы не просто играете
в ту же самую игру на чуть другом поле. Вы играете в очень опасную
игру на поле, кажущемся безопасным.

Раньше независимые люди защищали себя, собираясь в определенных
учреждениях – сначала в судах, а затем в университетах. Там они могли
в какой-то степени устанавливать свои собственные правила. В местах,
где люди работают с идеями, как правило, есть традиции, защищающие
свободу мысли. Это столь же очевидно, как мощные воздушные фильтры на
фабриках по производству микрочипов или шумоизоляция в студиях
звукозаписи. Когда агрессивные обыватели были чем-то взбешены,
последние пару столетий от них можно было укрыться в университетах.

Сейчас, к сожалению, этот трюк может не сработать – последняя волна
нетерпимости зародилась именно в университетах. Она началась в
середине 1980-х годов, к 2000 году, как казалось, умерла, но в
последнее время она вновь вспыхнула – с появлением социальных сетей. К
сожалению, похоже что именно этого добивались в Кремниевой Долине.
Несмотря на то, что люди, управляющие Кремниевой Долиной, в
большинстве своем независимы, они передали агрессивно настроенным
обывателям инструмент, о котором они могли только мечтать.

С другой стороны, возможно упадок духа свободы мысли в университетах –
это и симптом, и причина ухода множества мыслящих свободно людей.

У людей, которые 50 лет назад стали бы профессорами, сейчас есть и
другие возможности. Теперь они могут стать аналитиками или взяться за
стартапы. Обе эти сферы требуют умения мыслить независимо. Если бы эти
люди были профессорами, они бы оказали более сильное сопротивление в
борьбе за академическую свободу. Возможно, картина с бегством людей со
свободой мысли из университетов, находящихся в упадке, слишком мрачна.
Возможно, университеты идут под уклон, потому что многие из этих людей
уже ушли. [5]

Несмотря на то, что я потратил много времени на размышления об этой
ситуации, я не могу предсказать итог. Смогут ли университеты
переломить тенденцию и остаться учреждениями, где захотят собираться
свободомыслящие люди? Или эти люди постепенно от них откажутся? Если
это произойдет, то я боюсь представить потери.

Как бы то ни было, я храню надежду в долгосрочной перспективе. Люди с
независимым мышлением могут себя защитить. Если существующие институты
будут скомпрометированы, эти люди создадут новые. Для этого
потребуется воображение. Но воображение — это, в конце концов, их
специализация.

Примечания

[1] Конечно, я понимаю, что если человеческие качества отличаются в
любых двух отношениях, вы можете использовать их как оси и вывести
результирующие четыре квадранта типов личностей. Таким образом, я на
самом деле утверждаю, что оси ортогональны, и в обоих имеется
значительная вариативность.

[2] Агрессивно настроенные обыватели не несут ответственности за все
неприятности в мире. Куда больший источник неприятностей — это некий
харизматичный лидер, который приходит к власти, обращаясь к ним. Когда
появляется такой лидер, обыватели становятся намного опаснее.

[3] Когда я управлял Y Combinator, я никогда не боялся писать тексты,
которые оскорбляли обывателей. Если бы YC была компанией по
производству печенья, я бы столкнулся с трудным моральным выбором.
Обычные люди тоже едят печенье. Но они не запускают успешные стартапы.
Если бы я отговаривал их от вступления в YC, нам просто пришлось бы
читать меньше заявок.

[4] В одной области был замечен прогресс: наказания за разговоры о
запрещенных идеях менее суровы, чем в прошлом. Опасность быть убитым
невелика, по крайней мере, в более богатых странах. Агрессивно
настроенные обыватели в основном удовлетворены тем, что неугодных им
увольняют.

[5] Многие профессора независимы – особенно в математике, точных и
инженерных науках. Тем не менее, более точно положение дел можно
понять по студентам, представляющим большую часть общества (а значит,
среди них больше людей с традиционным укладом мысли). Таким образом,
конфликт между студентами и преподавателями по сути представляет собой
конфликт между людьми разных классов и типов.

\end{document}
