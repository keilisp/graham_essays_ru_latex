\documentclass[ebook,12pt,oneside,openany]{memoir}
\usepackage[utf8x]{inputenc} \usepackage[russian]{babel}
\usepackage[papersize={90mm,120mm}, margin=2mm]{geometry}
\sloppy
\usepackage{url} \title{Чем жизнь творца отличается от жизни
  менеджера} \author{Пол Грэм} \date{}
\begin{document}
\maketitle

Программисты так не любят совещания в том числе потому, что у них
совершенно особый график, отличающийся от графика других людей.
Совещания требуют от них большего.

Есть два типа графика, которые я называю графиком менеджера и графиком
творца. График менеджера – это для начальников. Оно воплощается в
традиционных органайзерах, разрезающих день на часовые интервалы. Вы
можете при необходимости выделить на какую-то задачу несколько часов,
но по умолчанию вы определяете, что будете делать в каждый из этих
часов.

Когда вы пользуетесь временем таким образом, встретиться с кем-то –
это чисто технический вопрос. Найдите в графике свободное место,
назначьте встречу, и все готово.

Большинство людей, обладающих властью, живут именно по такому графику.
Это административный график. Но есть и другой способ использования
времени, распространенный среди людей, которые что-то создают –
например, программистов или писателей. Они обычно предпочитают
использовать время, измеряя его блоками как минимум в полдня.
Невозможно писать или программировать, укладываясь в отрезки по часу.
За это время едва успеваешь разогнаться.

Когда вы живете в графике творца, совещания – это катастрофа. Одна
встреча может отъесть у вас полдня, разбивая время на два кусочка,
слишком мелких, чтобы что-то успеть. И еще вам нужно не забыть пойти
на совещание. Это не проблема для тех, кто живет по менеджерскому
графику. У них всегда что-то происходит в следующий час, вопрос только
– что именно. Но когда совещание назначено для кого-то, кто живет по
графику творца, ему нужно об этом подумать.

Для человека, живущего по графику творца, пойти на совещание – все
равно что сделать исключение. Это значит не просто переключиться от
одной задачи к другой. Это значит изменить режим, в котором вы
работаете.

Иногда совещание может повлиять на весь день. Совещание обычно
отнимает полдня, разбивая на части утро или послеобеденное время. Но в
дополнение к этому иногда возникает эффект домино. Если я знаю, что
мое утро будет поделено на части, я с меньшей вероятностью начну
делать что-то масштабное с утра. Я знаю, что это выглядит излишней
чувствительностью, но если вы творец, подумайте о своей ситуации.
Разве вы не испытываете волнение при мысли, что у вас есть целый день
на то, чтобы поработать, и никаких совещаний и встреч? И это значит,
что когда этого не получается, ваше настроение портится. А масштабные
проекты всегда требуют от вас работы на пределе возможностей.
Небольшого падения духа достаточно, чтобы подорвать их.

Каждый из этих графиков, взятый сам по себе, вполне нормален. Проблемы
начинаются, когда графики сталкиваются. Поскольку самые влиятельные
люди живут в графике менеджера, они могут позволить себе заставить
всех остальных существовать по этому графику. Но более умные
руководители воздерживаются от этого, если знают, что некоторым из их
сотрудников на работу нужны длинные блоки времени.

Наш случай – необычный. Почти все инвесторы и все венчурные инвесторы,
которых я знаю, живут по расписанию менеджеров. Но Y Combinator
работает по графику творца. Я не удивлюсь, если вскоре появится больше
компаний, работающих так же. Я подозреваю, что предприниматели все
больше будут сопротивляться своему превращению в менеджеров.

Как же нам удается работать с такой массой стартапов, действуя по
графику творца? Мы опираемся на классический инструмент, симулирующий
график менеджера внутри графика творца: офисный рабочий день.
Несколько раз в неделю я выделяю значительный кусок времени, чтобы
повстречаться с предпринимателями, которых мы финансируем. Это время
выделяется в конце моего рабочего дня, и я написал специальную
программу, с помощью которой все назначаемые встречи скапливаются к
концу дня. И поэтому встречи никогда не отвлекают меня от других дел.
Иногда это время офисных встреч затягивается, но оно никогда не
отрывает меня от другой работы.

Когда мы работали над нашим собственным стартапом, еще в 1990-х, я
изобрел еще один трюк, позволяющий делить день. Я программировал с
ужина до примерно 3 часов утра каждый день, потому что вечером никто
не отвлекал меня. Потом я спал до 11 часов утра, приходил в офис и до
ужина работал над «бизнес-делами». Я никогда не рассматривал это в
таком ключе, но по сути, каждый день у меня было два рабочих дня, один
по расписанию менеджера, другой – по расписанию творца.

Когда вы действуете по графику менеджера, вы можете заниматься тем,
чего не могли бы себе позволить в графике творца: проводить встречи
без конкретного результата, просто чтобы познакомиться с кем-то. Если
в вашем графике появилось пустое место, почему нет? Может быть,
окажется, что вы поможете друг другу как-то иначе.

Деловые люди в Кремниевой долине (да и во всем мире) проводят такие
встречи каждый день. Если вы живете по графику менеджера, то такие
встречи, по сути, ничего вам не стоят. Они настолько распространены,
что когда предлагают встретиться таким образом, используют особые
слова – например, «пойти выпить кофе».

Однако если вы живете по графику творца, такие встречи обходятся вам
исключительно дорого. И это нас в каком-то смысле ограничивает. Все
предполагают, что мы, как и другие инвесторы, живем по графику
менеджера. Так что они представляют нас кому-то, с кем нам, по их
мнению, стоит встретиться, или отправляют нам письмо с предложением
попить кофе. В этот момент перед нами встает выбор, и оба варианта не
очень радуют: мы можем встретиться с человеком и потерять полдня, или
же можем избежать встречи и, вероятно, обидеть человека.

До недавнего времени мы не понимали в полной мере, в чем проблема. Мы
просто принимали за данность, что приходится или разрывать график, или
обижать людей. Но теперь, когда я понял, что происходит, возможно,
есть третий вариант: написать некий текст, объясняющий эти два
графика. Может быть, со временем, если конфликт между графиком
менеджера и графиком творца начнут понимать лучше, это станет менее
серьезной проблемой.

Те из нас, кто живет по графику творца, склонны к компромиссам. Мы
знаем, что какое-то количество совещаний нам все равно придется
отсидеть. Все, о чем мы просим тех, кто живет по графику менеджера –
чтобы они понимали цену этих совещаний.

\end{document}
