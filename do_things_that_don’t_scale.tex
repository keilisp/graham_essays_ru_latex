\documentclass[ebook,12pt,oneside,openany]{memoir}
\usepackage[utf8x]{inputenc} \usepackage[russian]{babel}
\usepackage[papersize={90mm,120mm}, margin=2mm]{geometry} \sloppy
\usepackage{url} \title{Делайте вещи, которые не масштабируются}
\author{Пол Грэм} \date{}
\begin{document}
\maketitle

Один из наиболее универсальных советов, которые мы даем в Y
Combinator, это браться за сложную работу. Многие начинающие
основатели верят, что стартапы или «взлетают» или нет. Вы создаете
что-то, делаете это доступным, и, если вы придумали самую лучшую
мышеловку, люди, как и было обещано, сами придут к вам. Или не придут,
в таком случае у вас нет рынка. [1] \newline

На самом деле стартапы взлетают, потому что основатели заставляют их
взлетать. Существует лишь небольшое количество стартапов, которые
выросли сами, поскольку обычно для их раскачки требуются определенные
усилия. Здесь можно провести сравнение с заводной ручкой, которой
комплектовались автомобили до появления электрических стартеров. Если
двигатель заводили, он работал, однако запуск представлял собой
отдельный трудоемкий процесс. \newline

\subsection{Привлечение клиентов}

Самая распространенная сложная работа, за которую основатели должны
браться на старте, это самостоятельное привлечение пользователей. Этим
должны заниматься почти все стартапы. Нельзя ждать, когда пользователи
к вам придут. Вы должны сами пойти и привести их. \newline

Stripe – один из самых успешных стартапов, которые мы финансировали, и
проблема, которую они решили, была неотложной. Если кому и можно было
сидеть и ждать пользователей, так это Stripe. Однако в рамках Y
Combinator этот проект известен своим агрессивным привлечением
пользователей на ранней стадии. \newline

Стартапы, которые создают что-то для других стартапов, имеют большой
пул потенциальных клиентов других компаний, которые мы финансируем, и
никто не воспользовался этим лучше, чем Stripe. В Y Combinator мы
называем придуманный ими метод «установкой Коллисонов». Более робкие
основатели спрашивают: «Не хотите ли попробовать нашу бета-версию?» и
в случае положительного ответа говорят: «Отлично, мы пришлем вам
ссылку». Но братья Коллисоны ждать не собирались. Как только кто-то
соглашался попробовать Stripe, они говорили: «Отлично, а теперь дай-ка
мне свой ноутбук» и сразу же производили установку. \newline

Есть две причины, по которым основатели отказываются выходить в народ
и привлекать пользователей индивидуально. Первая – это сочетание
застенчивости и лени. Лучше сидеть дома и писать код, чем выйти и
разговаривать с кучей незнакомцев и, вероятнее всего, получить отказ
от большинства из них. Однако для успеха стартапа по меньшей мере один
основатель (обычно исполнительный директор) должен потратить много
времени на продажи и маркетинг. [2] \newline

Вторая причина, по которой основатели игнорируют этот путь, это то,
что конкретные цифры кажутся поначалу слишком маленькими. Крупные
известные стартапы не могли так начинать, думают они. И делают ошибку,
недооценивая силу совокупного роста. Мы просим каждый стартап измерять
прогресс недельными темпами роста. Если у вас 100 пользователей, вам
нужно получить еще 10 за следующую неделю, чтобы темпы роста составили
10\% в неделю. И даже если 110 кажутся не намного лучше, чем 100,
поддержав темпы роста на отметке 10\% в неделю, вы удивитесь, какие
большие цифры в итоге получите. Через год у вас будет 14000
пользователей, а через два – 2 миллиона. \newline

Вы будете действовать совершенно по-другому, когда будете получать
пользователей тысячами, и темпы роста постепенно замедлятся. Но если
существует рынок, то обычно можно начать привлекать пользователей
вручную и потом постепенно переключиться автоматизированные методы.
[3] \newline

Airbnb – классический пример этого метода. Рынок настолько сложно
сдвинуть с мертвой точки, что сначала нужно быть готовыми к
героическим подвигам. В случае Airbnb это было хождение по домам в
Нью-Йорке, привлечение новых пользователей и помощь уже существующим.
Я вспоминаю Airbnb во времена Y Combinator: вечно с сумками на
колесах, потому что на ужины по вторникам они постоянно откуда-то
прилетали. \newline

\subsection{Понимание хрупкости стартапа}

Сейчас Airbnb кажется силой, которую невозможно остановить, однако в
самом начале проект был настолько хрупким, что его успех или провал
зависел от тех приблизительно 30 дней хождения по квартирам с целью
индивидуального привлечения пользователей. \newline

Изначальная хрупкость была присуща не только Airbnb. Почти все
стартапы поначалу хрупкие. И это одна из главных вещей, которую
неопытные основатели и инвесторы (а также журналисты и всезнайки на
форумах) истолковывают неправильно, несознательно оценивая стартапы в
зачаточной стадии по стандартам уже сформировавшихся стартапов. Это
все равно что посмотреть на новорожденного и сказать: «Это крошечное
создание никогда не сможет ничего достичь». \newline

Нестрашно, если журналисты и всезнайки не верят в ваш стартап. Они
всегда все понимают неправильно. Также нормально, если инвесторы не
верят в ваш стартап, они изменят свое мнение, когда увидят ваш рост.
Самое опасное, если вы сами не верите в свой стартап. Мне часто
приходится подбадривать основателей, которые не видят полный потенциал
того, что создают. Даже Билл Гейтс совершил такую ошибку. Он вернулся
в Гарвард на осенний семестр после основания Microsoft. И хотя остался
он там ненадолго, мог бы вообще не возвращаться, если бы понимал, что
Microsoft станет пусть даже долей того, чем он стал. [4] \newline

На ранней стадии стартапа следует спрашивать не «Завоюет ли компания
мир?», а «Насколько крупной может стать компания, если ее основатели
все сделают правильно?». А правильные шаги могут одновременно казаться
трудоемкими и нелогичными. Microsoft не сильно впечатлял, когда это
были всего лишь пару парней в Альбукерке, которые писали
интерпретаторы Basic для рынка из нескольких тысяч «хоббистов» (как их
тогда называли), однако, если оглянуться назад, это был оптимальный
путь, который позволил компании занять доминирующую позицию на рынке
программного обеспечения для персональных компьютеров. И я знаю, что
Брайан Чески и Джо Геббия (основатели Airbnb) не думали, что находятся
на пути к большому успеху, когда делали «профессиональные» снимки
первых сдающихся квартир. Они просто пытались выжить. Но, как показало
время, это тоже был верный путь, который позволил занять доминирующую
позицию на огромном рынке. \newline

Как находить пользователей для привлечения вручную? Если вы создаете
что-то, что решает ваши собственные проблемы, тогда вам необходимо
лишь искать подобных себе, что обычно довольно просто. В остальных
случаях необходимо приложить более продуманные усилия и найти самый
перспективный поток пользователей. Обычно можно получить первую группу
пользователей, сделав сравнительно нецелевой запуск, а дальше
пронаблюдать, кто проявит больше энтузиазма и искать подобных им.
Например, Бен Зильберман заметил, что многие из первых пользователи
Pinterest интересовались дизайном, поэтому для привлечения
пользователей он посетил конференцию блогеров, пишущих о дизайне, и
это отлично сработало. [5] \newline

\subsection{Удовольствие для пользователя}

Необходимо делать все возможное не только для того, чтобы получить
пользователей, но также чтобы они чувствовали себя счастливыми. Wufoo,
например, посылали каждому новому пользователю написанное от руки
письмо с благодарностью столько времени, сколько могли (получилось на
удивление долго). Ваши первые пользователи должны чувствовать, что,
зарегистрировавшись у вас, сделали правильный выбор. А вы в свою
очередь должны находить новые способы их порадовать. \newline

Почему нам приходится учить этому стартапы? Почему основатели не
понимают этого сами? Я думаю, по трем причинам. \newline

Во-первых, многие основатели стартапов прошли подготовку как
разработчики, а работа с клиентами в такую подготовку не входит. От
вас ожидают разработки чего-то надежного и прекрасного, а не
раболепского внимания к каждому пользователю, как от продажника. Как
ни странно, разработка противопоставляется наставничеству, и эта
традиция возникла во времена, когда разработчики были менее
влиятельными и стояли лишь во главе своей узкой области разработки, а
не командовали парадом. Можно быть упрямым, если вы Скотти, но не
Кирк. \newline

Вторая причина, по которой основатели не уделяют должного внимания
отдельно взятым клиентам, это опасения, что такая тактика не
сработает. Когда основатели стартапов в зачаточной стадии выражают
такое беспокойство, я объясняю, что в текущем состоянии им нечего
терять. Может, если они приложат особые усилия, чтобы сделать
существующих пользователей суперсчастливыми, в один прекрасный день
пользователей станет так много, что они уже не смогут продолжать это
делать. Было бы замечательно столкнуться с такой проблемой. И когда
такое вдруг случится, вы поймете, что радовать пользователей проще,
чем казалось. Частично потому, что обычно можно найти более простые
способы, чем предполагалось изначально, а частично потому, что желание
радовать пользователей к тому времени прочно войдет в вашу культуру.
\newline

Я никогда не видел стартап, который бы зашел в тупик из-за того, что
слишком усердно пытался осчастливить своих первых пользователей.
\newline

Но самой главной причиной, по которой основатели не понимают,
насколько внимательными они могли бы быть к своим пользователям,
вероятно является то, что они сами никогда не испытывали такого
внимания. Их стандарты работы с клиентами сформировались под влиянием
компаний, клиентами которых они являются, а это в основном крупные
компании. Тим Кук не посылает вам написанную от руки благодарность за
покупку ноутбука. Он не может. А вы можете! Это одно из преимуществ
маленькой компании: вы можете обеспечить уровень обслуживания, который
не под силу обеспечить ни одной крупной компании. [6] \newline

Как только вы поймете, что взаимодействие с пользователем не
ограничивается существующими традициями, стоит подумать о том, как
далеко вы можете зайти, пытаясь порадовать своих пользователей.
\newline

\subsection{Опыт пользователя}

Я пытался придумать словосочетание, которое бы передавало, насколько
чрезмерным должно быть внимание к пользователю, и понял, что это уже
сделал Стив Джобс: оно должно быть безумно великим (insanely great).
Стив использовал слово «безумно» не как синоним слова «очень». Он в
буквальном смысле имел в виду, что необходимо сконцентрироваться на
качестве исполнения в такой степени, которая в повседневной жизни
считалась бы паталогической. \newline

Со всеми самыми успешными стартапами, которые мы финансировали, так и
было, и это вероятно не удивляет потенциальных основателей. Что
основатели-новички не понимают, так это во что превращает стремление к
«безумно великому» зачаточный стартап. Когда Стив Джобс начал
использовать это выражение, Apple уже был состоявшейся компанией. Стив
имел ввиду Mac (включая его документацию и даже упаковку – такова
природа одержимости), который должен был стать безумно великим в плане
задумки и исполнения. Разработчикам это понять несложно. Это просто
более бескомпромиссная версия разработки надежного и прекрасного
продукта. \newline

Основателям сложно понять (может Стиву самому было сложно это понять),
что дает «безумно великое» в первые несколько месяцев жизни стартапа.
Не продукт должен быть безумно великим, а user-experience ваших
пользователей. Для большой компании продукт — доминирующая
составляющая. Но вы можете и должны обеспечивать пользователям безумно
великий опыт с ранним, незавершенным и несовершенным продуктом,
компенсируя недоработки вниманием. \newline

Возможно, это можно сделать, а нужно ли? Да. Чрезмерное внимание к
первым пользователям – не просто допустимый метод стимуляции роста.
Для большинства успешных стартапов – это необходимый элемент
взаимосвязи, необходимой для получения хорошего продукта. Создание
лучшей мышеловки (ловушки для клиентов) – не неделимая операция. Даже
если вы начнете подобно самым успешным стартапам и создадите что-то,
что нужно вам самим, первый ваш продукт никогда не будет достаточно
правильным. Кроме областей, где ошибки стоят слишком много, часто
лучше не настраиваться на совершенство с самого начала. Особенно это
касается программного обеспечения: обычно лучше представить
пользователям продукт, как только он будет иметь некоторую полезность,
и далее посмотреть, что они будут с ним делать. Перфекционизм часто
является оправданием затягивания, в любом случае первая модель
пользователей всегда неточная, даже если выявляетесь одним из них. [7]
\newline

Отзывы, которые вы получаете, общаясь напрямую со своими первыми
пользователями, будут самыми полезными. Став крупной компанией, вы
будете вынуждены использовать фокус-группы и жалеть о том, что не
можете прийти к своим пользователям домой или на работу и посмотреть,
как они используют ваш продукт, как вы делали, когда этих
пользователей было совсем мало. \newline

\subsection{Сдерживание пламени}

Иногда правильным будет сконцентрироваться на очень узком рынке. Это
подобно разжиганию костра: сначала вы разводите небольшой огонь и
даете ему разгореться, а потом добавляете больше дров. \newline

Именно так было с Facebook. Сначала он предназначался лишь для
студентов Гарварда. В той форме его потенциальным рынком было лишь
несколько тысяч человек, но, поскольку ресурс действительно
удовлетворял их потребности, зарегистрировалась критическая масса.
Далее на некоторое время к Гарварду добавились определенные колледжи.
Когда я разговаривал с Марком Цукербергом в Школе стартапов, он
отметил, что составление списка курсов для каждого колледжа было
трудоемким занятием, но именно благодаря ему студенты чувствовали себя
на сайте как дома. \newline

Любой стартап, который можно описать как рынок, обычно начинается в
каком-либо сегменте рынка, но это может работать и для других
стартапов. Всегда стоит задуматься о том, существует ли сегмент рынка,
где можно быстро набрать критическую массу пользователей. [8] \newline

Большинство стартапов, которые используют стратегию сдерживания
пламени, делают это несознательно. Они создают что-то для себя и своих
друзей, которые становятся первыми клиентами, и только позже понимают,
что могут предложить свой продукт более широкому рынку. Стратегия
работает также хорошо при ее неосознанном применении. Однако самая
большая опасность неосознанного применения этой модели лежит в наивном
отказе от ее части. Например, если вы не создаете что-то для себя и
своих друзей, или даже если создаете, но пришли из корпоративного
мира, и ваши друзья не являются вашими первыми клиентами, вы больше не
получаете идеальный первоначальный рынок на блюдечке с голубой
каемочкой. \newline

Среди компаний лучшими первыми клиентами обычно являются другие
стартапы. Они более открыты новому по своей природе и, только начав
работу, еще не успели сделать выбор по всем пунктам. Плюс, если они
окажутся успешными и начнут быстро расти, вы будете расти вместе с
ними. Одним из многих непредусмотренных преимущества модели Y
Combinator (и особенно расширения Y Combinator) стало то, что B2B
стартапы сейчас имеют под рукой мгновенный рынок из сотен других
стартапов. \newline

\subsection{Путь Meraki}

Стартапы аппаратного обеспечения могут поступать подобно Meraki. И
хотя мы не финансировали Meraki, основателями были выпускники Роберта
Морриса, поэтому мы знаем их историю. Они начали действительно со
сложного: сами собирали свои роутеры. \newline

Стартапы аппаратного обеспечения сталкиваются с препятствием, которое
отсутствует на пути у программных проектов. Стоимость минимального
заказа для отправки на завод обычно составляет несколько сотен тысяч
долларов. Это может загнать вас в ловушку-22: без продукта невозможно
обеспечить рост, необходимый для привлечения средств для производства
продукта. Раньше, когда стартапы аппаратного обеспечения вынуждены
были полагаться на деньги инвесторов, нужно было быть очень
убедительным, чтобы преодолеть это препятствие. Возникновение
краудфандинга (или точнее предварительных заказов) очень помогло. Но
все равно я советую стартапам по возможности поступать как Meraki. Это
сделали Pebble. Они собрали первые несколько сотен часов
самостоятельно. Если бы не было этой стадии, они, наверное, не продали
бы часов на 10 миллионов долларов, оказавшись на Kickstarter. \newline

Как и чрезмерное внимание к первым клиентам, самостоятельное
производство оказывается ценным для стартапов аппаратного обеспечения.
Вы можете быстрее изменить дизайн, и вы узнаете то, чего другим
образом не смогли бы узнать. Эрик Мигиковски из Pebble рассказал, что
среди того, что он узнал, было «насколько важно иметь хорошие винты».
Кто бы мог подумать? \newline

\subsection{Консультирование}

Иногда мы советуем основателям B2B стартапов доводить внимание к
клиентам до крайности: выбрать одного пользователя и действовать так,
будто они консультанты, создающие что-то только для него одного.
Первый пользователь служит формой для отлива, продолжайте ее
корректировать, пока не будете идеально соответствовать его
потребностям, так обычно можно получить то, чем захотят
воспользоваться другие пользователи. Если даже их будет не много,
вероятно существуют смежные рынки, где можно будет найти больше. Если
вы можете найти одного пользователя, которому действительно что-то
нужно, и будете работать над удовлетворением этой потребности, вы
будете работать над тем, что нужно людям, именно это изначально должен
делать стартап. [9] \newline

Консультирование представляет собой типичный пример сложной работы.
Оно (как и другие способы проявления внимания) безопасно, пока вы не
получаете за это деньги. Именно здесь компании переступают черту. Пока
вы являетесь продуктовой компанией, которая просто чересчур
внимательна к клиенту, они вам благодарны, даже если вы не решаете все
их проблемы. Но как только за эту внимательность вам начинают платить,
от вас ожидают, что вы будете делать все. \newline

Еще один подобный консультированию метод привлечения изначально
неохотных пользователей – использование собственного ПО от их имени.
Мы делали это в Viaweb. Когда мы общались с продавцами и спрашивали,
хотели бы они использовать наше ПО для создания онлайн-магазинов,
некоторые отвечали, что не хотели бы, но позволяли нам создавать такие
онлайн-магазины для них. Поскольку мы были готовы на все для получения
пользователей, мы делали это. Мы чувствовали себя довольно паршиво в
то время. Вместо организации крупных стратегических партнерств в сфере
электронной коммерции мы пытались продавать чемоданы, ручки и мужские
рубашки. Сейчас мы понимаем, что тогда это было правильным занятием,
потому что благодаря ему мы поняли, как бы чувствовали себя продавцы,
используя наше ПО. Иногда обратная связь была почти мгновенной:
создавая сайт какого-либо магазина, я понимал, что необходима функция,
которой у нас нет, поэтому я тратил несколько часов на ее реализацию и
продолжал строить сайт. \newline

\subsection{Работа вручную}

Есть более крайний вариант, когда вы не просто используете свое ПО, а
сами работаете в качество ПО. Имея лишь небольшое количество
пользователей, иногда можно делать вручную то, что планируется
автоматизировать позже. Это позволяет запуститься быстрее, и когда вы
в конце концов будете автоматизировать процесс, вы будете точно знать,
что делать, потому что самостоятельное выполнение этих действий
обеспечит вам мышечную память. \newline

Когда ручные компоненты кажутся пользователю программными, это
начинает напоминать злую шутку. Например, когда Stripe раздавал
«мгновенные» аккаунты своим первым пользователям, основатели просто
вручную тайно регистрировали обычные аккаунты. \newline

Некоторые стартапы на начальном этапе могут быть полностью ручными.
Можно найти кого-то, перед кем стоит проблема, требующая решения,
решить эту проблему вручную и продолжать это делать настолько долго,
насколько это возможно, постепенно автоматизируя узкие места. Вас
может немного пугать то, что проблемы пользователей решаются еще не
автоматизированным способом, однако это не так страшно, как более
распространенный случай наличия автоматической функции, которая еще не
решает ничьи проблемы. \newline

\subsection{Отказ от громких запусков}

Следует отметить одну начальную тактику, которая обычно не работает:
громкий запуск. Я иногда встречаю основателей, которые верят, что
стартапы представляют собой неуправляемые ракеты, а не пилотируемые
самолеты, и что успеха можно добиться, только если запуск будет
произведен с достаточной начальной скоростью. Они желают получить
публикации сразу в 8 различных изданиях. И запуститься, конечно же, во
вторник, потому что они где-то читали, что это лучший день для
запуска. \newline

Понять, насколько мало значат запуски, очень просто. Подумайте о
каких-либо успешных стартапах. Сколько из их запусков вы помните? Все,
для чего нужен запуск, это некоторая начальная база пользователей.
Насколько хорошо пойдут дела через несколько месяцев зависит от того,
насколько счастливыми вы сделаете этих пользователей и сколько их
было. [10] \newline

Так почему основатели думают, что запуск имеет значение? Это сочетание
эгоизма и лени. Они думают, что то, что они создают, настолько
великое, что каждый, кто про это услышит, сразу же зарегистрируется.
Кроме того, работы будет существенно меньше, если можно будет получать
пользователей, просто заявляя о своем существовании, а не привлекая
каждого отдельно. Даже если вы действительно создаете что-то великое,
получение пользователей всегда будет процессом постепенным отчасти
потому, что великое обычно еще и новое, но в основном потому, что
пользователям есть о чем думать кроме вашего продукта. \newline

Партнерства также обычно не работают. Они не работают для стартапов в
общем, но особенно не работают, если их используют для стимулирования
роста. Распространенная ошибка неопытных основателей – вера в то, что
партнерство с крупной компанией обеспечит крупный прорыв. Через шесть
месяцев они все говорят одно и то же: работы оказалось намного больше,
чем мы ожидали, а в итоге мы практически ничего от этого не получили.
[11] \newline

Недостаточно просто создать что-то экстраординарное с самого начала. С
самого начала нужно приложить экстраординарные усилия. Любая
стратегия, которая не предполагает усилий – ожидание громкого запуска
для получения пользователей или крупного партнера – подозрительна в
силу самого факта. \newline

\subsection{Векторный подход}

Необходимость делать что-то пугающе трудоемкое в самом начале почти
универсальна, поэтому возможно стоит перестать рассматривать идеи
стартапов как скаляры. Вместо этого следует попробовать рассматривать
их как сочетание того, что вы собираетесь создать, и сложной работы,
которую вы собираетесь проделать, чтобы заставить компанию работать.
\newline

Такой подход к идеям стартапов может быть интересным, потому что
теперь, когда есть два компонента, можно постараться творчески подойти
как ко второму, так и к первому. Однако в большинстве случаев второй
компонент будет тем, чем он обычно является: ручным привлечением
пользователей и обеспечением им необыкновенно приятного опыта.
Основная польза векторного подхода к стартапам – это напоминание
основателям, что они должны усиленно работать в двух направлениях.
[12] \newline

В лучшем случае оба компонента вектора станут ценным вкладом в ДНК
вашей компании: сложная работа, которую вы должны проделать на старте,
это не неизбежное зло, а необратимые перемены к лучшему. Если вам
приходится агрессивно привлекать пользователей, пока вы еще малы,
вероятно, став крупными, вы не перестанете быть агрессивными. Если вам
приходится производить собственное аппаратное обеспечение или
использовать свое программное обеспечение от имени пользователей, вы
научитесь тому, чему в другой ситуации не смогли бы научиться. И самое
важное, если вам приходится тяжело работать, чтобы радовать
пользователей, когда у вас их совсем немного, вы продолжите это
делать, когда их число станет большим. \newline

\subsection{Примечания}

[1] На самом деле Эмерсон никогда конкретно не упоминал мышеловки. Он
писал: «Если у человека есть хорошее зерно, древесина, доски или
свиньи на продажу или он может делать стулья, ножи, котлы или
церковные органы лучше, чем кто-либо другой, к его дому будет вести
широкая протоптанная дорога, даже если этот дом стоит в лесу».
\newline

[2] Спасибо Сэму Альтману за совет, я скажу прямо. Нет, вы не можете
избежать занятия продажами, наняв кого-то, кто будет это делать за
вас. Сначала вы сами должны заниматься продажами. Впоследствии можно
будет нанять реального продажника, который вас заменит. \newline

[3] Это работает потому, что, по мере того как вы растете, ваш размер
помогает вам расти. Патрик Коллисон написал: «В какой-то момент
произошла очень заметная перемена в самоощущении Stripe. Он перестал
быть той глыбой, которую мы должны были толкать, и стал вагоном,
который движется по инерции». \newline

[4] Одним из более хитрых способов, которым Y Combinator может помочь
основателям, является корректировка амбиций, потому что мы точно
знаем, как выглядели многие успешные стартапы, когда только начинали.
\newline

[5] Если вы создаете что-то, для чего не можете собрать небольшую
группу пользователей, за которыми сможете наблюдать (например,
корпоративное ПО), и в сфере, в которой у вас нет связей, вам придется
положиться на холодные звонки и рекомендации. Но стоит ли вообще
браться за такую идею? \newline

[6] Гарри Тэн отметил интересные ловушки, в которые попадают
основатели на старте. Они настолько хотят казаться крупными, что
имитируют даже недостатки крупных компаний, такие как безразличие к
индивидуальным пользователям. Это кажется им более «профессиональным».
На самом деле лучше смириться с фактом того, что вы малы, и
использовать все преимущества этой ситуации. \newline

[7] Ваша пользовательская модель почти никогда не может быть идеально
точной, потому что потребности пользователей часто меняются в ответ на
то, что вы для них создаете. Создайте для них компьютер, и им вдруг
захочется работать на нем с электронными таблицами, потому что
благодаря появлению вашего компьютера кто-то изобрел электронные
таблицы. \newline

[8] Если вам необходимо выбрать между группой, которая
зарегистрируется быстрее всех, и группой, которая будет платить
больше, обычно лучше выбрать первую, потому что вероятно эти люди
станут вашими первыми клиентами. Они лучше повлияют на ваш продукт, и
вам не придется тратить так много усилий на продажи. У них меньше
денег, но вам и не нужно много, чтобы поддерживать целевые темпы роста
на ранней стадии. \newline

[9] Да, я могу представить случаи, когда можно в итоге работать над
чем-то, что в действительности было нужно только одному пользователю.
Но это обычно очевидно даже для неопытных основателей. Поэтому, если
это не очевидно, создавая что-то для рынка из одного пользователя, не
беспокойтесь об этом. \newline

[10] Может существовать даже обратная зависимость между грандиозностью
запуска и успехом. Единственные запуски, которые я помню, это такие
знаменитые провалы, как Segway и Google Wave. Wave – особенно
тревожный пример, так как я считаю, что изначально идея была
великолепной, но частично ее убил слишком раздутый запуск. \newline

[11] Google вырос на фоне Yahoo, но это не было партнерством. Yahoo
был их клиентом. \newline

[12] Это также напомнит основателям, что идея, второй компонент
которой пустой (идея, где вы не можете ничего сделать, чтобы заставить
ее работать, потому что не знаете, как найти пользователей, которых
можно привлечь вручную) вероятно плохая идея, по крайней мере для этих
основателей.

\end{document}
