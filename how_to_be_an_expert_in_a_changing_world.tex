\documentclass[ebook,12pt,oneside,openany]{memoir}
\usepackage[utf8x]{inputenc} \usepackage[russian]{babel}
\usepackage[papersize={90mm,120mm}, margin=2mm]{geometry}
\sloppy
\usepackage{url} \title{Как оставаться экспертом в постоянно
  меняющемся мире} \author{Пол Грэм} \date{}
\begin{document}
\maketitle

Если бы мир вокруг не менялся, наша уверенность в правильности своих
убеждений возрастала бы с монотонным постоянством. Наши взгляды
проходят проверку жизненным опытом, чем этого опыта больше (и чем он
разнообразнее), тем меньше вероятность того, что они ошибочны. Многие
люди полагают, что это действительно так. Такое отношение вполне
оправданно, когда речь идет о взглядах на вещи, практически неизменны
по своей сути, например, такие, как человеческая природа. Однако для
изменчивых явлений, к которым можно отнести практически все остальное,
вы уже не можете быть полностью уверены в своих взглядах.

Эксперты часто ошибаются в своих прогнозах просто потому, что они
являются экспертами по более ранней версий мира.

Возможно ли избежать этого? Можете ли вы защитить себя от устаревших
убеждений? В какой-то степени, да. Я провел почти десять лет,
вкладывая в стартапы, которые находятся на ранних стадиях развития.
Весьма любопытно, но мой опыт показывает, что, умение распознавать
неактуальные идеи и есть тот навык, который может сделать из вас
успешного стартап-инвестора.

Большинство по-настоящему хороших стартапов изначально выглядят
ужасно. Нередко они лишь кажутся нам такими, просто потому что мир
меняется, и изменения, которые в нем происходят превращают подобные
плохие идеи в хорошие. Я провел много времени, изучая способы
распознания идей, а приемы, которые я использовал могут быть применимы
к отбору идей в целом.

Первый шаг – беззаветная вера в постоянство изменений. Люди, которые
позволяют себе стать жертвами той самой монотонно возрастающей
уверенности в своих убеждениях — это люди, которые молчаливо решили
для себя, что мир, в котором мы живем, не меняется. Осознанно
напоминая себе, что это не так, вы сразу же начинаете искать перемены.

Где не следует их искать? К сожалению, помимо в меру полезного
обобщения о почти не меняющейся человеческой природе, сказать можно
только, что перемены сложно предсказать. Да, звучит как одна большая
тавтология, однако никогда не будет лишним напомнить: важные изменения
обычно приходят оттуда, откуда их никто не ожидал.

Поэтому я даже не пытаюсь делать предсказания. Периодически, во время
интервью меня просят сделать прогноз на будущее. В такие моменты я
всегда испытываю трудности, потому что мне приходиться на лету
придумывать какой-нибудь правдоподобный ответ, а чувствую я себя, как
студент, который не подготовился к экзамену.[1] Однако в отличии от
студента подготовиться мне мешает вовсе не лень. Все дело в том, что
предположения о будущем так редко бывают верны, что они, обычно, не
стоят тех дополнительных ограничений, которые они накладывают на образ
мышления их автора. Попытки определить правильное направление следует
прекратить, признав, вместо этого, что вы понятия не имеете, какое
направление является правильным, и став гораздо более восприимчивым к
ветру перемен.

Вообще, следить за развитием событий и пытаться угадать их ход — очень
увлекательное занятие. Рабочие гипотезы ограничивают свободу вашей
мысли, но тем не менее, они же могут стать для вас мотивирующим
фактором, поэтому строить их — вполне приемлемо. Следует, однако, быть
достаточно дисциплинированным и не давать вашим предположениям
превратиться во что-то более серьезное. [2]

Я верю, что такая, пассивная модель работы годиться не только для
оценки новых идей, но и для их создания. Новые идеи появляются не
тогда, когда вы намеренно пытаетесь их придумать, а тогда, когда вы
пытаетесь решить ту или иную проблему, не пренебрегая при этом
необычными вариантами, которые в процессе поиска решения подсказывает
вам интуиция.

Ветер перемен рождается бессознательно в умах специалистов той или
иной области знаний. Если вы состоявшийся эксперт в определенной сфере
деятельности, то для вас любой вопрос, не имеющий очевидного отношения
к делу, любая странная идея, которые приходят вам на ум, достойны
проверки уже в силу самого факта их возникновения.[3] Когда в рамках Y
Combinator мы называем идею безумной — это комплимент, при этом, как
правило, более похвальный чем просто "хорошая идея".

Инвесторы, вкладывающие в стартапы, имеют беспрецедентную мотивацию
для смены изживших себя взглядов на новые. Сумев раньше других понять,
что тот или иной, изначально безнадежный стартап, на самом деле
таковым не является, они способны заработать огромное количество
денег. Как бы то ни было, их мотивы лежат не только в финансовой
плоскости. В процессе выбора, их убеждения проходят прямую проверку:
когда стартапы приходят к инвестору, он должен сказать свое «да» или
«нет», чтобы после этого, довольно быстро узнать, правильный ли выбор
он сделал. Инвесторы, сказавшие «нет» Google (которых было несколько)
будут помнить об этом до конца своих дней.

Вообще, любой человек, которому приходится в том или ином смысле,
«делать ставки на идеи», а не просто комментировать их, имеет ту же
мотивацию, что и инвестор. Возможность почувствовать ее есть и у вас,
как и у любого другого комментатора. Вам лишь надо превратить свои
комментарии в ставки, написав, например, о чем-либо так, чтобы
написанное оставалось на всеобщем обозрений в течении длительного
периода времени. Уже в процессе написания, вы сразу же заметите, что
стали гораздо внимательнее относиться к тому, что вы пишите. В
случайной беседе все было бы гораздо «проще».[4]

Другая хитрость, которая позволяет мне защищать себя от устаревших
идей: я стараюсь с самого начала концентрировать свое внимание на
людях, а не идеях. Предсказать характер будущих открытий сложно, но я
однако, научился довольно хорошо определять то, какими должны быть
люди, которые их совершат. Хорошие новые идеи появляются у людей,
которые искренне в них верят, энергичны и имеют независимый образ
мышления.

Делать ставку на людей, а не на идеи. Этот подход спасал меня
бесчисленное множество раз. Мы, например, считали Airbnb плохой идеей,
однако точно могли сказать, что его основатели были энергичны,
настроены решительно и на все имели отличные от общепринятых взгляды
(даже доходили в этом до крайностей). Поэтому мы решили на какое-то
время отодвинуть свои сомнения в сторону и профинансировали их.

Есть и другая техника, которая, кажется мне вполне пригодной к
применению в общем случае. Окружите себя такими людьми, которые
создают новые идеи. Если вы хотите быстро определять, что ваши взгляды
стали неактуальны, нет лучше способа, чем подружиться с людьми, чьи
открытия сделают их такими.

Быть свободным от плена собственного профессионализма — задача, не
простая сама по себе. С каждым разом она будет становиться все сложнее
и сложнее, поскольку с течением времени изменения будут происходить
все быстрее и быстрее. История эта стара как мир: изменения начали
набирать обороты еще во времена палеолита. Идеи порождают идеи. Не
думаю, что это когда-либо изменится. Хотя, откуда мне знать наверняка?

Примечания

[1] Обычно мне удается выкрутиться следующим образом: я рассказываю о
настоящем, но рассказываю про такие предположения, о которых
большинство людей еще не знает.

[2] Особенно если они становятся настолько известными, что люди
начинают отождествлять их лично с вами. Необходимо крайне скептически
относиться к убеждениям, в которые вы склонны верить и как только та
или иная гипотеза становится связана с вашим именем, она почти
наверняка попадает в эту категорию

[3] На практике, понятие "состоявшийся эксперт" не требует от человека
получать признание в качестве эксперта со стороны других людей,
поскольку оно в любом случае является запоздалым индикатором вашей
способности оценивать идеи. Одного года упорной работы плюс проявление
заинтересованности будет достаточно для большинства сфер человеческой
деятельности.

[4] По моим эмпирическим наблюдениям, комментарии в таких местах, как,
например, форумы и Twitter на практике работают так же, как и
случайная беседа, несмотря на публичность и бессрочность их
существования. Возможно, роль разделительной черты между серьезным
обсуждением и случайной беседой играет наличие у вашего текста хорошо
различаемого заголовка.

\end{document}
