\documentclass[ebook,12pt,oneside,openany]{memoir}
\usepackage[utf8x]{inputenc} \usepackage[russian]{babel}
\usepackage[papersize={90mm,120mm}, margin=2mm]{geometry}
\sloppy
\usepackage{url} \title{Дети и стартапы} \author{Пол Грэм} \date{}
\begin{document}
\maketitle

Я боялся заводить детей до тех пор, пока я их не завел. До этого
момента, я относился к детям, как юный Августин к добродетельной
жизни. Мне было бы грустно думать, что у меня никогда не будет детей.
Но хотел ли я их в тот момент? Нет.

Если бы у меня были дети, я стал бы родителем, а родители, как я знал
с детства, были неклёвые. Они были скучными и ответственными, и им
было не до веселья. И хотя нет ничего удивительного в том, что дети
верят в это, честно говоря, я не так уж много повидал, чтобы изменить
свое мнение. Всякий раз, когда я замечал родителей с детьми, дети
казались ужасными, а родители — жалкими измученными существами, даже
когда они доминировали.

Когда знакомые заводили детей, я с энтузиазмом поздравлял их, потому
что, похоже, именно так все и поступали. Но я этого совсем не
чувствовал. «Лучше у вас, чем у меня», — думал я.

Теперь, когда у людей рождаются дети, я поздравляю их с энтузиазмом, и
я искренен. Особенно с первенцем. Я чувствую, что они только что
получили лучший подарок в мире.

Что изменилось, так это то, что у меня появились дети. То, чего я
боялся, оказалось чудесным.

Отчасти, и я не буду отрицать того, что это связано с серьезными
химическими изменениями, которые произошли почти мгновенно, когда
родился наш первый ребенок. Как будто кто-то щелкнул выключателем. Я
вдруг почувствовал себя защитником не только по отношению к нашему
ребенку, но и по отношению ко всем детям. Когда я вез жену и
новорожденного сына домой из больницы, мы подъехали к пешеходному
переходу, полному пешеходов, то поймал себя на мысли: «я должен быть
очень осторожен со всеми этими людьми. Каждый из них чей-то ребенок!»

Так что, в какой-то степени, вы не можете доверять мне, когда я
говорю, что иметь детей-это здорово. В какой-то степени я похож на
религиозного сектанта, который говорит вам, что вы будете счастливы,
если тоже примкнете к этому культу (но только потому, что
присоединение к этому культу изменит ваше сознание таким образом, что
вы будете счастливы стать членом этого культа). Но не совсем. Но были
некоторые вещи, которые я воспринимал неверно, прежде чем у меня
появились дети.

Мои наблюдения за родителями и детьми были очень сильно подвержены
когнитивному искажению "Систематической ошибке отбора" (Selection
bias). Некоторые родители, наверно, замечали что я пишу “Всякий раз,
когда замечал родителя с ребенком.” Конечно, когда я замечал, что дети
делают что-то плохое. Я замечал их только тогда, когда они шумели. И
где я был. Когда замечал их? Как правило, я не хожу в места с детьми,
поэтому единственные случаи, когда я сталкиваюсь с ними, это
общественные места, такие как самолет. Что не совсем типичный случай.
Полеты с ребенком — это то, что мало нравится родителям.

Что я не заметил, потому что те моменты, как правило, были намного
тише всех тех замечательных моментов, когда родители были с детьми.
Люди не говорят об этом очень много — магию сложно выразить словами, и
все остальные родители в любом случаем знают об этом, но одна из
замечательных особенностей рождения детей, это то, что на протяжении
длительного времени ты чувствуешь, что не хочешь быть нигде, кроме как
с ними, и нет ничего другого чем ты предпочитаешь заниматься. Ты не
должен делать что-то особенное. Вы можете просто делать что-то вместе,
или укладывать их спать, или раскачивать их на качелях в парке. Но ты
не променяешь эти моменты ни на что. Никто не ассоциирует детей с
миром, но это то, что ты чувствуешь. Тебе не нужно больше смотреть
дальше того момента, в котором ты находишься сейчас.

До того, как у меня появились дети, у меня были моменты такого покоя,
но они были более редкими. С детьми это может происходить несколько
раз в день.

Другим источником информации о детях было мое собственное детство, и
это тоже вводило в заблуждение. Я довольно плохо себя вел, и постоянно
попадал в неприятности из-за чего-то. Поэтому мне казалось, что
родительство — это, по сути, правоохранительная деятельность. Я и не
подозревал, что бывают и хорошие времена.

Я помню, как однажды, когда мне было около 30 лет, мама сказала, что
была счастлива растить меня и мою сестру. Боже мой, подумал я, эта
женщина — святая. Она не только терпела всю ту боль, которой мы ее
подвергали, но и наслаждалась ею? Теперь я понимаю, что она просто
говорила правду.

Она сказала, что одна из причин, по которой ей нравилось с нами
общаться, заключалась в том, что с нами было интересно разговаривать.
Это стало для меня сюрпризом, когда у меня появились дети. Вы не
просто любите их. Они тоже становятся твоими друзьями. Они
действительно интересны. И хотя я признаю, что маленькие дети
катастрофически любят повторения (все, что стоит сделать один раз,
приходится сделать пятьдесят раз), часто с ними действительно весело
играть. Это меня тоже удивило. Играть с двухлетним ребенком было
весело, когда мне самому было два года, и определенно не весело, когда
мне было шесть. Почему это снова стало весело позже? Но это так.

Конечно, бывают времена, когда это чистая рутина. Или, что еще хуже,
ужас. Иметь детей — это один из тех интенсивных типов переживаний,
которые трудно себе представить, если у вас их не было. Но дети — это
не просто ваша ДНК, направляющаяся к спасательным шлюпкам, как я
представлял до рождения детей.

Хотя некоторые мои опасения насчет детей были правильными. Они
определенно делают вас менее продуктивным. Я знаю, что наличие детей
заставляет некоторых людей действовать вместе, но если вы уже
действовали вместе, у вас будет меньше времени, чтобы сделать это. В
частности, вам придется работать по расписанию. У детей есть
расписание. Я не уверен, потому ли это, что таковы дети, или потому,
что это единственный способ интегрировать свою жизнь со взрослой, но
когда у вас есть дети, вам, как правило, приходится работать по их
расписанию.

У вас будет много отрезков времени чтобы работать. Но вы не сможете
позволить работе заполнить всю вашу жизнь, как я привык это делать до
появления детей. Вам придется работать в одно и то же время каждый
день, независимо от вдохновения или “потока”. И будут моменты, когда
придется остановиться, даже если вы уже в “потоке”.

Мне удалось приспособиться работать в новом образе жизни. Работа, как
и любовь, находит свой путь. Если даже крохам времени удастся найтись,
я использую их все для плодотворной работы. И хотя я не делаю так
много работы, как до того как у меня появились дети, я делаю
достаточно.

Не приятно такое говорить, ведь мне всегда были присущи амбиции, но
появление детей поубавит у вас амбиций. Больно видеть это написанным.
Я мучительно избегаю этого. Но если бы здесь не было чего-то
серьезного, с чего бы мне избегать? Дело в том, что когда у тебя есть
дети, ты заботишься о них больше, чем о себе. А внимание — это игра с
нулевой суммой. Только одна мысль в единицу времени может быть «в
топе». И если уж у тебя дети, скорее всего это будет мысль о них, и уж
потом только проект, над которым ты работаешь.

Есть у меня кое-какие хаки, чтобы балансировать на этом пути.
Например, когда я пишу эссе, то думаю о том, что бы хотел тем самым
поведать своим детям. И это заставляет правильнее смотреть на вещи.
Когда я писал Bel, то сказал детям, что возьму их в Африку, как только
закончу эту работу. Когда ты такое говоришь маленькому ребёнку, он
воспринимает это как обещание. То есть, я должен был или закончить,
или оставить их без поездки. И будь я везунчиком, такие трюки
продвинут меня сильно вперед. Но проблема все ещё существует, без
сомнения.

С другой стороны, что ж это за амбиции, если они не выживут, после
появления детей? У вас настолько мало вариантов?

И хотя появление детей может искажать мое нынешнее восприятие, это не
затерло мою память. Я прекрасно помню, какой была жизнь раньше.
Достаточно хорошо, чтобы скучать по многим вещам, по таким, как
возможность улететь в другую страну в любой момент. Это было так
здорово. Почему я никогда так не делал?

Видите, что я там сделал? Дело в том, что большую часть свободы,
которую я имел до детей, я никогда не использовал. Я заплатил за это
одиночеством, но я никогда не использовал это.

У меня было много счастливых моментов, прежде чем у меня появились
дети. Но если я посчитаю счастливые моменты, не просто потенциально
возможное счастье, а настоящие счастливые моменты, после появления
детей их будет больше, чем раньше. Сейчас я часто засыпаю с этой
мыслью.

Опыт родительства у каждого, и я знаю, что мне повезло. Мне кажется,
что то беспокойство, которое я ощущал еще до появления детей, должно
быть обычным явлением. И если судить по лицам других родителей, когда
они смотрят на своих детей, так и должно выглядеть счастье, которое
приносят дети.

Примечание

[1] Взрослые уже слишком изощрены видеть в двух-летних детях сложный и
очаровательный характер, в то время, как для большинства шестилетних
детей они просто “неполноценные” шестилетки.

\end{document}
