\documentclass[ebook,12pt,oneside,openany]{memoir}
\usepackage[utf8x]{inputenc} \usepackage[russian]{babel}
\usepackage[papersize={90mm,120mm}, margin=2mm]{geometry}
\sloppy
\usepackage{url} \title{Как потерять время и деньги} \author{Пол Грэм}
\date{}
\begin{document}
\maketitle

Когда мы продали наш стартап в 1998 году, я неожиданно получил большую
кучу денег. После этого мне пришлось думать о том, о чем я раньше
никогда не задумывался: как не потерять их. Мне было известно, что
богатые люди тоже становятся бедными, также как и бедные люди
становятся богатыми. Но так как последние несколько лет я провел за
изучением того, как бедные становятся богатыми, я совершенно не знал
как люди становятся бедными. И для того, чтобы избежать этого, мне
необходимо было понять, как это просходит.

Я стал уделять больше внимания тому, как люди теряют состояния. Если
бы в детстве меня спросили, как богатые люди становятся бедными, я бы
ответил, что это происходит потому, что они тратят все свои деньги.
Так это происходит в фильмах и в книгах, потому что это красочный
способ. Но на самом деле, большинство людей теряют состояния не из-за
неуемных трат, а из-за неправильных инвестиций.

Довольно сложно потратить все деньги и не заметить этого. Любому
нормальному человеку будет сложно профукать больше чем несколько
десятков тысяч долларов и не подумать при этом: “Кажется, я трачу
слишком много денег”. С другой стороны, если начать торговать
деривативами на бирже, можно потерять несколько миллионов долларов в
мгновение ока.

Большие траты денег на всякую роскошь и удовольствия включают
сигнализацию в голове большинства людей. В то время как инвестирование
денег этого не делает. Удовольствия и роскошь кажутся развращающими. И
если вы не унаследовали ваши деньги и не выиграли их в лотерею, то вы
скорее всего знаете, что подобные вещи ведут к проблемам.
Инвестирование обходит эту сигнализацию. Вы же не тратите деньги
впустую, вы просто переводите их из одного состояния в другое. Вот
почему люди, которые пытаются продать вам дорогую вещь всегда говорят:
“думайте об этом, как об инвестиции”.

Чтобы этого избежать, необходимо выработать в себе новый вид
сигнализации. Это может оказаться довольно непростой задачей. В то
время как защита, не позволяющая тратить слишком много денег, является
настолько естественной, что она возможно даже встроена в нашу ДНК,
защита, позволяющая избежать неправильных инвестиций, должна быть
выработана самостоятельно, а кроме того, она часто противоречит
интуиции.

Несколько дней назад я осознал нечто неожиданное: ситуация со временем
очень похожа на ситуацию с деньгами. Самый опасный способ потерять
время заключается не в том, чтобы провести его без удовольствия, а в
том, чтобы провести его, занимаясь ненастоящей работой. Когда вы
проводите время в развлечениях, вы понимаете, что потакаете своим
капризам. Сигнализация в таких случаях срабатывает довольно быстро.
Если бы я проснулся как-нибудь и провел весь день сидя на диване перед
телевизором, я бы чувствовал себя так, как будто произошло нечто очень
неправильное. Даже когда я просто думаю об этом, у меня возникает
неприятное ощущение. Я бы почуствовал себя некомфортно уже после двух
часов просмотра телевизора, не говоря о целом дне.

И несмотря на все это, у меня определенно бывали дни, которые я мог бы
равнозначно просидеть перед телевизром. Дни, в конце которых я
спрашивал себя – что я сделал сегодня, и ответом было: в основном,
ничего. В конце подобных дней я тоже чувствую себя плохо, но далеко не
так плохо, как если бы я просидел их перед телевизором. Если бы я весь
день смотрел телевизор, я бы чувствовал себя так, словно я падаю в
бездонную пропасть. Но такая сигнализация не работает в те дни, в
итоге которых у меня ничего не сделано потому, что я занимался вещами
очень похожими на настоящую работу. Например, разбирал почту. Ведь это
делается, сидя за столом. И не приносит удовольствия. Значит это,
конечно же, работа.

Со временем, также как и с деньгами, избегания развлечений больше
недостаточно для того, чтобы защитить вас. Вероятно, этого было
достаточно для древнего, и может быть для доиндустриального обществ.
Но с тех пор мир стал сложнее: наиболее опасными ловушками являются те
виды деятельности, которые обходят ествественную сигнализацию и
маскируются под полезную. И самое плохое в этом то, что они не
приносят даже удовольствия.

\end{document}
