\documentclass[ebook,12pt,oneside,openany]{memoir}
\usepackage[utf8x]{inputenc} \usepackage[russian]{babel}
\usepackage[papersize={90mm,120mm}, margin=2mm]{geometry}
\sloppy
\usepackage{url} \title{Пиши как говоришь} \author{Пол Грэм} \date{}
\begin{document}
\maketitle

Самым простым способом увеличения количества читателей ваших текстов
является употребление простого разговорного языка.

Большинство людей, начинающих писать, пишут на совершенно другом
языке, отличающемся от языка, который используется ими для общения,
причем отличается как структура употребляемых предложений, так и набор
слов. Например, в разговорном английском языке никто не использует
глагол «pen» (писать пером, используется в качестве синонима глагола
«write»). Кто-то мог бы почувствовать себя идиотом, услышав в
разговоре подобные слова.

Последней каплей для меня было высказывание, которое я прочитал пару
дней назад: The mercurial Spaniard himself declared: «After Altamira,
all is decadence!» (Изменчивый испанец провозгласил: «После Альтамиры
– всё упадок!» — прим. англоязычные господа любят использовать слово
«mercurial» в качестве метафоры для определения чего-нибудь
непостоянного, неустойчивого, переменчивого, так как одним из
переводов этого слова является слово «ртутный»).

Эта фраза из «Истории Древней Великобритании» Нила Оливера. Мне
немного не по себе, из-за того, что я использовал пример именно из
этой книги, потому что она написана не хуже множества других. Но
только представьте себе, что называете Пикассо «ртутным испанцем» при
разговоре с другом. Даже одно упоминание такого оборота речи заставит
кого-угодно удивленно вскинуть брови. И все же люди так пишут целые
книги.

Итак, письменный и устный язык различны. Действительно ли письменный
язык хуже?

Если вы хотите, чтобы люди читали и понимали о чем вы пишете, ответом
на вопрос выше будет — «да». Письменный язык сложней, что делает его
более трудноусваиваемым и требующим некоторых усилий для чтения. Он
также более формален и сух, что иногда приводит к дрейфу внимания
читателя. Но, пожалуй, хуже всего то, что сложные предложения и
замысловатые слова внушают пишущим ложное впечатление, что они говорят
больше, чем есть на самом деле.

На самом деле для выражения сложных идей не нужно сложных предложений.
Когда специалисты в какой-то трудной для понимания темы обсуждают друг
с другом некоторые отдельные идеи из этой области, они используют
речевые обороты не сложней, чем при обсуждении обеденного меню.
Конечно они используют и специфические слова, но даже те, не более,
чем необходимо. И по моему опыту, чем труднее предмет, тем менее
формально его обсуждают эксперты. Отчасти, я думаю, это связано с тем,
что они не стремятся что-то доказать, и отчасти потому, что сложные
идеи предполагают меньшую лингвистическую гибкость.

Неофициальный язык является своего рода «спортивной одеждой» для идей.

Я не утверждаю, что разговорный язык всегда работает лучше. Поэзией,
как и большинством музыкальных текстов, можно выразить такие вещи,
которые не поддаются разговорному стилю. Некоторые писатели могут
успешно использовать замысловатые слова в прозе. Есть случаи, когда
писатели не хотят упрощать понятие своих текстов, но почти всегда
разговорный язык лучше.

Может показаться, что большинству трудно писать разговорным языком. В
таком случае возможно, лучшим решением был бы набросок первого
черновика в обычном режиме и стиле. После чего нужно взглянуть на
каждое предложение и задать себе вопрос: «Смог бы я так выразиться,
если бы разговаривал с другом?» Если нет, то представьте, что бы вы
сказали и используйте это выражение вместо первоначального. Через
некоторое время этот фильтр начнет действовать автоматически. Когда вы
будете писать что-то непроизносимое, вы будете слышать воображаемый
скрежет, ложащихся на страницу, неудобоваримых слов.

Прежде, чем опубликовать новое эссе, я читаю его вслух и исправляю
все, что не похоже на речь. Также исправляются незначительные
фонетические шероховатости. Не знаю нужно ли это, но такие правки не
отнимают много сил и времени.

Конечно, подобный трюк не всегда помогает. Бывает настолько далекие от
разговорного языка тексты, что одни высказывания не могут быть
заменены на другие. Для таких случаев есть более радикальные решения.
После написания черновика, попробуйте объяснить другу написанное.
Затем замените черновик только что сказанным.

Люди часто говорят мне, что мои эссе звучат как обычная речь, и то,
что такой факт удостаивается отдельного комментария, показывает
насколько редко кому-нибудь удается писать в разговорном стиле.

Если вы просто постараетесь писать в разговорном стиле, уже будете на
голову выше 95\% пишущих. А это так легко сделать — просто не
позволяйте высказываниям быть более громоздкими, чем при общении с
друзьями.

\end{document}
