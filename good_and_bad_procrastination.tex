\documentclass[ebook,12pt,oneside,openany]{memoir}
\usepackage[utf8x]{inputenc} \usepackage[russian]{babel}
\usepackage[papersize={90mm,120mm}, margin=2mm]{geometry}
\sloppy
\usepackage{url} \title{Хорошая и плохая прокрастинация} \author{Пол
  Грэм} \date{}

\begin{document}
\maketitle

Все самые впечатляющие люди, с которыми я знаком — ужасные
прокрастинаторы. Так может, прокрастинация не всегда плоха? \newline

Обычно пишущие о прокрастинации пишут о том, как от неё избавиться —
что, строго говоря, невозможно. Существует бесконечное количество
вещей, которые нужно сделать, в то время как над чем бы вы ни
работали, вы не работаете над всем остальным. Так что вопрос
заключается не в том, как устранить прокрастинацию, а в том, как
прокрастинировать правильно. \newline

Существует три типа прокрастинации, в зависимости от того, что вы
делаете вместо работы: вы можете (А) ничего не делать, (Б) заниматься
чем-то менее важным и (В) заниматься чем-то более важным. Последний
тип, я убеждён, и есть хорошая, правильная прокрастинация. \newline

Это как «рассеянный профессор», который забывает или побриться, или
поесть, или посмотреть под ноги, когда размышляет над чем-нибудь
интересным. Его мозг отстранён от повседневного бытия, потому что
занят другими вещами. \newline

Именно в этом смысле я назвал прокрастинаторами всех впечатляющих
людей, которых знаю. Они прокрастинаторы типа В: избегают работы над
мелочами ради работы над чем-то большим. \newline

Что такое «мелочи»? Грубо говоря, работа, у которой нулевые шансы быть
упомянутой в некрологе. Конечно, сейчас трудно сказать, чему именно
повезёт оказаться вашим лучшим трудом (будет ли это магнум опус об
архитектуре шумерских храмов или детективный триллер, опубликованный
под псевдонимом), но существует целый класс задач, который мы можем
спокойно вычеркнуть из этого списка: бритьё, стирка, уборка, написание
благодарственных писем — всё, что может быть названо обязанностью. \newline

Хорошая прокрастинация — это уклонение от обязанностей ради выполнения
реальной работы. \newline

Хорошая по смыслу, по крайней мере. Те, кто хочет, чтобы вы выполняли
свои обязанности, вряд ли сочтут её хорошей. Но вам, видимо, придётся
расстроить их, чтобы действительно что-нибудь сделать. Даже самые
мягкие с виду люди становятся удивительно безжалостными ко всему, что
касается обязанностей, когда хотят совершить что-то великое. \newline

Некоторые обязанности, такие как отвечать на письма, устраняются сами
собой, если не обращать на них внимания (правда, иногда вместе с
друзьями). Другие, например, стрижка газона или оплата счетов,
становятся только хуже, если их запустить. Кажется, что не следует
откладывать в дальний ящик обязанности второго типа. Всё равно рано
или поздно ими придётся заняться. Почему бы не сделать этого сейчас? \newline

Причина, по которой даже эти обязанности всё-таки стоит отложить,
заключается в том, что действительно важные задачи требуют двух вещей,
в которых обязанности не нуждаются: больших непрерывных временных
отрезков и правильного настроения. Если вас вдохновляет какой-то
проект, лучшим решением может оказаться задвинуть все остальные дела
на несколько дней, чтобы как следует над ним поработать. Да, возможно,
обязанности займут больше времени, когда до них наконец-то дойдут
руки. Но сделав много всего за эти дни, в конечном итоге вы окажетесь
значительно более продуктивным. \newline

На самом деле, вполне вероятно, что разница заключается не в объёме
задачи, а в её типе. Возможно, некоторые виды работ могут быть
выполнены исключительно в порыве вдохновения и во время долгих,
непрерывных временных отрезков, а не в послушно распланированные
маленькие подходы. Эмпирически кажется, что так оно и есть. Когда я
думаю о людях, сделавших что-то великое, они не кажутся мне теми, кто
покорно вычёркивает из списка дел один пункт за другим. Они кажутся
мне теми, кто увиливает от обязанностей, чтобы поработать над какой-то
новой идеей. \newline

Обратное тоже верно — чем больше кого-то принуждают заниматься
обязанностями, тем больше это снижает его продуктивность. Это очень
дорогой подход не только из-за времени, которое обязанности отнимают
сами по себе, но и из-за того, что они разрушают работу над настоящей
проблемой. Вам достаточно отвлекать кого-нибудь пару-тройку раз за
день, чтобы этот человек в принципе не мог работать над большими
задачами. \newline

Я долго пытался понять, почему стартапы наиболее продуктивны в самом
начале, когда это просто несколько ребят, собирающихся дома у одного
из них. Думаю, основная причина заключается в том, что на данном этапе
их никто не отвлекает. В теории, когда у основателей наконец
появляется достаточно денег, чтобы кого-нибудь нанять — это хорошо. Но
возможно, что на самом деле переработать лучше, чем отвлечься. Как
только ты разбавляешь стартаперов типичными офисными работниками —
прокрастинаторами типа Б — вся компания начинает работать на их
частоте. Они мастера отвлекаться на обязанности, и вскоре ты
становишься таким же. \newline

Обязанности столь эффективно справляются с убийством больших проектов,
что многие используют их именно с этой целью. Например, человек,
решивший написать роман, внезапно обнаруживает, что нужно устроить
генеральную уборку. Люди, которым не удается написать роман, не терпят
крах со своей затеей, сидя перед чистым листом бумаги несколько дней
подряд. Они терпят крах, кормя кота, отправляясь купить что-нибудь для
дома, встречаясь с друзьями за чашечкой кофе и проверяя почту. «У меня
нет времени на работу», — говорят они. И времени действительно нет;
они об этом позаботились. \newline

(Еще одна распространенная вариация — «мне негде работать». Попробуйте
посетить места, в которых работали великие люди, и своими глазами
убедиться, сколь они неподходящи.) \newline

Я сам использовал обе эти уловки раз за разом. За последние 20 лет я
выучил кучу трюков, чтобы заставить себя работать, но даже сейчас не
всегда выигрываю. В одни дни я действительно выполняю много настоящей
работы, другие же съедаются обязанностями. И я знаю, что обычно это
моя вина: я позволяю обязанностям сожрать мой день, чтобы избежать
столкновения со сложными задачами. \newline

Это самая опасная форма прокрастинации, — неосознанная прокрастинация
типа Б, — потому что она не выглядит прокрастинацией. Ты «сделал много
дел». Просто не тех. \newline

И любой совет по борьбе с прокрастинацией, концентрирующийся на
вычеркивании задач из списка, является не просто неполным, а неверным
в корне, если даже не рассматривает возможности, что сам список дел
является прокрастинацией типа Б. Хотя, пожалуй, «возможность» —
слишком мягкое слово в данном контексте. Практически всегда он таковым
и является. Если ты не работаешь над самыми большими задачами, над
которыми только можешь, ты — прокрастинатор типа Б, и не имеет
значения, сколько всего ты успеваешь сделать. \newline

В своем знаменитом эссе «Вы и ваше исследование [оригинал,
хабраперевод]» (которое я рекомендую каждому амбициозному человеку вне
зависимости от того, над чем он работает), Ричард Хэмминг предлагает
задать себе три вопроса: \newline

\begin{itemize}
\item{Какие проблемы являются наиболее значимыми в твоей области
деятельности?}

\item{Ты работаешь над одной из них?}

\item{Почему нет?}
\end{itemize}

Хэмминг работал в Bell Labs, когда начал задавать эти вопросы
коллегам. По большому счету, каждому сотруднику Bell Labs следовало бы
работать над важнейшими проблемами в своей области. Считается, что не
все могут внести одинаково драматичный вклад в развитие мира; не знаю;
но какими бы ни были ваши возможности, определённо существуют проекты,
которые вам по зубам. Так что литанию Хэмминга можно сформулировать
более общим образом: \newline

\begin{itemize}
\item{Какая вещь является самой важной из всего, над чем ты мог бы работать,
и почему ты не работаешь над ней?}
\end{itemize}

Большинство людей уклоняется от ответа. Я сам уклоняюсь; я вижу вопрос
на странице и стараюсь как можно скорее перейти к следующему
предложению. Хэмминга, в своё время действительно задававшего его
коллегам, в конце концов стали избегать. Однако взглянуть в глаза
этому вопросу обязан каждый амбициозный человек. \newline

Беда в том, что на эту наживку можно поймать слишком большую рыбу.
Чтобы сделать что-то грандиозное, недостаточно просто найти хороший
проект. После того, как проект найден, нужно еще и заставить себя
работать над ним, а это может быть непросто. И чем больше проблема,
тем сложнее заставить себя над ней работать. \newline

Конечно, главной причиной, по которой люди считают сложной работу над
определённой задачей, является тот факт, что они не получают от этого
удовольствия. Мы часто обнаруживаем, особенно когда молоды, что
работаем над вещами, которые не слишком нам нравятся — просто потому,
что это выглядит впечатляюще, например, или потому, что кто-то другой
поручил нам заняться этим. Большинство аспирантов тянет лямку, работая
над большими поректами, которые им на самом деле не интересны, что
превращает магистратуру в синоним прокрастинации. \newline

Однако даже если вам нравится то, чем вы занимаетесь, намного проще
заставить себя работать над маленькими задачами, а не над большими.
Почему же? Почему так трудно работать над большими проектами? Первая
причина — вы не сможете извлечь из этого никакой выгоды в обозримом
будущем. Когда работаешь над чем-то, что можно завершить за день или
два, ожидаешь награды в виде приятного ощущения завершённости в
ближайшем времени. Когда награда находится неопределимо далеко, её
получение кажется менее реалистичным. \newline

Забавно, но другая причина, по которой люди предпочитают не работать
над большими проектами, это боязнь потратить время впустую. Что, если
ты ничего не добьёшься? Тогда всё затраченное время будет потеряно.
(На самом деле, это крайне маловероятно, потому что работа над
большими проектами почти всегда куда-нибудь, да приводит.) \newline

Тем не менее, проблема с большими задачами не может заключаться только
в том, что ты не получишь моментального вознаграждения, или в том, что
можно потерять много времени. Будь это полный список, работа над ними
стала бы не хуже поездки к тёще. Нет уж, там точно есть кое-что ещё.
Большие проекты ужасают. Столкновение с ними лицом к лицу причиняет
почти физическую боль. Они похожи на пылесос, подключённый к твоему
воображению — изначальные идеи моментально высасываются, новых нет, а
он все сосёт и сосёт. \newline

Ты не можешь взглянуть большой проблеме прямо в глаза. Ты вынужден
делать это наискосок. Но что можно — так это постепенно спрямлять
угол: нужно посмотреть на задачу достаточно прямо, чтобы ухватить
исходящее от неё вдохновение, но недостаточно, чтобы тебя парализовало
масштабом. И дальше ты сможешь с каждым разом смотреть всё более и
более смело, по аналогии с тем, как корабль переставляет паруса всё
ближе и ближе к ветру по ходу движения. \newline

Для работы над большими проектами, похоже, нужен ряд трюков,
позволяющих обмануть самого себя. Приходится работать над маленькими
вещами, которые вырастают в большие, или работать над постепенно
увеличивающимися задачами, или разделять моральную нагрузку с
коллегами. Идти на подобного рода ухищрения — не признак слабости.
Величайшие из свершений проложили эти пути. \newline

Когда я разговариваю с людьми, заставившими себя работать над чем-то
большим, я замечаю, что все они отодвинули обязанности в сторону и все
чувствуют себя виноватыми из-за этого. Не думаю, что им стоит винить
себя. Никто не способен переделать все дела в мире, так что каждому,
кто занимается чем-то действительно важным, неизбежно придётся
оставить много маленьких дел невыполненными. Мне кажется неправильным
испытывать из-за этого угрызения совести. \newline

На мой взгляд, путь к «решению» проблемы прокрастинации заключается в
том, чтобы позволить удовольствию увлечь тебя за собой, а не заставить
список дел толкать вперёд. Работай над амбициозными проектами, которые
тебе по-настоящему нравятся, подстраивай паруса по ветру — и
несделанным останется именно то, что должно.

\end{document}
