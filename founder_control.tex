\documentclass[ebook,12pt,oneside,openany]{memoir}
\usepackage[utf8x]{inputenc} \usepackage[russian]{babel}
\usepackage[papersize={90mm,120mm}, margin=2mm]{geometry}
\sloppy
\usepackage{url} \title{Нужно ли основателю сохранять контроль над
  компанией?} \author{Пол Грэм} \date{}
\begin{document}
\maketitle

Один предприниматель, которого мы в свое время профинансировали,
сейчас ведет переговоры с венчурными инвесторами. Он спросил меня –
часто ли бывает так, чтобы основатель сохранил контроль над компанией
после привлечения венчурного капитала? Инвесторы сказали ему, что это
почти нереальная ситуация.

Десять лет назад так всё и было. Традиционно после первого этапа
инвестиций совет директоров компании состоял из двух основателей, двух
инвесторов и одного независимого члена. В последнее время в состав
совета чаще входит один основатель, один инвестор и один независимый
член. В любом случае, основатели теряют большинство.

Но не всегда. Марк Цукерберг сохранил контроль над советом директоров
Facebook после первого этапа инвестиций и сохраняет по сей день. Марк
Пинкус также сохранил контроль над Zynga. Можно ли считать их
исключениями из правила? Как часто основатели продолжают в полной мере
управлять компанией после привлечения венчурных инвестиций? Я слышал о
нескольких таких случаях среди компаний, которые мы финансировали, но
не был уверен, о скольких именно. Чтобы выяснить это, я написал на
список рассылки ycfounders.

Ответы удивили меня. В десяти или двенадцати компаниях (из числа тех,
которые мы финансировали) после привлечения венчурных инвестиций
основатели по-прежнему имели большинство в советах директоров.

Я чувствую, что настал переломный момент. Многие инвесторы по-прежнему
действуют так, как будто никогда не слышали об основателях,
сохраняющих контроль над компанией после привлечения инвестиций. Даже
если вы просто спросите их об этом, многие из них попытаются унизить
вас – заставить почувствовать себя новичком или маньяком, который
хочет лично контролировать всё вокруг. Но основатели, про которых я
слышал, не новички и не маньяки контроля. А даже если они новички и
маньяки контроля, вроде Марка Цукерберга, именно в таких маньяков
инвесторам нужно пытаться вкладывать побольше.

Основатели, сохраняющие контроль над компанией после привлечения
инвестиций, уже давно не новость. И если не случится никакой
финансовой катастрофы, думаю, в 2011 году это станет нормой.

Контроль над компанией – это не просто большинство голосов в совете
директоров. Инвесторы обычно могут налагать вето на некоторые важные
решения, такие как продажа компании, вне зависимости от того, сколько
мест в совете они занимают. А в ходе голосования в советах директоров
редко бывает борьба. Решения принимаются не в ходе голосования, а во
время предшествующей ему дискуссии. Если в ходе дискуссии мнения
разделяются, сторона, которая знает, что проиграет голосование, вряд
ли будет настаивать на своем. Вот что контроль над советом означает в
действительности. Вы не получаете автоматически право делать всё, что
вам хочется; совет по-прежнему должен действовать в интересах
акционеров; но если у вас большинство в совете, тогда ваше мнение о
том, что именно входит в интересы акционеров, будет преобладать.

Итак, хотя контроль над советом директоров не означает тотального
контроля, он позволяет многое. Это неизбежно влияет на атмосферу в
компании. И если сохранение контроля основателя над компанией после
привлечения инвестиций станет нормой, это изменит атмосферу во всем
мире стартапов. Переход к этой новой норме может произойти удивительно
быстро, поскольку стартапы, контролируемые основателями, зачастую
оказываются успешнее других. Именно они определяют тенденции, как для
других стартапов, так и для венчурных инвесторов.

Инвесторы так жестко ведут переговоры со стартапщиками во многом из-за
того, что не хотят показаться слабыми перед своими партнерами. Когда
они подписывают соглашение, они хотят иметь возможность похвастаться
выгодными условиями, о которых им удалось договориться. Большинство из
них на самом деле не придает большого значения тому, сохранит ли
основатель контроль над компанией. Они просто не хотят показать, что
им пришлось пойти на уступки. Это значит, что если инвесторы
перестанут считать сохранение контроля основателя уступкой, эта
практика быстро станет общепринятой.

Это изменение не должно стать такой большой проблемой, какой его могут
видеть инвесторы (так же как и многочисленные другие изменения,
которые им пришлось принять). Инвесторы по-прежнему смогут убеждать,
они лишь потеряют возможность принуждать. А те стартапы, в которых
инвесторы прибегают к принуждению, слишком мало значат. Инвесторы
зарабатывают большую часть денег на нескольких хитовых проектах, и эти
стартапы не входят в их число.

Знание того, что основатели сохранят контроль над компанией, может
помочь инвесторам сделать лучший выбор. Если они будут знать, что не
смогут уволить основателей, они будут искать стартапщиков, которым
смогут доверять. А это как раз те люди, которые им нужны.

\end{document}
