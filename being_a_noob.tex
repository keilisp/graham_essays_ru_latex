\documentclass[ebook,12pt,oneside,openany]{memoir}
\usepackage[utf8x]{inputenc} \usepackage[russian]{babel}
\usepackage[papersize={90mm,120mm}, margin=2mm]{geometry}
\sloppy
\usepackage{url} \title{Чувствовать себя нубом — это хорошо}
\author{Пол Грэм} \date{}
\begin{document}
\maketitle

Когда я был молод, я думал, что взрослые всё знают. Теперь, когда я
стал взрослым, я понял, что это не так.

Я постоянно чувствую себя нубом (новичком). Это выглядит, как-будто я
начинаю стартап в новой области, про которую я ничего не знаю, или
читаю книгу по теме, в которой плохо разбираюсь, или посещаю страну, в
которой я не знаю как себя вести.

Быть нубом - не очень приятное занятие. Само слово "нуб" - это не
комплимент. Но даже сегодня я выясняю что-то интересное про то как
быть "нубом": чем больше ты нуб в мелочах (тактически), тем меньше ты
нуб в общем (стратегически).

Например, если ты остался в своей стране, ты меньше ощущаешь себя
нубом, чем если бы ты отправился Фарававию (Farawavia), где все
устроено по-другому. Но всё же вы будете знать больше, если вы
переедете. Чувствовать себя "нубом" обратно коррелирует с невежеством.

Но если быть нубом так полезно, почему же мы тогда так этого не любим?
Какая эволюционная ценность у такого отвращения?

Полагаю, что ответ в том, что есть два разных типа "чувствовать себя
нубом": быть тупым и делать что-то неизведанное. Наше отвращение к
чувству "быть нубом" - это то как наш мозг дает сигнал: "камон, а ну
быстро выясни как это устроено". И это было правильно так думать на
протяжении большей части человеческой истории. Жизнь
охотников-собирателей была сложносоставной, но их жизнь не менялась
так быстро как сейчас. Им не нужно было внезапно и срочно выяснять как
устроены криптовалюты. Так что имело смысл прокачивать свою
компетенцию в существующих областях и задачах, чем искать новые.
Чувствовать себя нубом было так же отвратно, как и чувствовать себя
голодным в мире, где еда была дефицитом.

Сейчас, когда еды завались, и проблема не в недостатке, а в
переизбытке, наша "боязнь голода" вводит нас в заблуждение. То же
самое и с тем, чтоб чувствовать себя нубом.

Хоть это неприятно, и люди высмеивают нас за это, но чем больше вы
чувствуете себя нубом, тем лучше.

\end{document}
