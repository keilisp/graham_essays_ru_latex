\documentclass[ebook,12pt,oneside,openany]{memoir}
\usepackage[utf8x]{inputenc} \usepackage[russian]{babel}
\usepackage[papersize={90mm,120mm}, margin=2mm]{geometry}
\sloppy
\usepackage{url} \title{Мелкие странности: как найти дело своей жизни}
\author{Пол Грэм} \date{}
\begin{document}
\maketitle

Мой отец математик. Большую часть моего детства он работал на компанию
Westinghouse, моделировал ядерные реакторы.

Он был одним из тех везунчиков, которые еще в юности поняли, чем хотят
заниматься. Когда он рассказывает о детстве, виден явный водораздел в
районе 12 лет, когда он «заинтересовался математикой». Он вырос в
небольшом прибрежном городе в Уэльсе. Мы рассматривали на Google
Street View его путь в школу, и он сказал, что было приятно расти в
сельской местности.

«А тебе не стало скучно там лет в 15?» — спросил я.

«Нет, — отвечал он, — тогда я уже увлекался математикой».

В другой раз он рассказал мне, что очень любил решать задачи. Для меня
упражнения в конце каждой главы учебника являются работой или, по
крайней мере, способом закрепления выученного. Для него эти задачи
были вознаграждением. Текст каждой главы был лишь набором
рекомендацией о том, как их решать. Он говорит, что как только получал
новый учебник, сразу же бросался решать все задачи — к некоторому
раздражению учителя, потому что класс должен был идти по книге
постепенно.

Немногие люди так рано и с такой уверенностью понимают, над чем хотят
работать в будущем. Но разговоры с отцом напомнили мне о
познавательном приеме, который могут использовать все. Если что-то
кажется другим людям работой, а вам не кажется — это что-то, к чему вы
хорошо приспособлены. К примеру, большинство программистов, которых я
знаю, в том числе и я сам, любят отладку программ. Люди обычно не
выстраиваются в очередь, чтобы этим заняться — это примерно как
выдавливать прыщи. Но наверное, вам нужно любить отладку, чтобы любить
и само программирование, учитывая, что программирование во многом из
нее и состоит.

Чем более странными ваши вкусы кажутся другим людям, тем более это
надежное доказательство, что возможно, этим вам и стоит заняться.
Когда я учился в колледже, я писал доклады за своих друзей. Было
довольно интересно писать доклад по предмету, который я сам не изучал.
И это приносило моим друзьям огромное облегчение.

Забавно, что одна и та же задача может быть болезненной для одних и
приятной для других. Но я тогда не понимал, что вытекает из этого
дисбаланса, потому что я не искал ответа. Я не понимал, как трудно
бывает решить, чем же тебе заниматься, и что иногда это приходится
устанавливать по крошечным намекам, как сыщику, который раскрывает
дело в детективном романе. И я уверен, что многим людям поможет, если
они зададут этот вопрос себе прямо: какое занятие кажется работой
другим людям, но не кажется вам?

\end{document}
