\documentclass[ebook,12pt,oneside,openany]{memoir}
\usepackage[utf8x]{inputenc} \usepackage[russian]{babel}
\usepackage[papersize={90mm,120mm}, margin=2mm]{geometry}
\sloppy
\usepackage{url} \title{Как справиться с патентными «оковами» без
  государства} \author{Пол Грэм} \date{}
\begin{document}
\maketitle

Недавно я понял, что частично решить проблему патентов можно и без
помощи государства.


Я никогда не был на 100 процентов уверен: патенты помогают
техническому прогрессу или препятствуют ему? Ребенком я думал, что
помогают. Я думал, патенты защищают изобретателя от больших компаний,
которые могут украсть его идею. А, может быть, это и было правдой в
прошлом, когда большинство вещей обладали физическими свойствами.
Несмотря на то, что патенты, в целом, вещь хорошая, некоторые
используют для плохих целей. А так как с таким использованием
встречаешься все чаще и чаще, растет потребность и в реформе патента.

Проблема с патентами потому и проблема, что приходится в процессе
задействовать правительство. Значит, дело будет идти медленно. Однако
недавно я осознал, что мы можем подступиться к этой проблеме следующим
образом: в некоторых случаях реально узнать, как используется патент,
отслеживая место, где он был выпущен.

Один из способов использовать патент явно не для поддержки инноваций —
имеет место быть тогда, когда корпорации, производящие плохой товар,
имея патент, вытесняют мелкие компании с качественными продуктами. С
этим мы можем справиться и без помощи государства.

Чтобы найти компании, которые выше таких фокусов, необходимо
потребовать с них публичные обещания этого не делать. Тогда, те, кто
не даст обещание, сразу будут казаться подозрительными. Потенциальные
работники не захотят туда наниматься. Инвесторы же смогут увидеть, что
компания на плаву благодаря судебным процессам, а не хорошим товарам.

Вот обещание:

Не применять первым патент на программное обеспечение компании, где
работает меньше 25 человек. (No first use of software patents against
companies with less than 25 people.)

Я намеренно привел точное высказывание для краткости. Это обещание не
имеет юридической силы. Оно похоже на слоган Гугла «Не быть злом».
Компания не определяет, что в их понимании зло, но заявила, что они
хотят соответствовать стандарту, которого не придерживается «Алстриа»
(американская компания, один из лидеров мирового рынка табачных
изделий). Несмотря на сдерживающую сторону этого слогана, он принес
Гуглу пользу. Технологические компании выигрывают, привлекая наиболее
работоспособных людей, а этих людей интересуют работодатели, которые
стремятся поднять стандарт качества, а не уменьшать требования.

Обещание, связанное с патентом, производит меньший эффект, но связано
с понятным принципом — «Не будь злом». Я поддерживаю каждую компанию,
которая возьмет этот принцип за основу. Если вы хотите помощь в
решении проблемы патентов — поддержите вашего работодателя.

Большинство технологических компаний уже не будет опускаться до
использования патентов на стартапы. Вы не увидите, как Гугл или
Фейсбук подают в суд на стартапы за нарушение патентных прав. Им это
не нужно. К счастью технологических компаний, обещание по патенту не
требует перемен в поведении. Они просто заявляют делать то, что они и
так обязаны. И когда все не использующие патенты на стартапы компании
дадут это обещание, несогласные (не давшие его) будут бросаться в
глаза.

Такого рода обещание не исчерпает проблему патентов. Оно не остановит
тех, кто не соблюдает патент, например; они уже неприкасаемые. Но это
поможет исправить то, что важнее патентных троллей — они всего лишь
паразиты. Нескладный паразит иногда может убить хозяина, но не это его
цель. Компании, возбуждающие судебные дела из-за нарушения патента
стартапом, хотят добиться одного: чтобы продукт этого стартапа не
вышел на рынок.

Компании предъявляющие требования по соблюдению патента к стартапам
губят инновации на корню. Здесь каждый может сделать что-то, не ожидая
правительства: спросить компании к какому лагерю они принадлежат.

\end{document}
